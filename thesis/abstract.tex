
\begin{abstract} % From rbgs-cal {{{
Formal methods have advanced to the point where
the functional correctness of various large
system components has been mechanically verified.
However,
the diversity of semantic models used across projects
makes it difficult to connect these component
to build larger certified systems.
%Therefore,
Given this,
we seek to embed these models and proofs
into a general-purpose framework
where they could interact. %be made interoperable.
%linked together to
%construct certified heterogeneous systems.
We believe that a synthesis of game
semantics, the refinement calculus, and algebraic effects can
provide such a framework.

To combine game semantics and refinement, we replace the downset
completion typically used to construct strategies from posets of plays.
Using the \emph{free completely distributive completion},
we construct \emph{strategy specifications}
equipped with arbitrary angelic and demonic choices
and ordered by a generalization of alternating refinement.
This provides a novel approach to nondeterminism in game semantics.

%To connect
Connecting algebraic effects and game semantics, we interpret effect
signatures as games and define two
categories %$\gcat^{ib}$ and $\gcat^b$
of effect signatures and strategy
specifications.
The resulting models are sufficient to represent the behaviors
of a variety of low-level components,
including the \emph{certified abstraction layers}
used to verify the operating system
kernel CertiKOS. %~\cite{popl15}.
\end{abstract}
%}}}

\begin{abstract} % From rbgs-compcert {{{
Since the introduction of CompCert,
researchers have been refining
its language semantics and correctness theorem,
and used them in
large-scale software verification efforts.
Meanwhile,
artifacts ranging from CPU designs to network protocols
have been successfully verified,
and there is interest in
making them interoperable
to tackle end-to-end verification
at an even larger scale.

To that end,
we believe that
a synthesis of existing research on
game semantics,
refinement-based methods, and
abstraction layers
has the potential to serve as a common theory
of certified components,
and that integrating CompCert to such a theory
is a critical goal.
However,
the requirements we have identified for
CompCert to be deployed in this context
are not met by any of its existing variants.

CompCertO is
a new extension of CompCert
which characterizes compiled program modules
in terms of their interaction with other components.
By extending the CompCert semantics
in a way that embraces relational reasoning,
we achieve this with only a minimal increase
in proof size.
\end{abstract}
%}}}

