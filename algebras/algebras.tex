\documentclass{beamer}
\usepackage{stmaryrd}
\usepackage{tikz-cd}
\usepackage{amsfonts}
\usepackage{bbm}
\usefonttheme[onlymath]{serif}
\setlength{\parskip}{1ex}

\newcommand{\fun}[1]{\mathrm{#1}}
\newcommand{\cat}[1]{\mathbf{#1}}

\title{Classifying simple game models}
\author{J\'er\'emie Koenig}

\AtBeginSection[]
{
    \begin{frame}
        \frametitle{Table of Contents}
        \tableofcontents[currentsection,currentsubsection]
    \end{frame}
}

\begin{document}

\begin{frame}
\titlepage
\end{frame}

\section{Introduction}

\section{Basic setup}

\begin{frame}{Algebraic effects} %{{{
%Recall that in the framework of algebraic effects,
Effectful computations are terms in an algebra where: \pause
\begin{itemize}
  \item operations are the possible effects;
  \item their operands are possible continuations.
\end{itemize}
\pause
Operations \emph{prepend} an effect to a computation:
$\fun{readbit}(x, y)$ reads a bit,
then continues as $x$ or $y$ depending on its value.
\pause

\vfill
An \emph{effect signature} gives a set $E$ of operations
and an assignment $\fun{ar} : E \rightarrow \cat{Set}$
of an arity to each operation.
We write:
\[
    E = \{ m_1 : \fun{ar}(m_1), \ldots, m_k : \fun{ar}(m_k) \}
\]
\pause
Arities correspond to possible outcomes of an effect
($\fun{readbit} : \mathbbm{2}$).
Effects which take parameters are \emph{families}
of operations, for example
$\{ \fun{print}[s] : \mathbbm{1} \mid s \in \fun{string} \}$.
\end{frame}
%}}}

\begin{frame}{Algebras for a signature} %{{{
To give effects a semantics,
we can provide an interpretation.
An~algebra for a an effect signature $E$
is given by:
\begin{itemize}
  \item A carrier set $A$
  \item For each $(m : N) \in E$, its interpretation
    $\llbracket m \rrbracket : A^{\fun{ar}(m)} \rightarrow A$
\end{itemize}
\pause

\vfill
Categorically, consider the functor
$\hat{E} : \cat{Set} \rightarrow \cat{Set}$
defined by:
\[
    \hat{E} X := \sum_{m \in E} X^{\fun{ar}(m)}
               = \sum_{m \in E} \prod_{n \in \fun{ar}(m)} X
\]
Then an $\hat{E}$-algebra is a set $A$
together with $\alpha : \hat{E} A \rightarrow A$.
%\begin{align*}
%    \hat{E} X &= \sum_{m \in E} X^{\fun{ar}(m)} \\
%              &= \sum_{m \in E} \prod_{n \in \fun{ar}(m)} X \,.
%\end{align*}
\pause

\vfill
Correspondance:
$\llbracket m \rrbracket(\vec{x}) = \alpha(\iota_m(\vec{x}))$
\end{frame}
%}}}

\begin{frame}[fragile]{Category of algebras} %{{{
The $\hat{E}$-algebras form a category $\hat{E} \cat{Alg}$
whose:
\begin{itemize}
  \item Objects are pairs $(X, \chi)$ with
    $\chi : \hat{E} X \rightarrow X \in \cat{Set}$
    \[
      \begin{tikzcd}
        \hat{E} A \ar[r, "\alpha"] & A
      \end{tikzcd}
    \]
  \item Homomorphisms
    $f : (A, \alpha) \rightarrow (B, \beta) \in \hat{E} \cat{Alg}$
    are functions $f : A \rightarrow B \in \cat{Set}$ such that
    $f \circ \alpha = \beta \circ \hat{E} f$:
    \[
      \begin{tikzcd}
        \hat{E} A \ar[r, "\alpha"] \ar[d, "\hat{E} f"'] &
        A \ar[d, "f"] \\
        \hat{E} B \ar[r, "\beta"'] &
        B
      \end{tikzcd}
    \]
\end{itemize}
\end{frame}
%}}}

\begin{frame}[fragile]{Initial algebra} %{{{
I will write $(\mu \hat{E}, c)$ for the \emph{initial $\hat{E}$-algebra},
characterized by:
\[
  \begin{tikzcd}
    \hat{E} \, \mu\hat{E} \ar[r, "c"] \ar[d, "\hat{E} d_\chi"', dashed] &
      \mu \hat{E} \ar[d, "d_\chi", "!"', dashed] \\
    \hat{E} X \ar[r, "\chi"] &
      X
  \end{tikzcd}
\]
\pause
It corresponds to:
\begin{itemize}
  \item the set of finite terms over the signature $E$;
  \item the inductive type with operations of $E$ as constructors;
  \item terminating strategies, seeing $E$ as an iterated game;
  \item the least fixed point of $\hat{E}$
        ($\mu\hat{E} \cong \hat{E}\,\mu\hat{E}$), hence:
    \[
      \text{``} \quad
      \mu \hat{E} \cong
        \sum_{m_1}
        \prod_{n_1}
        \sum_{m_2}
        \prod_{n_2}
        \sum_{m_3}
        \prod_{n_3}
        \cdots
      \quad \text{''}
    \]
\end{itemize}
\end{frame}
%}}}

\begin{frame}[fragile]{Free monad} %{{{
More generally, there is an adjunction:
\[
  \begin{tikzcd}
    \cat{Set}
    \arrow[r, "F_{\hat{E}}" name=F, shift left=2] &
    \hat{E} \cat{Alg}
    \arrow[l, "U_{\hat{E}}" name=G, shift left=2]
    \arrow[phantom, from=F, to=G, "\dashv" rotate=-90]
  \end{tikzcd}
  \qquad
  \begin{tikzcd}
    A \arrow[r, "\eta_A"] \arrow[rd, "f"'] &
    UFA \arrow[d, "U f^*"] &
    FA \arrow[d, "f^*", "!"', dashed] \\
    & U B & B
  \end{tikzcd}
\]
\pause

The forgetful functor
$U_{\hat{E}} : \hat{E} \cat{Alg} \rightarrow \cat{Set}$
keeps the carrier set of an algebra
and the underlying function of an algebra homomorphisms.
\pause

Its left adjoint
$F_{\hat{E}} : \cat{Set} \rightarrow \hat{E} \cat{Alg}$
takes a set of generators
and adds all possible terms that can be constructed from them.
\pause

The monad $U_{\hat{E}} F_{\hat{E}}$
associated with the adjunction
is one version of the \emph{free monad on $E$}.
Note also that $F_{\hat{E}}(\varnothing) = \mu \hat{E}$.
\end{frame}
%}}}

\section{Generalizing}

\begin{frame}{Base category}
Although we have done everything in $\cat{Set}$,

\end{frame}

\begin{frame}{Coalgebras}
\end{frame}

\begin{frame}{Dual functor}
\end{frame}

\begin{frame}[fragile]{Classification}
\begin{tabular}{l|c|c|c|c|}
  & $\mu \hat{E}$ & $\nu \hat{E}$ & $\mu \check{E}$ & $\nu \check{E}$ \\
  \hline
  $\cat{Set}$ & Finite terms & All terms & & \\
  \hline
  $\cat{Set}_\bot$ & & ITrees & & CAL \\
  \hline
  $\cat{CPPO}$ & \multicolumn{2}{c|}{} & \multicolumn{2}{c|}{Traditional strategies} \\
  \hline
% $\cat{Sup}_*$ & \multicolumn{2}{c|}{} & \multicolumn{2}{c|}{} \\
% \hline
% $\cat{Sup}$ & \multicolumn{2}{c|}{} & \multicolumn{2}{c|}{} \\
% \hline
  $\cat{CDLat}_*$ & \multicolumn{2}{c|}{Interaction specifications} & \multicolumn{2}{c|}{} \\
  \hline
% $\cat{CDLat}$ & \multicolumn{2}{c|}{} & \multicolumn{2}{c|}{} \\
% \hline
\end{tabular}
\end{frame}

\section{Going further}

\begin{frame}{Extending this}

Combining $\hat{E}$, $\check{E}$
to recover various kinds of strategies.
Profunctor?

\end{frame}


\end{document}

