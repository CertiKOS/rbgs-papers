\documentclass{beamer}
\usepackage{stmaryrd}
\usepackage{tikz-cd}
\usepackage{amsfonts}
\usepackage{bbm}
\usefonttheme[onlymath]{serif}
\setlength{\parskip}{1ex}

\newcommand{\fun}[1]{\mathrm{#1}}
\newcommand{\cat}[1]{\mathbf{#1}}

\title{Classifying simple game models}
\author{J\'er\'emie Koenig}

\AtBeginSection[]
{
    \begin{frame}
        \frametitle{Table of Contents}
        \tableofcontents[currentsection]
    \end{frame}
}

\begin{document}

\begin{frame}
\titlepage
\end{frame}

\section{Basic setup}

\begin{frame}{Algebraic effects} %{{{
%Recall that in the framework of algebraic effects,
Effectful computations are terms in an algebra where: \pause
\begin{itemize}
  \item operations are the possible effects;
  \item their operands are possible continuations.
\end{itemize}
\pause
Operations \emph{prepend} an effect to a computation:
$\fun{readbit}(x, y)$ reads a bit,
then continues as $x$ or $y$ depending on its value.
\pause

\vfill
An \emph{effect signature} gives a set $\Sigma$ of operations
and an assignment $\fun{ar} : \Sigma \rightarrow \cat{Set}$
of an arity to each operation.
We write:
\[
    \Sigma = \{ m_1 : \fun{ar}(m_1), \ldots, m_k : \fun{ar}(m_k) \}
\]
\pause
Arities correspond to possible outcomes of an effect
($\fun{readbit} : \mathbbm{2}$).
Effects which take parameters are \emph{families}
of operations, for example
$\{ \fun{print}[s] : \mathbbm{1} \mid s \in \fun{string} \}$.
\end{frame}
%}}}

\begin{frame}{Algebras for a signature} %{{{
To give effects a semantics,
we can provide an interpretation.
An~algebra for a an effect signature $\Sigma$
is given by:
\begin{itemize}
  \item A carrier set $A$
  \item For each $m \in \Sigma$, its interpretation
    $\llbracket m \rrbracket : A^{\fun{ar}(m)} \rightarrow A$
  \pause
  \item Or equivalently, a function
  $\alpha :
     \left(\sum_{m \in \Sigma} A^{\fun{ar}(m)} \right)
     \rightarrow A$
\end{itemize}
\pause

\vfill
Categorically, this is an algebra for
$\hat{\Sigma} : \cat{Set} \rightarrow \cat{Set}$
defined by:
\[
    \hat{\Sigma} X := \sum_{m \in \Sigma} \prod_{n \in \fun{ar}(m)} X \,
\]
That is a object $A \in \cat{Set}$
together with a morphism $\alpha : \hat{\Sigma} A \rightarrow A$.
%\begin{align*}
%    \hat{\Sigma} X &= \sum_{m \in \Sigma}} X^{\fun{ar}(m)} \\
%              &= \sum_{m \in \Sigma} \prod_{n \in \fun{ar}(m)} X \,.
%\end{align*}
%\pause
%\vfill
%Correspondance:
%$\llbracket m \rrbracket(\vec{x}) = \alpha(\iota_m(\vec{x}))$
\end{frame}
%}}}

\begin{frame}[fragile]{Category of algebras} %{{{
The $\hat{\Sigma}$-algebras form a category $\hat{\Sigma} \cat{Alg}$
whose:
\begin{itemize}
  \item Objects are pairs $(X, \chi)$ with
    $\chi : \hat{\Sigma} X \rightarrow X \in \cat{Set}$
    \[
      \begin{tikzcd}
        \hat{\Sigma} X \ar[r, "\chi"] & X
      \end{tikzcd}
    \]
  \item Homomorphisms
    $f : (A, \alpha) \rightarrow (B, \beta) \in \hat{\Sigma} \cat{Alg}$
    are functions $f : A \rightarrow B \in \cat{Set}$ such that
    $f \circ \alpha = \beta \circ \hat{\Sigma} f$:
    \[
      \begin{tikzcd}
        \hat{\Sigma} A \ar[r, "\alpha"] \ar[d, "\hat{\Sigma} f"'] &
        A \ar[d, "f"] \\
        \hat{\Sigma} B \ar[r, "\beta"'] &
        B
      \end{tikzcd}
    \]
\end{itemize}
\end{frame}
%}}}

\begin{frame}[fragile]{Initial algebra} %{{{
I will write $(\mu \hat{\Sigma}, c)$ for the \emph{initial $\hat{\Sigma}$-algebra},
characterized by:
\[
  \begin{tikzcd}
    \hat{\Sigma} \, \mu\hat{\Sigma} \ar[r, "c"] \ar[d, "\hat{\Sigma} d_\chi"', dashed] &
      \mu \hat{\Sigma} \ar[d, "d_\chi", "!"', dashed] \\
    \hat{\Sigma} X \ar[r, "\chi"] &
      X
  \end{tikzcd}
\]
\pause
It corresponds to:
\begin{itemize}
  \item the set of finite terms over the signature $\Sigma$;
  \item the inductive type with operations of $\Sigma$ as constructors;
  \item terminating strategies, seeing $\Sigma$ as an iterated game;
  \item the least fixed point of $\hat{\Sigma}$
        ($\mu\hat{\Sigma} \cong \hat{\Sigma}\,\mu\hat{\Sigma}$), hence:
    \[
      \text{``} \quad
      \mu \hat{\Sigma} \cong
        \sum_{m_1}
        \prod_{n_1}
        \sum_{m_2}
        \prod_{n_2}
        \sum_{m_3}
        \prod_{n_3}
        \cdots
      \quad \text{''}
    \]
\end{itemize}
\end{frame}
%}}}

\begin{frame}[fragile]{Free monad} %{{{
More generally, there is an adjunction:
\[
  \begin{tikzcd}
    \cat{Set}
    \arrow[r, "F_{\hat{\Sigma}}" name=F, shift left=2] &
    \hat{\Sigma} \cat{Alg}
    \arrow[l, "U_{\hat{\Sigma}}" name=G, shift left=2]
    \arrow[phantom, from=F, to=G, "\dashv" rotate=-90]
  \end{tikzcd}
  \qquad
  \begin{tikzcd}
    A \arrow[r, "\eta_A"] \arrow[rd, "f"'] &
    UFA \arrow[d, "U f^*"] &
    FA \arrow[d, "f^*", "!"', dashed] \\
    & U B & B
  \end{tikzcd}
\]
\pause

The forgetful functor
$U_{\hat{\Sigma}} : \hat{\Sigma} \cat{Alg} \rightarrow \cat{Set}$
keeps the carrier set of an algebra
and the underlying function of an algebra homomorphisms.
\pause

Its left adjoint
$F_{\hat{\Sigma}} : \cat{Set} \rightarrow \hat{\Sigma} \cat{Alg}$
takes a set of generators
and adds all possible terms that can be constructed from them.
\pause

The monad $U_{\hat{\Sigma}} F_{\hat{\Sigma}}$
associated with the adjunction
is one version of the \emph{free monad on $\Sigma$}.
Note also that $F_{\hat{\Sigma}}(\varnothing) = \mu \hat{\Sigma}$.
\end{frame}
%}}}

\section{Generalizing}

\begin{frame}[fragile]{Coalgebras}
The initial $\hat{\Sigma}$-algebra
gives \emph{finite} terms over the signature.
The~\emph{final coalgebra} also includes infinite terms:
\[
  \begin{tikzcd}
    X \arrow[r, "\chi"] \arrow[d, "c_\chi", "!"', dashed] &
    \hat{\Sigma} X \arrow[d, "\hat{\Sigma} c_\chi", dashed] \\
    \nu \hat{\Sigma} \arrow[r, "d"] &
    \hat{\Sigma} \, \nu\hat{\Sigma}
  \end{tikzcd}
\]
Here $\chi$ ``observes'' the first action performed by
an element $x \in X$ of the coalgebra,
and $c_\chi$ recursively unfolds these observations.

As before, there is an adjunction
$\begin{tikzcd}
    \cat{Set}
    \arrow[r, "\bar{F}_{\hat{\Sigma}}" name=F, shift left=2] &
    \hat{\Sigma} \cat{Coalg}
    \arrow[l, "\bar{U}_{\hat{\Sigma}}" name=G, shift left=2]
    \arrow[phantom, from=F, to=G, "\dashv" rotate=-90]
  \end{tikzcd}$.
%  \qquad
%  \begin{tikzcd}
%    A \arrow[r, "\eta_A"] \arrow[rd, "f"'] &
%    \bar{U} \bar{F} A \arrow[d, "\bar{U} f^*"] &
%    \bar{F} A \arrow[d, "f^*", "!"', dashed] \\
%    & \bar{U} B & B
%  \end{tikzcd}

\emph{Interaction trees} are given by
$\bar{U}_{\hat{\Sigma}_\tau} \bar{F}_{\hat{\Sigma}_\tau}$,
where
$\Sigma_\tau := \Sigma + \{ \tau : \mathbbm{1} \}$.
\end{frame}

\begin{frame}[fragile]{Base category}
Although we have done everything in $\cat{Set}$,
the same constructions can be carried out in
any category $\cat{C}$ with products ($\prod$) and coproducts ($\sum$).
If there is a forgetful functor $U_\cat{C} : \cat{C} \rightarrow \cat{Set}$
and a corresponding $F_\cat{C} : \cat{Set} \rightarrow \cat{C}$,
we can combine the adjunctions:
\[
  \begin{tikzcd}
    \cat{Set}
    \arrow[r, "F_\cat{C}" name=F1, shift left=2] &
    \cat{C}
    \arrow[l, "U_\cat{C}" name=G1, shift left=2]
    \arrow[r, "F_{\hat{\Sigma}}" name=F, shift left=2] &
    \hat{\Sigma} \cat{Alg}
    \arrow[l, "U_{\hat{\Sigma}}" name=G, shift left=2]
    \arrow[phantom, from=F1, to=G1, "\dashv" rotate=-90]
    \arrow[phantom, from=F, to=G, "\dashv" rotate=-90]
  \end{tikzcd}
\]
The result combines the structure of objects of $\cat{C}$
with the effect signature we are considering.

For example,
\emph{interaction specifications}
are obtained using $\cat{CDLat}^\bullet$.
\emph{ITrees up to $\tau$}
are coalgebras over $\cat{Set}^\bullet$.
\end{frame}

\begin{frame}{Dual functor}
Terms are obtained by iterating the functor:
\[
    \hat{\Sigma} X := \sum_{m \in \Sigma} \prod_{n \in \fun{ar}(m)} X
\]
They correspond to a client making a series of requests
$m \in \Sigma$ and getting responses $n \in \fun{ar}(n)$.

Conversely, the functor:
\[
    \check{\Sigma} X := \prod_{m \in \Sigma} \sum_{n \in \fun{ar}(m)} X
\]
corresponds to the server/environment/handler.

For examples, behaviors of
\emph{certified abstraction layers}
can be modeled as $\nu \check{\Sigma}$ in $\cat{Set}^\bullet$.
\end{frame}

\section{Conclusion}

\begin{frame}[fragile]{Conjectured classification}
\begin{centering}
\begin{tabular}{l|c|c|c|c|}
  & $\mu \hat{\Sigma}$ & $\nu \hat{\Sigma}$ & $\mu \check{\Sigma}$ & $\nu \check{\Sigma}$ \\
  \hline
  $\cat{Set}$ & Finite terms & All terms & & \\
  \hline
  $\cat{Set}^\bullet$ & & ITrees & & CAL \\
  \hline
  $\cat{CPPO}$ & \multicolumn{2}{c|}{} & \multicolumn{2}{c|}{Traditional strategies} \\
  \hline
% $\cat{Sup}_*$ & \multicolumn{2}{c|}{} & \multicolumn{2}{c|}{} \\
% \hline
% $\cat{Sup}$ & \multicolumn{2}{c|}{} & \multicolumn{2}{c|}{} \\
% \hline
  $\cat{CDLat}_*$ & \multicolumn{2}{c|}{Interaction specifications} & \multicolumn{2}{c|}{} \\
  \hline
% $\cat{CDLat}$ & \multicolumn{2}{c|}{} & \multicolumn{2}{c|}{} \\
% \hline
\end{tabular}
\end{centering}

\vfill
\pause
Questions I don't know the answer to:
\begin{itemize}
  \item Is this accurate?
  \item How can $\hat{\Sigma}$, $\check{\Sigma}$ interact:
  \begin{itemize}
    \item to play against each other?
    \item to be mixed in one model ($\Gamma \rightarrow \Sigma$)?
  \end{itemize}
  \item Can we generalize to richer games? % and keep it simple?
%  \item with and $\oplus$ seem straightforward to interpret,
%    what of par and $\otimes$?
\end{itemize}
\end{frame}

\end{document}

