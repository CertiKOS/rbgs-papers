\documentclass{beamer}
\usepackage{amsmath}
\usepackage{amssymb}
\usepackage{tikz-cd}

\title{The free monad monad as {!}}
\author{J\'er\'emie Koenig}

\setlength{\parskip}{1ex}

\begin{document}

\maketitle

\begin{frame}{} %{{{

  Dear L\'eo and Prof. Melli\`es,

  Thank you for your August 12 note and ensuing discussion.

  I still need to digest but generally speaking,
  squeezing out \\ the abstract structures and systematizing is very valuable:
  \begin{itemize}
    \item To understand our models in broader context
    \item To ensure the construction is sound
    \item To better understand the \emph{practical problem we start from}
    \item For engineering purposes
  \end{itemize}

\end{frame}
%}}}

\begin{frame}{Scope} %{{{

  Interesting angles (I think) in my LICS paper:
  \begin{enumerate}
    \item Games and dual nondeterminism
    \item Construction of certified abstraction layers
    \item Connecting games, effects, algebra
  \end{enumerate}

  You address {1}, {2} in your note.

  Today, I would like to touch on {3}.

\end{frame}
%}}}

\section{Introduction}
\frame{\sectionpage}

\begin{frame}{Starting point} %{{{

  You have observed in the past things along the lines:
  \begin{itemize}
    \item It may be possible to factor $E \rightarrow F$
      into ${!E} \multimap F$.
    \item Our $x[f]$ construction looks like a second order monad
  \end{itemize}

  I have been mulling them over and worked out some details.

  It's the ``free monad'' monad on $\mathbf{End}$!

\end{frame}
%}}}

\begin{frame}[fragile]{Outline} %{{{

  Some stories people like to tell about the free monad:
  \begin{itemize}
    \item $E \mathbf{Alg} : T_E \dashv U_\mathbf{Alg} : \mathbf{Set}$.
      \emph{(Yawn)}
    \item $\mathbf{Mnd} : T_{(-)} \dashv U_\mathbf{Mnd} : \mathbf{End}$.
      \emph{(Aha!)}
  \end{itemize}

  I have come to understand that these are categories of games:
  \begin{center}
    \begin{tabular}{llll}
      \hline
      $\mathbf{End}^\mathsf{op}$ &
        endofunctors &
        natural transformations &
        $E \multimap F$
      \\
      $\mathbf{Mnd}^\mathsf{op}$ &
        monads &
        monad homomorphisms &
        $E \rightarrow F$
      \\
      \hline
    \end{tabular}
  \end{center}

  This is quite obvious in retrospect!

\end{frame}
%}}}

\section{Construction}
\frame{\sectionpage}

\begin{frame}{Setting} %{{{

  For now I have only thought about the case of:
  \begin{itemize}
    \item deterministic, total strategies;
    \item endofunctors as a slight generalization of signatures.
  \end{itemize}

  So my games will be endofunctors on $\mathbf{Set}$.

\end{frame}
%}}}

\begin{frame}{Endofunctors as games} %{{{

  Effect signatures are interpreted as endofunctors:
  \[
    E \, X \:=\: \sum_{(m : N) \in E} X^N
        \:=\: \sum_{(m : N) \in E} \prod_{n \in N} \:\: X
  \]

  $E \, X$ are the depth 1 terms of the form
  $m(x_n)_{n \in N}$ where $x_n \in X$.

  \vfill
  Game interpretation:
  \begin{itemize}
    \item $\sum$ is choice of $\mathsf{O}$
    \item $\prod$ is choice of $\mathsf{P}$
    \item $X$ is ``done''
  \end{itemize}

\end{frame}
%}}}

\begin{frame}{Natural transformations as linear strategies} %{{{

  If $E$ and $F$ are endofunctors for signatures,
  \[
    \phi : E \rightarrow F \quad
    \text{corresponds to} \quad
    \sigma : F \multimap E
  \]

  When $(m \mathop: N) \mapsto (q_m \mathop: R)$ and
  $(r \in R) \mapsto (n_r \in N)$:
  \[
    \phi_X(m(x_n)_{n \in N}) = q_m(x_{n_r})_{r \in R}
  \]

  \vfill
  Game interpretation of $m(x_n)_{n\in N} \in E X$:
  \begin{itemize}
    \item $\mathsf{O}$ specifies a question $m$
      and a value $x_n$ for each answer $n$.
    \item $\mathsf{P}$ returns the value $x_n$
      corresponding to its answer.
    \item Naturality prevents funky business in the choice of $x_n$.
  \end{itemize}

\end{frame} 
%}}}

\begin{frame}{Free monad} %{{{

  The free monad on $E : \mathbf{Set} \rightarrow \mathbf{Set}$
  can be described as:
  \[
    T_E(X) = \sum_{n \in \mathbb{N}} E^n X = \mu K \,.\, (X + E K)
     = {!E}
  \]

  If $E$ comes from an effect signature, \\
  $T_E(X)$ is the terms on $E$ with variables in $X$.

  \vfill
  Game interpretation:
  \begin{itemize}
    \item $T_{(-)}$ is an exponential modality
    \item monad homomorphism $\Leftrightarrow$ \emph{innocent} strategy
  \end{itemize}

\end{frame}
%}}}

\begin{frame}{The ``free monad'' monad} %{{{

  The functor $T_{(-)} : \mathbf{End} \rightarrow \mathbf{Mnd}$ maps:
  \begin{itemize}
    \item the endofunctor $E$ to its free monad $T_E$;
    \item a natural transformation $E \xrightarrow{\mathbf{End}} F$ to \\
      a monad homomorphism $T_E \xrightarrow{\mathbf{Mnd}} T_F$, aka
    \item a strategy for $F \multimap E$ to
      an innocent strategy for ${!F} \multimap {!E}$.
  \end{itemize}

  The functor $U : \mathbf{Mnd} \rightarrow \mathbf{End}$ maps:
  \begin{itemize}
    \item the monad $\langle T, \eta, \mu \rangle$
      to its underlying endofunctor $T$;
    \item forgets a natural transformation is a monad homomorphism;
    \item forgets a strategy is innocent.
  \end{itemize}

  It turns out $T_{(-)} \dashv U$,
  and so $U T_{(-)}$ is a comonad in $\mathbf{End}^\mathsf{op}$
  which gives the structure associated with the exponential modality.

\end{frame}
%}}}

\section{Conclusion}
\frame{\sectionpage}

\begin{frame}[fragile]{Dialectical trinitarianism} %{{{

  Like computational trinitarianism but for interacting stuff:
  \[
    \begin{tikzcd}[column sep=tiny]
      & \text{(Co-)Algebra} \ar[dl, dash] \ar[dr, dash] & \\
      \text{Effects} \ar[rr, dash] & &
      \text{Games}
    \end{tikzcd}
  \]

\end{frame}
%}}}

\begin{frame}{Future work} %{{{

  Specialize:
  \begin{itemize}
    \item
      The ``free monad'' monad in $\mathbf{ESig}$. \\
      We get funky arities but it should work. \\
      Why? Because it will Coq-ify like butter!
  \end{itemize}

  Generalize:
  \begin{itemize}
    \item Your construction of nondeterminism
    \item Endofunctors and monads on something else than $\mathbf{Set}$
    \item Enrichment of some kind
    \item Templates as \emph{signaturoids}?
    \item How does coalgebra fit in?
  \end{itemize}

\end{frame}
%}}}

%
%Dear Paul-Andr\'e and L\'eo,
%thank you for your August 12 note
%presenting a categorical analysis of our layer models
%and the ensuing discussion.
%Below I outline some thoughts about this work
%and address some of the topics we touched on
%in our previous conversations.
%
%\section{Significance}
%
%In this first section
%I want to summarize why I think this general line of inquiry is promising
%and my point of view on what could come out of our collaboration.
%
%\subsection{As a way to guide our research}
%
%Systematizing and
%squeezing the abstract structures out of our models
%allows us to understand better how they emerge,
%how they relate to other models,
%and how they can be generalized.
%We can also make sure their construction is sound,
%in the sense that if every detail naturally ``falls out''
%of some abstract construction,
%we can be confident that a given model
%is the right one for the job.
%
%There is an interesting dynamic
%where 
%
%(ideally, our understanding of ``the job''
%will be made more crisp and formal
%by connecting it to the abstract description),
%and that it is free of peculiarities or arbitrary choices
%which may end up making things more complicated
%or less complete than they need to be
%(I am wondering about your observation
%regarding demonic choice in the first move
%in particular).
%
%\subsection{As a software engineering device}
%
%In the process of formalizing
%certified abstraction layers and game semantics.
%
%
%\section{Free monads and $!$ modality}
%
%You have suggested in the past that
%it may be possible to give an exponential construction $!E$
%on signature so that
%our strategies for $E \rightarrow F$
%could be reformulated as ${!E} \multimap F$,
%where $E \multimap F$ could be a very
%elementary 4-move game.
%For simple strategies in $\mathbf{Set}$
%this can be formalized as follows.
%
%\subsection{Signatures and endofunctors}
%
%An effect signature $E$
%can be represented as an \emph{endofunctor}
%$\hat{E} : \mathbf{Set} \rightarrow \mathbf{Set}$:
%\[
%  \hat{E} X := \sum_{(m \mathop: N) \in E} X^N
%  \,.
%\]
%The endofunctor $\hat{E} X$ constructs ``proto-terms''
%which consists of one operation of $E$
%applied to elements of $X$.
%
%An algebra for the signature $E$ with a carrier set $A$
%can be represented categorically as an $\hat{E}$-algebra
%$\alpha : \hat{E} A \rightarrow A$.
%A homomorphism of $\hat{E}$-algebras
%$f : \langle A, \alpha \rangle \rightarrow \langle B, \beta \rangle$
%is a function $f : A \rightarrow B$
%which makes the following diagram commute:
%\[
%  \begin{tikzcd}
%    \hat{E} A \ar[r, "\alpha"] \ar[d, "\hat{E} f"'] & A \ar[d, "f"] \\
%    \hat{E} B \ar[r, "\beta"] & B
%  \end{tikzcd}
%\]
%I will call the category of $\hat{E}$-algebras $E \mathbf{Alg}$.
%
%\subsection{Signatures and monads}
%
%The \emph{free monad} $T_E(X)$ for $E$ iterates this construction.
%It is characterized by the adjunction $F_E \dashv U$,
%where $U : E \mathbf{Alg} \rightarrow \mathbf{Set}$
%retains only the carrier set of an algebra
%and $F_E : \mathbf{Set} \rightarrow E \mathbf{Alg}$
%gives us the free algebra for a given set of generators,
%and the functor can be described as:
%\[
%  T_E(X) = \sum_{n \in \mathbb{N}} \hat{E}^n X
%  = \mu Y \,.\, X + \hat{E}(Y)
%  \,.
%\]
%This much is pretty well-known,
%there is a one-to-one correspondence between
%the algebras for the endofunctor $\hat{E}$ and
%the algebras for the monad $T_E$,
%blah blah\ldots
%
%\subsection{Strategies as natural transformations}
%
%A less commonly articulated part of the story is
%the relationship between
%\emph{natural transformations} on endofunctors
%and \emph{monad homomorphisms}.
%I have come to understand them as strategies.
%A monad homomorphism $\sigma : T_F \rightarrow_\mathbf{Mnd} T_E$
%correspond to a strategy for the game $E \rightarrow F$.
%A natural transformation $\eta : \hat{F} \rightarrow_\mathbf{End} \hat{E}$
%corresponds to a strategy for the 4-move game $E \multimap F$
%which you described.
%
%In fact,
%there is a ``next-level'' adjunction
%$T_{(-)} \dashv U$
%between the categories $\mathbf{Mnd}$ of monads
%and $\mathbf{End}$ of endofunctors on $\mathbf{Set}$,
%where $T_{(-)}$ constructs the free monad for an endofunctor
%and $U : \mathbf{Mnd} \rightarrow \mathbf{End}$
%takes $\langle T, \eta, \mu \rangle$ to $T$.
%The associated natural isomorphism is:
%\[
%  \forall F \in \mathbf{End},
%  \forall T \in \mathbf{Mnd},
%  \mathbf{Mnd}(T_F, T) \cong \mathbf{End}(F, UT)
%\]
%Plugging in $T_E$ for $E$
%we get that the ``cartesian'' strategies $E \rightarrow F$
%correspond to the linear strategies $UT_E \multimap F$.
%
%So to summarize,
%the ``free monad'' monad
%gives us a comonad in
%the category of games and strategies $\mathbf{End}^\mathrm{op}$,
%and can be used to define
%a version of the exponential modality $!$ as $UT_{(-)}$
%together with the associated ${!E} \multimap E$
%and the strategy $\sigma^\dagger : {!E} \multimap {!F}$
%for all $\sigma : {!E} \multimap F$.
%The associated co-Kleisli category is
%$\mathbf{Mnd}^\mathrm{op}$.
%
%I think you have also noted before
%that our construction of strategies
%from the interaction specification monad
%almost comes with ``two layers of monads'',
%with $x[\sigma]$
%
%
%
%\subsection{Templatification}
%
%\subsection{Sup as a model of linear logic}

\end{document}
