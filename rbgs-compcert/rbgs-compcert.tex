\documentclass[acmsmall,screen,review,anonymous]{acmart}

% Enable/disable optional sections here
\newif\ifopt \optfalse

% Packages {{{
\usepackage{bbm}
\usepackage{tikz}
\usetikzlibrary{calc}
\usetikzlibrary{graphs}
\usetikzlibrary{cd}
\usetikzlibrary{patterns}
\usetikzlibrary{backgrounds}
\usetikzlibrary{shapes}
\usepackage{bussproofs}
\usepackage{stmaryrd}
\usepackage{bbm}
\usepackage{soul}
\usepackage{booktabs}
% }}}

% Parameters {{{
\settopmatter{printfolios=true,printccs=false,printacmref=false}
\setcopyright{none}

\acmJournal{PACMPL}
\acmVolume{1}
\acmNumber{POPL}
\acmArticle{1}
%\acmYear{2021}
%\acmMonth{1}
%\acmDOI{}  \acmDOI{10.1145/nnnnnnn.nnnnnnn}
%\startPage{1}

\bibliographystyle{ACM-Reference-Format}
\citestyle{acmauthoryear}

\captionsetup[table]{position=bottom}
\hyphenation{Comp-Cert}
\hyphenation{Comp-CertO}

\newcommand{\figsize}{\small}
% }}}

% Macros {{{
\newcommand{\kw}[1]{\ensuremath{ \mathsf{#1} }}
\newcommand{\ifr}[1]{\mathrel{[{#1}]}}
\newcommand{\que}{\circ}
\newcommand{\ans}{\bullet}
\newcommand{\vref}{\le_\kw{v}}
\newcommand{\mext}{\le_\kw{m}}
\newcommand{\refby}{\preceq}
\newcommand{\scref}{\sqsupseteq}

% Pointers for justified sequences %{{{

% Parameters
\newcommand{\pshift}{1.6ex}
\newcommand{\pcdist}{2.5}
\newcommand{\pcangle}{60}

% Pointer hook
\newcommand{\ph}[1]{%
  \tikz[remember picture]{\coordinate (#1);}}

% Pointer to
\newcommand{\ptc}[2]{%
  \rule{0pt}{1.4em}%
  \tikz[remember picture, overlay]{
    \draw[->,#2]
      let \p{dest} = (#1),
          \n1 = {ln(veclen(\x{dest}, \y{dest}) + 1)},
          \p1 = ($(0,0)+(0,\pshift)$),
          \p4 = ($(#1)+(0,\pshift)$),
          \p2 = ($(\p1)!\n1*\pcdist!-\pcangle:(\p4)$),
          \p3 = ($(\p4)!\n1*\pcdist!+\pcangle:(\p1)$) in
        (\p1) .. controls (\p2) and (\p3) .. (\p4);}}
\newcommand{\pt}[1]{%
  \ptc{#1}{lightgray}}
\newcommand{\bpt}[1]}}

\ifopt
  \newcommand{\opt}[2]{#1}
  \newenvironment{optional}{\begin{color}{gray}}{\end{color}}
\else
  \newcommand{\opt}[2]{#2}
  \excludecomment{optional}
\fi
\newcommand{\anon}[2]{#1}

% }}}

\title{CompCertO: Compiling Certified Open C Components} %{{{

\author{J\'er\'emie Koenig}
\orcid{0000-0002-3168-5925}
\affiliation{
  \institution{Yale University}
  \city{New Haven}
  \state{CT}
  \country{USA}
}
\email{jeremie.koenig@yale.edu}

\author{Zhong Shao}
\affiliation{
  \institution{Yale University}
  \city{New Haven}
  \state{CT}
  \country{USA}
}
\email{zhong.shao@yale.edu}

%}}}

\begin{document}

\begin{abstract} %{{{
Since the introduction of CompCert,
researchers have been refining
its language semantics and correctness theorem,
and used them as components
in software verification efforts.
Meanwhile,
artifacts ranging from CPU designs to network protocols
have been successfully verified,
and there is interest in
making them interoperable
to tackle end-to-end verification
at an even larger scale.

To that end,
\anon{it has been proposed}{we believe} that
a synthesis of existing research on
game semantics,
refinement-based methods, and
abstraction layers
has the potential to serve as a common theory
of certified components.
Integrating CompCert to such a theory
is a critical goal.
However,
the requirements we have identified for
CompCert to be deployed in this context
are not met by any of its existing variants.

CompCertO is
a new extension of CompCert
which characterizes compiled program modules
in terms of their interaction with other components.
By extending the CompCert semantics
in a way that embraces relational reasoning,
we achieve this with only a minimal increase
in proof size.
\end{abstract}
%}}}

\maketitle

\section{Introduction} %{{{

\subsection{Large-scale verification} %{{{

The introduction a decade ago of
the certified compiler CompCert \cite{compcert}
represented a breakthrough
in the scale and feasability of
certified software.
Since then,
researchers have been able to formally verify the
total functional correctness
of various key components of computer systems,
including
%compilers \cite{compcert, vellvm},
operating system kernels \cite{sel4,popl15},
file systems \cite{fscq} and
processor designs \cite{safe,kami}.

Building on this legacy,
the research community is now undertaking the next challenge:
constructing large-scale, heterogenous certified systems
by using formal specifications as interfaces
between the correctness proofs of various components%, covering a range of abstraction levels
\cite{deepspec}.
To~this end,
\anon{}{we have proposed}
a synthesis of game semantics and the refinement calculus
\anon{has been proposed}{}
\cite{rbgs-cal}
as a common setting
in which various
certified component could be embedded and
made interoperable.
If successful,
this would represent a deployment of formal methods
at an unprecedented scale.

%Meanwhile,
%CompCert itself
%has been used as a platform other projects have built upon.
%For example,
%verification tools have been created with soundness proofs
%connecting to CompCert \cite{vst,verasco}.
%The composition techniques used to verify CompCert
%have been extended in various directions
%\cite{compcompcert,sepcompcert,compcertm}
%and have inspired the \emph{certified abstraction layers}
%used to verify the operating system kernel CertiKOS \cite{popl15}.

Once again,
CompCert may play a central role in this effort.
Given the importance of compilers in
the construction of present-day computer systems,
and of CompCert in the formal methods landscape,
its integration to any framework
attempting to tackle end-to-end verification
should be a litmus test.
%For the framework to be general,
%it cannot be designed around CompCert itself.
However, this new application of CompCert also
comes with its own challenges.

%}}}

\subsection{Decomposing heterogenous systems} %{{{

Existing work turns CompCert
into a platform enabling compositional verification (\S\ref{sec:rw}).
%\cite{compcompcert,sepcompcert,compcertm,vst,verasco}.
However, in most cases,
the horizon is a completed assembly program to be run as a user-level process.
%Indeed, while
%CompCertM allows the user to reason about components with C and assembly
%call interfaces in isolation and link them, its top-level correctness
%theorem only characterizes closed semantics where no calls to outside
%components remain.
%The line of work we are pursuing attempts to go beyond this horizon.
%We envision a verification infrastructure where,
%for example,
%software and hardware components could be freely combined and described
%at various levels of abstraction.
%In the context of the verification of heterogenous systems
%where software and hardware component
This becomes a limitation in the context of heterogenous systems.

For example,
consider the problem of verifying
a network interface card (NIC) driver.
The NIC and its driver are closely coupled,
but the details of their interaction
are irrelvant to the rest of the system
and should not leak into our reasoning at larger scales.
Instead,
we wish to treat them as a unit
and establish a direct relationship between calls into
the driver's C interface and network communication.
Together, the NIC and driver implement
a specification $\sigma$ of type
$\kw{Net} \rightarrow \mathcal{C}$ (see \S\ref{sec:gamesem}). %:
%\[
%  \sigma : \kw{Net} \rightarrow \mathcal{C} \,.
%\]
The driver code would be specified
($\sigma_\kw{drv}$)
and verified
at the level of CompCert semantics,
whereas device I/O primitives
($\sigma_\kw{io}$)
and the NIC
($\sigma_\kw{NIC}$)
would be specified as additional components:
\[
  \sigma_\kw{NIC} : \kw{Net} \rightarrow \kw{IO}
  \qquad
  \sigma_\kw{io} : \kw{IO} \rightarrow \mathcal{C}
  \qquad
  \sigma_\kw{drv} : \mathcal{C} \rightarrow \mathcal{C}
\]
By reasoning about their interaction,
it would be possible to establish a relationship between
the overall specification $\sigma$ and
the composition
$\sigma_\kw{drv} \circ \sigma_\kw{io} \circ \sigma_\kw{NIC}$.
Then a \emph{compiler of certified components}
would help us transport specifications and proofs
obtained with respect to the driver's C code
to the compiled code operating at the level of assembly
($\sigma' : \kw{Net} \rightarrow \mathcal{A}$).

In this paper, we show how to adapt
the semantic model and correctness proofs of CompCert
so that they can be used in this way.
Our model is not intended to reach the level of generality
required to handle all aspects of the problem above;
indeed we want it to remain tailored to CompCert's
verification as much as possible.
Instead, we envision a hierarchy
where components could be verified in suitable models,
then soundly embedded in more general ones
to be made interoperable.
To make it possible for CompCert to be used in this context,
the ability to treat the driver code as an
independent component is crucial.
This excludes approaches to
compositional compiler correctness
which are formulated in terms of completed programs.
%rather than characterizing a component's interactions
%with its environment directly.

%}}}

\subsection{Requirements for CompCert} \label{sec:compcertreq} %{{{

To handle use cases like the one we have presented above,
the compiler's correctness proof
should satisfy the following requirements:
\begin{enumerate}
\item \label{req:opensem}
  The semantics of the source and target languages
  should characterize the behavior of \emph{open} components
  in terms of their interactions with the rest of the program.
\item \label{req:opensim}
  The correctness theorem
  should go beyond refinement under a fixed notion of program context, and relate
  the interactions of the source and target modules directly.
\item \label{req:openabs}
  The abstraction gap between C and assembly-level
  interactions should be made explicit.
\item \label{req:linking}
  Some form of certified \emph{linking}
  should be provided as well as certified compilation.
\item \label{req:complexity}
  To facilitate integration into the official release,
  changes to the existing proofs of CompCert
  should be minimal.
\end{enumerate}
As outlined in Table~\ref{tbl:compcerts},
each of these requirements is fulfilled
by some exisiting CompCert extension,
however none satisfies them all.
%

%With each step,
%the user can gain more confidence in the reliability of CompCert:
%existing work %testing the correctness of existing compilers
%has found fewer bugs in CompCert,
%compared to unverified alternatives \cite{csmith},
%and efforts to make CompCert's correctness theorem more realistic
%have uncovered and removed some of the few remaining bugs \cite{sepcompcert}.
%Much of this work has focused on making
%the correctness proof of CompCert more compositional,
%making it possible to prove properties about
%programs constructed from several independently compiled units,
%and to compositionally verify programs.

%}}}

\begin{table} % tbl:compcerts {{{
  \small
  \begin{tabular}{l@{\qquad}r@{}l@{\qquad}c@{\:}c@{\:}c@{\:}c@{\:}c}
    \hline
    Variant & \multicolumn{2}{c}{Semantic model} & (1) & (2) & (3) & (4) & (5) \\
    \hline
    (Sep)CompCert \cite{compcert,sepcompcert} &
      $\chi : \mathbf{1} \twoheadrightarrow \mathcal{C}
      \vdash \mathbf{1} $ & ${} \twoheadrightarrow \mathcal{W}$ &
      & & & $\checkmark$ & $\checkmark$ \\
    CompCertX \cite{popl15} &
      $\chi : \mathbf{1} \twoheadrightarrow \mathcal{C} \times \mathcal{A}
       \vdash
       \mathbf{1} $ & ${} \twoheadrightarrow \mathcal{C} \times \mathcal{A}$
      & & & $\checkmark$ & $\checkmark$ & $\checkmark$ \\
    Compositional CompCert \cite{compcompcert} &
      $\mathcal{C}$ & ${} \twoheadrightarrow \mathcal{C}$ &
      $\checkmark$ & $\checkmark$ & & & \\
    CompCertM \cite{compcertm} &
      $\mathcal{C} \times \mathcal{A} $ & ${}\twoheadrightarrow
       \mathcal{C} \times \mathcal{A}$ &
      $\checkmark$ & & $\checkmark$ & $\checkmark$ & $\checkmark$ \\
    CompCertO &
      $A $ & ${}\twoheadrightarrow A \quad
      (A \in \mathbb{L})$ &
      $\checkmark$&$\checkmark$&$\checkmark$&$\checkmark$&$\checkmark$ \\
    \hline
  \end{tabular}
  \caption{Taxonomy of CompCert extensions
    in terms of the corresponding game models (\S\ref{sec:gamesem})
    and the requirements they satisfy (\S\ref{sec:compcertreq}).
    See \S\ref{sec:rw} for details.
    The parameter $\chi : \mathbf{1} \twoheadrightarrow \mathcal{C}$
    pre-specifies the behavior of external functions,
    whereas games on the left of arrows
    correspond to dynamic interactions.
    As a distinguishing feature,
    CompCertO's model is parametrized by a generic notion of
    \emph{language interface}
    $A \in \mathbb{L} \supseteq \{\mathcal{C}, \mathcal{A}\}$.
    %We also indicate for each variant of CompCert whether it provides:
    %(1)~open semantics;
    %(2)~open correctness;
    %(3)~explicit abstraction;
    %(4)~certified linking;
    %(5)~minimal changes.
    %The CompCertO model
    %is parametrized by the languages interfaces $A$, $B$.
  }
  \label{tbl:compcerts}
\end{table}
%}}}

\subsection{Contributions} %{{{

This paper introduces CompCertO,
the first extension of CompCert to address
all of these requirements simultaneously.
The key to this achievement
is the expressivity of our model
and our unreserved adoption of relational reasoning,
which provides flexibility in
describing complex phenomena
arising from compositional compilation,
and enables a more fine-grained separation of concerns.

We generalize CompCert semantics
to express interactions between components (\S\ref{sec:sem}),
using \emph{language interfaces}
to describe the form of these interactions
and \emph{simulation conventions}
to describe the correspondance between the interfaces
of source and target languages.
The behavior of
composite programs is specified by a
\emph{horizontal composition} operator (\S\ref{sec:sem:linker}),
which is shown to be correctly implemented
by the existing linking operator for assembly programs
(\S\ref{sec:sem:ref}).
To combine and reason about simulation proofs,
we introduce a rich \emph{simulation algebra} (\S\ref{sec:simalg})
and use it to derive our main compiler correctness statement
(\S\ref{sec:comppass}).

The simulation proofs for
most of CompCert's passes
can be updated with minimal effort.
%In the case of more complex passes,
Simulation conventions can be defined
which capture the internal invariants
used by existing proofs,
avoiding many sources of complexity found
in previous work.
To facilitate reasoning about
them, %simulation conventions,
we define \emph{CompCert Kripke logical relations}
(CKLR, \S\ref{sec:cklr}),
which unify
CompCert's memory transformations
as structure-preserving relations
over the memory model.
Our treatment of invariants is presented in \S\ref{sec:inv}
and more specialized simulation conventions
are discussed in \S\ref{sec:backend}.

We discuss related work in \S\ref{sec:rw}
and %conclude with a short evaluation of
the significance of our results in \S\ref{sec:concl}.

%[The public URL of the CompCertO repository is omitted
%for the purposes of the double-blind review process
%but has been provided in HotCRP alongside this submission.]

%}}}

%}}}

\section{Main ideas} \label{sec:mainideas} %{{{

\begin{optional}
\subsection{Principles for system construction} \label{sec:principles} %{{{

% preamble {{{

The goal of certified system design is
to create a formal description of
the system to be constructed (the program),
while ensuring
through careful analysis
that the system
will behave properly.
To this end,
we assign
to every system $p \in P$
a mathematical object $\llbracket p \rrbracket \in \mathbb{D}$
representing its behavior.
We will call the set $\mathbb{D}$ a \emph{semantic domain}.
In this section we elucidate
the structure and properties of $\mathbb{D}$
necessary to the process of builing
large-scale certified systems.

%}}}

\paragraph{Specifications and refinement} %{{{

System design starts with a set of requirements
on the behavior of the system to be constructed
(the specification).
These requirements do not capture every detail
of the eventual system,
but delineate a range of acceptable behaviors.

In refinement-based approaches,
programs and specifications are interpreted in the same
semantic domain $\mathbb{D}$,
which is equipped with a \emph{refinement} preorder
${\refby} \subseteq \mathbb{D} \times \mathbb{D}$.
The proposition $\sigma_1 \refby \sigma_2$
asserts that $\sigma_2$ is a refinement of $\sigma_1$,
and in particular a system description $p \in P$ is a correct implementation
of $\sigma \in \mathbb{D}$ when
$\sigma \refby \llbracket p \rrbracket$.

%}}}

\paragraph{Compositionality} %{{{

Complex systems are built by assembling components
whose behavior is understood,
such that their interaction achieves a desired effect.
The syntactic constructions of
the language used to describe systems
correspond to the ways in which they can be composed.

To enable compositional reasoning,
a suitable model must provide an account of
the behavior of the composite system
in terms of the behavior of its parts.
For instance,
if the language contains a binary operator
${+} : P \times P \rightarrow P$,
then the semantic domain should be equipped with
a corresponding operation
${\oplus} : \mathbb{D} \times \mathbb{D} \rightarrow \mathbb{D}$
such that
$\llbracket p_1 + p_2 \rrbracket =
 \llbracket p_1 \rrbracket \oplus \llbracket p_2 \rrbracket$.

%}}}

\paragraph{Monotonicity} %{{{

Once a component has been shown to conform to a given specification,
we want to abstract it as a ``black box''
so that further reasoning can be done in terms of
the component's specification rather than its implementation details.
To support this,
we must establish that semantic composition operators
are compatible with refinement:
\[ \sigma_1 \refby \sigma_1' \wedge
   \sigma_2 \refby \sigma_2' \Rightarrow
   \sigma_1 \oplus \sigma_2 \refby \sigma_1' \oplus \sigma_2' \,. \]
Suppose we have two components $p_1$ and $p_2$,
where $p_2$ relies on $p_1$ for its operation,
and we want to verify that their combination $p_1 + p_2$
satisfies a specification $\sigma$.
Once we verify $p_1$ against its own specification
$\sigma_1 \refby \llbracket p_1 \rrbracket$,
by the monotonicity of ${\oplus}$ it is sufficient to show that
$\sigma \refby \sigma_1 \oplus \llbracket p_2 \rrbracket$:
\[
   \sigma \:\refby\:
   \sigma_1 \oplus \llbracket p_2 \rrbracket \:\refby\:
   \llbracket p_1 \rrbracket \oplus \llbracket p_2 \rrbracket \:=\:
   \llbracket p_1 + p_2 \rrbracket
\]

%}}}

\paragraph{Abstraction} %{{{

Large-scale systems operate across multiple layers of abstraction.
Each abstraction layer defines its own understanding of the interaction
between a component and its environment.
To relate abstraction layers we need to give
an explicit account of how their formulations of the interaction
correspond to one another.
In this work we address this by defining a heterogenous version
of the refinement relation
${\refby_\mathbb{R}} \subseteq
 \mathbb{D}_1 \times \mathbb{D}_2$ between
the abstract domain $\mathbb{D}_1$ and
the more concrete one $\mathbb{D}_2$, where
$\mathbb{R}$ specifies a correspondance between
the two views of the system.

%In related work we show that for a sufficiently general
%semantic domain, the simulation convention $\mathbb{R}$
%defines a Galois connection
%$(\mathbb{D}^\sharp, {\refby})
% \galois{\gamma}{\alpha}
% (\mathbb{D}^\natural, {\refby})$
%between the abstract and concrete domains,
%which allows refinement properties to be transferred between
%the abstract and concrete domains through the defining property
%$
%    \gamma(\sigma^\sharp) \refby \sigma^\natural \Leftrightarrow
%    \sigma^\sharp \refby \alpha(\sigma^\natural)
%$.

%}}}

%\paragraph{Compilers} %{{{
%
%This is particularly relevant
%in the context of compilers.
%Between C and assembly,
%interactions across compilation units
%are understood very differently.
%At the level of C,
%cross-module interaction is defined in terms of
%function calls;
%invoking a function consists of assigning values
%to the function's parameters,
%initializing a new stack frame,
%and finally executing the function's body.
%At the assembly level, cross-module
%interactions simply consist in branching to an address
%outside the current module with
%a certain register state.
%
%In that context,
%the correpondance between the source and target semantic domains
%depends on the \emph{calling convention} used by the compiler.
%The correctness property of a C-to-assembly compiler
%can then be stated as:
%\[ \llbracket p \rrbracket_s \sqsubseteq_\mathbb{R}
%   \llbracket C(p) \rrbracket_t \,. \]
%where
%$p$ is the source program, $C(p)$ the compiler's output,
%$\llbracket - \rrbracket_s$ gives the semantics of the source language,
%$\llbracket - \rrbracket_t$ gives the semantics of the target language,
%and $\mathbb{R}$ formalizes the C calling convention.
%
%%}}}

%}}}
\end{optional}

\subsection{Game semantics} \label{sec:gamesem} %{{{

% preamble {{{

Game semantics is a form of denotational semantics which
incorporates some operational aspects.
An early success of this approach was
the formulation of the first fully abstract models
of the programming language PCF \cite{pcfajm,pcfho}.
%In this section,
%we give an overview of this line of research
%and how it can be applied in the context of CompCert.
Typically,
game semantics interpret
types as two-player games
and terms as strategies for these games.
Games describe the form of the interaction
between a program component %of the corresponding type
(the \emph{system})
and its execution context
(the \emph{environment}).
Strategies
specify which move the system plays
for all relevant positions in the game.

Positions are usually identified with sequences of moves,
and strategies with the set of positions
a component can reach.
% depending on the possible behaviors of the environment
This representation makes
game semantics similar to
trace semantics of process algebras,
but it is distinguished
by a strong polarization between
actions of the system and the environment,
and between outputs and inputs.
This confers an inherent ``rely-guarantee'' flavor
to games which facilitates compositional reasoning
\cite{cspgs}.

%}}}

\paragraph{Games} \label{sec:mainideas:gs:games} %{{{

A game is defined by a set $M$ of moves
players will choose from,
as well as a stipulation of which
sequences of moves are valid.
We focus on two-player, alternating games
where the environment plays first and
where the players
each contribute every other move.

When typesetting examples,
we underline the moves of the system.
For chess,
moves are taken in the set $\{a1 \ldots h8\} \times \{a1 \ldots h8\}$,
and a valid sequence of moves may look like:
\[ e2e4 \cdot \underline{c7c5} \cdot c2c3 \cdot \underline{d7d5} \cdots \]
The games we use to model low-level components
will rely on the following constructions.

%Most game semantics
%include additional structure
%in the description of games.
%The set of moves is usually partitioned
%into proponent and opponent moves ($M = M^\kw{O} \uplus M^\kw{P}$),
%and into questions and answers ($M = M^\que \uplus M^\ans$).
%Game models for high-order languages are often more complex.

%}}}

\paragraph{Type structure} \label{sec:mainideas:gs:types} %{{{

%While the game $\mathcal{C}$
%is extremely simple,
The expressive power of game semantics
comes from the ways in which complex games can be derived from simple ones,
and used to interpret compound types.
For example,
in the game $A \times B$
the environment initially chooses whether to play
an instance of $A$ or an instance of $B$.
The game $A \rightarrow B$ usually consists of
an instance of $B$ played
together with instances of $A$
started at the discretion of the system,
where the roles of the players are reversed.
%and which correspond to
%the multiple accesses to the argument values
%allowed by most $\lambda$-calculi.

The games we start from have a particularly simple structure.
We call each one a \emph{language interface}.
Their moves $M = M^\que \uplus M^\ans$ are partitionned into
questions $q \in M^\que$ and answers $r \in M^\ans$.
Questions correspond to function invocations
and answers return control to the caller.
Formally,
a language interface is defined as follows.

\begin{definition} \label{def:li}
A \emph{language interface} is a tuple
$A = \langle A^\que, A^\ans \rangle$, where
$A^\que$ is a set of \emph{questions} and
$A^\ans$ is a set of \emph{answers}.
\end{definition}

We focus on games of the form $A \rightarrow B$,
where $A$ and $B$ are language interfaces.
In this setting,
the valid positions of $A \rightarrow B$ are
sequences of the form:
\[
  q\ph{qpos} \cdot
    \underline{m_1}\ph{m1pos} \cdot \pt{m1pos}n_1 \cdots
    \underline{m_k}\ph{m2pos} \cdot \pt{m2pos}n_k \cdot
    \pt{qpos}\underline{r} \in
  B^\que ( {A^\que} A^\ans )^* {B^\ans}
\]
and all their prefixes.
In our context,
this describes a program component responding to
an incoming call $q$.
The component performs a series of external calls $m_1 \ldots m_k$
yielding the results $n_1 \ldots n_k$,
and finally returns from the top-level call
with the result $r$.
The arrows show the correspondance between questions and answers
%in a play of this form
but are not part of the model.

\begin{figure} % fig:abc {{{
  \figsize
  \tt
  \begin{tabular}{ll @{\hspace{3em}} rrl}
    \hline
    \underline{A.c} & int mult(int n, int p) \{ &
    \underline{A.s} & mult: & \%eax := \%ebx \\
                    & \quad return n * p; &
                    & & \%eax *= \%ecx \\
                    & \} &
                    & & ret \\
    \hline
    \underline{B.c} & int sqr(int n) \{ &
    \underline{B.s} & sqr: & \%ecx := \%ebx \\
                    & \quad return mult(n, n); &
                    & & call mult \\
                    & \} &
                    & L1: & ret \\
    \hline
  \end{tabular}
  \caption{Two simple C compilation units and corresponding assembly code.
    For this example,
    the calling convention stores arguments in
    the registers
    \texttt{\%ebx} and \texttt{\%ecx}
    and return values in
    the register
    \texttt{\%eax}.}
  \label{fig:abc}
\end{figure}
%}}}

\begin{example} \label{ex:abc} %{{{
We use a simplified version of C and assembly
to illustrate some of the principles behind our model.
Consider the program components in Fig.~\ref{fig:abc}.
The behavior of $\textsf{B.c}$
as it interacts with $\textsf{A.c}$
uses plays of the form:
\begin{equation} \label{eqn:cplay}
  \mathsf{sqr}(3) \cdot
    \underline{\mathsf{mult}(3,3)} \cdot 9 \cdot \underline{9}
\end{equation}
This corresponds to the game
$\tilde{\mathcal{C}} \rightarrow \tilde{\mathcal{C}}$
for a language interface
$\tilde{\mathcal{C}} :=
 \langle \kw{ident} \times \kw{val}^*, \kw{val} \rangle$.
Questions specify the function to invoke
and its arguments;
answers carry the value returned by the function.

To describe the behavior of \texttt{A.s} and \texttt{B.s},
we use a set of registers
$R := \{ \kw{pc}, \kw{eax}, \kw{ebx}, \kw{ecx} \}$
together with a stack of pending return addresses
($\kw{pc}$ is the program counter).
The corresponding language interface can be defined as
$\tilde{\mathcal{A}} :=
 \langle \kw{val}^R \times \kw{val}^*, \,
         \kw{val}^R \times \kw{val}^* \rangle$.
A possible execution of \texttt{B.s}
is: % described by the play:
\begin{equation} \label{eqn:splay}
{
  \figsize
  \left[
    \begin{array}{l@{{} \mapsto {}}r}
      \kw{pc}  & \kw{sqr} \\
      \kw{eax} & 42 \\
      \kw{ebx} & 3 \\
      \kw{ecx} & 7 \\
      \multicolumn{2}{c}{\textit{stack: } \hfill x \cdot \vec{k}}
    \end{array}
  \right] \cdot
  \underline{
    \left[
      \begin{array}{l@{{} \mapsto {}}r}
        \kw{pc}  & \kw{mult} \\
        \kw{eax} & 42 \\
        \kw{ebx} & 3 \\
        \kw{ecx} & 3 \\
        \multicolumn{2}{c}{\textit{stack: } \: \kw{L1} \cdot x \cdot \vec{k}}
      \end{array}
    \right]} \cdot
  \left[
    \begin{array}{l@{{} \mapsto {}}r}
      \kw{pc}  & \kw{L1} \\
      \kw{eax} & 9 \\
      \kw{ebx} & 3 \\
      \kw{ecx} & 3 \\
      \multicolumn{2}{c}{\textit{stack: } \hfill x \cdot \vec{k}}
    \end{array}
  \right] \cdot
  \underline{
    \left[
      \begin{array}{l@{{} \mapsto {}}r}
        \kw{pc}  & x \\
        \kw{eax} & 9 \\
        \kw{ebx} & 3 \\
        \kw{ecx} & 3 \\
        \multicolumn{2}{c}{\textit{stack: } \hfill \vec{k}}
      \end{array}
    \right]}
}
\end{equation}
The correspondance between (\ref{eqn:cplay}) and (\ref{eqn:splay})
is determined by the C calling convention in use.
We will discuss this point in more detail in \S\ref{sec:simconv}.
\end{example}
%}}}

%}}}

%}}}

%%\subsection{Refinement} \label{sec:mainideas:gs:ref} %{{{
%
%Existing work on denotational game semantics of
%typed functional languages
%has not placed much emphasis on the notion of refinement,
%focusing instead on program equivalence and
%the problem of full abstraction.
%While there have been attempts to integrate nondeterminism
%to game semantics \cite{gsnondet,gsndsheaves},
%we believe that this work
%is limited by a lack of distinction between angelic and demonic
%nondeterminism.
%
%The distinction between system and environment actions
%present in game-theoretical approaches
%leads to a notion of \emph{alternating} refinement:
%a behavior $x$ refines a behavior $y$ if
%all \emph{system} actions in $x$ are also possible in $y$, and if
%all \emph{environment} actions in $y$ are also possible in $x$
%\cite{altref,gmos}.
%This makes it possible for specifications to
%constrain the environment as well as the system,
%and enables a rely-guarantee style of reasoning.
%
%In the realm of imperative program semantics and verification,
%it has long been recognized that predicate transformers
%can model both \emph{angelic} and \emph{demonic} nondeterminism,
%and that this duality has important connections to games.
%In particular,
%the imperative programs and specifications
%of the \emph{refinement calculus} are understood as
%\emph{contracts} between a component and its environment,
%and their behavior is described in terms of
%an operational game semantics \cite{refcal,refcalgames}.
%The calculus incorporates both kinds of nondeterminism
%as the meet and join operations of the calculus' refinement lattice.
%
%We build on recent work extending this approach to the realm
%of functional programming \cite{augtyp,dndf}
%to incorporate dual nondeterminism to the game semantics
%we present in \S\ref{sec:gamesem}.
%
%%}}}

\subsection{CompCertO} \label{sec:mainideas:compcerto} %{{{

The semantic model of CompCert corresponds to
a game $\mathcal{E} \rightarrow \mathcal{W}$.
Programs are run without any parameters
and produce a single integer denoting their exit status.
This is described by the language interface
$\mathcal{W} := \langle \mathbbm{1}, \kw{int} \rangle$,
where $\mathbbm{1} = \{ * \}$ is the unit set
and $\kw{int}$ is the set of machine integers.
Interaction with the environment
is captured as a trace of events from a predefined set,
each with an output and input component.
These events,
described by the language interface $\mathcal{E}$,
correspond to system calls and accesses to volatile variables.

\paragraph{Semantic model} %{{{

In order to model open components and cross-component interactions,
we generalize CompCert's labelled transition systems
to describe strategies for games of the form:
\[ A \twoheadrightarrow B \: := \:
   A \times \mathcal{E} \rightarrow B \,. \]
The language interface $B$ describes how a component can be activated,
and the ways in which it can return control to the caller.
The language interface $A$ describes the external calls that the component
may perform in the course of its execution.

This flexibility allows us to treat interactions
at a level of abstraction adapted to each language.
For example, CompCertO's source language \kw{Clight} uses the game
\mbox{$\mathcal{C} \twoheadrightarrow \mathcal{C}$}.
The questions of $\mathcal{C}$ specify a function to call,
argument values,
and the state of the memory at the time of invocation;
the answers specify a return value and an updated memory state.
On the other hand, the target language \kw{Asm} uses
$\mathcal{A} \twoheadrightarrow \mathcal{A}$,
where $\mathcal{A}$ describes control transfers
in terms of processor registers
rather than function calls.
%The original whole-program model can be recovered as
%$\mathbf{1} \twoheadrightarrow \mathcal{W}$,
%where $\mathbf{1}$ represents the empty language interface.
The language interfaces used in CompCertO
are described in \S\ref{sec:sem:open}.

%This allows to accurately model assembly-level control transfers,
%which is important when verifying system code
%incorporating hand-written assembly components
%which may not follow the C calling convention.
%As demonstrated in \S\ref{sec:passes},
%this gain in expressivity
%allows us to side-step a number of issues
%that have plagued previous work on CompCert compositionality.

%}}}

\paragraph{Simulations} %{{{

CompCert uses simulation proofs
to establish a correspondence between
the externally observable behaviors of
the source and target programs of each compilation pass.
The internal details of simulation relations
have no bearing on this correspondance,
so these details can remain hidden
to fit a uniform and transitive notion of pass correctness.
This makes it easy to derive the correctness
of the whole compiler
from the correctness of each pass.

To achieve compositionality across compilation units,
our model must reveal details
about component interactions
which were previously internal.
Since many passes transform
%the memory states and runtime values which constitute
these interactions in
%non-trivial and
specialized ways,
this breaks the uniformity
of pass correctness properties.

Existing work attempts to recover this uniformity
by developing a general notion of correctness
covering all passes
%\cite{compcompcert,compcertm}
or by delaying pass composition so that
it operates on closed semantics only.
%\cite{sepcompcert,compcertm}.
Unfortunately, these techniques either
conflict with our requirement~\#2,
make proofs more complex,
or cascade into subtle ``impedence mismatch'' problems
requiring their own solutions
(see \S\ref{sec:rw}).

By contrast,
we capture the particularities of each simulation proof
by introducing a notion of \emph{simulation convention}
expressing the correspondance between
source- and target-level interactions.
To describe simulation conventions
and reason about them,
%compositionally,
we use logical relations.

%}}}

%}}}

\subsection{Logical relations} \label{sec:logrel} %{{{

Logical relations are structure-preserving relations
in the way homomorphisms are structure-preserving maps.
However,
logical relations are more compositional than homomorphisms,
because they do not suffer from the same problems
in the presence of mixed-variance constructions
like the function arrow %$\rightarrow$
\cite{lrp}.
In the context of typed languages,
this means that type-indexed logical relations
can be defined by recursion over the structure of types.

%Logical relations have found widespread use in programming language theory.
%Unary logical relations can be used to establish
%various properties of type systems:
%a type-indexed predicate expressing a property of interest
%is shown to be compatible with the language's reduction,
%and to contain all of the well-typed terms of the language.
%Binary logical relations can be used to capture
%contextual equivalence between terms,
%as well as notions such as non-interference or compiler correctness.
%Relational models of type quantification yield
%Reynold's well-known theory of relational parametricity,
%and can be used to prove \emph{free theorems} that
%all terms of a given parametric type must satisfy.

Logical relations can be of any arity,
but
we restrict our attention to
binary logical relations.
Given an algebraic structure $\mathcal{S}$,
a \emph{logical relation}
between two instances $S_1, S_2$ of $\mathcal{S}$
will be a relation $R$
between their carrier sets,
such that the corresponding operations of $S_1$ and $S_2$
take related arguments to related results.
We write $R \in \mathcal{R}(S_1, S_2)$.

\begin{example}%[Logical relation of monoids] %{{{
\label{ex:monoid}
A monoid is a set with
an associative operation $\cdot$ and
an identity element $\epsilon$.
A~\emph{logical relation of monoids} between
$\langle A, \cdot_A, \epsilon_A \rangle$ and
$\langle B, \cdot_B, \epsilon_B \rangle$
is a relation $R \subseteq A \times B$
such that:
\begin{equation}
\label{eqn:monoidrel}
(u \mathrel{R} u' \wedge v \mathrel{R} v' \: \Rightarrow \:
 u \cdot_A v \: \mathrel{R} \: u' \cdot_B v')
\: \wedge \:
\epsilon_A \mathrel{R} \epsilon_B \,.
\end{equation}
\end{example}
%}}}

Logical relations between multisorted structures
consist of one relation for each sort,
between the corresponding carrier sets.
In the case of structures which include type operators,
we can associate to each base type $A$
a relation over its carrier set $\llbracket A \rrbracket$,
and to each type operator $T(A_1, \ldots, A_n)$
a corresponding \emph{relator}:
given relations $R_1, \ldots, R_n$ over
the carrier sets $\llbracket A_1 \rrbracket, \ldots, \llbracket A_n \rrbracket$,
the relator for $T$
will construct a relation $T(R_1, \ldots, R_n)$
over $\llbracket T(A_1, \ldots, A_n) \rrbracket$.
Relators for some common constructions are shown in Fig.~\ref{fig:relators}.
In this framework, the proposition (\ref{eqn:monoidrel}) can be reformulated as:
\[
  \cdot_A \ifr{R \times R \rightarrow R} \cdot_B
  \: \wedge \:
  \epsilon_A \mathrel{R} \epsilon_B \,.
\]

\begin{example} \label{ex:simrel} %{{{
Simulation relations are
logical relations of transition systems.
Consider the transition systems
$\alpha : A \rightarrow \mathcal{P}(A)$ and
$\beta : B \rightarrow \mathcal{P}(B)$.
A simulation relation $R \in \mathcal{R}(A, B)$
satisfies:
\[
  \begin{tikzcd}[scale=0.8]
    s_1 \arrow[r, "\alpha"]
        \arrow[d, dash, "R"'] &
    s_1' \arrow[d, dashed, dash, "R"] \\
    s_2 \arrow[r, dashed, "\beta"] &
    s_2'
  \end{tikzcd}
  \qquad
  \label{eqn:simrel}
  \begin{array}{r@{\,.\,}l}
    \forall s_1 \, s_2 \, s_1' &
      \alpha(s_1) \ni s_1' \wedge s_1 \mathrel{R} s_2 \Rightarrow
    \\[0.25ex]
    \exists s_2' &
      \beta(s_2) \ni s_2' \wedge s_1' \mathrel{R} s_2'
  \end{array}
\]
Using the relators in Fig.~\ref{fig:relators},
we can express the same property
concisely and compositionally as:
\[
  \alpha \ifr{R \rightarrow \mathcal{P}^\le(R)} \beta \,.
\]
\end{example}
%}}}

\begin{figure} % fig:relators {{{
  \figsize
  \begin{align*}
    x \ifr{R_1 \times R_2} y \ \Leftrightarrow\  &
      \pi_1(x) \ifr{R_1} \pi_1(y) \wedge
      \pi_2(x) \ifr{R_2} \pi_2(y) \\
    x \ifr{R_1 + R_2} y \ \Leftrightarrow\  &
      (\exists \, x_1 \, y_1 \,.\,
        x_1 \ifr{R_1} y_1 \wedge
        x = i_1(x_1) \wedge
        y = i_1(y_1)) \\ \vee\ &
      (\exists \, x_2 \, y_2 \,.\,
        x_2 \ifr{R_2} y_2 \wedge
        x = i_2(x_2) \wedge
        y = i_2(y_2)) \\
    f \ifr{R_1 \rightarrow R_2} g \ \Leftrightarrow\  &
      \forall \, x \, y \,.\,
        x \ifr{R_1} y \Rightarrow
        f(x) \ifr{R_2} g(y) \\
    A \ifr{\mathcal{P}^\le(R)} B \ \Leftrightarrow\  &
      \forall \, x \in A \,.\,
      \exists \, y \in B \,.\,
      x \ifr{R} y
  \end{align*}
  \caption{A selection of relators}
  \label{fig:relators}
\end{figure}
%}}}

%Logical relations used to reason about contextual equivalence
%are often partial equivalence relations (PER).
%By contrast, since we mainly focus on refinement,
%most of the relations we consider will not be symmetric.

\paragraph{Kripke relations}

Since relations for stateful languages
often depend on the current state,
Kripke logical relations
are parametrized over a set of state-dependent \emph{worlds}.
Components related at the same world
are guaranteed to be related in compatible ways.
We use the following notations.

\begin{definition} \label{def:klr} %{{{
A \emph{Kripke} relation is
a family of relations $(R_w)_{w \in W}$.
We write $R \in \mathcal{R}_W(A, B)$
for a Kripke relation between the sets $A$ and $B$.
For $w \in W$ we write:
\[
\begin{array}{c@{\qquad}c}
    [w \Vdash R] \: := \: R_w &
    [\Vdash R] \: := \: \bigcap_{w} R_w
\end{array}
\]
\end{definition}
%}}}

A simple relation $R \in \mathcal{R}(A, B)$
can be promoted to a Kripke relation
$\lceil R \rceil \in \mathcal{R}_W(A, B)$
by defining $[w \Vdash \lceil R \rceil] := R$ for all $w \in W$.
More generally, for an $n$-ary relator $F$ we have:
\[
  \AxiomC{$
    F :
      \mathcal{R}(A_1, B_1) \,\times\,\cdots\,\times\,\mathcal{R}(A_n, B_n)
      \rightarrow \mathcal{R}(A, B)$}
  \UnaryInfC{$
    \lceil F \rceil :
      \mathcal{R}_W(A_1, B_1) \times \cdots \times \mathcal{R}_W(A_n, B_n)
      \rightarrow \mathcal{R}_W(A, B)$}
  \DisplayProof
\]
where for the Kripke relations $R_i \in \mathcal{R}_W(A_i, B_i)$:
\[
  [w \Vdash \lceil F \rceil (R_1, \ldots, R_n)] \: := \:
    F(w \Vdash R_1, \ldots, w \Vdash R_n)
\]
We use $\lceil - \rceil$ implicitly
when a relator appears in a context where
a Kripke logical relation is expected.
Since reasoning with logical relations
often involves self-relatedness,
we use the notation
$x :: R$ to denote $x \mathrel{R} x$.
For legibility, we will also write
$w \Vdash x \mathrel{R} y$ for $x \ifr{w \Vdash R} y$
and $\Vdash x \mathrel{R} y$ for $x \ifr{\Vdash R} y$.

%}}}

\subsection{Simulation conventions} \label{sec:simconv} %{{{

The framework of Kripke relations allows us
to define simulation conventions as follows.
The worlds ensure that corresponding pairs of
questions and answers are related consistently.

\begin{definition} \label{def:simconv} % Simulation convention %{{{
A \emph{simulation convention} between the language interfaces
$A_1 = \langle A_1^\que, A_1^\ans \rangle$ and
$A_2 = \langle A_2^\que, A_2^\ans \rangle$
is a tuple $\mathbb{R} = \langle W, \mathbb{R}^\que, \mathbb{R}^\ans \rangle$
with $\mathbb{R}^\que \in \mathcal{R}_W(A_1^\que, A_2^\que)$
and $\mathbb{R}^\ans \in \mathcal{R}_W(A_1^\ans, A_2^\ans)$.
We will write $\mathbb{R} : A_1 \Leftrightarrow A_2$.
In particular,
for a language interface $A$,
the \emph{identity} simulation convention
is defined as
$\kw{id}_A := \langle \mathbbm{1}, {=}, {=} \rangle
  : A \Leftrightarrow A$.
We will usually omit the subscript $A$.
\end{definition}
%}}}

A simulation between the transition systems
$L_1 : A_1 \twoheadrightarrow B_1$ and
$L_2 : A_2 \twoheadrightarrow B_2$
is then assigned a type $\mathbb{R}_A \twoheadrightarrow \mathbb{R}_B$,
where %the expressions
$\mathbb{R}_A : A_1 \Leftrightarrow A_2$ and
$\mathbb{R}_B : B_1 \Leftrightarrow B_2$
are simulation conventions
relating the corresponding language interfaces.
We write
$L_1 \le_{\mathbb{R}_A \twoheadrightarrow \mathbb{R}_B} L_2$.

Table~\ref{tbl:notations} presents a summary of notations.
%(notations are summarized in Table~\ref{tbl:notations}.
Simulation conventions
will often be derived from
more elementary relations,
following the internal structure of questions and answers
(see \S\ref{sec:cklr}).

\begin{example} %{{{
The calling convention we used in Example~\ref{ex:abc}
can be formalized as a simulation convention
$\tilde{\mathbb{C}} :=
  \langle \kw{val}^*, \tilde{\mathbb{C}}^\que, \tilde{\mathbb{C}}^\ans \rangle :
    \tilde{\mathcal{C}} \Leftrightarrow \tilde{\mathcal{A}}$.
We use the set of worlds $\kw{val}^*$
to relate the stack of
assembly questions to that of the corresponding answers.
The relations $\tilde{\mathbb{C}}^\que, \tilde{\mathbb{C}}^\ans$
are defined by:
%can be expressed using the rules:
\[
  \AxiomC{$\mathit{rs}[\kw{pc}] = f$}
  \AxiomC{$\vec{v} \text{ is is contained in } \mathit{rs}[\kw{ebx}, \kw{ecx}]$}
  \BinaryInfC{$
    x \vec{k} \Vdash
    f(\vec{v}) \mathrel{\tilde{\mathbb{C}}^\que} \mathit{rs}@ x \vec{k}
  $}
  \DisplayProof
  \qquad
  \AxiomC{$\mathit{rs}[\kw{eax}] = v'$}
  \AxiomC{$\mathit{rs}[\kw{pc}] = x$}
  \BinaryInfC{$
    x \vec{k} \Vdash
    v' \mathrel{\tilde{\mathbb{C}}^\ans} \mathit{rs}@\vec{k}
  $}
  \DisplayProof
\]
For a C-level function invocation $f(\vec{v})$,
we expect the register $\kw{pc}$ to point to
the beginning of the function $f$,
and the registers $\kw{ebx}$ and $\kw{ecx}$
to contain the first and second arguments (if applicable).
Other registers may contain arbitrary values.
The stack $x \vec{k}$ has no relationship to the C question,
however the assembly answer is expected to pop the return address
and branch to it, setting the program counter $\kw{pc}$ accordingly.
In addition,
the return value $v'$ must be stored
in the register $\kw{eax}$.
\end{example}
%}}}

\begin{table} % tbl:notations {{{
  \figsize
  \begin{tabular}{lcl}
    \hline
    Notation & Examples & Description \\
    \hline
    $R \in \mathcal{R}(S_1, S_2)$ &
      $\vref$ &
      Simple relation \\
    $R \in \mathcal{R}_W(S_1, S_2)$ &
      $\hookrightarrow_\kw{m}$ &
      Kripke relation (Def.~\ref{def:klr}) \\
    $w \Vdash R$ & &
      Kripke relation at world $w$ \\
    $w \Vdash x \mathrel{R} y$ & &
      $x$ and $y$ related at world $w$ \\
    \hline
    $A, B, C$ &
      $\mathcal{C}, \mathcal{A}, \mathbf{1}$ &
      Language interface (Def.~\ref{def:li}) \\
    %$M_A^\que, M_A^\ans$ & &
    %  Questions and answers of $A$ \\
    $L : A \twoheadrightarrow B$ &
      $\kw{Clight}(p)$ &
      LTS for $A \twoheadrightarrow B$ (Def.~\ref{def:lts}) \\
    $\mathbb{R} : A_1 \Leftrightarrow A_2$ &
      $\kw{alloc}, \mathbf{R}_\mathcal{C}$ &
      Simulation convention (Def.~\ref{def:simconv}) \\
    $L_1 \le_{\mathbb{R} \twoheadrightarrow \mathbb{S}} L_2$ &
      Thm.~\ref{thm:compc} &
      Simulation property (Def.~\ref{def:fsim}) \\
    $L_1 \oplus L_2$ & &
      Horizontal composition (Def.~\ref{def:hcomp}) \\
    %$p_1 + p_2$ & &
    %  Program linking \\
    %$\mathbf{R}$ &
    %  $\kw{ext}, \kw{inj}$ &
    %  CKLR (Def.~\ref{def:cklr}) \\
    \hline
  \end{tabular}
  \caption{Summary of notations}
  \label{tbl:notations}
\end{table}
%}}}

The expressive power of simulation conventions
makes the adaptation of existing correctness proofs
for the various passes of CompCert straightforward.
Instead of forcing all passes into a single one-size-fits-all mold,
we can choose conventions matching
the simulation relation and invariants
used in each pass.
Simulations for each pass can then be composed
in the way shown in Fig.~\ref{fig:simcomp}.

\begin{figure} % fig:simcomp {{{
  \figsize
  $\begin{array}{c}
    \kw{id} : A \Leftrightarrow A
    \qquad
    L \le_{\kw{id} \twoheadrightarrow \kw{id}} L
    \\[1.5em]
    \AxiomC{$\mathbb{R} : A_1 \Leftrightarrow A_2$}
    \AxiomC{$\mathbb{S} : A_2 \Leftrightarrow A_3$}
    \BinaryInfC{$\mathbb{R} \cdot \mathbb{S} : A_1 \Leftrightarrow A_3$}
    \DisplayProof
    \\[1.5em]
    \AxiomC{$L_1 \le_{\mathbb{R}_A \twoheadrightarrow \mathbb{R}_B} L_2$}
    \AxiomC{$L_2 \le_{\mathbb{S}_A \twoheadrightarrow \mathbb{S}_B} L_3$}
    \BinaryInfC{$
      L_1 \le_{\mathbb{R}_A \cdot \mathbb{S}_A \twoheadrightarrow
               \mathbb{R}_B \cdot \mathbb{S}_B} L_3$}
    \DisplayProof
  \end{array}
  \qquad
  \begin{tikzpicture}[baseline,scale=0.66]
    \node (A1) at (-1,  2) {$A_1$};
    \node (A2) at (-1,  0) {$A_2$};
    \node (A3) at (-1, -2) {$A_3$};
    \node (B1) at ( 1,  2) {$B_1$};
    \node (B2) at ( 1,  0) {$B_2$};
    \node (B3) at ( 1, -2) {$B_3$};
    \draw (A1) edge[->>] node[auto] {$L_1$} (B1);
    \draw (A2) edge[->>] node[auto] {$L_2$} (B2);
    \draw (A3) edge[->>] node[auto] {$L_3$} (B3);
    \begin{scope}[double equal sign distance, {Implies[]}-{Implies[]}]
      \draw (A1) edge[double] node[auto,swap] {$\mathbb{R}_A$} (A2);
      \draw (B1) edge[double] node[auto] {$\mathbb{R}_B$} (B2);
      \draw (A2) edge[double] node[auto,swap] {$\mathbb{S}_A$} (A3);
      \draw (B2) edge[double] node[auto] {$\mathbb{S}_B$} (B3);
    \end{scope}
  \end{tikzpicture}
  $
  \caption{Simulation identity and vertical composition}
  \label{fig:simcomp}
\end{figure}
%}}}

%}}}

\subsection{Simulation algebra} \label{sec:mainideas:simalg} %{{{

CompCert's \emph{injection} passes (see \S\ref{sec:cklr})
pose a particular challenge,
already encountered
in the work on Compositional CompCert
\cite{compcompcert}:
injection passes
make stronger assumptions on external calls
than they guarantee for incoming calls.
Our framework
can express this situation by using
different simulation conventions
for external and incoming calls
($\kw{injp} \twoheadrightarrow \kw{inj}$).
However,
since
the external calls of one component and
the incoming calls of another
will not be related in compatible ways,
this asymmetry breaks %the simulation's
horizontal compositionality (Thm.~\ref{thm:fsim-hcomp}).

In CompCertO,
we rectify this imbalance \emph{outside}
of the simulation proof itself.
Simulations for individual passes are not always
horizontally compositional,
but we are able to derive a symmetric simulation convention
for the compiler as a whole.
Properties of the $\kw{Clight}$ and $\kw{RTL}$ languages
allow us to strengthen the correctness proof.
These properties are encoded as self-simulations
and inserted as pseudo-passes.
We can then perform algebraic manipulations
on simulation statements
to rewrite the overall simulation convention
used by the compiler into a symmetric one.

These algebraic manipulations are
based on a notion of
simulation convention refinement ($\scref$)
allowing a simulation convention
to replace another in all simulation statements.
We construct a typed Kleene algebra \cite{tka}
based on this order,
and use it to ensure that
our compiler correctness statement
is both compositional
and insensitive to the inclusion of optional passes.

%}}}

%}}}

\section{Operational semantics} \label{sec:sem} %{{{

%This section describes CompCertO's semantic infrastructure.
%In \S\ref{sec:sem:overview}--\ref{sec:sem:mm},
%we review the techniques used in CompCert
%to formalize the semantics of the various languages involved,
%then starting with \S\ref{sec:sem:open}
%we describe the extensions we use
%to model the behavior of \emph{open} programs.

\subsection{Whole-program semantics in CompCert} \label{sec:sem:closed} %{{{

The semantics of CompCert languages
are given in terms of a simple notion of process behavior.
By \emph{process}, we mean a self-contained computation
which can be characterized by
the sequence of system calls it performs.
%Several steps are involved to go
%from a source program to an executing process;
%these steps are modeled by CompCert
%with varying degrees of sophistication and accuracy.
For a C program to be executed as a process,
its translation units must be compiled to object files,
then linked together
into an executable binary
loaded by the system.

\begin{optional}
\begin{figure} % fig:process {{{
    \begin{tikzpicture}[scale=1.25]
        \tt
        \node (Ms) at (1.5, 1) {whole-program.s};
        \node (M1c) at (0, 3) {M1.c};
        \node (M2c) at (1, 3) {M2.c};
        \node (etc) at (2, 3) {\ldots};
        \node (Mnc) at (3, 3) {M$n$.c};
        \node (M1s) at (0, 2) {M1.s};
        \node (M2s) at (1, 2) {M2.s};
        \node (ets) at (2, 2) {\ldots};
        \node (Mns) at (3, 2) {M$n$.s};
        \draw (M1c) edge[->] (M1s);
        \draw (M2c) edge[->] (M2s);
        \draw (Mnc) edge[->] (Mns);
        \draw (M1s) edge[->] (Ms);
        \draw (M2s) edge[->] (Ms);
        \draw (Mns) edge[->] (Ms);
        \rm
        \node at (-1.3, 2.5) {Compilation};
        \node at (-1.3, 1.5) {Linking};

        %\node (Mb) at (1.5, 0) {Running process};
        %\node at (-1.3, 0.5) {Loading};
        %\draw (Ms) edge[->] (Mb);

        %\tt
        %\node (Mc) at (1.5, 4) {whole-program.c};
        %\draw (M1c) edge[->,dotted] (Mc);
        %\draw (M2c) edge[->,dotted] (Mc);
        %\draw (Mnc) edge[->,dotted] (Mc);
    \end{tikzpicture}
    \caption{CompCert's approximation of the C toolchain}
    \label{fig:process}
\end{figure}
%}}}
\end{optional}

%Finally, the system loads the executable into memory
%where initialization code sets up the process
%and invokes the program's \texttt{main} function.
%Since CompCert is a compiler from C to assembly,
%but does not include an assembler or linker,
The model used for verifying CompCert accounts for this
\opt{in the way depicted in Fig.~\ref{fig:process}.}%
    {in the following way.}
%The lowest-level objects considered
%are CompCert \kw{Asm} programs.
Linking is approximated by
merging programs, seen as sets of global definitions.
The execution
of a program composed of the translation units
$\texttt{M1.c} \ldots \texttt{M$n$.c}$
which compile to
$\texttt{M1.s} \ldots \texttt{M$n$.s}$
is modeled as:
\[
    L_\kw{tgt} :=
    \kw{Asm}(\texttt{M1.s} +
             \cdots +
             \texttt{M$n$.s}) \,.
\]
Here,
$+$ denotes CompCert's linking operator and
$\kw{Asm}$ maps an assembly program to its semantics.
Note that the loading process is encoded
as part of the definition of $\kw{Asm}$,
which constructs a global environment
laying out the program's code and static data
into the runtime address space,
and models the conventional invocation of \texttt{main}.
To formulate compiler correctness,
we must also specify the behavior of the source program.
To this end,
CompCert defines a linking operator
and semantics
for the language $\kw{Clight}$,%
\footnote{
  Although CompCert features a frontend for a richer version
  of the C language,
  the simplified intermediate dialect \kw{Clight}
  is usually used as the source language
  when using CompCert to build certified artifacts.
  %In particular it is the target language of the
  %\texttt{clightgen} utility which
  %produces Coq code for a C program's AST,
  %which the user can then prove to be correct
  %either manually or with tools such as the VST program logic
  %\cite{vst}.
}
allowing the desired behavior to be specified as:
\[
    L_\kw{src} :=
    \kw{Clight}(\texttt{M1.c} + \cdots + \texttt{M$n$.c}) \,.
\]
The correctness of CompCert
can then be stated as $L_\kw{src} \refby L_\kw{tgt}$.

\paragraph{Transition systems} %{{{

In the original CompCert, language semantics are
given as labelled transition systems (LTS),
which characterize a program's behavior in terms of
sequences of observable events.
Schematically, a CompCert LTS
is a tuple
$L = \langle S, I, {\rightarrow}, F \rangle$
consisting of
a set of states $S$,
a subset $I \subseteq S$ of initial states,
a labelled transition relation
${\rightarrow} \subseteq S \times \mathbb{E}^* \times S$,
and a set
$F \subseteq S \times \kw{int}$
of final states associated with exit statuses.
The relation~$s \stackrel{t}{\rightarrow} s'$
indicates that the state $s$ may transition to the state $s'$
through an interaction $t \in \mathbb{E}^*$.
%The possible behaviors of CompCert LTS fall into four categories:
%\begin{itemize}
%\item An execution reaching a final state is said to
%    \emph{terminate}.
%    For example,
%    the following execution generates
%    the event trace $t_1 t_2 \cdots t_{n-1}$
%    and terminates with status $r$:
%    \[
%        I \ni s_1 \stackrel{t_1}{\rightarrow}
%          s_2 \stackrel{t_2}{\rightarrow}
%          \cdots \stackrel{t_{n-1}}{\rightarrow}
%          s_n \mathrel{F} r \,.
%    \]
%\item An execution reaching
%    an infinite sequence of $\epsilon$ transitions
%    is said to \emph{silently diverge}.
%    The following execution diverges after
%    generating the trace $t_1 t_2 \cdots t_{n-1}$:
%    \[
%        I \ni s_1 \stackrel{t_1}{\rightarrow}
%          s_2 \stackrel{t_2}{\rightarrow}
%          \cdots \stackrel{t_{n-1}}{\rightarrow}
%          s_n \stackrel{\epsilon}{\rightarrow}
%          s_{n+1} \stackrel{\epsilon}{\rightarrow}
%          \cdots
%    \]
%\item By contrast,
%    infinite executions which keep interacting
%    are said to exhibit \emph{reactive} behavior.
%    The following execution
%    is reactive if and only if
%    $\forall i \, \exists j \,.\, i \le j \wedge t_j \ne \epsilon$:
%    \[
%        I \ni s_1 \stackrel{t_1}{\rightarrow}
%          s_2 \stackrel{t_2}{\rightarrow}
%          s_3 \stackrel{t_3}{\rightarrow}
%          \cdots
%    \]
%\item Finally, an execution which reaches a stuck state
%    is said to \emph{go wrong}. It will have the shape:
%    \[
%        I \ni s_1 \stackrel{t_1}{\rightarrow}
%          s_2 \stackrel{t_2}{\rightarrow}
%          \cdots \stackrel{t_{n-1}}{\rightarrow}
%          s_n \,,
%    \]
%    with no $t, s'$ such that
%    $s_n \stackrel{t}{\rightarrow} s'$.
%    This models \emph{undefined behavior}
%    and can be refined by any behavior
%    admitting $t_1 t_2 \cdots t_{n-1}$ as a prefix.
%\end{itemize}

%The model outlined above
%makes no attempt to model interactions across components,
%and is only ever used to
%describe the behavior of whole programs.
%This makes it challenging to reason about
%the behavior of individual compilation units,
%although there have been successful attempts in this direction
%\cite{sepcompcert,compcertm}.
%In any case,
%the compiler's correctness property
%as described in \S\ref{sec:sem:intro}
%only considers uses which follow
%the pattern approximated in Fig.~\ref{fig:process}.
%
%To formulate a more fine-grained and flexible
%version of the correctness theorem of CompCert,
%we need an account of
%the behavior of individual translation units.
%The model presented in \S\ref{sec:sem:open}
%achives this by recoding control transfers
%to and from the modeled system explicitly.
%These control transfer will need to expose more information
%about the internal structure of program states;
%in the following section we review the design of
%CompCert's memory model,
%which all program states are constructed from.

The construction of states in CompCert language semantics
follow common patterns.
In particular,
all languages start with
the same notion of \emph{memory state}.

%}}}

\paragraph{Memory model} \label{sec:sem:mm} %{{{

The CompCert memory model \cite{compcertmm,compcertmmv2}
is the core algebraic structure
underlying the semantics of CompCert's languages.
Some of its operations
are shown in Fig.~\ref{fig:mm}.
The idealized version presented here
involves
the type of memory states \kw{mem},
the type of runtime values \kw{val}, and
the types of pointers \kw{ptr} and address ranges \kw{ptrrange}.
To keep our exposition concise and clear,
we gloss over the technical details
associated with modular arithmetic and overflow constraints.

\begin{figure} % fig:mm (The CompCert memory model) {{{
  \figsize
  \[
    v \in \kw{val} ::=
          \kw{undef} \mid
          \kw{int}(n) \mid
          \kw{long}(n) \mid
          \kw{float}(x) \mid
          \kw{single}(x) \mid
          \kw{vptr}(b, o)
  \]
  \[
    (b, o) \in \kw{ptr} =
      \kw{block} \times \mathbb{Z}
    \qquad
    (b, l, h) \in \kw{ptrrange} =
      \kw{block} \times \mathbb{Z} \times \mathbb{Z}
  \]
  \begin{align*}
    \kw{alloc} &:
      \kw{mem} \rightarrow \mathbb{Z} \rightarrow \mathbb{Z} \rightarrow
      \kw{mem} \times \kw{block}
    \\
    \kw{free} &:
      \kw{mem} \rightarrow
      \kw{ptrrange} \rightarrow
      \kw{option}(\kw{mem})
    \\
    \kw{load} &:
      \kw{mem} \rightarrow \kw{ptr} \rightarrow \kw{option}(\kw{val})
    \\
    \kw{store} &:
      \kw{mem} \rightarrow \kw{ptr} \rightarrow \kw{val} \rightarrow \kw{option}(\kw{mem})
%    \\
%    \kw{perm} &:
%      \kw{mem} \rightarrow \kw{ptr} \rightarrow \mathcal{P}(\kw{perm})
  \end{align*}
  \caption{Outline of the CompCert memory model}
  \label{fig:mm}
\end{figure}
%}}}

The memory is organized into a finite number of \emph{blocks}.
Each memory block has a unique identifier $b \in \kw{block}$
%represented as a positive integer,
and is equipped with its own linear address space.
Block identifiers and offsets are often manipulated together
as pointers $p = (b, o) \in \kw{ptr} = \kw{block} \times \mathbb{Z}$.
New blocks are created with prescribed boundaries
using the primitive $\kw{alloc}$.
A runtime value $v \in \kw{val}$ can be stored at
a given address using the primitive \kw{store},
and retreived using the primitive \kw{load}.
Values can be integers (\kw{int}, \kw{long}) and
floating point numbers (\kw{float}, \kw{single})
of different sizes,
as well as pointers (\kw{vptr}).
The special value \kw{undef}
represents an undefined value.
Simulation relations
often allow $\kw{undef}$
to be refined into a more concrete value;
we write value refinement as
${\vref} := \{(\kw{undef}, v), (v, v) \mid v \in \kw{val}\}$.

The memory model is shared by all of the languages in CompCert.
States always consist of
a memory component $m \in \kw{mem}$,
alongside language-specific components
which may contain additional values ($\kw{val}$).

%}}}

%}}}

\subsection{Open semantics in CompCertO} \label{sec:sem:open} %{{{

The memory model also plays a central role
when describing interactions between program components.
In our approach, %to compositional semantics,
the memory state is passed %back-and-forth
alongside all control transfers.

\paragraph{Language interfaces} %{{{

\begin{table} % tbl:li Language interfaces {{{
  \figsize
  \begin{tabular}{clll}
    \hline
    Name & Question & Answer & Description \\
    \hline
    $\mathcal{C}$ &
      $\mathit{vf}[\mathit{sg}](\vec{v})@m$ & $v'@m'$ &
      C calls \\
    $\mathcal{L}$ &
      $\mathit{vf}[\mathit{sg}](\mathit{ls})@m$ & $\mathit{ls}'@m'$ &
      Abstract locations \\
    $\mathcal{M}$ &
      $\mathit{vf}(\mathit{sp},\mathit{ra},\mathit{rs})@m$ & $\mathit{rs}'@m'$ &
      Machine registers \\
    $\mathcal{A}$ &
      $\mathit{rs}@m$ & $\mathit{rs}'@m'$ &
      Arch-specific \\
    $\mathbf{1}$ & n/a & n/a &
      Empty interface \\
    $\mathcal{W}$ & * & $r$ &
      Whole-program \\
    \hline
  \end{tabular}
  \caption{Language interfaces used in CompCertO}
  \label{tbl:li}
\end{table}
%}}}

Our models of cross-component interactions in CompCert languages
are shown in Table~\ref{tbl:li}.
At the source level ($\mathcal{C}$),
questions consist of
the address of the function being invoked
($\mathit{vf} \in \kw{val}$),
its signature
($\mathit{sg} \in \kw{signature}$),
the values of its arguments
($\vec{v} \in \kw{val}^*$),
and the state of the memory at the point of entry
($m \in \kw{mem}$);
answers
consist of the function's return value
and the state of the memory at the point of exit.
%In these terms,
%a Clight component describes a strategy for the game
%$\mathcal{C} \twoheadrightarrow \mathcal{C}$.
%The component is activated by an incoming $\mathcal{C}$ question
%in the right-hand side game,
%where it is expected to produce a $\mathcal{C}$ answer.
%In the meantime,
%it may perform external calls by
%asking questions and expecting answers
%in the left-hand side $\mathcal{C}$ game.
This language interface is used for \kw{Clight} and
for the majority of CompCert's intermediate languages.
%including \kw{RTL} which is used to perform
%most of the compiler's optimizations.

As we move towards lower-level languages,
this is reflected in the language interfaces we use:
function arguments are first mapped into
abstract locations alongside local temporary variables
($\mathcal{L}$,~used by \kw{LTL} and \kw{Linear}).
These locations are eventually split between
in-memory stack slots and a fixed number of machine registers
($\mathcal{M}$, used by \kw{Mach}).
Finally, the target assembly language \kw{Asm}
stores the program counter, stack pointer,
and return address into their own machine registers,
which is reflected in its interface $\mathcal{A}$.

The interface of whole-program execution
can also be described in this setting:
the language interface $\mathbf{1}$ contains no move;
per \S\ref{sec:mainideas:compcerto},
the interface $\mathcal{W}$ has a single trivial question $*$,
and the answers $r \in \kw{int}$
give the exit status of a process.
%XXX already said that, refer to earlier
Hence the original CompCert semantics described in
\S\ref{sec:sem:closed}
can be seen to define strategies for
$\mathbf{1} \twoheadrightarrow \mathcal{W}$:
the process can only be started in a single way,
cannot perform any external calls,
and indicates an exit status upon termination.

%}}}

\paragraph{Transition systems} %{{{

To account for the cross-component interactions
described by language interfaces,
CompCertO extends
the transition systems described in \S\ref{sec:sem:closed}
as follows.

\begin{definition} \label{def:lts}
Given an \emph{incoming} language interface $B$
and an \emph{outgoing} language interface $A$,
a \emph{labelled transition system for the game $A \twoheadrightarrow B$}
is a tuple $L = \langle S, \rightarrow, D, I, X, Y, F \rangle$.
The relation
${\rightarrow} \subseteq S \times \mathbb{E}^* \times S$ is
a \emph{transition relation} on the set of states $S$.
The set $D \subseteq B^\que$ specifies which
questions the component accepts;
$I \subseteq D \times S$ then
assigns to each one a set of \emph{initial states}.
$F \subseteq S \times B^\ans$
designates \emph{final states} together with corresponding answers.
External calls are specified by
$X \subseteq S \times A^\que$,
which designates \emph{external states} together with
a question of $A$, and
$Y \subseteq S \times A^\ans \times S$,
which is used to select a \emph{resumption state}
to follow an external state
based on the answer provided by the environment.
We write $L : A \twoheadrightarrow B$ when
$L$ is a labelled transition system for $A \twoheadrightarrow B$.
\end{definition}

We use infix notation for the various transition relations
$I, X, Y, F$.
In particular we write $n \mathrel{Y}^s s'$
to denote that $n \in A^\ans$
resumes the suspended external state $s$
to continue with state $s'$.
%
%The interpretation of the generalized LTS described above
%follows the one given in \S\ref{sec:sem:closed},
%with interactions over the game $A$
%as a new source of observable actions.
The main reason for treating
events $e \in \mathbb{E}$ and
external calls $m n \in A^\que A^\ans$
differently is that
while events are expected to be the same
between the source and target programs,
the form of external calls varies significantly
across languages
and the simulation convention they follow
must be defined explicitly.
In addition,
while events and event traces
bundle together the output and input
components of the interaction,
our representation of external calls
separates them,
which simplifies the formulation of
horizontal composition and open simulations.

%For a small-step semantics
%$L = \langle S, {\rightarrow}, I, X, Y, F \rangle : \kw{semantics}(A,B)$,
%we recognize those states using the predicate:
%\[
%    \kw{stuck}_L(s) :=
%      ({\rightarrow}(s) = \varnothing) \wedge
%      (X(s) = \varnothing) \wedge
%      (F(s) = \varnothing)
%\]
%Taking this into account,
%the immediate behavior of a state $s \in S$
%can be expressed as the interactive computation
%$\kw{step}_L(s) : \mathcal{I}_{M_A^\que,M_A^\ans}(S)$
%defined as follows:
%\[
%  \kw{step}_L(s) :=
%    \begin{cases}
%      \top & \mbox{if } \kw{stuck}(s) \\
%      {\rightarrow}(s) \sqcup
%      (X(s) \bind \mathbf{I} \bind Y(s)) & \mbox{otherwise,}
%   \end{cases}
%\]
%The overall external behavior of $L$
%can then be given as
%$
%    \llbracket L \rrbracket :
%      M_B^\que \rightarrow \mathcal{I}_{M_A^\que,M_A^\ans}(M_B^\ans)
%$
%defined by:
%%Then $\llbracket L \rrbracket$ can be defined as:
%\begin{align*}
%  \llbracket L \rrbracket (q) :=
%    \begin{cases}
%       \top & \mbox{if } I(q) = \varnothing \\
%       I(q) \bind \kw{step}^\infty \bind F & \mbox{otherwise.}
%     \end{cases}
%\end{align*}

%}}}

\paragraph{Horizontal composition} \label{sec:sem:linker} %{{{

To model linking,
we need to express the external behavior
of a collection of components
in terms of the behaviors of
individual components.
The operator we use for this purpose
can be described in the following way.

Consider the components $L_1, L_2 : A \twoheadrightarrow A$.
When $L_1$ is running and performs an external call
to one of the functions implemented by $L_2$,
the execution of $L_1$ is suspended.
The question of $L_1$ to $L_2$
is used to initialize a new state for $L_2$,
and $L_2$ becomes the active component.
Once $L_2$ reaches a final state,
the corresponding answer is used to resume
the execution of $L_1$.
In the process
$L_2$ may itself perform cross-component calls,
instantiating \emph{new} executions of $L_1$.
Therefore,
in addition to the state of the active component,
we need to maintain a \emph{stack} of suspended states
for component instances awaiting resumption.
The corresponding transition system
is described in Fig.~\ref{fig:hcomp}.

\begin{figure} % fig:hcomp {{{
  \figsize
  \begin{tikzpicture}[xscale=0.5,yscale=0.7,
      baseline=(current bounding box.center)]
    \node[draw] (L1) at (0, 1) {$L_1$};
    \node[draw] (L2) at (0,-1) {$L_2$};
    \begin{scope}[circle, minimum size=3, inner sep=0]
      \node[draw, fill=black] (X) at (-2,0) {};
      \node[draw, fill=black] (I) at (+2,0) {};
    \end{scope}
    \tiny
    \node[below right] at (I) {$\kw{inc}$};
    \draw (L1) -| (-1,0);
    \draw (L2) -| (-1,0) -- (X);
    \draw[->] (X) -- node[below] {$\kw{ext}$} (-3,0);
    \draw[->] (X) |- (0,2) node[above] {$\kw{push}/\kw{pop}$} -| (I);
    \draw (3,0) -- (1,0);
    \draw[->] (1,0) |- (L1);
    \draw[->] (1,0) |- (L2);
  \end{tikzpicture}
  \qquad
  \begin{minipage}{0.66\textwidth}
    \begin{gather*}
        \AxiomC{$q \in D_i$}
        \AxiomC{$q \mathrel{I_i} s$}
        \RightLabel{\tiny $\kw{inc}^\que$}
        \BinaryInfC{$q \mathrel{I} (i, s)$}
        \DisplayProof
        \quad
        \AxiomC{$s \stackrel{t}{\rightarrow}_i s'$}
        \RightLabel{\tiny $\kw{run}$}
        \UnaryInfC{$
            (i, s) \, k
            \stackrel{t}{\rightarrow}
            (i, s') \, k$}
        \DisplayProof
        \quad
        \AxiomC{$s \mathrel{F_i} r$}
        \RightLabel{\tiny $\kw{inc}^\ans$}
        \UnaryInfC{$(i, s) \mathrel{F} r$}
        \DisplayProof
        \\[1ex]
        \AxiomC{$s \mathrel{X_i} q$}
        \AxiomC{$q \in D_j$}
        \AxiomC{$q \mathrel{I_j} s'$}
        \RightLabel{\tiny $\kw{push}$}
        \TrinaryInfC{$
            (i, s) \, k
            \stackrel{\epsilon}{\rightarrow}
            (j, s') \, (i, s) \, k$}
        \DisplayProof
        \quad
        \AxiomC{$s' \mathrel{F_j} r$}
        \AxiomC{$r \mathrel{Y_i^s} s''$}
        \RightLabel{\tiny $\kw{pop}$}
        \BinaryInfC{$
            (j, s') \, (i, s) \, k
            \stackrel{\epsilon}{\rightarrow}
            (i, s'') \, k$}
        \DisplayProof
        \\[1ex]
        \AxiomC{$s \mathrel{X_i} q$}
        \AxiomC{$\forall j \,.\, q \notin D_j$}
        \RightLabel{\tiny $\kw{ext}^\que$}
        \BinaryInfC{$(i, s) \, k \mathrel{X} q$}
        \DisplayProof
        \qquad
        \AxiomC{$r \mathrel{Y_i^s} s'$}
        \RightLabel{\tiny $\kw{ext}^\ans$}
        \UnaryInfC{$r \mathrel{Y^{(i, s) \, k}} (i, s') \, k$}
        \DisplayProof
    \end{gather*}
  \end{minipage}
    \caption{Horizontal composition of open semantics.
      The state is a stack of alternating activations
      of the two components,
      initialized as a singleton by an incoming question ($\kw{inc}^\que$).
      During normal execution ($\kw{run}$),
      the top-level state is updated.
      Calls to functions provided by the other component ($\kw{push}$)
      are handled by pushing a new state onto the stack,
      initialized to handle the call in question.
      If a final state is reached
      while there are suspended activations ($\kw{pop}$),
      the result is used to resume the most recent one.
      External calls which are provided by neither component
      ($\kw{ext}^\que$, $\kw{ext}^\ans$),
      and final states encountered at the top level
      ($\kw{inc}^\ans$),
      are simply passed along to the environment.
    }
    \label{fig:hcomp}
\end{figure}
%}}}

\begin{definition}[Horizontal composition] \label{def:hcomp} %{{{
For two transition systems $L_1, L_2 : A \twoheadrightarrow A$
with
$L_i = \langle S_i, {\rightarrow}_i, D_i, I_i, X_i, Y_i, F_i \rangle$,
the \emph{horizontal composition} of $L_1$ and $L_2$
is defined as:
\[
    L_1 \oplus L_2 :=
    \langle
      (S_1 + S_2)^*, {\rightarrow}, D_1 \cup D_2, I, X, Y, F
    \rangle
\]
where the components $\rightarrow$, $I$, $X$, $Y$, $F$
are defined by
the rules shown in Fig.~\ref{fig:hcomp}.
\end{definition}
%}}}

%}}}

%}}}

\subsection{Open simulations} \label{sec:sem:ref} %{{{

CompCert is proved correct using a simulation
between the transition semantics of the source and target programs.
This \emph{forward}%
\footnote{In this usage, \emph{forward} pertains to
  the compilation process,
  rather than the execution of programs.}
simulation is used to establish a \emph{backward} simulation.
Backward simulations
are in turn proved to be sound with respect to trace containement.
We have updated forward and backward simulations to
work with CompCertO's semantic model.
In this section we present forward simulations,
which are used as our primary notion of refinement.
%in the remainder of the paper.

%The set of worlds $W$ is used to
%make sure that questions on one hand,
%and answers on the other hand,
%are related consistently.
%For instance,
%in CompCertO's overall simulation convention
%$\mathbb{C} : \mathcal{C} \Leftrightarrow \mathcal{A}$,
%the registers used to store a function's return value
%may depend on the signature $\kw{sg}$ used for the call.
%As another example,
%when an injection pass makes an external call,
%we need to make sure the call's final memory states
%are related by the same memory injection that was used
%for the initial states (or a successor, see \S\ref{sec:corr:inj}).

A forward simulation asserts that any transition in the source program
has a corresponding transition sequence in the target.
The sequence may be empty,
but to ensure the preservation of silent divergence
this can only happen for finitely many consecutive source transitions.
This is enforced by indexing the simulation relation
over a well-founded order,
and requiring the index to decrease
whenever an empty transition sequence is used.
This mechanism is unchanged in CompCertO
and is largely orthogonal to the techniques we introduce,
so we omit this aspect of forward simulations
in our exposition below.

\paragraph{Forward simulations} %{{{

Our transition systems introduce
various forms of external communication,
which must be taken into account by our notions of simulation.
In CompCertO,
a forward simulation between the small-step semantics
$L_1 : A_1 \rightarrow B_1$ and
$L_2 : A_2 \rightarrow B_2$
operates in the context of the simulation conventions
$\mathbb{R}_A : A_1 \Leftrightarrow A_2$ and
$\mathbb{R}_B : B_1 \Leftrightarrow B_2$.

As depicted in Fig.~\ref{fig:fsim},
if questions of $B_1$ and $B_2$
respectivelly used to activate $L_1$ and $L_2$
are related by the simulation convention $\mathbb{R}_B$
at a world $w_B$,
simulations guarantee that the corresponding answers will be related
by $\mathbb{R}_B$ as well:
the diagrams
can be pasted together horizontally
to follow the executions of $L_1$ and $L_2$.
Note that
the simulation relation $R \in \mathcal{R}_{W_B}(S_1, S_2)$
is itself indexed by $w_B$
to ensure that answers
are related consistently with the corresponding questions.

\begin{figure} % fig:fsim {{{
  \[
    \begin{array}{c@{\qquad}c@{\qquad}c}
      \begin{tikzcd}[sep=large]
        q_1 \ar[d, "w_B \Vdash \mathbb{R}_B^\que"', dash] \ar[r, dash, "I_1"] &
        s_1 \ar[d, "w_B \Vdash R", dash, dashed] \\
        q_2 \ar[r, "I_2"', dash, dashed] &
        s_2
      \end{tikzcd}
      &
      \begin{tikzcd}[sep=large]
        s_1 \ar[r, "t"] \ar[d, "w_B \Vdash R"', dash] &
        \!\!{}_1 \:\, s_1' \ar[d, "w_B \Vdash R", dash, dashed] \\
        s_2 \ar[r, "t", dashed] &
        \!\!{}_2^* \:\, s_2'
      \end{tikzcd}
      &
      \begin{tikzcd}[sep=large]
        s_1 \ar[r, "F_1", dash] \ar[d, "w_B \Vdash R"', dash] &
        r_1 \ar[d, "w_B \Vdash \mathbb{R}_B^\ans", dash, dashed] \\
        s_2 \ar[r, "F_2"', dash, dashed] &
        s_2'
      \end{tikzcd}
      \\[3.5em]
      I_1 \ifr{\Vdash \mathbb{R}_B^\que \rightarrow \mathcal{P}^\le(R)} I_2
      &
      {\rightarrow_1}
      \ifr{\Vdash R \rightarrow {=} \rightarrow \mathcal{P}^\le(R)}
      {\rightarrow_2^*}
      &
      F_1
      \ifr{\Vdash R \rightarrow \mathcal{P}^\le(\mathbb{R}_B^\ans)}
      F_2
      \\[1em]
      \small \textsf{(a) Initial states} &
      \small \textsf{(b) Internal states} &
      \small \textsf{(c) Final states}
    \end{array}
  \]
  \caption{Forward simulation properties for initial, internal and final states.
    (a)~When the execution is initiated by two related questions,
    any initial state of $L_1$ is matched by
    a related initial state of $L_2$.
    (b)~Every internal transition of $L_1$ is then matched by
    a sequence of transitions of $L_2$,
    preserving the simulation relation.
    (c)~When a final state is eventually reached by $L_1$,
    any related state in $L_2$ is final as well and
    produces a related answer.
    The indexing of the relations
    $\mathbb{R}_B^\que$, $R$, $\mathbb{R}_B^\ans$
    by the Kripke world $w_B$
    guarantees that the original questions and their eventual answers
    are related in a consistent way.}
  \label{fig:fsim}
\end{figure}
%}}}

The simulation convention $\mathbb{R}_A$
determines the correspondance between
outgoing questions triggered by
the transition systems' external states.
The corresponding simulation properties
are shown in Fig.~\ref{fig:fsim-ext}.
Compared with the treatment of incoming questions,
the roles of the system and environment are reversed:
the simulation proof can choose $w_A$
to relate the outgoing questions,
and the environment guarantees that any corresponding answers
will be related at that world.

\begin{figure} % fig:fsim-ext {{{
  \[
      \begin{tikzcd}[sep=large,baseline=-3pt]
        s_1 \ar[r, "X_1", dash] \ar[d, "w_B \Vdash R"', dash] &
        m_1 \ar[r, dotted, dash] \ar[d, "w_A \Vdash \mathbb{R}_A^\que"', dash, dashed] &
        n_1 \ar[r, "Y_1^{s_1}", dash] \ar[d, "w_A \Vdash \mathbb{R}_A^\ans"', dash] &
        s_1' \ar[d, "w_B \Vdash R", dash, dashed]
        \\
        s_2 \ar[r, "X_2"', dash, dashed] &
        m_2 \ar[r, dotted, dash] &
        n_2 \ar[r, "Y_2^{s_2}"', dash, dashed] &
        s_2
      \end{tikzcd}
      \small
      \begin{array}{
          r @{\,.\,} l @{{} \Vdash {}} c@{}c@{}c @{\:\wedge\:}
                                    c@{}c@{}c @{\:\:} l}
        \forall \, w_B \, s_1 \, s_2 \, m_1 & w_B & s_1 & {}\mathrel{R}{} & s_2 &
                        s_1 & {}\mathrel{X_1}{} & m_1 & \Rightarrow \\
        \exists \, w_A \, m_2 & w_A & m_1 & {}\mathrel{\mathbb{R}_A^\que}{} & m_2 &
                        s_2 & {}\mathrel{X_2}{} & m_2 & \wedge \\
        \forall \, n_1 \, n_2 \, s_1' & w_A & n_1 & {}\mathrel{\mathbb{R}_A^\ans}{} & n_2 &
                        n_1 & {}\mathrel{Y_1^{s_1}}{} & s_1' & \Rightarrow \\
        \exists \, s_2' & w_B & s_1' & {}\mathrel{R}{} & s_2' &
                        n_2 & {}\mathrel{Y_2^{s_2}}{} & s_2' &
      \end{array}
  \]
  \caption{Forward simulation property for external states.
    When $L_1$ assigns a question $m_1$ to an external state $s_1$,
    $L_2$ assigns a corresponding question $m_2$ to any related state $s_2$.
    The questions are related according to
    the simulation convention $\mathbb{R}_A$,
    at a world $w_A$ chosen by the simulation.
    When $L_1$ is resumed by an answer $n_1$,
    then a related answer $n_2$ also resumes $L_2$,
    and reestablishes the simulation relation
    between resulting states.}
  \label{fig:fsim-ext}
\end{figure}
%}}}

\begin{definition}[Forward simulation] \label{def:fsim} %{{{
Given
two simulation conventions
$\mathbb{R}_A : A_1 \Leftrightarrow A_2$ and
$\mathbb{R}_B : B_1 \Leftrightarrow B_2$,
and given
the transition systems
$L_1 : A_1 \twoheadrightarrow B_1 = \langle S_1, {\rightarrow}_1, D_1, I_1, X_1, Y_1, F_1 \rangle$ and
$L_2 : A_2 \twoheadrightarrow B_2 = \langle S_2, {\rightarrow}_2, D_2, I_2, X_2, Y_2, F_2 \rangle$,
a \emph{forward simulation} between $L_1$ and $L_2$
consists of a relation
$R \in \mathcal{R}_{W_B}(S_1, S_2)$
satisfying the properties shown in
Fig.~\ref{fig:fsim} and Fig.~\ref{fig:fsim-ext}.
In addition, the domains of $L_1$ and $L_2$
must satisfy:
\[
  (\lambda q_1 \, . \, (q_1 \in D_1))
  \ifr{\Vdash \mathbb{R}_B^\que \rightarrow {\Leftrightarrow}}
  (\lambda q_2 \, . \, (q_2 \in D_2))
\]
We will write $L_1 \le_{\mathbb{R}_A \twoheadrightarrow \mathbb{R}_B} L_2$.
\end{definition}
%}}}

%}}}

\paragraph{Horizontal composition} %{{{

The horizontal composition operator
described by Def.~\ref{def:hcomp}
preserves simulations.
Roughly speaking,
whenever new component instances are created
by cross-component calls,
the simulation property for the new components
can be stitched in-between
the two halves of the callers' simulation property
described in Fig.~\ref{fig:fsim-ext}.

\begin{theorem}[Horizontal composition of simulations] \label{thm:fsim-hcomp} %{{{
For a simulation convention
$\mathbb{R} : A_1 \Leftrightarrow A_2$
and transition systems
$L_1, L_1' : A_1 \twoheadrightarrow A_1$ and
$L_2, L_2' : A_2 \twoheadrightarrow A_2$,
the following propery holds:
\[
    \AxiomC{$L_1 \le_{\mathbb{R} \twoheadrightarrow \mathbb{R}} L_2$}
    \AxiomC{$L_1' \le_{\mathbb{R} \twoheadrightarrow \mathbb{R}} L_2'$}
    \BinaryInfC{$L_1 \oplus L_1'
      \le_{\mathbb{R} \twoheadrightarrow \mathbb{R}}
      L_2 \oplus L_2'$}
    \DisplayProof
\]
\begin{proof}
See \texttt{common/SmallstepLinking.v}
in the Coq development.
\end{proof}
\end{theorem}
%}}}

One interesting and novel aspect of the proof
is the way worlds are managed.
Externally,
only the worlds corresponding to incoming and outgoing
questions and answers are observed.
Internally,
the proof of Thm.~\ref{thm:fsim-hcomp}
maintains a \emph{stack} of
worlds
to relate the corresponding stack of activations
in the source and target composite semantics.
See also \S\ref{sec:cklr-worlds}.

Horizontal composition of simulations
allows us to decompose the verification of a complex program
into the verification of its parts.
To establish the correctness of the linked assembly program,
we can then use the following result.

\begin{theorem} \label{thm:asmlinking} %{{{
Linking \kw{Asm} programs
yields a correct implementation of
horizontal composition:
\[
    \forall p_1 p_2 \,.\,
      \kw{Asm}(p_1) \oplus \kw{Asm}(p_2)
      \le_{\kw{id} \twoheadrightarrow \kw{id}}
      \kw{Asm}(p_1 + p_2)
\]
\begin{proof}
See \texttt{x86/AsmLinking.v} in the Coq development.
\end{proof}
\end{theorem}
%}}}

%}}}

%}}}

%\subsection{Loaders} \label{sec:sem:loader} %{{{
%
%XXX: Here, could describe how loaders for various language
%interfaces can be defined.
%Loaders turn an open semantics such as
%$L : \mathcal{C} \rightarrow \mathcal{C}$
%into a closed whole-program semantics of type
%$\kw{load}(L) : \mathbf{1} \rightarrow \mathcal{W}$.
%
%Problem: the definitions of loaders involve
%program skeletons and symbol tables,
%which so far I have left out of the semantic model.
%
%}}}

%}}}

\section{Simulation algebra} \label{sec:simalg} %{{{

Now that we have described the structures we use
to model and relate the execution of program components,
we shift our attention to the techniques we use
to derive a correctness statement for the whole compiler
from the correctness properties of each pass.

\subsection{Vertical composition} %{{{

The first step is to combine the
simulation properties for successive compilation passes
into a single one.
The convention used by the resulting simulation
can be described as follows.

\begin{definition}[Composition of simulation conventions] %{{{
The composition of two Kripke relations
$R \in \mathcal{R}_{W_R}(X, Y)$ and
$S \in \mathcal{R}_{W_S}(Y, Z)$
is the Kripke relation
$R \cdot S \in \mathcal{R}_{W_R \times W_S}(A, C)$
defined by:
\[
  (w_R, w_S) \Vdash x \ifr{R \cdot S} z \Leftrightarrow
  \exists y \in Y \,.\,
    w_R \Vdash x \mathrel{R} y \wedge
    w_S \Vdash y \mathrel{R} z \,.
\]
Then for the simulation conventions
$\mathbb{R} : A \Leftrightarrow B$ and
$\mathbb{S} : B \Leftrightarrow C$,
we define
$\mathbb{R} \cdot \mathbb{S} : A \Leftrightarrow C$ as:
\[
  \mathbb{R} \cdot \mathbb{S} :=
  \langle
    W_\mathbb{R} \times W_\mathbb{S}, \:
    \mathbb{R}^\que \cdot \mathbb{S}^\que, \:
    \mathbb{R}^\ans \cdot \mathbb{S}^\ans
  \rangle
\]
\end{definition}
%}}}

\begin{theorem} \label{thm:fsim-vcomp} %{{{
Simulations compose vertically
in the way depicted in Fig.~\ref{fig:simcomp}.
\begin{proof}
Visually speaking,
the diagrams shown in Figs.~\ref{fig:fsim} and~\ref{fig:fsim-ext}
can be pasted vertically.
For details,
see the Coq proofs
\texttt{identity\_forward\_simulation}
and \texttt{compose\_forward\_simulations}
in the file \texttt{common/Smallstep.v}.
\end{proof}
\end{theorem}
%}}}

%\begin{theorem}[Vertical composition of simulations] \label{thm:simcomp} %{{{
%For the transition systems
%$L_1 : A_1 \twoheadrightarrow B_1$,
%$L_2 : A_2 \twoheadrightarrow B_2$,
%$L_3 : A_3 \twoheadrightarrow B_3$,
%and the simulation conventions
%$\mathbb{R}_A : A_1 \Leftrightarrow A_2$,
%$\mathbb{R}_B : B_1 \Leftrightarrow B_2$,
%$\mathbb{S}_A : A_2 \Leftrightarrow A_3$, and
%$\mathbb{S}_B : B_2 \Leftrightarrow B_3$,
%the following holds:
%\[
%  \AxiomC{$L_1 \le_{\mathbb{R}_A \twoheadrightarrow \mathbb{R}_B} L_2$}
%  \AxiomC{$L_2 \le_{\mathbb{S}_A \twoheadrightarrow \mathbb{S}_B} L_3$}
%  \BinaryInfC{
%    $L_1 \le_{\mathbb{R}_A \cdot \mathbb{S}_A \twoheadrightarrow
%                   \mathbb{R}_B \cdot \mathbb{S}_B} L_3$}
%  \DisplayProof
%\]
%\begin{proof}
%\texttt{compose\_open\_fsim} in \texttt{common/Smallstep.v}.
%\end{proof}
%\end{theorem}
%%}}}

%}}}

\subsection{Refinement of simulation conventions} \label{sec:scref} %{{{

As discussed in \S\ref{sec:mainideas:simalg},
the composite simulation conventions obtained
when we compose the passes of CompCertO
are not immediately satisfactory.
In the remainder of this section,
we introduce the algebraic infrastructure we use
to rewrite them into an acceptable form,
centered around
a notion of refinement for simulation conventions.

A refinement $\mathbb{R} \scref \mathbb{S}$
captures the idea that the convention $\mathbb{S}$
is more general than $\mathbb{R}$,
so that any simulation accepting $\mathbb{S}$ as its
incoming convention can accept $\mathbb{R}$ as well.
The shape of the symbol illustrates its meaning:
related questions of $\mathbb{R}$ can be transported to
related question of $\mathbb{S}$;
when we get a response, the
related answers of $\mathbb{S}$ can be transported back to
related answers of $\mathbb{R}$.

\begin{definition}[Simulation convention refinement] %{{{
Given two simulation conventions
$\mathbb{R}, \mathbb{S} : A_1 \Leftrightarrow A_2$,
we say that
$\mathbb{R}$ refines $\mathbb{S}$ and write
$\mathbb{R} \scref \mathbb{S}$
when the following holds:
\begin{align*}
  \forall w \, m_1 \, m_2 \,.\,
  w \Vdash m_1 \mathrel{R^\que} m_2 &\Rightarrow
  \exists \, v \,.\, (
  v \Vdash m_1 \mathrel{S^\que} m_2
  \: \wedge \\
  \forall \, n_1 \, n_2 \,.\,
  v \Vdash n_1 \mathrel{S^\ans} n_2 &\Rightarrow
  w \Vdash n_1 \mathrel{R^\ans} n_2) \,.
\end{align*}
We write $\mathbb{R} \equiv \mathbb{S}$ when both
$\mathbb{R} \scref \mathbb{S}$ and
$\mathbb{S} \scref \mathbb{R}$.
\end{definition}
%}}}

\begin{theorem} % assoc, id, monot {{{
For
$\mathbb{R} : A_1 \Leftrightarrow A_2$,
$\mathbb{S} : A_2 \Leftrightarrow A_3$,
$\mathbb{T} : A_3 \Leftrightarrow A_4$,
the following properties hold:
\begin{gather*}
  ({\cdot}) :: {{\scref} \times {\scref} \rightarrow {\scref}}
  \qquad
  (\mathbb{R} \cdot \mathbb{S}) \cdot \mathbb{T} \equiv
    \mathbb{R} \cdot (\mathbb{S} \cdot \mathbb{T})
  \qquad
  \mathbb{R} \cdot \kw{id} \equiv
  \kw{id} \cdot \mathbb{R} \equiv
  \mathbb{R}
\end{gather*}
In addition, when
$\mathbb{R} \scref \mathbb{R}' : A_1 \Leftrightarrow A_2$ and
$\mathbb{S}' \scref \mathbb{S} : B_1 \Leftrightarrow B_2$,
for all
$L_1 : A_1 \twoheadrightarrow B_1$ and $L_2 : A_2 \twoheadrightarrow B_2$:
\[
      L_1 \le_{\mathbb{R} \twoheadrightarrow \mathbb{S}} L_2
      \: \Rightarrow \:
      L_1 \le_{\mathbb{R}' \twoheadrightarrow \mathbb{S}'} L_2 \,.
\]
\begin{proof}
See \texttt{open\_fsim\_ccref} in \texttt{common/CallconvAlgebra.v}.
\end{proof}
\end{theorem}
%}}}

In CompCert,
the passes \kw{Cshmgen}, \kw{Renumber}, \kw{Linearize},
\kw{CleanupLabels} and \kw{Debugvar}
restrict the source and target
memory states and values to be identical.
Their simulation proofs require very few changes
and can be assigned the convention $\kw{id} \twoheadrightarrow \kw{id}$
(see Table~\ref{tbl:passes}).
The properties above ensure that these passes
have no impact on the overall simulation convention
of CompCertO.

%}}}

\subsection{Kleene algebra} %{{{

Given a collection of simulation conventions,
it is possible to combine them
by allowing questions to be related by any one of them.
This is the additive operation of our Kleene algebra.
The Kleene star combines all possible finite iterations
of a simulation convention.

This construction is key in our treatment of
asymmetric injection passes
($\kw{injp} \twoheadrightarrow \kw{inj}$).
The details of $\kw{inj}$ and $\kw{injp}$
are discussed in \S\ref{sec:cklr},
but are not an essential part of the technique.
Schematically,
by pre- and post-composing injection passes
with self-simulations of types
$\kw{injp}^* \twoheadrightarrow \kw{injp}^*$ and
$\kw{inj} \twoheadrightarrow \kw{inj}$,
we obtain the simulation conventions:
\[
  \kw{injp}^* \cdot \kw{injp} \cdot \kw{inj}
  \twoheadrightarrow
  \kw{injp}^* \cdot \kw{inj} \cdot \kw{inj}
\]
The property $\kw{injp}^* \cdot \kw{injp} \scref \kw{injp}^*$
on the left-hand side,
and the idempotency of $\kw{inj}$
on the right-hand side (Thm.~\ref{thm:cklr-props})
allow us to rewrite the above into
a symmetric simulation convention.

\begin{definition} \label{def:joins} % joins and Kleene star {{{
Consider $(\mathbb{R}_i)_{i \in I}$
a family of conventions
with
$\mathbb{R}_i = \langle W_i, R_i^\que, R_i^\ans \rangle
  : A_1 \Leftrightarrow A_2$
for all $i \in I$.
We define the simulation convention
$\sum_{i \in I} \mathbb{R}_i := \langle W, R^\que, R^\ans \rangle$,
where:
\[
  W := \sum_{i \in I} W_i  \qquad
  \begin{array}{l}
  (i, w) \Vdash R^\que \: := \: w \Vdash R_i^\que \\[1ex]
  (i, w) \Vdash R^\ans \: := \: w \Vdash R_i^\ans \,.
  \end{array}
\]
We will write $\mathbb{R}_1 + \cdots + \mathbb{R}_n$
for the finitary case $\sum_{i=1}^n \mathbb{R}_i$.
Then for $\mathbb{R} : A_1 \Leftrightarrow A_2$,
we define
$\mathbb{R}^* := \sum_{n \in \mathbb{N}} \mathbb{R}^n$,
where
$\mathbb{R}^0 := \kw{id}$ and
$\mathbb{R}^{n+1} := \mathbb{R}^n \cdot \mathbb{R}$.
\end{definition}
%}}}

\begin{theorem}[Kleene iteration of simulations] \label{thm:simk} %{{{
The constructions ${\scref}, {\cdot}, {+}, {*}$
work together as a typed Kleene algebra.
Moreover, the following properties hold:
\[
  \AxiomC{$\forall i \,.\,
    L_1 \le_{\mathbb{R} \twoheadrightarrow \mathbb{S}_i} L_2$}
  \UnaryInfC{$
    L_1 \le_{\mathbb{R} \twoheadrightarrow \sum_i \mathbb{S}_i} L_2$}
  \DisplayProof
  \qquad
  \AxiomC{$L \le_{\mathbb{R} \twoheadrightarrow \mathbb{S}} L$}
  \UnaryInfC{$L \le_{\mathbb{R}^* \twoheadrightarrow \mathbb{S}^*} L$}
  \DisplayProof
\]
\begin{proof}
See \texttt{cc\_star\_fsim} and the preceding definitions
in \texttt{common/CallconvAlgebra.v}.
\end{proof}
\end{theorem}
%}}}

%}}}

\subsection{Compiler correctness} \label{sec:comppass} %{{{

\begin{table} % tbl:passes Passes of CompCertO %{{{
  \figsize
  \begin{tabular}{l r @{$\: \twoheadrightarrow \:$} l r @{\ } r r}
    \hline
    Language/Pass & Outgoing & Incoming & \multicolumn{2}{c}{SLOC}
      & See also \\
    \hline
    \textbf{Clight} & $\mathcal{C}$ & $\mathcal{C}$ & +17 & (+3\%) \\
    \emph{Thm.~\ref{thm:cklr-props}} &
      $(\kw{ext} + \kw{injp})^*$ &
      $(\kw{ext} + \kw{injp})^*$ &
      & &
      \S\ref{sec:cklr-props} \\
    \kw{SimplLocals} & $\kw{injp}$ & $\kw{inj}$ & -4 & (-1\%) &
      \S\ref{sec:injp} \\
    \kw{Cshmgen} & \kw{id} & \kw{id} & +0 & (+0\%) &
      \S\ref{sec:scref} \\
    \hline
    \textbf{Csharpminor} & $\mathcal{C}$ & $\mathcal{C}$ & +15 & (+4\%) \\
    \kw{Cminorgen} & $\kw{injp}$ & $\kw{inj}$ & -21 & (-2\%) &
      \S\ref{sec:injp} \\
    \hline
    \textbf{Cminor} & $\mathcal{C}$ & $\mathcal{C}$ & +27 & (+3\%) \\
    \kw{Selection} & $\kw{wt} \cdot \kw{ext}$ & $\kw{wt} \cdot \kw{ext}$ &
      +43 & (+0\%) &
      \S\ref{sec:wt} \\
    \hline
    \textbf{CminorSel} & $\mathcal{C}$ & $\mathcal{C}$ & +15 & (+3\%) \\
    \kw{RTLgen} & $\kw{ext}$ & $\kw{ext}$ & +8 & (+0\%) &
      \S\ref{sec:cklrsc} \\
    \hline
    \textbf{RTL} & $\mathcal{C}$ & $\mathcal{C}$ & +11 & (+2\%) \\
    $\kw{Tailcall}^\dagger$ & $\kw{ext}$ & $\kw{ext}$ & -1 & (-1\%) &
      \S\ref{sec:cklrsc} \\
    \kw{Inlining} & $\kw{injp}$ & $\kw{inj}$ & +58 & (+3\%) &
      \S\ref{sec:injp} \\
    \kw{Renumber} & $\kw{id}$ & $\kw{id}$ & -14 & (-7\%) &
      \S\ref{sec:scref} \\
    \emph{Thm.~\ref{thm:cklr-props}} &
      $\kw{inj}$ &
      $\kw{inj}$ &
      & &
      \S\ref{sec:cklr-props} \\
    $\kw{Constprop}^\dagger$ &
      $\kw{va} \cdot \kw{ext}$ & $\kw{va} \cdot \kw{ext}$ &
      -17 & (-2\%) &
      \S\ref{sec:va} \\
    $\kw{CSE}^\dagger$ &
      $\kw{va} \cdot \kw{ext}$ & $\kw{va} \cdot \kw{ext}$ &
      +3 & (+0\%) &
      \S\ref{sec:va} \\
    $\kw{Deadcode}^\dagger$ &
      $\kw{va} \cdot \kw{ext}$ & $\kw{va} \cdot \kw{ext}$ &
      -7 & (-1\%) &
      \S\ref{sec:va} \\
    %\st{Unusedglob} \\
    \kw{Allocation} &
      $\kw{wt} \cdot \kw{alloc}$ & $\kw{wt} \cdot \kw{alloc}$ &
      +13 & (+0\%) &
      \S\ref{sec:alloc} \\
    \hline
    \textbf{LTL} & $\mathcal{L}$ & $\mathcal{L}$ & +15 & (+6\%) \\
    \kw{Tunneling} & $\kw{ext}$ & $\kw{ext}$ & +2 & (+0\%) &
      \S\ref{sec:cklrsc} \\
    \kw{Linearize} & \kw{id} & \kw{id} & -15 & (-3\%) &
      \S\ref{sec:scref} \\
    \hline
    \textbf{Linear} & $\mathcal{L}$ & $\mathcal{L}$ & +16 & (+7\%) \\
    \kw{CleanupLabels} & \kw{id} & \kw{id} & -10 & (-4\%) &
      \S\ref{sec:scref} \\
    \kw{Debugvar} & \kw{id} & \kw{id} & -12 & (-3\%) &
      \S\ref{sec:scref} \\
    \kw{Stacking} & \kw{stacking} & \kw{stacking} & +291 & (+11\%) &
      \S\ref{sec:stacking} \\
    \hline
    \textbf{Mach} & $\mathcal{M}$ & $\mathcal{M}$ & +100 & (+26\%) \\
    \kw{Asmgen} & \kw{asmgen} & \kw{asmgen} & +179 & (+6\%) &
      \S\ref{sec:asmgen} \\
    \hline
    \textbf{Asm} & $\mathcal{A}$ & $\mathcal{A}$ & +53 & (+5\%) \\
    \hline
    \multicolumn{3}{r}{Total:} & +765 & (+2\%)
  \end{tabular}
  \caption{Languages and passes of CompCertO.
    Passes are grouped by their source language.
    $\dagger$ indicates an optional optimization.
    The simulation conventions $\kw{ext}, \kw{inj}, \kw{injp}$
    are explained in \S\ref{sec:cklr}.
    The invariants $\kw{wt}, \kw{va}$
    are explained in \S\ref{sec:inv}.
    The conventions $\kw{alloc}, \kw{stacking}, \kw{asmgen}$
    are explained in \S\ref{sec:backend}.
    To handle the asymmetry of injection passes
    ($\kw{injp} \twoheadrightarrow \kw{inj}$),
    self-simulation properties are inserted as pseudo-passes
    (Thm.~\ref{thm:cklr-props}).
    Significant lines of code (SLOC) measured by \texttt{coqwc},
    compared to CompCert v3.6.}
  \label{tbl:passes}
\end{table}
%}}}

We can now present our central result.
The passes of CompCertO are shown in Table~\ref{tbl:passes}.
The techniques outlined above allow us to formulate
a simulation convention
$\mathbb{C} : \mathcal{C} \Leftrightarrow \mathcal{A}$
for the compiler, %which is both compositional
%and independent of the exact sequence of passes
%included in the compiler.
and to establish the following correcntess property.

\begin{theorem}[Compositional Correctness of CompCertO] \label{thm:compc} %{{{
For a \kw{Clight} program $p$
and an \kw{Asm} program $p'$ such that
$\kw{CompCert}(p) = p'$,
the following simulation holds:
\[
    \kw{Clight}(p) \le_{\mathbb{C} \twoheadrightarrow \mathbb{C}}
    \kw{Asm}(p') \,,
\]
where the simulation convention $\mathbb{C}$ is defined as:
\[
    \mathbb{C} := (\kw{ext} + \kw{injp})^* \cdot \kw{inj} \cdot
      (\kw{va} \cdot \kw{ext})^3 \cdot
      \kw{wt} \cdot \kw{alloc} \cdot
      \kw{ext} \cdot %_\mathcal{L} \cdot
      \kw{stacking} \cdot
      \kw{asmgen}
      \,.
\]
\begin{proof}
%\begin{optional}
Use Thm.~\ref{thm:fsim-vcomp} to compose
the correctness proofs of the compiler passes and
self-simulations shown in Table~\ref{tbl:passes}.
By properties of the Kleene star,
the outgoing simulation convention of each of the
passes \kw{SimplLocals}--\kw{Renumber} %\kw{Unusedglob}
can be folded into $(\kw{ext} + \kw{injp})^*$
to obtain the overall outgoing convention $\mathbb{C}$.
Likewise, by properties of $\kw{inj}$ and $\kw{ext}$
%(Thm.~\ref{thm:cklr-props}),
their incoming simulation conventions
can be folded into $\kw{inj}$
to obtain the overall incoming convention $\mathbb{C}$.
%\end{optional}
For details, see \texttt{driver/Compiler.v}.
\end{proof}
\end{theorem}
%%}}}

%}}}

\subsection{Compositional compilation and verification} \label{sec:cver} %{{{

%Recalling Fig.~\ref{fig:process},
Consider the translation units
$\mathtt{M1.c}, \ldots, \mathtt{Mn.c}$
compiled and linked to
$\mathtt{M1.s} + \ldots + \mathtt{Mn.s} = \mathtt{M.s}$.
We~can use
Thms.~\ref{thm:fsim-hcomp},
\ref{thm:asmlinking} and
\ref{thm:compc}
to establish the following separate compilation property:
\begin{equation}
  \label{eqn:sepcomp}
  \kw{Clight}(\mathtt{M1.c}) \oplus \cdots \oplus \kw{Clight}(\mathtt{Mn.c})
  \le_{\mathbb{C} \twoheadrightarrow \mathbb{C}}
  \kw{Asm}(\mathtt{M.s})
\end{equation}
That is,
the linked \kw{Asm} program
$\mathtt{M.s}$
faithfully implements
the horizontal composition of the source modules' behaviors,
following the simulation convention $\mathbb{C}$.

Additionally,
suppose we wish to verify that the overall program
satisfies a specification $\Sigma$,
also expressed as a transition system
for $\mathcal{C} \twoheadrightarrow \mathcal{C}$.
We must first decompose $\Sigma$
%along the lines of the program's components:
into:
\[
    \Sigma \: \le_{\kw{id} \twoheadrightarrow \kw{id}} \:
    \Sigma_1 \oplus \cdots \oplus \Sigma_n \,.
\]
Then for each module, we prove that
$\Sigma_i \le_{\kw{id} \twoheadrightarrow \kw{id}} \kw{Clight}(\mathtt{Mi.c})$.
This can be combined with Thm.~\ref{thm:fsim-hcomp} and Eqn.~\ref{eqn:sepcomp}
to establish the correctness property
$\Sigma \le_{\mathbb{C} \twoheadrightarrow \mathbb{C}}
        \kw{Asm}(\mathtt{M.s})$.

Note that Eqn.~\ref{eqn:sepcomp} can be established
as long as the property
$\kw{Clight}(\mathtt{Mi.c})
 \le_{\mathbb{C} \twoheadrightarrow \mathbb{C}}
 \kw{Asm}(\mathtt{Mi.s})$
holds independently for each module.
It is possible for the different $\mathtt{Mi.s}$
to be obtained by different compilers,
as long as each one satisfies a correctness property
following the simulation convention in Thm.~\ref{thm:compc}.
Indeed this is the case for versions of CompCertO
obtained by enabling different optimization passes.
Moreover, if some of the $\mathtt{Mi.s}$
are hand-written assembly modules satisfying
$\mathcal{C}$-style specifications,
then we can prove on a case-by-case basis that
$\Sigma_i \le_{\mathbb{C} \twoheadrightarrow \mathbb{C}} \kw{Asm}(\mathtt{Mi.s})$
and proceed with the rest of the proof as before.

Using C functions from
arbitrary assembly contexts is also possible,
because the compiler's simulation convention $\mathbb{C}$
captures all of the guarantees provided by CompCertO and
directly specifies the behavior of the compiled assembly code.
A proof involving an arbitrary assembly context
which invokes a function compiled by CompCertO
must establish that the call is performed
according to the C calling convention.
Then Thm.~\ref{thm:compc}
can be used to reason about the behavior of the call
in terms of the semantics of the source code
or a $\mathcal{C}$ specification that it satisfies.

%}}}

%}}}

\section{Logical relations for CompCert} \label{sec:cklr} %{{{

In the next few sections,
we examine the correctness properties
and simulation conventions
at the level of individual passes.
In this section,
we focus on simulations which use
$\kw{ext}$, $\kw{inj}$ and $\kw{injp}$.

The questions, answers and states
used to describe the semantics of CompCert languages all contain
a memory state and surrounding runtime values.
Likewise, simulation conventions and relations
are constructed around \emph{memory transformations}
and relate the surrounding components in ways that %,
%in addition to reflecting any structural changes
%between to source and target language interfaces,
are compatible with the chosen memory transformation.

\subsection{Memory extensions} \label{sec:memext} %{{{

For passes where strict equality is too restrictive,
but the source and target programs
use similar memory layouts,
CompCert uses the \emph{memory extension} relation,
which allows the values
stored in the target memory state to refine
the values stored in the source memory at the same location.
%the target memory state can also
%provide more permissions than the source memory,
%and in particular blocks may be allocated with
%a broader range of addresses in the target memory
%than in the source memory.

By analogy with
the value refinement relation $v_1 \vref v_2$
introduced in \S\ref{sec:sem:mm},
we write $m_1 \mext m_2$ to signify that
the source memory $m_1$ is extended by
the target memory $m_2$.
Together,
the relations $\vref$ and $\mext$
constitute a logical relation for the memory model,
in the sense that
loads from memory states related by extension
yield values related by refinement,
writing values related by refinement
preserves memory extension,
and similarly for the remaining memory operations.

%}}}

\subsection{Memory injections} \label{sec:meminj} %{{{

The most complex simulation relations of CompCert
allow memory blocks to be dropped, added, or
mapped at a given offset within a larger block.
These transformations of the memory structure
are specified by partial functions of type:
\[
  \kw{meminj} := \kw{block} \rightharpoonup \kw{block} \times \mathbb{Z} \,,
\]
We will call $f \in \kw{meminj}$
an \emph{injection mapping}.
An entry $f(b) = (b', o)$
means that the source memory block with identifier $b$
is mapped into the target block $b'$
at offset $o$.

As with refinement and extension,
an injection mapping determines both
a relation on values and
a relation on memory states,
which work together
as a logical relation for the CompCert memory model.
The relation $f \Vdash v_1 \hookrightarrow_\kw{v} v_2$
allows $v_2$ to refine $v_1$,
but also requires any pointer present in $v_1$ 
to be transformed according to $f$.
The relation $f \Vdash m_1 \hookrightarrow_\kw{m} m_2$
requires that the corresponding addresses of $m_1$ and $m_2$
hold values that are related by $f \Vdash {\hookrightarrow_\kw{v}}$.
%The injection mapping $f$ plays the role of a Kripke world,
%ensuring that the memory states and surrounding values
%are related consistently.

Note that corresponding memory allocations
in the source and target states cause $f$ to
evolve into a more defined injection mapping $f \subseteq f'$
relating the two newly allocated blocks.
To handle this,
we introduce the following constructions.

%}}}

\subsection{Modal Kripke relators} %{{{

We defined general Kripke relations in \S\ref{sec:logrel}.
We add structure to sets of Kripke worlds,
specifying how they can evolve.

\begin{definition} %{{{
A \emph{Kripke frame} is a tuple
$\langle W, {\leadsto} \rangle$, where
$W$ is a set of \emph{possible worlds} and
$\leadsto$ is a
binary \emph{accessibility relation} over $W$.
Then the Kripke relator $\Diamond$ is defined by:
\[
  w \Vdash x \ifr{\Diamond R} y \: \Leftrightarrow \:
    \exists \, w' \,.\, w \leadsto w' \wedge
      w' \Vdash x \mathrel{R} y
\]
\end{definition}
%}}}

\begin{example}[Injection simulation diagrams] \label{ex:sim} %{{{
Building on Ex.~\ref{ex:simrel},
consider once again the simple transition systems
$\alpha : A \rightarrow \mathcal{P}(A)$ and
$\beta : B \rightarrow \mathcal{P}(B)$.
An injection-based simulation relation between them
will be a Kripke relation
$R \in \mathcal{R}_\kw{meminj}(A, B)$
satisfying the property:
\begin{equation}
    \label{eqn:hairy-sim}
  \begin{tikzcd}[scale=0.8]
    s_1 \arrow[r, "\alpha"]
        \arrow[d, dash, "f \Vdash R"'] &
    s_1' \arrow[d, dashed, dash, "f' \Vdash R \quad (f \subseteq f')"] \\
    s_2 \arrow[r, dashed, "\beta"] &
    s_2'
  \end{tikzcd}
  \qquad
    \begin{array}{r@{}l}
    \forall f \, s_1 \, s_2 \, s_1' \,.\,
      f \Vdash s_1 \mathrel{R} s_2 &{} \wedge
      \alpha(s_1) \ni s_1' \Rightarrow \\
    \exists f' \, s_2' \,.\,
      f \subseteq f' &{} \wedge
      \beta(s_2) \ni s_2' \wedge
      f' \Vdash s_1' \mathrel{R} s_2'
    \end{array}
\end{equation}
The new states may be related according to
a new injection mapping $f'$,
but in order to preserve existing relationships
between any surrounding source and target pointers,
the new mapping must include
the original one ($f \subseteq f'$).
This pattern is very common in CompCert
and appears in a variety of contexts.
By using $\langle \kw{meminj}, {\subseteq} \rangle$
as a Kripke frame,
we can express
(\ref{eqn:hairy-sim}) as:
\[
  \alpha \ifr{\Vdash R \rightarrow \mathcal{P}^\le(\Diamond R)} \beta \,.
\]
\end{example}
%}}}

%}}}

\subsection{CompCert Kripke logical relations} \label{sec:cklrdef} %{{{

We can now formalize the idea that
extensions and injections
constitute logical relations for the CompCert memory model.

\begin{definition}[CompCert Kripke Logical Relation] \label{def:cklr} %{{{
For a tuple $R = (W, \leadsto, f, R^\kw{mem})$,
where
$\langle W, \leadsto \rangle$ is a Kripke frame,
$f : W \rightarrow \kw{meminj}$
associates an injection mapping to each world, and where
$R^\kw{mem} \in \mathcal{R}_{W}(\kw{mem})$
is a Kripke relation on memory states,
the Kripke relations
$R^\kw{ptr} \in \mathcal{R}_W(\kw{ptr})$ and
$R^\kw{ptrrange} \in \mathcal{R}_W(\kw{ptrrange})$
are defined by the rules:
\begin{gather*}
  \AxiomC{$f_w(b) = (b', \delta)$}
  \UnaryInfC{$w \Vdash (b, o) \mathrel{R^\kw{ptr}} (b', o + \delta)$}
  \DisplayProof
  \\[1ex]
  \AxiomC{$w \Vdash (b_1, l_1) \mathrel{R^\kw{ptr}} (b_2, l_2)$}
  \AxiomC{$h_1 - l_1 = h_2 - l_2$}
  \BinaryInfC{$w \Vdash (b_1, l_1, h_1) \mathrel{R^\kw{ptrrange}} (b_2, l_2, h_2)$}
  \DisplayProof
\end{gather*}
and
$R^\kw{val} \in \mathcal{R}_W(\kw{val})$
is the smallest Kripke relation satisfying:
\begin{gather*}
  \forall \, v \in \kw{val} \,.\,
    \Vdash \kw{undef} \mathrel{R^\kw{val}} v \qquad
  \kw{vptr} :: {\Vdash R^\kw{ptr} \rightarrow R^\kw{val}} \\
  \kw{int}, \kw{long}, \kw{float}, \kw{single} ::
    {\Vdash {=} \rightarrow R^\kw{val}} \,.
\end{gather*}
We say that $R$ is a \emph{CompCert Kripke logical relation}
if the properties shown in Fig.~\ref{fig:cklr-def} are satisfied.
\end{definition}
%}}}

\begin{figure} % fig:cklr-def (Definition of CKLRs) {{{
  \begin{gather*}
    {\leadsto} \mbox{ is reflexive and transitive} \\
    w \leadsto w' \Rightarrow f(w) \subseteq f(w')
  \end{gather*}
  \begin{align*}
      \kw{alloc} ::
        &\Vdash R^\kw{mem} \rightarrow {=} \rightarrow {=} \rightarrow
        \Diamond (R^\kw{mem} \times R^\kw{block})
      \\
      \kw{free} ::
        &\Vdash R^\kw{mem} \rightarrow R^\kw{ptrrange} \rightarrow
        \kw{option}^\le(\Diamond R^\kw{mem})
      \\
      \kw{load} ::
        &\Vdash R^\kw{mem} \rightarrow R^\kw{ptr} \rightarrow
        \kw{option}^\le(R^\kw{val})
      \\
      \kw{store} ::
        &\Vdash R^\kw{mem} \rightarrow R^\kw{ptr} \rightarrow R^\kw{val} \rightarrow
        \kw{option}^\le(\Diamond R^\kw{mem})
%      \\
%      \kw{perm} ::
%        &\Vdash R^\kw{mem} \rightarrow R^\kw{ptr} \rightarrow {\subseteq}
  \end{align*}
  \caption{Defining properties of CKLRs.
    Note the correspondance with
    the types of operations in Fig.~\ref{fig:mm}.}
  \label{fig:cklr-def}
\end{figure}
%}}}

\paragraph{Rationale} %{{{

The relation $R^\kw{mem}$ is given.
We expect $R^\kw{ptr}$ to map
each source pointer to at most one target pointer
and to be shift-invariant in the following sense:
\[
  \AxiomC{$w \Vdash (b_1, o_1) \mathrel{R^\kw{ptr}} (b_2, o_2)$}
  \UnaryInfC{$w \Vdash (b_1, o_1 + \delta) \mathrel{R^\kw{ptr}} (b_2, o_2 + \delta)$}
  \DisplayProof
\]
Any such relation can be uniquely specified by
the injection mapping $f$.
We expect the other relations to be consistent with $R^\kw{ptr}$
and $\kw{undef}$ to act as a least element for $R^\kw{val}$,
which determines them completely.

%}}}

Note that $R^\kw{mem}$
is the central component driving world transitions,
as witnessed by the uses of $\Diamond$ in Fig.~\ref{fig:cklr-def}.
The surrounding relations are monotonic in $w$,
so that any extra state
constructed from pointers and runtime values
will be able to ``follow along'' when
world transitions occur.

\begin{theorem}
Extensions and injections
correspond to the CompCert Kripke logical relations:
\begin{align*}
  \kw{ext} &:=
    \langle \{*\}, \: \{(*,*)\}, \: * \mapsto (b \mapsto (b, 0)), \:
    {\mext} \rangle
  \\
  \kw{inj} &:=
    \langle \kw{meminj}, {\subseteq}, f \mapsto f,
      {\hookrightarrow_\kw{m}} \rangle
\end{align*}
\begin{proof}
The correspondance between $R^\kw{val}_\kw{inj}$ and
$\hookrightarrow_\kw{v}$ is easily verified,
as is the correspondance between
$* \Vdash R^\kw{val}_\kw{ext}$ and $\vref$.
The properties of Fig.~\ref{fig:cklr-def}
reduce to well-known properties of the memory model
already proven in CompCert.
See \texttt{cklr/Extends.v} and \texttt{cklr/Inject.v}
for details.
\end{proof}
\end{theorem}

%}}}

\subsection{From CKLRs to simulation conventions} \label{sec:cklrsc} %{{{

In simulations,
the accessibility relation
allows us to update the world after each step
in the program's execution.
Transitivity allows us to combine
multiple steps in one:
\[
  q\ph{qh} \cdot
    \bpt{qh}s_1\ph{s1} \cdot
    \bpt{s1}s_2\ph{s2} \cdots
    s_k\ph{sn} \cdot
    \bpt{sn}r
  \quad \Rightarrow \quad
  q\ph{qh} \cdot
    s_1 \cdot
    s_2 \cdots
    s_k \cdot
    \bpt{qh}r
\]
In our approach to simulation conventions,
the accessibility relation is not part of
the interface of simulations.
Instead,
a single world is used to formulate
the 4-way relationship between
pairs of questions and answers.
As shown below,
in the case of simulation conventions
based on CKLRs,
this relation does involve the accessibility relations
which CKLRs introduce.

Given a specific language interface $\mathcal{X}$,
the components of any CKLR
$R = \langle W, {\leadsto}, f, R^\kw{mem} \rangle$
can be used to construct a simulation relation
$R_\mathcal{X} : \mathcal{X} \Leftrightarrow \mathcal{X}$.
For instance:
\[
  R_\mathcal{C} :=
    \langle
      W, \:
      (R^\kw{val} \times {=} \times \vec{R}^\kw{val} \times R^\kw{mem}), \:
      \Diamond (R^\kw{val} \times R^\kw{mem})
    \rangle \,.
\]
We will often implicitly promote $R$ to $R_\mathcal{C}$.
%
Furthermore,
since the semantics of CompCert languages
are built out of the operations of the memory model,
they are well-behaved with respect to CKLRs and
we can prove the following parametricity theorems.

\begin{theorem}[Relational parametricity of $\kw{Clight}$ and $\kw{RTL}$] %{{{
\label{thm:param}
For all programs $p$ and CKLR $R$:
\[
      \kw{Clight}(p)
        \le_{R \twoheadrightarrow R}
      \kw{Clight}(p)
      \qquad
      \kw{RTL}(p)
        \le_{R \twoheadrightarrow R}
      \kw{RTL}(p)
\]
\begin{proof}
See \texttt{cklr/*rel.v} in the Coq development,
in particular \texttt{Clightrel.v} and \texttt{RTLrel.v}.
\end{proof}
\end{theorem}
%}}}

The passes of CompCert which use memory extensions
do not feature complex invariants
which must be preserved at call sites;
it is enough for external calls to preserve
the memory extension.
Consequently,
they are not much more difficult to update
than identity passes,
and can be assigned the type $\kw{ext} \twoheadrightarrow \kw{ext}$.
By contrast,
injection passes are trickier to handle.

%}}}

\subsection{External calls in injection passes} \label{sec:injp} %{{{

Passes which alter the block structure of the memory
use \emph{memory injections} (\S\ref{sec:meminj}).
The convention \kw{inj} can be used for incoming calls,
but it is insufficient for outgoing calls.

Consider the \kw{SimplLocals} pass,
which removes some local variables %whose addresses are not taken
from the memory.
The corresponding values are instead stored
as temporaries in the target function's local environment,
and the correspondance between the two
is enforced by the simulation relation.
To maintain it,
we need to know that
external calls do not modify
the corresponding source memory blocks.

More generally,
as depicted in Fig.~\ref{fig:injp},
injection passes expect external calls
to leave regions outside of the injection's footprint untouched.
This expectation is reasonable because
external calls
should behave uniformly between the source and target executions.
%An external function acting on the memory state
%should not forge block identifiers and pointers,
%but can only follow pointers passed as arguments
%or fetched from the memory itself
%(as explained in \S\ref{sec:cklr},
%this is a form of parametricity).
These requirements can be formalized as the
following CompCert Kripke logical relation:
\begin{gather*}
  \kw{injp} :=
    \langle
      \kw{meminj} \times \kw{mem} \times \kw{mem}, \:
      {\leadsto_\kw{injp}}, \:
      \pi_1, \:
      R^\kw{mem}_\kw{injp}
    \rangle
  \\
  \AxiomC{$f \Vdash m_1 \hookrightarrow_\kw{m} m_2$}
  \UnaryInfC{$(f, m_1, m_2) \Vdash m_1 \mathrel{R^\kw{mem}_\kw{injp}} m_2$}
  \DisplayProof
\end{gather*}
where $(f, m_1, m_2) \leadsto_\kw{injp} (f', m_1', m_2')$
ensures that $f \subseteq f'$ and that the memory states
satisfy the constraints in Fig.~\ref{fig:injp}
(for details,
see \texttt{cc\_injp} in \texttt{common/LanguageInterface.v}).

\begin{figure} % fig:injp {{{
  \begin{tikzpicture}[scale=0.4,>=stealth]
    \node at (-2,4.5) {$m_1$};
    \node at (-2,2.5) {$f$};
    \node at (-2,0.5) {$m_2$};
    \draw ( 0,4) rectangle +(2,1);
    \draw[pattern=north east lines] ( 3,4) rectangle +(2,1);
    \draw ( 6,4) rectangle +(3,1);
    \draw[pattern=north east lines] (10,4) rectangle +(1,1);
    \draw[pattern=north east lines] ( 0,0) rectangle +(8,1);
    \draw[pattern=north east lines] ( 9,0) rectangle +(2,1);
    \draw[->] (0,4) -- (1,1); \draw[dotted] (2,4) -- (3,1);
    \draw[->] (6,4) -- (4,1); \draw[dotted] (9,4) -- (7,1);
    \draw[fill=white] (1,0) rectangle +(2,1);
    \draw[fill=white] (4,0) rectangle +(3,1);
    %
    \draw[dashed] (12,5.5) -- (12,-0.5);
    %
    \draw (13,4) rectangle +(2,1);
    \draw (16,4) rectangle +(2,1);
    \draw (13,0) rectangle +(1,1);
    \draw (15,0) rectangle +(3,1);
    \draw[dashed,->] (13,4) -- (15,1);
    \draw[dotted] (15,4) -- (17,1);
    \draw[dotted] (17,1) -- (17,0);
    \node at (20,4.5) {$m_1'$};
    \node at (20,2.5) {$f'$};
    \node at (20,0.5) {$m_2'$};
  \end{tikzpicture}
  \[ (f, m_1, m_2) \leadsto_\kw{injp} (f', m_1', m_2') \]
  \caption{External calls and memory injections.
    The source and target memory states are
    depicted at the top and bottom
    of the figure. Arrows describe the injection mapping.
    The memory block on the left of the dashed line
    are present at the beginning of the call.
    Memory blocks on the right
    are allocated during the call,
    adding a new entry to the injection mapping.
    The shaded areas must not be modified by the call.
  }
  \label{fig:injp}
\end{figure}
%}}}

%}}}

\subsection{Discussion: world transitions and compositionality} \label{sec:cklr-worlds} %{{{

The $\kw{injp}$ convention
illustrates a key novelty
in the granularity at which we deploy Kripke worlds.
In previous work,
Kripke worlds are usually assumed to evolve linearly
with the execution.
Writing $s_i$ for internal states,
this can be depicted as:
\[
  q\ph{qh} \cdot
    \bpt{qh}s_1\ph{s1} \cdot
    \bpt{s1}s_2\ph{s2} \cdot
    \bpt{s2}m\ph{m} \cdot
    \bpt{m}s_1'\ph{s1'} \cdot
    \bpt{s1'}s_2'\ph{s2'} \cdot
    \bpt{s2'}n\ph{n} \cdot
    \bpt{n}s_3 \cdots
    s_k\ph{sn} \cdot
    \bpt{sn}r
\]
To enable horizontal compositionality,
the challenge is then to construct worlds,
accessibility relations, and simulation relations
which are sophisticated enough
to express ownership constraints
like the ones discussed in \S\ref{sec:injp},
which evolve and shift as the execution
switches between components.

In our open simulations,
worlds can be deployed
independently for incoming and outgoing calls,
in a way which follows the structure of plays,
as depicted here:
\[
  q\ph{qpos} \cdot
    m_1\ph{m1pos} \cdot \ptc{m1pos}{black}n_1 \cdots
    m_k\ph{m2pos} \cdot \ptc{m2pos}{black}n_k \cdot
    \ptc{qpos}{black}r
\]
Internal steps
are not part of a component's observable behavior,
and individual simulation proofs are free to use any
simulation relation establishing
the simulation convention at interaction sites.

Two examples illustrate this flexibility.
First, as explained in \S\ref{sec:sem:ref},
to handle nested cross-component calls,
composite simulations
use an internal stack of worlds.
A situation where $m_1$ and $m_2$ are
cross-component calls and $m_3$ is an external call
can be described as:
\[
  q\ph{qpos} \cdot
    m_1\ph{m1pos} \cdot
    m_2\ph{m2pos} \cdot
    m_3\ph{m3pos} \cdot
    \ptc{m3pos}{black}n_3 \cdot
    \pt{m2pos}n_2 \cdot
    \pt{m1pos}n_1 \cdot
    \ptc{qpos}{black}r
\]
Second,
in simulations which use CKLRs,
the simulation relation can simply
be qualified as $w_B \Vdash \Diamond R$
to allow the world to evolve as the execution progresses.
The corresponding shape is:
\[
  q\ph{qh} \cdot
    \pt{qh}s_1\ph{s1} \cdot
    \pt{qh}s_2\ph{s2} \cdot
    \pt{qh}s_3 \cdots
    \pt{qh}s_k\ph{sn} \cdot
    \ptc{qh}{black}r
\]
Since $\Diamond \Diamond R = \Diamond R$,
per-step world transitions can easily be folded
into the overall constraint.
Moreover,
this allows steps which \emph{do not}
individually conform to world transitions
($\Vdash R \rightarrow \Diamond R$),
but \emph{do} maintain $\Diamond R$ with respect
to the initial world
($w_B \Vdash \Diamond R \rightarrow \Diamond R$).

For instance,
the simulation convention of the \kw{Stacking} pass
is based on $\kw{injp} \twoheadrightarrow \kw{injp}$.
\kw{Stacking} stores the contents of some
temporaries used by the source program
into \emph{spilling locations}
of the target in-memory stack frames.
To prove correctness,
we must ensure that spilling locations
are only accessed as intended,
by enforcing their separation
%in the target memory
from the injected source memory.
%
This property is maintained by $\leadsto_\kw{injp}$,
which most internal steps and external calls conform to.
On the other hand,
internal steps which \emph{do} access spilling locations
in the expected way
do not conform to $\leadsto_\kw{injp}$ at a granular level.
However,
since the stack frame is a \emph{new}
memory block allocated after a function is called,
these steps do maintain $\leadsto_\kw{injp}$
with repect to the initial world.
This allows us to encode CompCert's original and
``instantaneous'' assumptions about external calls directly,
and existing simulation proofs relying on them
can be updated with almost no changes.

Combined together,
the two examples above are sufficient to express
ownership constraints which require
sophisticated permission maps in other approaches,
by using conditions already present in CompCert.

%}}}

\subsection{Properties} \label{sec:cklr-props} %{{{

Finally,
we state some properties which are used
to derive Thm.~\ref{thm:compc}.

\begin{theorem} \label{thm:cklr-props}
For all $\kw{Clight}$ and $\kw{RTL}$ programs:
\begin{gather*}
\forall \, p \,.\,
  \kw{Clight}(p)
  \le_{(\kw{ext} + \kw{injp})^* \twoheadrightarrow (\kw{ext} + \kw{injp})^*}
  \kw{Clight}(p) \\
\forall \, p \,.\,
  \kw{RTL}(p)
  \le_{\kw{inj} \twoheadrightarrow \kw{inj}}
  \kw{RTL}(p)
\end{gather*}
In addition,
the simulation relations derived from $\kw{ext}$ and $\kw{inj}$
compose in the following way:
\[
  \kw{ext} \cdot \kw{inj} \equiv
  \kw{inj} \cdot \kw{ext} \equiv
  \kw{inj} \cdot \kw{inj} \equiv
  \kw{inj} \,.
\]
\begin{proof}
The first statement is derived from
Thms.~\ref{thm:simk} and~\ref{thm:param};
see \texttt{driver/Compiler.v}.
For the second statement, see
\texttt{ext\_inj}, \texttt{inj\_ext}, \texttt{inj\_inj}
found under \texttt{cklr/}.
\end{proof}
\end{theorem}

%}}}

%}}}

\section{Invariants} \label{sec:inv} %{{{

% preamble {{{

Several passes of CompCert
rely on the preservation of invariants
by their source program.
This situation is illustrated in Fig.~\ref{fig:fsim-inv}:
when the semantics of a language preserves an invariant,
the preservation properties can assist
in proving forward simulations
which use the language as their source.
This allows us to decompose the simulation proof,
and in the case of $\kw{RTL}$
the preservation proofs can be reused for multiple passes.

In CompCert,
this technique is deployed in an ad-hoc manner:
for each pass using an invariant,
the simulation relation is strengthened to assert that
the invariant holds on the source state,
and the preservation properties for the source language
are used explicitly in the simulation proof
to maintain this invariant.
In CompCertO,
this becomes more involved,
because the simulation convention must be altered
to ensure that invariants are preserved
by external calls.

On the other hand,
our simulation infrastructure offers the opportunity
to capture and reason about invariants explicitly,
and to further decouple preservation and simulation proofs.
In this section,
we give an overview of our treatment of invariants.
For details,
see \texttt{common/Invariant.v}
in the Coq development.

%}}}

\subsection{Invariants and language interfaces} %{{{

First, we define a sort of ``invariant convention'',
which describes how a given invariant impacts the questions and answers
of the language under consideration.

\begin{definition} % Invariant for a language interface {{{
An \emph{invariant for a language interface} $A$
is a tuple
$\mathbb{P} = \langle W, \mathbb{P}^\que, \mathbb{P}^\ans \rangle$,
where $W$ is a set of worlds
and $\mathbb{P}^\que, \mathbb{P}^\ans$
are families of predicates on $A^\que, A^\ans$
indexed by $W$.
\end{definition}
%}}}

\begin{example} \label{ex:wt} %{{{
Typing constraints for the language interface $\mathcal{C}$
can be expressed as the invariant:
\begin{gather*}
  \kw{wt} :=
    \langle
      \kw{sig},
      \mathbb{P}_\kw{wt}^\que,
      \mathbb{P}_\kw{wt}^\ans
    \rangle
  \\
  \AxiomC{$\vec{v} <: \mathit{sg}.\kw{args}$}
  \UnaryInfC{$\mathit{sg} \Vdash
    \mathit{vf}[\mathit{sg}](\vec{v})@m \in \mathbb{P}_\kw{wt}^\que$}
  \DisplayProof
  \qquad
  \AxiomC{$v' <: \mathit{sg}.\kw{res}$}
  \UnaryInfC{$\mathit{sg} \Vdash
    v'@m' \in \mathbb{P}_\kw{wt}^\ans$}
  \DisplayProof
\end{gather*}
The proposition $\vec{v} <: \mathit{sg}.\kw{args}$
asserts that the types of the arguments $\vec{v}$
match those specified by the signature $\mathit{sg}$.
The proposition $v' <: \mathit{sg}.\kw{res}$
asserts a similar property for the return value $v'$.
\end{example}
%}}}

Invariants can be seen as a special case of simulation convention
which constrain the source and target questions and answers
to be equal.
This can be formalized as follows.

\begin{definition}[Simulation conventions for invariants]
A $W$-indexed predicate $P$ on a set $X$
can be promoted to a Kripke relation
$\hat{P} \in \mathcal{R}_W(X, X)$
defined by the rule:
\[
  \AxiomC{$w \Vdash x \in P$}
  \UnaryInfC{$w \Vdash x \mathrel{\hat{P}} x$}
  \DisplayProof
\]
Then an invariant
$\mathbb{P} = \langle W, \mathbb{P}^\que, \mathbb{P}^\ans \rangle$
can be promoted to a simulation convention:
$\hat{\mathbb{P}} :=
 \langle W, \hat{\mathbb{P}}^\que, \hat{\mathbb{P}}^\ans \rangle$.
\end{definition}

%}}}

\begin{figure} % fig:fsim-inv {{{
  \[
    \begin{tikzcd}[execute at end picture={
        \begin{scope}[radius=0.25, on background layer]
        \draw (node cs:name=I1,anchor=center) circle;
        \draw[dashed] (node cs:name=I2,anchor=center) circle;
        \draw (node cs:name=R1,anchor=center) circle;
        \draw[dashed] (node cs:name=R2,anchor=center) circle;
        \draw (node cs:name=X1,anchor=center) circle;
        \draw[dashed] (node cs:name=X2,anchor=center) circle;
        \draw (node cs:name=Y1,anchor=center) circle;
        \draw[dashed] (node cs:name=Y2,anchor=center) circle;
        \draw (node cs:name=F1,anchor=center) circle;
        \draw[dashed] (node cs:name=F2,anchor=center) circle;
        \draw (node cs:name=I11,anchor=center) circle;
        \draw (node cs:name=I21,anchor=center) circle;
        \draw (node cs:name=R11,anchor=center) circle;
        \draw (node cs:name=R21,anchor=center) circle;
        \draw (node cs:name=X11,anchor=center) circle;
        \draw (node cs:name=X21,anchor=center) circle;
        \draw (node cs:name=Y11,anchor=center) circle;
        \draw (node cs:name=Y21,anchor=center) circle;
        \draw (node cs:name=F11,anchor=center) circle;
        \draw (node cs:name=F21,anchor=center) circle;
        \end{scope}
      }]
    %
      |[alias=I1]| q_1 \ar[r, "I_1"] &
      |[alias=I2]| s_1 &
      |[alias=R1]| s_1 \ar[r, "t"] &
      |[alias=R2]| s_1' &
      |[alias=X1]| s_1 \ar[r, "X_1"] &
      |[alias=X2]| m_1 \ar[r, dotted, dash] &
      |[alias=Y1]| n_1 \ar[r, "Y_1^{s_1}"] &
      |[alias=Y2]| s_1' &
      |[alias=F1]| s_1 \ar[r, "F_1"] &
      |[alias=F2]| r_1
      \\
      |[alias=I11]| q_1 \ar[r, "I_1"] \ar[d, dash] &
      |[alias=I21]| s_1 \ar[d, dashed, dash] &
      |[alias=R11]| s_1 \ar[r, "t"] \ar[d, dash] &
      |[alias=R21]| s_1' \ar[d, dashed, dash] &
      |[alias=X11]| s_1 \ar[r, "X_1"] \ar[d, dash] &
      |[alias=X21]| m_1 \ar[r, dotted, dash] \ar[d, dashed, dash] &
      |[alias=Y11]| n_1 \ar[r, "Y_1^{s_1}"] \ar[d, dash] &
      |[alias=Y21]| s_1' \ar[d, dashed, dash] &
      |[alias=F11]| s_1 \ar[r, "F_1"] \ar[d, dash] &
      |[alias=F21]| r_1 \ar[d, dashed, dash]
      \\
      |[alias=I12]| q_2 \ar[r, "I_2"', dashed] &
      |[alias=I22]| s_2 &
      |[alias=R12]| s_2 \ar[r, "t"', dashed] &
      |[alias=R22]| \!\!\! {}^* \: s_2' &
      |[alias=X12]| s_2 \ar[r, "X_2"', dashed] &
      |[alias=X22]| m_2 \ar[r, dotted, dash] &
      |[alias=Y12]| n_2 \ar[r, "Y_2^{s_2}"', dashed] &
      |[alias=Y22]| s_2' &
      |[alias=F12]| s_2 \ar[r, "F_1"', dashed] &
      |[alias=F22]| r_2
    \end{tikzcd}
  \]
  \caption{Simulation with invariants (see \S\ref{sec:fsim-inv}).
    Circles indicate questions, answers and states
    which satisfy the appropriate invariants.
    When the transition system $L_1$ preserves the invariants
    in the way shown in the top row,
    a simulation of $L_1$ by $L_2$ can be established through
    the weakened diagrams shown in the bottom row.
    The resulting simulation uses the convention
    $\mathbb{P}_A \cdot \mathbb{R}_A \twoheadrightarrow
     \mathbb{P}_B \cdot \mathbb{R}_B$,
    ensuring that the environment
    establishes and preserves the appropriate invariants
    on questions and answers.
    The simulation relation $P \cdot R$ then ensures that
    the strengthened assumptions used by the
    weakened simulation diagrams can be satisfied.}
  \label{fig:fsim-inv}
\end{figure}
%}}}

\subsection{Simulations modulo invariants} \label{sec:fsim-inv} %{{{

The top row in Fig.~\ref{fig:fsim-inv}
illustrates the preservation of invariants by transition systems.
In the context of a transition system
$L_1 : A_1 \rightarrow B_1$,
we consider three invariants working together:
\begin{itemize}
  \item an invariant $\mathbb{P}_A$ for the language interface $A$;
  \item an invariant $\mathbb{P}_B$ for the language interface $B$;
  \item a $W_B$-indexed predicate $P$ on the states of $L_1$.
\end{itemize}
The preservation of
$\langle \mathbb{P}_A, \mathbb{P}_B, P \rangle$
is then analogous to a unary simulation property,
where $\mathbb{P}_A \twoheadrightarrow \mathbb{P}_B$
play the roles of the simulation conventions,
and $P$ plays the role of the simulation relation.
%In fact,
%when $L_1$ preserves these invariants,
%the following property holds:
%\[
%    L_1 \le_{\hat{\mathbb{P}}_A \twoheadrightarrow \hat{\mathbb{P}}_B} L_1
%\]

Once we have established that
the source language preserves the invariants,
we wish to use this fact to help prove the forward simulation
for a given pass.
To this end,
we define a \emph{strengthened} transition system
$L_1^\mathbb{P} : A_1 \twoheadrightarrow B_1$,
with the property that
$
   L_1 \le_{\hat{\mathbb{P}}_A \twoheadrightarrow \hat{\mathbb{P}}_B}
   L_1^\mathbb{P}
$.
For a target transition system $L_2 : A_2 \twoheadrightarrow B_2$,
it then suffices to show that
$
  L_1^\mathbb{P}
  \le_{\mathbb{R}_A \twoheadrightarrow \mathbb{R}_B}
  L_2
$
to establish:
\[
  L_1
  \le_{\hat{\mathbb{P}}_A \cdot \mathbb{R}_A \twoheadrightarrow
       \hat{\mathbb{P}}_B \cdot \mathbb{R}_B}
  L_2 \,.
\]
Simulations from $L_1^\mathbb{P}$
are easier to prove,
because $L_1^\mathbb{P}$
provides assumption that the invariants hold
on all source questions, answers and states.
The simulation diagrams
reduce to those shown in the bottom row of
Fig.~\ref{fig:fsim-inv}.
However, since they are formulated in terms of
Def.~\ref{def:fsim},
the standard forward simulation techniques
defined by CompCert
in \texttt{Smallstep.v}
remain available.

%}}}

\subsection{Typing invariants} \label{sec:wt} %{{{

The typing invariant described in Ex.~\ref{ex:wt}
is used by the $\kw{Selection}$ and $\kw{Allocation}$ passes.
We have updated their correctness proofs
as well as the preservation proofs in
$\texttt{Cminortyping.v}$ and $\texttt{RTLtyping.v}$
to use our framework.

The invariant $\kw{wt}$ satisfies one key property:
when a simulation convention $\mathbb{R}$
consists of a sequence of CKLRs and other invariants,
the following property holds:
\[
  \kw{wt} \cdot \mathbb{R} \cdot \kw{wt} \equiv
  \mathbb{R} \cdot \kw{wt}
\]
This means CompCertO's overall simulation convention
can eliminate the typing invariant for the $\kw{Selection}$ pass,
retaining only that used for $\kw{Allocation}$.
In turn, this facilitates the simplification of the convention for
the passes from \kw{Clight} to \kw{Inlining}.

%}}}

\subsection{Value analysis} \label{sec:va} %{{{

The passes
$\kw{Constprop}$, $\kw{CSE}$ and $\kw{Deadcode}$
use CompCert's value analysis framework.
Abstract interpretation is performed %at compilation time
on their source program,
and the resulting information is used to carry out
the corresponding optimizations.
The correctness proofs for these passes then rely
on the invariant $\kw{va}$,
which asserts that the concrete runtime states
satisfy the constraints encoded in the corresponding
abstract states.

We have updated the value analysis framework
and associated pass correctness proofs
to fit the invariant infrastructure described in this section.
Value analysis passes use the convention
$\kw{va} \cdot \kw{ext} \twoheadrightarrow \kw{va} \cdot \kw{ext}$.
Unfortunately,
because it combines constraints with mixed variance,
the invariant $\kw{va}$ does not propagate in the same way as $\kw{wt}$,
so the compiler's simulation convention must retain
the component $(\kw{va} \cdot \kw{ext})^3$ as-is.
When some of the corresponding optimization passes are disabled,
we use self-simulations of the $\kw{RTL}$ language
to match this convention nonetheless.

%}}}

%}}}

\section{Specialized simulation conventions} \label{sec:backend} %{{{

For the \kw{Alloc}, \kw{Stacking} and \kw{Asmgen} passes,
we construct more specific simulation conventions
which express the correspondance between
the higher-level and lower-level representations
of function calls and returns.
These passes use identical conventions for
incoming and external calls.

\subsection{The \kw{Allocation} pass} \label{sec:alloc} %{{{

The \kw{Allocation} pass from \kw{RTL} to \kw{LTL}
is the first pass to modify the interface of function calls.
\kw{LTL} uses \emph{abstract locations}
which represent the stack slots and machine registers
eventually used in the target assembly program.
Abstract locations contain arguments, temporaries and return values.
The contents are stored in a \emph{location map},
passed across components by the interface $\mathcal{L}$
alongside memory states.
%Up to \kw{RTL},
%arguments are passed as standalone values,
%but in \kw{LTL}
%they are mapped to locations.
The compiler also expects the values of
abstract locations designated as \emph{callee-save}
to be preserved by function calls.

To express the simulation convention used by the \kw{Allocation} pass,
we will use the following notations.
For a signature $\mathit{sg}$ and a location map $\mathit{ls}$,
we write $\kw{args}(\mathit{sg}, \mathit{ls})$
to represent the argument values stored in $\mathit{ls}$.
Likewise,
$\kw{retval}(\mathit{sg}, \mathit{ls})$ extracts
the contents of locations used to store the return value.
The relation $\equiv_\kw{CS}$ asserts that
two location maps agree on callee-save locations.

The simulation convention
$\kw{alloc} : \mathcal{C} \Leftrightarrow \mathcal{L}$
uses its worlds to remember the initial location map
and the signature associated with a call,
and can then be defined by:
\[
  \kw{alloc} := \langle
      \kw{signature} \times \kw{locmap}, \:
      R_\kw{alloc}^\que, \:
      R_\kw{alloc}^\ans \rangle
\]
\[
  \AxiomC{$\vec{v} \vref \kw{args}(\mathit{sg}, \mathit{ls})$}
  \AxiomC{$m_1 \mext m_2$}
  \BinaryInfC{$
      (\mathit{sg}, \mathit{ls}) \Vdash
      \mathit{vf}[\mathit{sg}](\vec{v})@m_1
      \mathrel{R_\kw{alloc}^\que}
      \mathit{vf}[\mathit{sg}](\mathit{ls})@m_2$}
  \DisplayProof
\]
\[
  \AxiomC{$v' \vref \kw{retval}(sg, ls')$}
  \AxiomC{$ls \equiv_\kw{CS} ls'$}
  \AxiomC{$m_1' \mext m_2'$}
  \TrinaryInfC{$
      (sg, ls) \Vdash
      v'@m_1'
      \mathrel{R_\kw{alloc}^\ans}
      ls'@m_2'$}
  \DisplayProof
\]

%}}}

\subsection{The \kw{Stacking} pass} \label{sec:stacking} %{{{

The \kw{Stacking} pass
consolidates the information which
\kw{Linear} stores in abstract stack locations
into in-memory stack frames.
The simulation proof uses a memory injection,
and involves maintaining separation properties
ensuring that the source memory and
the regions of stack frames
introduced by \kw{Stacking}
occupy disjoint areas of the target memory.

With regards to the memory state,
the \kw{stacking} simulation convention
is essentially identical to \kw{injp}.
Since the new regions of stack frames
are outside the image of source memory,
and most of them are local to
function activations,
the properties of \kw{injp}
are largely sufficient
(see also \S\ref{sec:cklr-worlds}).

The exception to the rule pertains to argument passing.
Loads from a function's argument locations
access the caller's stack frame.
If the area used to store arguments
overlaps with the injected source memory state,
then the source program and external calls
may alter them in unexpected ways.
In previous CompCert extensions,
sophisticated techniques were required
to prevent this from happening.

In our model,
we simply encode the required separation condition
in the simulation convention
$\kw{stacking} : \mathcal{L} \Leftrightarrow \mathcal{M}$.
The convention
asserts that the contents of argument locations
are stored into corresponding stack slots
within the target memory,
but additionally
requires that the region of the target memory
used to store arguments within the caller's stack
must be disjoint from the injection image of the source memory:
\[
  \begin{tikzpicture}[scale=0.4,>=stealth]
    \draw ( 1,3) rectangle +(3,1); \node[left] at (1,3.5) {$m_1$};
    \draw ( 0,0) rectangle +(9,1); \node[left] at (0,0.5) {$m_2$};
    \node (V1) at (6.5,3.5) {$\vec{v}_1$};
    \node (V2) at (2.5,0.5) {$\vec{v}_2$};
    \draw[->] (1,4) -- (1,1); \draw[dotted] (1,1) -- (1,0); \draw[dotted] (4,4) -- (4,0);
    \draw[->] (V1) -- (V2);
  \end{tikzpicture}
  \qquad
  \begin{tikzpicture}[scale=0.4,>=stealth]
    \draw ( 1,3) rectangle +(3,1); \node[left] at (1,3.5) {$m_1$};
    \draw ( 0,0) rectangle +(9,1); \node[left] at (0,0.5) {$m_2$};
    \node (V1) at (6.5,3.5) {$\vec{v}_1$};
    \node (V2) at (6.5,0.5) {$\vec{v}_2$};
    \draw[->] (1,4) -- (1,1); \draw[dotted] (1,1) -- (1,0); \draw[dotted] (4,4) -- (4,0);
    \draw[->] (V1) -- (V2);
  \end{tikzpicture}
\]
For details,
see \texttt{cc\_stacking} in \texttt{backend/Mach.v}.

%}}}

\subsection{The \kw{Asmgen} pass} \label{sec:asmgen} %{{{

The \kw{Asmgen} pass from \kw{Mach} to \kw{Asm}
uses a memory extension.
\kw{Asm} introduces explicit registers for the
program counter, stack pointer and return address.
The corresponding simulation convention
$\kw{asmgen} : \mathcal{M} \Leftrightarrow \mathcal{A}$
ensures that the appropriate components of
\kw{Mach}-level queries are mapped to the new registers.
In addition,
we must ensure that the call returns the stack pointer
to its original value,
and set the program counter to the return address
specified by the caller.

\begin{optional}
[XXX: we don't talk about nextblock in our simplified
exposition of the memory model]
A more complex challenge is that the $\kw{Asm}$ language
does not have a control stack,
but instead executes instruction after instruction
in a ``flat'' manner,
making it difficult to distinguish final states.
To address this,
we keep track of the initial value $\mathit{nb}_0$ of
the memory's $\kw{nextblock}$ counter,
and use it to distinguish between inner and outer
stack pointers.
When $\mathit{rs}[\kw{sp}] \ge \mathit{nb}_0$,
we interpret the \kw{ret} instruction as an \emph{internal} return
and simply jump to the location pointed to by $\mathit{rs}[\kw{ra}]$.
However, when $\mathit{rs}[\kw{sp}] < \mathit{nb}_0$,
we intepret \kw{ret} as a top-level return to the environment,
and model its behavior as a final state
rather than an internal step.
In addition,
the simulation convention $\kw{asmgen}$
must ensure that the initial value of the stack pointer
points to a valid block of the initial memory.
\end{optional}

This is expressed as:
\begin{gather*}
  \kw{asmgen} := \langle \kw{val} \times \kw{val},
    R_\kw{asmgen}^\que, R_\kw{asmgen}^\ans \rangle
  \\[1ex]
  \AxiomC{$\mathit{rs}_1 \uplus
    [\kw{sp} := \mathit{sp}, \kw{ra} := \mathit{ra}, \kw{pc} := \mathit{vf}]
    \vref \mathit{rs}_2$}
  \AxiomC{$m_1 \mext m_2$}
  \BinaryInfC{
    $(\mathit{sp}, \mathit{ra}) \Vdash
     \mathit{vf}(\mathit{sp}, \mathit{ra}, \mathit{rs}_1)@m_1
     \mathrel{R_\kw{asmgen}^\que}
     \mathit{rs}_2@m_2$}
  \DisplayProof
  \\[1ex]
  \AxiomC{$
    \mathit{rs}_1' \uplus [\kw{sp} := \mathit{sp}, \kw{pc} := \mathit{ra}]
    \vref \mathit{rs}_2'$}
  \AxiomC{$m_1' \mext m_2'$}
  \BinaryInfC{$
    (\mathit{sp}, \mathit{ra}) \Vdash \mathit{rs}_1'@m_1'
    \mathrel{R_\kw{asmgen}^\ans}
    \mathit{rs}_2'@m_2'$}
  \DisplayProof
\end{gather*}

%}}}

%}}}

\section{Related work and evaluation} \label{sec:rw} %{{{

A general survey,
discussion and synthesis of various
compositional compiler correctness results
is provided by \citet{next700}.
We focus on CompCert extensions.
Our conceptual framework
can be used to classify previous work
on semantic models of CompCert.
By reinterpreting these models
in terms of games and strategies,
we can establish the taxonomy presented in
Table~\ref{tbl:compcerts}.

\paragraph{CompCert and SepCompCert} %{{{

As noted in \S\ref{sec:mainideas:compcerto},
the whole-program semantics used by CompCert
express strategies for the game
$\mathbf{1} \twoheadrightarrow \mathcal{W}$.
CompCert's original correctness theorem
stated the refinement property
$\kw{C}_\kw{wp}(p) \refby \kw{Asm}_\kw{wp}(p')$,
where $\kw{C}_\kw{wp}$ and $\kw{Asm}_\kw{wp}$
denote the source and target whole-program semantics.
SepCompCert \cite{sepcompcert}
later introduced the linking operator $+$
and generalized the correctness theorem to
the form discussed in \S\ref{sec:sem:closed}.
%Our work uses the same notion of linking
%for target \kw{Asm} programs.

Since external calls are not accounted for explicitly
in this semantic model,
they are interpreted %in every language semantics
using a common global parameter $\chi$
specifying their behavior.
The correctness proof assumes that $\chi$ is deterministic
and that it satisfies a number of healthiness requirements
with respect to the memory transformations
used in CompCert's correctness proof,
corresponding roughly to self-simulation
under the CKLRs $\kw{ext}$ and $\kw{injp}$.

%}}}

\paragraph{Contextual compilation} %{{{

CompCertX \cite{popl15} and
Stack-Aware CompCert \cite{stackaware}
generalize
the incoming interface of programs
from $\mathcal{W}$ to $\mathcal{C}$,
and as such characterizes the behavior
not only of \texttt{main}
but of any function of the program,
called with any argument values.
This allows CompCertX and its correctness theorem
to be used in the layer-based verification of
the CertiKOS kernel:
once the code of a given abstraction layer has been verified
and compiled using CompCertX,
that layer's specification can be used as the new $\chi$
when the next layer is verified.
%This enabled the verification of CertiKOS ---
%an artifact of significant size ---
%at the level of assembly code,
%by integrating CompCert as part of its proof of correctness.
However,
this approach does not support
mutually recursive components,
and requires the healthiness conditions on $\chi$
to be proved before the next layer is added.

%}}}

\paragraph{Compositional CompCert} %{{{

The \emph{interaction semantics} of
Compositional CompCert \cite{compcompcert}
are closer to our own model
but are limited to the language interface $\mathcal{C}$.
Likewise, the simulations used in Compositional CompCert
correspond to our notion of forward simulation
for a single convention called \emph{structured injections},
which we will write $\mathbb{SI}$.
Simulation proofs are updated to follow this model,
and the \emph{transitivity} of $\mathbb{SI}$ is established
($\mathbb{SI} \cdot \mathbb{SI} \equiv \mathbb{SI}$),
so that passes can be composed
to obtain a simulation for the whole compiler.

Compositional CompCert also introduced a notion of \emph{semantic linking}
similar to our horizontal composition
(\S\ref{sec:sem:linker}).
As in our case,
semantic linking is shown to preserve simulations
(Thm.~\ref{thm:fsim-hcomp}),
however semantic linking is not related to
syntactic linking of assembly programs,
and this was later shown to be problematic \cite{compcertm}.

Another limitation of Compositional CompCert
is the complexity of the theory
and the proof effort required.
Because of the use of a single simulation convention,
many assumptions naturally expressed as
relational invariants in the simulation relations of CompCert
must be either captured by $\mathbb{SI}$
or handled at the level of language semantics,
and simulation proofs
essentially had to be rewritten and adapted to
structured injections.

%}}}

\paragraph{CompCertM} %{{{

The most recent extension of CompCert is CompCertM \cite{compcertm},
which shares common themes and was developped concurrently
with our work.
While its correctness
is ultimately stated in terms of closed semantics,
CompCertM uses a notion of open semantics
as an intermediate construction
to enable compositional compilation and verification.

The open semantics used in CompCertM
builds on interaction semantics
by incorporating an assembly language interface.
The resulting semantic model can be characterized as
$\mathcal{C} \times \mathcal{A} \twoheadrightarrow
 \mathcal{C} \times \mathcal{A}$.
%XXX or is it (C -> C) x (A -> A) ?
Simulations
are parametrized by Kripke relations similar to CKLRs (\S\ref{sec:cklr})
and predicates similar to our invariants (\S\ref{sec:inv}).
While simulations do not directly compose,
a new technique called \emph{refinement under self-related context}
(RUSC)
can nonetheless be used to derive a contextual refinement theorem
for the whole compiler with minimal overhead.

This approach has many advantages.
CompCertM avoids much of the complexity
of Compositional CompCert
when it comes to composing passes,
and the flexibility of the simulations used
makes updating the correctness proofs of passes much easier.
CompCertM also charts new ground in several directions.
The RUSC relation used to state the final theorem
is shown to be adequate with respect to the trace semantics
of closed programs.
CompCertM has improved support for static variables
%and module-local state,
%can be used to carry out compositional verification,
and the verification of
the assembly runtime function \texttt{utod} is demonstrated.

In other aspects,
CompCertM inherits the limitation of previous approaches
whereas CompCertO goes further.
Because the compiler correctness theorem
is not itself expressed as a simulation,
it fails requirement \#\ref{req:opensim}
laid out in \S\ref{sec:compcertreq}.
While it may be possible to accomodate \#\ref{req:openabs},
the parametrization of simulations
does not offer the same flexibility as
our full-blown notion of simulation convention.
As a consequence, a cascade of techniques
(\emph{repaired} interaction semantics,
\emph{enriched} memory injections,
the mixed simulations of \cite{pilsner})
need to be deployed to enforce invariants
which find a natural relational expression
under our appoach.

Therefore,
an interesting question for future investigation
will be to determine to what extent
the techniques used by CompCertM and CompCertO
could be integrated to combine
the strengths of both developments.

%}}}

\paragraph{CompCertO}

\begin{table} % tbl:slocs {{{
  \figsize
  \begin{tabular}{lrr}
    \hline
    Component & SLOC & \\ %\multicolumn{2}{r}{SLOC} \\
    \hline
    Semantic framework (\S\ref{sec:sem}) & $+782$ & ($+14\%$) \\
    Horizontal composition (\S\ref{sec:sem:linker}) & $676$ & \\
    Simulation convention algebra (\S\ref{sec:simalg}) & $1{,}052$ & \\
    CKLR theory and instances (\S\ref{sec:cklr}) & $1{,}807$ & \\
    \kw{Clight} and \kw{RTL} parametricity (\S\ref{sec:cklrsc}) & $2{,}741$ & \\
    Invariant preservation proofs (\S\ref{sec:inv}) & $+549$ & ($+7\%$) \\
    Pass correctness proofs (Tbl.~\ref{tbl:passes}) & $+765$ & ($+2\%$) \\
    \textbf{Total} & \textbf{8372} & \\
    \hline
  \end{tabular}
  \caption{Significant lines of code in CompCertO
    relative to CompCert v$3.6$.
    See Table~\ref{tbl:passes}
    for a per-pass breakdown of the increase in size
    of pass correctness proofs,
    and \texttt{overhead.py} in the Coq development
    for the list of files included in each group.}
  \label{tbl:slocs}
\end{table}
%}}}

To give a sense of the overall complexity of CompCertO,
we list in Table~\ref{tbl:slocs}
the increase in significant lines of code it introduces
compared to CompCert v$3.6$.
As shown in Table~\ref{tbl:passes},
our methodology comes with a negligible increase
in the complexity of most simulation proofs.
Although SLOC is an imperfect measure,
and a 1:1 comparison between developments which
prove different things is difficult,
our numbers represent
a drastic improvement over Compositional CompCert,
and compare favorably
or are on par with
the corresponding sections of CompCertM.

% XXX the family of CompCert variants not yet introduced
Our use of the simulation conventions
\kw{injp}, \kw{alloc}, \kw{stacking} and \kw{asmgen}
in particular
underscores the benefits of our approach.
The corresponding passes are the root of
much complexity
in Compositional CompCert, CompCertX and CompCertM.
For instance,
to express the requirement on
the areas protected by \kw{injp},
both CompCompCert and CompCertM
introduce general mechanisms for tracking ownership of
different regions of memory
as part of an extended notion of memory injection.
The approach taken here demonstrates that
the requirements placed on external functions
by the original CompCert
are already good enough for the job!
Because our framework is expressive enough to capture them,
the corresponding passes barely need any modifications,
and the associated issues are resolved before they even show up.

Likewise, the preservation of callee-save registers
ensured by the \kw{Allocation} pass,
and the subtle issues associated with argument-passing
in the \kw{Stacking} pass
have been the cause of much pain
in previous CompCert extensions.
The ease with which they are addressed here
demonstrates the power of
an explicit treatment of abstraction,
made possible
by our notions of language interface and simulation convention.

%}}}

\section{Conclusion} \label{sec:concl} %{{{

The distinguishing feature of CompCertO
is the expressivity of our model,
which allowed us to formulate a more precise correctness theorem
and offered flexibility
in the formalization and deployment of
our composition techniques.
It is worth noting that
the resulting structures have much in common with
those found in
certified abstraction layers \cite{popl15,rbgs-cal}.
Hence,
we hope that our work will not only
provide a compiler
suitable for use in the end-to-end verification of
large-scale heterogenous systems,
but also constitute an important contribution
to the conceptual apparatus
required for this formidable undertaking.

%}}}

\bibliography{../references}

\end{document}
