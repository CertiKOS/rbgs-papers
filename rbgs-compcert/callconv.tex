\documentclass[11pt]{article}
\usepackage{geometry}
\usepackage{amsmath}
\usepackage{amssymb}
\usepackage{ebproof}

\title{Establishing the CompCertO calling convention}

\begin{document}

\maketitle

\section{Background}

The \emph{simulation conventions} of CompCertO
specify the relationship between
source and target interactions
in open simulations.
We use three kinds of simulation conventions:
\begin{itemize}
  \item Memory transformations ($\mathsf{inj}$, $\mathsf{ext}$, $\mathsf{injp}$)
    do not alter the structure of queries and replies but
    allow the memory states and surrounding values
    to be transformed according to a CompCert KLR
    (see paper draft).
  \item Calling conventions ($\mathcal{C\!L}, \mathcal{L\!M}, \mathcal{M\!A}$)
    encode the correspondance between the various language interfaces.
  \item Invariants ($\mathsf{wt}, \mathsf{wtl}, \mathsf{va}$)
    require that source and target queries are identical,
    and additionally that they satify some property of interest.
    Schematically, they lift a predicate to a relation:
    \[
      R[I] := \{ (x, x) \mid x \in I \}
    \]
\end{itemize}
Simulation conventions support a Kleene algebra,
which involves notions of refinement ($\supseteq$),
vertical composition ($\cdot$),
identity ($\mathsf{id}$),
environment choice ($+$),
and finite iteration ($*$) of simulation conventions.

Based on these constructions,
we can assign to each pass of CompCert
a simulation type $\mathbb{R}_A \twoheadrightarrow \mathbb{R}_B$
where $\mathbb{R}_A, \mathbb{R}_B$ are simulation conventions.
Here are some examples:
\begin{center}
  \begin{tabular}{lc}
    \hline
    Pass & Type \\
    \hline
    Typical extension pass &
      $\mathsf{ext} \twoheadrightarrow \mathsf{ext}$ \\
    Typical injection pass &
      $\mathsf{injp} \twoheadrightarrow \mathsf{inj}$ \\
    $\mathsf{Alloc}$ &
      $\mathsf{wt} \cdot \mathsf{ext} \cdot \mathcal{C\!L}
       \twoheadrightarrow
       \mathsf{wt} \cdot \mathsf{ext} \cdot \mathcal{C\!L}$ \\
    Value analysis passes &
      $\mathsf{va} \cdot \mathsf{ext} \twoheadrightarrow
       \mathsf{va} \cdot \mathsf{ext}$ \\
    \hline
  \end{tabular}
\end{center}
When the correctness proofs for the various passes
are vertically composed into one big simulation,
their simulation conventions are likewise composed
to obtain the simulation convention for the result.

Unfortunately,
the simulation convention obtained through this process
cannot be used as-is for the whole compiler:
\begin{itemize}
  \item Because of the asymmetry of injection passes
    ($\mathsf{injp}$ vs. $\mathsf{inj}$),
    the result is not horizontally compositional:
    it provides weaker guarantees on incoming calls ($\mathsf{inj}$)
    than it expects on outgoing calls ($\mathsf{injp}$).
  \item The result is also tied to the structure of the compiler,
    and varies with the inclusion of optional passes.
\end{itemize}
To address this,
we use the Kleene algebra and
properties of specific simulation conventions
to rewrite the simulation convention for the whole compiler
into a more palatable form.

\section{Approach}

\subsection{The CompCertO simulation convention}

Leaving aside invariants for a moment,
my strategy is to establish the following
simulation convention for CompCertO:
\[
  \mathbb{C} :=
  (\mathsf{ext} + \mathsf{injp})^* \cdot
  (\mathcal{C\!L} \cdot
   \mathcal{L\!M} \cdot
   \mathcal{M\!A}) \cdot
  \mathsf{inj}
\]
The structure is as follows:
\begin{itemize}
  \item
    The first component
    allows the environment to introduce
    memory injections and extensions
    between the source and target-level interactions.
    It is similar in spirit to the $\mathsf{extcall\_properties}$
    formalized in CompCert for external calls.
  \item
    The second component 
    expresses the correspondance between
    C-style and Asm-style interactions.
    In the original CompCert this is encoded by
    the external call case of the Asm semantics,
    in particular through the predicate
    $\mathsf{Asm.extcall\_arguments}$.
  \item
    The third component allows the compiler to introduce an injection.
    This is necessary to model the transformation in the structure of the memory
    between the C and assembly program:
    local variables and arguments become consolidated into stack frames, etc.
\end{itemize}

\subsection{Matching it with the passes}

To establish the simulation convention above
for the whole compiler,
we need to show two properties:
\begin{enumerate}
  \item Incoming calls performed according to $\mathbb{C}$
    should be compatible with the composite incoming convention
    of CompCertO's passes ($\mathbb{C} \sqsupseteq \mathbb{C}_\mathsf{in}$).
  \item Outgoing calls performed by the passes
    should be compatible with the overall simulation convention
    ($\mathbb{C}_\mathsf{out} \sqsupseteq \mathbb{C}$).
\end{enumerate}
Together, this allows us to establish a final correctness theorem for CompCertO
expressed exclusively in terms of $\mathbb{C}$:
\[
  \mathsf{CompCert}(p_s) = p_t
  \: \Rightarrow \:
  \mathsf{Clight}(p_s)
  \le_{\mathbb{C} \twoheadrightarrow \mathbb{C}}
  \mathsf{Asm}(p_t)
\]
Below I outline a number of strategies
we can use to make this possible.

\subsection{Self-simulations}

The first arrow in our quiver
is to introduce identity pseudo-passes
in the compiler,
in the form of self-simulations for various language semantics.
In particular,
we can establish parametricity properties
for $\mathsf{Clight}$ and $\mathsf{Asm}$,
showing that for any CompCert KLR $R$:
\begin{align*}
  \mathsf{Clight}(p) &\le_{R \twoheadrightarrow R} \mathsf{Clight}(p) \\
  \mathsf{Asm}(p) &\le_{R \twoheadrightarrow R} \mathsf{Asm}(p)
\end{align*}
These properties can be used to satisfy
the $(\mathsf{ext} + \mathsf{injp})^*$ and $\mathsf{inj}$
components of $\mathbb{C}$.

\subsection{Properties of specific conventions}

Once we introduce the self-simulations,
the conventions for CompCertO passes look something like:
\begin{gather*}
  \mathbb{C}_\mathsf{in} :=
    (\mathsf{ext} + \mathsf{injp})^* \cdot
    \mathsf{inj} \cdot
    \mathsf{inj} \cdot
    \mathsf{ext} \cdot
    \mathsf{ext} \cdots
    \mathsf{ext} \cdot
    \mathcal{C\!L} \cdot
    \mathsf{ext} \cdot
    \mathcal{L\!M} \cdot
    \mathsf{inj} \cdot
    \mathcal{M\!A} \cdot
    \mathsf{ext} \cdot
    \mathsf{inj}
    \\
  \mathbb{C}_\mathsf{out} :=
    (\mathsf{ext} + \mathsf{injp})^* \cdot
    \mathsf{injp} \cdot
    \mathsf{injp} \cdot
    \mathsf{ext} \cdot
    \mathsf{ext} \cdots
    \mathsf{ext} \cdot
    \mathcal{C\!L} \cdot
    \mathsf{ext} \cdot
    \mathcal{L\!M} \cdot
    \mathsf{injp} \cdot
    \mathcal{M\!A} \cdot
    \mathsf{ext} \cdot
    \mathsf{inj}
\end{gather*}
Again, they need to be matched against the overall simulation convention:
\[
  \mathbb{C} :=
  (\mathsf{ext} + \mathsf{injp})^* \cdot
  \mathcal{C\!L} \cdot
  \mathcal{L\!M} \cdot
  \mathcal{M\!A} \cdot
  \mathsf{inj}
\]
Fortunately,
we can use properties of various simulation conventions to achieve this.

First,
it is possible to use the final $\mathsf{inj}$
to satisfy the various memory transformations
expected by the passes on incoming calls.
More precisely,
the following properties hold:
\[
  \mathsf{inj} \sqsupseteq \mathsf{ext} \cdot \mathsf{inj}
  \qquad
  \mathsf{inj} \sqsupseteq \mathsf{inj} \cdot \mathsf{inj}
\]
Then,
I was able to defined the calling conventions
$\mathcal{C\!L}, \mathcal{L\!M}, \mathcal{M\!A}$
so that they commute with arbitrary CompCert KLRs
in the following way:
\[
  \mathcal{X\!Y} \cdot R
  \sqsupseteq
  R \cdot \mathcal{X\!Y}
\]
This allows us to rewrite the convention for incoming calls
as follows:
\begin{align*}
  \mathbb{C} &=
    (\mathsf{ext} + \mathsf{injp})^* \cdot
    \mathcal{C\!L} \cdot
    \mathcal{L\!M} \cdot
    \mathcal{M\!A} \cdot
    \mathsf{inj} \\
  &\sqsupseteq
    (\mathsf{ext} + \mathsf{injp})^* \cdot
    \mathcal{C\!L} \cdot
    \mathcal{L\!M} \cdot
    \mathcal{M\!A} \cdot
    \mathsf{inj} \cdot
    \mathsf{inj} \cdot
    \mathsf{ext} \cdot
    \mathsf{ext} \cdots
    \mathsf{ext} \cdot
    \mathsf{ext} \cdot
    \mathsf{inj} \cdot
    \mathsf{ext} \cdot
    \mathsf{inj} \\
  &\sqsupseteq
    (\mathsf{ext} + \mathsf{injp})^* \cdot
    \mathsf{inj} \cdot
    \mathsf{inj} \cdot
    \mathsf{ext} \cdot
    \mathsf{ext} \cdots
    \mathsf{ext} \cdot
    \mathcal{C\!L} \cdot
    \mathsf{ext} \cdot
    \mathcal{L\!M} \cdot
    \mathsf{inj} \cdot
    \mathcal{M\!A} \cdot
    \mathsf{ext} \cdot
    \mathsf{inj}
  =
    \mathbb{C}_\mathsf{in}
\end{align*}

For outgoing calls,
we can use the same commutation properties
to continue pushing the memory transformations
towards the source level,
where they can be absorbed
by the $(\mathsf{ext} + \mathsf{injp})^*$ component
thanks to the Kleene algebra properties:
\begin{align*}
  \mathbb{C}_\mathsf{out}
  &=
    (\mathsf{ext} + \mathsf{injp})^* \cdot
    \mathsf{injp} \cdot
    \mathsf{injp} \cdot
    \mathsf{ext} \cdot
    \mathsf{ext} \cdots
    \mathsf{ext} \cdot
    \mathcal{C\!L} \cdot
    \mathsf{ext} \cdot
    \mathcal{L\!M} \cdot
    \mathsf{injp} \cdot
    \mathcal{M\!A} \cdot
    \mathsf{ext} \cdot
    \mathsf{inj} \\
  &\sqsupseteq
    (\mathsf{ext} + \mathsf{injp})^* \cdot
    \mathsf{injp} \cdot
    \mathsf{injp} \cdot
    \mathsf{ext} \cdot
    \mathsf{ext} \cdots
    \mathsf{ext} \cdot
    \mathsf{ext} \cdot
    \mathsf{injp} \cdot
    \mathsf{ext} \cdot
    \mathcal{C\!L} \cdot
    \mathcal{L\!M} \cdot
    \mathcal{M\!A} \cdot
    \mathsf{inj} \\
  &\sqsupseteq
    (\mathsf{ext} + \mathsf{injp})^* \cdot
    \mathcal{C\!L} \cdot
    \mathcal{L\!M} \cdot
    \mathcal{M\!A} \cdot
    \mathsf{inj}
  = \mathbb{C}
\end{align*}

By combining these two approaches,
we can establish the calling convention
by rewriting in the overall simulation for the compiler:
\[
  \begin{prooftree}
    \hypo{\mathbb{C}_\mathsf{out} \sqsupseteq \mathbb{C}}
    \hypo{\mathsf{Clight}(p_s)
          \le_{\mathbb{C}_\mathsf{out} \twoheadrightarrow \mathbb{C}_\mathsf{in}}
          \mathsf{Asm}(p_t)}
    \hypo{\mathbb{C} \sqsupseteq \mathbb{C}_\mathsf{in}}
    \infer3{\mathsf{Clight}(p_s)
          \le_{\mathbb{C} \twoheadrightarrow \mathbb{C}}
          \mathsf{Asm}(p_t)}
  \end{prooftree}
\]

\section{Challenges}

\subsection{Invariants in CompCertO}

The basic principle outlined above is fairly straightforward;
unfortunately the big picture is complicated by
three different invariants required by various passes:
\begin{itemize}
  \item The $\mathsf{Selection}$ and $\mathsf{Allocation}$ passes,
    which use the $\mathcal{C}$ language interface,
    rely on a typing invariant:
    the values for function call arguments and returns
    must have the type specified by the corresponding signature.
    This is expressed by the invariant $\mathsf{wt}_\mathcal{\!C}$.
  \item The $\mathsf{Stacking}$ pass translates from
    $\mathsf{Linear}$ to $\mathsf{Mach}$.
    At the $\mathsf{Linear}$ level,
    calls are specified by assigning values to
    \emph{abstract locations} ($\mathsf{loc} \rightarrow \mathsf{val}$)
    which represent stack slots and registers.
    Each possible location ($\mathsf{loc}$) is assigned a type,
    which expresses whether they can hold 32 or 64 bits values.
    The $\mathsf{Stacking}$ proofs relies on the fact that
    the location maps found in calls and returns
    assign to each location a value of the right type.
    This is expressed as $\mathsf{wt}_\mathcal{\!L}$.
  \item Finally, passes which use the $\mathsf{ValueAnalysis}$ framework
    need the memory to satify some properties.
    In particular, global constants should be initialized
    as prescribed by the symbol table,
    and should be read-only.
\end{itemize}

The main difficulty with these invariants
is that although they are preserved by the passes that use them,
they do not commute with CKLRs
in the way that $\mathcal{C\!L}, \mathcal{L\!M}, \mathcal{M\!A}$ do.
This is due to issues of mixed variance
which arise when both queries and replies are involved,
as well as with the invariant $\mathsf{va}$ itself.

\subsection{Typing invariants}

For typing invariants,
this can be addressed by defining a simulation convention
$\hat{\mathsf{wt}}$
which commutes with CKLRs and is weaker than the original $\mathsf{wt}$,
but can be shown to be equivalent in some contexts.
Specifically,
it satisfies the following properties:
\[
  \hat{\mathsf{wt}} \cdot R \equiv \mathsf{wt} \cdot R
  \qquad
  \hat{\mathsf{wt}} \equiv \mathsf{wt} \cdot \hat{\mathsf{wt}}
  \qquad
  \hat{\mathsf{wt}} \cdot R \sqsupseteq R \cdot \hat{\mathsf{wt}}
  \,.
\]
We can then modify the CompCertO simulation convention
to require the appropriate typing properties:
\[
  \mathbb{C} :=
    (\mathsf{ext} + \mathsf{injp})^* \cdot
    \hat{\mathsf{wt}}_\mathcal{\!C} \cdot
    \mathcal{C\!L} \cdot
    \hat{\mathsf{wt}}_\mathcal{\!L} \cdot
    \mathcal{L\!M} \cdot
    \mathcal{M\!A} \cdot
    \mathsf{inj}
\]
The approach outlined in the previous section
can then be extended to take the $\mathsf{wt}$s into account.

The idea is that $\hat{\mathsf{wt}}$
will force the source value to $\mathsf{Vundef}$
when the corresponding target value
doesn't have the right type.

\subsection{Value analysis invariant}

The value analysis invariant $\mathsf{va}$
is more challenging.
It includes constraints with mixed variance
with respect to CKLRs.
For example,
the global constants should contain the right values
(covariant with CKLRs),
but should only be assigned limited permissions
(contravariant with CKLRs).

On the other hand,
the $\mathsf{va}$ properties
are preserved by all of the operations of the memory model,
because once a global constant has been properly initialized
and set to read-only,
it cannot be modified any more.
Also,
unlike typing invariants,
the $\mathsf{va}$ invariant is uniform
with all language interfaces'
(although it is currently defined for $\mathcal{C}$-style interactions only).
This means the invariant applies in the same way
to all values that appear in queries and replies
surrounding the memory model,
unlike typing which depends on \emph{how} a given value is used
(say, for which function argument).

So my plan is to encode the $\mathsf{va}$
invariant in terms of CKLRs,
and use the existing parametricity properties of languages
to handle it.
Namely, I want to define a CKLR operator $R^\mathsf{va}$
which should have properties like:
\[
  R^\mathsf{va}_\mathcal{\!C} \equiv \mathsf{va} \cdot R_\mathcal{\!C}
  \qquad
  \mathsf{inj}^\mathsf{va} \sqsupseteq \mathsf{injp}
  \qquad
  \mathsf{inj}^\mathsf{va} \sqsupseteq \mathsf{inj}^\mathsf{va} \cdot \mathsf{inj}
\]
as well as other things.
I still need to figure out the details,
but the idea is that we would use this instead of the final $\mathsf{inj}$
in $\mathbb{C}$ to make sure the proper invariant holds,
and that as a CKLR it will easily commute with the calling conventions
and $\hat{\mathsf{wt}}$.

\section{Plan}

\begin{itemize}
  \item Show the parametricity of $\mathsf{Asm}$ wrt. CompCert KLRs
  \item Define a CKLR capturing the $\mathsf{va}$ requirements
  \item Finish the compiler correctness theorem
    to confirm that things will work out if we can achieve this.
\end{itemize}

\end{document}
