\documentclass[aspectratio=1610,mathserif]{beamer}
\usepackage{amsmath}
\usepackage{cmll}
\usepackage{stmaryrd}

\title{CompCertOX: Outline and Plan}

\begin{document}

\maketitle

\begin{frame}{Paper Outline}
  \begin{enumerate}
    \item Introduction
    \item Main Ideas
    \item Interaction Specifications
    \item CompCertO Semantics
    \item Refinement Conventions
    \item Implementing CAL
    \item Related Work
    \item Conclusion
  \end{enumerate}
\end{frame}

\begin{frame}{Section 1: Introduction}
  \begin{description}
    \item[Context]
      Certified software needs large-scale compositionality.
      Usual compositional proof techniques only operate locally.
      Compositional semantics offer a principled solution.
    \item[Gap]
      The problem is,
      existing work focuses on theoretical understanding
      of languages and features,
      not verification.
      In particular,
      treatment of refinement and abstraction not satisfactory.
      Some progress:
      interaction trees, LICS'20, CompCertO, POPL'22.
      But not many applications.
    \item[Innovation] 
      We show that LICS'20 model can support CompCertO,
      formalize it in Coq,
      and demonstrate verification of bounded queue.
  \end{description}
\end{frame}

\begin{frame}{Section 2: Main Ideas}
  \begin{itemize}
    \item OG CAL
    \item CompCertO
    \item Dual nondeterminism
    \item LICS'20 model
  \end{itemize}
\end{frame}

\begin{frame}{Section 3: Interaction Specifications}
  Describe and revisit the LICS '20 model
  in the ACT '21 framework:
  \begin{itemize}
    \item Interaction spec monad as a \emph{free} monad
    \item Associated induction principle to make proofs easier
  \end{itemize}

  Specifically, $\mathcal{I}_E(X)$ is a free algebra for
  the functor $E : \mathbf{CDLat} \rightarrow \mathbf{CDLat}$
  defined as:
  \[
    EY :=
     \sum_{m \in E^\circ} \left( \prod_{n \in E^\bullet_m} Y \right)_\bot
     + X
  \]
  Not sure how to define $\sum$ in $\mathbf{CDLat}$ but that's fine,
  can just describe algebra $\alpha : EA \rightarrow A$
  as family of functions $(\alpha^m)_{m \in E^\circ}$
  plus valuation $X \rightarrow A$.
\end{frame}

\begin{frame}{Section 4: CompCertO Semantics}
  Here we give the embedding of CompCertO LTS into the LICS '20 model,
  based on Yu's work and the text from my thesis:
  \begin{itemize}
    \item Angelic interpretation only
    \item Focus on categorical composition
    \item Don't use footprints for $\circ$
    \item Linking still implements $\circ$ as Yu showed
  \end{itemize}
\end{frame}

\begin{frame}{Section 5: Refinement Conventions}
  Generalize simulation conventions
  from language interfaces to effect signatures.
  The result is a category where $\otimes/@$ is a bifunctor.
  Refinement conventions can still be represented as
  formal adjunctions in the LICS '20 model.

  \begin{itemize}
    \item Explain that
    \item Relations on state -> refinement conventions
    \item Simulation conventions -> refinement conventions
    \item Soundness of forward simulations
  \end{itemize}

  Sets the stage for CompCertOX.
\end{frame}

\begin{frame}{Section 6: Implementing CAL}
  \begin{itemize}
    \item Interfacing CompCertO and underlays
    \item Layer patterns and proof methods
    \item Soundness of CompCertO and Asm linking
    \item Example: bounded queue
  \end{itemize}
\end{frame}

\begin{frame}{Status and Plan}
  Current status:
  \begin{itemize}
    \item Good picture of the whole paper
    \item Many fragments for \S\S 1--5
    \item Some code related to \S 3 (interaction spec. algebras)
    \item Older stuff + Yu's work will fit into \S 6
  \end{itemize}

  By Friday:
  \begin{itemize}
    \item Jeremie: consolidate what I have into a complete draft of \S\S 1--5
    \item Jeremie: finish proving new induction principle for $\mathcal{I}_E$
    \item Yu: make sure refinement conventions work
  \end{itemize}

  ASAP:
  \begin{itemize}
    \item Jeremie: sketch out \S 6
    \item Yu: progress on verifying actual code
  \end{itemize}
\end{frame}

\end{document}
