\documentclass{article}
%\usepackage{geometry}
\usepackage[colorlinks]{hyperref}
\usepackage{ebproof}
\usepackage{amsmath}
\usepackage{amssymb}
\usepackage{libertine}
\usepackage{stmaryrd}
\usepackage{tikz}
\usepackage{tikz-cd}

\newtheorem{remark}{Remark}

\title{CompCertO\hspace{-0.5ex}\raisebox{-0.5ex}{X}!}
\author{J\'er\'emie Koenig}

\begin{document}

\maketitle

\vfill

\begin{figure}[h] % fig:ex {{{
  \begin{center}
    \begin{tikzcd}[row sep=1cm, column sep=1.5cm]
      1 \ar[d, double, dash] \ar[rrr, ->>, "L''"] &&&
      \mathcal{C}@K''
        \ar[d, <->, "S"]
        \ar[r, ->>, "\llbracket C \rrbracket@K''"] &
      \mathcal{C}@K''
        \ar[ddd, <->, "R \circ S"]
      \\
      1 \ar[d, double, dash] \ar[rr, ->>, "L'"] &&
      \mathcal{C}@K'
        \ar[d, <->, "R"]
        \ar[r, ->>, "\llbracket N \rrbracket@K'"] &
      \mathcal{C}@K'
        \ar[d, <->, "R"]
      \\
      1 \ar[d, double, dash] \ar[r, ->>, "L"] &
      \mathcal{C}@K
        \ar[d, double, dash] 
        \ar[r, ->>, "\llbracket M \rrbracket@K"] &
      \mathcal{C}@K
        \ar[r, ->>, "\llbracket N \rrbracket@K"] &
      \mathcal{C}@K
        \ar[d, double, dash]
      \\
      1 \ar[d, double, dash] \ar[r, ->>, "L"] &
      \mathcal{C}@K
        \ar[d, double, dash] 
        \ar[rr, ->>, "\llbracket M + N \rrbracket@K"] &&
      \mathcal{C}@K
        \ar[r, ->>, "\llbracket C \rrbracket@K"] &
      \mathcal{C}@K
        \ar[d, double, dash] 
      \\
      1 \ar[r, ->>, "L"] &
      \mathcal{C}@K
        \ar[rrr, ->>, "\llbracket C + M + N \rrbracket@K"] &&&
      \mathcal{C}@K
    \end{tikzcd}
  \end{center}
  \caption}}

\vfill

\section{Overview} %{{{

Our usual
\href{https://certikos.github.io/rbgs-papers/thesis/thesis.pdf\#chapter.4}%
  {theory of abstraction layers}
can be formulated in CompCertO's double category of
language interfaces, simulation conventions and transition systems:
\begin{itemize}
  \item A layer interface $L$ with abstract states in $K$
    is represented as $L : 1 \twoheadrightarrow \mathcal{C}@K$
  \item Clight semantics define
    $\llbracket M \rrbracket : \mathcal{C} \twoheadrightarrow \mathcal{C}$
    and lift to
    $\llbracket M \rrbracket @ K :
     \mathcal{C}@K \twoheadrightarrow \mathcal{C}@K$
  \item Abstraction relations define simulation conventions
    $R : \mathcal{C}@K' \Leftrightarrow \mathcal{C}@K$
  \item Layer correctness
    $L \vdash_R M : L'$
    is encoded as
    $L' \le_{\mathsf{id} \twoheadrightarrow R}
     \llbracket M \rrbracket @K \circ L$
\end{itemize}
\autoref{fig:ex}
demonstrates how these ingredients may fit together
in a typical situation.

\newpage
The following operators
are used to formulate vertical and horizontal composition
of abstraction layers.
In the homogenous case
$(A \twoheadrightarrow A) \times
 (A \twoheadrightarrow A) \rightarrow
 (A \twoheadrightarrow A)$,
they under-approximate $\oplus$
and can therefore be implemented by linking.
\begin{itemize}
  \item Categorical composition \hfill
    $\circ :
      (B \twoheadrightarrow C) \times
      (A \twoheadrightarrow B) \rightarrow
      (A \twoheadrightarrow C) \qquad$
  \item Flat composition \hfill
    $\uplus :
      (A \twoheadrightarrow B) \times
      (A \twoheadrightarrow B) \rightarrow
      (A \twoheadrightarrow B) \qquad$
\end{itemize}
To reconnect formally with Yu's work,
we can investigate the following further:
\begin{itemize}
  \item A CompCertO transition system $L : A \twoheadrightarrow B$
    can be embedded into $\dagger A \multimap B$.
  \item The cliques of $\dagger \mathcal{C}$
    are universal abstract states.
  \item We can map between effect signatures $E$
    and the language interfaces $\mathcal{C}, \mathcal{A}$.
\end{itemize}
We can then define
a principled embedding
into game semantics or coherence spaces.

%A certified abstraction layer
%$L_2 \vdash_R M : L_1$
%establishes that:
%\[
%  L_1 \le_{\mathsf{id} \twoheadrightarrow R}
%  \llbracket M \rrbracket @K_2 \circ L_2
%\]
%Various properties of the operators involved
%ensure that
%for any context $C$,
%\[
%  \llbracket C \rrbracket@K_1 \circ L_1
%  \: \le_{\mathsf{id} \twoheadrightarrow R} \:
%  \llbracket C \rrbracket@K_2 \circ \llbracket M \rrbracket@K_2 \circ L_2
%  \: \le_{\mathsf{id} \twoheadrightarrow \mathsf{id}} \:
%  \llbracket C + M \rrbracket@K_2 \circ L_2
%  \,.
%\]
%In particular,
%this enables vertical composition:
%\[
%  L_1
%  \: \le_{\mathsf{id} \twoheadrightarrow R} \:
%  \llbracket M \rrbracket@K_2 \circ L_2
%  \: \le_{\mathsf{id} \twoheadrightarrow S} \:
%  \llbracket M + N \rrbracket@K_3 \circ L_3
%\]
%On the other hand,
%the flat composition operator:
%\[
%  L_1, L_2 : A \twoheadrightarrow B
%  \vdash
%  L_1 \uplus L_2 : A \twoheadrightarrow B
%\]
%allows us to carry out a similar

%}}}

\section{Basic setup} %{{{

I will use the notations and concepts outlined in
\href{https://certikos.github.io/rbgs-papers/thesis/thesis.pdf\#chapter.4}{Chapter 4}
of my thesis.
In particular,
\href{https://certikos.github.io/rbgs-papers/thesis/thesis.pdf#section.4.4}{\S 4.4}
reframes our CompCertX-based approach
into our more abstract formalism
and is a starting point for the following definitions.

\subsection{Abstract states} %{{{

In CompCertX,
the memory model is extended with an \emph{abstract state} component
which is used to specify the behavior of underlay primitives.

To extend CompCertO in a similar way,
given a set $K$ of abstract states,
we can introduce an operator to transform the language interface
$A = \langle A^\circ, A^\bullet \rangle$
into
\[
    A@K := \langle A^\circ \times K, \: A^\bullet \times K \rangle
    \,.
\]
Then,
a CompCertO transition system $L : A \twoheadrightarrow B$
can be lifted to \[ L@K : A@K \twoheadrightarrow B@K \,, \]
which maintains an abstract state component
and threads it through the computation.
If $L = \langle S, {\rightarrow}, D, I, X, Y, F \rangle$,
then we can define:
\[
    L@K \:\: := \:\:
      \langle S \times K, \:\: {\rightarrow} \times {=}_K, \:\:
            D \times K, \:\: I \times {=}_K, \:\: X \times {=}_K, \:\:
            Y_K, \:\: F \times {=}_K \rangle
\]
The state is extended with an abstract state component ($S \times K$).
Most of the relations involved in $L@K$
simply thread this component through unchanged (${-} \times {=}_K$).
At external calls,
we update it with the
corresponding component of the reply:
\[
  \begin{prooftree}
    \hypo{n \mathrel{Y^s} s'}
    \infer1{n@{k'} \mathrel{Y_K^{s@k}} s'@k'}
  \end{prooftree} 
\]

%}}}

\subsection{Layer interfaces} %{{{

Per the definitions in my thesis and our LICS'20 paper,
a layer interface can be described as a family of specifications:
\[
    \sigma^m : K \rightarrow \mathcal{P}^1(N \times K)
\]
where $(m \mathbin: N) \in E$ is an operation of the layer's signature.
In the case of the $\mathcal{C}$ language interface of CompCertO,
operations are of the form:
\[
    f(\vec{v}) : \mathsf{val}
    \qquad
    \text{where}
    \qquad
    f \in \mathsf{val}
    \qquad
    \vec{v} \in \mathsf{val}^*
\]

A layer interface specified in this style
can easily be represented as a CompCertO transition system
$\hat{\sigma} : 1 \twoheadrightarrow \mathcal{C}@K$,
defined as:
\[
  \hat{\sigma} := \langle
    \mathsf{val} \times \mathsf{mem} \times K,
    \varnothing,
    D,
    I,
    \varnothing,
    \varnothing,
    F
  \rangle
\]
A call into this transition system involves a single state.
At invocation,
we immediately query $\sigma$ to obtain the call's outcome
and save it in the transition system's state:
\[
  \begin{prooftree}
    \hypo{\sigma^{f(\vec{v})}(k) \ni (v', k')}
    \infer1{f(\vec{v})@m@k \mathrel{I} (v', m, k')}
  \end{prooftree}
\]
This single state admits no transition but is immediately final:
\[
  \begin{prooftree}
    \infer0{(v', m, k') \mathrel{F} v'@m@k'}
  \end{prooftree}
\]

The domain $D$ has to be specified in addition to $\sigma$,
which does not carry this information.
Alternatively,
we could use a more sophisticated embedding,
where $\sigma$ is defined in terms of a more abstract
effect signature $E$,
and where we specify a correspondance between
the operations of $E$ and $\mathcal{C}$ calls.

Using the definitions above
and the categorical composition $\circ$ which I define later,
the semantics of a Clight program $M$
on top of a layer interface
$L : 1 \twoheadrightarrow \mathcal{C}@K$
can be given as
\[
  \llbracket M \rrbracket @K \circ L \: : \:
  1 \twoheadrightarrow \mathcal{C}@K
  \,.
\]

%}}}

\subsection{Abstraction relations} %{{{

In CompCertX-based CertiKOS,
the abstraction relation between
an overlay with abstract states in $K_1$ and
an underlay with abstract states in $K_2$
is given as a pair of relations:
\[
  R^\mathsf{r} \subseteq K_1 \times K_2
  \qquad
  R^\mathsf{m} \subseteq K_1 \times \mathsf{mem}
\]
We also associate with each layer a set of global variables $G$
such that:
\[
  \begin{prooftree}
    \hypo{k_1 \mathrel{R^\mathsf{r}} m_2}
    \hypo{m_2 \cong_G m_2'}
    \infer2{k_1 \mathrel{R^\mathsf{r}} m_2'}
  \end{prooftree}
\]
where $\cong_G$ denotes the usual $\mathsf{Mem.unchanged\_on}$
relationship asserting that the two memories
associate the same contents to the global variables in $G$.

In CompCertO,
we can use these relations to define a
memory-extension-based simulation convention
$R : \mathcal{C}@K_1 \Leftrightarrow \mathcal{C}@K_2$
which captures CertiKOS-style abstraction between
overlay and underlay behaviors:
\begin{gather*}
 {\begin{prooftree}
    \hypo{k_1 \mathrel{R^\mathsf{r}} k_2}
    \hypo{k_1 \mathrel{R^\mathsf{m}} m_2}
    \hypo{m_1 \le_\mathsf{m} m_2}
    \hypo{m_1 \mathrel\text{no-perms-on} G}
    \hypo{\vec{v}_1 \le_\mathsf{v} \vec{v}_2}
    \infer5{f(\vec{v}_1)@m_1@k_1 \mathrel{R^\circ} f(\vec{v}_2)@m_2@k_2}
  \end{prooftree}}
\\[1em]
 {\begin{prooftree}
    \hypo{k_1 \mathrel{R^\mathsf{r}} k_2}
    \hypo{k_1 \mathrel{R^\mathsf{m}} m_2}
    \hypo{m_1 \le_\mathsf{m} m_2}
    \hypo{m_1 \mathrel\text{no-perms-on} G}
    \hypo{v'_1 \le_\mathsf{v} v'_2}
    \infer5{v'_1@m_1@k_1 \mathrel{R^\bullet} v'_2@m_2@k_2}
  \end{prooftree}}
\end{gather*}
Then we can
formulate the layer correctness property
$L \vdash_R M : L'$ as
\[
    L' \le_{\mathsf{id} \twoheadrightarrow R}
    \mathsf{Clight}(M)@K \circ L
    \,.
\]

%}}}

%}}}

\section{Categorical composition} %{{{

I have used the categorical composition operator
\[
  \begin{prooftree}
    \hypo{L_1 : B \twoheadrightarrow C}
    \hypo{L_2 : A \twoheadrightarrow B}
    \infer2{L_1 \circ L_2 : A \twoheadrightarrow C}
  \end{prooftree}
\]
without spelling out the details.
Categorical composition is both
more general and more restricted
than the horizontal composition operator $\oplus$:
\begin{itemize}
  \item The types of $L_1$ and $L_2$
    are more general than $A \twoheadrightarrow A$.
  \item 
    $L_1$ is allowed to call into $L_2$
    but not the other way around.
\end{itemize}

\subsection{Definition} %{{{

Suppose we have the transition systems:
\begin{align*}
  L_1 &= \langle S_1, {\rightarrow_1}, D_1, I_1, X_1, Y_1, F_1 \rangle
    : B \twoheadrightarrow C
  \\
  L_2 &= \langle S_2, {\rightarrow_2}, D_2, I_2, X_2, Y_2, F_2 \rangle
    : A \twoheadrightarrow B
\end{align*}
The composite transition system is defined as
\[
  L_1 \circ L_2 :=
  \langle S, {\rightarrow}, D_1 \cup D_2, I, X, Y, F \rangle
\]
with the following components.
States are of the form:
\[
    S := S_1 + (S_2 \times S_1)
\]
Initially, the environment question activates $L_1$:
\[
  \begin{prooftree}
    \hypo{q_C \mathrel{I_1} s_1}
    \infer1{q_C \mathrel{I} \iota_1(s_1)}
  \end{prooftree}
  \qquad
  \begin{prooftree}
    \hypo{s_1 \rightarrow_1 s_1'}
    \infer1{\iota_1(s_1) \rightarrow \iota_1(s_1')}
  \end{prooftree}
  \qquad
  \begin{prooftree}
    \hypo{s_1 \mathrel{F_1} r_C}
    \infer1{\iota_1(s_1) \mathrel{F} r_C}
  \end{prooftree}
\]
When an external call is encountered,
the question is used to activate $L_2$:
\[
  \begin{prooftree}
    \hypo{s_1 \mathrel{X_1} q_B}
    \hypo{q_B \mathrel{I_2} s_2}
    \infer2{\iota_1(s_1) \rightarrow \iota_2(s_2, s_1)}
  \end{prooftree}
\]
Execution proceeds according to $L_2$,
\[
  \begin{prooftree}
    \hypo{s_2 \rightarrow_2 s_2'}
    \infer1{\iota_2(s_2, s_1) \rightarrow \iota_2(s_2', s_1)}
  \end{prooftree}
  \qquad
  \begin{prooftree}
    \hypo{s_2 \mathrel{X_2} q_A}
    \infer1{\iota_2(s_2, s_1) \mathrel{X} q_A}
  \end{prooftree}
  \qquad
  \begin{prooftree}
    \hypo{r_A \mathrel{Y_2^{s_2}} s_2'}
    \infer1{r_A \mathrel{Y^{\iota_2(s_2, s_1)}} \iota_2(s_2', s_1)}
  \end{prooftree}
\]
until a final state of $L_2$ is reached,
at which point $L_1$ is resumed:
\[
  \begin{prooftree}
    \hypo{s_2 \mathrel{F_2} r_B}
    \hypo{r_B \mathrel{Y_1^{s_1}} s_1'}
    \infer2{\iota_2(s_2, s_1) \rightarrow \iota_1(s_1')}
  \end{prooftree}
\]

%}}}

\subsection{Identity} %{{{

The identity transition system $\mathsf{id}_A : A \twoheadrightarrow A$
can be defined as:
\[
  \mathsf{id}_A :=
    \langle A^\circ + A^\bullet,\: \varnothing,\: \varnothing,\: I,\: X,\: Y,\: F \rangle
\]
The components are defined by:
\[
  \begin{prooftree}
    \infer0{q \mathrel{I} \iota_1(q) \mathrel{X} q}
  \end{prooftree}
  \qquad
  \begin{prooftree}
    \infer0{r \mathrel{Y^{\iota_1(q)}} \iota_2(r) \mathrel{F} r}
  \end{prooftree}
\]
It should be straightforward to verify:
\[
  L
  \le_{\mathsf{id} \twoheadrightarrow \mathsf{id}}
  (\mathsf{id}_B \circ L)
  \le_{\mathsf{id} \twoheadrightarrow \mathsf{id}}
  L
  \qquad
  L
  \le_{\mathsf{id} \twoheadrightarrow \mathsf{id}}
  (L \circ \mathsf{id}_A)
  \le_{\mathsf{id} \twoheadrightarrow \mathsf{id}}
  L
\]

\begin{remark}[Domains and categorical composition]
  Unlike the horizontal composition operator $\oplus$
  and the flat composition operator $\uplus$ introduced in the next section,
  categorical composition does not make use of the component's domains
  to compute the behavior of the composite transition system.
  This means that in general,
  a component may exhibit meaningful behaviors on queries outside its domain.
  This is the case in particular for $\mathsf{id}_A$,
  where the domain is $\varnothing$ but
  every possible query is associated with a behavior.
  In turn it suggests a modification to CompCertO's
  language semantics which would make them act as ``passthough''
  on queries outside of a module's domain.

  I should also note that categorical composition as
  given in this section will require a modifying
  the definition of CompCertO's transition systems slightly.
  As it stands,
  a component's domain $D$ is a set of queries of
  the incoming language interface.
  However,
  for the domain $D_1 \cup D_2$ used in $L_1 \circ L_2$
  to make sense when $B \neq C$,
  we will have to change it to a language-independent
  notion of domain, for example a set of identifiers.
  Language interfaces
  would then provide a way to recognize whether queries
  are part of a domain expressed in this general way.
\end{remark}

%}}}

\subsection{Associativity} %{{{

Associativity can be verified using
the simulation relation:
\[
  \begin{array}{rr}
    \hline
    L_1 \circ (L_2 \circ L_3) & (L_1 \circ L_2) \circ L_3 \\
    \hline
    \iota_1(s_1) & \iota_1(\iota_1(s_1)) \\
    \iota_2(\iota_1(s_2), s_1) & \iota_1(\iota_2(s_2, s_1)) \\
    \iota_2(\iota_2(s_3, s_2), s_1) & \iota_2(s_3, \iota_2(s_2, s_1)) \\
    \hline
  \end{array}
\]

%}}}

\subsection{Simulations} %{{{

Categorical composition is compatible with simulations in the following sense.
Given
$L_1 : B \twoheadrightarrow C$ and $L_2 : A \twoheadrightarrow B$,
which are simulated respectively by
$L_1' : B' \twoheadrightarrow C'$ and $L_2' : A' \twoheadrightarrow B'$,
the following property holds:
\[
  \begin{prooftree}
    \hypo{L_1 \le_{S \twoheadrightarrow T} L_1'}
    \hypo{L_2 \le_{R \twoheadrightarrow S} L_2'}
    \infer2{L_1 \circ L_2 \le_{R \twoheadrightarrow T} L_1' \circ L_2'}
  \end{prooftree}
\]
Diagrammatically,
this allows us to paste simulations squares horizontally:
\[
  \begin{tikzcd}
    A \ar[r, ->>, "L_1"] \ar[d, <->, "R"] &
    B \ar[r, ->>, "L_2"] \ar[d, <->, "S"] &
    C \ar[d, <->, "T"] \\
    A' \ar[r, ->>, "L_1'"'] &
    B' \ar[r, ->>, "L_2'"'] &
    C'
  \end{tikzcd}
  \qquad \Longrightarrow \qquad
  \begin{tikzcd}
    A \ar[rr, ->>, "L_1 \circ L_2"] \ar[d, <->, "R"] &&
    C \ar[d, <->, "T"] \\
    A' \ar[rr, ->>, "L_1' \circ L_2'"'] &&
    C'
  \end{tikzcd}
\]

Note that the vertical composition of simulation squares
corresponds to the usual composition of simulations already given
in the CompCertO paper:
\[
  \begin{prooftree}
    \hypo{L_1 \le_{R \twoheadrightarrow S} L_2}
    \hypo{L_2 \le_{T \twoheadrightarrow U} L_3}
    \infer2{L_1 \le_{R \cdot T \twoheadrightarrow S \cdot U} L_3}
  \end{prooftree}
  \hspace{3em}
  \begin{tikzcd}
    A_1 \ar[r, ->>, "L_1"] \ar[d, <->, "R"] & B_1 \ar[d, <->, "S"] \\
    A_2 \ar[r, ->>, "L_2"] \ar[d, <->, "T"] & B_2 \ar[d, <->, "U"] \\
    A_3 \ar[r, ->>, "L_3"] & B_3
  \end{tikzcd}
  \quad
  \begin{tikzcd}[row sep=large]
    A_1 \ar[r, ->>, "L_1"] \ar[dd, <->, "R \cdot T"] & B_1 \ar[dd, <->, "S \cdot U"] \\ \\
    A_3 \ar[r, ->>, "L_3"] & B_3
  \end{tikzcd}
\]

One last observation is that the identity transition system
allows us to formulate the refinement of simulation conventions itself
as a simulation square.
There is a nice symmetry
with the refinement of transition systems:
\[
  \begin{tikzcd}[sep=tiny]
    A \ar[dd, double, dash, "\mathsf{id}_A"'] \ar[rr, ->>, "L_1"] &&
    B \ar[dd, double, dash, "\mathsf{id}_B"] \\
    & L_1 \le L_2 & \\
    A \ar[rr, ->>, "L_2"'] &&
    B
  \end{tikzcd}
  \hspace{5em}
  \begin{tikzcd}[sep=tiny]
    A \ar[dd, <->, "S"'] \ar[rr, double, dash, "\mathsf{id}_A"] &&
    A \ar[dd, <->, "R"] \\
    & R \sqsupseteq S & \\
    B \ar[rr, double, dash, "\mathsf{id}_B"'] &&
    B
  \end{tikzcd}
\]

%}}}

\subsection{Relationship with $\oplus$} %{{{

In the case $A = B = C$,
the categorical composition $L_1 \circ L_2$
is an under-approximation of
the horizontal composition $L_1 \oplus L_2$:
\[
    L_1 \circ L_2
    \le_{\mathsf{id} \twoheadrightarrow \mathsf{id}}
    L_1 \oplus L_2
\]

%}}}

%To obtain a proper categorical structure,
%we will consider $L$ to be ``pass-through''
%outside of its domain $D$.
%In other words,
%composition will work as follows:
%\[
%  \begin{tikzpicture}[xscale=2]
%    \node[draw,rectangle] (L1) at (2, 2) {$L_1$};
%    \node[draw,rectangle] (L2) at (1, 1) {$L_2$};
%    \draw[->] (3,2) -- node[above, near start] {$D_1$} (L1);
%    \draw[->] (3,1) -- node[above, near start] {$D_2$} (L2);
%    \draw[->] (3,0) -- node[above] {$\overline{D_1 \cup D_2}$} (0,0);
%    \draw[->] (L1) -- node[above] {$\overline{D_2}$} (0,2);
%    \draw[->] (L1) .. controls +(-0.5,0) and +(0.5,0) .. (L2);
%    \draw[->] (L2) -- (0,1);
%  \end{tikzpicture}
%\]

%}}}

\section{Flat composition}

\section{Coherence spaces}

\section{Effect signatures}

\end{document}
