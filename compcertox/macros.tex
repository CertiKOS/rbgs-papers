% Some of the macros I defined trip up latexdiff,
% so I separate them in this file.
% vim: foldmethod=marker

% Notations
\newcommand{\kw}[1]{\ensuremath{ \mathsf{#1} }}
\newcommand{\ifr}[1]{\mathrel{[{#1}]}}
\newcommand{\que}{\circ}
\newcommand{\ans}{\bullet}
\newcommand{\vref}{\le_\kw{v}}
\newcommand{\mext}{\le_\kw{m}}
\newcommand{\refby}{\preceq}
\newcommand{\scref}{\sqsupseteq}
\newcommand{\screfd}{\sqsubseteq}
\newcommand{\unitset}{\mathds{1}}

% Multi-letter language interfaces
\newcommand{\li}[1]{\mathit{#1}}
% Calling conventions (language interface boundaries)
\newcommand{\cc}[2]{{ \kw{#1#2} }}

% Pointers for justified sequences %{{{

% Parameters
\newcommand{\pshift}{1.6ex}
\newcommand{\pcdist}{1}
\newcommand{\pcangle}{60}

% Pointer hook
\newcommand{\ph}[1]{%
  \tikz[remember picture]{\coordinate (#1);}}

% Pointer to
\newcommand{\ptc}[2]{%
  \tikz[remember picture,baseline,>={Latex[round,length=3.6pt]}]{
    \draw[->,#2]
      let \p{dest} = (#1),
          \n1 = {pow(veclen(\x{dest}, \y{dest}), 0.5) * 1.5},
          \p1 = ($(0,0)+(0,\pshift)$),
          \p4 = ($(\x{dest},0)+(0,\pshift)$),
          \p2 = ($(\p1)!\n1*\pcdist!-\pcangle:(\p4)$),
          \p3 = ($(\p4)!\n1*\pcdist!+\pcangle:(\p1)$) in
        (\p1) .. controls (\p2) and (\p3) .. node[pos=0.5] (top) {} (\p4);
    \pgfresetboundingbox
    \path[use as bounding box] (0,0 |- top);
}}
\newcommand{\pt}[1]{%
  \ptc{#1}{gray}}
\newcommand{\bpt}[1]{%
  \ptc{#1}{black,thick,>={Latex[round,length=4pt]}}}

% TikZ setup
\pgfdeclarelayer{tint}
\pgfdeclarelayer{nodes}
\pgfdeclarelayer{background}
\pgfsetlayers{tint,background,main,nodes}
\selectcolormodel{cmyk}

% Parameters for diagrams
\newcommand{\stens}{0.6}

% The intensity of colors in figures and row highlighting respectively.
% These should be the same, otherwise they are just confusing to look at
% side by side, especially on a printout.
\newcommand{\filltint}{!35}
\newcommand{\tbltint}{\filltint}

% Colors used in the World transitions section
\newcommand{\colorA}{ACMDarkBlue}
\newcommand{\colorB}{ACMDarkBlue}
\newcommand{\internalA}[1]{\textcolor{\colorA}{#1}}
\newcommand{\internalB}[1]{\textcolor{\colorB}{#1}}

% Refinement tiles {{{

%       -1.5    -1    0    +1   +1.5
%  +1.5 (TLC)--(LT)--(T)--(RT)--(TRC) +1.5
%        |                         |
%  +1   (TL)   (TLI) (TI) (TRI)  (TR) +1
%        |                         |
%   0   (L)    (LI)  (C)  (RI)    (R)  0
%        |                         |
%  -1   (BL)   (BLI) (BI) (BRI)  (BR) -1
%        |                         |
%  -1.5 (BLC)--(LB)--(B)--(RB)--(BRC) -1.5
%       -1.5    -1    0    +1   +1.5

\newenvironment{tile}[1]{%
  \begin{tikzpicture}[baseline,yscale=0.36,xscale=0.5]
    \tikzset{to path={
      .. controls ($(\tikztostart)!\stens!(\tikztostart -| \tikztotarget)$)
              and ($(\tikztotarget)!\stens!(\tikztotarget -| \tikztostart)$) ..
      (\tikztotarget) \tikztonodes}}
    \tikzset{#1}
    % Coordinates for things on the left
    \coordinate (TL) at (-1.5,1);
    \coordinate (L) at (-1.5,0);
    \coordinate (BL) at (-1.5,-1);
    \coordinate (TLI) at (-1,1);
    \coordinate (BLI) at (-1,-1);
    % Coordinates for things on the right
    \coordinate (TR) at (1.5,1);
    \coordinate (R) at (1.5,0);
    \coordinate (BR) at (1.5,-1);
    \coordinate (TRI) at (1,1);
    \coordinate (BRI) at (1,-1);
    % Coordinates for things on top
    \coordinate (LT) at (-1,+1.5);
    \coordinate (T) at (0,+1.5);
    \coordinate (RT) at (+1,+1.5);
    \coordinate (TI) at (0,+1);
    % Coordinates for things at the bottom
    \coordinate (LB) at (-1,-1.5);
    \coordinate (B) at (0,-1.5);
    \coordinate (RB) at (+1,-1.5);
    \coordinate (BI) at (0,-1);
    % Center node, for crossing
    \node[circle,inner sep=2pt] (C) at (0,0) {};
    % Computed coordinates
    \coordinate (TLC) at ($(T-|L)$);
    \coordinate (BLC) at ($(B-|L)$);
    \coordinate (TRC) at ($(T-|R)$);
    \coordinate (BRC) at ($(B-|R)$);
}{%
  \end{tikzpicture}%
}
\newcommand{\simproof}[2]{%
  \begin{pgfonlayer}{nodes}
    \node[draw,rectangle,fill=white,rounded corners=2pt,minimum height=0.5cm,minimum width=0.8cm] at #1 {#2};
  \end{pgfonlayer}
}
\newcommand{\drawst}{%
  \draw[thick,rounded corners=1mm]
}
\newcommand{\fillleft}[1]{%
  \begin{pgfonlayer}{tint}
    \fill[#1] (TLC) rectangle (B);
  \end{pgfonlayer}
}
\newcommand{\fillright}[1]{%
  \begin{pgfonlayer}{tint}
    \fill[#1] (T) rectangle (BRC);
  \end{pgfonlayer}
}
\newcommand{\filltop}[1]{%
  \begin{pgfonlayer}{tint}
    \fill[#1] (TLC) rectangle (R);
  \end{pgfonlayer}
}
\newcommand{\fillbot}[1]{%
  \begin{pgfonlayer}{tint}
    \fill[#1] (L) rectangle (BRC);
  \end{pgfonlayer}
}
\newcommand{\fillboth}[1]}}

