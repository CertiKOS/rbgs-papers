\section{Broader Impacts}
\label{sec:impact}

\paragraph*{Technology Transfer}
The proposed project aims at developing a compositional verified
compiler toolchain for C and related languages and creating a {\em
  certified} application binary interfaces (ABIs) for future
trustworthy heterogeneous systems. Such certified toolchain and ABIs
will greatly facilitate the verification of large-scale system
software, which in turn have a profound impact on the software
industry and the society in general.
Artifacts resulting from the project will be
made open-source to ensure rapid dissemination of ideas.

We plan to aggressively pursue technology transfer through several
different channels. First, PI Shao is the co-founder of CertiK,
a startup focusing on formal verification and auditing of smart
contracts and blockchain ecosystems. To date,
CertiK has provided services to over 4,000 clients and detected
over 70,000 vulnerabilities in blockchain code. It has also
successfully raised \$200 million of VC funding. CertiK has retargeted
DeepSEA to build verified smart contracts and developed a new
verified compiler backend for generating EVM code. We believe
that our new compositional verified compilation technology
can be readily applied to greatly enhance the power of this version
of DeepSEA as well.

Second, together with his Columbia colleagues, PI Shao is working on
a DARPA V-SPELLS project (called REFUEL)
which aims to develop verified composition and flattening of secure
enclaves for cyber-physical systems with large legacy code base.
REFUEL will deploy SeKVM and CertiKOS on the self-driving car
and UAV platforms to build VM-based and ARM-TrustZone-based secure
enclaves. The compositional compilation and verification technologies
developed in this proposal will be tested and transferred into realistic
case studies under the DARPA project.

\vspace*{-2ex}
\paragraph*{Curriculum Development}
We plan to develop innovative certified system software and formal
verification curriculum at both the undergraduate and graduate
level. The two PIs plan to develop a new course on Formal Semantics
(CPSC 430) at Yale based on the Nominal CompCertO developed in this
project. Existing courses and textbooks on formal semantics focus on
somewhat outdated material and do not cover exciting new technologies
such as game semantics, nominal techniques, and verified
compilation. By using our verified compiler toolchain and common
composition patterns in CertiKOS as the basis, we plan to
create a set of programming labs (including specifying, programming,
compiling, verifying, and linking certified abstraction layers) where
students can learn modern compositional techniques through a more
hands-on approach.

PI Shao also plans to revamp his course on Language-Based Security
(CPSC 428) based on the verified OS kernel (CertiKOS) and hypervisor
(SeKVM) and compiler toolchain (Nominal CompCertO) and certified
programming tools (DeepSEA and VST) where students can gain hands-on
experience on both system design and verification.  Language-Based
Security aims to completely eliminate specific classes of
vulnerabilities introduced by the semantics of OS
kernels, virtual machines, and programming languages. Language-based
technologies, including program verification, formal methods, static
analyses, interpreters, and compilers are the key
components of next generation security systems. The newly revamped
course will survey the most promising compositional
techniques and show how to apply them to build
dependable system software.  We will
make the courses accessible to undergraduate students so they
can participate in the related
research. We will make our course materials freely available and
encourage our colleagues to use them at other universities.

\vspace*{-2ex}
\paragraph*{BPC Plan and Outreach}
The supplemental BPC document contain a detailed plan on how the PIs
will engage in the wide range of BPC activities organized by the
Computer Science department at Yale. In addition, the PIs will
continue to recruit women and people from URGs (Under Represented
groups) to research.  PI Shao has a strong track record of mentoring
women and URGs in the past few years: he served as the PostDoc advisor
(and is still serving as a mentor) for the 2021 Computing Innovation
(CI) Fellow Anitha Gollamudi; he was the PostDoc advisor for
Dr. Jung-Eun Kim during 2017-2021, an MIT EECS Rising Star and now
Assistant Professor of Computer Science at NCSU; he was also the
research advisor for Valerie Chen during 2018-2020, a CRA
Undergraduate Research Award Finalist and now a PhD student at CMU.
If this project is funded, the PIs will seek undergraduate students –
especially women and underrepresented minorities – through programs
like REU to join the work on building end-to-end verified compilation
toolchain.


