

\section{Broader Impact}

Our proposed research will contribute greatly to build a safe and
bug-free distributed software ecosystem. Especially, the parameterized
verification framework and atomic-distributed-object-based composite
reasoning of distributed systems capture the core properties of
distributed systems. Because most seemingly different distributed
systems are developed and optimized while preserving these core
properties, our verification approach will be able to host a wide
range of distributed systems and tolerate frequent system updates to
easily verify the safety properties. In addition, our verification
framework will guarantee the safety of large-scale distributed systems
that were considered very challenging. We expect that our research
will considerably lighten the burden of distributed system testing and
debugging by making distributed system verification greatly
approachable and applicable.

Furthermore, our research can be easily extended to different domains,
such as cyber physical systems and internet of things, where hundreds
of devices realize a form of a distributed system. For example,
mission critical health monitoring wireless sensor nodes and
coordination of self-driving cars and avionics would be able to
directly benefit from our research. As computing devices are becoming
more and more common, our research will be able to verify the
correctness of their interactions and have a greater impact.

We plan to develop innovative distributed system software and formal
verification curriculum at both the undergraduate and graduate
level. PI Shao recently started to teach a course on language-based
security at Yale and the term projects are designed based on initial
findings of our proposal.  Extending the term project, co-PI Shin is
planning to develop a distributed system verification course, where
students can gain hands-on experience on both system design and
verification. We plan to extend the courses to undergraduate and
Master's research program where students can participate in the
proposed research. We will make our course materials freely available
and encourage our colleagues to use them at other universities.

We are planning to further use our research materials to foster
diversity and inclusion. Co-PI Soul\'e has been specially focusing
on supporting Latin American and Hispanic students for few years with
scholarship programs, student visit programs, and diversity seminars. 
The PIs plan to continue these efforts jointly and to recruit 
underrepresented students to get involved in our research.

Several composite distributed system examples that we plan to model
and verify are motivated by experiences in the commercial sector, 
and therefore has a clear promise for industrial impact. 
For example, co-PI Soul\'{e} has worked closely with Western Digital 
on a project that uses network hardware to build a new breed of 
distributed shared memory system. Successful modeling and
verification of the system can directly influence the safety and 
correctness of the production prototype. 

The design, implementation, and validation of a software artifact 
is one of our great contributions to our proposal. 
We plan to release the software developed in this project under 
a flexible, open-source license.

