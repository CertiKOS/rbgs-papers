\section{Compositional Verified Compilation into ELF Object Files}

Together with his students, PI Shao has recently developed
CompCertELF~\cite{compcertelf20}, the first extension to CompCert that
supports verified compilation from C programs all the way to a
standard binary file format, i.e., the ELF object format.  However,
CompCertELF only supports {\em{}verified separate
compilation}~\cite{sepcompcert} which cannot handle compositional linking of
heterogeneous components.  For this proposal, we will integrate the
technologies in CompCertELF into Nominal CompCertO to build an {\em
  end-to-end and compositional verified compiler} that can compile C
components all the way into ELF object files.

\paragraph*{Task 3a: Stack-Aware Norminal CompCertO.}
To support concrete memory layout required by the binary machine code,
we will first extend Nominal CompCertO to support compilation with a
single, continuous, and finite stack by incorporating the key ideas in
Stack-Aware CompCert~\cite{wang2019,compcertelf20}.  Stack-Aware
CompCert explicitly manages the call stack by adding a data type
called \emph{abstract stack} to memory states. The abstract stack
records the history of memory consumption incurred by stack allocation
and maintains fine-grained \emph{stack permissions}. By exploiting
that information, Stack-Aware CompCert achieves contextual compilation
of single-threaded C programs into an assembly language that is aware
of a single and finite stack.  In our structured nominal memory model,
the abstract stack can be readily absorbed into the support. We can
drop stack permissions from the abstract stack in Stack-Aware CompCert
because CompCertO already separately enforces them as part of its
reasoning framework (e.g., simulation conventions).  Furthermore, by
enriching supports with multiple abstract stacks following the idea of
Task 2c, we will build Multi-Stack Nominal CompCertO which can compile
multi-threaded programs onto multi-stack machine models. These ideas
will form a complete solution to thread-safe contextual compilation
needed to support CCAL~\cite{ccal18}.

\paragraph*{Task 3b: Verified Compilation into Relocatable ELF Files}
CompCertELF is built on top of Stack-Aware CompCert. In essence, it
modifies and extends Stack-Aware CompCert's compilation chain with a
\emph{verified assembler} that transforms the realistic assembly
programs output by Stack-Aware CompCert to sequences of bytes that
represent relocatable ELF files.  This assembler consists of several
passes that successively merge code and data into atomic blocks (known
as sections), generate information for relocation and linking (such as
symbol tables and relocation tables), encode the assembly instructions
and the global data into their binary forms, and finally embed them
into the ELF format with meta-data (such as section headers and ELF
headers). A big part of this task is to leverage the rich structure
from the nominal memory model to enable richer ABIs.  Another
challenge is to develop automation support building certified
instruction encoders and decoders for different machine
architectures~\cite{xu21}.

\paragraph*{Task 3c: Certified Compositional ELF Linker and Loader}
To support end-to-end certified linking, CompCertELF has to prove
compilation commutes with two different linkers: one used in CompCert
and the other for relocatable object files. On one hand, in CompCert,
the linker views programs as partial mappings from identifiers to
global definitions. Therefore, it is insensitive to and may
arbitrarily rearrange the order of definitions. On the other hand, to
generate relocatable object files, the function and variable
definitions are merged into atomic sections following some particular
order. The ELF linker simply concatenates sections of the same type
together like they are ``black boxes.''  Here, the order in which
definitions are merged and sections are concatenated significantly
affects the result of linking.  The consequence is that commutativity
between linking and compilation breaks at the transitional phase from
CompCert to the assembler where the latter links programs as
relocatable objects. We believe that the language interfaces in
CompCertO and the richer permutation support in our nominal memory
model can provide a better and cleaner solution toward a certified
linker and loader.

