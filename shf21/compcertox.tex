\section{Secure Compilation}
\label{sec:compcertox}

In general,
even a correct compiler may not always preserve
information flow security properties.
This creates a challenge when such properties
are proved at the source level
but must be established for the target program.

The culmination of our proposed research will be to incorporate
information flow security and secure compilation
into our open semantics framework.
While there are still many fundamental questions to answer
about the ways in which interaction and compositionality
interact with secure compilation techniques,
we outline our basic approach below.

\subsection{Full Abstraction}

A fully abstract compiler preserves and reflects program equivalence:
\begin{equation} \label{eqn:fac}
  \llbracket p_1 \rrbracket_\kw{S} \equiv
  \llbracket p_2 \rrbracket_\kw{S}
  \quad \Leftrightarrow \quad
  \llbracket \kw{C}(p_1) \rrbracket_\kw{T} \equiv
  \llbracket \kw{C}(p_2) \rrbracket_\kw{T}
\end{equation}
Full abstraction is particularly relevant
in the context of secure compilation,
since it can be used to propagate properties such as non-interference
from the source code to the target level.
However,
most C compilers do \emph{not} provide such a guarantee,
because assembly programs may differ in the details
of their use of registers, memory layout, etc.
while still being faithful realizations of a single C-level behavior.

One way to side-step this issue
is to characterize a more limited version of (\ref{eqn:fac})
which actually \emph{does} hold for the compiler.
For example,
the compiler may not preserve source-level equivalences
when we are able to observe the details of the assembly execution,
but the equivalence may still hold when
we observe the behavior of target programs
in terms of the \emph{source}-level behavior that they realize,
ignoring for example the value of caller-save registers
returned by the program.
Such a property may be sufficient for some applications.

Secondly,
it may be possible to add a wrapper to the target code
to normalize assembly behaviors
so that a given C behavior always yields
the same assembly behavior in the target program.
Such a wrapper would for example make sure
to erase the value of caller-save registers
before the target code returns control to the environment.

\subsection{Security-preserving Simulations}

Finally,
full abstraction is sufficient but not necessary
for preserving information-flow security.
Prior work conducted in PI Shao's group
introduced a notion of \emph{security-preserving} simulation
\cite{costanzo16}.
There,
security policies are formulated in terms
of sets of \emph{observations} 
which each principal can make about the system.
Transition systems are equipped with observation functions of type
$\mathcal{O} : S \rightarrow \Omega$,
where $S$ is the set of states of the transition system and
$\Omega$ is a set of observations.
Two states $s_1, s_2 \in S$
are said to be \emph{indistinguishable}
when they yield the same observation:
\[
  s_1 \simeq s_2 \::\Leftrightarrow\: \mathcal{O}(s_1) = \mathcal{O}(s_2)
\]
In this context,
a security-preserving simulation
is a simulation where
the simulation relation
preserves $\simeq$.

\subsection{Proposed Research}

We will investigate how these different approaches to secure compilation
can be incorporated into our open semantics framework.

\vspace*{-2ex}
\paragraph*{Task 4a: Observations for Language Interfaces}

[Here the problem is to extend the notion of security-preserving simulation
to open semantics. We must define observations for language interfaces
$\mathcal{O} : A \rightrightarrows \Omega$
and can induce a simulation convention $\mathcal{O} \mathbin; \mathcal{O}^{-1}$
which expresses indistinguisability.
Then self-simulation shows that a transition system
obeys the security policy expressed by $\mathcal{O}$.]

\vspace*{-2ex}
\paragraph*{Task 4b: Equivalence \emph{Modulo} Calling Conventions}




The completion of Task~1c, described in \S\ref{sec:compcerto},
will provide us with way to incorporate C and assembly components
into the framework of \emph{refinement-based game semantics} \cite{koenig20},
a fairly general game semantics
equipped with the structure of a category
and with both angelic and demonic choice operators.
Specifically,
a CompCertO component $L : A \twoheadrightarrow B$
will be interpreted as a \emph{strategy specification} of type
$\llbracket L \rrbracket : A \rightarrow B$.

In addition, dual nondeterminism provides strong support for abstraction.
In particular,
a simulation convention $\mathbb{R} : A \Leftrightarrow B$
can be represented as a pair of adjoint morphisms
$R^* \dashv R_* : B \rightarrow A$
such that:
\[
  \mathrm{id}_A \le R_* \circ R^*
  \qquad \text{and} \qquad
  R^* \circ R_* \le \mathrm{id}_B
  \,.
\]
The components $R^*$ and $R_*$
derived from $\mathbb{R}$,
can be used
to \emph{translate} between the source and target language interfaces.
A simulation $L_1 \le_{\mathbb{R} \twoheadrightarrow \mathbb{S}} L_2$
establishes the following equivalent refinement properties:
\[
  L_1 \le R_* \circ L_2 \circ S^* \quad \Leftrightarrow \quad
  L_1 \circ S_* \le R_* \circ L_2 \quad \Leftrightarrow \quad
  R^* \circ L_1 \le L_2 \circ S^* \quad \Leftrightarrow \quad
  R^* \circ L_2 \circ S_* \le L_2
\]
In other words,
whereas the simulations used in CompCertO
combine \emph{abstraction} and \emph{refinement}
into a single relation,
they can now be disentangled:
since adjunctions allow us to translate between abstraction levels,
refinement can once again take the form of a simple partial order.

Moreover,
it is now possible to express that two assembly programs
have equivalent behaviors
\emph{from the point of view of the C calling convention}.
Given a simulation convention $\mathbb{R}$,
we can use the closure operator defined by the associated adjunction
to compare two target behaviors:
\[
  L_1 \simeq_\mathbb{R} L_2 \: \Leftrightarrow \:
  R^* \circ R_* \circ L_1 \circ R^* \circ R_* =
  R^* \circ R_* \circ L_2 \circ R^* \circ R_*
\]
We will explore these connections and attempt to characterize
the security properties of CompCert,
using formulations such as:
\[
  \llbracket p_1 \rrbracket_\kw{S} \equiv
  \llbracket p_2 \rrbracket_\kw{S}
  \quad \Leftrightarrow \quad
  \llbracket \kw{C}(p_1) \rrbracket_\kw{T} \simeq_\mathbb{C}
  \llbracket \kw{C}(p_2) \rrbracket_\kw{T}
\]

\vspace*{-2ex}
\paragraph*{Task 4c: Normalization of Assembly Behaviors}

Give two wrappers $W^+, W^- : \mathcal{A} \twoheadrightarrow \mathcal{A}$
such that
\begin{enumerate}
  \item the C-level behaviors of $L$ and $W^+ \odot L \odot W^-$
    are equivalent;
  \item when the C-level behaviors of $L_1, L_2$ are equivalent,
    then the assembly behaviors
    $W^+ \odot L_1 \odot W^-$ and
    $W^+ \odot L_2 \odot W^-$
    are equivalent as well.
\end{enumerate}
Then by implementing this wrapper
we can obtain a fully abstract compiler.
