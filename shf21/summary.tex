\noindent{\bf Project Summary}

{\bf Overview:} Distributed systems are difficult to verify due to
their inherent complexity, which stems from handling concurrency and
network asynchrony.  Significant advances have been made in formally
reasoning and verifying distributed systems, but most existing
approaches focus on reasoning about specific instances of distributed
systems and do little to explore how to expose the common high-level
behaviors of distributed systems while hiding the implementation
details.  Therefore, verifying individual distributed systems requires
redundant reasoning and the absence of a high-level model makes it
difficult to address the new challenges that modern applications are
often composed of multiple distributed systems.  Most current
verification tools and approaches are not well-equipped to handle
multiple distributed system interactions and the PIs aim to address
this problem by proposing a high-level uniform model of a distributed
system that facilitates reasoning and verification of both individual
and composition of distributed systems.

{\bf Keywords:} compositional compiler correctness; verified C compiler;
formal specification and verification; nominal techniques; 
secure compilation

{\bf Intellectual Merit:} In this effort, the PIs propose to design a
novel compositional atomic distributed object model that can be used
to reduce the complexity of distributed system reasoning and
verification. The atomic distributed object model is defined using
modified push/pull operations from work on shared-memory concurrency
and maintains a logical history of state change requests. The model
hides details such as interleaved network messages and failures that
complicates the reasoning about the core safety properties of a
distributed system. Still, the model provides enough detail to reason
about a system at a much higher abstraction level.

The atomic distributed object model encapsulates the key safety
properties of individual distributed systems and the PIs plan to build
multiple network-based specification that captures the common
network-level behavior of similar classes of distributed systems.  The
network-based specification helps individual systems to verify their
refinement relation to the atomic distributed object model, provides
reusable proofs that are derived from common system behaviors, and
acts as a verification template that verifies the safety properties
encapsulated in the atomic object model for free.

Once individual distributed systems are verified to be correct and
safe based on the atomic distributed object model, the high-level
abstraction of the model can be used to reason about multiple
distributed system interactions. Verification of multiple distributed
system interactions is challenging to do with existing approaches,
which lack a common high-level model of individual systems.  The
atomic distributed object model can play a key role to stitch multiple
verified distributed systems together into a larger distributed object
without having to deal with the low level details of individual
systems.

The PIs will build in Coq a distributed system verification
framework, \sysname{}, based around the atomic distributed object
model to verify individual distributed systems and their interactions.
The PIs will demonstrate through concrete examples that proving
properties even of composite distributed systems can be
straightforward with the atomic distributed object model due to the
elegantly simple object interface.  The PIs plan to verify real-world
cutting-edge distributed systems written in C.  One of the target
systems is a distributed shared memory that uses a programmable
switch, a low-latency network, and multiple sharded distributed
components that run consensus protocols.


{\bf Broader Impact:} The technology for simplifying distributed
system reasoning and allowing for verification of distributed system
composition will have a profound impact on the software industry and
the society in general. It will dramatically improve the reliability
and security of large-scale software infrastructures, such as the
cloud, and applications that run on top of the infrastructure.  The
atomic distributed object specification and the verification framework
will make distributed infrastructures easier to understand and verify.
The applicability of the research outcome can be easily extended to
relevant fields such as cyber physical systems or internet of things
that use multiple sensors and devices over the network.  On the
educational side, this project will push new courses on distributed
system design and verification and will broaden the participation of
underrepresented groups.

