\subsubsection*{Overview}

In the last 50 years, the C programming language and its associated
toolchain (e.g., compiler, assembler, linker, loader) have played a
dominant role in the development of today's software and hardware
systems.  Over the past decade, researchers have been able to formally
verify various key components of these systems, including compilers,
OS kernels, file systems, and processor designs. Building on these
successes, the research community is attempting to construct
large-scale certified heterogeneous systems by using formal {\em deep}
specifications as interfaces between the correctness proofs of various
components.

The formally verified C compiler, CompCert, is a major breakthrough
that holds the promise to become the bedrock of future {\em certified}
heterogeneous system stack. In recent years, researchers have been
refining the CompCert language semantics and correctness theorem, and
used them in various software verification efforts.  Unfortunately,
CompCert still suffers from several major limitations: it does
not support compositional verified compilation and linking with
heterogeneous components; its rigid memory model is incompatible with
concurrency; it does not generate binary machine code; and it does not
support secure compilation in that the verified compiler could still
introduce information leaks during compilation.

In this effort, the PIs propose to develop a novel verified
compilation toolchain that addresses all of these shortcomings. In
doing so, the project will explore, refine, and discover new semantic
models and formal frameworks for supporting compositional
specification, abstraction, refinement of heterogeneous systems.

\vspace{+2mm}
\noindent{\bf Keywords}:~~{Compositional Compiler Correctness; Verified C Compiler; Compositional Semantics; Formal Specification and Verification; Nominal Techniques; Secure Compilation.}

\subsubsection*{Intellectual Merit}

The project will make five related scientific
contributions. First, it will contribute new technologies for supporting
{\em compositional verified compilation} by using (1) novel
game-semantics-inspired language interfaces to specify sophisticated
calling conventions, and then (2) richer simulation conventions
% (based on logical relations)
to support compositional proofs.  Second, it
will develop a novel {\em nominal memory model}---an enhancement to
CompCert's block-based memory model with nominal techniques---to
remove the global constraints for managing memory blocks, and enable
flexible memory structures for open and concurrent programs. Third, it
will develop an {\em end-to-end and compositional verified compiler}
that can compile C components all the way into ELF object files; and
build verified compositional linker and loader that can work directly
with ELF binaries.  Fourth, it will support {\em verified
security-preserving compilation} by developing a general framework for
specification, abstraction, and refinement, and connecting certified
abstraction layers with verified compilation. Fifth, it will evaluate the
effectiveness of the new verified compiler by applying it to build
more advanced certified OS kernels (e.g., CertiKOS) and certified
programming tool (e.g., DeepSEA).

\subsubsection*{Broader Impacts}

The proposed project aims at developing a compositional verified
compiler toolchain for C and related languages and creating {\em
certified} application binary interfaces (ABIs) for future trustworthy
heterogeneous systems. The new technologies for compositional
specification and refinement will greatly facilitate the verification
of large-scale system software, which in turn will have a profound impact
on the software industry and the society in general. The applicability
of the research outcome can be easily extended to relevant fields such
as operating systems, blockchain and smart contracts, and
cyber-physical systems.  On the educational side, this project will
push new courses on formal semantics, compilers and interpreters, and
language-based security, and will broaden the participation of
underrepresented groups.  Artifacts resulting from the project will be
made open-source to ensure rapid dissemination of ideas.

