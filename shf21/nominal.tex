\section{Verified Compilation with a Nominal Memory Model}
\label{sec:nominal}

In this section, we describe some limitations of CompCert's block-based
memory model and present our new 
{\em nominal memory model}, which removes the global constraints for
managing memory blocks and enables flexible memory structures for open
programs and concurrent threads.

Memory models are critical components for formalizing the semantics of
imperative programming languages. They determine an abstraction over
memory and provide necessary operations for manipulating memory states
at the corresponding level of abstraction.
The \emph{block-based} memory model
used in CompCert~\cite{leroy08,compcert-mem-v2}
provides a medium level of
abstraction---neither too abstract nor too concrete---by modeling
memory space as a collection of contiguous memory regions (also called
\emph{blocks}).
% Each block is given a unique identifier
%(called its \emph{block id}). Furthermore, pointers are naturally
%defined as pairs of block ids and offsets to memory cells, and
%definitions of pointer operations follow in a straightforward manner.

The block-based memory model has been highly successful. By uniformly
applying it to all of CompCert's languages, the developers of CompCert
were able to verify over 20 compiler passes containing advanced
optimizations.  It has also been widely adopted in other verification
projects, including various extensions of CompCert for supporting
compositionality and concurrency
(e.g.,~\cite{sevcik13,compcompcert,compcertm,cascompcert,wang2019,compcertelf20,sepcompcert,compcerts})
and formalization of language semantics for program verification
(e.g.,~\cite{appel11:vst,dscal15,ccal18}).

\subsection{The CompCert Memory Model}
\label{ssec:nominal-bbmm}

In the block-based memory model, the memory space is represented as a
countably infinite set of blocks where each block is given a unique
identifier $b$ called its \emph{block id}, represented
as a postitive number. The entire memory space is
divided into two parts by a special block id called
\nextblock, which
denotes the id of the next block to be allocated
and is increased after each allocation.
A block with id less than \nextblock has been
allocated and is called a \emph{valid block}.
Unallocated blocks are called \emph{invalid blocks}.

A valid memory block is a
finite array of bytes with a lower and upper bound. A \emph{memory
state} consists of a value for \nextblock,
as well as a mapping from addresses to \emph{in-memory values}
for each block id below \nextblock.
A pointer or a memory address is represented as a pair $(b, o)$ where $b$ is a
memory block id and $o$ is an offset in bytes
within the corresponding block.
%
This separation of the address space into
independent blocks
provides basic support for memory isolation.
% instead of:
\ignore{
Pointer arithmetic operates by adjusting offsets, so that for
example $(b, o) + n$ is defined as $(b, o+n)$.
Note that
this simple block-based abstraction already provides basic support for
memory isolation, in the sense that given a pointer to block $b_1$ we
can never reach $b_2$ by performing pointer arithmetic if $b_1 \neq b_2$.
}

%\begin{figure}
%  \begin{subfigure}[b]{.45\textwidth}
%    \begin{coq}
%Definition block : Type := positive.
%Record mem: Type := { 
%  mem_contents: block -> Z -> memval;
%  nextblock: block; }.
%    \end{coq}
%    \caption{Blocks and Memory States}
%    \label{sfig:memstate}
%  \end{subfigure}
%  \begin{subfigure}[b]{.5\textwidth}  
%    \begin{coq}
%alloc : mem -> Z -> Z -> mem \times block
%free  : mem -> block -> Z -> Z -> $\some{\code{mem}}$
%load  : mem -> chunk -> block -> Z -> $\some{\code{val}}$
%store : mem -> chunk -> block -> Z -> val -> $\some{\code{mem}}$
%    \end{coq}
%    \caption{Memory Operations}
%    \label{sfig:memops}
%  \end{subfigure}
%  \caption{Definitions for the Block-Based Memory Model}
%  \label{fig:bbmem}
%%  \vspace{-0.1cm}
%\end{figure}
%
%\figref{sfig:memops} shows the main operations over memory where
%\coqe{val} is the type of \emph{regular values}:
%%
%\begin{coq}
%   Inductive val := Vundef | Vint $i_{32}$ | Vlong $i_{64}$ | Vsingle $f_{32}$ | Vfloat $f_{64}$ | Vptr $(b,o)$.
%\end{coq}
%%
%Here, \coqe{Vundef} represents undefined values, \coqe{Vptr $(b,o)$}
%represents a pointer $(b,o)$, and the remaining forms denote 32- and
%64-bit integer and floating point values.
%%
%\coqe{chunk} is the type of \emph{memory chunks} containing
%information necessary for performing conversion between in-memory
%values and regular values.
%%
%Note that the option type $\lfloor \rfloor$ is used to describe the
%possible failure of some operations.  Given a memory state $m$, a
%lower bound $l$ and a higher bound $h$, \coqe{alloc $m$ $l$ $h$}
%returns a new block whose id is \coqe{$m$.(nextblock)} and whose valid
%offsets are in the range of $[l,h)$. It also returns a new memory state
%  where \nextblock is increased by $1$ and the newly allocated memory
%  cells are initialized with undefined values.
%%
%Note that \coqe{alloc} \emph{always succeeds} because the memory space
%has infinitely many blocks. The \coqe{free} operation frees memory
%cells in a certain range. Given $m$, $k$ $b$ and $o$,
%\coqe{load $m$ $k$ $b$ $o$}
%loads a value starting from the address $(b,o)$ such
%that the size and type of value are determined by $k$.  Conversely,
%\coqe{store} stores a value into a certain location in memory.

A main task of compilers is to transform abstract data structures
(e.g., stacks) for describing the semantics of the high-level source
languages into concrete data structures for the low-level target
languages. These transformations often operate on certain part of
memory and have local effects.
%
CompCert introduces \emph{injections} to capture such changes. An
injection $j$ is a partial function of type
$\code{block} \rightharpoonup \code{block} \times \mathbb{Z}$.
When $j(b) = (b',o)$,
the transformation inserts the source block $b$ into
the target block $b'$ at offset $o$.
When $j(b)$ is undefined,
the transformation removes block $b$ from the memory.

\subsection{Limitations}
\label{ssec:intro-nmm}

Despite its success, CompCert's block-based memory model is
still quite primitive and inflexible, making it difficult to support
end-to-end and compositional verified compilation: 

\begin{itemize}
\item 
\textbf{Inflexibility 1: Fixed Representation of Block IDs.}
%
The fixed representation of block ids by positive numbers
makes it impossible to distinguish between memory regions
with different roles (such as the stack, heap, and global
data), making it difficult to reason about specific parts of
memory in isolation.

\item
\textbf{Inflexibility 2: Sequential Numbering of Valid Blocks.}  
%
The ids of valid
blocks must be numbered sequentially by $[1,\ldots,\nextblock-1]$.
This creates unintuitive and unnecessary dependency between valid
blocks playing different roles.

\item
\textbf{Inflexibility 3: Global Constraint for Allocation.}
%
In any memory state, there is only a single \nextblock for assigning
new block ids upon allocation. In the setting where multiple open
programs or concurrent threads work on the same memory state, every
open program or thread must keep track of how other programs or
threads modify \nextblock. This global constraint imposes a complex
dependency between open programs and threads.
\end{itemize}

Because of the complex dependencies created by the above
inflexibilities, verification of any compiler transformation must
treat the memory space \emph{as a whole}. This incurs significant
difficulties.
For example, many
compiler transformations only operate on certain sub-regions of memory
(\eg, transformations on stack frames). However, because of
\textbf{Inflexibilities 1 and 2}, the reasoning must be lifted to the
whole memory, therefore becoming significantly more involved and less
intuitive.

When proving the correctness of memory-transforming passes,
we would like to exploit the locality of memory
transformations to construct intuitive proofs. For example, for
\code{SimplLocals}, \coqe{Cminorgen} and \coqe{Stacking}, one would
like to exploit the facts that \emph{1)} blocks for global definitions
are unchanged, and \emph{2)} the changes to individual stack frames
cannot interfere with each other. However, it is impossible to
formalize these facts directly because we can neither distinguish
blocks for global definitions from blocks in stack frames, nor tell
the differences between blocks in different stack frames: they are all
indexed by positive numbers. This problem is exactly caused by
\textbf{Inflexibility 1} introduced above.

Moreover,
verification techniques for open programs are compositional only if
the semantics of open programs are compatible with each other.
The existence of a global \nextblock makes this compatibility
difficult to establish, even when different open programs operate
on completely separate memory regions.
\ignore{
Below we take the compilation and linking of Certified Concurrent
Abstraction Layers (CCAL) as an example to illustrate the above
problem~\cite{ccal18}.

A CCAL object is a formally verified open program built on top of some
layer $L$. A multi-threaded program is developed by abstracting
individual threads into CCAL objects. These objects are then
compiled by an extension of CompCert called Thread-Safe CompCertX into
assembly code. Finally, CCAL objects need to be linked together to
form a complete program. 
This requires CCAL object to use compatible views of the memory.
Achieving this
compatibility is non-trivial since 
a new stack block is
allocated for every function call in CompCert's assembly. To
link threads together, it is necessary for each thread to have
the same sequences of
stack blocks and \nextblock, meanwhile preventing one
thread from accessing the private stack memory of the others.
%
%
These problems are exactly caused by
\textbf{Inflexibilities 2 and 3} in~\S\ref{ssec:intro-nmm}.
}

Problems of this kind plague many projects
extending CompCert to support compositionality or concurrency (\eg,
CASCompCert~\cite{cascompcert} and Thread-Safe
CompCertX~\cite{ccal18}). To circumvent them, various ad hoc
modifications to the block-based memory model were invented, leading to
verification results that are overly technical and less reusable.
We will solve these problems by getting rid of the inflexibilites of
the block-based memory model through a generalization based on nominal
techniques.

\subsection{The Nominal Memory Model}
\label{ssec:nominal-nmm}

Nominal techniques~\cite{pitts-nominal,gabby2002} provide a
mathematical foundation for managing named resources. 
%
In this setting, any countably infinite set $\nameset{A}$ can
represent a set of available names. Elements in these sets are called
\emph{atomic names} (or simply \emph{names} in short). Dependency of
objects upon names is captured by the notions of \emph{permutations}
and \emph{supports}. A permutation $\pi$ is a bijection from
$\nameset{A}$ to itself that only renames a finite subset of names in
$\nameset{A}$. Given a set of objects $X$ and some $x \in X$, $A
\subseteq \nameset{A}$ supports $x$ (or $A$ is a support of $x$) iff,
for any permutation $\pi$ on $\nameset{A}$ that is an identity mapping
for names in $A$, we have $\pi \cdot x = x$ where $\_ \cdot \_$
denotes the ``application'' (known as an \emph{action}) of $\pi$ to the
object $x$. This effectively means that $x$ is independent of any name
outside of $A$.
%
Only objects that can be supported by a finite set of names are of
interest to us.
%
A binary relation called \emph{freshness} makes the independence
relation concrete. A name $a\in \nameset{A}$ is fresh w.r.t. $x$
(written as $\freshness{a}{x}$) iff $x$ is supported by some finite set $A
\subseteq \nameset{A}$ and $a \not\in A$.

The above concepts can be used to characterize the block-based
memory model. By taking memory states as objects containing names, 
we adopt the following analogies:
%
\begin{itemize}\itemsep 0pt
\item Block ids represent names that memory states depend upon;
\item Given a memory state $m$, the set of valid blocks in $m$
  represents a support of $m$;
\item Given a memory state $m$, its \nextblock is fresh w.r.t. $m$.
\end{itemize}
%
%Note that the set of valid blocks is always finite.

To some extent, the
existing memory operations in CompCert already exploit the properties
of atomic names and supports. For example,
\coqe{alloc} always succeeds because there is always an infinite amount
of ids outside the set of valid blocks.
%
However, 
the block-based memory also makes rigid assumptions
about names and supports.
These limitations have been outlined in \S\ref{ssec:intro-nmm}
and can be reframed in the context of nominal sets:
%
\begin{enumerate} \itemsep 0pt
\item Block ids are fixed to positive numbers;
\item For any memory state $m$ whose \nextblock is $n$, its support
  must be the set $\{1,\ldots,n-1\}$;
\item For any memory state $m$ whose \nextblock is $n$, its fresh
  block must be assigned with the id $n$.
\end{enumerate}

Nominal CompCert \cite{wang2022}
lifts these restrictions
by generalizing the interface of
CompCert's memory model.
The new interface,
based on nominal sets,
is formalized as a Coq module type.
%
This establishes a clean separation between
the requirements of language semantics and simulation proofs
(now expressed in terms of nominal sets)
and the possible implementations of the memory model.
Because the requirements of the existing code
already largely (though implicitly) matched
the properties used to define the nominal model,
we were able to update
the language semantics and correctness proof of CompCert
with minimal changes.
While the block-based memory model
can still serve as a simple implementation of nominal memory,
it becomes possible to introduce
more sophisticated models,
which can then provide additional functionality
and enable forms of reasoning
which are impossible
or excessively complex in the block-based model.

\subsection{Proposed Work}

We plan to
further study the advantages of the nominal memory model,
incorporate it into CompCertO,
and take advantage of its possibilities
to enable various forms of reasoning
within our framework.

\ignore{
\vspace*{-2ex}
\paragraph*{Task 2a: Nominal CompCert with structural memory space.}
We will develop various instantiations
of the nominal memory model where the whole {\em memory space} is
divided into memory regions with well-defined structures and clear
roles. One promising design~\cite{wang2022} separates global memory
(for global variables and functions) from stack memory; and then
organizes the stack into a tree structure that mirrors the ``call
activation'' tree.  Because most memory transformations do not change
the order of function calls and returns, the stack tree remains stable
under compilation; this allows us to recover the locality decompose a
transformation on the entire stack into local transformations on
individual stack frames, which could lead to much simpler and more
intuitive proofs for memory injection phases.
}

\vspace*{-2ex}
\paragraph*{Task 2a: Nominal CompCertO for open programs.}
Another key task is to integrate the nominal memory model into our
new compositional verified compilation framework in
\S\ref{sec:compcerto}.  Compositional compilation needs to not only
keep track of how memory is transformed by internal functions, but
also do that for external functions.  If the context is fixed, the
transformation on contextual memory should always be described as
\emph{an identity mapping} from source to target memory blocks,
regardless of how complicated the memory transformation is for
internal executions.  However, this seemingly simple task is extremely
difficult to complete in the original CompCert because it lacks the
ability to distinguish memory blocks allocated by internal functions
from those allocated by external ones. With the nominal memory model and its
instances, contextual memory can be separated from internal memory and
be reasoned about via a well-defined interface, leading to simpler and
more elegant correctness proofs for compiling certain kinds of open
programs. We believe that these techniques together with others we
have built on top of the nominal memory model could be particularly
beneficial in the general context of compositional compiler
correctness.

\vspace*{-2ex}
\paragraph*{Task 2b: Develop a separation algebra for CompCert memory}
In \S\ref{sec:xxx}
we explained the role separation algebras can play
in consolidating high-level, abstract, compositional state
into a single concrete CompCert memory state.
%
Unfortunately,
due to the issues with \nextblock
discussed in this section,
defining a separation algebra for CompCert memory
is difficult.
We will use the extended capabilities of
the nominal memory model
to overcome this challenge.

\vspace*{-2ex}
\paragraph*{Task 2c: Nominal CompCertO for open threads}
We will extend Nominal CompCertO to support
multi-threaded programs. The existing solutions invented ad hoc
mechanisms~\cite{ccal18} to cope with the global \nextblock which
prevent further compilation to a realistic machine model in which each
thread has its own contiguous and private stack. We plan to
instantiate supports with multiple stack trees where we can grow
the stacks individually without interference with each other, thereby
eliminating the problems with \nextblock.



