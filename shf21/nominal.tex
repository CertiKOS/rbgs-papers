\section{Verified Compilation with a Nominal Memory Model}
\label{sec:nominal}

In this section, we describe the limitations of the block-based
memory model and present our new 
{\em nominal memory model} which removes the global constraints for
managing memory blocks and enables flexible memory structures for open
programs and concurrent threads.

\subsection{Problems with the Block-Based Memory Model}
\label{ssec:nominal-bbmm}

In the block-based memory model, the memory space is represented as a
countably infinite set of blocks where each block is given a unique
identifier called its \emph{block id} (denoted by $b$, represented
as postitive numbers). The entire memory space is
divided into two parts by a special block id called
\emph{\nextblock}. A block with id less than \nextblock has been
allocated (called a \emph{valid block}). Otherwise, it has not been
allocated yet (called an \emph{invalid block}). \nextblock---initially
with the value $1$---denotes the id of the next block to be allocated
and will be increased after each allocation. A valid memory block is a
finite array of bytes with a lower and upper bound. A \emph{memory
state} (denoted by $m$) consists of a mapping from addresses to
\emph{in-memory values} in valid blocks and the value of
\nextblock. Its type \coqe{mem} is defined in~\figref{sfig:memstate}
where \coqe{block} is the type of block ids, \coqe{Z} is the type of
integers, \coqe{memval} is the type of in-memory values, and
\coqe{$m$.(mem_contents) $b$ $o$ = $v$} iff $b$ is a valid block in
$m$ and $v$ is the in-memory value at the $o$-th memory cell (byte) of
$b$.

A pointer or a memory address is a pair $(b, o)$ where $b$ is the
memory block it points to and $o$ is an index to a memory cell in block
$b$ (also called an offset).
%
Pointer arithmetic is represented by adjustments to offsets. For
example, $(b, o) + n$ is defined as $(b, o+n)$.
%
This simple block-based abstraction already provides basic support for
memory isolation, in the sense that given a pointer to block $b_1$ we
can never reach $b_2$ by performing pointer arithmetic if $b_1 \neq b_2$.

\begin{figure}
  \begin{subfigure}[b]{.45\textwidth}
    \begin{coq}
Definition block : Type := positive.
Record mem: Type := { 
  mem_contents: block -> Z -> memval;
  nextblock: block; }.
    \end{coq}
    \caption{Blocks and Memory States}
    \label{sfig:memstate}
  \end{subfigure}
  \begin{subfigure}[b]{.5\textwidth}  
    \begin{coq}
alloc : mem -> Z -> Z -> mem \times block
free  : mem -> block -> Z -> Z -> $\some{\code{mem}}$
load  : mem -> chunk -> block -> Z -> $\some{\code{val}}$
store : mem -> chunk -> block -> Z -> val -> $\some{\code{mem}}$
    \end{coq}
    \caption{Memory Operations}
    \label{sfig:memops}
  \end{subfigure}
  \caption{Definitions for the Block-Based Memory Model}
  \label{fig:bbmem}
%  \vspace{-0.1cm}
\end{figure}

\figref{sfig:memops} shows the main operations over memory where
\coqe{val} is the type of \emph{regular values}:
%
\begin{coq}
   Inductive val := Vundef | Vint $i_{32}$ | Vlong $i_{64}$ | Vsingle $f_{32}$ | Vfloat $f_{64}$ | Vptr $(b,o)$.
\end{coq}
%
Here, \coqe{Vundef} represents undefined values, \coqe{Vptr $(b,o)$}
represents a pointer $(b,o)$, and the remaining forms denote 32- and
64-bit integer and floating point values.
%
\coqe{chunk} is the type of \emph{memory chunks} containing
information necessary for performing conversion between in-memory
values and regular values.
%
Note that the option type $\lfloor \rfloor$ is used to describe the
possible failure of some operations.  Given a memory state $m$, a
lower bound $l$ and a higher bound $h$, \coqe{alloc $m$ $l$ $h$}
returns a new block whose id is \coqe{$m$.(nextblock)} and whose valid
offsets are in the range of $[l,h)$. It also returns a new memory state
  where \nextblock is increased by $1$ and the newly allocated memory
  cells are initialized with undefined values.
%
Note that \coqe{alloc} \emph{always succeeds} because the memory space
has infinitely many blocks. The \coqe{free} operation frees memory
cells in a certain range. Given $m$, $k$ $b$ and $o$,
\coqe{load $m$ $k$ $b$ $o$}
loads a value starting from the address $(b,o)$ such
that the size and type of value are determined by $k$.  Conversely,
\coqe{store} stores a value into a certain location in memory.

A main task of compilers is to transform abstract data structures
(e.g., stacks) for describing the semantics of the high-level source
languages into concrete data structures for the low-level target
languages. These transformations often operate on certain part of
memory and have local effects.

CompCert introduces \emph{injections} to capture such changes. An
injection $j$ is a partial function of type
\coqe{block -> $\some{\code{block} \times \code{Z}}$}, such that $j(b) =
\some{(b',o)}$ iff a block $b$ is inserted into block $b'$ at offset
$o$ by the transformation. If $j(b) = \none$ (we use $\none$ to
represent the \coqe{None} constructor), then $b$ is removed from the
memory after the transformation.

When proving the correctness of these memory-transforming passes, one
would expect that we could exploit the locality of memory
transformations to construct intuitive proofs. For example, for
\code{SimplLocals}, \coqe{Cminorgen} and \coqe{Stacking}, one would
like to exploit the facts that \emph{1)} blocks for global definitions
are unchanged, and \emph{2)} the changes to individual stack frames
cannot interfere with each other. However, it is impossible to
formalize these facts directly because we can neither distinguish
blocks for global definitions from blocks in stack frames, nor tell
the differences between blocks in different stack frames: they are all
indexed by positive numbers. This problem is exactly caused by
\textbf{Inflexibility 1} introduced in~\S\ref{ssec:intro-nmm}.


Verification techniques for open programs are compositional only if
the semantics of open programs are compatible with each other.
However, the existence of a global \nextblock makes this compatibility
very difficult to establish, even when different open programs operate
on completely separate memory regions.

\begin{wrapfigure}[12]{r}{0.5\textwidth}
  \begin{tikzpicture}[scale=0.6, every node/.style={scale=0.9},auto]
    \def\bh{0.2cm}
    \tikzstyle{af}=[draw,minimum width=0.5cm, minimum height=0.5cm, rounded corners=.05cm];
    
    \node[af,fill=yellow] (t1f) {$f$};
    \node[af, below=\bh of t1f, fill=yellow] (t1g) {$g$};
    \node[af, below=\bh of t1g, pattern = north east lines] (t1i) {};
    \node[af, below=\bh of t1i, pattern = north east lines] (t1j) {};
    \node[af, below=\bh of t1j, fill=yellow] (t1h) {$h$};
    %% \node[draw = none, below left =\bh and -0.1cm of t1h, rotate = 90] (t1end) {$\ldots$};
    
    \draw[->,>=stealth] (t1f) -- (t1g);
    \draw[->,>=stealth] (t1g) -- (t1i);
    \draw[->,>=stealth] (t1i) -- (t1j);
    \draw[->,>=stealth] (t1j) -- (t1h);
    \draw[->,>=stealth] (t1h) -- ($(t1h.south) + (0cm, -0.3cm)$);

    \node[above = 0.1cm of t1f] {Thread $1$};

    \node[af, right = 2cm of t1f, pattern = north east lines] (t2f) {};
    \node[af, below=0.22cm of t2f, pattern=north east lines] (t2g) {};
    \node[af, below=\bh of t2g,fill=green] (t2i) {$i$};
    \node[af, below=\bh of t2i,fill=green] (t2j) {$j$};
    \node[af, below=\bh of t2j, pattern=north east lines] (t2h) {};
    %% \node[draw = none, below left =\bh and -0.1cm of t2h, rotate = 90] (t2end) {$\ldots$};

    \draw[->,>=stealth] (t2f) -- (t2g);
    \draw[->,>=stealth] (t2g) -- (t2i);
    \draw[->,>=stealth] (t2i) -- (t2j);
    \draw[->,>=stealth] (t2j) -- (t2h);
    \draw[->,>=stealth] (t2h) -- ($(t2h.south) + (0cm, -0.3cm)$);

    \node[above = 0.1cm of t2f] (t2txt) {Thread $2$};

    \node[draw=none, right = 1cm of t2i] (comp) {\large\bf $\Longrightarrow$};
    \node[draw=none, above = 0.1cm of comp] {\small \emph{Thread Linking}};

    \draw[->, dashed, >=stealth] (t1g) -- node [sloped, above] {\small \emph{yield}} (t2i);
    \draw[->, dashed, >=stealth] (t2j) -- node [sloped, above] {\small \emph{yield}} (t1j);


    \node[af, right = 3cm of t2f,fill=yellow] (tf) {$f$};
    \node[af, below=\bh of tf,fill=yellow] (tg) {$g$};
    \node[af, below=\bh of tg,fill=green] (ti) {$i$};
    \node[af, below=\bh of ti,fill=green] (tj) {$j$};
    \node[af, below=\bh of tj,fill=yellow] (th) {$h$};
    %% \node[draw = none, below left =\bh and -0.1cm of th, rotate = 90] (tend) {$\ldots$};

    \draw[->,>=stealth] (tf) -- (tg);
    \draw[->,>=stealth] (tg) -- (ti);
    \draw[->,>=stealth] (ti) -- (tj);
    \draw[->,>=stealth] (tj) -- (th);
    \draw[->,>=stealth] (th) -- ($(th.south) + (0cm, -\bh)$);
    
    \node[right = 2cm of t2txt] {Threads $\{1,2\}$};
    

  \end{tikzpicture}
  \vspace{-0.2cm}
  \caption{\scriptsize{}Linking of Multiple Stacks into a Single Stack in CCAL}
  \vspace{-0.2cm}
  \label{fig:stack-linking}
\end{wrapfigure}

We take the compilation and linking of Certified Concurrent
Abstraction Layers (CCAL) as an example to illustrate the above
problem~\cite{ccal18}. A concurrent certified abstraction layer $L$
consists of shared and private memory states, abstract states, and a
set of possibly shared primitive operations (like external
functions). The semantics of a language (\eg, C or assembly) built on
top of $L$ forms an abstract machine in which concurrent programs can
be formally described. 
%
A CCAL object is a formally verified open program built on top of some
layer $L$. A multi-threaded program is developed by abstracting
individual threads into CCAL objects. These objects are then
compiled by an extension of CompCert called Thread-Safe CompCertX into
assembly code. Finally, CCAL objects need to be linked together to
form a complete program. 

For the above linking to be possible, the views of memory of CCAL
objects must be compatible with each other. Achieving this
compatibility is a non-trivial task.
%
To understand this, observe that a new stack block is
allocated for every function call in CompCert's assembly. To
link threads together, it is necessary for each thread to have
the same sequences of
stack blocks and \nextblock, meanwhile preventing one
thread from accessing the private stack memory of the others.

To solve the above problem, the authors of Thread-Safe
CompCertX~\cite{ccal18} modify the assembly semantics so that when a
thread yields to its context, a sequence of dummy stack blocks is
allocated to increase \nextblock in accordance with the actual
allocation of stack blocks by the context. Moreover, to avoid any
interference between memory operations on the stacks in different
threads, the dummy blocks do not carry any read or write
permission---they are ``dead'' memory cells from the perspective of
the focused thread. With those devices, it is possible to ``thread''
the private stack frames of each thread into a single stack. As an
example, the linking of stacks for two threads is depicted in
\figref{fig:stack-linking}. Here, the call to the yield primitive from
thread $1$ to $2$ in the function $g$ allocates two dummy blocks
(marked with diagonal lines) for the corresponding execution in thread
$2$. Accesses to these dummy blocks are invalid in thread $1$.


The solution above has two problems.
%
First, intuitively, each thread should have it own private stack:
the context should not interfere with operations on this stack.
Contrary to this assumption, the growth of dummy frames depends
on how contextual threads change \nextblock. This introduces
unnecessary complication to verification of compilation.
%
Second, in the linked program, stack frames for different threads are
interleaved with each other. This makes the semantics of
linked programs much more complex. It is also extremely difficult to
further verify their compilation to actual machine code where each
thread should have its own contiguous and private stack.
%
These problems are exactly caused by
\textbf{Inflexibility 2 and 3} in~\S\ref{ssec:intro-nmm}.

\subsection{The Nominal Memory Model}
\label{ssec:nominal-nmm}

We solve the above problems by getting rid of the inflexibilites of
the block-based memory model through a generalization based on nominal
techniques.

Nominal techniques~\cite{pitts-nominal,gabby2002} provide a
mathematical foundation for managing named resources. 
%
In this setting, any countably infinite set $\nameset{A}$ can
represent a set of available names. Elements in these sets are called
\emph{atomic names} (or simply \emph{names} in short). Dependency of
objects upon names is captured by the notions of \emph{permutations}
and \emph{supports}. A permutation $\pi$ is a bijection from
$\nameset{A}$ to itself that only renames a finite subset of names in
$\nameset{A}$. Given a set of objects $X$ and some $x \in X$, $A
\subseteq \nameset{A}$ supports $x$ (or $A$ is a support of $x$) iff,
for any permutation $\pi$ on $\nameset{A}$ that is an identity mapping
for names in $A$, we have $\pi \cdot x = x$ where $\_ \cdot \_$
denotes the ``application'' (known as an \emph{action}) of $\pi$ to the
object $x$. This effectively means that $x$ is independent of any name
outside of $A$.
%
Only objects that can be supported by a finite set of names are of
interest to us.
%
A binary relation called \emph{freshness} makes the independence
relation concrete. A name $a\in \nameset{A}$ is fresh w.r.t. $x$
(written as $\freshness{a}{x}$) iff $x$ is supported by some finite set $A
\subseteq \nameset{A}$ and $a \not\in A$.

\begin{figure}[t]
\begin{coq}
  Module Type BLOCK.
    Parameter block : Type.
    Parameter eq_block : $\forall\app x\app y: \scode{block}, \sumbool{x=y}{x \neq y}$.
  End BLOCK.

  Module Type SUP.
    Parameter sup: Type.
    (** Operations *)
    Parameter sup_empty : sup.  (* Empty support *)
    Parameter fresh_block : sup -> block.  (* Generation of fresh blocks *)
    Parameter sup_incr : sup -> sup.  (* Increment of supports *)
    Parameter sup_in : block -> sup -> Prop.   (* Check validity of block ids *)
    (** Properties *)
    Parameter sup_dec : $\forall\app b\app s, \sumbool{\scode{sup\_in}\app b\app s}{\neg \scode{sup\_in}\app b\app s}$.
    Parameter empty_in: $\forall\app b, \neg \scode{sup\_in}\app b\app \scode{sup\_empty}$.
    Parameter freshness : $\forall\app s, \neg \scode{sup\_in}\app (\scode{fresh\_block}\app s)\app s$.
    Parameter sup_incr_in : $\forall\app a\app s, \scode{sup\_in}\app a\app (\scode{sup\_incr}\app s) 
    \leftrightarrow a = (\scode{fresh\_block}\app s) \lor \scode{sup\_in}\app a\app s$.
  End SUP.  
\end{coq}
%  \vspace{-0.2cm}
  \caption{Interfaces of the Nominal Memory Model}
  \label{fig:nm-interface}
%  \vspace{-0.1cm}
\end{figure}

The above concepts can be used to characterize the block-based
memory model. By taking memory states as objects containing names, 
we adopt the following analogies:
%
\begin{itemize}\itemsep 0pt
\item Block ids represent names that memory states depend upon;
\item Given a memory state $m$, the set of valid blocks in $m$
  represents a support of $m$;
\item Given a memory state $m$, its \nextblock is fresh w.r.t. $m$.
\end{itemize}
%
Note that the set of valid blocks is always finite. To some extent, the
existing memory operations in CompCert already exploit the properties
of atomic names and supports. For example,
\coqe{alloc} always succeeds because there is always an infinite amount
of ids outside the set of valid blocks.

However, the block-based memory also makes rigid assumptions
about names and supports:
%
\begin{itemize} \itemsep 0pt
\item Block ids are fixed to positive numbers;
\item For any memory state $m$ whose \nextblock is $n$, its support
  must be the fixed set of consecutive numbers $\{1,\ldots,n-1\}$;
\item For any memory state $m$ whose \nextblock is $n$, its fresh
  block must be assigned with the id $n$.
\end{itemize}
%

We shall remove these rigid assumptions and generalize the block-based
memory model into the nominal memory model by introducing an
abstraction of block ids and supports as module types, as depicted
in~\figref{fig:nm-interface}.
%
By the definition of \coqe{BLOCK}, block ids are names with decidable equality.
%
By the definition of \coqe{SUP}, a support type must be accompanied by
four kinds
of operations: checking the membership of blocks in supports
(\coqe{sup_in}), creating an empty support (\coqe{sup_empty}),
generating fresh blocks (\coqe{fresh_block}) and increasing supports
with new blocks (\coqe{sup_incr}). Furthermore, those operations must
satisfy some well-formedness properties. 

\begin{figure}
\begin{coq}
  Module Block <: BLOCK.
    Definition block := positive.         Definition eq_block := peq.
  End Block.

  Module Sup <: SUP.
    Definition sup := list block.         Definition sup_in ($b$: block) ($s$: sup) : Prop := $b \in s$.
    Definition sup_empty : sup := [].     Definition fresh_block ($s$: sup) := (find_max_pos $s$) + 1.
    Definition sup_incr ($s$ :sup) := $(\scode{fresh\_block }s) \cons s$.   $\qquad$ ...
  End Sup.  
\end{coq}  
  \vspace{-0.2cm}
  \caption{Block-based Memory Model as an Instance}
  \vspace{-0.1cm}
  \label{fig:bm-instance}
\end{figure}

We also note that the above generalization does not exactly match with the
standard definitions in nominal techniques. For example, \coqe{BLOCK} does not
enforce that block ids are from a countably infinite set. Instead,
the \coqe{freshness} property guarantees that any block fresh w.r.t a
support $s$ must not be already in $s$. Together with the totality of
\coqe{fresh_block}, it implies the existence of an infinite number of
block ids.
%
We also define supports to be any type that has the interface of
\coqe{SUP}, instead of a finite set of block ids. This generalization
allows us to instantiate \coqe{sup} with complex data structures for
formalizing memory space in fine-grained styles. 

To make use of the above interfaces, we instantiate block ids and
supports as follows:
%
\begin{coq}
  Module Block <: BLOCK. ... End Block.           Module Sup <: SUP. ... End Sup.
\end{coq}
%
Then the original \coqe{block} type is instantiated with
\coqe{Block.block}. Moreover, the memory state carries a support instead of
\nextblock:
%
\begin{coq}
  Record mem: Type := { mem_contents: block -> Z -> memval;   support: Sup.sup;}.
\end{coq}
%
The memory operations are adapted accordingly. For example, checking
of valid blocks is done by using \coqe{sup_in} instead of comparing
with \nextblock:
%
\begin{coq}
  Definition valid_block ($m$:mem) ($b$:block) := $b$ < $m$.(nextblock).   (* Old *)
  Definition valid_block ($m$:mem) ($b$:block) := sup_in $b$ $m$.(support).  (* New *)
\end{coq}
%
For another example, \coqe{alloc} now invokes \coqe{fresh_block} to
allocate a new block instead of consulting \nextblock.
The properties of all memory operations can be easily reestablished
because they are already ignorant of particular instantiations of block ids
and supports. 

Finally, the block-based memory model becomes a special case of the
nominal memory model, as depicted in~\figref{fig:bm-instance} where
\coqe{find_max_pos} finds the maximal positive number in a list.

As a preliminary evidence for its effectiveness, we have recently
successfully applied the nominal memory model to the full compilation
chain of CompCert to get Nominal CompCert~\cite{wang2022}.  We updated
the semantics of every language of CompCert by replacing the old
memory operations with the new one. We then also updated the
simulation proofs. Because the generalization of block ids and
supports is mostly orthogonal to the simulation proofs in CompCert,
the changes are minimal.

\vspace*{-2ex}
\paragraph*{Task 2a: Nominal CompCert with structural memory space.}
A big part of our proposed effort is to develop various instantiations
of the nominal memory model where the whole {\em memory space} is
divided into memory regions with well-defined structures and clear
roles. One promising design~\cite{wang2022} separates global memory
(for global variables and functions) from stack memory; and then
organizes the stack into a tree structure that mirrors the ``call
activation'' tree.  Because most memory transformations do not change
the order of function calls and returns, the stack tree remains stable
under compilation; this allows us to recover the locality decompose a
transformation on the entire stack into local transformations on
individual stack frames, which could lead to much simpler and more
intuitive proofs for memory injection phases.

\vspace*{-2ex}
\paragraph*{Task 2b: Norminal CompCertO for open programs.}
Another key task is to integrate the norminal memory model into our
new compositional verified compilation framework in
\S\ref{sec:compcerto}.  Compositional compilation needs to not only
keep track of how memory is transformed by internal functions, but
also do that for external functions.  If the context is fixed, the
transformation on contextual memory should always be described as
\emph{an identity mapping} from source to target memory blocks,
regardless of how complicated the memory transformation is for
internal executions.  However, this seemingly simple task is extremely
difficult to complete in the original CompCert because it lacks the
ability to distinguish memory blocks allocated by internal functions
from those by external ones. With the nominal memory model and its
instances, contextual memory can be separated from internal memory and
be reasoned about via a well-defined interface, leading to simpler and
more elegant correctness proofs for compiling certain kinds of open
programs. We believe that these techniques together with others we
have built on top of the nominal memory model could be particularly
beneficial in the general context of compositional compiler
correctness.

\ignore{
In CompCert, the complexity of memory injections is largely confined
to the correctness proofs of individual passes: injections are
existentially quantified as part of the simulation relation and do not
appear in the correctness statement itself.  This is possible because
memory states are not part of the externally observable behavior of
programs. This assumption must be relaxed in compositional extensions
of CompCert which model interactions between compilation units.  In
work such as Compositional CompCert~\cite{stewart15},
CompCertX~\cite{dscal15,wang2019}, CompCertM~\cite{compcertm},
CASCompCert~\cite{cascompcert} and CompCertO~\cite{koenig21}, memory
states appear as part of the interactions between components.  As a
consequence, the memory relations used by compilation passes become
part of their correctness statements.  This makes correctness theorems
and the vertical composition of passes much more complex.

Our techniques based on the nominal memory model have the potential to
significantly simplify the above proofs from the following
perspectives.  First, the evolving Kripke worlds used in compositional
compiler correctness are used to store both memory injections (which
can be determined ahead of time using more structural injection
functions) and additional permission information (which could be
stored in our model as part of the information associated with block
identifiers).  As a result, by incorporating our approach, it may be
possible to eliminate Kripke worlds entirely, which would
significantly simplify the semantic frameworks used in these projects.
Moreover, our techniques could help isolate private memory as internal
states, for example through the use of a partial commutative monoid
structure on memory states which would facilitate splitting and
merging private memory regions used by individual components.
Finally, they could provide a seamless way to incorporate stack
awareness to compositional compilers of this kind.
}

\vspace*{-2ex}
\paragraph*{Task 2c: Norminal CompCertO for open threads}
We will extend Nominal CompCertO to support
multi-threaded programs. The existing solutions invented ad hoc
mechanisms~\cite{ccal18} to cope with the global \nextblock which
prevent further compilation to a realistic machine model in which each
thread has its own contiguous and private stack. We plan to
instantiate supports with multiple stack trees where we can grow
the stacks individually without interference with each other, thereby
eliminating the problems with \nextblock.



