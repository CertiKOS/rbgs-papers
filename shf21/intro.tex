\section{Introduction}
\label{sec:intro}

Over the past decade, researchers have been able to formally verify
various key components of computer systems, including compilers
\cite{compcert,cakeml,vellvm}, OS kernels
\cite{sel4,popl15,osdi16}, file systems \cite{fscq} and processor
designs \cite{safe,kami}.  Building on these successes, the research
community is attempting to construct large-scale, heterogeneous
certified systems by using formal specifications as interfaces between
the correctness proofs of various components \cite{deepspec}.  The
ongoing design of suitable semantic frameworks is an important step
towards this goal.  However, incorporating certified compilers into
frameworks of this kind presents a number of difficulties.

\subsection{Compositional Compiler Correctness} %{{{

Compiler correctness is often formulated as a \emph{semantics
preservation} property, asserting that the semantics of the compiled
program $\kw{C}(p)$ are related in some particular way to the
semantics of the source program $p$:
\begin{equation} \label{eqn:semp}
  \llbracket p \rrbracket_\kw{S} \sim
  \llbracket \kw{C}(p) \rrbracket_\kw{T}
  \,.
\end{equation}
For whole-program compilers, semantics preservation is straightforward
enough.  In CompCert~\cite{compcert}, the semantics of the source and
target programs (denoted as $\llbracket \cdot \rrbracket_\kw{S}$
and $\llbracket \cdot \rrbracket_\kw{T}$)
are given as labeled transition systems, and the
relation $\sim$ is a simulation property.

However,
practical applications involve
program \emph{components} which we want to compile
and verify separately from each other.
In principle,
the use of a compositional semantics
enables the formulation of (\ref{eqn:semp})
at the level of individual components.
Unfortunately, traditional approaches to compositional semantics
fare poorly in the presence of advanced language features,
or of the kind of abstraction
involved in the compilation process.
For CompCert,
early attempts along these lines
have proven
challenging \cite{cpp15,compcompcert}.

As a result,
common wisdom holds semantics preservation
to be a lost cause
for compositional compiler correctness \cite{next700}.
Instead,
research has focused on
compositional reasoning methods
based on contextual refinement,
side-stepping the need for compositional semantics preservation
\cite{sepcompcert,compcertm}.

%}}}

\subsection{Decomposing Heterogeneous Systems} %{{{

Unfortunately,
these methods share an intrinsic limitation:
they presuppose the existence of a completed system
to be proven correct,
and compositionality only operates within its boundary.
This becomes a serious impediment
in the context of large-scale heterogeneous systems.

\begin{example} \label{ex:nicdriver} %{{{
Consider the problem of verifying
a network interface card (NIC) driver.
The NIC and its driver are closely coupled,
but the details of their interaction
are irrelevant to the rest of the system;
these details should not leak into our large-scale reasoning.
Instead,
we wish to treat the NIC and its driver as a unit,
and establish a direct relationship between C calls into
the driver and network communication.
Together, the NIC and driver implement
a specification $\sigma :
\li{Net} \rightarrow \mathcal{C}$,
meaning they \emph{use} the interface $\li{Net}$
modeling the network,
and \emph{provide} the interface $\mathcal{C}$
modeling C function calls.

The driver code could be specified
($\sigma_\kw{drv}$)
and verified
at the level of CompCert semantics,
whereas device I/O primitives
($\sigma_\kw{io}$)
and the NIC itself
($\sigma_\kw{NIC}$)
would be specified as additional components
in the context of a richer model:
\[
  \sigma_\kw{NIC} : \li{Net} \rightarrow \li{IO}
  \qquad
  \sigma_\kw{io} : \li{IO} \rightarrow \mathcal{C}
  \qquad
  \sigma_\kw{drv} : \mathcal{C} \rightarrow \mathcal{C}
\]
By reasoning about their interaction,
it would be possible to establish a relationship between
the overall specification $\sigma : \li{Net} \rightarrow \mathcal{C}$ and
the composition
$\sigma_\kw{drv} \circ \sigma_\kw{io} \circ \sigma_\kw{NIC}$.
Then a \emph{compiler of certified components}
should help us transport specifications and proofs
obtained at the C level %with respect to the driver's C code
to the compiled code operating at the level of assembly
($\sigma' : \li{Net} \rightarrow \mathcal{A}$).
\end{example}
%}}}

Under existing contextual approaches,
the NIC driver can only be specified and verified in terms of
its interactions at the boundary of the C program code.
Since abstracting away from these interactions
is the role of a driver in the first place,
this is a serious limitation.
%
To be sure,
existing techniques could be extended
to address this specific problem.
For example,
the NIC hardware model could be brought
within the scope of the ``whole program'' being considered,
and the exchange of Ethernet packets
modeled as part of its observable behavior.
However, this approach does not scale.

Example~\ref{ex:nicdriver}
is by no means a contrived corner case.
In fact, patterns of this kind are pervasive even in more mundane situations.
Programmers often use libraries
which mediate access to external resources
(network services, file systems, user interfaces).
Proper high-level specifications
for software components of this kind
must model these resources.
It would rapidly become burdensome to expect
the verification framework
to fix in advance
the dozens or hundreds
of kinds of resources
which may be involved
in the course of verifying a large-scale system.

Fortunately,
advances in compositional semantics
offer a realistic path
to tackling problems of this kind.
In particular,
game semantics (\S\ref{sec:gamesem})
provides a general and expressive framework
to model interactions between typed components.
Recent work proposes integrating
dual nondeterminism and refinement
into this framework,
extending it with powerful mechanisms
to account for abstraction \cite{rbgs-cal}.
Establishing a compatible compiler correctness result
is an important test of this approach
and practical next step.

%}}}


In the last 50 years, the C programming language and its associated
toolchain (e.g., compiler, assembler, linker, loader, memory model,
and application binary interface or ABI) have served as the primary
buiding blocks in the development of today's mainstream system
software stack. The formally verified C compiler CompCert is a recent
breakthrough that holds the promise to become the bedrock of future
{\em certified} heterogeneous system stack.  Indeed, during the last
decade, researchers have been refining the CompCert language semantics
and correctness theorem, and used them as components in various
software verification efforts. Artifacts such as OS kernels and
hypervisors, processor designs, file systems, and network protocols
have been successfuly verified. The significance of CompCert rests not
only on its verified compiler but also on the full formalization of
the ANSI C memory model and ABIs as well as the operational semantics
for all of its source, intermediate, target languages.

Unfortunately, the CompCert ecosystem today still suffers from the
following major shortcomings. First, it does not support compositional
compilation and linking of heterogeneous C and assembly components.
Second, it uses a restricted memory model that has a rigid name space
and is incompatible with multithreaded or multicore programs.  Third,
it is not end-to-end in that CompCert can only produce assembly code
but not actually binary machine code; there are still no certified
linker and loader that can work directly with ELF (Executable and
Linkable Format) binaries.  Fourth, it does not support secure
compilation; source program that is provably information-flow secure
may be compiled into target code that contains information leaks.

In this effort, the PIs propose to
develop a novel verified compilation toolchain that would address
all of the above shortcomings. More specifically,
\begin{itemize}
%%%%%%%%%%%%%%%
\item by using game semantics techniques and richer language interfaces,
they propose to develop a novel {\bf compositional verified compiler} that
can establish a truly compositional compiler correctness theorem for
open heterogeneous components;
%by relating the behaviors of source and target components directly;
%%%%%%%%%%%%%%%
\item they propose to develop a novel {\bf nominal memory model}---an
enhancement to CompCert's block-based memory model with nominal
techniques---to remove the global constraints for managing memory
blocks, and enable flexible memory structures for open and concurrent
programs;
%%%%%%%%%%%%%%%
\item basing upon their prior work on CompCertELF, they will develop an
{\bf end-to-end and compositional verified
compiler} that can compile C components all the way into ELF
object files; they will also build {\em verified compositional linker
and loader} that can work directly with ELF binaries;
%%%%%%%%%%%%%%%
\item they will develop a general
framework for specification, abstraction, and refinement in
compositional semantics, and apply this framework to connect
certified abstraction layer with verified compiler and support
{\bf verified security-preserving compilation};
%%%%%%%%%%%%%%%
\item they will evaluate the effectiveness of the new verified compiler
toolchain in two case studies: one is to use it to build more sophisticated
certified concurrent OS kernels and hypervisors; another is to
use it as a compiler backend for the DeepSEA certified programming tool.
\end{itemize}


We plan to divide \sysname{} into four major components
(Figure~\ref{fig:advert}) and implement each component in
Coq~\cite{coq}: 1) the ADO specification, 2) a multi-ADO reasoning
framework, 3) a network-based specification, and 4) a code-level
verification framework.  The ADO specification sits at the center of
\sysname{} to model individual distributed systems using a simple
atomic interface. The multi-ADO reasoning framework allows composite
systems to be built from ADOs and reasoned about modularly. \sysname{}
fills in the missing details from below with the network-based
specification, and provides a clear path between implementation and
ADO by proving a refinement relation. This specification will be
parameterized such that it can model most protocols with similar
network patterns. Along the way we will solve the challenges that stem
from mapping arbitrarily delayed future events in the network-level
specification to an atomic function call in the ADO model. Verifying
that different systems refine the ADO model will be made simple and
convenient through the network-based specification because the
properties captured by the common network pattern could allow for the
reuse of proofs. Finally, the code level verification framework will
use the certified concurrent abstraction layer
(CCAL)~\cite{concurrency} approach (the PI's prior work).  \sysname{}
will take care of each step to connect the code-level implementation
of a system up to multiple distributed system reasoning.

Our proposed research consists of the following four components:
\begin{itemize}%[leftmargin=*]
\item We will identify and propose an atomic distributed object model
  that preserves the key characteristics of distributed systems, but
  hides the implementation details.
\item We will explore different distributed protocols to create
  parameterized network-based specifications that connect individual
  system implementations to the atomic distributed object model
  through a refinement relation. Each network-based specification will
  reuse the refinement proofs and act as a template that connects a
  similar class of individual systems to the atomic distributed object
  model.
\item We will conduct studies of composite distributed applications to
  illustrate how \sysname{} and atomic distributed object model enable
  simple reasoning about multiple distributed system interactions. We
  will explore various patterns of system compositions for broad
  impact.
\item To demonstrate the power and real-world applicability of
  \sysname{}, we will carry out concrete instantiations multiple
  distributed protocols, including multi-Paxos, and Raft, and extend
  the instantiations to a large-scale distributed shared memory
  system.
\end{itemize}
