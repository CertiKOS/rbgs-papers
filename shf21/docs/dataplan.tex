\documentclass[11pt]{article}
\usepackage{xspace}
\pagestyle{empty}

\usepackage{vmargin}
\setpapersize{USletter}
\setmarginsrb{25mm}{26mm}{26mm}{18mm}{0mm}{0mm}{7mm}{7mm}

\begin{document}

\appendix

\section*{Data Management Plan}

\subsection*{Data Generated}

The proposed project will produce data in the form of source code,
measurement results, formal specifications and proofs, published
papers, and technical reports. The source code developed in this
project will consist chiefly of the tools described in the project
narrative, as well as application benchmarks for evaluating the
developed verification techniques.  The source code, formal
specification, and proofs will be in various languages: C, C++, OCaml,
Coq, and custom languages that the PIs develop and document. The
measurement results will be text files consisting of experimental
logs, gathered by benchmarking logic inside the developed tools.

This project will not involve the acquisition of either animal or
human subjects data.

\subsection*{Availability of Data}

During the period of research, all code and documentation will be kept
under version control using freely available tools, and the
respository backed up or mirrored to ensure redundancy. 

Data resulting from this research will be made widely available to
other researchers. New software and modifications to software that
comprise the multi-scale object implementation and toolchain will be
managed and maintained according to the open standards. Data related
to the implementation of practical tools along with documentation will
be made available to other researchers under an open-source
license.  
Documentation of theoretical developments will either be
published or presented in a peer-reviewed forum, or made available
online and put in the public domain.
The software, data, and documents will be made public through public
GitHub repositories and websites hosted by Yale University.

\subsection*{Data Management}

The faculty, researchers, and students in the Department of Computer
Science have access to a wide variety of ever-changing
state-of-the-art computing resources, ranging from laptops,
conventional PCs and scientific workstations to high-powered
compute-servers and workstation clusters used as parallel computers.

The Computer Science department routinely performs data backup to
prevent against loss of data and ensure all research artifacts are
properly archived. Yale University ITS offers hosting and
administration in a wide variety of reliable, secure, and
cost-effective solutions. Yale ITS offers cloud-based, virtual, and
data center solutions.  This includes hosting for computing and
storage systems in a secure, scalable, and environmentally controlled
space in a Yale Data Center with redundant facilities and resilient
technologies. The Yale ITS private cloud is based on redundant, highly
reliable infrastructure with replicated storage and load and
performance management.

The Department maintains a website, \texttt{www.cs.yale.edu}, which is
directly accessible by members of our community for authoring web
content or for sharing data of any kind.

In accordance with NSF requirements, archived data will be maintained
on these storage systems for a minimum of 3 years after the project
ends in the event that the PIs are called upon to share or otherwise
produce the data. 

\end{document}
