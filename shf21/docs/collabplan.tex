\documentclass[11pt]{article}
\usepackage{xspace}
\pagestyle{empty}

\usepackage{vmargin}
\usepackage{color}
\setpapersize{USletter}
\setmarginsrb{25mm}{26mm}{26mm}{18mm}{0mm}{0mm}{7mm}{7mm}

%\title{Collaboration Plan}
%\date{}
%\author{}
\begin{document}
%\maketitle
\thispagestyle{empty}
\section*{\underline{Collaboration Plan}}
%\subsection*{\underline{Expertise Related to the Project}}
\section{Expertise and Specific Roles}

\paragraph{PI Shao} is the leader of the FLINT team at Yale
which developed both the CertiKOS kernel (OSDI 2016) and the
associated certified programming tools (POPL 2015, PLDI
2018).  He brings to the collaboration his many years of expertise in
research and development of new programming languages and formal
methods, certified compiler and system software, and new formal
techniques for reasoning about concurrent programs. His recent work on
nominal memory model for verified compilation of C programs (POPL
2022a), Stack-Aware CompCert (POPL 2019), CompCertELF (OOPSLA 2020),
CompCertO (PLDI 2021), game semantics for certified abstraction layers
(LICS 2021 and POPL 2022b), certified abstraction layers (PLDI 2018,
POPL 2015), and security-preserving compilation (PLDI 2016a) will be
highly relevant and can serve as a starting point toward the
end-to-end and compositional verified secure compiler described in
the current proposal. PI Shao will be responsible for the overall
administration and direction of this project. He will focus his
effort on realizing the end-to-end aspect of our verified
compilation effort by integrating Stack-Aware CompCert and new nominal
techniques with CompCertO. 


\paragraph{Co-PI Koenig} is an expert in the area of compiler
verification, refinement-based game semantics, and certified
abstraction layers. He developed the very first certified layer
linking library based on CompCertX (POPL 2015) which is used in all
versions of the CertiKOS kernel. His LICS 2021 paper and PhD
dissertation work developed a novel compositional refinement-based
game semantics for certified heterogeneous components. He designed and
developed CompCertO (PLDI 2021) which is the first compositional
verified C compiler that can support comping and linking certified
{\em open} C and assembly components. PI Koenig will focus his effort
on extending his CompCertO work to support nominal techniques,
verified linking of certified abstraction layers, and
security-preserving compilation.

%\subsection*{\underline{Collaboration}}
\section{Collaboration}

Nearly every component of this project involves tight collaboration
between the two PIs and their students. Each PI brings a unique set of
expertise to this problem, complementing the expertise of the
others. The PIs' preliminary meetings surrounding the creation of this
proposal will serve as a model for how they plan to proceed. Frequent
project meetings will provide a context for overall operations
management. They will also hold many more, smaller meetings with more
specific objectives.

When it comes to the compositional framework design, a theoretical
study of various semantic models and analysis of its potential
application targets (e.g., heterogeneous systems or concurrent OS
kernels) are necessary. Such tasks require joint efforts and expertise
of all PIs from both formal methods and certified system software
fields. The research proposal includes iterative research tasks, which
involve continuous exchange of ideas and understandings of both formal
methods and certified systems, to adjust and find proper logical
models and structures of open systems software and their
verification. The project meetings and jointly advised students will
play the key roles to realize the project artifacts.

One of the primary goals of the proposed research is on the
educational side where we train students to become experts in both
formal methods and certified systems. We plan
to jointly advise students in the following three ways.

First, students will have opportunities to interact with all PIs during
project meetings. Students will be naturally exposed to how 
researchers from different fields approach and 
solve problems and learn from the commonalities and differences. The students
will discuss their ideas, receive feedbacks from all PIs, and learn how to 
become successful in these fields. 

Seconds, we will recommend jointly-advised students to take advanced
courses in formal methods (e.g., software analysis and verification
and language-based security) and systems (e.g., operating systems,
compilers and interpreters). The
courses will include term projects where the students can bring their
own research projects as long as they are relevant to the course.  We
expect the students to learn about advanced knowledge that is
necessary for our proposed research project and gain a hands-on
experience by applying the knowledge to the term project. The term
project can be a part of our proposed research and students will be
able to interact with the PIs through classes.

Finally, in addition to project meetings and classes, the students
will be able to closely interact with the PIs when working on this
project. PI Shao will provide expert advice for general programming
languages and verification research. Co-PI Koenig will advise at the
concrete implementation level (in the Coq proof assistant). The PIs
will use a shared source-code repository for distributed access by all
team members. There will be frequent code reviews and following online
and offline discussions to improve the code.

The success of the research can be directly measured by the ability to
produce the desired artifacts and how well they are picked up by
others. The main artifacts are various versions of the certified
end-to-end and compositional secure compiler which we plan to develop
under this project.  To demonstrate its effectiveness, we plan to
deploy it in multiple different application domains including
the CertiKOS project which would benefit from having such a certified
compiler to build more sophisticated certified OS kernels and hypervisors;
and the DeepSEA programming tool which would benefit from
having such a compiler as its backend. 

Overall, the new verified compiler framework should also greatly
simplify the verification of large-scale system software and their
compositions. The compositional verified compiler should be able to
compile and compose open system components at different abstraction
levels.  The security-preserving refinement proofs should formally
connect different levels and establish end-to-end correctness and
security guarantees.


\end{document}
