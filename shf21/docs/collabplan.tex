\documentclass[11pt]{article}
\usepackage{xspace}
\pagestyle{empty}

\usepackage{vmargin}
\usepackage{color}
\setpapersize{USletter}
\setmarginsrb{25mm}{26mm}{26mm}{18mm}{0mm}{0mm}{7mm}{7mm}

%\title{Collaboration Plan}
%\date{}
%\author{}
\begin{document}
%\maketitle
\thispagestyle{empty}
\section*{\underline{Collaboration Plan}}
%\subsection*{\underline{Expertise Related to the Project}}
\section{Expertise Related to the Project}

\paragraph{PI Shao} is the leader of the CertiKOS team at Yale
which developed both the CertiKOS kernel (OSDI 2016) and the
associated certified programming tools and compilers (POPL 2015, PLDI
2018).  He brings to the collaboration his many years of expertise in
research and development of new programming languages and formal
methods, certified system software, and new formal techniques for
reasoning about concurrent and distributed programs.  His work on
certified concurrent abstraction layers (PLDI 2018, POPL 2015),
information-flow security (PLDI 2016a), certified OS kernels (PLDI
2016b, OSDI 2016), and compositional verification of
termination-preserving refinement of concurrent objects (LICS 2014,
CONCUR 2013) will be highly relevant and can serve as a starting point
toward the design and implementation of the ADVERT framework.

\paragraph{Co-PI Soul\'e} is an expert in data management systems,
storage systems, and networking. 
As a research co-op in the Data Intensive Systems and Analytics Group
at IBM T. J. Watson Research Center, PI Soul\'{e} worked on IBM's distributed
stream processing engine, System S, and its query language,
SPL. Much of his work in this area focused on query
optimization (CSUR 2014) and proving the correctness of
optimizations (ESOP 2010).  In addition to his work on System S, PI
Soul\'{e} has developed techniques to optimize disk block layout for
graph databases (VLDBJ 2016), and algorithms to automatically
transform relational queries into streaming queries (VLDB 10)
Most relevant to this proposal is his work on a policy architecture,
named PADS (NSDI 2009), that allows developers to build
storage systems with tunable consistency levels by specifying routing
and blocking policies. His work on Callinicos (ATC 16) uses a
domain-specific language to declare transactions and to safely and
efficiently execute distributed transactions that span database
partitions.  He has also worked on gossip
protocols (ECOOP 2014) and content distribution
networks (NSDI 2007). He is recipient of a Google Faculty Research Award for work on network
modeling, best paper awards at ACM DEBS 2012 and NSDI 2018, and an IBM
Invention Plateau Award for work on streaming databases.

\paragraph{Co-PI Shin} is an expert in distributed systems and 
cloud storage systems, and at the same time
has expertise in distributed system verification. His PhD work
showed how to design a cloud storage system that can guarantee
performance and transactional isolation under a distributed concurrent
accesses (FAST 2013, FAST 2016) and created a new class of
local storage system that can take advantage of distributed weak
consistency semantics (SoCC 2016). He started working on co-designing
distributed systems and formal verification methods
in recent years and published a paper on a modular distributed
system design and an incremental verification method (SoCC 2019) 
which is highly related to this proposal. 

%\subsection*{\underline{Collaboration}}
\section{Collaboration}

%All Track I projects are required to include a Collaboration Plan submitted by the lead institution as a separate Supplementary Document (limited to 3 pages or less). 
%1) This plan must describe the distinct expertise provided by the PIs as required above under “Who May Serve as PI” ------------DONE------------
%2) as well as plans for working together to advance knowledge in both formal methods and at least one “field” area. ------------FIELD STORY IS MISSING--------------
%3)Joint supervision of students and postdoctoral researchers is strongly encouraged. 
%4) The collaboration plan must also describe clear measures of success for both the formal method and “field” aspects of the project and a plan for evaluating success. ---- MEASURE OF SUCCESS AND EVAL PLAN MISSING ----

Nearly every component of this project involves tight collaboration
between the three PIs and their students. Each PI brings a unique set
of expertise to this problem, complementing the expertise of the
others. PI Shao brings core expertise in the formal verification of
large-scale system software, the development of certified compilers
and proof-assistant languages,  and the design and
implementation of domain-specific program logics for reasoning about
low-level programs. Co-PI Soul\'e brings strong expertise in stream
processing engine, data management engine, and 
content distribution networks, which all involve some form of
distributed systems. His recent industry experience also brings
invaluable expertise to relate our research to up-to-date industry 
practices of building, testing, and verifying distributed systems. 
Co-PI Shin brings strong expertise in both system design and formal
verification and will serve as a bridge to promote strong collaboration.
Shin specializes in designing cloud storage systems and 
distributed system building blocks and developing an approachable
formal verification method from a system designer point of view. 

\subsection{Project meetings}

The PIs' preliminary meetings surrounding the creation of this
proposal will serve as a model for how they plan to proceed. Frequent
project meetings will provide a context for overall operations
management. They will also hold many more, smaller meetings
with more specific objectives.

When it comes to the verification framework design, a theoretical
study of various distributed protocols and analysis of the code-level 
implementation of distributed systems are necessary. Such tasks require
joint efforts and expertise of all PIs from both formal methods and 
distributed system fields. The research proposal includes 
iterative research tasks, which involve continuous exchange 
of ideas and understandings of both formal methods and distributed 
systems, to adjust and find proper logical models and structures
of distributed systems and their verification. The project meetings 
and jointly advised students will play the key roles to realize the 
project artifacts. 

\subsection{Joint Student Advising}

One of the primary goals of the proposed research is on the educational
side where we train students to become experts in both formal
methods and distributed systems. We plan to jointly advise students
in the following three ways. 

First, students will have opportunities to interact with all PIs during
project meetings. Students will be naturally exposed to how 
researchers from formal methods and distributed systems approach and 
solve problems and learn from the commonalities and differences. The students
will discuss their ideas, receive feedbacks from all PIs, and learn how to 
become successful in both fields. 

Seconds, we will recommend jointly-advised students to take advanced 
courses in formal methods (e.g., software analysis and verification and language-based security) 
and distributed systems (e.g., advanced distributed systems and advanced networking), 
which the PIs teach. The courses will include term projects where the students 
can bring their own research projects as long as they are relevant to the course. 
We expect the students to learn about advanced knowledge that is necessary for 
our proposed research project and gain a hands-on experience by applying 
the knowledge to the term project. The term project can be a part of our 
proposed research and students will be able to interact with the PIs through
classes.

Finally, in addition to project meetings and classes, the students will be
able to closely interact with the PIs when developing the verification
framework for distributed systems, writing and evaluating distributed system 
codes, and verifying the code with the framework. PI Shao and co-PI Soul\'e
will provide expert advice for verification code and distributed system code,
respectively. Co-PI Shin will advise at a lower level to structure the verification 
code and the system code so that both pieces can be connected efficiently. 
The PIs will use a shared source-code repository for distributed 
access by all team members. There will be frequent code reviews and following
online and offline discussions to improve the code. 

\section{Success Criteria and Evaluation}

The success of the research can be directly measured by the ability to produce the desired artifacts
and how well they are picked up by others.

\paragraph{Formal Methods} The main artifact is the distributed system verification framework, 
which demonstrates that the new theoretical model and the verification approach are practical and sound.
The framework will consist of four parts: 1) code-level verifier; 2) network-based specification;
3) atomic distributed object (ADO) specification; and 4) multiple ADO reasoning
tools. 

The code-level verifier builds on top of a prior work on the certified 
concurrent abstraction layer research which was led by PI Shao. The code-level 
verifier in our framework should identify reusable tactics and patterns to speed-up 
multiple distributed system verification. The code-level verifier's success can be
evaluated as we verify multiple systems and measure the proof effort (e.g., lines 
of code and time) that takes to verify individual systems. 

A network-based specification is parameterized to model a particular class of 
different distributed systems. The network-based specification will act as a verification 
template to verify the safety property that is exposed in the network level. 
The success of the network-based specifications depends on whether each
specification can verify the corresponding class of system instances.
The system classes include multi-Paxos variants and primary-backup variants.

The ADO specification is intended to model common high-level behaviors of distributed systems
that satisfy the replicated state safety. We intend to iterate through the design so that
the ADO fully represents the system with this safety property while hiding the network-level
details. Then the ADO model will be used to reason about multiple distributed system interactions. 
The ADO model and the multiple ADO reasoning tools should be able to model and verify
horizontal and vertical composition of distributed systems without relying on the low-level detail.

Overall, the entire verification framework should be able to simplify
the verification of individual distributed systems and their compositions. Each component
of the framework should be able to verify distributed system models at a different abstraction level
to be considered successful. The refinement proofs should formally connect different levels
and the ADO and the level above should be oblivious of implementation details. We plan to verify 
different classes of distributed systems from the code level to measure the proof effort and 
applicability of our framework. We plan to compare these metrics against existing approaches. 
Detailed case studies that verify known composite distributed systems and an instance of distributed 
shared memory system will evaluate each component of the framework and the entire framework as a whole.


\paragraph{Fields} Being able to verify multiple individual distributed systems using parameterized 
network-based specifications and the system composition with the ADO model will be a great 
indication of success: these capabilities will significantly simplify distributed system verification
and make the verification much approachable to developers. 
Since formal verification is often considered too time-consuming and impractical for everyday 
distributed system development, use of our research to distributed system development 
cycles would be considered a greater success.

Our framework explains the same distributed system behavior repeatedly at a different 
level of abstraction, and we expect it to promote the understanding of distributed protocols. 
Use of our findings, in addition to the verification framework, by other research 
groups, industrial labs, and educators throughout the world is another criteria that can 
indicate the success of our research project. 
% Section D: References
%\bibliographystyle{abbrv}
%\bibliographystyle{acm}
%\bibliography{biblio,refs,shao,mahesh}

\end{document}
