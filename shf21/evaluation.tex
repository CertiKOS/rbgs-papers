\section{Evaluation and Integration}

To evaluate and demonstrate the power and real-world applicability of
our new verified compiler, we will apply it to enhance existing and
build more advanced certified systems and programming tools.

\vspace*{-2ex}
\paragraph*{Task 5a: Use CompCertO in a heterogeneous context}

In Example~\ref{ex:nicdriver},
we outlined a possible scenario where C components
are made to interact with components of different kinds
to build heterogeneous certified systems.
We will use this approach to
mechanize the verification of a simple heterogeneous system
in the Coq proof assistant,
incorporating certified compilation using CompCertO's correctness theorem.

\vspace*{-2ex}
\paragraph*{Task 5b: Use CompCertO to implement Certified Abstraction Layers}

The original formulation of certified abstraction layers
we used to verify the CertiKOS kernel \cite{popl15}
was very tightly coupled with CompCertX,
an extension of CompCert specifically designed for this application.
By contrast,
our more recent work establishes a theory of certified abstraction layers
within refinement-based game semantics
which is independent of any compiler
or language semantics.
We will show that CompCertO can be interfaced with this new formulation
and used to implement certified abstraction layers within this framework.

\vspace*{-2ex}
\paragraph*{Task 5c: Compiling and composing CCAL layers.}
CertiKOS~\cite{certikos-osdi16} is a certified concurrent OS kernel
that supports fine-grained locking and can boot Linux in its guest VM
running on stock multicore x86 hardware.  CertiKOS is built from a
collection of certified concurrent abstraction layers
(CCAL)~\cite{ccal18} written in C or assembly.  The thread-safe
variant of CompCertX~\cite{dscal15} used by CCAL, however, suffers
from many limitations: it is not end-to-end and does not general
machine code; it does not support concrete machine-level finite
stacks; and it does not support certified linking (and loading) of
user- and kernel-level components. We plan to show that our new
verified compositional compiler can indeed address all these problems
and allow us to build much more powerful OS kernels.

Another target of application is SeKVM~\cite{sekvm21a,sekvm21b,tao21},
a new secure and formally verified Linux KVM Hypervisor developed at
Columbia University. SeKVM uses the same CCAL-based layered technology 
but only does the verification at the C source code level. We plan to
apply our new compositional verified compiler to enhance SeKVM and support
more extensible hypervisors with smaller TCBs and richer features.

%\vspace*{-2ex}
%\paragraph*{Task 5d: Security-preserving compilation for CertiKOS and SeKVM.}
%Both CertiKOS and SeKVM verify the information-flow security of their
%entire systems by first verifying a generalized noninterference
%property (with declassification)~\cite{costanzo16} over the abstract
%(deep) specification of their system-call layers, and then relying on
%security-preserving contextual refinement to propagate the security
%guarantees to cover the entire C and assembly implementation. The
%initial single-core version of CertiKOS~\cite{costanzo16}, in
%particular, used CompCertX~\cite{dscal15} to construct an end-to-end
%security proof all the way to the assembly level.  We will apply our
%new security-preserving compiler to Concurrent CertiKOS and SeKVM and
%build end-to-end security proofs to the binary machine code level.

\vspace*{-2ex}
\paragraph*{Task 5d: Connecting and enhancing certified programming tools.}
The process of manually connecting the C and assembly semantics from
CompCert with the Certified Abstraction Layers framework in the
original development of CertiKOS led to the development of the DeepSEA
programming language~\cite{deepsea19}.  The DeepSEA code is compiled
into C (and then compiled to assembly using CompCert) plus a deep
specification of its behavior in Coq and an automatically generated
proof of refinement between them. The DeepSEA platform has been
revamped so that it now follows the compositional semantics described
in \S\ref{sec:compcerto}. We will connect our new compositional
verified compiler with DeepSEA and enhance DeepSEA with better support
to concurrency and name-space management (based on the nominal techniques).

The verified software toolchain VST~\cite{appel11:vst}) is another
certified programming tool that supports a rich separation logic for C
and also connects with CompCert. Connecting VST with our end-to-end
compositional verified compiler would allow user-level programs
verified using VST to be compiled into ELF object files and loaded
into the certified OS kernel to establish an end-to-end assurance case
that covers both the OS kernel and user-level applications.

