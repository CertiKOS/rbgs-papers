\documentclass[acmsmall,timestamp,review]{acmart}

% Packages {{{
\usepackage{hyperref}
\usepackage{amsmath}
\usepackage{amssymb}
\usepackage{bbm}
\usepackage{tikz}
\usetikzlibrary{calc}
\usetikzlibrary{graphs}
\usetikzlibrary{cd}
\usepackage{bussproofs}
\usepackage{stmaryrd}
\usepackage{calc}
\usepackage{bbm}
% }}}

% Parameters {{{

\hyphenation{Comp-Cert}

% }}}

% Macros {{{

%\newtheorem{example}{Example}
%\newtheorem{definition}{Definition}
%\newtheorem{lemma}{Lemma}

\newcommand{\kw}[1]{\ensuremath{ \mathsf{#1} }}
\newcommand{\ifr}[1]{\ [{#1}]\ }
\newcommand{\ifrw}[2]{\ [{#1}]_{#2}\ }
\newcommand{\alt}{\ |\ } % use \mid instead
\newcommand{\bind}{\gg\!\!=}

\newcommand{\EC}{\kw{C}}
\newcommand{\simrel}{\kw{simrel}}

% Moves
\newcommand{\mcall}[3]{\kw{#1}({#2})@{#3}}
\newcommand{\pcall}[3]{%
  \underline{\mcall{#1}{#2}{#3}}%
}
\newcommand{\mret}[2]{{#1}@{#2}}
\newcommand{\pret}[2]{%
  \underline{\mret{#1}{#2}}%
}
\newcommand{\mretx}[3]{{#1}@{#2}/{#3}}
\newcommand{\pretx}[3]{%
  \underline{\mretx{#1}{#2}{#3}}%
}

% Pointers for justified sequences %{{{

% Parameters
\newcommand{\pshift}{1.6ex}
\newcommand{\pcdist}{2.5}
\newcommand{\pcangle}{60}

% Pointer hook
\newcommand{\ph}[1]{%
  \tikz[remember picture]{\coordinate (#1);}}

% Pointer to
\newcommand{\pt}[1]{%
  \rule{0pt}{1.4em}%
  \tikz[remember picture, overlay]{
    \draw[->]
      let \p{dest} = (#1),
          \n1 = {ln(veclen(\x{dest}, \y{dest}) + 1)},
          \p1 = ($(0,0)+(0,\pshift)$),
          \p4 = ($(#1)+(0,\pshift)$),
          \p2 = ($(\p1)!\n1*\pcdist!-\pcangle:(\p4)$),
          \p3 = ($(\p4)!\n1*\pcdist!+\pcangle:(\p1)$) in
        (\p1) .. controls (\p2) and (\p3) .. (\p4);}}

%}}}

% }}}

\title{Refinement-Based Game Semantics for CompCert}

\author{J\'er\'emie Koenig}
\affiliation{Yale University}
\email{jeremie.koenig@yale.edu}

\author{Zhong Shao}
\affiliation{Yale University}
\email{zhong.shao@yale.edu}

\begin{document}

\maketitle

\section{Introduction} %{{{

Our goal is end-to-end verification of computer systems.

\subsection{Scaling Up Verification} %{{{

%}}}

\subsection{Certified Compilers} %{{{

Compilers are essential component;
quitessential tool for working across abstraction layers.
On one hand good "litmus test" for framework
of how certified compilers fit into the picture.
On this other hand suggests thinking about not just
"certified compilers" but "compilers of certified components".

%}}}

\subsection{Refinement-Based Game Semantics} %{{{

Mix of refinement, abstraction, game semantics
is how we should go about this.

%}}}

\subsection{Logical Relations} %{{{

%}}}

\subsection{Contributions} %{{{

\begin{itemize}
\item Present a conceptual framework for thinking about
  the construction of large-scale, end-to-end verified systems;
\item Show how ideas from refinement-based verification,
  abstraction layers, game semantics, relators
  can be synthesized
  to build a compositional semantics for CompCert open modules;
\item In particular, expressivity of abstraction helps us
  update compiler correctness proof
  is a more conservative way,
  improving on Compositional CompCert.
\item New model for interaction that resolves the tension between
  operational and trace semantics
  by using monad.
\item[+] [Logical relations, verified in Coq]
\end{itemize}

%}}}

%}}}

\section{Main Ideas} %{{{

[XXX update and trim down for the reduced scope of this paper]

\subsection{Principles for system construction}

The goal of certified system design is
to create a formal description of
the system to be constructed (the program),
while ensuring through careful analysis that the system
will behave properly.

To carry out this analysis,
we need a model
which assigns to each possible system description $p \in P$
a mathematical object $\llbracket p \rrbracket \in \mathbb{D}$
representing its behavior.
We will call the set $\mathbb{D}$ a \emph{semantic domain}.
In this section we elucidate
the structure and properties of $\mathbb{D}$
necessary to the process of builing
large-scale certified systems.

\subsubsection{Specifications and refinement}

System design starts with a set of requirements
that constrain the behavior of the system to be constructed
(the specification).
These requirements may not capture every detail
of the behavior of the eventual system,
but instead delineate the range of acceptable behaviors
from the point of view of its environment.

In the context of refinement-based methods,
the specifications are themselves elements of
the semantic domain $\mathbb{D}$,
which is equipped with a \emph{refinement} preorder
${\sqsubseteq} \subseteq \mathbb{D} \times \mathbb{D}$.
The proposition $\sigma_1 \sqsubseteq \sigma_2$
assert that $\sigma_2$ is a less restrictive specification than $\sigma_1$,
and in particular a system description $p \in P$ is a correct implementation
of $\sigma \in \mathbb{D}$ when:
\[ \llbracket p \rrbracket \sqsubseteq \sigma \]

\subsubsection{Compositionality}

Complex systems are built by assembling components
whose behavior is understood,
such that their interaction achieves a desired effect.
The syntactic constructions of
the language used to describe systems
correspond to the ways in which they can be composed.

To enable compositional reasoning,
a suitable model must provide an account of
the behavior of the composite system
in terms of the behavior of its parts.
For instance,
if the language contains a binary operator
${+} : P \times P \rightarrow P$,
then the semantic domain should be equipped with
a corresponding operation
${\bullet} : \mathbb{D} \times \mathbb{D} \rightarrow \mathbb{D}$
such that:
\[ \llbracket p_1 + p_2 \rrbracket =
   \llbracket p_1 \rrbracket \bullet \llbracket p_2 \rrbracket \,. \]

We can use this property
to show that a composite system
satisfies a given specification,
for instance
$\llbracket p_1 + p_2 \rrbracket \sqsubseteq \sigma$.
Once this has been established,
we will want to abstract the composite system as a ``black box''
which can in turn be used as a primitive component in a larger system.
Further reasoning can be done in terms of
the new component's specification $\sigma$ rather than
its concrete behavior $\llbracket p_1 + p_2 \rrbracket$.
To support this,
we must establish that semantic composition operators
are compatible with refinement in the following sense:
\[ \sigma_1 \sqsubseteq \sigma_1' \wedge
   \sigma_2 \sqsubseteq \sigma_2' \Rightarrow
   \sigma_1 \bullet \sigma_2 \sqsubseteq \sigma_1' \bullet \sigma_2' \,. \]
Should $p_1 + p_2$ be used as a component in a larger system,
this property will allow us to establish that:
\[ \llbracket (p_1 + p_2) + p_3 \rrbracket \sqsubseteq
   \sigma \bullet \llbracket p_3 \rrbracket \,. \]

\subsubsection{Abstraction}

Large-scale systems operate across many layers of abstraction.
Each abstraction layer defines its own understanding of the interaction
between the system and its environment.
To relate abstraction layers we will need to give an explicit account
of how their formulations of the interaction correspond to one another.

For example,
at the physical level,
communication over a wire may involve a series of voltages through time,
but at a higher level of abstraction
we will only be concerned with the sequence of bytes
being transmitted.
Serial communication hardware (for instance a UART device)
serves as a bridge between these two views,
implementing a byte-oriented communication channel
in a voltage-oriented world;
to formally verify such a component we will need to
express the correspondance between these two
layers of abstraction.

This can be done by defining an embedding
$\mathbb{C} : \mathbb{D}_s \rightarrow \mathbb{D}_t$
between the semantic domains
used to reason at different levels of abstraction.

[...]

\subsubsection{Compilers}


\subsubsection{Heterogeneity}

[What happens to $\mathbb{D}$ in the context of typed languages,
how this can be used to gain expressivity and construct heterogenous systems.]


\subsection{Game semantics}

[introductory remarks]

Types are interpreted as two-player games,
which specify the form of valid interactions
between the system and the environment.
Terms are interpreted as strategies for these games,
which for any position in the game
specify the next move to be played by the system.

\subsubsection{Games}

A game is usually defined by giving a set of moves $M$
that players can choose from,
as well as a specification of which
sequences of moves are considered valid.
For chess,
moves will be taken in the set $(\{a, \ldots, h\} \times \{1, \ldots, 8\})^2$,
and a sequence of moves may look like:
\[ e2e4 \cdot \underline{c7c5} \cdot c2c3 \cdot \underline{e7d5} \cdots \]
For a program expected to produce a natural number,
the possible moves may be a request to evaluate ($\kw{q}$)
and primitive values ($42$),
so that an interaction between the expression $21 \times 2 : \kw{nat}$
and its evaluation context will proceed as:
\[ \kw{q} \cdot \underline{42} \]
Note that we underline the moves of the second player,
namely the system.

We will focus on two-player, alternating games
where the environment and the system each contribute every other move.
The environment plays first.
Positions in the game can be given
as sequences of moves;
even-length sequences correpond to positions
where the environment is expected to move,
and odd-length sequences correspond to positions
where the system is expected to move.

\subsubsection{Strategies}

In this context,
it is natural to think of system strategies
as functions from odd-length positions to moves:
\[ \sigma : \bigcup_n M^{2n+1} \rightharpoonup M \]
For example,
the strategy $\llbracket 21 \times 2 \rrbracket$
will be the partial function with a single mapping $\kw{q} \mapsto 21$,
whereas the black player in the chess play above
has used a strategy $\sigma$ containing:
\[
  \sigma(e2e4) = c7c5 , \quad
  \sigma(e2e4 \cdot c7c5 \cdot c2c3) = e7d5 , \quad
  \ldots
\]

This is suitable to express the behavior of a concrete system.
However, for specifications allowing a range of behavior,
we must generalize strategies to allow nondeterminism:
\[ \sigma : \bigcup_n M^{2n+1} \rightarrow \mathcal{P}(M) \]
This is isomorphic to
$\mathcal{P}(\bigcup_n M^{2n+2})$
and corresponds to the usual encoding of strategies
as sets of traces.

Equivalently,
a strategy can be given as a
nonbranching
\emph{alternating transition system},
with environment moves performing transitions from
a set of continuations $K$ to a set of states $S$,
and conversely
transitions from a state $S$ to a continuation $K$
emitting a system move.
We obtain a strategy by
specifying an initial continuation and
abstracting over $K$ and $S$:
\[ \sigma : \exists \, K S \,.\, K \times
      (K \times M \rightharpoonup S) \times
      (S \times M \rightharpoonup K) \]
Two strategies will be equivalent
when simulations can be established between them.

Any strategy given as a transition system
defines a partial mapping between the positions in the game
where the system is to play
and the set of states $S$ used by the strategy,
which can then be queries for the set of possible
system moves associated with the position.
The transition-system presentation of strategies
is less standard in existing game semantics literature
but is easier to relate to operational semantics
and may be easier for formalize.

\subsubsection{Compositionality}

While the game for $\kw{nat}$ given above
is extremely simple,
the expressive power of game semantics
comes from the way in which complex games can be derived
from simple ones to interpret compound types.

For instance,
given two games $A$ and $B$,
the game $A \otimes B$ is traditionally defined
to allow $A$ and $B$ to be played side by side.
The set of moves will be $M_{A \otimes B} = M_A + M_B$,
and for any valid position of $A \otimes B$,
the subsequence consisting of the moves of a constituent game
is expected to be valid in that game.
An interaction in the game
$\llbracket \kw{nat} \rrbracket \otimes
 \llbracket \kw{nat} \rrbracket$ could be:
\[ i_2(q) \cdot \underline{i_2(42)} \cdot
   i_1(q) \cdot \underline{i_1(7)} \]

There are two other important constructions on games.
In the game $A \multimap B$,
we play the games $A$ and $B$ simultaneously,
but in the game $A$ we play the role of the environment.
In other words,
a strategy for $A \multimap B$
primarily implements the game $B$,
but can also rely on an implementation of $A$.
Finally,
the game $!A$
consists of multiple copies of the game $A$,
which are instantiated at the discretion of the environment.

These constructions can be used together
to express the structure of various kinds of interactions.
For instance,
the type $!A \multimap B$ is traditionally used
to model the behavior of a $\lambda$-term of type $A \rightarrow B$,
which can access its argument multiple times.
In the context of CompCert,
the game $!C \multimap C$
represents the way a C module behaves in response to an activation
(as specified by the game $C$),
while being able to perform any number of external calls ($!C$),
and corresponds to the \emph{interaction semantics} defined in
\cite{compcompcert}.
In this work,
we will extend this to the game $!C \multimap !C$,
which will allow components to maintain
persitent, hidden state across activations
and express the behavior of reentrant calls.

\subsubsection{Refinement}

[alternating refinement, $\lightning$, etc]

\subsubsection{Abstraction}

This concept of refinement can be extended to account for
different levels of abstraction.


[etc]

The polarisation inherent in game semantics
can be seen to naturally encode the rely-guarantee aspects
of the C calling convention.

\subsection{Logical relations}

Given an algebraic structure $\mathcal{S}$
involving a number of operations over a carrier set,
a \emph{logical relation}
between two instances $S_1, S_2$ of $\mathcal{S}$
will be a relation $R \subseteq |S_1| \times |S_2|$
between their carrier sets,
such that the operations of $\mathcal{S}$
take related arguments to related results.
We write $R : \mathcal{R}(S_1, S_2)$.

\begin{example}[Logical relation of monoids]
\label{ex:monoid}
A \emph{monoid} is a set $A$ equipped with
an associative binary operation $\cdot$ and
an identity element $\epsilon$.
A \emph{logical relation of monoids} between
a monoid $\langle A, \cdot_A, \epsilon_A \rangle$ and
a monoid $\langle B, \cdot_B, \epsilon_B \rangle$
is a relation $R \subseteq A \times B$
such that:
\begin{gather*}
u \ifr{R} u' \wedge v \ifr{R} v' \Rightarrow u \cdot_A v \ifr{R} u' \cdot_B v' \\
\epsilon_A \ifr{R} \epsilon_B
\end{gather*}
\end{example}

Logical relations between multisorted structures
will include one relation for each sort,
between the corresponding carrier sets.
In the case of structures which include type operators,
we can associate to each base type $A$
a relation over its carrier set $\llbracket A \rrbracket$,
and to each type operator $T(A_1, \ldots, A_n)$
a corresponding \emph{relator}:
given relations $R_1, \ldots, R_n$ over
the carrier sets $\llbracket A_1 \rrbracket, \ldots, \llbracket A_n \rrbracket$,
the relator for $T$
will construct a relation $T(R_1, \ldots, R_n)$
over $\llbracket T(A_1, \ldots, A_n) \rrbracket$.

\begin{figure} % fig:relators {{{
  {\small
  \begin{align*}
    x \ifr{R_1 \times R_2} y \ \Leftrightarrow\  &
      \pi_1(x) \ifr{R_1} \pi_1(y) \wedge
      \pi_2(x) \ifr{R_2} \pi_2(y) \\
    x \ifr{R_1 + R_2} y \ \Leftrightarrow\  &
      (\exists \, x_1 \, y_1 \,.\,
        x_1 \ifr{R_1} y_1 \wedge
        x = i_1(x_1) \wedge
        y = i_1(y_1)) \\ \vee\ &
      (\exists \, x_2 \, y_2 \,.\,
        x_2 \ifr{R_2} y_2 \wedge
        x = i_2(x_2) \wedge
        y = i_2(y_2)) \\
    f \ifr{R_1 \rightarrow R_2} g \ \Leftrightarrow\  &
      \forall \, x \, y \,.\,
        x \ifr{R_1} y \Rightarrow
        f(x) \ifr{R_2} g(y) \\
    A \ifr{\mathcal{P}^+(R)} B \ \Leftrightarrow\  &
      \forall \, x \in A \,.\,
      \exists \, y \in B \,.\,
      x \ifr{R} y \\
    A \ifr{\mathcal{P}^-(R)} B \ \Leftrightarrow\  &
      \forall \, y \in B \,.\,
      \exists \, x \in A \,.\,
      x \ifr{R} y
  \end{align*}
  }%
  \caption{A selection of relators}
  \label{fig:relators}
\end{figure}
%}}}

Relators for some common constructions are shown in Fig.~\ref{fig:relators}.
Note that the first requirement given in Ex.~\ref{ex:monoid}
can be expressed as:
\[
  \cdot_A \ifr{R \times R \rightarrow R} \cdot_B
\]

Logical relations used to establish contextual equivalence
are often partial equivalence relations (PER);
by contrast, our focus is refinement,
so that most of the relations we consider will not be symmetric.

\subsection{Kripke logical relations} %{{{
\label{sec:klr}

The ways in which the components of a complex structure should be related
are not always independent.
To preserve compositionality while ensuring that
components are related in a consistent way,
we can parametrize a logical relation
over a set $W$ of \emph{possible worlds}.
Components related at the same world will be related
in compatible ways.

\begin{definition}
For a given set $W$,
a \emph{Kripke logical relation} is
$W$-indexed family of logical relations $(R_w)_{w \in W}$.
We write $R : \mathcal{R}_W(S_1, S_2)$
for a Kripke logical relation between structures $S_1$ and $S_2$,
and define the following relations:
\begin{align*}
  x \ifr{w \Vdash R} y &\Leftrightarrow x \ifr{R_w} y \\
  x \ifr{\Vdash R} y &\Leftrightarrow \forall w \,.\, x \ifr{R_w} y \,.
\end{align*}
\end{definition}

In the remainder of this section,
we introduce some useful constructions on
Kripke logical relations.

\subsubsection{Kripke relators}

A logical relation $R : \mathcal{R}(A, B)$
can be promoted to a $W$-indexed Kripke logical relation $\lceil R \rceil$
which ignores the index, so that $\lceil R \rceil_w = R$.
Likewise,
a relator
  $F : \mathcal{R}(A_1, B_1) \,\times\,\cdots\,\times\,\mathcal{R}(A_n, B_n) \rightarrow \mathcal{R}(A, B)$
can be promoted to its Kripke version
by pointwise extension over the set of possible worlds:
\begin{gather*}
  \lceil F \rceil : \mathcal{R}_W(A_1, B_1) \times \cdots \times \mathcal{R}_W(A_n, B_n) \rightarrow \mathcal{R}_W(A, B) \\
  \lceil F \rceil (R_1, \ldots, R_n)_w = F(R_{1,w}, \ldots, R_{n,w})
\end{gather*}
We use $\lceil - \rceil$ implicitly
when a relator appears in a context where
a Kripke logical relation is expected.

\subsubsection{Modalities}

Next we define Kripke relators
allowing us to [move between possible worlds].

\begin{definition}
A \emph{Kripke frame}
is a labelled transition system
$\langle W, (\stackrel{l}{\leadsto})_{l \in \Lambda} \rangle$, where
$W$ is a set of \emph{possible worlds},
$\Lambda$ is a set of labels, and
$(\stackrel{l}{\leadsto})_{l \in \Lambda}$ is a family of
binary \emph{accessibility relations} over $W$.
\end{definition}

For a Kripke frame
$\langle W, (\stackrel{l}{\leadsto})_{l \in \Lambda} \rangle$,
we define the Kripke relators $\langle l \rangle$, $[l]$ as follows:
\begin{align*}
  x \ifr{w \Vdash \langle l \rangle R} y & \: \Leftrightarrow \:
    \exists \, w' \,.\, w \stackrel{l}{\leadsto} w' \wedge
      x \ifr{w' \Vdash R} y \\
  x \ifr{w \Vdash [ l ] R} y & \: \Leftrightarrow \:
    \forall \, w' \,.\, w \stackrel{l}{\leadsto} w' \Rightarrow
      x \ifr{w' \Vdash R} y
\end{align*}
In the context of an unlabelled Kripke frame ($\Lambda = \{ * \}$),
we will write $\langle * \rangle$ as $\Diamond$ and
$[ * ]$ as $\Box$.

\begin{example}[Simulation diagram]
\label{ex:sim}
Consider a Kripke logical relation of sets $R : \mathcal{R}_W(A, B)$,
and two transition relations $\alpha : A \rightarrow \mathcal{P}(A)$
and $\beta : B \rightarrow \mathcal{P}(B)$.
The simulation diagram:
\[
  \begin{tikzcd}
    s_1 \arrow[r, "\alpha"]
        \arrow[d, dash, "R_w"'] &
    s_1' \arrow[d, dotted, dash, "R_{w'} \quad (w \leadsto w')"] \\
    s_2 \arrow[r, dotted, "\beta"] &
    s_2'
  \end{tikzcd}
\]
can be written as:
\[
  \alpha \ifr{\Vdash R \rightarrow \mathcal{P}^+(\Diamond R)} \beta \,.
\]
\end{example}

%\subsection{CompCert}

%}}}

%}}}

\section{The Interaction Monad} \label{sec:monad} %{{{

% preamble {{{

The traditional representation of strategies as
prefix-closed sets of traces
offers many advantages:
it is simple;
the ordering properties of down-set lattices
are straightforward and well-understood;
as a trace semantics
it is well-suited as a fully abstract description
of a system's external behavior.

On the other hand,
the transition systems
used to define the semantics of CompCert languages
correspond more closely to the internal operation of the system
and lend themselves better
to operational resoning and intuition,
but phenomena such as branching and divergence
make it harder to pin down a single notion of equivalence \cite{blts}
and in some cases complicates the corresponding definitions of simulations.

The \emph{interaction monad} $\mathcal{I}_{M,N}(-)$
defined in this section offers advantages of both of these models.
Following traditional work on game semantics,
we use a simple trace model to formalize
the behavior of interactive components.
However,
the accompanying monadic structure
introduces a notion of sequential composition
which makes it possible to define strategies
in an operational way.
We present this approach in \S\ref{sec:monad:def}--\S\ref{sec:monad:iter}.

The interaction monad,
can be used to formalize various notions of alternating strategies.
In \S\ref{sec:monad:games},
we show a simple encoding of the
innocent, well-bracketed strategies
of a specific class of games,
we we will use as the foundation for
the semantic model presented in \S\ref{sec:modsem}.

%}}}

\subsection{Overview} \label{sec:monad:overview} %{{{

The interaction monad specifies
the behavior of a system which interacts with its environment.
At any point,
the system can perform an output $m \in M$ and
wait for a subsequent input $n \in N$ from the environment.
This is modelled by the operation:
\[
    \kw{interact} : M \rightarrow \mathcal{I}_{M,N}(N) \,.
\]

Additionally,
to accomodate specifications which permit a range of possible behaviors,
the interaction monad is equipped with a complete refinement lattice.
Given $x, y : \mathcal{I}_{M,N}(A)$,
their supremum $x \vee y : \mathcal{I}_{M,N}(A)$
is the smallest specification that permits the behavior of either;
it can also be interpreted as non-deterministic choice
of the system.
Conversely, $x \wedge y : \mathcal{I}_{M,N}(A)$ can be interpreted as
the largest specification requiring that $x$ and $y$ both be satisfied.
The least element $\bot$
is a specification that can never be satisfied;
the greatest element $\top$
is the specification that is always satisfied ---
or represents a computation whose behavior is entierely undefined.

Finally,
we model non-deterministic iteration with the operator:
\[
     -^\infty : (A \rightarrow \mathcal{I}_{M,N}(A)) \rightarrow
                (A \rightarrow \mathcal{I}_{M,N}(A)) \,.
\]
Notably,
$-^\infty$ is different from
the Kleene star associated with the refinement lattice,
because we account for silent divergence as a specific behavior,
incomparable with terminating ones,
rather than identifying it with
the unsatisfiable specification $\bot$
or the undefined behavior $\top$.

%}}}

\subsection{Definition} \label{sec:monad:def} %{{{

Following the traditional game semantics approach,
we formalize interactive behaviors as prefix-closed sets of plays.
We do not restrict the move available to $\kw{O}$,
so the plays we will consider are odd-length sequences of moves.
To support the monadic structure and account
for silent divergence and undefined behaviors,
we augment the set of outputs with the
distinguished terminal moves $v \in X$, $\Delta$, and $\lightning$:
\[
    s, t \in
    \mathcal{T}_{M,N}(X) ::=
    v \mid \Delta \mid \lightning \mid m \mid mnt \,.
\]
Any trace is considered a prefix of $\lightning$,
so that our extended prefix relation on $\mathcal{T}_{M,N}(X)$
is defined by the rules:
\begin{gather*}
  v \sqsubseteq v , \quad
  \Delta \sqsubseteq \Delta , \quad
  t \sqsubseteq \lightning , \quad
  m \sqsubseteq m , \quad
  m \sqsubseteq mnt ,
  \quad
  \AxiomC{$s \sqsubseteq t$}
  \UnaryInfC{$mns \sqsubseteq mnt$}
  \DisplayProof
\end{gather*}
An interactive behavior is
a prefix-closed set of traces:
\[
    \mathcal{I}_{M,N}(X) :=
    \{ T \subseteq \mathcal{T}_{M,N}(X) \mid
       \forall t \in T \,.\, \forall s \sqsubseteq t \,.\, s \in T \}
\]

Note that since any trace is a prefix of $\lightning$,
a behavior which admits a trace ending with $\lightning$
will also admit all possible interactions
sharing the same initial segment.
This allows us to define our notion of refinement
as simple trace containment.
For $x, y \in \mathcal{I}_{M,N}(X)$, refinement is defined as:
\[
    x \sqsubseteq y \Leftrightarrow x \subseteq y
\]
Since unions and intersections
preserve prefix closure,
they induce a lattice structure on $\mathcal{I}_{M,N}(X)$.

%}}}

\subsection{Monad operations} %{{{

The monad's unit associates to each value $v \in X$
the computation with a single trace $v$:
\[
    \kw{ret}_X(v) := \{ v \} \,.
\]
The binding operation corresponds to
the sequential composition of
a behavior $x \in \mathcal{I}_{M,N}(A)$ and
a function $f : A \rightarrow \mathcal{I}_{M,N}(B)$.
The result is an interactive behavior in $\mathcal{I}_{M,N}(B)$ which
contains the traces of $x$ where
any final value $v$ has been replaced with
all possible traces in $f(v)$.
For a single trace $t \in x$ we define $t \bind f$
using the following rules:
\begin{gather*}
  \AxiomC{}
  \UnaryInfC{$\Delta \in (\Delta \bind f)$}
  \DisplayProof
  \quad
  \AxiomC{}
  \UnaryInfC{$t \in (\lightning \bind f)$}
  \DisplayProof
  \quad
  \AxiomC{}
  \UnaryInfC{$m \in (m \bind f)$}
  \DisplayProof
  \\[1ex]
  \AxiomC{$t \in f(v)$}
  \UnaryInfC{$t \in (v \bind f)$}
  \DisplayProof
  \quad
  \AxiomC{$s \in (t \bind f)$}
  \UnaryInfC{$mns \in (mnt \bind f)$}
  \DisplayProof
\end{gather*}
Then $x \bind f$ can be defined as:
\[
    (x \bind f) = \bigcup_{t \in x} (t \bind f)
\]
It is straightforward to verify that
the monad laws hold:
\begin{align*}
  (\kw{ret}(v) \bind f) &= f(v) \\
  (x \bind \kw{ret}) &= x \\
  (x \bind (v \mapsto f(v) \bind g)) &= ((x \bind f) \bind g) \,.
\end{align*}

%}}}

\subsection{Kleisli morphisms} \label{sec:monad:kmor} %{{{

While interactive behaviors $x : \mathcal{I}_{M,N}(A)$
represent active computations,
the monad's Kleisli morphisms $f : A \rightarrow \mathcal{I}_{M,N}(B)$
can be interpreted as \emph{interactive continuations}:
suspended computations which can be resumed by an input $v : A$.
Kleisli morphisms can be used to
model strategies for games in which $\kw{O}$ plays first,
as demonstrated in \S\ref{sec:monad:games}.
In addition,
a Kleisli morphism
$f : A \rightarrow \mathcal{I}_{M,N}(A)$
with identical argument and result types
can be interpreted as an \emph{interactive transition relation},
as demonstrated in \S\ref{sec:monad:powerset}.

The binding operation $\bind$
and the lattice structure on $\mathcal{I}_{M,N}(A)$
can be extended to Kleisli morphisms as follows:
\begin{align*}
    (f \cdot g)(a) &:= f(a) \bind g &
    \mathbf{1}(a) &:= \kw{ret}(a) \\
    (f \oplus g)(a) &:= f(a) \vee g(a) &
    \mathbf{0}(a) &:= \bot
\end{align*}
Together with the iteraction principles
defined in the following section,
these operations constitute a structure
analogous to a typed Kleene algebra \cite{tka}.

%}}}

\subsection{Iteration} \label{sec:monad:iter} %{{{

A Kleisli morphism $f : A \rightarrow \mathcal{I}_{M,N}(A)$
can be iterated as follows.
We start by defining
the $n$-fold sequential composition of $f$
and the associated notion of Kleene star as:
\begin{align*}
    f^0(a) &:= \kw{ret}(a) \\
    f^{n+1}(a) &:= f^n(a) \bind f \\
    f^*(a) &:= \bigcup_n f^n(a) \,.
\end{align*}
In order to recognize silent divergence,
we introduce $f_\Delta$,
defined by the following coinductive rule.
Note that $f_\Delta(v) \sqsubseteq \{\Delta\}$.
\[
    \AxiomC{$v' \in f(v)$}
    \AxiomC{$\Delta \in f_\Delta(v')$}
    \doubleLine
    \BinaryInfC{$\Delta \in f_\Delta(v)$}
    \DisplayProof
\]
Then the non-deterministic, infinite iteration of $f$ is:
\[
    f^\infty := (f \oplus f_\Delta)^*
\]

%}}}

\subsection{Sets and Functions} \label{sec:monad:powerset} %{{{

The powerset monad $\mathcal{P}$
can be embedded into the monad $\mathcal{I}_{M,N}$
using the natural transformation
$\eta^\mathcal{P}_X : \mathcal{P}(X) \rightarrow \mathcal{I}_{M,N}(X)$
defined as:
\[
    \eta^\mathcal{P}_X(V) := \{ v \mid v \in V \}
\]
In particular,
a relation $R : A \rightarrow \mathcal{P}(B)$
can be interpreted as the Kleisli morphism
$\eta^\mathcal{P}_B \circ R : A \rightarrow \mathcal{I}_{M,N}(B)$.

\begin{example} \label{ex:ts} % Transition system {{{
Consider a transition system $\alpha = (S, I, {\rightarrow}, F)$,
where
$S$ is a set of states,
$I : \mathcal{P}(S)$
is a set of initial states,
${\rightarrow} : S \rightarrow \mathcal{P}(S)$
is a transition relation, and
$F : S \rightarrow \mathcal{P}(A)$
associate potential output values to each state.
The behavior of $\alpha$ can be expressed as:
\[
    \llbracket \alpha \rrbracket :=
    \eta^\mathcal{P}_S(I) \bind
    (\eta^\mathcal{P}_S \circ {\rightarrow})^\infty \bind
    (\eta^\mathcal{P}_S \circ F)
    : \mathcal{I}_{M,N}(A) \,.
\]
\end{example}
%}}}

We use a similar pattern in \S\ref{sec:modsem:def}
to express the behavior of CompCert small-step semantics
in terms of interactive behaviors.
For conciseness,
we will sometimes rely implicitely on $\eta_X^\mathcal{P}$
when a set $x : \mathcal{P}(A)$ is used
in a context where an interactive behavior
of type $\mathcal{I}_{M,N}(A)$ is expected,
for instance expressing $\llbracket \alpha \rrbracket$ above as
$I \bind {\rightarrow}^\infty \bind F$.

Another application of $\eta^\mathcal{P}$
is its use for computing the \emph{preimage} of a function
$f : A \rightarrow B$ as the Kleisli morphism
$f^{-1} : B \rightarrow \mathcal{I}_{M,N}(A)$ defined by:
\[
    f^{-1}(b) = \eta^\mathcal{P}_A(\{ a \mid f(a) = b \}) \,.
\]
For symmetry we will sometimes implicitely promote
$f : A \rightarrow B$ to the Kleisli morphism
$\kw{ret} \circ f : A \rightarrow \mathcal{I}_{M,N}(B)$,
so that for instance given two Kleisli morphisms $f$ and $g$,
their coproduct in the Kleisli category can be expressed as:
\[
    f + g = (i_1^{-1} \cdot f \cdot i_1) \oplus (i_2^{-1} \cdot g \cdot i_2) \,.
\]

%}}}

\subsection{Interaction} %{{{

The interaction primitive
$\kw{interact} : M \rightarrow \mathcal{I}_{M,N}(N)$
can be defined as follows:
\[
    \kw{interact}(m) := \{ m, mnn \mid n \in N \}
\]
Note that in the trace $mnn$,
the first occurence of $n$ denotes an input,
whereas the second one denotes the value returned by $\kw{interact}$.

To eliminate interaction,
we can use the the following operator to
to ``silently unfold'' the next action of a behavior $x$:
\[
    \kw{next} :
       \mathcal{I}_{M,N}(X) \rightarrow
       \mathcal{I}_{P,Q}(X + M \times (N \rightarrow \mathcal{I}_{M,N}(X)))
\]
When $x$ diverges or goes wrong, so does $\kw{next}(x)$.
If $x$ returns a value $a$, then $\kw{next}(x)$
has a similar behavior:
\[
    \kw{next}(\kw{ret}(a)) = \kw{ret}(i_1(a))
\]
However,
if $x$ attempts to interact,
then $\kw{next}(x)$ will instead
return the output $m \in M$ attempted by $x$,
as well as a continuation $N \rightarrow \mathcal{I}_{M,N}(X)$
which can be used to resume the evaluation of $x$
on a subsequent input $n \in N$:
\[
    \kw{next}(\kw{interact}(m) \bind f) = \kw{ret}(i_2(m), f) \,.
\]

\begin{definition}[Silent unfolding]
For an interactive behavior $x : \mathcal{I}_{M,N}(A)$,
the behavior $\kw{next}(x)$ is defined by the rules:
\begin{gather*}
    \AxiomC{$v \in x$}
    \UnaryInfC{$i_1(v) \in \kw{next}(x)$}
    \DisplayProof
    \quad
    \AxiomC{$\Delta \in x$}
    \UnaryInfC{$\Delta \in \kw{next}(x)$}
    \DisplayProof
    \quad
    \AxiomC{$\lightning \in x$}
    \UnaryInfC{$\lightning \in \kw{next}(x)$}
    \DisplayProof
    \\[1ex]
    \AxiomC{$m \in x$}
    \UnaryInfC{$i_2(m, n \mapsto \delta_{mn}(x)) \in \kw{next}(x)$}
    \DisplayProof
\end{gather*}
where $\delta_{mn}(x) := \{ t \mid mnt \in x \}$.
\end{definition}

Interaction introduces a second notion of composition
besides the one induced by the $\bind$ operator:
given
$x : \mathcal{I}_{M,N}(A)$ and
$f : M \rightarrow \mathcal{I}_{P,Q}(N)$,
the behavior $x[f] : \mathcal{I}_{P,Q}(A)$
will proceed according to $x$,
but whenever $x$ attempts to perform an output $m \in M$,
the behavior $f(m)$ will be substituted,
and whenever $f(m)$ returns a value $n \in N$,
the value will be used as an input to resume to evaluation of $x$.

Note that $x[f]$ may introduce silent divergence;
for instance:
\[
    \kw{interact}^\infty(*) [\kw{ret}] = \{ \Delta \} \,,
\]
even though neither $\kw{interact}^\infty(*)$
nor $\kw{ret}$ silently diverge on their own.
To formulate the definition of $x[f]$,
we will use an interactive transition system
based on $\kw{next}$,
over states of type $\mathcal{I}_{M,N}(A) + A$.
A state $i_1(\hat{x})$ denotes that
the evaluation of $x$ is still active,
with $\hat{x}$ the remaining suffix of its behavior.
A state $i_2(v)$ denotes that
$x$ has terminated with the return value $v$.

\begin{definition}[Substitution]
For a type $A$ and
an interactive continuation $f : M \rightarrow \mathcal{I}_{P,Q}(N)$,
we define
$\vec{f}_A : \mathcal{I}_{M,N}(A) + A \rightarrow
 \mathcal{I}_{P,Q}(\mathcal{I}_{M,N}(A) + A)$
as the interactive transition system:
\begin{align*}
  \vec{f}_A(s) := i_1^{-1}(s) \bind \kw{next} \bind
      [i_2, (m, k) \mapsto f(m) \bind n \mapsto \kw{ret}(i_1(k(n)))]
%        (&i_1^{-1} \bind i_2) \\ {} \oplus
%        (&i_2^{-1} \bind (m, k) \mapsto f(m) \bind k \bind i_1)
\end{align*}
Then for a behavior $x : \mathcal{I}_{M,N}(A)$,
the \emph{substitution} $x[f] : \mathcal{I}_{P,Q}(A)$ is defined as:
\[
    x[f] := \kw{ret}(i_1(x)) \bind \vec{f}_A^\infty \bind i_2^{-1} \,.
\]
For $g : A \rightarrow \mathcal{I}_{M,N}(B)$,
the \emph{interactive composition}
$g \odot f : A \rightarrow \mathcal{I}_{P,Q}(B)$ is defined as:
\[
    (g \odot f)(a) := g(a) [f] \,.
\]
\end{definition}

[Properties:
\begin{itemize}
\item $x[\kw{interact}] = x$, $\kw{interact}(m)[f] = f(m)$
\item $x[g \circ f] = x[g][f]$
\end{itemize}
etc.]

Similarly to the way
the Kleene composition $f \cdot g$ can be iterated as $f^\infty$,
we can let a behavior interact with itself
by iterating the interactive composition $f \odot g$ as $f^\oast$.
For $f : M \rightarrow \mathcal{I}_{M,N}(N)$,
The transition system used to defined $f^\oast$
uses states of type $(N \rightarrow \mathcal{I}_{M,N}(N)$

\begin{definition}[Interactive iteration]
[A bit more involved because we need to keep a stack of continuations.
Maybe define in terms of traces if that's more transparent.]
\end{definition}

[Rename $\kw{interact}$ as $\mathbf{I}$
given that it is the identity for $\odot$?]

%}}}

\subsection{Abstraction} \label{sec:monad:abs} %{{{

We now consider the problem of relating the interactive behaviors
$x_1 : \mathcal{I}_{M_1,N_1}(X_1)$ and
$x_2 : \mathcal{I}_{M_2,N_2}(X_2)$
whose inputs, outputs, and results are taken in different sets.
We will use Kripke logical relations to relate these components,
so that the correspondance between the two interactions
can be sensitive to the history of the computation.

\begin{definition}
For sets of inputs $M_1, M_2$, outputs $N_1, N_2$, and results $X_1, X_2$,
a \emph{simulation convention} between them
is a tuple $\mathbb{R} = \langle W, \leadsto, R_M, R_N, R_X \rangle$
where $\langle W, \leadsto \rangle$ is a Kripke frame, and
$R_M : \mathcal{R}_W(M_1, M_2)$,
$R_N : \mathcal{R}_W(N_1, N_2)$,
$R_X : \mathcal{R}_W(X_1, X_2)$
are $W$-indexed Kripke logical relations
between the respective sets of inputs, outputs and results.
\begin{itemize}
\item
The identity simulation convention is defined as
$\mathbbm{1} := \langle \{*\}, \{(*,*)\}, {=}, {=}, {=} \rangle$.
\item
The composition of
the simulation conventions $\mathbb{R}$ and $\mathbb{S}$ is:
\[
    \mathbb{R} \cdot \mathbb{S} :=
      \langle
        W_\mathbb{R} \times W_\mathbb{S}, \:
        {\leadsto}_\mathbb{R} \times {\leadsto}_\mathbb{S}, \:
        R_M \cdot S_M, \:
        R_N \cdot S_N, \:
        R_X \cdot S_X
      \rangle \,,
\]
where $R \cdot S$ denotes the Kripke relation defined by:
\[
    (w_1, w_2) \Vdash R \cdot S \: := \:
      (w_1 \Vdash R) \cdot (w_2 \Vdash S) \,.
\]
\item
Given
$\mathbb{R} = \langle W, \leadsto, R_M, R_N, R_V \rangle$ and
$\mathbb{R}' = \langle W', \leadsto, R_M', R_N', R_V' \rangle$,
we say that $\mathbb{R}$ \emph{refines} $\mathbb{R}'$
and write $\mathbb{R} \sqsubseteq \mathbb{R}'$
when:
\begin{align*}
  \forall w \in W \,.\, (
    &\forall (v_1, v_2) \in [w \Vdash R_V] \,.\, \\
    &\exists w' \in W' \,.\, (v_1, v_2) \in [w' \Vdash R_V']) \\
  \wedge \: (
    &\forall (m_1, m_2) \in [w \Vdash R_M] \,.\, \\
    &\exists w' \in W' \,.\, (m_1, m_2) \in [w' \Vdash R_M'] \: \wedge \\
    &\forall v' \in W' \,.\, w' \leadsto v' \Rightarrow \\
    &\forall (n_1, n_2) \in [v' \Vdash R_N'] \,.\, \\
    &\exists v \,.\, w \leadsto v \wedge (n_1, n_2) \in [v \Vdash R_N] )
\end{align*}
\end{itemize}
\end{definition}

Simulation conventions relate interactions which have the same ``shape'',
in the sense that there is a one-to-one correspondance between
the inputs and outputs of $x_1$ and $x_2$.
Nevertheless,
extending a simulation convention $\mathbb{R}$ to behaviors
is complicated by the alternating roles of the system and the environment.
Our goal will be to define a Kripke relator:
\[ {\le}_\mathbb{R} \: = \:
   \mathcal{I}^\le_{R_M,R_N}(R_X) \: : \:
   \mathcal{R}_W(\mathcal{I}_{M_1,N_1}(X_1), \mathcal{I}_{M_2,N_2}(X_2)) \]
such that $x_1 \le_\mathbb{R} x_2$
whenever $x_1$ is simulated by $x_2$ according to $\mathbb{R}$.
We will proceed by defining a mapping:
\[ \mathbb{R}^*_w : \mathcal{I}_{M_2,N_2}(X_2) \rightarrow
                    \mathcal{I}_{M_1,N_1}(X_1) \, \]
such that $\mathbb{R}^*_w(x_2)$ is
the largest computation in $\mathcal{I}_{M_1,N_1}(X_1)$
simulated by $x_2$.
Accordingly,
we will define:
\[ w \Vdash x_1 \le_\mathbb{R} x_2 \: \Leftrightarrow \:
   x_1 \sqsubseteq \mathbb{R}^*_w(x_2) \,. \]

In the presence of different levels of abstraction
related by a convention $\mathbb{R}$,
the mapping $\mathbb{R}^*$ will allow us to embed a high-level specification
into a low-level semantic domain
where it can be compared with
the concrete behavior of the system we seek to verify.
In a complementary way, $\le_\mathbb{R}$
allows us to express abstraction relationally;
specifically, the primitives of the interaction monad
will enjoy properties with respect to $\le_\mathbb{R}$
which will help us construct
simulations between
structurally similar computations
operating at different levels of abstraction.

\begin{definition}
For a simulation convention $\mathbb{R} = \langle R_M, R_N, R_X \rangle$
and a behavior $x_2 : \mathcal{I}_{M_2, N_2}(X_2)$,
the behavior $\mathbb{R}^*_w(x_2)$ is defined by the rules:
\begin{gather*}
  \AxiomC{$(\varepsilon, v_2) \in x_2$}
  \AxiomC{$v_1 \ifr{w \Vdash R_X} v_2$}
  \BinaryInfC{$(\varepsilon, v_1) \in \mathbb{R}^*_w(x_2)$}
  \DisplayProof
  \qquad
  \AxiomC{$(\varepsilon, \Delta) \in x_2$}
  \UnaryInfC{$(\varepsilon, \Delta) \in \mathbb{R}^*_w(x_2)$}
  \DisplayProof
  \qquad
  \AxiomC{$(\varepsilon, \lightning) \in x_2$}
  \UnaryInfC{$t \in \mathbb{R}^*_w(x_2)$}
  \DisplayProof
  \\[1ex]
  \AxiomC{$(\varepsilon, m_2) \in x_2$}
  \AxiomC{$w \leadsto w'$}
  \AxiomC{$m_1 \ifr{w' \Vdash R_M} m_2$}
  \TrinaryInfC{$(\varepsilon, m_1) \in \mathbb{R}^*_w(x_2)$}
  \DisplayProof
  \\[1ex]
  \AxiomC{$
    \begin{array}{c}
      (\varepsilon, m_2) \in x_2 \qquad
      w \leadsto w' \qquad
      m_1 \ifr{w' \Vdash R_M} m_2
      \\[.3ex]
      \forall \: w'' \, n_2 \:.\:
        w' \leadsto w'' \: \wedge \:
        n_1 \ifr{w'' \Vdash R_N} n_2 \: \Rightarrow \:
        t_1 \in \mathbb{R}^*_{w''}(\delta(x_2, m_2)(n_2))
    \end{array}$}
  \UnaryInfC{$m_1 n_1 t_1 \in \mathbb{R}^*_w(x_2)$}
  \DisplayProof
\end{gather*}
\end{definition}

The mapping $\mathbb{R}^*$ preserves our constructions
on simulation conventions in the following ways:
\begin{gather*}
\mathbbm{1}^*_w(x) = x \\
(\mathbb{R} \cdot \mathbb{S})^* = \mathbb{S}^* \circ \mathbb{R}^* \\
-^* : {\sqsubseteq} \rightarrow {\sqsubseteq} \rightarrow {\sqsubseteq}
\end{gather*}
These properties ensure that
the relational version of $\mathbb{R}^*$,
defined below as $\le_\mathbb{R}$,
behaves as a relator.

\begin{definition}
For a simulation convention $\mathbb{R} = \langle R_M, R_N, R_X \rangle$,
a world $w$,
and two interactive behaviors
$x_1 : \mathcal{I}_{M_1, N_1}(X_1)$ and
$x_2 : \mathcal{I}_{M_2, N_2}(X_2)$,
we say that
\emph{$x_1$ is simulated at $w$ by $x_2$ according to $\mathbb{R}$}
and write:
\[
    x_1 \ifr{w \Vdash {\le}_\mathbb{R}} x_2
    \quad \mbox{or} \quad
    x_1 \ifr{w \Vdash \mathcal{I}^\le_{R_M,R_N}(R_X)} x_2
\]
whenever $x_1 \sqsubseteq \mathbb{R}^*_w(x_2)$.
\end{definition}

Whenever possible,
for brevity's sake we will use the notation $\le_\mathbb{R}$.
When we need to make the components relations of $\mathbb{R}$ explicit,
we will use the notation $\mathcal{I}_{R_M,R_N}(R_X)$ instead.

Identity and composition yield
${\le}_\mathbbm{1} = {\sqsubseteq}$ and
${\le}_{\mathbb{R} \cdot \mathbb{S}} =
 {\le}_\mathbb{R} \cdot {\le}_\mathbb{S}$.
More generally,
the following properties make
$\le_\mathbb{R}$ a \emph{relator} \cite{something}:
\begin{align*}
  {=} &\subseteq {\le}_\mathbbm{1} \\
  {\le}_{\mathbb{R} \cdot \mathbb{S}} &\subseteq
    {\le}_\mathbb{R} \cdot {\le}_\mathbb{S} \\
  \mathbb{R} \sqsubseteq \mathbb{S} &\Rightarrow
    {\le}_\mathbb{R} \subseteq {\le}_\mathbb{S}
\end{align*}
In particular, $\le_\mathbb{R}$ is compatible with $\sqsubseteq$
in the following sense:
\begin{align*}
    {\sqsubseteq} \cdot {\le}_\mathbb{R} &\:\subseteq\: {\le}_\mathbb{R} \\
    {\le}_\mathbb{R} \cdot {\sqsubseteq} &\:\subseteq\: {\le}_\mathbb{R}
\end{align*}

We can now formulate the following properties,
which describe the behavior of the monad's primitives
with respect to abstraction:
\begin{align*}
  \kw{ret} &:
    {}\Vdash R_X \rightarrow \mathcal{I}^\le_{R_M,R_N}(R_X) \\
  \bind &:
    (\Vdash R_X \rightarrow
     \mathcal{I}^\le_{R_M,R_N}(R_Y)) \rightarrow
    (\Vdash \mathcal{I}^\le_{R_M,R_N}(R_X) \rightarrow
     \mathcal{I}^\le_{R_M,R_N}(R_Y)) \\
  \kw{interact} &:
    (\Vdash (\Diamond R_M) \rightarrow
     \mathcal{I}^\le_{R_M,R_N}(R_N)) \\
  \kw{next} &:
    (\Vdash \mathcal{I}^\le_{R_M,R_N}(R_X) \rightarrow
     R_X +
     \Diamond (R_M \times
     \Box (R_N \rightarrow \mathcal{I}^\le_{R_M,R_N}(R_X)))) \\
  -^\infty &:
    (\Vdash R_X \rightarrow \mathcal{I}^\le_{R_M,R_N}(R_X)) \rightarrow
    (\Vdash R_X \rightarrow \mathcal{I}^\le_{R_M,R_N}(R_X)) \\
  \eta^\mathcal{P} &:
    (\Vdash \mathcal{P}^\le(R_X) \rightarrow
     \mathcal{I}^\le_{R_M,R_N}(R_X))
\end{align*}
Together,
these properties allow us to construct
heterogenous simulations
between monadic terms with similar structures.

\begin{example} \label{ex:sim}
Building on our previous example,
consider
$\alpha_1 = (S_1, I_1, {\rightarrow}_1, F_1)$ and
$\alpha_2 = (S_2, I_2, {\rightarrow}_2, F_2)$
two transition systems,
together with a relation
$R : \mathcal{R}(S_1, S_2)$
satisfying:
\begin{gather*}
  I_1 \ifr{\mathcal{P}^\le(R)} I_2 \\
  {\rightarrow}_1 \ifr{R \rightarrow \mathcal{P}^\le(R)} {\rightarrow}_2 \\
  F_1 \ifr{R \rightarrow \mathcal{P}^\le(=)} F_2
\end{gather*}
That is, $R$ is a simulation relation between $\alpha_1$ and $\alpha_2$.
Then by using the properties above and
following the structure of $\llbracket - \rrbracket$,
we can show that:
\[
    \llbracket \alpha_1 \rrbracket \sqsubseteq
    \llbracket \alpha_2 \rrbracket \,.
\]
\end{example}

%}}}

\subsection{Games and Strategies} \label{sec:monad:games} %{{{

[Just define our notion of elementary game / language interface,
and the notion of strategy for $A \rightarrow B$
that we will use for CompCert semantics.
Connect to HON games]

%}}}

%}}}

\section{Semantics of CompCert Modules} %{{{

% preamble {{{

Having laid out our basic semantic framework,
we now use it to construct a compositional semantic model
for open modules in CompCert.

%}}}

\subsection{Overview} %{{{

CompCert is a C-to-assembly compiler written and verified
in the Coq proof assistant.
The compiler is accompanied by
a formal semantics of the source, target, and intermediate languages,
and a proof that if the compiler succeeds,
then the behavior of the emitted target program
refines that of the source program.

The semantics of CompCert languages
are given as labeled transition systems (LTS)
modeling the execution of a whole program's \kw{int main()} function.
The semantics of external calls
is a global parameter common to all languages,
and may use a fixed set of actions as a rudimentary model
of the program's interaction with the environment.
The externally observable behavior of the program
is given by defining the set of traces associated with any LTS,
and the various kinds of simulations used by CompCert
are shown to be sound with respect to trace containement.

CompCert, however,
does not attempt to model the semantics of open modules,
nor to define a semantic notion of module composition.
We remedy this using the following approach:
\begin{itemize}
\item Following Compositional CompCert \cite{compcompcert},
  we extend CompCert's LTS model
  to explicitely account for the two-way interaction
  between a program module and its environment
  in terms of control transfers associated with
  function calls and returns.
\item We further extend this approach by
  using elementary games to parametrize the LTS model, and
  using refinement conventions to parametrize
  CompCert's notions of simulations.
  Games specify the form of function calls and returns,
  whereas refinement conventions serve to express
  a generalized calling convention between the various
  languages involved in the compilation process.
\end{itemize}

In \S\ref{sec:modsem:def},
we introduce our changes to CompCert's LTS model,
and define their behavior in terms of interactive computations.
In \S\ref{sec:modsem:ref},
we carry out those changes to the notions of
forward and backward simulations
present in CompCert,
and show them sound with respect to refinement.
In \S\ref{sec:modsem:comp},
[composition].
[Figure out how abstraction fits in].

%}}}

\subsection{Definition} %{{{
\label{sec:modsem:def}

The semantics of CompCert languages are given mainly
as labeled transition systems (LTS).
In the original CompCert,
a labeled transition system is given as
a set of states $S$,
a set of initial states
$I \subseteq S$,
a labeled transition relation
${\rightarrow} \subseteq S \times \mathbb{E}^* \times S$,
and a set
$F \subseteq S \times \kw{int}$
of final states associated with an integer result.
Compcert LTS support a notion of interaction with the environment
in the form of event traces:
a transition $s \stackrel{t}{\rightarrow} s'$,
indicates that the state $s$ may transition to state $s'$
through an interaction recorded as the event trace $t \in \mathbb{E}^*$.

For the range of languages that we are considering
($\kw{Clight}$--$\kw{Asm}$),
non-empty event traces are only generated by
the predefined semantics of external calls.
Since this model is insufficient to express
the semantics of open C and assembly modules,
we replace this facility with
and explicit treatment of incoming and outgoing calls
at the level of transition systems.

\begin{definition}[Small-step semantics]
Given two elementary games $A, B$,
a \emph{small-step semantics} for the game $A \rightarrow B$
is a tuple $L = \langle S, \rightarrow, I, X, R, F \rangle$.
$S$ is a set of states,
with ${\rightarrow} \subseteq S \times S$ a \emph{transition relation} on $S$.
The handling of incoming calls is specified by
$I \subseteq M_B^\kw{Q} \times S$, which
assigns a set of \emph{initial states} to each question of $B$, and
$F \subseteq S \times M_B^\kw{A}$,
which designates \emph{final states} together with corresponding answers.
The handling of external calls is specified by
$X \subseteq S \times M_A^\kw{Q}$,
which identifies \emph{external states} together with
corresponding questions of $A$ directed to the environment, and
$R \subseteq S \times M_A^\kw{A} \times S$,
which is used to select a \emph{resumption state}
based on the outcome of the external call
after the environment returns control to the module.

We write $L : \kw{semantics}(A, B)$ to indicate that
$L$ is a small-step semantics for the game $A \rightarrow B$.
\end{definition}

CompCert interprets the lack of any transition (``getting stuck'')
as undefined behavior.
This convention fits the context of
language semantics defined through inductive sets of rules:
the behavior of a state for which there are no rules is undefined.
For a small-step semantics
$L = \langle S, {\rightarrow}, I, X, R, F \rangle : \kw{semantics}(A,B)$,
we recognize those states using the predicate:
\[
    \kw{stuck}_L(s) :=
      ({\rightarrow}(s) = \varnothing) \wedge
      (X(s) = \varnothing) \wedge
      (F(s) = \varnothing)
\]
Taking this into account,
the immediate behavior of a state $s \in S$
can be expressed as the interactive computation
$\kw{step}_L(s) : \mathcal{I}_{M_A^\kw{Q},M_A^\kw{A}}(S)$
defined as follows:
\[
  \kw{step}_L(s) :=
    \begin{cases}
      \top & \mbox{if } \kw{stuck}(s) \\
      {\rightarrow}(s) \vee
      (X(s) \bind \kw{interact} \bind R(s)) & \mbox{otherwise,}
   \end{cases}
\]
The overall external behavior of $L$
can then be given as an interactive computation
$
    \llbracket L \rrbracket :
      M_B^\kw{Q} \rightarrow \mathcal{I}_{M_A^\kw{Q},M_A^\kw{A}}(M_B^\kw{A})
$
defined by:
%Then $\llbracket L \rrbracket$ can be defined as:
\begin{align*}
  \llbracket L \rrbracket (q) :=
    \begin{cases}
       \top & \mbox{if } I(q) = \varnothing \\
       I(q) \bind \kw{step}^\infty \bind F & \mbox{otherwise.}
     \end{cases}
\end{align*}

[Perhaps make this point here:
For now,
we can remark that transition systems in the form defined above
describe the behavior of a \emph{single} invocation
of the program module being modeled.
With each incoming question,
we will instantiate a new, independent state
using the initial-state predicate $I$.
The derived strategies will be innocent,
in the sense that they will not maintain
any hidden state across subsequent or reentrant
invocations from the environment.
Instead,
relevant global state will be passed back into the module
as a component of the environment's question
(for instance in the form of the memory state component
of the question as described in Table~\ref{tbl:li}).]

%}}}

\subsection{Refinement and Abstraction} %{{{
\label{sec:modsem:ref}

The correctness of CompCert is established in terms of
a \emph{backward simulation}
between the small-step semantics of the source and target programs.
In turn, backward simulations
are shown to be sound with respect to trace containement.
In this section,
we define backward simulations for our updated notion
of small-step semantics,
and show that they are sound with respect to
the refinement of their behaviors.

% Footnote pointing out difference w/ "Fw & Bw Sim" terminology?
A backward simulation asserts that any transition in the target program
has a corresponding transition sequence in the source program.
A transition in the target program can be matched with
an empty transition sequence in the source program;
however, to ensure the preservation of silent divergence,
this can only happen for finitely many consecutive target transitions.
This restriction is enforced by indexing the simulation relation
over a well-founded order,
and making sure that the index decreases
whenever a potentially empty sequence of source transitions is used.

In our setting,
the definition of backward simulations needs to be extended
to take into account the ways in which the questions and answers
at the source and target level ought to be related.
This can be specified using the notion of simulation convention
defined in \S\ref{sec:monad:abs}:
a simulation between the small-step semantics
$L_1 : \kw{semantics}(A_1, B_1)$ and
$L_2 : \kw{semantics}(A_2, B_2)$ will
operate in the context of the refinement convention
$\mathbb{C}_A : \mathcal{R}_{W_A}(A_1, A_2)$ and
$\mathbb{C}_B : \mathcal{R}_{W_B}(B_1, B_2)$.

Note that initial and resumption states on one hand,
and external and final states on the other hand,
correspond to respective actions of the environment and the system;
as such, they need to be treated differently.
Furthermore,
the potential non-determinism present in small-step semantics
is interpreted as \emph{system} non-determinism,
so that when relating sets of states
we will want to make sure that
each \emph{target} state has a corresponding \emph{source} state,
but the source program may allow additional behaviors
that are not realized in the target program.
This rule is slightly altered to take into account the convention that
empty sets of states correspond to undefined behaviors,
so that if the source set is empty,
no restrictions are placed on the target state.
This is formalizes as follows.

\begin{definition}[Backward simulation between sets of states]
We say that $R : \mathcal{R}(S_1, S_2)$ is a
\emph{backward simulation relation
  between the sets $X_1 \subseteq S_1$ and  $X_2 \subseteq S_2$}
if the following conditions hold:
\begin{enumerate}
\item
  If $X_1$ is non-empty,
  then $X_2$ is non-empty as well.
\item
  If $X_1$ is non-empty,
  then for all $s_2 \in X_2$,
  there exists $s_1 \in X_1$
  such that $(s_1, s_2) \in R$.
\end{enumerate}
We will write $X_1 \ge_R X_2$.
\end{definition}

A state $s$ \emph{goes wrong}
if it is neither an external nor a final state,
and if there no transition $s \rightarrow s'$;
a state $s$ is \emph{safe}
if no state that goes wrong is reachable from $s$
by the transition relation $\rightarrow$.
Safe states will be denoted using the predicate $\kw{safe}(s)$.
With this we are ready to define backward simulations.

\begin{definition}[Backward simulation]
Given
two simulation conventions
$\mathbb{C}_A : \mathcal{R}(A_1, A_2)$ and
$\mathbb{C}_B : \mathcal{R}(B_1, B_2)$,
and given
$L_1 : \kw{semantics}(A_1, B_1)$ and
$L_2 : \kw{semantics}(A_2, B_2)$
two small-step semantics,
a \emph{backward simulation} between $L_1$ and $L_2$
consists in a
well-founded order $(I, <)$
together with a family of relations
$(R_i : \mathcal{R}_{W_B}(S_1, S_2))_{i \in I}$
satisfying the following properties:
\begin{description}
\item[Initial states]
  For all
  $q_1 \ifr{w \Vdash {\preceq}_B^\kw{Q}} q_2$,
  the condition $I_1(q_1) \ge_{\exists i . R_i} I_2(q_2)$ holds.
\item[External calls]
  For all $s_1 \ifr{w \Vdash R_i} s_2$
  with $s_1$ a safe state, and
  for any question $m_2 \in M_{A_2}^\kw{Q}$
  such that $(s_2, m_2) \in X_2$,
  there exists $w' \in W_A$ and $q_1 \in M_{A_1}^\kw{Q}$
  such that $q_1 \ifr{w' \Vdash {\preceq}_A^\kw{Q}} q_2$.
  In addition, for all corresponding answers
  $r_1 \ifr{w' \Vdash {\preceq}_A^\kw{A}} r_2$,
  the condition $R_1(s_1, r_1) \ge_{\exists i . R_i} R_2(s_2, r_2)$ holds.
\item[Final states]
  For all $s_1 \ifr{w \Vdash R_i} s_2$
  with $s_1$ a safe state, and
  for any answer $r_2 \in M_{B_2}^\kw{A}$
  such that $(s_2, r_2) \in F_2$,
  there exists a state $s_1'$ reachable from $s_1$ and
  an answer $r_1 \in M_{B_1}^\kw{A}$ such that
  $(s_1', r_1) \in F_1$ and $r_1 \ifr{w \Vdash {\preceq}_B^\kw{A}} r_2$.
\item[Progress]
  For all $s_1 \ifr{w \Vdash R_i} s_2$
  with $s_1$ a safe state,
  $s_2$ does not go wrong.
\item[Simulation]
  For all $s_1 \ifr{w \Vdash R_i} s_2$,
  if $s_1$ is a safe state and $s_2 \rightarrow s_2'$,
  then there exists $i' \in I$ and $s_1' \in S_1$
  such that $s_1' \ifr{w \Vdash R_{i'}} s_2'$ and
  such that one the following conditions hold:
  \[
    s_1 \rightarrow^+ s_1' \,, \quad \mbox{or} \quad
    s_1 \rightarrow^* s_1' \:\wedge\: i' < i \,.
  \]
\end{description}
We will write $L_1 \ge_{\mathbb{C}_A \rightarrow \mathbb{C}_B} L_2$.
\end{definition}

[Maybe short commentary to unpack some of that.]

Stated in their full generality,
CompCert backward simulations are notably complex.
Fortunately,
many simpler formulations and related constructions
can be used to establish
the existence of backward simulations
for specific compiler passes.
However,
backward simulations are not easily amenable to
meta-theoretical analysis
where they can appear both as hypotheses and conclusions,
which is an additional motivation for introducing
our game-based framework.
In the rest of this section we show that
backward simulations between small-step semantics
are sound with respect to
our much simpler notion of refinement
over interactive computations.

[Do that.]

%}}}

\subsection{Composition} \label{sec:modsem:comp} %{{{

Given two semantic objects
$\sigma_1 : B \Rightarrow C$ and
$\sigma_2 : A \Rightarrow B$,
their categorical composition $\sigma_1 \odot \sigma_2 : A \Rightarrow C$,
per the definition in \S\ref{sec:monad:kmor},
computes the behavior of $\sigma_1$ when $\sigma_2$ is used
to interpret its external calls.
In addition,
to model cross-module interactions in low-level programs,
we define the symmetric,
\emph{horizontal composition}
of
$\sigma_1 : A \Rightarrow A$ and
$\sigma_2 : A \Rightarrow A$,
which allows $\sigma_1$ and $\sigma_2$
to interact with one another
in a mutually recursive fashion.

\begin{definition}[Horizontal composition]
For an elementary game $A$,
the \emph{horizontal composition} of the interactive computations
$\sigma_1, \sigma_2 : M_A^\kw{Q} \rightarrow
 \mathcal{I}_{M_A^\kw{Q},M_A^\kw{A}}(M_A^\kw{A})$
is the interactive computation
$\sigma_1 \bullet \sigma_2 : M_A^\kw{Q} \rightarrow
 \mathcal{I}_{M_A^\kw{Q},M_A^\kw{A}}(M_A^\kw{A})$
defined as:
\[
    \sigma_1 \bullet \sigma_2 :=
      (\sigma_1 \oplus \sigma_2)^\oast \,.
\]
\end{definition}

The commutativity of $\bullet$ trivially follows from that of $\oplus$,
and associativity can be derived from properties of $\oplus$ and $\oast$.
The empty behavior $\mathbf{0}$ and
the ``pass-through'' behavior $\kw{interact}$
are both identities for $\bullet$.
In addition,
the following approximations that can be useful for proofs:
\[
    \sigma_1 \odot \sigma_2 \quad \sqsubseteq \quad
    \sigma_1 \odot (\sigma_2 \oplus \kw{interact}) \quad \sqsubseteq \quad
    \sigma_1 \bullet \sigma_2 \,.
\]

%}}}

\subsection{Hiding} %{{{

Note that
in the behavior $\sigma_1 \bullet \sigma_2$,
cross-component calls remain observable:
while the external calls of $\sigma_1$
can be resolved to behaviors implemented in $\sigma_2$
(and conversely),
the original external calls remain
as non-deterministic alternatives in the final result.
[Only way to ensure associativity
because nothing prevents other components
to \emph{also} implement these functions,
in which case eliminating the calls
would make the result sensitive to the order of composition.]

This phenomenon is reminiscent of a similar one observed
in the process calculus CCS \cite{ccs}
in the context of \emph{parallel composition}.
Following this precedent,
we then define a separate \emph{hiding} operator
which retains only the external calls
which belong to a given set.

\begin{definition}[Hiding]

\end{definition}

%}}}

\subsection{Multi-module programs} %{{{

[Following composition + hiding in the last section,
extend semantic model to include component domains and/or range
and construct a notion of composition with built-in hiding
that is nevertheless associative.
Build corresponing syntactic notation
for multi-module programs.]

[Alt., syntactic linking]

%}}}

%}}}

\section{Compiler Correctness} %{{{

In the previous section
we have shown how to extend
the semantic model used by CompCert
to describe the behavior of open modules.
In this section
we show how the correctness proof of CompCert
can be updated to account for this additional structure.

\subsection{Overview} %{{{

Based on the existing CompCert semantics,
we will define the elementary games
$\mathcal{C}$ and $\mathcal{A}$
describing how C and assembly modules
interact with their respective environments.
The semantics of CompCert $\kw{Clight}$ and $\kw{Asm}$ programs
can then be expressed as interactive behaviors of types:
\[
    \kw{Clight}(p_s) : \mathcal{C} \Rightarrow \mathcal{C} \,, \qquad
    \kw{Asm}(p_t) : \mathcal{A} \Rightarrow \mathcal{A} \,.
\]
We will show that there exists a simulation convention
$\mathbb{R} : \mathcal{A} \Leftrightarrow \mathcal{C}$
such that whenever $p_t = \kw{CompCert}(p_s)$,
the following refinement property holds:
\begin{equation}
    \label{eqn:correctness}
    \kw{Asm} \llbracket p_t \rrbracket
    \, \sqsubseteq_{\mathbb{R} \Rightarrow \mathbb{R}}
    \kw{Clight} \llbracket p_s \rrbracket
\end{equation}

In Compositional CompCert,
this is achieved independently for each compilation pass.
Stated in our terms,
Compositional CompCert's \emph{interaction semantics}
employ a fixed interface $\mathcal{C}$,
and \emph{structured injections} define a single refinement convention
$\mathbb{S} : \mathcal{C} \Leftrightarrow \mathcal{C}$.
A theorem similar to (\ref{eqn:correctness}) is proved for each pass,
and structured injections are show to compose
($\mathbb{S} \cdot \mathbb{S} \equiv \mathbb{S}$),
so that a simulation theorem can be derived for the whole compiler.
However,
a major challenge encountered in the course of this work
was the asymmetry of requirements vs. guarantees
present in the existing proofs:
while CompCert imposes strong requirements
on the semantics of external functions,
the simulation relations used by its correctness proofs
are too weak to establish corresponding guarantees
for a module's own function executions,
preventing horizontal compositionality.
Because of this,
Compositional CompCert required significant changes
to all compilation passes,
strengthening their simulation relations
to conform to $\mathbb{S}$.

However,
our explicit treatment of abstraction
and reified notion of simulation convention
makes another approach possible.
Using existing proofs, it is reasonably straightforward
to establish a theorem of the form:
\begin{equation}
    \label{correctness-alt}
    \llbracket p_t \rrbracket
    \sqsubseteq_{\mathbb{R} \Rightarrow \mathbb{R}'}
    \llbracket p_s \rrbracket
\end{equation}
for each compilation pass,
where the simulation conventions $\mathbb{R}$ and $\mathbb{R}'$
characterize the assumptions and guarantees
of the existing proof.
In addition,
relevant properties of key source, target, and intermediate languages
can be expressed in relational form,
and can be used to strengthen the resulting theorem:
by composing these proofs together,
and using algebraic properties of
the simulation conventions involved,
we will be able to bridge the gap
between the domain and codomain simulation conventions
and be able to prove a correctness statement
in the form of (\ref{eqn:correctness}).

In the remainder of this section,
[dig deeper into the actual semantics and
simulation conventions used in CompCert.]
etc.

%}}}

\subsection{The CompCert Memory Model} %{{{

\begin{figure} % fig:mm (The Compcert memory model) {{{
  \begin{gather*}
    v : \kw{val} ::=
      \kw{Vundef} \alt
      \kw{Vint}(n) \alt
      \kw{Vlong}(n) \alt
      \kw{Vfloat}(x) \alt
      \kw{Vsingle}(x) \alt
      \kw{Vptr}(b, o)
    \\
    (b, o) : \kw{ptr} :=
      \kw{block} \times \mathbb{Z}
    \\
    (b, l, h) : \kw{ptrrange} :=
      \kw{block} \times \mathbb{Z} \times \mathbb{Z}
  \end{gather*}
  \begin{align*}
    \kw{Mem.alloc} &:
      \kw{mem} \rightarrow \mathbb{Z} \rightarrow \mathbb{Z} \rightarrow
      \kw{mem} \times \kw{block}
    \\
    \kw{Mem.free} &:
      \kw{mem} \rightarrow
      \kw{ptrrange} \rightarrow
      \kw{option}(\kw{mem})
    \\
    \kw{Mem.load} &:
      \kw{mem} \rightarrow \kw{ptr} \rightarrow \kw{option}(\kw{val})
    \\
    \kw{Mem.store} &:
      \kw{mem} \rightarrow \kw{ptr} \rightarrow \kw{val} \rightarrow \kw{option}(\kw{mem})
    \\
    \kw{Mem.perm} &:
      \kw{mem} \rightarrow \kw{ptr} \rightarrow \mathcal{P}(\kw{perm})
  \end{align*}
  \caption{Outline of the Compcert memory model}
  \label{fig:mm}
\end{figure}
%}}}

The Compcert memory model \cite{compcertmmv2}
is the core algebraic structure
which underlies the semantics of Compcert's languages.
Some of its operations
are shown in Fig.~\ref{fig:cklr}.
The idealized version presented here
involves
the type of memory states \kw{mem},
the types of pointers \kw{ptr} and address ranges \kw{ptrrange}, and
the type of runtime values \kw{val}.
To keep our exposition concise and clear,
we will gloss over the technical details
associated with the encoding of offsets
as concrete binary integers,
and the associated modular arithmetic and overflow constraints.
[But see artefact for the details.]

The memory is organized into a finite number of \emph{blocks}.
Each memory block has a unique identifier ($b : \kw{block}$)
represented as a positive integer,
and is equipped with its own independent linear address space.
Block identifiers and offsets are often manipulated together,
as a pair $p = (b, o) : \kw{ptr} = \kw{block} \times \mathbb{Z}$.
New blocks are created by the primitive $\kw{Mem.alloc}$,
with prescribed boundaries for their usable offsets.

A runtime value ($v : \kw{val}$) can be stored at
a given address using the primitive \kw{Mem.store},
and retreived using the primitive \kw{Mem.load}.
Values can be integers (\kw{Vint}, \kw{Vlong}) and
floating point numbers (\kw{Vfloat}, \kw{Vsingle})
of different sizes,
as well as pointers (\kw{Vptr}).
The special value \kw{Vundef}
represents an undefined value;
the simulation relations used by Compcert
usually allow $\kw{Vundef}$
to be refined into a more concrete value.

The memory model is shared by all of the languages in CompCert.
The states used to define their semantics consist of
a memory component $m : \kw{mem}$,
together with language-specific components
built mainly out of additional run-time values ($\kw{val}$).
For higher-level languages,
the language-specific component may consist of
a control stack with local environments for each activation;
for lower-level languages,
the additional component will mainly contain
the values stored in various registers.
Similarly,
our models of cross-module communication
will be centered around a memory component,
accompanied by language-specific information
about function call or return events.

%}}}

\subsection{Language Interfaces} %{{{

\begin{table*} % tbl:li Language interfaces {{{
  \begin{tabular}{clll}
    \hline
    Name & Questions & Answers & Description \\
    \hline
    $\mathcal{C}$ & $(\kw{id}, \kw{sg}, \vec{v}, m)$ & $(v', m')$ &
      C-style function calls (Clight--RTL) \\
    $\mathcal{L}$ & $(\kw{id}, \kw{sg}, \kw{ls}, m)$ & $(\kw{ls}', m')$ &
      Arguments passed in abstract locations (LTL, Linear) \\
    $\mathcal{M}$ & $(\kw{id}, \kw{sp},\kw{ra},\kw{rs}, m)$ & $(\kw{rs}', m')$ &
      Arguments passed through in-memory stack (Mach) \\
    $\mathcal{A}$ & $(\kw{rs}, m)$ & $(\kw{rs}', m')$ &
      Assembly-style control transfers (Asm) \\
    \hline
  \end{tabular}
  \caption{Language interfaces for the various Compcert intermediate languages.}
  \label{tbl:li}
\end{table*}
%}}}

The elementary games we use to model
the cross-module interactions of CompCert languages
are shown in Table~\ref{tbl:li}.
Moves correspond to control transfers
between a module and its environment:
questions correspond to function invocations;
answers return control to the caller.

At the source level ($\mathcal{C}$),
questions consist of
the name and signature of the function being invoked,
the values of its arguments,
and the state of the memory at the point of entry;
answers
consist of the function's return value
and the state of the memory at the point of exit.
This language interface is used for Clight and
for the majority of CompCert's intermediate languages.

However, starting with LTL, [...]
Once we reach Asm ($\mathcal{A}$),
queries and replies simply specify
the values of registers and the state of the memory.

%}}}

\subsection{Simulation Relations} %{{{

[Simulation relations account for refinement and abstraction,
for the change in structure between the states of different languages.
Simulation convention is the ``externally visible'' component of that.
Since core of states is mem,
core of simulation relations and conventions
are logical relation on memories.
Specific cases used in CompCert are
injections and extensions (explain).
We're going to generalize to CKLR.]

Although block identifiers and offsets
are concretely represented as integers,
there is an implicit expectation throughout CompCert
that constructions using the memory model
will not depend on specific block identities
or absolute pointer offsets.
[Explain connection to relational parametricity.]

{\color{gray}
--- [Some former material in the rest of this section:] ---

Besides equality,
the simulation proofs used
to establish the correctness of Compcert passes
relate the runtime values and memory state
that make up the states for a given language
in one of two possible ways:
memory extensions
allow the target values and memory to be more defined
than the source ones;
memory injections
additionally permits a remapping of source memory blocks
into target memory blocks at given offsets.
This remapping is specified by an injection mapping
$f : \kw{meminj}$,
where the type $\kw{meminj}$ is defined as
$\kw{block} \rightarrow \kw{option}(\kw{block} \times \mathbb{Z})$.
An entry of the form $f(b) = (b', \delta)$
signifies that the block $b$ of the source memory
is mapped into the block $b'$ of the target memory,
with offsets shifted so that
offset $0$ in $b$ will be mapped at offset $\delta$ in $b'$.

Compatibility of a given component
with extensions and injections
asserts that given two input memory states and surrounding values
in a ``memory extension'' relation
(resp. in a specific ``memory injection'' relation),
the component's outputs will preserve that relationship.
If the component allocates new memory blocks,
the injection mapping may be correspondingly extended
to introduce entries for the allocated blocks.
This is an informal description,
and the exact formulation of such theorems
vary on a component-by-component basis.
Moreover,
for many components,
CompCert needs to introduce two separate,
but very similar proofs:
one for extensions and one for injections.

As we generalize from extensions and injections
to define a family of Kripke logical relations
over the Compcert memory model,
we will be able to formalize these restrictions
in the form of unified relational parametricity theorems:
for any Compcert KLR,
such components will be related to themselves
by a relation constructed according to their type,
from the basic components provided by the KLR.
Because the KLRs are by definition compatible
with the memory model's basic operations,
such theorems will be relatively straightforward to prove,
as long as the constructions under consideration
do not access the values and memory states in exotic ways.
In addition,
because the parametricity theorems
and the properties of components
are expressed in terms of a unified relational language,
it will be possible to mechanize these proofs
to some extent.

}

%}}}

\subsection{Kripke Logical Relations} %{{{

\begin{figure} % fig:cklr-derived (Derived components of CKLRs) {{{
  \[ R = (W, \leadsto, f, R^\kw{mem}) \]

  \noindent \fbox{$R_w^\kw{ptr}$} \hfill \ 
  \[
    \AxiomC{$f_w(b) = (b', \delta)$}
    \UnaryInfC{$(b, o) \ifr{R_w^\kw{ptr}} (b', o + \delta)$}
    \DisplayProof
  \]

  \noindent \fbox{$R_w^\kw{ptrrange}$} \hfill \ 
  \[
    \AxiomC{$(b_1, l_1) \ifr{R^\kw{ptr}_w} (b_2, l_2)$}
    \AxiomC{$h_1 - l_1 = h_2 - l_2$}
    \BinaryInfC{$(b_1, l_1, h_1) \ifr{R^\kw{ptrrange}_w} (b_2, l_2, h_2)$}
    \DisplayProof
  \]

  \noindent \fbox{$R_w^\kw{val}$} \hfill \ 
  \begin{gather*}
    \kw{Vundef} \ifr{\Box R^\kw{val}} v \quad \\
    \kw{Vint} \ifr{(=) \rightarrow \Box R^\kw{val}} \kw{Vint} \\
    \kw{Vlong} \ifr{(=) \rightarrow \Box R^\kw{val}} \kw{Vlong} \\
    \kw{Vfloat} \ifr{(=) \rightarrow \Box R^\kw{val}} \kw{Vfloat} \\
    \kw{Vsingle} \ifr{(=) \rightarrow \Box R^\kw{val}} \kw{Vsingle} \\
    \kw{Vptr} \ifr{\Box (R^\kw{ptr} \rightarrow R^\kw{val})} \kw{Vptr}
  \end{gather*}

  \caption{Derived components of CKLRs}
  \label{fig:cklr-derived}
\end{figure}
%}}}

\begin{figure} % fig:cklr-axioms (Axioms for CKLRs) {{{
  Kripke frame
  \vspace{1em}
  \begin{gather*}
    w \leadsto w \\
    w \leadsto w' \wedge w' \leadsto w'' \Rightarrow w \leadsto w'' \\
    f \ifr{(\leadsto) \rightarrow \kw{inject\_incr}} f
  \end{gather*}

  \vspace{2em}
  Memory operations ($R^\kw{mem}$)
  \vspace{1em}
  \[
    \begin{array}{c}
      \kw{Genv.init\_mem}
      \ifr{(\approx) \rightarrow \Diamond R^\kw{mem}}
      \kw{Genv.init\_mem}
      \\
      \kw{Mem.alloc}
      \ifr{\Box(R^\kw{mem} \rightarrow (=) \rightarrow (=) \rightarrow
        \Diamond (R^\kw{mem} \times R^\kw{block}))}
      \kw{Mem.alloc}
      \\
      \kw{Mem.free}
      \ifr{\Box(R^\kw{mem} \rightarrow R^\kw{ptrrange} \rightarrow
        \kw{option}^+(\Diamond R^\kw{mem}))}
      \kw{Mem.free}
      \\
      \kw{Mem.load}
      \ifr{\Box(R^\kw{mem} \rightarrow R^\kw{ptr} \rightarrow
        \kw{option}^+(R^\kw{val}))}
      \kw{Mem.load}
      \\
      \kw{Mem.store}
      \ifr{\Box(R^\kw{mem} \rightarrow R^\kw{ptr} \rightarrow R^\kw{val} \rightarrow
        \kw{option}^+(\Diamond R^\kw{mem}))}
      \kw{Mem.store}
      \\
      \kw{Mem.perm}
      \ifr{\Box(R^\kw{mem} \rightarrow R^\kw{ptr} \rightarrow (\subseteq))}
      \kw{Mem.perm}
    \end{array}
  \]
  \caption{Axioms for CKLRs}
  \label{fig:cklr-axioms}
\end{figure}
%}}}

\begin{figure} % fig:cklr-props (Notable properties of CKLRs) {{{
  \vspace{2em}
  Pointers ($R^\kw{ptr}, R^\kw{ptrrange}$)
  \vspace{1em}
  \vspace{1em}
  \[
    \AxiomC{$w \leadsto w'$}
    \UnaryInfC{$R^\kw{ptr}_w \subseteq R^\kw{ptr}_{w'}$}
    \DisplayProof
    \quad
    \AxiomC{$w \leadsto w'$}
    \UnaryInfC{$R^\kw{ptrrange}_w \subseteq R^\kw{ptrrange}_{w'}$}
    \DisplayProof
    \quad
    \AxiomC{$w \leadsto w'$}
    \UnaryInfC{$R^\kw{val}_w \subseteq R^\kw{val}_{w'}$}
    \DisplayProof
  \]
  \caption{Notable properties of CKLRs}
  \label{fig:cklr-props}
\end{figure}
%}}}

Memory extensions and injections
are useful for simulations
because they are specific instances
of logical relations for the CompCert memory model:
they are compatible with memory operations
in the sense that
performing similar operations on related arguments
yields related results.

We formalize this idea by defining
a notion of Kripke logical relations over the CompCert memory model,
closed under composition, and which
admits memory extensions and injections as particular instances.
As we will see,
more complex relations can also be defined,
and this will make relational parametricity theorems
particularly useful.

\begin{definition}[Compcert Kripke simulation relation]
A \emph{Compcert Kripke simulation relation} $R$
is a tuple $(W_R, \leadsto_R, f_R, R^\kw{mem})$
such that:
\begin{itemize}
\item $\langle W_R, \leadsto_R \rangle$
  is the Kripke frame associated with $R$;
\item $f_R : W_R \rightarrow \kw{meminj}$
  is an injection mapping specifying how
  pointers and values should be related;
\item $R^\kw{mem} : W_R \rightarrow \mathcal{R}(\kw{mem})$
  is the relation's component for memory states.
\end{itemize}
From $f_R$,
we can derive the components
$R^\kw{ptr}$, $R^\kw{ptrrange}$, $R^\kw{block}$ and $R^\kw{val}$.
The components of $R$ must satisfy
a number of properties which are shown in Fig.~\ref{fig:cklr-axioms}
and discussed below.
We will omit the $R$ subscripts when discussing a single relation.
\end{definition}

Note that only the $R^\kw{mem}$ component is given direcly.
We expect $R^\kw{ptr}$ to be functional
(so that each source pointer has at most one corresponding target pointer),
and to satisfy the following shift-invariance property:
\[
  \AxiomC{$(b_1, o_1) \ifr{R^\kw{ptr}_w} (b_2, o_2)$}
  \UnaryInfC{$(b_1, o_1 + \delta) \ifr{R^\kw{ptr}_w} (b_2, o_2 + \delta)$}
  \DisplayProof
\]
Any such relation can be uniquely specified by
an injection mapping such as $f_R$.
We expect the remaining components to be consistent with $R^\kw{ptr}$
and $\kw{Vundef}$ to act as a bottom element for $R^\kw{val}$,
so that the definitions provided in Fig.~\ref{fig:cklr-derived}
are the only possible ones.

Note that the relational property associated to $f$,
together with the definitions of
derived relations such as $R^\kw{ptr}$ and $R^\kw{val}$,
ensure that these relations are monotonic in $w$,
in the sense that if $w \leadsto w'$
then $R^x_w \subseteq R^x_{w'}$.
However,
this is not necessarily the case for $R^\kw{mem}$.

%}}}

\subsection{Composition of passes} %{{{

\begin{table*} % tbl:passes Passes of Composable Compcert %{{{
  \begin{tabular}{lllp{.5\textwidth}}
    \hline
    Language/Pass & Outgoing & Incoming & Description \\
    \hline
    \textbf{Clight} & $\mathcal{L}_\kw{C}$ & $\mathcal{L}_\kw{C}$ &
      A simpler version of CompCert C
      where expressions contain no side-effects. \\
    -- & $R^*; \kw{injn}$ & $R^*; \kw{injn}$ & \emph{Clight properties} \\
    \kw{Cshmgen} & \kw{id} & \kw{id} &
      Simplification of control structures;
      explication of type-dependent computations. \\
    \hline
    \textbf{Csharpminor} & $\mathcal{L}_\kw{C}$ & $\mathcal{L}_\kw{C}$ &
      Low-level structured language. \\
    \kw{Cminorgen} & \kw{injp} & \kw{injt} &
      Stack allocation of local variables whose address is taken;
      simplification of switch statements. \\
    \hline
    \textbf{Cminor} & $\mathcal{L}_\kw{C}$ & $\mathcal{L}_\kw{C}$ &
      Low-level structured language,
      with explicit stack allocation of certain local variables. \\
    \kw{Selection} & \kw{extp} & \kw{extt} &
      Recognition of operators and addressing modes. \\
    \hline
    \textbf{Cminorsel} & $\mathcal{L}_\kw{C}$ & $\mathcal{L}_\kw{C}$ &
      Like Cminor, with machine-specific operators and addressing modes. \\
    \kw{RTLgen} & \kw{extp} & \kw{extt} &
      Construction of the CFG, 3-address code generation. \\
    \hline
    \textbf{RTL} & $\mathcal{L}_\kw{C}$ & $\mathcal{L}_\kw{C}$ &
      Register transfer language
      (3-address code, control-flow graph, infinitely many pseudo-registers). \\
    -- & $\kw{injn}; \kw{inj};$ & $\kw{injn}; \kw{inj};$ & \emph{RTL properties} \\
    \kw{Allocation} & \kw{wt}; \kw{ext}; &
                      \kw{wt}; \kw{ext}; &
      Register allocation \\
    & \kw{alloc} & \kw{alloc} & \\
    \hline
    \textbf{LTL} & $\mathcal{L}_\kw{loc}$ & $\mathcal{L}_\kw{loc}$ &
      Location transfer language
      (3-address code, control-flow graph of basic blocks,
      finitely many physical registers, infinitely many stack slots). \\
    \kw{Linearize} & \kw{id} & \kw{id} &
      Linearization of the CFG \\
    \hline
    \textbf{Linear} & $\mathcal{L}_\kw{loc}$ & $\mathcal{L}_\kw{loc}$ &
      Like LTL, but the CFG is replaced by
      a linear list of instructions with explicit branches and labels \\
    \kw{Stacking} & \kw{wt};\kw{stacking} & \kw{wt};\kw{stacking} &
      Laying out the activation records \\
    \hline
    \textbf{Mach} & $\mathcal{L}_\kw{mach}$ & $\mathcal{L}_\kw{mach}$ &
      Like Linear, with a more concrete view of the activation record \\
    \kw{Asmgen} & \kw{asmgen} & \kw{asmgen} &
      Emission of assembly code \\
    \hline
    \textbf{Asm} & \kw{\bf li\_asm} & \kw{\bf li\_asm} &
      Assembly language for x86 machines \\
    \hline
  \end{tabular}
  \caption}}

Table~\ref{tbl:passes} shows the passes of Composable Compcert,
together with the calling conventions for external calls and module interaction
used at each pass.
The calling conventions for each pass are chosen for convenience:
they reflect most closely the way the proof was written
in Compcert and CompCertX,
rather than a particularly useful or meaningful theorem for that pass.
In particular,
note that for most passes
the calling convention used for external calls is different from
that used for the module's outer interface,
preventing horizontal compositionality.
This section explains how the passes can nonetheless be composed
and how a satisfactory theorem can be derived for the whole compiler.

%}}}

%}}}

\section{Discussion and Related Work} %{{{

%}}}

\bibliographystyle{abbrv}
\bibliography{lwcc}

\end{document}
