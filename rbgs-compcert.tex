\documentclass[acmsmall,timestamp,review,anonymous]{acmart}

% Packages {{{
\usepackage{hyperref}
\usepackage{amsmath}
\usepackage{amssymb}
\usepackage{bbm}
\usepackage{tikz}
\usetikzlibrary{calc}
\usetikzlibrary{graphs}
\usetikzlibrary{cd}
\usepackage{bussproofs}
\usepackage{stmaryrd}
\usepackage{calc}
\usepackage{bbm}
% }}}

% Parameters {{{

\hyphenation{Comp-Cert}

% }}}

% Macros {{{

%\newtheorem{example}{Example}
%\newtheorem{definition}{Definition}
%\newtheorem{lemma}{Lemma}

\newcommand{\kw}[1]{\ensuremath{ \mathsf{#1} }}
\newcommand{\ifr}[1]{\ [{#1}]\ }
\newcommand{\ifrw}[2]{\ [{#1}]_{#2}\ }
\newcommand{\alt}{\ |\ } % use \mid instead
\newcommand{\bind}{\gg\!\!=}

\newcommand{\EC}{\kw{C}}
\newcommand{\simrel}{\kw{simrel}}

% Moves
\newcommand{\mcall}[3]{\kw{#1}({#2})@{#3}}
\newcommand{\pcall}[3]{%
  \underline{\mcall{#1}{#2}{#3}}%
}
\newcommand{\mret}[2]{{#1}@{#2}}
\newcommand{\pret}[2]{%
  \underline{\mret{#1}{#2}}%
}
\newcommand{\mretx}[3]{{#1}@{#2}/{#3}}
\newcommand{\pretx}[3]{%
  \underline{\mretx{#1}{#2}{#3}}%
}

% Pointers for justified sequences %{{{

% Parameters
\newcommand{\pshift}{1.6ex}
\newcommand{\pcdist}{2.5}
\newcommand{\pcangle}{60}

% Pointer hook
\newcommand{\ph}[1]{%
  \tikz[remember picture]{\coordinate (#1);}}

% Pointer to
\newcommand{\pt}[1]{%
  \rule{0pt}{1.4em}%
  \tikz[remember picture, overlay]{
    \draw[->]
      let \p{dest} = (#1),
          \n1 = {ln(veclen(\x{dest}, \y{dest}) + 1)},
          \p1 = ($(0,0)+(0,\pshift)$),
          \p4 = ($(#1)+(0,\pshift)$),
          \p2 = ($(\p1)!\n1*\pcdist!-\pcangle:(\p4)$),
          \p3 = ($(\p4)!\n1*\pcdist!+\pcangle:(\p1)$) in
        (\p1) .. controls (\p2) and (\p3) .. (\p4);}}

%}}}

% }}}

\title{Refinement-Based Game Semantics for CompCert}

\author{J\'er\'emie Koenig}
\affiliation{Yale University}
\email{jeremie.koenig@yale.edu}

\author{Zhong Shao}
\affiliation{Yale University}
\email{zhong.shao@yale.edu}

\begin{document}

\begin{abstract} %{{{
In recent years,
formal verification of computer systems
of increasing size has become practical.
Researchers have been able to verify complex artifacts
at a variety of abstraction levels spanning
from CPU designs to network protocols.

These achievements remain discrete efforts, but
there has been increasing interest in rendering them interoperable,
as exemplified by the DeepSpec project.
The ability to connect correctness proofs of components
verified at various levels of abstraction
would enable the construction of extremely reliable heterogenous systems
through end-to-end verification.

In this paper,
we propose that a successful synthesis of existing research on
game semantics,
refinement-based methods,
abstraction layers and
logical relations
has the potential to serve as a common theory
of certified components.
We apply this methodology to define
a compositional semantics for the verified compiler
CompCert,
and demonstrate that
the compiler's correctness proof can be extended
to this compositional setting
much more easily than using previous approaches.
\end{abstract}
%}}}

\maketitle

\section{Introduction} %{{{

% preamble {{{

Over the past decade,
researchers have been able to formally verify the
total functional correctness
of various key components of computer systems,
including
compilers \cite{compcert, vellvm},
operating system kernels \cite{sel4, popl15},
file systems \cite{fscq}, and
processor designs \cite{safe}.
Such verification efforts
consist of
a formal semantics describing the system's behavior,
a formal specification,
and a mechanized proof that
the system conforms to the specification.
Assuming the semantic model and specification accurately describe
the system's operation and requirements,
this approach can provide
strong reliability and security guarantees.
Moreover,
if the proof is mechanized in a general-purpose proof assistant,
it can be validated by a third-party with minimal effort
and without the need to trust the original developers.

%}}}

\subsection{Specifications as Intefaces} %{{{

Such achievements remain discrete efforts, but
there has been increasing interest in rendering them interoperable,
as exemplified by the DeepSpec project \cite{deepspec}.
By using formal specifications as interfaces
between the correctness proofs of various components ---
showing that the requirements of one component
are met by the guarantees of another ---
the researchers in DeepSpec hope to
construct correctness proofs for composite systems
operating over a wide range of abstraction layers.
If successful,
such an achievement would represent a deployment of formal methods
at a scale far beyond the current state of the art.

In addition,
this approach could provide reliability and security guarantees
even beyond those of existing formally verified components.
A certified component is only as trustworthy as its specification;
having specifications tied too much to a single verification effort
increases the risk of
implementation bugs being simply propagated to
the specification.
These bugs would become apparent in an effort to
connect the specification's guarantees to
the requirements of another component.
More generally,
specifications used as internal interfaces
in a larger certified system cease to be part of
the overall system's trusted base;
in this way,
building larger certified systems
by composing existing ones
would reduce the
ratio between the size of the external specification to
the overall size of the system.

%}}}

\subsection{The CompCert C Compiler} %{{{

These principles are illustrated in
the certified C compiler CompCert \cite{compcert}.
The semantics of the various intermediate languages
used by CompCert are formalized as transition systems,
and the correctness of each pass is proved
in a self-contained way
by establishing a simulation between
the source and target programs for that pass.
As individual passes are composed into a complete compiler,
their correctness proofs can be composed into
an overall semantic preservation theorem.
This final theorem no longer involves
the semantics of intermediate languages,
but only that of the source C program
and the final assembly program emitted by the compiler.

Since the introduction of CompCert a decade ago \cite{compcert},
there have been very successful efforts aimed at
interfacing it with other verification tools (VST),
using it as a component in larger verification projects (CertiKOS),
and refining its correctness theorem
to model real-world compiler use
in increasingly realistic detail
\cite{qompcert,sepcompcert,compcompcert,compcerttso,compcertshm}.
With each step,
the user can gain more confidence in the reliability of CompCert:
existing work testing the correctness of existing compilers
has found fewer bugs in CompCert,
compared to unverified alternatives \cite{csmith},
and efforts to make CompCert's correctness theorem more realistic
have uncovered and removed some of the few remaining bugs \cite{sepcompcert}.
Yet, most of this work
focuses on the reliability of the compiler
as an individual component.
The role this component plays in the construction of larger systems
is usually treated informally:
real-world use case scenarios are presented
to explain the meaning and justify the suitability
of the correctness theorem being proved.

However,
beyond \emph{system components that are certified},
achieving end-to-end verification of large-scale systems
will require \emph{components of certified systems},
which can in turn be used and composed
to build larger certified systems.
[In some ways,
CompCert illustrates the difficulty with compositionality:
to this day an appropriate compositional semantic model
for CompCert remains open question]

[Move some of this to the Related Work section?]

%}}}

\subsection{Challenges} %{{{

At present, many verification projects
use ad-hoc semantic models and specifications,
expressed in paradigms chosen first and foremost
to make modeling and verification tasks tractable.
Since verification is challenging,
the freedom to choose an appropriate semantic model
is quite valuable.
On the other hand,
the diversity of paradigms used for specification and verification
makes it difficult
to interface projects using different approaches.
These competing concerns
suggest the need to identify a common,
flexible semantic framework
into which existing models could be embedded
for integration purposes.

To enable the construction of certified systems
by assembling off-the-shelf components,
this framework should be equipped with
a rich composition infrastructure
and high-level principles for
reasoning about composite systems.
This represent another point of tension,
between the operational models
often used for verification
and the more compositional, denotational models
supporting equational reasoning.

%}}}

\subsection{Contributions} %{{{

In this paper,
we argue that
a successful synthesis of existing research
on game semantics,
refinement-based verification,
and certified abstraction layers
could provide a general framework
for the construction of large-scale certified systems.
We call this approach \emph{refinement-based game semantics}.

In \S\ref{sec:mainideas},
we examine from a more formal perspective
the principles enabling the construction
of large-scale certified systems.
We describe our high-level vision of
refinement-based game semantics,
and explain the ways in which
[we think it's up to the task].

Our first technical contribution
appears in \S\ref{sec:monad}:
we construct a semantic framework, formalized in the Coq proof assistant,
by augmenting a simple game model
with abstraction and refinement,
and we show how a traditional
trace-based strategy model
can be made more amenable to operational reasoning
by extending it with monadic structure.

While our game model remains much simpler than
those used in the semantics of high-order
functional programming languages,
we demonstrate its applicability in \S\ref{sec:modsem}
by using it to formulate a compositional module semantics
for CompCert.
In contrast with previous attempts \cite{compcompcert,cpp15},
our model incorporates \emph{abstraction},
taking into account the differences in
the nature of cross-module communication
between the various languages involved in CompCert.

This is used to full effect in \S\ref{sec:compcert},
where we show that a high-level
\emph{calling convention algebra}
can be used to render the correctness theorem of CompCert
fully compositional,
with only minimal changes to the existing correctness proofs
of compiler passes.

%}}}

%}}}

\section{Main Ideas} \label{sec:mainideas} %{{{

\subsection{Principles for system construction} %{{{

% preamble {{{

The goal of certified system design is
to create a formal description of
the system to be constructed (the program),
while ensuring through careful analysis that the system
will behave properly.
To carry out this analysis,
we must assign
to a system description $p \in P$
a mathematical object $\llbracket p \rrbracket \in \mathbb{D}$
representing its behavior.
We will call the set $\mathbb{D}$ a \emph{semantic domain}.
In this section we elucidate
the structure and properties of $\mathbb{D}$
necessary to the process of builing
large-scale certified systems.

%}}}

\subsubsection{Specifications and refinement} %{{{

System design starts with a set of requirements
that constrain the behavior of the system to be constructed
(the specification).
These requirements may not capture every detail
of the behavior of the eventual system,
but instead delineate a range of acceptable behaviors
from the point of view of its environment.

In refinement-based approaches,
programs and specifications are interpreted in the same
semantic domain $\mathbb{D}$,
which is equipped with a \emph{refinement} preorder
${\sqsubseteq} \subseteq \mathbb{D} \times \mathbb{D}$.
The proposition $\sigma_1 \sqsubseteq \sigma_2$
asserts that $\sigma_1$ is a more restrictive specification than $\sigma_2$,
and in particular a system description $p \in P$ is a correct implementation
of $\sigma \in \mathbb{D}$ when
$\llbracket p \rrbracket \sqsubseteq \sigma$.

%}}}

\subsubsection{Compositionality} %{{{

Complex systems are built by assembling components
whose behavior is understood,
such that their interaction achieves a desired effect.
The syntactic constructions of
the language used to describe systems
correspond to the ways in which they can be composed.

To enable compositional reasoning,
a suitable model must provide an account of
the behavior of the composite system
in terms of the behavior of its parts.
For instance,
if the language contains a binary operator
${+} : P \times P \rightarrow P$,
then the semantic domain should be equipped with
a corresponding operation
${\oplus} : \mathbb{D} \times \mathbb{D} \rightarrow \mathbb{D}$
such that:
$\llbracket p_1 + p_2 \rrbracket =
 \llbracket p_1 \rrbracket \oplus \llbracket p_2 \rrbracket$.

%}}}

\subsubsection{Monotonicity} %{{{

Once a component has been shown to conform to a given specification,
say $\llbracket p_1 \rrbracket \sqsubseteq \sigma_1$,
we will want to abstract the component as a ``black box'',
so that further reasoning can be done in terms of
the component's specification rather than its implementation details.
To support this,
we must establish that semantic composition operators
are compatible with refinement:
\[ \sigma_1 \sqsubseteq \sigma_1' \wedge
   \sigma_2 \sqsubseteq \sigma_2' \Rightarrow
   \sigma_1 \oplus \sigma_2 \sqsubseteq \sigma_1' \oplus \sigma_2' \,. \]
Then to establish that
$\llbracket p_1 + p_2 \rrbracket \sqsubseteq \sigma$,
it is sufficient to show that
$\sigma_1 \oplus \llbracket p_2 \rrbracket \sqsubseteq \sigma$:
\[
   \llbracket p_1 + p_2 \rrbracket \: = \:
   \llbracket p_1 \rrbracket \oplus \llbracket p_2 \rrbracket \:\sqsubseteq\:
   \sigma_1 \oplus \llbracket p_2 \rrbracket \:\sqsubseteq\:
   \sigma
\]

%}}}

\subsubsection{Abstraction} %{{{

Large-scale systems operate across many layers of abstraction.
Each abstraction layer defines its own understanding of the interaction
between a component and its environment.
To relate abstraction layers we will need to give an explicit account
of how their formulations of the interaction correspond to one another.

For example,
at the physical level,
communication over a wire may involve a series of voltages through time,
but at a higher level of abstraction
we will only be concerned with the sequence of bytes
being transmitted.
Serial communication hardware (for instance a UART device)
serves as a bridge between these two views,
implementing a byte-oriented communication channel
in a voltage-oriented world;
to formally verify such a component we will need to
express the correspondance between these two
layers of abstraction.

This can be done by defining an concretization embedding
$\gamma : \mathbb{D}_s \rightarrow \mathbb{D}_t$
between the semantic domains
used to reason at different levels of abstraction.
This embedding should be monotonic,
and should ideally preserve any relevant structure of $\mathbb{D}_s$
with a counterpart in $\mathbb{D}_t$.

%}}}

\subsubsection{Compilers} %{{{

Abstraction and concretization are
also involved when reasoning about compilers.
For example,
between C and assembly,
interactions across compilation units
are understood very differently.
At the level of C,
cross-module interaction is defined in terms of
function calls,
where invoking a function consists of assigning values
to the function's parameters,
initializing a new stack frame,
and tranferring control to the function's code.
By contrast,
at the assembly level we think of cross-module
interactions simply as branching to an address
outside the current module with
registers in a certain state.

To state a correctness property for a C-to-assembly compiler
such as CompCert,
we will need to formulate the correspondance between the two,
in the form of a concretization function $\gamma$
explaining how a C invokation is realized at the assembly level.
For a compiler $C$ we can then state correctness as:
\[ \llbracket C(p) \rrbracket_t \sqsubseteq
   \gamma(\llbracket p \rrbracket_s) \,, \]
where
$\llbracket - \rrbracket_s$ gives the semantics of the source language and
$\llbracket - \rrbracket_t$ gives the semantics of the target language.
Note that $\gamma$ can be understood as the semantic counterpart to $C$,
and is a formalization of the C \emph{calling convention}
for the target architecture.

%}}}

\subsubsection{Heterogeneity} %{{{

[What happens to $\mathbb{D}$ in the context of typed languages,
how this can be used to gain expressivity and construct heterogenous systems.]

%}}}

%}}}

\subsection{Game semantics} %{{{

% preamble {{{

The mathematical study of the semantics of programming languages
has traditionally opposed denotational and operational approaches.
Operational semantics describes
the behavior of a program in terms of
a state evolving across time.
Denotational semantics is a more abstract approach,
whereby the meaning of a program fragment (its denotation)
is computed from the meanings of its constituents.
This compositionality makes denotational semantics
more amenable to some forms of large-scale reasoning,
but its abstract character makes it more difficult
to connect to the concrete behavior of the system.

Game semantics is a denotational approach that
incoroporates some operational aspects.
Each type in the language
is interpreted as a two-player game,
which corresponds the interaction
between a program component of this type
(the player \kw{P}, for \emph{proponent})
and its execution context
(the player \kw{O}, for \emph{opponent})
The behavior of a term or component
is then modeled as a strategy for \kw{P} in this game,
which specifies the next move of the component
for all relevant positions in the game.

Positions are usually identified with sequences of moves,
and strategies can be identified with the set of positions
that a component can reach.
In this sense,
game semantics is similar to
the trace semantics of process algebras.
However, game semantics is distinguished
by a strong polarization between
the system and the environment,
and a strong distinction between outputs and inputs.
This confers an inherent ``rely-guarantee'' flavor
to games which facilitates compositional reasoning
in the context of heterogenous systems \cite{cspgs}.

%}}}

\subsubsection{Games} %{{{

A game is usually defined by giving a set of moves $M$
that players can choose from,
as well as a specification of which
sequences of moves are considered valid.
For chess,
moves will be taken in the set $(\{a, \ldots, h\} \times \{1, \ldots, 8\})^2$,
and a sequence of moves may look like:
\[ e2e4 \cdot \underline{c7c5} \cdot c2c3 \cdot \underline{e7d5} \cdots \]
As a more relevant example,
the game $\mathcal{C}$ is defined in \S\ref{sec:compcert:li}
to model the semantics of C compilation units.
Its plays are of the form:
\[ f[sg](\vec{v})@m \cdot \underline{v'@m'} \]
The environment indicates
which function $f$ should be invoked
using what signature $sg$,
and specifies the values $\vec{v}$ of the arguments
as well as the state $m$ of the memory
at the time of invocation.
When the function returns,
the system replies with a move $v'@m'$,
transferring control back to the environment
with the return value $v'$
and the updated memory state $m'$.
For simplicity,
we will omit signatures and memory states
from examples given in this section.

We will focus on two-player, alternating games
where the environment and the system each contribute every other move.
The environment plays first.
Positions in the game can be given
as sequences of moves;
even-length sequences correpond to positions
where the environment is expected to move,
and odd-length sequences correspond to positions
where the system is expected to move.
Note that when typesetting examples,
we underline the moves of the second player,
namely the system.

Most game semantics
include additional structure
in the description of games.
The set of moves is usually partitioned
into proponent and opponent moves ($M = M^\kw{O} \uplus M^\kw{P}$),
and into questions and answers ($M = M^\kw{Q} \uplus M^\kw{A}$).
The structure of valid plays
is often constrained by an \emph{enabling relation} $\vdash$
on moves,
which prevents some moves from being played
until a related move has been made by the other player.
[We stick to first-order stuff and don't need any of this.]

%}}}

\subsubsection{Strategies} %{{{

Following trace semantics of process calculi,
traditional work on game semantics
represents strategies as
the set of positions that the player may reach.
Hence,
a strategy is a prefix-closed set of sequences of moves,
subject to certain well-formedness constraints.
For example,
the strategy $\sigma$ used by
the black player in the chess play above
contained the plays:
\[
  \sigma \: = \: \{
    e2e4, \quad
    e2e4 \cdot \underline{c7c5}, \quad
    e2e4 \cdot \underline{c7c5} \cdot c2c3, \quad
    \ldots
  \}
\]

Strategies are usually subject to a number
of well-formedness constraints beyond prefix closure.
Commonly used contraints include the following:
\begin{description}
\item[Receptivity]
  requires that all possible behaviors of the environment
  be included in the strategy,
  so that if $s \in \sigma$ is an even-length play and
  $m$ is an environment move such that $sm$ is a valid play of the game,
  then $sm \in \sigma$ as well.
\item[Determinism]
  prevents the strategy from allowing
  more than one behavior of the system at any given point,
  so that if $s \in \sigma$ is an odd-length play,
  then any moves $m, n$ of the system
  such that $sm \in \sigma$ and $sn \in \sigma$
  must be equal ($m = n$).
\item[Well-bracketing]
  only allows the most recently asked question
  to be answered at any given point.
  That is,
  the game proceeds according to a strict stack discipline.
\item[Innocence]
  requires that the behavior of a strategy
  only depends on the most recent move
  as well as the chain of moves enabling it.
  This means the strategy is not allowed to maintain
  private state across
  successive but independent queries.
\end{description}
Game semantics of PCF \cite{pcfajm,pcfho,gamesem99}
impose all of these constraints,
and subsequent work was able to model additional language features
by relaxing a combination of them.
Following \cite{gsnondet},
we will relax the determinism constraint in order
to model specifications permitting
a range of implementation behaviors.
The family of games we consider,
and the representation we use for strategies,
are restricted enough that
the well-bracketing and innocence constraints
will always be satisfied.

%}}}

\subsubsection{Compositionality} \label{sec:mainideas:gs:comp} %{{{

While the game $\mathcal{C}$ used as an example above
is extremely simple,
the expressive power of game semantics
comes from the way in which complex games can be derived from simple ones,
and used to interpret compound types.
In the following we recall a few standard constructions
and demonstrate how they can be used
to complete our semantic model for C compilation units.

For a game $A$,
the game $!A$ consists of multiple copies of $A$,
instantiated at the discretion of the environment.
For example,
the following is a play for the game $!\mathcal{C}$:
\[
    \kw{pow}(3,2) \cdot
    \underline{9} \cdot
    \kw{pi}() \cdot
    \underline{3.1416} \cdot
    \kw{pow}(5,0) \cdot
    \underline{1}
\]
Because $\mathcal{C}$ is an elementary game
and our strategies are alternating,
in this examples the different copies of $\mathcal{C}$
will always be played in succession.
However,
in richer models multiple copies may proceed
concurrently
and be played in an interleaved manner.

Another common construction is the game $A \multimap B$
which consist of the games $A$ and $B$ played side-by-side.
However, in $A$ the roles of the system and the environment
are reversed.
In our context,
this means the game will always start
with a move of $B$ played by the environment,
but the player may initiate the game $A$.
For instance, a strategy $\sigma$
for the game $\mathcal{C} \multimap \mathcal{C}$
may include plays of the form:
\[
    \sigma \: \ni \:
    \kw{pow}(3,2) \cdot
    \underline{\kw{mult}(3,3)} \cdot
    r \cdot
    \underline{r}
\]
Here,
the component's function $\kw{pow}$
is invoked (in the right-hand side game),
but instead of immediately returning a result,
the component first performs an external call to $\kw{mult}$.
After the environment replies to this request with the result $r$,
the component in turn answers the invocation of $\kw{pow}$
with the same value.
If another component implements $\kw{mult}$,
so that its behavior is described by a strategy $\rho$
for the game $\mathcal{C}$ defined as follows:
\[ \rho := \{ \kw{mult}(n,p) \cdot \underline{r} \mid r = n p \} \,, \]
then their composition may include plays such as:
\[ \sigma \circ \rho \ni \kw{pow}(3,2) \cdot \underline{9} \]

The constructions presented above can be used together
to express the structure of various kinds of interactions.
For instance,
the type $!A \multimap B$ is traditionally used
to model the behavior of a $\lambda$-term of type $A \rightarrow B$,
which can access its argument multiple times.
In our context,
the game $!\mathcal{C} \multimap \mathcal{C}$
represents the way a C compilation unit behaves in response to
an activation,
while being able to perform any number of external calls.

%}}}

\subsubsection{Refinement} \label{sec:mainideas:gs:ref} %{{{

The distinction between system and environment actions
present in game semantics
leads to a notion of \emph{alternating} refinement:
a behavior $x$ refines a behavior $y$ if
all \emph{system} actions in $x$ are also possible in $y$, and if
all \emph{environment} actions in $y$ are also possible in $x$
\cite{altref,gmos}.
This makes it possible for specifications to
place constraints on the environment as well as the system,
and enables a rely-guarantee style of reasoning.

When the receptivity requirement is imposed on strategies,
all environment actions are always possible,
so that refinement is ``flattened'' back to trace containement.
However,
rely-guarantee reasoning can still be supported:
bad moves of the environment are technically allowed,
but the specification places no requirement on
the subsequent behavior of the system.
In \S\ref{sec:monad:def},
we will go one step further and introduce a distinguished
system action $\lightning$,
which is refined by all other actions,
and reifies this notion of undefined behavior.

%}}}

%}}}

\subsection{Logical relations} %{{{

In the broadest sense,
logical relations are structure-preserving relations,
in the same way that homomorphisms are structure-preserving maps.
However,
logical relations are more compositional than homomorphisms,
because they do not suffer from the same problems
in the presence of mixed-variance constructions,
such as the function arrow $\rightarrow$ \cite{lrp}.
This is a major advantage
for reasoning about typed languages,
where type-indexed logical relations
can be defined by recursion over the structure of types.

Logical relations have found widespread use in programming language theory.
Unary logical relations can be used to establish
various properties of type systems:
a type-indexed predicate expressing a property of interest
is shown to be compatible with the language's reduction,
and to contain all of the well-typed terms of the language.
Binary logical relations can be used to capture
contextual equivalence between terms,
as well as notions such as non-interference or compiler correctness.
Relational models of type quantification yield
Reynold's well-known theory of relational parametricity,
and can be used to prove \emph{free theorems} that
all terms of a given parametric type must satisfy.

Logical relations can be of any arity,
but in the present work
we will restrict our attention to
binary logical relations.
Given an algebraic structure $\mathcal{S}$
involving a number of operations over a carrier set,
a \emph{logical relation}
between two instances $S_1, S_2$ of $\mathcal{S}$
will be a relation $R$
between their carrier sets,
such that the corresponding operations of $S_1$ and $S_2$
take related arguments to related results.
We write $R : \mathcal{R}(S_1, S_2)$.

\begin{example}[Logical relation of monoids]
\label{ex:monoid}
A \emph{monoid} is a set $A$ equipped with
an associative binary operation $\cdot$ and
an identity element $\epsilon$.
A \emph{logical relation of monoids} between
a monoid $\langle A, \cdot_A, \epsilon_A \rangle$ and
a monoid $\langle B, \cdot_B, \epsilon_B \rangle$
is a relation $R \subseteq A \times B$
such that:
\begin{gather*}
u \ifr{R} u' \wedge v \ifr{R} v' \Rightarrow u \cdot_A v \ifr{R} u' \cdot_B v' \\
\epsilon_A \ifr{R} \epsilon_B
\end{gather*}
\end{example}

Logical relations between multisorted structures
will include one relation for each sort,
between the corresponding carrier sets.
In the case of structures which include type operators,
we can associate to each base type $A$
a relation over its carrier set $\llbracket A \rrbracket$,
and to each type operator $T(A_1, \ldots, A_n)$
a corresponding \emph{relator}:
given relations $R_1, \ldots, R_n$ over
the carrier sets $\llbracket A_1 \rrbracket, \ldots, \llbracket A_n \rrbracket$,
the relator for $T$
will construct a relation $T(R_1, \ldots, R_n)$
over $\llbracket T(A_1, \ldots, A_n) \rrbracket$.

\begin{figure} % fig:relators {{{
  {\small
  \begin{align*}
    x \ifr{R_1 \times R_2} y \ \Leftrightarrow\  &
      \pi_1(x) \ifr{R_1} \pi_1(y) \wedge
      \pi_2(x) \ifr{R_2} \pi_2(y) \\
    x \ifr{R_1 + R_2} y \ \Leftrightarrow\  &
      (\exists \, x_1 \, y_1 \,.\,
        x_1 \ifr{R_1} y_1 \wedge
        x = i_1(x_1) \wedge
        y = i_1(y_1)) \\ \vee\ &
      (\exists \, x_2 \, y_2 \,.\,
        x_2 \ifr{R_2} y_2 \wedge
        x = i_2(x_2) \wedge
        y = i_2(y_2)) \\
    f \ifr{R_1 \rightarrow R_2} g \ \Leftrightarrow\  &
      \forall \, x \, y \,.\,
        x \ifr{R_1} y \Rightarrow
        f(x) \ifr{R_2} g(y) \\
    A \ifr{\mathcal{P}^\le(R)} B \ \Leftrightarrow\  &
      \forall \, x \in A \,.\,
      \exists \, y \in B \,.\,
      x \ifr{R} y \\
    A \ifr{\mathcal{P}^\ge(R)} B \ \Leftrightarrow\  &
      \forall \, y \in B \,.\,
      \exists \, x \in A \,.\,
      x \ifr{R} y
  \end{align*}
  }%
  \caption{A selection of relators}
  \label{fig:relators}
\end{figure}
%}}}

Relators for some common constructions are shown in Fig.~\ref{fig:relators}.
Note that the first requirement given in Example~\ref{ex:monoid} above
can be expressed as:
\[
  \cdot_A \ifr{R \times R \rightarrow R} \cdot_B
\]

Logical relations used to reason about contextual equivalence
are often partial equivalence relations (PER).
By contrast, since we mainly focus on refinement,
most of the relations we consider will not be symmetric.

%}}}

\subsection{Kripke logical relations} %{{{
\label{sec:klr}

For stateful languages,
which terms should be related
will often depend on the current state.
This motivated the introduction of Kripke logical relations,
which are parametrized over a set of state-dependent \emph{worlds}.
Different components related at the same world
will be guaranteed to be related in compatible ways.

In the following,
we give a general account of Kripke logical relations
by drawing on their connection with
the Kripke semantics of modal logic.
We apply this framework
in our treatment of refinement and abstraction
in the context of game semantics (\S\ref{sec:monad:abs}).
Then in \S\ref{sec:compcert:cklr},
we use it to develop a logical-relations
understanding of some key aspects of CompCert,
and show how parametricity
can be used to derive important properties
of CompCert languages.

\begin{definition}
For a given set $W$,
a \emph{Kripke logical relation} is
$W$-indexed family of logical relations $(R_w)_{w \in W}$.
We write $R : \mathcal{R}_W(S_1, S_2)$
for a Kripke logical relation between structures $S_1$ and $S_2$,
and define the following relations:
\begin{align*}
  x \ifr{w \Vdash R} y &\Leftrightarrow x \ifr{R_w} y \\
  x \ifr{\Vdash R} y &\Leftrightarrow \forall w \,.\, x \ifr{R_w} y \,.
\end{align*}
\end{definition}

\begin{example}[CompCert memory injections] \label{ex:meminj} %{{{
The memory model of CompCert divides the memory into blocks,
whose contents are addressed by an integer
(see \S\ref{sec:compcert:mm}).
Between the states of the source and target programs,
a block may be dropped, added, or
mapped at a given offset within a larger block.
These transformation of the memory structure
are specified by partial functions of type:
\[
  \kw{meminj} := \kw{block} \rightharpoonup \kw{block} \times \mathbb{Z} \,,
\]
The simulation relations of CompCert,
and the component relations that they are built from,
are often parametrized by such a partial function $f : \kw{meminj}$,
which we call an \emph{injection mapping}.

For example,
CompCert represents pointers as pairs $(b, o)$, where
$b : \kw{block}$ designates a memory block, and
$o : \mathbb{Z}$ specifies an offset within the block.
The correspondance between source and target pointers
depends on the injection mapping being used.
To make sure pointers are related consistently
with each other and with the memory states,
in \S\ref{sec:compcert:mm}
we model this correspondance as a $\kw{meminj}$-indexed
Kripke logical relation $(R^\kw{ptr}_f)_{f : \kw{meminj}}$
defined by:
\[
    (b_1, o_1) \ifr{f \Vdash R^\kw{ptr}} (b_2, o_2) \:\Leftrightarrow\:
    f(b_1) = (b_2, o_2 - o_1) \,.
\]
\end{example}
%}}}

\subsubsection{Kripke relators}

A logical relation $R : \mathcal{R}(A, B)$
can be promoted to a $W$-indexed Kripke logical relation $\lceil R \rceil$
which ignores the index, so that $\lceil R \rceil_w = R$ for all $w \in W$.
Likewise,
a relator
  $F : \mathcal{R}(A_1, B_1) \,\times\,\cdots\,\times\,\mathcal{R}(A_n, B_n) \rightarrow \mathcal{R}(A, B)$
can be promoted to its Kripke version
by pointwise extension over the set of possible worlds:
\begin{gather*}
  \lceil F \rceil : \mathcal{R}_W(A_1, B_1) \times \cdots \times \mathcal{R}_W(A_n, B_n) \rightarrow \mathcal{R}_W(A, B) \\
  \lceil F \rceil (R_1, \ldots, R_n)_w = F(R_{1,w}, \ldots, R_{n,w})
\end{gather*}
We use $\lceil - \rceil$ implicitly
when a relator appears in a context where
a Kripke logical relation is expected.

\subsubsection{Modalities}

Kripke logical relations
can be used to ensure that the components of complex states
are related consistently (at the same world).
By adding structure to the set of worlds,
we will be able to go one step further and
specify how these worlds can evolve,
for instance across time.

\begin{definition} %{{{
A \emph{Kripke frame} is a tuple
$\langle W, {\leadsto} \rangle$, where
$W$ is a set of \emph{possible worlds} and
$\leadsto$ is a
binary \emph{accessibility relation} over $W$.
For a Kripke frame
$\langle W, \leadsto \rangle$,
we define the Kripke relators $\Diamond$, $\Box$ as follows:
\begin{align*}
  x \ifr{w \Vdash \Diamond R} y & \: \Leftrightarrow \:
    \exists \, w' \,.\, w \leadsto w' \wedge
      x \ifr{w' \Vdash R} y \\
  x \ifr{w \Vdash \Box R} y & \: \Leftrightarrow \:
    \forall \, w' \,.\, w \leadsto w' \Rightarrow
      x \ifr{w' \Vdash R} y
\end{align*}
\end{definition}
%}}}

Building on Example~\ref{ex:meminj},
we continue to examine the ways in which
some aspects of CompCert can be understood
in terms of Kripke logical relations.

\begin{example}[CompCert simulation diagrams] \label{ex:sim} %{{{
Many simulation diagrams in CompCert
involve a memory injection.
We can make them more compositional by
expressing them in the framework of Kripke logical relations.
Making the injection mapping explicit,
such a simulation relation will be a Kripke logical relation
$R \in \mathcal{R}_\kw{meminj}(A, B)$,
which will relate two transition systems
$\alpha : A \rightarrow \mathcal{P}(A)$ and
$\beta : B \rightarrow \mathcal{P}(B)$
according to the diagram:
\[
  \begin{tikzcd}
    s_1 \arrow[r, "\alpha"]
        \arrow[d, dash, "R_f"'] &
    s_1' \arrow[d, dashed, dash, "R_{f'} \quad (f \subseteq f')"] \\
    s_2 \arrow[r, dashed, "\beta"] &
    s_2'
  \end{tikzcd}
\]
More precisely, this diagram represents the following condition:
\[
    \forall f \, s_1 \, s_2 \, s_1' \,.\,
      s_1 \ifr{f \Vdash R} s_2 \wedge
      \alpha(s_1) \ni s_1' \Rightarrow
    \exists f' \, s_2' \,.\,
      \beta(s_2) \ni s_2' \wedge
      f \subseteq f' \wedge
      s_1' \ifr{f' \Vdash R} s_2'
\]
Here, the new states may be related according to
a new memory injection $f'$,
but in order to preserve existing relationships
between source and target pointers,
the new memory injection should include
the original one ($f \subseteq f'$).

This pattern is very common in CompCert
and appears in a variety of contexts.
By using $\langle \kw{meminj}, {\subseteq} \rangle$
as a Kripke frame,
we can express it concisely and compositionally
in our relational framework.
For instance,
the simulation diagram shown above can be written as:
\[
  \alpha \ifr{\Vdash R \rightarrow \mathcal{P}^\le(\Diamond R)} \beta \,.
\]
\end{example}
%}}}

%}}}

%}}}

\section{Semantic Framework} \label{sec:monad} %{{{

\subsection{Overview} %{{{

As a first step towards
the development of refinement-based game semantics,
we present a semantic framework
integrating refinement and abstraction
into a simple game model.
To facilitate the expression of operational constructions
within this framework,
we formulate our model around an \emph{interaction monad},
which generalizes strategies to enable
sequential composition and iteration.

\subsubsection{Games} %{{{

We will consider the alternating, well-bracketed, receptive strategies
for the game $!A \multimap B$ (see \S\ref{sec:mainideas:gs:comp})
where $A$ and $B$ are games with a very simple structure.

\begin{definition} \label{def:elemgame} % elementary game {{{
An \emph{elementary game} is a pair
$A = \langle M_A^\kw{Q}, M_A^\kw{A} \rangle$, where
$M_A^\kw{Q}$ is a set of \emph{questions}, and
$M_A^\kw{A}$ is a set of \emph{answers}.
\end{definition}
%}}}

Elementary games proceed as follows:
first $\kw{O}$ chooses a question,
then $\kw{P}$ chooses an answer.
More interestingly,
the composite game $!A \multimap B$,
consists of
an instance of $B$, played concurrently with
instances of $A$ in which the roles of $\kw{P}$ and $\kw{O}$ are exchanged.
Hence,
the opponent moves for the game $!A \multimap B$ are
$M^\kw{O}_{!A \multimap B} = M^\kw{Q}_B + M^\kw{A}_A$,
the proponent moves are
$M^\kw{P}_{!A \multimap B} = M^\kw{A}_B + M^\kw{Q}_A$,
and alternating, well-bracketed plays
proceed as follows:
\begin{itemize}
  \item The environment first plays a question from the set $M_B^\kw{Q}$.
  \item The system may then ask a question from $M_A^\kw{Q}$,
    which will be answered by the environment with a move in $M_A^\kw{A}$.
    This can be iterated any number of times, at the system's discretion.
  \item The system then plays a move from $M_B^\kw{A}$
    to answer the environment's original question.
\end{itemize}
%\[
%  \begin{tikzpicture}[baseline=(O.base)]
%    \node (I) at (0,0) {};
%    \node (P) at (1,0) {\kw{P}};
%    \node (O) at (2,0) {\kw{O}};
%    \path [->] (I) edge[bend left] node[auto] {$m$} (P);
%    \path [->] (P) edge[bend left] node[auto] {$n$} (O);
%    \path [->] (P) edge[bend left] node[auto] {$q$} (I);
%    \path [->] (O) edge[bend left] node[auto] {$r$} (P);
%  \end{tikzpicture}
%  \quad
%  \begin{array}{c@{\,}l@{\quad}c@{\,}l}
%    m &\in M_B^\kw{Q} & q &\in M_A^\kw{Q} \\[1ex]
%    n &\in M_B^\kw{A} & r &\in M_A^\kw{A}
%  \end{array}
%\]

The simplicity of our ``ground-level'' games
and the fact that we remain within the realm of
first-order types
makes our game model much simpler than
those typically used in the semantics of functional languages.
Nevertheless,
our model is expressive enough to model low-level components
of the kind treated in \S\ref{sec:modsem}.

%}}}

\subsubsection{Strategies} %{{{

Following the usual approach,
we will represent strategies as prefix-closed sets of alternating plays.
Because we limit ourselves to receptive strategies,
odd-length plays can remain implicit and be left out of the representation.
Taking well-bracketing into account,
a first approximation of the type of strategies
for the game $!A \multimap B$ is:
\[
  \mathcal{S}_{!A \multimap B} \: \approx \:
   \big\{ \sigma \subseteq
      M^\kw{Q}_B
      \big( M^\kw{Q}_A M^\kw{A}_A \big)^*
      \big( M^\kw{Q}_A \cup M^\kw{A}_B \big) \: \big| \:
      \sigma \mbox{ prefix-closed} \big\}
\]

%}}}

\subsubsection{Composition} \label{sec:sem:games:comp} %{{{

Consider two strategies
$\sigma : \mathcal{S}_{!A \multimap B}$ and
$\tau : \mathcal{S}_{!B \multimap C}$.
In the composite strategy
$\tau \circ \sigma : \mathcal{S}_{!A \multimap C}$,
$\sigma$ and $\tau$ interact
with each other over their common game $B$,
and with the environment over the respective games $A$ and $C$.
Note that because $\tau$ may invoke $\sigma$ multiple times,
we need to first promote $\sigma$ to a strategy
$\sigma^\dagger : \mathcal{S}_{{!A} \multimap {!B}}$,
which duplicates the behavior of $\sigma$ to answer several independent questions.
Writing $s \restriction X,Y$ for
the subsequence of $s$ consisting of the moves played in the games $X$ or $Y$,
the traditional formulation of composition
can be given as:
\[
    \tau \circ \sigma =
    \{ s \restriction {A,C} \mid
       s \restriction {A,B} \in \sigma^\dagger \wedge
       s \restriction {B,C} \in \tau \} \,.
\]

Note that composition may turn two reactive strategies
into a strategy exhibiting silent divergence.
This will happen in cases where
the two strategies only interact over the middle game ($s \in B^*$).
Since in this case $s \restriction A,C$ will be empty,
the formulation given above is appropriate when
silent divergence is identified with the empty interaction.
However,
in the presence of nondeterminism,
this representation of divergence is insufficient \cite{gsnondet}.
In \S\ref{sec:monad:int}
we present an alternative
account of strategy composition,
suitable in the context of our framework.

%}}}

\subsubsection{Nondeterminism} %{{{

Since we wish to model specifications
allowing a range of possible concrete behaviors,
we do not impose
the usual determinism requirement on strategies.
At any given point,
a strategy may allow several possible behaviors
of the system, or none,
and we will use trace containement as
the corresponding notion of refinement.
The introduction of nondeterminisim raises the question of
where divergence fits with respect to that refinement preorder.

%}}}

\subsubsection{Divergence} %{{{

Identifying silent divergence with $\bot$
would allow a diverging program to refine any specification.
Instead,
we wish to treat divergence as a behavior on par with others.
This means our strategy model must be extended
with an explicit representation of silent divergence.
In \cite{gsnondet},
this is accomplished by adjoining to every strategy
a set of the plays which may trigger silent divergence.
Equivalently,
we extend our model by allowing plays to contain
the special move $\Delta$ as the last move of the system.

%Note that by contrast with \cite{gsnondet},
%[b/c monad gives us just enough branching-time info,
%we don't run into problems with unbounded nondeterminism]

%}}}

\subsubsection{Abstraction} %{{{

We consider forms of abstractions where
the abstract and concrete interactions have the same structure,
but use different representations for their questions and answers.
The correspondance is be specified
using Kripke logical relations,
so that we can ensure that questions and answers are related consistently.
Once a concretization relation has been specified
for the constituent games,
this can be extended to a concretization embedding for strategies.
This is treated in detail in \S\ref{sec:monad:abs}.

%}}}

\subsubsection{Undefined Behaviors} %{{{

As discussed in \S\ref{sec:mainideas:gs:ref},
we will sometimes want to express undefined behaviors:
specifications which place no restriction on
the actual behavior of the system.
Such specifications could simply be expressed
as a nondeterministic choice between all possible defined behaviors,
however in the presence of abstraction this is unsatisfactory.

Specifically,
the concretization of undefined behaviors
should itself be fully undefined,
but the naive approach would yield
a nondeterminic choice between all defined concrete behaviors
\emph{for which there is a corresponding abstract behavior},
which may be too restricted.
To avoid such issues,
we introduce a distinguished representation for undefined behaviors,
refined by all others,
in the form of an another terminal move $\lightning$.

%}}}

\subsubsection{Sequential Compositionality} %{{{

Finally,
we generalize from strategies to
define the \emph{interaction monad} $\mathcal{I}_{M,N}(-)$
so that the well-bracketed strategies for $!A \multimap B$
correspond to the special case:
\[
    \mathcal{S}_{!A \multimap B} \: \approx \:
    M_B^\kw{Q} \rightarrow \mathcal{I}_{M_A^\kw{Q},M_A^\kw{A}}(M_B^\kw{A}) \,.
\]
A computation $x \in \mathcal{I}_{M,N}(X)$ in this monad
specifies the behavior of an open system,
which may at any point perform an output $m \in M$ and
wait for a subsequent input $n \in N$ from the environment.
This is modelled by the operation
$\mathbf{I} : M \rightarrow \mathcal{I}_{M,N}(N)$.
Following the approach outlined above,
the interaction monad also features nondeterminism
and is equipped with a notion of iteration $-^\infty$
treating silent divergence as a specific behavior.

%Additionally,
%to accomodate specifications which permit a range of possible behaviors,
%the interaction monad is equipped with a complete refinement lattice.
%Given $x, y : \mathcal{I}_{M,N}(A)$,
%their supremum $x \sqcup y : \mathcal{I}_{M,N}(A)$
%is the smallest specification that permits the behavior of either;
%it can also be interpreted as non-deterministic choice
%of the system.
%Conversely, $x \sqcap y : \mathcal{I}_{M,N}(A)$ can be interpreted as
%the largest specification requiring that $x$ and $y$ both be satisfied.
%The least element $\bot$
%is a specification that can never be satisfied;
%the greatest element $\top$
%is the specification that is always satisfied ---
%or represents a computation whose behavior is entierely undefined.
%
%Finally,
%we model non-deterministic iteration with the operator:
%\[
%     -^\infty : (A \rightarrow \mathcal{I}_{M,N}(A)) \rightarrow
%                (A \rightarrow \mathcal{I}_{M,N}(A)) \,.
%\]
%Notably,
%$-^\infty$ is different from
%the Kleene star associated with the refinement lattice,
%because we account for silent divergence as a specific behavior,
%incomparable with terminating ones,
%rather than identifying it with
%the unsatisfiable specification $\bot$
%or the undefined behavior $\top$.

%}}}

\begin{figure} % fig:notations {{{
  \begin{center}
    \begin{tabular}{lccccccc}
      \hline
      Category & Appl. & Comp. & Unit & Join & Zero & Finite & Infinite \\
      \hline
      Functions & $f(x)$ & $f \circ g$ & \kw{id} & & & \\
      Monad &
        $x \bind f$ & $g \cdot f$ & $\mathbf{1}$ &
        ${\sqcup}, {\oplus}$ & $\bot, \mathbf{0}$ & $*$ & $\infty$ \\
      Interaction &
        $x[f]$ & $g \odot f$ & $\mathbf{I}$ &
        ${\sqcup}, {\oplus}$ & $\bot, \mathbf{0}$ & & \\
      \hline
    \end{tabular}
  \end{center}
  \caption{Summary of notations}
  \label{fig:notations}
\end{figure}
%}}}

%[the monadic approach gives us just enough
%local/intermediate branching-time info
%that we can avoid the usual problems
%around unbounded nondeterminism.]

%}}}

\subsection{The Interaction Monad} \label{sec:monad:def} %{{{

We now formally define the monad $\mathcal{I}_{M,N}$,
starting with the representation of plays. 
Note that
in an interactive behavior
$x \in \mathcal{I}_{M,N}(X)$,
the system is active first,
and may return control to the environment
by performing an output.
The plays we will consider for $\mathcal{I}_{M,N}(X)$
therefore start with a move $m \in M$ of player $\kw{P}$
and be of odd length,
and we use Kleisli morphisms
($f : A \rightarrow \mathcal{I}_{M,N}(B)$)
to encode strategies
which expect $\kw{O}$ to play first.

To support the monadic structure and account
for silent divergence and undefined behaviors,
we augment the set $M$ of outputs with the
distinguished terminal moves
$v \in X$ (returned value),
$\Delta$ (silent divergence), and
$\lightning$ (undefined behavior):
\[
    s, t \in
    \mathcal{T}_{M,N}(X) ::=
    v \mid \Delta \mid \lightning \mid m \mid mnt \,.
\]
Any trace refines $\lightning$, so that
the extended prefix relation on $\mathcal{T}_{M,N}(X)$
is defined by the rules:
\begin{gather*}
  v \sqsubseteq v , \quad
  \Delta \sqsubseteq \Delta , \quad
  t \sqsubseteq \lightning , \quad
  m \sqsubseteq m , \quad
  m \sqsubseteq mnt ,
  \quad
  \AxiomC{$s \sqsubseteq t$}
  \UnaryInfC{$mns \sqsubseteq mnt$}
  \DisplayProof
\end{gather*}
An interactive behavior is
a prefix-closed set of traces:
\[
    \mathcal{I}_{M,N}(X) :=
    \{ T \subseteq \mathcal{T}_{M,N}(X) \mid
       \forall t \in T \,.\, \forall s \sqsubseteq t \,.\, s \in T \}
\]

Note that since any trace is a prefix of $\lightning$,
a behavior which admits a trace ending with $\lightning$
will also admit all possible interactions
sharing the same initial segment.
This allows us to define our notion of refinement
as simple trace containment.
For $x, y \in \mathcal{I}_{M,N}(X)$, refinement is defined as:
\[
    x \sqsubseteq y \Leftrightarrow x \subseteq y
\]
Since unions and intersections
preserve prefix closure,
they induce a lattice structure on $\mathcal{I}_{M,N}(X)$.

%}}}

\subsection{Monad operations} %{{{

The monad's unit associates to each value $v \in X$
the interactive behavior with a single trace $v$:
\[
    \kw{ret}_X(v) := \{ v \} \,.
\]
The binding operation corresponds to
the sequential composition of
a behavior $x \in \mathcal{I}_{M,N}(A)$ and
a Kleisli morphism $f : A \rightarrow \mathcal{I}_{M,N}(B)$.
The result is an interactive behavior
$(x \bind f) \in \mathcal{I}_{M,N}(B)$
containing the traces of $x$ where
any final value $v$ has been replaced with
all possible traces in $f(v)$.
For a single trace $t \in x$ we define $t \bind f$
using the following rules:
\begin{gather*}
  \AxiomC{}
  \UnaryInfC{$\Delta \in (\Delta \bind f)$}
  \DisplayProof
  \quad
  \AxiomC{}
  \UnaryInfC{$t \in (\lightning \bind f)$}
  \DisplayProof
  \quad
  \AxiomC{}
  \UnaryInfC{$m \in (m \bind f)$}
  \DisplayProof
  \\[1ex]
  \AxiomC{$t \in f(v)$}
  \UnaryInfC{$t \in (v \bind f)$}
  \DisplayProof
  \quad
  \AxiomC{$s \in (t \bind f)$}
  \UnaryInfC{$mns \in (mnt \bind f)$}
  \DisplayProof
\end{gather*}
Then $x \bind f$ can be defined as:
\[
    (x \bind f) = \bigcup_{t \in x} (t \bind f)
\]
It is straightforward to verify that
the monad laws hold:
\begin{align*}
  (\kw{ret}(v) \bind f) &= f(v) \\
  (x \bind \kw{ret}) &= x \\
  (x \bind (v \mapsto f(v) \bind g)) &= ((x \bind f) \bind g) \,.
\end{align*}

The binding operation $\bind$
and the lattice structure on $\mathcal{I}_{M,N}(A)$
can be extended to Kleisli morphisms as follows:
\begin{align*}
    (f \cdot g)(a) &:= f(a) \bind g &
    \mathbf{1}(a) &:= \kw{ret}(a) \\
    (f \oplus g)(a) &:= f(a) \sqcup g(a) &
    \mathbf{0}(a) &:= \bot
\end{align*}
Together with the iteration principles
defined in the following section,
these operations constitute a structure
analogous to a typed Kleene algebra \cite{tka}.

%}}}

\subsection{Iteration} \label{sec:monad:iter} %{{{

A Kleisli morphism $f : A \rightarrow \mathcal{I}_{M,N}(A)$
can be iterated as follows.
We start by defining
the $n$-fold sequential composition of $f$
and the associated notion of Kleene star as:
\[
  \begin{array}{rl}
    f^0(a) &:= \kw{ret}(a) \\
    f^{n+1}(a) &:= f^n(a) \bind f
  \end{array}
  \qquad
  f^*(a) := \bigsqcup_n f^n(a) \,.
\]
In order to recognize silent divergence,
we introduce $f_\Delta$,
defined by the following coinductive rule.
Note that $f_\Delta(v) \sqsubseteq \{\Delta\}$.
\[
    \AxiomC{$v' \in f(v)$}
    \AxiomC{$\Delta \in f_\Delta(v')$}
    \doubleLine
    \BinaryInfC{$\Delta \in f_\Delta(v)$}
    \DisplayProof
\]
Then the non-deterministic, infinite iteration of $f$ is:
\[
    f^\infty := (f \oplus f_\Delta)^*
\]

%}}}

\subsection{Sets and Functions} \label{sec:monad:powerset} %{{{

The powerset monad $\mathcal{P}$
can be embedded into the monad $\mathcal{I}_{M,N}$
through the natural transformation
$\eta^\mathcal{P}_X : \mathcal{P}(X) \rightarrow \mathcal{I}_{M,N}(X)$
defined as
$\eta^\mathcal{P}_X(V) := \{ v \mid v \in V \}$.
In particular,
a relation $R : A \rightarrow \mathcal{P}(B)$
can be interpreted as
$\eta^\mathcal{P}_B \circ R : A \rightarrow \mathcal{I}_{M,N}(B)$.

\begin{example} \label{ex:ts} % Transition system {{{
Consider a transition system $\alpha = (S, I, {\rightarrow}, F)$,
where
$S$ is a set of states,
$I : \mathcal{P}(S)$
is a set of initial states,
${\rightarrow} : S \rightarrow \mathcal{P}(S)$
is a transition relation, and
$F : S \rightarrow \mathcal{P}(A)$
associates potential output values to each state.
The behavior of $\alpha$ can be expressed as:
\[
    \llbracket \alpha \rrbracket :=
    \eta^\mathcal{P}_S(I) \bind
    (\eta^\mathcal{P}_S \circ {\rightarrow})^\infty \bind
    (\eta^\mathcal{P}_S \circ F)
    : \mathcal{I}_{M,N}(A) \,.
\]
\end{example}
%}}}

We use a similar approach in \S\ref{sec:modsem:def}
to express the behavior of CompCert small-step semantics
in terms of interactive behaviors.
For conciseness,
we will sometimes rely implicitly on $\eta_X^\mathcal{P}$
when a set $x \in \mathcal{P}(A)$ is used
in a context where an interactive behavior
of type $\mathcal{I}_{M,N}(A)$ is expected,
for instance expressing $\llbracket \alpha \rrbracket$ above as
$I \bind {\rightarrow}^\infty \bind F$.

Given a function $f : A \rightarrow B$,
its \emph{preimage}
$f^{-1} : B \rightarrow \mathcal{I}_{M,N}(A)$
can be defined as:
\[
    f^{-1}(b) = \eta^\mathcal{P}_A(\{ a \mid f(a) = b \}) \,.
\]
For instance, given two Kleisli morphisms
$f : A \rightarrow \mathcal{I}_{M,N}(X)$ and
$g : B \rightarrow \mathcal{I}_{M,N}(X)$,
their copairing $[f, g] : A + B \rightarrow \mathcal{I}_{M,N}(X)$
in the Kleisli category can be expressed as
$(i_1^{-1} \cdot f) \oplus (i_2^{-1} \cdot g)$.
For the sake of symmetry we will sometimes implicitly promote
$f : A \rightarrow B$ to the Kleisli morphism
$\kw{ret} \circ f : A \rightarrow \mathcal{I}_{M,N}(B)$.

%}}}

\subsection{Interaction} \label{sec:monad:int} %{{{

The interaction primitive
$\mathbf{I}_{M,N} : M \rightarrow \mathcal{I}_{M,N}(N)$
can be defined as follows:
\[
    \mathbf{I}_{M,N}(m) := \{ m, mnn \mid n \in N \}
\]
Note that in the trace $mnn$,
the first occurence of $n$ denotes an input,
whereas the second one denotes the value returned by $\mathbf{I}$.

Interaction introduces a second notion of composition
besides the one induced by the $\bind$ operator:
given
$x \in \mathcal{I}_{M,N}(A)$ and
$f : M \rightarrow \mathcal{I}_{P,Q}(N)$,
the behavior $x[f] : \mathcal{I}_{P,Q}(A)$
will proceed according to $x$,
but whenever $x$ attempts to perform an output $m \in M$,
the behavior $f(m)$ will be substituted;
if $f(m)$ returns a value $n \in N$,
the value will be used as an input to resume to evaluation of $x$.
This corresponds to the composition of strategies
discussed in \S\ref{sec:mainideas:gs:comp}.

To formally define the behavior $x[f]$,
we first introduce the following decomposition of $x$:
\[
    x \: = \: \rho(x) \: \sqcup \:
        (\mu(x) \bind m \mapsto \mathbf{I}(m) \bind n \mapsto \delta_{mn}(x))
\]
The residual $\rho(x)$
contains the traces of $x$ that are of the form $v$, $\Delta$ or $\lightning$
(closed under the prefix relation as appropriate).
The behavior $\mu(x)$ contains the traces of $x$ of the form $m$.
The derivative $\delta_{mn}(x)$ is defined as
$\delta_{mn}(x) := \{ t \mid mnt \in x \}$.
Substitution can then be defined as follows.

\begin{definition}[Substitution]
For
$x \in \mathcal{I}_{M,N}(A)$ and
$f : M \rightarrow \mathcal{I}_{P,Q}(N)$,
we define the interactive substitution
$x[f] \in \mathcal{I}_{P,Q}(A)$ as:
\[
    x[f] \: :=
      \kw{ret}(x) \bind
      (s \mapsto \mu(s) \bind
       m \mapsto f(m) \bind
       n \mapsto \kw{ret}(\delta_{mn}(s)))^\infty \bind
      \rho \,.
\]
Additionally, for
$g : A \rightarrow \mathcal{I}_{M,N}(B)$,
the interactive composition $g \odot f : A \rightarrow \mathcal{I}_{P,Q}(B)$
is defined by $(g \odot f)(a) := g(a)[f]$.
\end{definition}

The definition above uses a transition system
over states $s \in \mathcal{I}_{M,N}(A)$.
At any point,
$x[f]$ exhibits the immediate behavior $\rho(s)$.
If $s$ attempts to interact with an output $m$,
$x[f]$ will exhibit the behavior $f(m)$ instead,
and continue whenever $f$ returns $n$
in the next state $\delta_{mn}(s)$.

The interactive substitution and composition
enjoy the following properties:
\begin{align*}
  x[\mathbf{I}] &= x &
  \mathbf{I}(m)[f] &= f(m) &
  x[g \odot f] &= x[g][f] \\
  g \odot \mathbf{I} &= g &
  \mathbf{I} \odot f &= f &
  h \odot (g \odot f) &= (h \odot g) \odot f
\end{align*}

%}}}

\subsection{Abstraction} \label{sec:monad:abs} %{{{

We now consider the problem of relating interactive behaviors
whose inputs, outputs, and results are taken in different sets.
Specifically,
for a \emph{concrete} behavior
$x^\natural : \mathcal{I}_{M^\natural,N^\natural}(X^\natural)$
on one hand, and an \emph{abstract} behavior
$x^\sharp : \mathcal{I}_{M^\sharp,N^\sharp}(X^\sharp)$
on the other hand,
we wish to describe the correspondance between
the two interactions
and establish that
$x^\natural$ is indeed a sound realization of $x^\sharp$.

We use Kripke logical relations to ensure that
outputs are related consistently with their subsequent inputs.
For
a relation $R_M : \mathcal{R}_W(M^\natural, M^\sharp)$ on outputs,
a relation $R_N : \mathcal{R}_W(N^\natural, N^\sharp)$ on inputs,
and a relation $R_X : \mathcal{R}(X^\natural, X^\sharp)$ on results,
our goal will be to define a relator:
\[ \mathcal{I}^\le_{R_M,R_N}(R_X) \: : \:
   \mathcal{R}(\mathcal{I}_{M^\natural,N^\natural}(X^\natural),
               \mathcal{I}_{M^\sharp,N^\sharp}(X^\sharp)) \]
such that $x^\natural \ifr{\mathcal{I}^\le_{R_M,R_N}(R_X)} x^\sharp$
whenever $x^\natural$ is simulated by $x^\sharp$ according to
the correspondance specified by $R_M$, $R_N$ and $R_X$.
We will proceed by defining a concretization mapping:
\[ \gamma_{R_M,R_N,R_X} : \mathcal{I}_{M^\sharp,N^\sharp}(X^\sharp) \rightarrow
                    \mathcal{I}_{M^\natural,N^\natural}(X^\natural) \, \]
such that $\gamma_{R_M, R_N, R_X}(x^\sharp)$ is
the largest computation in $\mathcal{I}_{M^\natural,N^\natural}(X^\natural)$
simulated by $x^\sharp$.
Accordingly,
we will define:
\[ x^\natural \ifr{\mathcal{I}^\le_{R_M,R_N}(R_X)} x^\sharp
   \: \Leftrightarrow \:
   x^\natural \sqsubseteq \gamma_{R_M,R_N,R_X}(x^\sharp) \,. \]

\begin{definition}
For the relations $R_M : \mathcal{R}_W(M^\natural, M^\sharp)$,
$R_N : \mathcal{R}_W(N^\natural, N^\sharp)$,
$R_X : \mathcal{R}(X^\natural, X^\sharp)$, and
for the behavior $x^\sharp : \mathcal{I}_{M^\sharp, N^\sharp}(X^\sharp)$,
the behavior $\gamma_{R_M,R_N,R_X}(x^\sharp)$ is defined by the rules:
\begin{gather*}
  \AxiomC{$v^\natural \ifr{R_X} v^\sharp$}
  \AxiomC{$v^\sharp \in x^\sharp$}
  \BinaryInfC{$v^\natural \in \gamma_{R_M,R_N,R_X}(x^\sharp)$}
  \DisplayProof
  \qquad
  \AxiomC{$\Delta \in x^\sharp$}
  \UnaryInfC{$\Delta \in \gamma_{R_M,R_N,R_X}(x^\sharp)$}
  \DisplayProof
  \qquad
  \AxiomC{$\lightning \in x^\sharp$}
  \UnaryInfC{$t \in \gamma_{R_M,R_N,R_X}(x^\sharp)$}
  \DisplayProof
  \\[1ex]
  \AxiomC{$m^\natural \ifr{w \Vdash R_M} m^\sharp$}
  \AxiomC{$m^\sharp \in x^\sharp$}
  \BinaryInfC{$m^\natural \in \gamma_{R_M,R_N,R_X}(x^\sharp)$}
  \DisplayProof
  \\[1ex]
  \AxiomC{$m^\natural \ifr{w \Vdash R_M} m^\sharp$}
  \AxiomC{$m^\sharp \in x^\sharp$}
  \AxiomC{$
      \forall \, n^\sharp \,.\,
        n^\natural \ifr{w \Vdash R_N} n^\sharp \Rightarrow
        t^\natural \in \gamma_{R_M,R_N,R_X}(\delta(x^\sharp, m^\sharp)(n^\sharp))$}
  \TrinaryInfC{$m^\natural n^\natural t^\natural \in \gamma_{R_M,R_N,R_X}(x^\sharp)$}
  \DisplayProof
\end{gather*}
Given $x^\natural \in \mathcal{I}_{M^\natural,N^\natural}(X^\natural)$,
we say that
\emph{$x^\natural$ is simulated by $x^\sharp$ according to
$R_M$, $R_N$, and $R_X$} and write:
\[
    x^\natural \ifr{\mathcal{I}^\le_{R_M,R_N}(R_X)} x^\sharp
\]
whenever $x^\natural \sqsubseteq \gamma_{R_M,R_N,R_X}(x^\sharp)$.
\end{definition}

The mapping $\gamma$ preserves constructions
on relations in the following ways:
\begin{gather*}
\gamma_{=,=,=}(x) = x \\
\gamma_{R_M \cdot S_M, R_N \cdot S_N, R_X \cdot S_X} =
  \gamma_{S_M, S_N, S_X} \circ \gamma_{R_M, R_N, R_X} \\
\gamma :: {\subseteq} \rightarrow {\subseteq} \rightarrow
  {\subseteq} \rightarrow {\sqsubseteq}
\end{gather*}
These properties ensure that
$\mathcal{I}^\le$ behaves as a \emph{relator} \cite{something}.
Specifically:
\begin{align*}
  \mathcal{I}^\le_{=,=}(=) &= {\sqsubseteq} \\
  \mathcal{I}^\le_{R_M \cdot S_M, R_N \cdot S_N}(R_X \cdot S_X) &=
    \mathcal{I}^\le_{R_M, R_N}(R_X) \cdot
    \mathcal{I}^\le_{S_M, S_N}(S_X) \\
  R_M \subseteq S_M \wedge
  R_N \subseteq S_N \wedge
  R_X \subseteq S_X &\Rightarrow
    \mathcal{I}^\le_{R_M, R_N}(R_X) \subseteq
    \mathcal{I}^\le_{S_M, S_N}(S_X)
\end{align*}

We can now formulate the following relational properties,
which describe the behavior of the monad's primitives
with respect to abstraction
(and refinement when $R_M$, $R_N$ and $R_X$ are $=$):
\begin{align*}
  \kw{ret} &:
    R_X \rightarrow \mathcal{I}^\le_{R_M,R_N}(R_X) \\
  \bind &:
    \mathcal{I}^\le_{R_M,R_N}(R_X) \rightarrow
    (R_X \rightarrow
     \mathcal{I}^\le_{R_M,R_N}(R_Y)) \rightarrow
    \mathcal{I}^\le_{R_M,R_N}(R_Y) \\
  \mathbf{I} &:
    {}\Vdash R_M \rightarrow
    \mathcal{I}^\le_{R_M,R_N}(R_N) \\
%  \kw{next} &:
%    (\Vdash \mathcal{I}^\le_{R_M,R_N}(R_X) \rightarrow
%     R_X +
%     \Diamond (R_M \times
%     \Box (R_N \rightarrow \mathcal{I}^\le_{R_M,R_N}(R_X)))) \\
  -^*,
  -^\infty &:
    (R_X \rightarrow \mathcal{I}^\le_{R_M,R_N}(R_X)) \rightarrow
    (R_X \rightarrow \mathcal{I}^\le_{R_M,R_N}(R_X)) \\
  \eta^\mathcal{P} &:
    \mathcal{P}^\le(R_X) \rightarrow
    \mathcal{I}^\le_{R_M,R_N}(R_X)
\end{align*}
Together,
these properties allow us to construct
heterogenous simulations
between monadic terms with similar structures.

\begin{example} \label{ex:sim}
Building on Example~\ref{ex:ts},
consider
$\alpha_1 = (S_1, I_1, {\rightarrow}_1, F_1)$ and
$\alpha_2 = (S_2, I_2, {\rightarrow}_2, F_2)$
two transition systems,
together with a relation
$R : \mathcal{R}(S_1, S_2)$
satisfying:
\begin{gather*}
  I_1 \ifr{\mathcal{P}^\le(R)} I_2 \\
  {\rightarrow}_1 \ifr{R \rightarrow \mathcal{P}^\le(R)} {\rightarrow}_2 \\
  F_1 \ifr{R \rightarrow \mathcal{P}^\le(=)} F_2
\end{gather*}
That is, $R$ is a simulation relation between $\alpha_1$ and $\alpha_2$.
Then by using the properties above and
following the structure of $\llbracket - \rrbracket$,
we can show that:
\[
    \llbracket \alpha_1 \rrbracket \sqsubseteq
    \llbracket \alpha_2 \rrbracket \,.
\]
\end{example}

%}}}

\subsection{Strategies and Simulation Conventions} \label{sec:monad:games} %{{{

We conclude the exposition of our framework
by defining our representation
of strategies for the game $!A \multimap B$,
where $A$ and $B$ are elementary games.
We write:
\[ A \Rightarrow B := M_B^\kw{Q} \rightarrow
   \mathcal{I}_{M_A^\kw{Q},M_A^\kw{A}}(M_B^\kw{A}) \]
for the type of such strategies.
Strategies inherit the algebraic structures
of the interaction monad,
so that for example
given $\sigma : A \Rightarrow B$ and $\tau : B \Rightarrow C$,
their composition can be written as $\tau \odot \sigma$, and
given $\sigma, \tau : A \Rightarrow B$,
their least upper bound
can be written $\sigma \oplus \tau$.

In addition,
refinement and abstraction can also be lifted to strategies.
Given $\sigma, \tau : A \Rightarrow B$,
we will write $\sigma \sqsubseteq \tau$
for pointwise refinement,
that is whenever $\sigma \ifr{{=} \rightarrow {\sqsubseteq}} \tau$.
To talk about abstraction in the context of strategies,
we will introduce the following notion.

\begin{definition} % Simulation convention %{{{
A \emph{simulation convention} between the elementary games
$A^\natural = \langle M_{A^\natural}^\kw{Q}, M_{A^\natural}^\kw{A} \rangle$ and
$A^\sharp = \langle M_{A^\sharp}^\kw{Q}, M_{A^\sharp}^\kw{A} \rangle$
is as a tuple $\mathbb{R} = \langle W, R^\kw{Q}, R^\kw{A} \rangle$
with $R^\kw{Q} \in \mathcal{R}_W(M_{A^\natural}^\kw{Q}, M_{A^\sharp}^\kw{Q})$
and $R^\kw{A} \in \mathcal{R}_W(M_{A^\natural}^\kw{A}, M_{A^\sharp}^\kw{A})$.
We will write $\mathbb{R} : A^\natural \Leftrightarrow A^\sharp$.
Given two simulation conventions
$\mathbb{R} : A^\natural \Leftrightarrow A^\sharp$ and
$\mathbb{S} : B^\natural \Leftrightarrow B^\sharp$,
and for two strategies
$\sigma^\natural : A^\natural \Rightarrow B^\natural$ and
$\sigma^\sharp : A^\sharp \Rightarrow B^\sharp$,
we write $\sigma^\natural \le_{\mathbb{R} \Rightarrow \mathbb{S}} \sigma^\sharp$
and say that $\sigma^\natural$ is
\emph{simulated by $\sigma^\sharp$ according to the convention}
$\mathbb{R} \Rightarrow \mathbb{S}$
when the following property holds:
\[
  \sigma^\natural \le_{\mathbb{R} \Rightarrow \mathbb{S}} \sigma^\sharp
  \: :\Leftrightarrow \:
  \sigma^\sharp
  \ifr{\Vdash S^\kw{Q} \rightarrow
       \mathcal{I}_{R^\kw{Q}, R^\kw{A}}(S^\kw{A})}
  \sigma^\natural \,.
\]
\end{definition}
%}}}

We can define a Kleene algebra on simulation conventions as follows.
This will be used in \S\ref{sec:compcert}
as a crucial ingredient of our overall simulation convention for CompCert.

\begin{definition}[Constructions on simulation conventions] %{{{
Given two simulation conventions
$\mathbb{R}, \mathbb{S} : A^\natural \Leftrightarrow A^\sharp$,
we say that
$\mathbb{R}$ refines $\mathbb{S}$ and write
$\mathbb{R} \preceq \mathbb{S}$
when the following holds for all $w, m_1, m_2$:
\[
      m_1 \ifr{w \Vdash R^\kw{Q}} m_2 \Rightarrow
      \exists \, v \,.\,
        m_1 \ifr{v \Vdash S^\kw{Q}} m_2 \wedge
        \forall \, n_1 \, n_2 \,.\,
          n_1 \ifr{v \Vdash S^\kw{A}} n_2 \Rightarrow
          n_1 \ifr{w \Vdash R^\kw{A}} n_2
\]
The least upper bound of $\mathbb{R}$ and $\mathbb{S}$
is defined as
$
    \mathbb{R} \curlyvee \mathbb{S} :=
      \langle
        W_R + W_S,
        R^\kw{Q} + S^\kw{Q},
        R^\kw{A} + S^\kw{A}
      \rangle
$,
where for $R \in \mathcal{R}_{W_R}(X,Y)$
and $S \in \mathcal{R}_{W_S}(X,Y)$,
the Kripke logical relation $R + S$ is defined as:
\[
    (R + S)_w =
    \begin{cases}
      R_u & \mbox{if } w = i_1(u) \\
      S_v & \mbox{if } w = i_2(v) \,.
    \end{cases}
\]
Given
$\mathbb{R} : A^\flat \Leftrightarrow A^\natural$ and
$\mathbb{S} : A^\natural \Leftrightarrow A^\sharp$,
the composite convention
$\mathbb{R} \cdot \mathbb{S} : A^\flat \Leftrightarrow A^\sharp$ is defined as:
\[
    \mathbb{R} \cdot \mathbb{S} :=
      \langle
        W_R \times W_S, \:
        R^\kw{Q} \cdot S^\kw{Q}, \:
        R^\kw{A} \cdot S^\kw{A}
      \rangle \,.
\]
Finally,
for $\mathbb{R} : A \Leftrightarrow A$,
we define the $n$-fold composition
$\mathbb{R}^n := \mathbb{R} \cdot \mathbb{R} \cdots \mathbb{R}$,
and the finite iteration:
\[
    \mathbb{R}^* :=
      \bigcurlyvee_{n \in \mathbb{N}} \mathbb{R}^n \,.
\]
\end{definition}
%}}}

Simulations interact with this structure in the following way.

\begin{lemma} \label{lemma:kleenesim} %{{{
Consider the strategies
$\sigma^\natural : A^\natural \Rightarrow B^\natural$,
$\sigma^\sharp : A^\sharp \Rightarrow B^\sharp$ and
the refinement conventions
$\mathbb{R}, \mathbb{R}' : A^\natural \Leftrightarrow A^\sharp$ and
$\mathbb{S}, \mathbb{S}' : B^\natural \Leftrightarrow B^\sharp$.
The following properties hold:
\[
  \AxiomC{$\mathbb{R}' \preceq \mathbb{R}$}
  \AxiomC{$\sigma^\natural \le_{\mathbb{R} \Rightarrow \mathbb{S}} \sigma^\sharp$}
  \AxiomC{$\mathbb{S} \preceq \mathbb{S}'$}
  \TrinaryInfC{$\sigma^\natural \le_{\mathbb{R}' \Rightarrow \mathbb{S}'} \sigma^\sharp$}
  \DisplayProof
  \qquad
  \AxiomC{$\sigma^\natural \le_{\mathbb{R} \Rightarrow \mathbb{S}} \sigma^\sharp$}
  \AxiomC{$\sigma^\natural \le_{\mathbb{R} \Rightarrow \mathbb{S}'} \sigma^\sharp$}
  \BinaryInfC{$\sigma^\natural
     \le_{\mathbb{R} \Rightarrow \mathbb{S} \curlyvee \mathbb{S}'} \sigma^\sharp$}
  \DisplayProof
\]
For the strategies
$\sigma^\flat : A^\flat \Rightarrow B^\flat$,
$\sigma^\natural : A^\natural \Rightarrow B^\natural$,
$\sigma^\sharp : A^\sharp \Rightarrow B^\sharp$ and
the refinement conventions
$\mathbb{R} : A^\flat \Leftrightarrow A^\natural$,
$\mathbb{R}' : A^\natural \Leftrightarrow A^\sharp$, and
$\mathbb{S} : B^\flat \Leftrightarrow B^\natural$,
$\mathbb{S}' : B^\natural \Leftrightarrow B^\sharp$,
the following property holds:
\[
  \AxiomC{$\sigma^\flat \le_{\mathbb{R} \Rightarrow \mathbb{S}} \sigma^\natural$}
  \AxiomC{$\sigma^\natural \le_{\mathbb{R}' \Rightarrow \mathbb{S}'} \sigma^\sharp$}
  \BinaryInfC{$\sigma^\flat
    \le_{\mathbb{R} \cdot \mathbb{R}' \Rightarrow \mathbb{S} \cdot \mathbb{S}'}
    \sigma^\sharp$}
  \DisplayProof
\]
Finally,
for
$\sigma, \tau : A \Rightarrow B$
and for
$\mathbb{R} : A \Leftrightarrow A$ and
$\mathbb{S} : B \Leftrightarrow B$,
the following property holds:
\[
  \AxiomC{$\sigma \le_{\mathbb{R} \Rightarrow \mathbb{S}} \tau$}
  \UnaryInfC{$\sigma \le_{\mathbb{R}^* \Rightarrow \mathbb{S}^*} \tau$}
  \DisplayProof
\]
\end{lemma}
%}}}

%}}}

%}}}

\section{Semantics of CompCert Modules} \label{sec:modsem} %{{{

\subsection{Overview} %{{{

Having laid out our basic semantic framework,
we now use it to construct a compositional semantic model
for CompCert open modules.
CompCert is a C-to-assembly compiler written and verified
in the Coq proof assistant.
The compiler is accompanied by
a formal semantics of the source, target, and intermediate languages,
and a proof that if the compiler succeeds,
then the behavior of the emitted target program
refines that of the source program.

The semantics of CompCert languages
are given as labeled transition systems (LTS)
modeling the overall execution of a complete program.
The semantics of external calls
is a global parameter common to all languages,
and may use a fixed set of actions as a rudimentary model
of the program's interaction with the environment.
The externally observable behavior of the program
is given by defining the set of traces associated with any LTS,
and the various kinds of simulations used by CompCert
are shown to be sound with respect to trace containement.

CompCert, however,
does not attempt to model the semantics of open modules,
nor to define a semantic notion of module composition.
We achive this using the following approach:
\begin{itemize}
\item Following Compositional CompCert \cite{compcompcert},
  we modify CompCert's LTS model
  to explicitely account for the two-way interaction
  between a program module and its environment.
  This interaction is formulated
  in terms of incoming and outgoing function calls and returns.
\item We generalize this model further by
  using elementary games (Def.~\ref{def:elemgame})
  to parametrize the LTS model, and
  using simulation conventions to parametrize
  CompCert's notions of simulations.
  Games specify the form of calls and returns;
  simulation conventions express
  generalized calling conventions between the various
  languages involved in the compilation process.
\end{itemize}

In \S\ref{sec:modsem:def},
we introduce our changes to CompCert's LTS model,
and define their behavior in terms of interactive computations.
In \S\ref{sec:modsem:ref},
we carry out those changes to the notions of
forward and backward simulations
present in CompCert,
and show them sound with respect to refinement.
In \S\ref{sec:modsem:comp},
[composition].
[Figure out how abstraction fits in].

%}}}

\subsection{Definition} %{{{
\label{sec:modsem:def}

The semantics of CompCert languages are given mainly
as labeled transition systems (LTS).
In the original CompCert,
a labeled transition system is given as
a set of states $S$,
a subset $I \subseteq S$ of initial states,
a labeled transition relation
${\rightarrow} \subseteq S \times \mathbb{E}^* \times S$,
and a set
$F \subseteq S \times \kw{int}$
of final states associated with integer results.
CompCert LTS support a notion of interaction with the environment
in the form of event traces:
a transition $s \stackrel{t}{\rightarrow} s'$,
indicates that the state $s$ may transition to state $s'$
through an interaction recorded as the event trace $t \in \mathbb{E}^*$.

For the range of languages that we are considering
($\kw{Clight}$--$\kw{Asm}$),
non-empty event traces are only generated by
the given semantics of external calls.
Since this model is insufficient to express
the semantics of open C and assembly modules,
we replace this facility with
an explicit treatment of incoming and outgoing calls
at the level of transition systems.

\begin{definition}[Small-step semantics]
Given two elementary games $A, B$,
a \emph{small-step semantics} for $A \Rightarrow B$
is a tuple $L = \langle S, \rightarrow, I, X, Y, F \rangle$.
${\rightarrow} \subseteq S \times S$ is a \emph{transition relation} on
the set of states $S$.
The handling of incoming calls is specified by
$I \subseteq M_B^\kw{Q} \times S$, which
assigns a set of \emph{initial states} to each question of $B$, and by
$F \subseteq S \times M_B^\kw{A}$,
which designates \emph{final states} together with corresponding answers.
The handling of external calls is specified by
$X \subseteq S \times M_A^\kw{Q}$,
which identifies \emph{external states} together with
corresponding questions of $A$ directed to the environment, and
$Y \subseteq S \times M_A^\kw{A} \times S$,
which is used to select a \emph{resumption state}
based on the outcome of the external call
after the environment returns control to the module.

We write $L : \kw{semantics}(A, B)$ to indicate that
$L$ is a small-step semantics for $A \Rightarrow B$.
\end{definition}

In CompCert, the absence of any transition (``getting stuck'')
denotes an undefined behavior.
This convention fits the context of
language semantics defined through inductive sets of rules:
the behavior of a state for which there are no rules is undefined.
For a small-step semantics
$L = \langle S, {\rightarrow}, I, X, Y, F \rangle : \kw{semantics}(A,B)$,
we recognize those states using the predicate:
\[
    \kw{stuck}_L(s) :=
      ({\rightarrow}(s) = \varnothing) \wedge
      (X(s) = \varnothing) \wedge
      (F(s) = \varnothing)
\]
Taking this into account,
the immediate behavior of a state $s \in S$
can be expressed as the interactive computation
$\kw{step}_L(s) : \mathcal{I}_{M_A^\kw{Q},M_A^\kw{A}}(S)$
defined as follows:
\[
  \kw{step}_L(s) :=
    \begin{cases}
      \top & \mbox{if } \kw{stuck}(s) \\
      {\rightarrow}(s) \sqcup
      (X(s) \bind \mathbf{I} \bind Y(s)) & \mbox{otherwise,}
   \end{cases}
\]
The overall external behavior of $L$
can then be given as
$
    \llbracket L \rrbracket :
      M_B^\kw{Q} \rightarrow \mathcal{I}_{M_A^\kw{Q},M_A^\kw{A}}(M_B^\kw{A})
$
defined by:
%Then $\llbracket L \rrbracket$ can be defined as:
\begin{align*}
  \llbracket L \rrbracket (q) :=
    \begin{cases}
       \top & \mbox{if } I(q) = \varnothing \\
       I(q) \bind \kw{step}^\infty \bind F & \mbox{otherwise.}
     \end{cases}
\end{align*}

%}}}

\subsection{Refinement and Abstraction} \label{sec:modsem:sim} %{{{
\label{sec:modsem:ref}

The correctness of CompCert is established in terms of
a so-called \emph{backward simulation}%
\footnote{In this usage, \emph{backward} pertains to
  the compilation process,
  rather than the execution of programs.}
between the small-step semantics of the source and target programs.
In turn, backward simulations
are shown to be sound with respect to trace containement.
In this section,
we define backward simulations for our updated notion
of small-step semantics,
and show that they are sound with respect to
refinement and abstraction of interactive behaviors.

% Footnote pointing out difference w/ "Fw & Bw Sim" terminology?
A backward simulation asserts that any transition in the target program
has a corresponding transition sequence in the source program.
A transition in the target program can be matched with
an empty transition sequence in the source program;
however, to ensure the preservation of silent divergence,
this can only happen for finitely many consecutive target transitions.
This restriction is enforced by indexing the simulation relation
over a well-founded order,
and making sure that the index decreases
whenever an empty sequence of source transitions is used.

In our setting,
the definition of backward simulations needs to be extended
to take into account the correspondance between
the source and target questions and answers
This can be specified using the notion of simulation convention
defined in \S\ref{sec:monad:abs}:
a simulation between the small-step semantics
$L_1 : \kw{semantics}(A_1, B_1)$ and
$L_2 : \kw{semantics}(A_2, B_2)$ will
operate in the context of the refinement convention
$\mathbb{C}_A : \mathcal{R}_{W_A}(A_1, A_2)$ and
$\mathbb{C}_B : \mathcal{R}_{W_B}(B_1, B_2)$.

Note that initial and resumption states on one hand,
and external and final states on the other hand,
correspond to respective actions of the environment and the system;
as such, they need to be treated differently.
Furthermore,
the potential non-determinism present in small-step semantics
is interpreted as \emph{system} non-determinism,
so that when relating sets of states
we will want to make sure that
each \emph{target} state has a corresponding \emph{source} state,
but the source program may allow additional behaviors
that are not realized in the target program.
This rule is slightly altered to take into account the convention that
empty sets of states correspond to undefined behaviors,
so that if the source set is empty,
no restrictions are placed on the target state.

\begin{definition}[Backward simulation between sets of states]
We say that $R : \mathcal{R}(S_1, S_2)$ is a
\emph{backward simulation relation
  between the sets $X_1 \subseteq S_1$ and  $X_2 \subseteq S_2$}
if the following conditions hold:
\begin{enumerate}
\item
  If $X_1$ is non-empty,
  then $X_2$ is non-empty as well.
\item
  If $X_1$ is non-empty,
  then for all $s_2 \in X_2$,
  there exists $s_1 \in X_1$
  such that $(s_1, s_2) \in R$.
\end{enumerate}
We will write $X_1 \ge_R X_2$.
\end{definition}

A state $s$ \emph{goes wrong}
if it is neither an external nor a final state,
and if there is no transition $s \rightarrow s'$;
a state $s$ is \emph{safe}
if no state that goes wrong is reachable from $s$
by the transition relation $\rightarrow$.
Safe states will be denoted using the predicate $\kw{safe}(s)$.
With this we are ready to define backward simulations.

\begin{definition}[Backward simulation]
Given
$\mathbb{C}_A : \mathcal{R}(A_1, A_2)$ and
$\mathbb{C}_B : \mathcal{R}(B_1, B_2)$
two simulation conventions,
and given
$L_1 : \kw{semantics}(A_1, B_1)$ and
$L_2 : \kw{semantics}(A_2, B_2)$
two small-step semantics,
a \emph{backward simulation} between $L_1$ and $L_2$
consists in a
well-founded relation $(I, <)$
together with a family of relations
$(R_i : \mathcal{R}_{W_B}(S_1, S_2))_{i \in I}$
satisfying the following properties:
\begin{description}
\item[Initial states]
  For all
  $q_1 \ifr{w \Vdash {\preceq}_B^\kw{Q}} q_2$,
  the condition $I_1(q_1) \ge_{\exists i . R_i} I_2(q_2)$ holds.
\item[Progress]
  For all $s_1 \ifr{w \Vdash R_i} s_2$
  with $s_1$ a safe state,
  $s_2$ does not go wrong.
\item[Simulation]
  For all $s_1 \ifr{w \Vdash R_i} s_2$,
  if $s_1$ is a safe state and $s_2 \rightarrow s_2'$,
  then there exists $i' \in I$ and $s_1' \in S_1$
  such that $s_1' \ifr{w \Vdash R_{i'}} s_2'$ and
  such that one the following conditions hold:
  \[
    s_1 \rightarrow^+ s_1' \,, \quad \mbox{or} \quad
    s_1 \rightarrow^* s_1' \:\wedge\: i' < i \,.
  \]
\item[External calls]
  For all $s_1 \ifr{w \Vdash R_i} s_2$
  with $s_1$ a safe state, and
  for any question $m_2 \in M_{A_2}^\kw{Q}$
  such that $(s_2, m_2) \in X_2$,
  there exists $w' \in W_A$ and $q_1 \in M_{A_1}^\kw{Q}$
  such that $q_1 \ifr{w' \Vdash {\preceq}_A^\kw{Q}} q_2$.
  In addition, for all corresponding answers
  $r_1 \ifr{w' \Vdash {\preceq}_A^\kw{A}} r_2$,
  the condition $Y_1(s_1, r_1) \ge_{\exists i . R_i} Y_2(s_2, r_2)$ holds.
\item[Final states]
  For all $s_1 \ifr{w \Vdash R_i} s_2$
  with $s_1$ a safe state, and
  for any answer $r_2 \in M_{B_2}^\kw{A}$
  such that $(s_2, r_2) \in F_2$,
  there exists a state $s_1'$ reachable from $s_1$ and
  an answer $r_1 \in M_{B_1}^\kw{A}$ such that
  $(s_1', r_1) \in F_1$ and $r_1 \ifr{w \Vdash {\preceq}_B^\kw{A}} r_2$.
\end{description}
We will write $L_1 \ge_{\mathbb{C}_A \Rightarrow \mathbb{C}_B} L_2$.
\end{definition}

Stated in their full generality,
CompCert backward simulations are notably complex.
Fortunately,
many simpler formulations and related constructions
can be used to establish
the existence of backward simulations
for specific compiler passes.
However,
backward simulations are not easily amenable to
meta-theoretical analysis
where they can appear both as hypotheses and conclusions,
which is an additional motivation for introducing
our game-based framework:
by embedding simulations into our game framework,
we will be able to conduct large-scale reasoning
in a more convenient setting.

\begin{lemma}[Soundness of backward simulations]
Consider
$\mathbb{C}_A : \mathcal{R}(A_1, A_2)$ and
$\mathbb{C}_B : \mathcal{R}(B_1, B_2)$
two simulation conventions,
and
$L_1 : \kw{semantics}(A_1, B_1)$ and
$L_2 : \kw{semantics}(A_2, B_2)$
two small-step semantics
such that
$L_1 \ge_{\mathbb{C}_A \Rightarrow \mathbb{C}_B} L_2$.
Then,
$
    \llbracket L_2 \rrbracket
    \le_{\mathbb{C}_A \Rightarrow \mathbb{C}_B}
    \llbracket L_1 \rrbracket
$.
\end{lemma}

%}}}

\subsection{Composition} \label{sec:modsem:comp} %{{{

Given two semantic objects
$\sigma_1 : B \Rightarrow C$ and
$\sigma_2 : A \Rightarrow B$,
their composition $\sigma_1 \odot \sigma_2 : A \Rightarrow C$,
per the definition in \S\ref{sec:monad:int},
computes the behavior of $\sigma_1$ when $\sigma_2$ is used
to interpret its external calls.
In addition,
to model cross-module interactions in low-level programs,
we define the symmetric,
\emph{horizontal} composition
$\sigma_1 \bullet \sigma_2$
of
$\sigma_1 : A \Rightarrow A$ and
$\sigma_2 : A \Rightarrow A$,
which allows $\sigma_1$ and $\sigma_2$
to interact with one another
in a mutually recursive fashion.
We will proceed first by computing their union $\sigma_1 \oplus \sigma_2$,
then resolving their recursive calls
through a unary operator $\sigma^\circledcirc$,
which nondeterministically feeds the external calls of $\sigma$
back to itself.

To define $\sigma^\circledcirc$,
we use a transition system which will keep track,
on one hand, of the currently active behavior, and
on the other, of a stack of suspended continuations ready to be resumed.
Whenever the active behavior performs an external call,
we may instead suspend it and instantiate a new copy of $\sigma$.
When the active behavior returns,
we will resume the next continuation on the stack.
This corresponds to the step relation:
\begin{align*}
   \kw{step}(s, k) :=
 \:  &\mu(s) \bind m \mapsto \mathbf{I}(m)
               \bind n \mapsto \kw{ret}(\delta_{mn}(s), k) \\ \sqcup
 \:  &\mu(s) \bind m \mapsto \kw{ret}(\sigma(m),
                        \kw{cons}(n \mapsto \delta_{mn}(s), k)) \\ \sqcup
 \:  &\rho(s) \bind n \mapsto
      \kw{cons}^{-1}(k) \bind (k_0, k') \mapsto
      \kw{ret}(k_0(n), k')
\end{align*}

\begin{definition}[Horizontal composition]
For an elementary game $A$
and for a strategy $\sigma : A \Rightarrow A$,
the recursive calls of $\sigma$ can be nondeterministically resolved as:
\[
\sigma^\circledcirc(m) :=
  \kw{ret}(\mathbf{I}(m), \kw{nil}) \bind 
      \kw{step}^\infty \bind
   (s, k) \mapsto
       \kw{nil}^{-1}(k) \bind - \mapsto
       \rho(s)
\]
The \emph{horizontal composition} of the interactive behaviors
$\sigma_1, \sigma_2 : A \Rightarrow A$
is the interactive behavior
$\sigma_1 \bullet \sigma_2 : A \Rightarrow A$
defined as:
\[
    \sigma_1 \bullet \sigma_2 :=
      (\sigma_1 \oplus \sigma_2)^\circledcirc \,.
\]
\end{definition}

The commutativity of $\bullet$ follows from that of $\oplus$,
and associativity can be derived from properties of $\oplus$ and $\circledcirc$.
The behaviors $\mathbf{0}$ and $\mathbf{I}$
are both identities for $\bullet$.

%}}}

\subsection{Hiding} %{{{

Note that
in the behavior $\sigma_1 \bullet \sigma_2$,
cross-component calls remain observable:
while the external calls of $\sigma_1$
can be resolved to behaviors implemented in $\sigma_2$
(and conversely),
the original external calls remain
as non-deterministic alternatives in the final result.
[Only way to ensure associativity
because nothing prevents other components
to \emph{also} implement these functions,
in which case eliminating the calls
would make the result sensitive to the order of composition.]

This phenomenon is reminiscent of a similar one observed
in the process calculus CCS \cite{ccs}
in the context of \emph{parallel composition}.
Following this precedent,
we then define a separate \emph{hiding} operator
which retains only the external calls
which belong to a given set.

\begin{definition}[Hiding]
For a strategy $\sigma : A \Rightarrow B$ and
a set of questions $D \subseteq M_A^\kw{Q}$,
we define $\bar{D} : A \Rightarrow A$ as
$\bar{D} := \{ m, mnn \mid m \in M_A^\kw{Q} \setminus D, n \in M_A^\kw{A} \}
\sqsubseteq \mathbf{I}$
and $\sigma \setminus D : A \Rightarrow B$ as
$\sigma \setminus D := \sigma \odot \bar{D} \sqsubseteq \sigma$.
\end{definition}

%}}}

%}}}

\section{Compiler Correctness} \label{sec:compcert} %{{{

In the previous section
we have shown how to extend
the semantic model used by CompCert
to describe the behavior of open modules.
In this section
we show how the correctness proof of CompCert
can be updated to account for this additional structure.

\subsection{Overview} \label{sec:compcert:overview} %{{{

Based on the existing CompCert semantics,
we will define the elementary games
$\mathcal{C}$ and $\mathcal{A}$
describing how C and assembly modules
interact with their respective environments.
The semantics of CompCert $\kw{Clight}$ and $\kw{Asm}$ programs
can then be expressed as interactive behaviors of types:
\[
    \kw{Clight} \llbracket p_s \rrbracket :
      \mathcal{C} \Rightarrow \mathcal{C} \,, \qquad
    \kw{Asm} \llbracket p_t \rrbracket :
      \mathcal{A} \Rightarrow \mathcal{A} \,.
\]
We will show that there exists a simulation convention
$\mathbb{R}_\kw{CompCert} : \mathcal{A} \Leftrightarrow \mathcal{C}$
such that whenever $p_t = \kw{CompCert}(p_s)$,
the following refinement property holds:
\begin{equation}
    \label{eqn:correctness}
    \kw{Asm} \llbracket p_t \rrbracket \,
    \sqsubseteq_{\mathbb{R}_\kw{CompCert} \Rightarrow \mathbb{R}_\kw{CompCert}}
    \kw{Clight} \llbracket p_s \rrbracket
\end{equation}

In Compositional CompCert,
this is achieved independently for each compilation pass.
Stated in our terms,
Compositional CompCert's \emph{interaction semantics}
employ a fixed interface $\mathcal{C}$,
and \emph{structured injections} define a single refinement convention
$\mathbb{S} : \mathcal{C} \Leftrightarrow \mathcal{C}$.
A theorem similar to (\ref{eqn:correctness}) is proved for each pass,
and structured injections are show to compose
($\mathbb{S} \cdot \mathbb{S} \equiv \mathbb{S}$),
so that a simulation theorem can be derived for the whole compiler.

A major challenge encountered in this context
is the asymmetry of requirements vs. guarantees
present in the existing proofs:
while CompCert imposes strong requirements
on the semantics of external functions,
the simulation relations used by its correctness proofs
are too weak to establish corresponding guarantees
for a module's own function executions,
preventing horizontal compositionality.
Because of this,
Compositional CompCert required significant changes
to all compilation passes,
strengthening their simulation relations
to conform to $\mathbb{S}$.

However,
our explicit treatment of abstraction
and reified notion of simulation convention
makes another approach possible.
Using existing proofs, it is reasonably straightforward
to establish a theorem of the form
for each compilation pass:
\begin{equation}
    \label{correctness-alt}
    \llbracket p_t \rrbracket
    \sqsubseteq_{\mathbb{R} \Rightarrow \mathbb{R}'}
    \llbracket p_s \rrbracket \,,
\end{equation}
where the simulation conventions $\mathbb{R}$ and $\mathbb{R}'$
characterize the assumptions and guarantees
of the existing proof.
In addition,
relevant properties of key source, target, and intermediate languages
can be expressed in relational form,
and can be used to strengthen the resulting theorem:
by composing these proofs together,
and using algebraic properties of
the simulation conventions involved,
we will be able to bridge the gap
between the domain and codomain simulation conventions
and be able to prove a correctness statement
in the form of (\ref{eqn:correctness}).

In the remainder of this section,
[dig deeper into the actual semantics and
simulation conventions used in CompCert.]
etc.

%}}}

\subsection{The CompCert Memory Model} \label{sec:compcert:mm} %{{{

\begin{figure} % fig:mm (The CompCert memory model) {{{
  \begin{gather*}
    v : \kw{val} ::=
      \kw{Vundef} \alt
      \kw{Vint}(n) \alt
      \kw{Vlong}(n) \alt
      \kw{Vfloat}(x) \alt
      \kw{Vsingle}(x) \alt
      \kw{Vptr}(b, o)
    \\
    (b, o) : \kw{ptr} :=
      \kw{block} \times \mathbb{Z}
    \\
    (b, l, h) : \kw{ptrrange} :=
      \kw{block} \times \mathbb{Z} \times \mathbb{Z}
  \end{gather*}
  \begin{align*}
    \kw{Mem.alloc} &:
      \kw{mem} \rightarrow \mathbb{Z} \rightarrow \mathbb{Z} \rightarrow
      \kw{mem} \times \kw{block}
    \\
    \kw{Mem.free} &:
      \kw{mem} \rightarrow
      \kw{ptrrange} \rightarrow
      \kw{option}(\kw{mem})
    \\
    \kw{Mem.load} &:
      \kw{mem} \rightarrow \kw{ptr} \rightarrow \kw{option}(\kw{val})
    \\
    \kw{Mem.store} &:
      \kw{mem} \rightarrow \kw{ptr} \rightarrow \kw{val} \rightarrow \kw{option}(\kw{mem})
    \\
    \kw{Mem.perm} &:
      \kw{mem} \rightarrow \kw{ptr} \rightarrow \mathcal{P}(\kw{perm})
  \end{align*}
  \caption{Outline of the CompCert memory model}
  \label{fig:mm}
\end{figure}
%}}}

The CompCert memory model \cite{compcertmmv2}
is the core algebraic structure
which underlies the semantics of CompCert's languages.
Some of its operations
are shown in Fig.~\ref{fig:mm}.
The idealized version presented here
involves
the type of memory states \kw{mem},
the types of pointers \kw{ptr} and address ranges \kw{ptrrange}, and
the type of runtime values \kw{val}.
To keep our exposition concise and clear,
we will gloss over the technical details
associated with the encoding of offsets
as concrete binary integers,
and the associated modular arithmetic and overflow constraints.
[But see artefact for the details.]

The memory is organized into a finite number of \emph{blocks}.
Each memory block has a unique identifier ($b : \kw{block}$)
represented as a positive integer,
and is equipped with its own independent linear address space.
Block identifiers and offsets are often manipulated together,
as a pair $p = (b, o) : \kw{ptr} = \kw{block} \times \mathbb{Z}$.
New blocks are created by the primitive $\kw{Mem.alloc}$,
with prescribed boundaries for their usable offsets.

A runtime value ($v : \kw{val}$) can be stored at
a given address using the primitive \kw{Mem.store},
and retreived using the primitive \kw{Mem.load}.
Values can be integers (\kw{Vint}, \kw{Vlong}) and
floating point numbers (\kw{Vfloat}, \kw{Vsingle})
of different sizes,
as well as pointers (\kw{Vptr}).
The special value \kw{Vundef}
represents an undefined value;
the simulation relations used by CompCert
usually allow $\kw{Vundef}$
to be refined into a more concrete value.

The memory model is shared by all of the languages in CompCert.
The states used to define their semantics consist of
a memory component $m : \kw{mem}$,
together with language-specific components
containing additional run-time values ($\kw{val}$).
For higher-level languages,
the language-specific component may consist of
a control stack with local environments for each activation;
for lower-level languages,
the additional component will mainly contain
the values stored in various registers.

Similarly,
our models of cross-module communication
will be built around a memory component,
accompanied by language-specific information
about function call or return events.

%}}}

\subsection{Language Interfaces} \label{sec:compcert:li} %{{{

\begin{table*} % tbl:li Language interfaces {{{
  \begin{tabular}{clll}
    \hline
    Name & Questions & Answers & Description \\
    \hline
    $\mathcal{C}$ & $(\kw{id}, \kw{sg}, \vec{v}, m)$ & $(v', m')$ &
      C-style function calls (Clight--RTL) \\
    $\mathcal{L}$ & $(\kw{id}, \kw{sg}, \kw{ls}, m)$ & $(\kw{ls}', m')$ &
      Arguments passed in abstract locations (LTL, Linear) \\
    $\mathcal{M}$ & $(\kw{id}, \kw{sp},\kw{ra},\kw{rs}, m)$ & $(\kw{rs}', m')$ &
      Arguments passed through in-memory stack (Mach) \\
    $\mathcal{A}$ & $(\kw{rs}, m)$ & $(\kw{rs}', m')$ &
      Assembly-style control transfers (Asm) \\
    \hline
  \end{tabular}
  \caption{Language interfaces for the various languages of CompCert.}
  \label{tbl:li}
\end{table*}
%}}}

The elementary games we use to model
the cross-module interactions of CompCert languages
are shown in Table~\ref{tbl:li}.
Moves correspond to control transfers
between a module and its environment:
questions correspond to function invocations;
answers return control to the caller.

At the source level ($\mathcal{C}$),
questions consist of
the name and signature of the function being invoked,
the values of its arguments,
and the state of the memory at the point of entry;
answers
consist of the function's return value
and the state of the memory at the point of exit.
This language interface is used for Clight and
for the majority of CompCert's intermediate languages.

However, starting with LTL, [...]
Once we reach Asm ($\mathcal{A}$),
queries and replies simply specify
the values of registers and the state of the memory.

%}}}

\subsection{Simulations} %{{{

The correctness of each pass of CompCert
is proved by establishing a simulation between
the semantics of the source and target programs
of that pass.
To generalize the existing proofs to our framework,
we extend them to fit the model presented in \S\ref{sec:modsem:sim}.
To this end,
we need to specify for each pass
two simulation conventions:
one for incoming and one for external calls.

In the same way language interfaces
reveal details of language semantics
that were previously internal
and establish an external interface for
the function call mechanism,
simulation conventions reveal
details of the simulation relations
used in CompCert passes
that were previously hidden from
the user of the correctness proof.
In practice,
the states of all CompCert languages
embed a memory state,
and passes fall into three categories
depending on how they relate
the memory states at their source and target languages:
\begin{itemize}
\item \emph{Equalty} is used when the memory states
  in the source and target execution remain identical,
  for instance when the differences between
  the source and target programs are mostly a matter of syntax,
  but they execute in the same way.
\item \emph{Memory extensions} are somewhat less constrained:
  the target memory is allowed to contain values that are
  \emph{more defined} than that of the source memory,
  and to have less restrictive permissions,
  so that all operations succeeding on the source memory
  will succeed on the target memory as well.
  However,
  the two memory states must share the same overall structure.
\item \emph{Memory injection} relations
  are the most general ones used in CompCert,
  and allow the structures of the source and target memories
  to differ, as specified by an injection mapping
  (see Example~\ref{ex:meminj}).
\end{itemize}

Table~\ref{tbl:passes} shows
the intermediate languages and compilation passes
used in our version of CompCert,
together with the respective
language interfaces and
simulation conventions
that are used for each one.
In the remainder of this section,
we will explain how these simulation conventions are constructed
and how the compiler's overall incoming and outgoing conventions
can be reconciled.

\begin{table*} % tbl:passes Passes of Composable CompCert %{{{
  \footnotesize
  \begin{tabular}{lllp{.55\textwidth}}
    \hline
    Language/Pass & Outgoing & Incoming & Description \\
    \hline
    \textbf{Clight} & $\mathcal{C}$ & $\mathcal{C}$ &
      A simpler version of CompCert C
      where expressions contain no side-effects. \\
    \emph{Eqn.}~(\ref{eqn:clight}) & $\mathcal{C}[\kw{injp}]^*$ & $\mathcal{C}[\kw{injp}]^*$ &
      \emph{Clight properties} \\
    \kw{SimplLocals} & $\mathcal{C}[\kw{injp}]$ & $\mathcal{C}[\kw{inj}]$ &
      Pulling non-adressable scalar local variables out of memory. \\
    \kw{Cshmgen} & \kw{id} & \kw{id} &
      Simplification of control structures;
      explication of type-dependent computations. \\
    \hline
    \textbf{Csharpminor} & $\mathcal{C}$ & $\mathcal{C}$ &
      Low-level structured language. \\
    \kw{Cminorgen} & $\mathcal{C}[\kw{injp}]$ & $\mathcal{C}[\kw{inj}]$ &
      Stack allocation of local variables whose address is taken;
      simplification of switch statements. \\
    \hline
    \textbf{Cminor} & $\mathcal{C}$ & $\mathcal{C}$ &
      Low-level structured language,
      with explicit stack allocation of certain local variables. \\
    \kw{Selection} & $\mathcal{C}[\kw{ext}]$ & $\mathcal{C}[\kw{ext}]$ &
      Recognition of operators and addressing modes. \\
    \hline
    \textbf{Cminorsel} & $\mathcal{C}$ & $\mathcal{C}$ &
      Like Cminor, with machine-specific operators and addressing modes. \\
    \kw{RTLgen} & $\mathcal{C}[\kw{ext}]$ & $\mathcal{C}[\kw{ext}]$ &
      Construction of the CFG, 3-address code generation. \\
    \hline
    \textbf{RTL} & $\mathcal{C}$ & $\mathcal{C}$ &
      Register transfer language
      (3-address code, control-flow graph, infinitely many pseudo-registers). \\
    \kw{Tailcall} & $\mathcal{C}[\kw{ext}]$ & $\mathcal{C}[\kw{ext}]$ &
      Recognition of tail calls. \\
    \kw{Renumber} & $\kw{id}$ & $\kw{id}$ &
      Postorder renumbering of the CFG. \\
    \emph{Eqn.}~(\ref{eqn:rtl}) & $\mathcal{C}[\kw{inj}]$ & $\mathcal{C}[\kw{inj}]$ &
      \emph{RTL properties} \\
    \kw{Allocation} & \kw{alloc} & \kw{alloc} &
      Register allocation \\
    \hline
    \textbf{LTL} & $\mathcal{L}$ & $\mathcal{L}$ &
      Location transfer language
      (3-address code, control-flow graph of basic blocks,
      finitely many physical registers, infinitely many stack slots). \\
    \kw{Linearize} & \kw{id} & \kw{id} &
      Linearization of the CFG \\
    \hline
    \textbf{Linear} & $\mathcal{L}$ & $\mathcal{L}$ &
      Like LTL, but the CFG is replaced by
      a linear list of instructions with explicit branches and labels \\
    \kw{CleanupLabels} & \kw{id} & \kw{id} &
      Removal of unreferenced labels. \\
    \kw{Debugvar} & \kw{id} & \kw{id} &
      Synthesis of debugging information. \\
    \kw{Stacking} & \kw{stacking} & \kw{stacking} &
      Laying out the activation records \\
    \hline
    \textbf{Mach} & $\mathcal{M}$ & $\mathcal{M}$ &
      Like Linear, with a more concrete view of the activation record \\
    \kw{Asmgen} & \kw{asmgen} & \kw{asmgen} &
      Emission of assembly code \\
    \hline
    \textbf{Asm} & $\mathcal{A}$ & $\mathcal{A}$ &
      Assembly language for x86 machines \\
    \hline
  \end{tabular}
  \caption}}

%}}}

\subsection{Kripke Logical Relations} \label{sec:compcert:cklr} %{{{

Memory extensions and injections
are useful for simulations
because they are specific instances
of logical relations for the CompCert memory model:
they are compatible with memory operations
in the sense that
performing similar operations on related arguments
yields related results.

\subsubsection{Definition} %{{{

We formalize this idea by defining
a notion of Kripke logical relations over the CompCert memory model,
closed under composition, and which
admits memory extensions and injections as particular instances.
As we will see,
more complex relations can also be defined,
and this will make relational parametricity theorems
particularly useful.

\begin{definition}[CompCert Kripke logical relation] \label{def:cklr} %{{{
Consider a tuple $R = (W, \leadsto, f, R^\kw{mem})$,
where
$\langle W, \leadsto \rangle$ is a Kripke frame,
$f : W \rightarrow \kw{meminj}$
associates an injection mapping to each world, and
$R^\kw{mem} : \mathcal{R}_{W}(\kw{mem})$
is a Kripke relation on memory states.
We introduce the Kripke relations
$R^\kw{ptr} : \mathcal{R}_W(\kw{ptr})$ and
$R^\kw{ptrrange} : \mathcal{R}_W(\kw{ptrrange})$
defined by the rules:
\[
  \AxiomC{$f_w(b) = (b', \delta)$}
  \UnaryInfC{$(b, o) \ifr{w \Vdash R^\kw{ptr}} (b', o + \delta)$}
  \DisplayProof
  \qquad
  \AxiomC{$(b_1, l_1) \ifr{w \Vdash R^\kw{ptr}} (b_2, l_2)$}
  \AxiomC{$h_1 - l_1 = h_2 - l_2$}
  \BinaryInfC{$(b_1, l_1, h_1) \ifr{w \Vdash R^\kw{ptrrange}} (b_2, l_2, h_2)$}
  \DisplayProof
\]
and the Kripke relation
$R^\kw{val} : \mathcal{R}_W(\kw{val})$
defined by the rules:
\begin{gather*}
  \forall \, v : \kw{val} \,.\,
    \kw{Vundef} \ifr{\Vdash R^\kw{val}} v \qquad
  \kw{Vptr} : {}
    [\Vdash R^\kw{ptr} \rightarrow R^\kw{val}] \\
  \kw{Vint}, \kw{Vlong}, \kw{Vfloat}, \kw{Vsingle} :
    [\Vdash {=} \rightarrow R^\kw{val}] \,.
\end{gather*}
We say that $R$ is a \emph{CompCert Kripke logical relation}
if the properties shown in Fig.~\ref{fig:cklr-def} are satisfied.
\end{definition}
%}}}

\begin{figure} % fig:cklr-def (Definition of CKLRs) {{{
  \begin{gather*}
    w \leadsto w \\
    w \leadsto w' \wedge w' \leadsto w'' \Rightarrow w \leadsto w'' \\
    f \ifr{(\leadsto) \rightarrow \kw{inject\_incr}} f
  \end{gather*}
  \begin{align*}
      \kw{Mem.alloc} :
        &\Vdash R^\kw{mem} \rightarrow {=} \rightarrow {=} \rightarrow
        \Diamond (R^\kw{mem} \times R^\kw{block})
      \\
      \kw{Mem.free} :
        &\Vdash R^\kw{mem} \rightarrow R^\kw{ptrrange} \rightarrow
        \kw{option}^\le(\Diamond R^\kw{mem})
      \\
      \kw{Mem.load} :
        &\Vdash R^\kw{mem} \rightarrow R^\kw{ptr} \rightarrow
        \kw{option}^\le(R^\kw{val})
      \\
      \kw{Mem.store} :
        &\Vdash R^\kw{mem} \rightarrow R^\kw{ptr} \rightarrow R^\kw{val} \rightarrow
        \kw{option}^\le(\Diamond R^\kw{mem})
      \\
      \kw{Mem.perm} :
        &\Vdash R^\kw{mem} \rightarrow R^\kw{ptr} \rightarrow {\subseteq}
  \end{align*}
  \caption{Axioms for CKLRs}
  \label{fig:cklr-def}
\end{figure}
%}}}

The $R^\kw{mem}$ component is given direcly.
We expect $R^\kw{ptr}$ to be functional
(each source pointer has at most one corresponding target pointer),
and to satisfy the following shift-invariance property:
\[
  \AxiomC{$(b_1, o_1) \ifr{R^\kw{ptr}_w} (b_2, o_2)$}
  \UnaryInfC{$(b_1, o_1 + \delta) \ifr{R^\kw{ptr}_w} (b_2, o_2 + \delta)$}
  \DisplayProof
\]
Any such relation can be uniquely specified by
an injection mapping such as $f$.
We expect the remaining components to be consistent with $R^\kw{ptr}$
and $\kw{Vundef}$ to act as a bottom element for $R^\kw{val}$.
Definition~\ref{def:cklr} follows from these conditions.

Note that $w \Vdash R^\kw{val}$
is equivalent to $\kw{Val.inject}(f_w)$
as defined in CompCert.
Furthermore, the relational property associated to $f$,
together with the definitions of
derived relations such as $R^\kw{ptr}$ and $R^\kw{val}$,
ensure that these relations are monotonic in $w$,
in the sense that:
\[
  \AxiomC{$w \leadsto w'$}
  \UnaryInfC{$R^\kw{ptr}_w \subseteq R^\kw{ptr}_{w'}$}
  \DisplayProof
  \quad
  \AxiomC{$w \leadsto w'$}
  \UnaryInfC{$R^\kw{ptrrange}_w \subseteq R^\kw{ptrrange}_{w'}$}
  \DisplayProof
  \quad
  \AxiomC{$w \leadsto w'$}
  \UnaryInfC{$R^\kw{val}_w \subseteq R^\kw{val}_{w'}$}
  \DisplayProof
\]
However,
this is not necessarily the case for $R^\kw{mem}$.

%}}}

\subsubsection{Extensions} %{{{

The simplest CompCert KLR corresponds to memory extensions.
It uses a trivial Kripke frame and injection mapping:
\[
  \kw{ext} :=
    \langle \{*\}, \{(*,*)\}, * \mapsto (b \mapsto b), \kw{Mem.extends} \rangle
\]
Note that $\kw{ext}^\kw{val} = \kw{Val.inject}(b \mapsto b)$
is equivalent to $\kw{Val.lessdef}$,
which is the relation on runtime values that
CompCert uses in conjunction with \kw{Mem.extends}.

%}}}

\subsubsection{Injections} %{{{

In the same vein,
we can reify memory injections as the CompCert KLR:
\[
  \kw{inj} :=
    \langle
      \kw{meminj},
      {\subseteq}, %\kw{inject\_incr},
      f \mapsto f,
      \kw{Mem.inject}
    \rangle
\]
Here the type \kw{meminj} is used directly
as the KLR's worlds.
The accessibility relation $f \subseteq g$
is the inclusion of partial functions,
called $\kw{inject\_incr}$ in CompCert,
which asserts that for all block identifiers $b, b'$ and offsets $\delta$
such that $f(b) = (b', \delta)$,
then $g(b) = (b', \delta)$ as well.

%}}}

\subsubsection{Composition} %{{{

Given
$R = \langle W_R, {\leadsto}_R, f_R, R^\kw{mem} \rangle$ and
$S = \langle W_S, {\leadsto}_S, f_S, S^\kw{mem} \rangle$,
we can define
their composition $R ; S$ as:
\[
  R ; S := \langle
    W_R \times W_S, \:
    {\leadsto}_R \times {\leadsto}_S, \:
    f_S \circ f_R, \:
    R^\kw{mem} \cdot S^\kw{mem}
  \rangle
\]
Note that
$(R \cdot S)^\kw{val} = R^\kw{val} \cdot S^\kw{val}$,
and similarly for other derived relations,
so that in particular CKLR composition is associative.
In addition, the CKLR \kw{inj} and \kw{ext}
satisfy the following composition properties.

\begin{lemma}[Composition properties of CKLRs]
Consider $R, S, T$ arbitrary CompCert Kripke logical relations.
The composition of CKLRs satisfies the following properties.
\begin{gather*}
  (R ; S) ; T \equiv R ; (S ; T) \\
  \kw{inj} ; \kw{inj} \equiv
    \kw{ext} ; \kw{inj} \equiv
    \kw{inj} ; \kw{ext} \equiv
    \kw{inj} \,.
\end{gather*}
\end{lemma}

These properties will be used to obtain the overall
compositional correctness proof for CompCert
after composing our updated compilation passes.

%}}}

\subsubsection{Footprint Preservation}

In the correctness proof of the passes of CompCert
which use a memory injection as part of their simulation relation,
the semantics of external calls is expected to preserve that injection,
so that the simulation relation can be reestablished
after external calls.
In addition,
external calls are expected to only modify
the parts of the source and target memories
that are within the injection's footprint,
which is to say that their addresses are
related by the injection mapping,
and are granted non-empty permissions
in both memory states.

For instance,
in the \kw{Stacking} pass,
the target memory's stack frames
are extended with spilling locations.
These locations are used in the target program
to keep track of temporary values,
which in the source program are kept
in a local environment separate from the memory state.
The spilling locations are not part of
the injection footprint
because they are unallocated in the source memory,
and their preservation across external calls
is essential to the correctness of \kw{Stacking}.

This expectation is reasonable because
external calls
should behave uniformly between the source and target executions.
In particular,
an external function acting on the memory state
should not synthesize block identifiers and pointers,
but can only follow pointers passed as arguments
or fetched from the memory itself.
If it were to access a spilling location in this way in the target execution,
the corresponding source pointer would have to be
related to the target pointer by \kw{Val.inject},
and consequently would either be undefined
or invalid (point to a location without permissions),
so that the source program would go wrong.

Because the CKLR framework is rich enough to express
temporal properties such as the preservation of injection footprints,
this argument can be made precise as a \emph{parametricity} property
(\S\ref{sec:compcert:param}).
For now,
we show how to encode this requirement as a CKLR.

\begin{definition}
The components of the CompCert Kripke logical relation:
\[ \kw{injp} :=
  \langle
    W \times \kw{mem} \times \kw{mem},
    \leadsto_\kw{injp}, \pi_1, R_\kw{injp}^\kw{mem}
  \rangle \]
are defined as follows.
For the injection mappings $f, f'$ and
the memory states $m_1, m_2, m_1', m_2'$,
the relation $(f, m_1, m_2) \leadsto_\kw{injp} (f', m_1', m_2')$
holds when the following conditions are satisfied:
\begin{itemize}
\item $f \subseteq f'$, and any new mappings in $f'$
  must be between newly allocated blocks in $m_1'$ and $m_2'$;
\item any blocks of $m_1$ unmapped by $f$
  must remain unchanged in $m_2$;
\item any locations in $m_2$ unrelated by $f$
  to valid locations of $m_1$ must remain unchanged in $m_2'$;
\item within allocated blocks of $m_1$,
  the permissions assigned to any offset may not increase in $m_1'$
  (this helps ensure that $\leadsto_\kw{meminj}$ is transitive).
\end{itemize}
The relation $R_\kw{injp}^\kw{mem}$ is defined by the rule:
\[
  \AxiomC{$m_1 \ifr{f \Vdash \kw{Mem.inject}} m_2$}
  \UnaryInfC{$m_1 \ifr{(f, m_1, m_2) \Vdash R_\kw{injp}^\kw{mem}} m_2$}
  \DisplayProof
\]
\end{definition}

Note that the CKLR $\kw{injp}$
does not enjoy composition properties
similar to those of $\kw{ext}$ and $\kw{inj}$.
Instead,
we use the iterated simulation convention $\mathcal{C}[\kw{injp}]^*$
to absorb instances of $\kw{injp}$ in the domain convention.
In the following section,
we show how a relational parametricity theorem for
Clight can be used to introduce this component
in the compiler's overall simulation convention.

%}}}

\subsection{Relational Parametricity} \label{sec:compcert:param} %{{{

For each language interface
$\mathcal{X} \in \{ \mathcal{C}, \mathcal{L}, \mathcal{M}, \mathcal{A} \}$,
a CKLR
$R = \langle W, {\leadsto}, f, R^\kw{mem} \rangle$ can be promoted to
a simulation convention
$\mathcal{X}[R] : \mathcal{X} \Leftrightarrow \mathcal{X}$.
For example
we define $\mathcal{C}[R] : \mathcal{C} \Leftrightarrow \mathcal{C}$
as follows:
\[
    \mathcal{C}[R] := \langle
      W, \:
      ({=} \times {=} \times {R^\kw{val}}^* \times R^\kw{mem}), \:
      \Diamond (R^\kw{val} \times R^\kw{mem})
    \rangle \,.
\]
Such operators preserve composition
in the sense that $\mathcal{X}[R ; S] = \mathcal{X}[R] \cdot \mathcal{X}[S]$.

Because the semantics of CompCert languages
are expected to be well-behaved with respect to
invariance properties of the memory model,
we can prove for each language of CompCert
a parametricity theorem of the form:
\[
    \forall R \,.\,
      \llbracket p \rrbracket
        \le_{\mathcal{X}[R] \Rightarrow \mathcal{X}[R]}
      \llbracket p \rrbracket \,.
\]
Since both sides of the relation are identical,
the simulations can be iterated
to obtain similar properties in terms of $\mathcal{X}[R]^*$.

In particular,
we use this approach to establish the following properties,
which will be useful to derive our final correctness theorem:
\begin{gather}
    \forall R \,.\,
      \kw{Clight} \llbracket p \rrbracket
        \le_{\mathcal{C}[R]^* \Rightarrow \mathcal{C}[R]^*}
      \kw{Clight} \llbracket p \rrbracket
      \label{eqn:clight}
      \\
    \forall R \,.\,
      \kw{RTL} \llbracket p \rrbracket
        \le_{\mathcal{C}[R] \Rightarrow \mathcal{C}[R]}
      \kw{RTL} \llbracket p \rrbracket
      \label{eqn:rtl}
\end{gather}

%}}}

\subsection{Compiler Correctness} %{{{

Following the approach outlined in \S\ref{sec:compcert:overview},
and referring to Table~\ref{tbl:passes},
the overall simulation convention for the whole compiler
can be given as:
\[
  \mathbb{R}_\kw{CompCert} : \mathcal{C} \Leftrightarrow \mathcal{A} :=
    \mathcal{C}[\kw{injp}]^* \cdot
    \mathcal{C}[\kw{inj}] \cdot
    \kw{alloc} \cdot
    \kw{stacking} \cdot
    \kw{asmgen} \,.
\]
To verify that the passes indeed compose to
estblish correctness with respect to the simulation convention
$\mathbb{R}_\kw{CompCert} \Rightarrow \mathbb{R}_\kw{CompCert}$,
it suffices to show that
the incoming and outgoing simulations can be reconciled,
or more precisely:
\[
    \mathcal{C}[\kw{injp}]^* \cdot
    \mathcal{C}[\kw{injp}] \cdots
    \mathcal{C}[\kw{ext}] \cdot
    \mathcal{C}[\kw{inj}]
    \: \sqsubseteq \:
    \mathbb{R}_\kw{CompCert}
    \: \sqsubseteq \:
    \mathcal{C}[\kw{injp}]^* \cdot
    \mathcal{C}[\kw{inj}] \cdots
    \mathcal{C}[\kw{ext}] \cdot
    \mathcal{C}[\kw{inj}] 
\]
This follows from the properties outlined in \S\ref{sec:compcert:cklr}
and \S\ref{sec:compcert:param},
so that we can conclude:
\[
    \forall p_s \,.\,
      \kw{CompCert}(p_s) = p_t \Rightarrow
      \kw{Clight} \llbracket p_s \rrbracket
      \sqsubseteq_{\mathbb{R}_\kw{CompCert} \Rightarrow
                   \mathbb{R}_\kw{CompCert}}
      \kw{Asm} \llbracket p_t \rrbracket \,.
\]



%}}}

%}}}

\section{Related and Future Work} %{{{

\subsection{Certified Compilation} %{{{

CompCert,
Compositional CompCert,
Tahina's semantics,
SepCompCert.

Ahmed's work to mix languages.

%}}}

\subsection{Game Semantics} %{{{

Game semantics first came to prominence
in the context of functional programming languages when
it was used to construct
the first fully abstract models for PCF
\cite{pcfajm,pcfho},
building on partial game models for linear logic
\cite{gsll,gsllaj}.
Subsequent work
extended the approach to account for a variety of
language features such as
state \cite{gsia},
control \cite{gscontrol} and
nondeterminism \cite{gsnondet}.
Although much of this work initially
targeted sequential programming languages,
investigations of \emph{concurrent} game models
have been paritcularly prolific and fruitful
\cite{x,y,z,gsconcur},
eventually coming full circle with Meli\`es
fully complete model of linear logic \cite{t}.

More low-level applications of games have also been proposed,
for instance see the work on interface theories
and interface automata \cite{ia,gmos,itcd,gtf},
and games have been used in the context of modal logic
to model properties of open systems
\cite{atl,altref}.

Although in this paper we have only considered a very simple game model,
in the future we hope to draw from the slew of existing game semantics research
to extend the framework of refinement-based game semantics
to a variety of settings,
and in particular to extend our model
to stateful and concurrent strategies.

%}}}

\subsection{Abstraction} %{{{

Although our treatment of abstraction (\S\ref{sec:monad:abs})
remains quite rudimentary,
we hope to develop a more comprehensive understanding
of abstraction in the context of refinement-based game semantics
by exploring connections with existing research,
in particular with
the theory of abstract interpretation \cite{absint,aif}
and its applications to game models \cite{aigp}.

On a more practical side,
\emph{certified abstraction layers} \cite{popl15,osdi16,ccal}
have been proposed as a general verification methodology
and used to prove the correctness of the operating system kernel CertiKOS.
We hope refinement-based game semantics
can eventually be used to generalize
certified abstraction layers to heterogenous settings.

%}}}

\subsection{Monads and Kleene Algebra} %{{{

Monads equipped with
typed Kleene algebra structures
have been studied under the names of
\emph{preordered monads} \cite{pom} and
\emph{Kleene monads} \cite{kleenem},
in the latter case establishing various decidability results.
Although our interaction monad (\S\ref{sec:monad:def})
comes short of constituting a Kleene algebra
(in particular, as in FailKAT \cite{failkat}
the axiom $x \cdot \mathbf{0} = \mathbf{0}$ does not hold),
it may be possible to embed it in a more well-behaved setting
and take advantage of some of these results.
In particular,
Damien Pous' work on Kleene algebra with converse
and extensive library of associated Coq tactics
offer promising opportunities for proof automation.

Interaction trees \cite{itrees}
represent another approach
related to our interaction monad,
formalized in Coq in the context of the DeepSpec project \cite{deepspec}.
Interaction trees
come out of the research on \emph{algebraic effects},
and are constructed as the free monad
for a user-specified effect signature.
Rather than sets of traces,
their representation uses Coq's coinductive types.

%}}}

\subsection{Logical Relations and Relators} %{{{

Esp. importance of relators.
KLRs.

%}}}


\subsection{TODO}

\begin{itemize}
\item
  Limitation:
  we model linking as in CompCompCert,
  but do not yet verify a form of linking as specified in SepCompCert
\item
  Limitation:
  liveness properties not expressible as refinement
\item
  Linear vs branching time,
  Vardi's argument for trace semantics
  as the externally observable things
\end{itemize}

%}}}

\bibliographystyle{abbrv}
\bibliography{lwcc}

\end{document}
