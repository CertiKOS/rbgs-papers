\begin{figure} % fig:readwritehello {{{
  \centering
  \begin{minipage}{0.39\textwidth}
      %\begin{lstlisting}[title={secret.c}]
      %#include <unistd.h>
      %char msg[] = "uryyb, jbeyq!\n";
      %int main()
      %{
      %        write(1, msg, sizeof msg - 1);
      %        return 0;
      %}
      %\end{lstlisting}
    \begin{center}
      secret.s
    \end{center}
    \vspace{-1.5em}
      \begin{minted}[fontsize=\small,frame=single,numbersep=0.3em]{gas}
.globl main
main: pushl $13
      pushl $msg
      call rot13
      pushl $1
      call write
      addl $12, %esp
      movl $0, %eax
      ret
.data
msg:  .string "hello, world!\n"
      \end{minted}
    \vspace{-1em}
      \begin{minted}[fontsize=\small,frame=single,numbersep=0.3em]{bash}
$ cc -o secret secret.s rot13.c
$ ./secret
uryyb, jbeyq!
$ cc -o decode decode.c rot13.c
$ ./secret | ./decode
hello, world!
  \end{minted}
  \end{minipage}
  \hspace{0.8em}
  \begin{minipage}{.57\textwidth}
    \begin{center}
      rot13.c
    \end{center}
    \vspace{-1.5em}
      \begin{minted}[fontsize=\small,frame=single,numbersep=0.3em]{c}
void rot13(char *buf, int len) {
  for (int i = 0; i < len; i++)
    if ('a' <= buf[i] && buf[i] <= 'z')
      buf[i] = (buf[i] - 'a' + 13) % 26 + 'a';
}
      \end{minted}
    \vspace{0.4em}
    \begin{center}
      decode.c
    \end{center}
    \vspace{-1.5em}
      \begin{minted}[fontsize=\small,frame=single,numbersep=0.3em]{c}
#include <unistd.h>
extern void rot13(char *, int);
int main() {
  char buf[100];
  int n = read(0, buf, sizeof buf);
  rot13(buf, n);
  write(1, buf, n);
  return 0;
}
      \end{minted}
  \end{minipage}
  \caption{Two programs which use a common library
    are compiled and made to
  interact through a pipe.}
  \label{fig:readwritehello}
\end{figure}