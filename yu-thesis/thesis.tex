%%%%%%%%%%%%%%%%%%%%%%%%%%%%%%%%%%%%%%
% yale_thesis.tex
% Alexander Cerjan
% 2014/04/07
%
% A bare, sample template for a Yale PhD thesis using yalephd.cls
%%%%%%%%%%%%%%%%%%%%%%%%%%%%%%%%%%%%%%

\documentclass[letterpaper,11pt]{yalephd}
% remove draft option for final printing.
% font size must be between 10pt-12pt.

\usepackage{geometry} % you need this for yalephd.cls to work.
\usepackage{graphicx} % you probably want the rest of these.
\usepackage{dcolumn}
\usepackage{bm}
\usepackage{amsmath}
\usepackage{amsfonts}
\usepackage{amssymb}
\usepackage{appendix}
\usepackage{comment}
\usepackage{cmll}
% \usepackage{cite}
% \usepackage{notoccite}

\usepackage{ebproof}
\usepackage{amsthm}
\usepackage{stmaryrd}
\usepackage{subcaption}

\usepackage[square,sort,semicolon]{natbib}
%\usepackage[draft=false,allcolors=blue,colorlinks]{hyperref}
\usepackage[hidelinks]{hyperref}

\usepackage{minted}
\usepackage{tikz}
\usepackage{tikz-cd}
\usepackage{listings}
\usepackage{booktabs}

\usetikzlibrary{arrows.meta,positioning,fit,calc}

\newtheorem{theorem}{Theorem}[chapter]
\newtheorem{lemma}[theorem]{Lemma}
\newtheorem{example}[theorem]{Example}
\newtheorem{remark}[theorem]{Remark}
\theoremstyle{definition}
\newtheorem{definition}[theorem]{Definition}
\newtheorem{challenge}{Challenge}

\hyphenation{Comp-Cert}
\hyphenation{Comp-CertX}
\hyphenation{Comp-CertO}
\hyphenation{Comp-CertOX}
\hyphenation{Comp-CertOE}
\hyphenation{Comp-CertM}
\hyphenation{Certi-KOS}

% \bibliographystyle{abbrvunsrt}

\begin{document}

% Need to define title before the abstract.
\title{Building Certified Systems with Compositional Semantics}
% capitalize the first letter in every word except conjunctions and prepositions
\author{Yu Zhang}
\advisor{Zhong Shao}
\date{December, 2025} % usually not \today.

\newcommand{\todo}[1]{\textcolor{red}{#1}}
\newcommand{\kw}[1]{\ensuremath{ \mathsf{#1} }}
\newcommand{\C}{\ensuremath{ \mathcal{C} }}
\newcommand{\A}{\ensuremath{ \mathcal{A} }}
\newcommand{\que}{\circ}         % superscript for questions
\newcommand{\ans}{\bullet}       % superscript for answers
\newcommand{\termi}[1]{\langle {#1} ]}
\newcommand{\encap}[1]{[ {#1} \rangle}
\newcommand{\at}{\mathbin@}
\newcommand{\red}[1]{\textcolor{red}{#1}}
\newcommand{\ifr}[1]{\mathrel{[{#1}]}}
\newcommand{\idsc}{\mathbf{id}} % identity simulation convention
\newcommand{\intl}[1]{#1^0}
\newcommand{\envstep}{\leadsto}
\newcommand{\sysstep}{\rightarrowtail}
\newcommand{\emptysig}{0}
\newcommand{\vcomp}{\fatsemi}
\newcommand{\jr}{\mathsf{Y}}
\newcommand*{\twoheadleftrightarrow}{%
  \twoheadleftarrow
  \mathrel{\mkern-15mu}%
  \twoheadrightarrow
}
\newcommand{\lensarrow}{\rightleftarrows}

% String diagrams {{{
\colorlet{scsdbg}{lightgray!50!white}
\colorlet{tssdbg}{lightgray!50!white}
\colorlet{memsdbg}{blue!50!white}
\colorlet{mmemsdbg}{blue!50!white}
\colorlet{penvsdbg}{green!50!white}
% }}}

% All the stuff at the front of your thesis.
\frontmatter

\begin{abstract}

  Formal verification has emerged as
  the most rigorous approach for building reliable computer systems.
  Certified systems, in particular,
  are accompanied by a formal specification
  and a machine-checkable proof of correctness,
  enabling independent verification by third parties.

  However, verifying large-scale, heterogeneous systems
  presents significant challenges that current techniques cannot adequately address.
  Such systems require compositional methods
  that can flexibly integrate diverse verification approaches,
  including program logic, certified compilation, and data abstraction.
  While compositional semantics offers a promising foundation,
  existing frameworks are typically built on specialized operational models
  that resist integration with one another.

  This dissertation presents a three-dimensional algebraic compositional structure
  that unifies diverse verification techniques within a single coherent foundation
  and enables systematic composition across program modules,
  abstraction levels,
  and system state components.
  This work first extends CompCertO to create the OpenTX and OpenTE frameworks,
  which conform to the three-dimensional compositional structure
  and demonstrate practical verification capabilities.
  Building on these foundations, the dissertation then develops
  a more general strategy-based semantic model with refinement conventions
  that also conforms to the three-dimensional approach
  and provides the core framework for compositional verification.
  All components are fully mechanized in the Rocq proof assistant,
  and their capabilities are demonstrated
  through multiple case studies.

\end{abstract}

\maketitle
\makecopyright{2025} % change as needed.
\tableofcontents
\listoffigures % remove this if you have no figures.
\listoftables % remove this if you have no tables.

\chapter{Acknowledgements} % this needs to be before \mainmatter.

I am deeply grateful to my advisor,
Prof. Zhong Shao, for his invaluable guidance,
encouragement,
and unwavering support throughout my PhD journey.
During challenging times---especially throughout the pandemic and
while navigating VISA issues---he stood by me,
offering both understanding and
practical help so that I could remain focused on my research.

I owe special thanks to Jérémie Koenig,
with whom I had the great fortune to work closely
for countless hours.
Together,
we explored sophisticated semantic models,
developed Rocq proofs,
and co-authored papers.
His generosity in sharing knowledge,
his patience in explanation,
and his enthusiasm in research
have been a constant source of inspiration.
Many of the core ideas in this dissertation
originate from his insights,
for which I am sincerely indebted.

I would also like to thank
Profs. Ruzica Piskac, Yuting Wang, and Alex Lew
for serving on my thesis committee.
Their thoughtful feedback has greatly improved this work.
I am particularly grateful to Prof. Yuting Wang
and his student Ling Zhang,
whose discussions have been especially helpful.

I am fortunate to have shared my time
at Yale
with the supportive community of the FLINT group
and the broader CS department.
I would like to especially acknowledge
Yuyang Sang, Vilhelm Sjöberg, Bowen Huang, Jialu Zhang,
Ning Luo, Arthur Oliveira Vale, and Yixuan Chen,
along with many others,
for their collaboration, advice, and friendship.

Finally,
I owe my deepest gratitude to my parents,
whose unwavering love and support
have been the foundation of all my pursuits.
Their encouragement has given me
the strength to follow my path with confidence.

% Starts proper arabic numbering of pages and chapters.
\mainmatter

\chapter{Introduction}

Formal verification has emerged as the gold standard
for ensuring correctness in computer systems,
offering mathematical proofs that provide
unparalleled assurances compared to conventional testing.
However, formal verification faces significant challenges
when applied to large-scale and heterogeneous systems.
Such systems combine diverse operational paradigms,
complex stateful interactions,
and multiple abstraction levels,
creating substantial obstacles
for traditional verification approaches.

Traditional verification frameworks
rely on simplified, monolithic operational semantics models.
While mathematically tractable,
these models have significant limitations
for complex, real-world systems.
They assume closed, static environments
that cannot adequately model
the dynamic interactions characteristic of modern computing systems.
Moreover, monolithic models
lack the flexibility to integrate
multiple verification methodologies and tools,
limiting their practical scalability.

\section{Verification Challenges in Complex Systems}

\subsection{Motivating Examples}

Software development typically follows two broad programming patterns.
The first pattern focuses on implementing data structures or algorithms
as reusable libraries for other programs.
For this style of programming,
the bounded queue example from \citet{rbgs-cal} is adopted,
whose source code is shown in Fig.~\ref{fig:bq-code}.
The queue is built in two layers:
$\kw{rb.c}$ exposes a ring buffer
backed by an array along with two counters
that wrap around on overflow or underflow,
while $\kw{bq.c}$ builds on this abstraction
to provide queue operations.

\begin{figure}[t!]
  \center
  \hspace{-5.5em}
  \begin{subfigure}{0.6\textwidth}
\begin{minted}[fontsize=\small,frame=single,numbersep=0.3em]{c}
static int c1, c2;
static V buf[N];

int inc1() { int i = c1++; c1 %= N; return i; }
int inc2() { int i = c2++; c2 %= N; return i; }
V get(int i) { return buf[i]; }
void set(int i, V val) { buf[i] = val; }
\end{minted}
    \vspace{-1em}
    \subcaption{The translation unit $\kw{rb.c}$}
    \label{fig:rb}
  \end{subfigure}
  \hspace{0.5em}
  \begin{subfigure}{0.48\textwidth}
\begin{minted}[fontsize=\small,frame=single,numbersep=0.3em]{c}
extern int inc1(void);
extern int inc2(void);
extern V get(int i);
extern void set(int i, V val);

void enq(V val) { set(inc2(), val); }
V deq() { return get(inc1()); }
\end{minted}
    \vspace{-1em}
    \subcaption{The translation unit $\kw{bq.c}$}
    \label{fig:bq}
  \end{subfigure}
  \hspace{-5.5em}
  \caption{Running example, adapted from \citet{rbgs-cal}.
    The component $\kw{rb.c}$
    implements a ring buffer of capacity $N$
    by encapsulating an array
    and two counters. It is used by %the component
    $\kw{bq.c}$ to implement a
  bounded queue.}
  \label{fig:bq-code}
  %\caption{The state of a ring buffer,
  %  made of two counters and a fixed-size array,
  %  is encapsulated behind a simple interface.}
  %\label{fig:rb}
  %\caption{This component relies on the ring buffer primitives
  %  provided in Fig.~\ref{fig:rb} to implement a bounded-size queue.}
  %\label{fig:bq}
\end{figure}


The second pattern centers on executable programs
that are algorithmically simple
but interact extensively with their external environment.
A representative example is shown in Fig.~\ref{fig:readwritehello},
where two programs share a common C library
and are designed to work together.
The 32-bit x86 assembly program $\kw{secret.s}$ produces an encoded message,
which is then passed through a pipe to $\kw{decode.c}$ for decoding.
Unlike the bounded queue,
the difficulty here lies not in the data structure itself
but in capturing and verifying
the correctness of rich external interactions across heterogeneous components.

\begin{figure} % fig:readwritehello {{{
  \centering
  \begin{minipage}{0.39\textwidth}
      %\begin{lstlisting}[title={secret.c}]
      %#include <unistd.h>
      %char msg[] = "uryyb, jbeyq!\n";
      %int main()
      %{
      %        write(1, msg, sizeof msg - 1);
      %        return 0;
      %}
      %\end{lstlisting}
    \begin{center}
      secret.s
    \end{center}
    \vspace{-1.5em}
      \begin{minted}[fontsize=\small,frame=single,numbersep=0.3em]{gas}
.globl main
main: pushl $13
      pushl $msg
      call rot13
      pushl $1
      call write
      addl $12, %esp
      movl $0, %eax
      ret
.data
msg:  .string "hello, world!\n"
      \end{minted}
    \vspace{-1em}
      \begin{minted}[fontsize=\small,frame=single,numbersep=0.3em]{bash}
$ cc -o secret secret.s rot13.c
$ ./secret
uryyb, jbeyq!
$ cc -o decode decode.c rot13.c
$ ./secret | ./decode
hello, world!
  \end{minted}
  \end{minipage}
  \hspace{0.8em}
  \begin{minipage}{.57\textwidth}
    \begin{center}
      rot13.c
    \end{center}
    \vspace{-1.5em}
      \begin{minted}[fontsize=\small,frame=single,numbersep=0.3em]{c}
void rot13(char *buf, int len) {
  for (int i = 0; i < len; i++)
    if ('a' <= buf[i] && buf[i] <= 'z')
      buf[i] = (buf[i] - 'a' + 13) % 26 + 'a';
}
      \end{minted}
    \vspace{0.4em}
    \begin{center}
      decode.c
    \end{center}
    \vspace{-1.5em}
      \begin{minted}[fontsize=\small,frame=single,numbersep=0.3em]{c}
#include <unistd.h>
extern void rot13(char *, int);
int main() {
  char buf[100];
  int n = read(0, buf, sizeof buf);
  rot13(buf, n);
  write(1, buf, n);
  return 0;
}
      \end{minted}
  \end{minipage}
  \caption{Two programs which use a common library
    are compiled and made to
  interact through a pipe.}
  \label{fig:readwritehello}
\end{figure}

\subsection{Core Verification Challenges}

These examples illustrate the core challenges
that effective verification frameworks must address
for real-world, large-scale systems:

\begin{challenge}
  \label{challenge:abstraction}
  \textbf{Supporting Multiple Levels of Abstraction.}
  Large-scale systems require reasoning
  across multiple abstraction levels,
  each governed by different mathematical frameworks.
  The challenge lies in coherently connecting
  these diverse perspectives.
  For instance,
  microprocessor circuit algebra
  differs fundamentally from
  assembly language operational semantics,
  which in turn differs from
  operating system abstractions
  and high-level programming language semantics.
  Functional specification languages
  provide clean, abstract behavioral views
  precisely because they avoid low-level operational details.

  This challenge applies beyond entire systems
  to individual data structures.
  Complex structures like maps, queues, or trees
  are typically specified functionally in terms of mathematical sets or relations
  but implemented through intricate low-level memory manipulations.
  Data abstraction exemplifies the broader challenge:
  coherently managing relationships
  between multiple views of the same system
  at different abstraction levels.

  Concretely,
  in the bounded queue example,
  the buffer can be specified abstractly
  as a function from integers to values,
  rather than as an array in the C memory model.
  Likewise, the queue itself is naturally described
  as a sequence of values,
  independent of the details of its implementation.
  Handling these different views coherently---whether across entire systems
  or within a single data structure---is essential for scalable and trustworthy verification.

\end{challenge}

\begin{challenge}
  \label{challenge:interaction}
  \textbf{Reasoning in Open Context.}
  Complex programs rarely execute in isolation.
  They interact with other components, libraries,
  or even entirely separate processes,
  and these interactions cannot always be anticipated
  in advance.
  For a verification framework to be realistic and scalable,
  it must support reasoning about program behavior
  in such open contexts%
  ---without relying on the closed-world assumption
  that all possible interactions are fixed
  and known ahead of time.
  Fundamentally, closed-world reasoning requires
  the behavior of external modules to be fixed upfront,
  while open-world reasoning requires only their signatures
  to be fixed,
  allowing the actual behavior to vary
  as long as it respects the interface contract.

  To understand this limitation,
  consider how traditional program semantics handle the bounded queue example.
  Under the closed-world assumption,
  the behavior of \kw{bq.c} remains unclear
  without having the underlying ring buffer functions provided in advance.
  This creates two unsatisfactory options.
  First, one could verify \kw{bq.c} together with
  \kw{rb.c}'s concrete implementation,
  but this approach obviously lacks modularity.
  The second option is to provide the behavior through
  a fixed specification for the ring buffer, say \kw{Spec}$_\kw{rb}$.
  While this seems more modular at first,
  it requires the specification to be completely fixed upfront
  and forces us to choose a particular data representation.
  Once committed to \kw{Spec}$_\kw{rb}$,
  the verification becomes locked into that specific way of thinking
  about the ring buffer's state and operations,
  preventing later substitution of different implementations
  or reasoning about the queue's behavior
  in contexts where the ring buffer might behave differently.

  In contrast, open-world reasoning would allow
  the bounded queue's correctness to be verified
  without presuming how the environment will use it.
  Similarly, when the queue invokes the underlying ring buffer,
  the framework must not hard-code assumptions
  about how the buffer will respond;
  rather, the framework must allow
  the queue's behavior to be modeled
  independently of its precise context of use.
  Reasoning in this way
  ensures that properties established
  for the queue remain valid no matter where, or how,
  it is deployed.

  The same challenge appears
  in the example of two programs communicating through a pipe.
  Each program is simple on its own,
  but their joint behavior
  depends on a dynamic, asynchronous exchange of data.
  Here too,
  verification cannot assume a fixed pattern of interaction.
  Instead,
  it must treat each program as an open component,
  reasoning about its behavior abstractly
  in terms of the inputs
  it may receive and the outputs it may produce.

  In short,
  the difficulty lies not only in proving
  that individual components behave correctly,
  but also in ensuring that
  such proofs remain robust
  when those components are placed
  into unpredictable or evolving environments.
  Supporting this style of open-world reasoning%
  ---where correctness is preserved under arbitrary contexts
  and interactions---is a central challenge
  addressed by this work.

\end{challenge}

\begin{challenge}
  \label{challenge:encapsulation}
  \textbf{State Encapsulation.}
  In complex systems,
  interactions between stateful components
  can cause unintended interference.
  Scalable verification requires
  rigorous state partitioning and encapsulation
  to ensure modular correctness proofs remain valid
  despite dynamic component interactions.

  State encapsulation provides exactly this safeguard.
  By restricting all or part of a component's internal state
  from being accessed directly by its environment,
  it guarantees that
  the state can only be influenced
  through the component's well-defined interface.
  A concrete example is the bounded queue implementation:
  the underlying ring buffer maintains its data in an array
  along with counters for indexing,
  but these internal details should remain hidden.
  The interface of the ring buffer exposes only operations
  such as get and set,
  and the queue itself in turn exposes only its abstract operators
  without revealing its internal representation.
  This disciplined boundary ensures that clients of the queue
  cannot tamper with the ring buffer's internal state,
  and the queue cannot depend on how the ring buffer itself is implemented,
  beyond its interface contract.

  Such encapsulation not only prevents accidental interference
  but also enforces a disciplined boundary of interaction,
  allowing proofs about one component
  to remain stable regardless of the behaviors of others.
  In this way,
  encapsulation strengthens modular reasoning
  and supports the scalable verification of complex systems.

\end{challenge}

\begin{challenge}
  \label{challenge:compilation}
  \textbf{Certified Compilation.}
  Verifying high-level source code alone is insufficient
  because compilation itself can introduce errors.
  End-to-end correctness requires frameworks
  that integrate with certified compilers,
  extending correctness guarantees
  from high-level specifications to machine-executable binaries.

  Certified compilers such as CompCert
  address part of this challenge
  by proving the correctness
  of the compilation process.
  However, ensuring end-to-end correctness
  requires the verification framework
  to seamlessly integrate with this correctness proof.
  Only then can guarantees obtained
  at the source-level components be reliably
  transferred to the generated target program.

  Additionally,
  this integration requires accurately modeling
  how compiled programs transition into executable entities.
  Crucially, for a proof artifact to be truly end-to-end,
  the framework must also model
  the behavior of the compiled assembly programs themselves.
  Capturing these low-level details,
  together with all relevant aspects of
  runtime behavior,
  ensures that the reasoning chain---from high-level specification
  through compiler transformations to final execution---remains unbroken.
  Without this,
  gaps can arise between source-level proofs
  and target-level execution,
  undermining the entire verification effort.
  Bridging this gap is therefore critical.

\end{challenge}

\begin{challenge}
  \label{challenge:multi-language}
  \textbf{Handling Heterogeneous Components.}

  Real-world systems are often heterogeneous,
  combining components written in different programming languages,
  from low-level assembly routines to high-level,
  expressive source code.
  An effective verification framework
  must therefore provide a unified way
  to reason across this spectrum,
  enabling correctness guarantees
  to span both abstract specifications
  and highly optimized machine-level implementations.

  This heterogeneity is not just a matter of convenience
  but of necessity.
  Performance-critical software%
  ---such as operating systems,
  cryptographic libraries,
  or multimedia codecs%
  ---frequently implements their hot paths
  in hand-written assembly
  to maximize performance
  through hardware-specific optimizations.
  For instance,
  cryptographic primitives
  like AES or SHA are often coded
  in assembly to leverage specialized CPU instructions,
  while the surrounding protocol logic
  remains in C or higher-level languages.
  Similarly, real-time systems
  such as network drivers or signal-processing routines
  may utilize assembly for performance-critical sections,
  while delegating control flow and system integration
  to higher-level code.

  The motivating example,
  shown in Figure \ref{fig:readwritehello},
  illustrates this phenomenon directly:
  the component $\kw{secret.s}$ is implemented in x86 assembly to encode a message,
  while the companion program $\kw{decode.c}$---written in C---deciphers it.
  Moreover, the pipe connecting them can itself be viewed as a heterogeneous component,
  since its semantics is defined neither in C nor assembly
  but rather through operating system abstractions.
  Although each component is algorithmically simple,
  their correctness hinges on coherent cross-language reasoning
  about how the two interact through this heterogeneous communication mechanism.

  A verification framework
  that cannot bridge this gap risks
  leaving its most critical
  and performance-sensitive components
  outside the scope of formal guarantees.
  By contrast,
  a robust framework must treat assembly
  and high-level languages
  as part of a single, coherent reasoning space,
  allowing correctness proofs
  to follow programs seamlessly across abstraction levels.

\end{challenge}

\section{Insufficient Existing Approaches}
\label{sec:intro:litreview}

Substantial research has sought to
address the challenges outlined above.
The key to formal verification
is the underlying semantic model that defines
how program behavior is represented and reasoned about.
This section reviews existing approaches
according to their choice of semantic model,
examining how each partially addresses
the verification challenges,
highlighting their contributions and limitations.

\subsection{Operational Semantics}

Operational semantics is the most commonly used approach
because it is intuitive and amenable to reasoning.
Programs are modeled through state transitions
that directly correspond to execution steps,
making the semantic model easy to understand and manipulate.

% Existing work has been conducted
% along the following lines.
% First,
% verification frameworks
% have been built on top of CompCert
% so that an end-to-end
% correctness proof can be delivered,
% thereby solving challenge (c).
% Second,
% a series of work
% have extended CompCert to support
% more advanced programming paradigms,
% including enabling compositional verification.

A prominent example is the Verified Software Toolchain (VST)\citep{vst},
which builds program logic on top of CompCert's Clight semantics.
VST integrates separation logic for state separation,
connects with CompCert's compiler correctness,
and provides strong automation support.
It has been extended with ITree\citep{itree}
to connect with higher-level semantic models
and successfully verify complex systems like network servers\citep{itrees}.
However, traditional operational semantics approaches
face fundamental limitations:
the semantic model is fixed
and interaction patterns are predetermined,
making it difficult to model flexible interactions
like those in the pipe-based rot13 example
or to support heterogeneous component interaction seamlessly.
While new operational models can be developed for specific tasks,
each new verification challenge
often requires developing yet another operational model.

% At the same time,
% VST is implemented as an extension to CompCert and
% is tied closely to the Clight semantics.
% It is hard to use as the starting point for a larger ecosystem,
% as it can only verify systems
% whose functionality can be specified completely
% within the VST logic.

\subsection{Denotational Semantics}

A more flexible approach uses denotational semantics
that associates program behavior with mathematical domains,
leveraging compositional structures and orders
in those domains to develop verification results.
The Interaction Specification (ISpec) framework \citep{rbgs-cal}
exemplifies this category,
using the freely completely distributive (FCD) lattice \citep{cspdnd}
to model component interaction with environments.
ISpec supports open context reasoning through dual-nondeterminism,
refinement, and flexible compositional structures.
It also enables interaction between heterogeneous components.
However, denotational approaches typically lack connection
to practical compiler correctness.
There is a significant gap between
the mathematical elegance of denotational models
and the concrete semantic models used in certified compilers,
which often employ entirely different notions of correctness.

\subsection{Compositional Certified Compilers}

Another line of work focuses on compositional certified compilers
that use open semantics.
Examples include CompCertX\citep{popl15}, CompCertO\citep{compcerto}, CompCertM\citep{compcertm},
and Compositional CompCert\citep{compcompcert}.
By modeling context directly within program semantics,
compositional certified compilers can handle open contexts
and support flexible interaction.
However, such semantics typically focus on compiler correctness
and do not support data abstraction and encapsulation
for general verification tasks.

CompCertX addresses Challenge~\ref{challenge:abstraction}
since it is integrated with the Certified Abstraction Layer (CAL) framework\citep{popl15}
for data abstraction,
enabling large-scale verification through a layered approach.
However, its openness is limited compared to fully open semantics,
and as a consequence,
verification of an individual component
relies on its downstream layers being verified first,
limiting its ability to address Challenge~\ref{challenge:interaction}.

Conversely, CompCertO \citep{compcerto} introduces fully open semantics
to describe individual component behavior
and model their interaction with environments,
effectively addressing Challenge~\ref{challenge:interaction}.
It uses simulation conventions to relate
program component behaviors
across different abstraction levels,
enabling more flexible reasoning
about multi-language systems.
However, CompCertO lacks built-in data abstraction support,
making it difficult to address Challenge~\ref{challenge:abstraction}
and Challenge~\ref{challenge:encapsulation}
in practice.

\subsection{Event-Based Semantics}

Event-based semantics like game semantics offers another option,
providing a hybrid approach between operational and denotational semantics
that describes program behavior in terms of traces of observable events.
This approach is conceptually appealing,
but most existing work focuses on traces at a single abstraction level,
which does not address the need to reason across multiple levels of abstraction.
Additionally, event-based approaches are typically not connected
with practical compiler correctness guarantees.

The DimSum framework \citep{dimsum}
exemplifies this approach,
employing a language-agnostic, event-based semantics
as a generic framework for multi-language semantics.
Component state implicitly evolves as events accumulate
and cannot be accessed by the environment,
thus achieving state encapsulation
and addressing Challenge~\ref{challenge:encapsulation}.
DimSum's key feature is the semantics wrapper
that translates components written in high-level languages
into components using low-level interaction interfaces,
addressing Challenge~\ref{challenge:multi-language}.
The semantics wrapper translates events
at different abstraction levels
using angelic and demonic choices,
which can be viewed as
the functional form of CompCertO's simulation conventions.
While DimSum provides an elegant solution to Challenge~\ref{challenge:abstraction}
through its multi-level event translation mechanism,
it uses only a simplified compiler,
failing to address Challenge~\ref{challenge:compilation}
and the broader issue of connecting with real compiler correctness guarantees.

% \subsection{Miscellaneous Frameworks}

% \subsubsection{Conditional Contextual Refinement (CCR)}

% CCR combined (vertical) refinement and
% (spatial) separation logic
% into a unified, mechanized framework.
% It also supports certified compilation and state encapsulation.

\section{Compositional Semantics Along Three Dimensions}

\begin{figure}
  \[
    \begin{array}{c@{\qquad}c}
      \begin{tikzpicture}[x=1cm,y=1cm,line cap=round]
        % Tunables
        \def\L{3}        % length of the left horizontal lines
        \def\bx{1.5}      % x center of the refinement boxes
        \def\bw{2.5}       % box width (cm)
        \def\bh{0.5}      % box height (cm)
        \def\gap{0.5}      % vertical gap between a line and the box below it

        % Helper macro for one tier: (line_y, label_text)
        \newcommand{\tier}[2]{%
          \draw[line width=0.6pt] (0,#1) -- (\L,#1);
          \node[draw,rounded corners=8pt,minimum width=\bw cm,minimum height=\bh cm,inner sep=5pt]
          at (\bx,#1-\gap) {refinement};
          \node[anchor=west] at (\L,#1) {#2};
        }

        % Tiers
        \tier{3.0}{high-level spec}
        \tier{2.0}{program spec}
        \tier{1.0}{C program}

        % Bottom line + last label (no box below)
        \draw[line width=0.6pt] (0,0) -- (\L,0);
        \node[anchor=west] at (\L,0) {Asm program};
      \end{tikzpicture}
      &
      \begin{tikzpicture}[>=latex]

        % Nodes
        \node[draw,rounded corners=10pt,minimum width=1.5cm,minimum height=2cm] (handler) {handler};
        \node[draw,rounded corners=10pt,minimum width=2cm,minimum height=2cm,right=0.5cm of handler] (refine) {component};
        \node[draw,rounded corners=10pt,minimum width=1.5cm,minimum height=2cm,right=0.5cm of refine] (client) {client};

        % Double arrows between handler and refinement

        \draw[->] ([yshift=0.6cm]handler.east) -- ([yshift=0.6cm]refine.west);
        \draw[<-] ([yshift=0.2cm]handler.east) -- ([yshift=0.2cm]refine.west);
        \draw[->] ([yshift=-0.2cm]handler.east) -- ([yshift=-0.2cm]refine.west);
        \draw[<-] ([yshift=-0.6cm]handler.east) -- ([yshift=-0.6cm]refine.west);

        \draw[->] ([yshift=0.6cm]refine.east) -- ([yshift=0.6cm]client.west);
        \draw[<-] ([yshift=0.2cm]refine.east) -- ([yshift=0.2cm]client.west);
        \draw[->] ([yshift=-0.2cm]refine.east) -- ([yshift=-0.2cm]client.west);
        \draw[<-] ([yshift=-0.6cm]refine.east) -- ([yshift=-0.6cm]client.west);

        % Horizontal lines above and below refinement
        \draw (1,1.3) -- (3.6,1.3);
        \draw (1,-1.3) -- (3.6,-1.3);
        \draw[dashed] (1,1.3) -- (1,-1.3);
        \draw[dashed] (3.6,1.3) -- (3.6,-1.3);

        \node at (1,1.8) {\shortstack[c]{interaction\\[-2pt] convention}};
        \node at (3.6,1.8) {\shortstack[c]{interaction\\[-2pt] convention}};
      \end{tikzpicture}
      \vspace{1.2ex}
      \\
      (a) & (b)
    \end{array}
  \]
  \caption{
    (a) Vertical composition that enables
    transitive composition of refinement properties
    across different abstraction levels.
    (b) Horizontal composition within a single abstraction level.
    The client, component, and handler form a call hierarchy.
  }
  \label{fig:refinement-diagram}
\end{figure}

The core insight underlying this work is that
effective verification frameworks require
\emph{compositionality along multiple dimensions}.
Rather than treating composition as a single concept,
this work recognizes that complex systems require
systematic decomposition along three orthogonal axes.

Compositional semantics provides
a systematic approach to addressing verification challenges
by enabling decomposition along these multiple dimensions.
This multi-dimensional perspective allows
large verification tasks to be broken down
into smaller, independently verifiable components
that can be reliably composed.

Specifically,
compositional semantics acts as
``semantic glue'',
enabling the use of diverse verification methods—such as program logic,
compiler correctness,
type systems,
and manual proofs—within a single cohesive framework.
This allows verification engineers
to select and apply the most suitable tools
for each component,
significantly enhancing verification flexibility and effectiveness.

This framework decomposes verification
along three orthogonal dimensions:

\begin{itemize}
  \item \textbf{Vertical Composition}:
    As illustrated in Fig.~\ref{fig:refinement-diagram}(a),
    vertical composition manages multiple levels of abstraction,
    enabling systems to be verified incrementally
    from high-level specifications down to low-level implementations.
    For example, a high-level functional specification
    can be refined through program specifications to C code,
    and finally to assembly code through compiler correctness.
    Each refinement step is verified independently,
    and the results compose transitively.
  \item \textbf{Horizontal Composition}:
    As shown in Fig.~\ref{fig:refinement-diagram}(b),
    horizontal composition addresses modular reasoning
    within a single abstraction level
    along the \emph{call hierarchy}.
    A component can be invoked by clients
    and can in turn invoke downstream handlers,
    forming a chain of function calls.
    Each component is verified independently
    in a context where it interacts with its environment
    according to specified interaction conventions.
    When verified components adhere to compatible interaction conventions,
    they can be seamlessly composed along this horizontal axis.
  \item \textbf{Spatial Composition}:
    Spatial composition introduces
    a third dimension of compositionality
    by structuring state interactions across different state representations.
    At the lowest level,
    all state is concretized in global memory.
    At higher abstraction levels,
    however, spatial composition allows
    state to be isolated and encapsulated across components,
    with each component responsible for a distinct portion of the state space.
    This enables independent reasoning about disjoint state components—whether
    concrete memory regions, abstract data structures, or logical state predicates—%
    while ensuring that interactions
    occur through disciplined interfaces.
    This explicit management of state boundaries
    strengthens modular reasoning
    and prevents unintended interference.
\end{itemize}

This work develops a
\emph{unified three-dimensional refinement algebra}
that uniformly captures
module composition,
abstraction levels,
and system states.
This algebra enables compositional reasoning
and intuitive graphical representations
through refinement diagrams,
serving as a rigorous foundation
for compositional verification
and supporting clear, intuitive reasoning about complex systems.

\section{Contributions}

\subsubsection{Contribution I: OpenTX and OpenTE Frameworks}

My first main contribution
is the development of the OpenTX and OpenTE frameworks,
which are compositional verification frameworks
based on CompCertO.
OpenTX (Open Transition system with eXternal state)
focuses on layered and spatial composition,
while OpenTE (Open Transition system with Encapsulated state)
further adds state encapsulation capabilities.

The starting point of the framework is the CompCertO's
\emph{open transition system semantics},
a well-established semantic model
explicitly designed to represent interactions
between program components and their environments.
To model interactions among components
and the abstraction levels,
the CompCertO's semantics
has a clear definition of component boundaries,
which naturally aligns with
the three-dimensional algebraic framework.
Moreover,
CompCertO employs \emph{simulation conventions}
to formally establish behavioral refinement relationships
between source- and target-level programs.
These conventions systematically link behaviors
across multiple abstraction levels%
---such as between C source code and
compiled assembly code.

Building upon the CompCertO's open transition systems
and simulation conventions,
this work makes the following extensions
to enhance CompCertO
from a compositional certified compiler
to a compositional verification framework:
\begin{itemize}
  \item Layered Composition:
    this work supplements CompCertO's model of
    translation unit linking
    with a more fundamental layered composition operator.
  \item Spatial Composition:
    this work extends the model
    with a compositional treatment of state,
    allowing both specifications
    and data abstraction
    to act independently
    on different components of the global state.
  \item Memory Separation:
    this work develops a partial commutative monoid structure
    defined on the CompCert memory model
    which bridges the gap between this compositional view
    and the global memory used by CompCert.
  \item State Encapsulation:
    this work extends the model with state encapsulation,
    and explains how the associated primitives
    interact with the model's compositional structure
  \item ClightP Language:
    this work extends the compiler's source language Clight
    to take advantage of the treatment of state.
    The resulting language ClightP
    allows the definition of component-local private variables
    which are kept separate from the global memory.
\end{itemize}

Starting from the CompCertO compiler,
this work extends it with layered composition,
spatial composition,
and memory separation
to obtain the OpenTX framework.
This design is inspired
by the earlier CompCertX work
but surpasses it
by offering a more flexible compositional structure
that fully aligns with the three-dimensional composition principle.
Building on OpenTX,
this work further introduces state encapsulation
and the ClightP semantics
to obtain the OpenTE framework.
These extensions are novel contributions
not found in the existing CompCert literature.
Importantly, all these extensions
are achieved without intrusive modifications
to the original CompCertO semantics,
thanks to the robust foundation
provided by its compositional semantics.

The OpenTE framework
conforms to the three-dimensional algebraic structure,
making it a comprehensive compositional verification framework.
However,
specifications in the OpenTE framework
must be expressed as transition systems,
which primarily emphasize internal transition steps
tailored for compiler correctness verification.
This characteristic limits
the framework's generalizability for broader verification tasks.

\subsubsection{Contribution II: Strategy Model with Refinement Conventions}

My second major contribution
is the development of a
\emph{strategy model}
based on game semantics.

The strategy model serves two key purposes.
First,
it characterizes component behaviors
exclusively through interaction events,
omitting unnecessary internal details
and thereby enabling broader applicability.
Second,
the model is highly flexible and generic,
accounting for various sophisticated reasoning techniques
such as data abstraction and memory separation.
It captures---under a uniform notion of refinement---properties as varied as program correctness results,
the semantics preservation theorem of CompCertO,
the frame property of separation logic,
and representation independence for encapsulated state.

Concretely,
this work highlights the following features
of the strategy model:
\begin{itemize}
  \item Strategy Model:
    this work presents a formalization of the semantics
    that characterizes component behaviors
    as event traces,
    and develops a series of
    composition operators
    to facilitate the definition of components.
  \item Refinement Conventions and Refinement Squares:
    this work extends the idea
    of simulation conventions in CompCertO
    further through \emph{refinement conventions},
    a generalized approach that
    explicitly relates interactions
    across different levels of abstraction.
  \item Embedding CompCertO:
    importantly, CompCertO's semantics
    can be naturally embedded
    within this strategy model,
    enhancing its versatility and integration capability.
\end{itemize}

The theory of game semantics
and its various variants
have been widely studied.
However,
the notion of refinement convention and refinement square
is novel in the context of game semantics.
Furthermore,
to my knowledge,
this is the first mechanized
semantics model based on game semantics
that is capable of
handling real-world verification tasks.

The refinement conventions
and refinement squares
are the key ingredients
to solve Challenge \ref{challenge:abstraction}
by enabling components at different abstraction levels
to be related by the same notion of refinement.
The semantics of the strategy model
captures the interaction of components
as event traces
so that the semantics inherently models
components in open context.
Various composition operators
are developed within the strategy
to make it conform to the three-dimensional algebraic structure
and to accurately represent process interactions.
These efforts effectively solve Challenge \ref{challenge:interaction}.
The strategy model defines component behaviors
through execution traces,
representing the component states entirely
as the interaction histories.
This approach inherently achieves state encapsulation,
directly resolving Challenge \ref{challenge:encapsulation}.
Additionally,
the correctness proofs of
CompCertO's compiler are seamlessly integrated into
the strategy model.
% Furthermore, the strategy model
% itself serves as a suitable semantic specification,
% effectively solving challenge \ref{challenge:specification}.
To resolve the challenge associated with
modeling program loading and execution events,
this work develops a \emph{loader} mechanism
converting CompCertO's open transition systems
into closed, standalone transition systems.
Together, the compiler correctness proofs
and the loader mechanism
effectively solve Challenge \ref{challenge:compilation}.
Lastly, the capability for multi-language reasoning (Challenge \ref{challenge:multi-language}) arises from
leveraging CompCertO's Kripke logical relations.

\subsubsection{Contribution III: Case Studies and Full Mechanization in Rocq}

This work provides a series of case studies
to demonstrate the practicality of the framework.

The first case study concerns
the verification of a bounded queue implementation (Fig.~\ref{fig:bq-code}),
adapted from \citet{rbgs-cal}.
The implementation builds a bounded queue
on top of a ring buffer.
While \citet{rbgs-cal} focused solely
on verifying the high-level specification,
this is extended to an end-to-end result
by verifying CompCertO programs as concrete implementations.
This example also highlights
how state encapsulation enables
a more modular and robust verification approach.

The second case study
examines the verification of programs
shown in Fig.~\ref{fig:readwritehello},
where two components communicate through a pipe:
one encodes a message and writes it to standard output,
while the other reads it from standard input
and decodes it.
This example captures richer patterns of component interaction
and requires mixed-language reasoning,
demonstrating the framework's ability
to handle complex external interactions.

The third case study is
the construction of a CAL instance.
This work has implemented it
using OpenTX, OpenTE, and the strategy model,
respectively.
This staged development demonstrates
how the framework evolves
to support increasingly scalable
and practical verification.
The CAL case study also underscores
how the framework's compositional structures
support generic verification frameworks
instead of ad-hoc verification tasks.

Finally,
the framework is fully mechanized
and validated
within the Rocq proof assistant,
ensuring rigorous, machine-checked proofs.
This mechanization provides practical tools
for verification engineers,
enabling them to conduct
robust and scalable verification efforts confidently.

\section{Structure of the Dissertation}

The remainder of this dissertation is structured as follows:

Chapter \ref{ch:bg} provides the necessary background
on formal verification methodologies,
certified compilation in
CompCert, CompCertO's open semantics,
certified abstraction layers,
and relevant category theory definitions.

Chapter \ref{ch:ox} presents the development of
the OpenTX framework,
introducing layered composition,
state lifting,
and the three-dimensional refinement algebra.
Chapter \ref{ch:oe} explores state encapsulation
and the ClightP language,
further detailing the OpenTE framework's compositional structure.

Chapter \ref{ch:strat} introduces the strategy model,
demonstrating its capabilities
through practical verification examples
and highlighting its balanced abstraction level.
Chapter \ref{ch:rc} elaborates on refinement conventions
and refinement squares,
showcasing how the framework uniformly
captures complex compositional verification scenarios.

Finally,
Chapter \ref{ch:application}
summarizes the applications of the framework
and evaluates its practicality.
Chapter \ref{ch:related}
discusses the related work.
Chapter \ref{ch:conclusion}
concludes the thesis and discusses future work.

%\autoref{ch:conc} concludes the thesis, summarizing the key contributions and discussing future research directions.

Portions of the work presented in this thesis
have previously appeared in peer-reviewed form.
In particular, Chapters \ref{ch:strat} and \ref{ch:rc}
are mostly based on the following publication:

\begin{itemize}
  \item Yu Zhang, Jeremie Koenig, Zhong Shao, and Yuting Wang. ``Unifying Compositional Verification and Certified Compilation with a Three-Dimensional Refinement Algebra.'' In Proceedings of the ACM on Programming Languages (PACMPL), Volume 9, Number POPL, Article 64 (January 2025), 31 pages.
\end{itemize}

This thesis extends and integrates that work with additional material, results, and perspectives.


\chapter{Background}
\label{ch:bg}

This chapter introduces the background material
needed to understand the developments in subsequent chapters.
It covers CompCert and CompCertO for certified compilation and compositional semantics,
the Certified Abstraction Layer (CAL) framework,
game semantics for modeling component interactions,
and relevant concepts from category theory.

\section{Notations}

This section introduces basic mathematical notations
used throughout the thesis.
$\varnothing$ denotes the empty set,
while $\{*\}$ denotes the singleton set
with $*$ as its only element.
For sets $S_1$ and $S_2$,
$S_1 + S_2$ represents their disjoint union,
with $\iota_1$ and $\iota_2$ denoting the canonical injections.

For relations,
I will use $\top$ for the universal relation,
relating every possible pair of elements,
while $\bot$ denotes the empty relation.
$R_1 \times R_2$ denotes the product of two relations,
and $R_1 \cdot R_2$ denotes their composition.

I will use infix notation $x \mathrel{R} y$
for $(x, y) \in R$ when convenient.
In addition, I will often write $x \mathrel{R} y \mathrel{S} z$
to mean $x \mathrel{R} y \mathrel\wedge y \mathrel{S} z$.

To reason about relations
that may evolve over time or context,
Kripke relations are used.
A Kripke relation is a relation parameterized
by a set of worlds $W$,
which models how the relation can change as the world evolves.
Concretely, I will write
\[
  R \in \mathcal{R}_W(A, B)
\]
for a family of relations $R_{w \in W} \subseteq A \times B$,
The set $W$ is equipped with
an accessibility relation $\sysstep$
to describe possible world evolutions. I write
\[
  w \Vdash a \mathbin{R} b
\]
to indicate that $a$ and $b$ are related at the world $w$.

\section{CompCert}

As a \emph{certified} compiler,
CompCert provides a mechanized proof of semantics preservation
between its source language, Clight,
and its target language, Asm,
with the entire development formalized
in the Rocq proof assistant.
To make this possible,
CompCert includes rigorous formalizations of
the semantics of both languages,
together with a precise definition of
what it means for compilation to preserve semantics.

\subsection{Transition Systems and Semantics Preservation}

CompCert defines language semantics
using \emph{transition systems}
that abstract over concrete language differences.
A transition system consists of
a set of states $S$ and the following components:
\begin{itemize}
  \item a distinguished subset of \emph{initial} states $I \subseteq S$;
  \item a step transition relation ${\rightarrow} \subseteq S \times S$;
  \item a relation $F \subseteq S \times \kw{int}$ which identifies
    \emph{final} states along with the exit codes.
\end{itemize}

For example,
the semantics $\kw{Clight}[p]$
is an instance of a transition system.
An execution of $p$ starts with an initial state $s_0 \in I$,
performs a number of transitions
\[
  I \ni s_0 \rightarrow s_1 \rightarrow \cdots \rightarrow s_n \mathrel{F} x
  \,,
\]
and terminates with status $x$ when the final state $s_n$ is reached.
States with no $\rightarrow$ or $F$ successors
\emph{go wrong} and represent undefined behavior.

\begin{remark}
  \label{rem:compcert-semantics}
  This overview abstracts away some technical details.
  For instance, CompCert uses transition labels
  to capture interactions with the operating system
  and takes special care in handling infinite executions.
  However, these aspects are largely orthogonal
  to the concerns of this thesis.
  In practice, the relevant behaviors can be
  recovered through the standard interactions
  among components via a unified interface.
  For this reason,
  I will not discuss them further in depth here.
\end{remark}

In the Clight semantics,
a program state consists of
the current control stack, environments mapping temporary variables to values,
and a global memory state.
By contrast,
states in the Asm semantics are simpler,
comprising only the machine registers
together with global memory.
In addition to these two main languages,
CompCert also formalizes
several intermediate languages.
Although these are not part of the external specification,
they play a crucial role in structuring the compiler
and in the construction of its correctness proof.

CompCert formalizes semantics preservation
as a \emph{simulation} relation $\le$
between source and target languages.
In particular,
the overall correctness theorem of CompCert
is established as follows:
\begin{equation}
  \kw{CompCert}(\kw{p.c}) = \kw{p.s}
  \quad\Longrightarrow\quad
  \kw{Clight}[\kw{p.c}] \le \kw{Asm}[\kw{p.s}]
  \,.
  \label{eqn:ccc-wp}
\end{equation}

A simulation between
two transition systems $L_1$ and $L_2$
is witnessed by a simulation relation
$R \subseteq I \times S_1 \times S_2$
between the states of the source $L_1$
and the states of the target $L_2$
that is indexed by a well-founded order $(I, \le)$.
This relation must satisfy several conditions
which ensure that
every execution of $L_1$ gives rise
to a corresponding execution of $L_2$:
\begin{itemize}
  \item For all initial states $s_1 \in I_1$,
    there exist an index $i\in I$
    and a related target initial state
    $s_2 \in I_2$ such that $(i, s_1, s_2) \in \mathrel{R}$;
  \item For all related states $(i, s_1, s_2) \in \mathrel{R}$
    and every source transition $s_1 \rightarrow_1 s_1'$,
    there exist $i' \in I$ and $s_2' \in S_2$
    such that $(i', s_1', s_2') \in \mathrel{R}$
    and either $s_2 \rightarrow_2^+ s_2'$
    or $s_2 = s_2' \wedge i' \le i$;
  \item For all related states $(i, s_1, s_2) \in \mathrel{R}$
    and every result $r$ that $s_1 \mathbin{F_1} r$,
    there exist $s_2'$ such that $s_2 \rightarrow_2^* s_2'$
    and $s_2' \mathbin{F_2} r$.
\end{itemize}
The simulation $L_1 \le L_2$
holds when such a simulation relation $R$ exists.

\begin{remark}
  Two points are worth highlighting:
  \begin{itemize}
    \item
      The definition above corresponds to \emph{forward simulation},
      which roughly states that every execution path in the source language
      can be matched by some execution path in the target language.
      Ultimately, forward simulations are often used to
      establish \emph{backward simulations},
      which reverse the simulation direction
      but involve additional subtleties.
      Under certain conditions---specifically
      when the target language is deterministic
      and the source language is receptive---the two notions coincide.

    \item As noted in Remark~\ref{rem:compcert-semantics},
      CompCert transition systems emit system call events
      via labels.
      Consequently, the final semantics-preservation theorem
      in CompCert is not stated as a simulation,
      but rather as a behavioral refinement
      between traces of events.
  \end{itemize}
  For a more detailed account of forward and backward simulations,
  and their connection to angelic and demonic nondeterminism,
  see \citet{thesis}.
  In this work,
  the focus is on compositional verification;
  therefore, the finer distinctions between simulation styles
  and the details of labeled events
  remain largely orthogonal to the development.
\end{remark}

\subsection{Compositionality}

CompCert's proof strategy relies on compositional simulations.
The compiler uses multiple compilation phases
that progressively transform programs:
$
p = p_0 \longmapsto p_1 \longmapsto \cdots \longmapsto p_n = p'
$.
To derive the correctness theorem,
a simulation proof is established for each phase:
\begin{equation}
  \kw{Clight}[p] \:=\:
  \kw{Clight}[p_0] \:\le\: \kw{RTL}[p_1] \:\le\: \cdots \:\le\: \kw{Asm}[p_n]
  \:=\: \kw{Asm}[p']
  \label{eqn:corrsteps}
\end{equation}
When the target $L_2$ of a simulation $\pi : L_1 \le L_2$
is the source of a simulation $\rho : L_2 \le L_3$,
the two can be combined, and the composite
$\pi \vcomp \rho$
is in turn a simulation of type $L_1 \le L_3$.
This allows
the successive simulation proofs in (\ref{eqn:corrsteps})
to be combined into the correctness property (\ref{eqn:ccc-wp}).

A key limitation of CompCert
is that its semantics only describe complete programs.
While CompCert supports separate compilation,
it uses a \emph{syntactic} approach to correctness
that can be formulated as:
\[
  \begin{prooftree}
    \hypo{\kw{CompCert}(\kw{a.c}) = \kw{a.s}}
    \hypo{\kw{CompCert}(\kw{b.c}) = \kw{b.s}}
    \infer2{\kw{CompCert}(\kw{a.c} + \kw{b.c}) = \kw{a.s} + \kw{b.s}}
  \end{prooftree}
\]
where the $+$ operator links two programs.

Translation units without a $\kw{main()}$ function
have undefined semantics.
The CompCert compiler can compile
these kind of programs,
but the correctness property (\ref{eqn:ccc-wp})
does not provide any guarantees.
To account for this situation
at the semantic level,
it is necessary to assign a behavior $\kw{Clight}(\kw{a.c})$
to individual translation units such as $\kw{a.c}$,
and to define an operator $\oplus$ to
model the \emph{semantic} linking process
which happens before $\kw{a.c}$
is run as part of a larger program.
This operator should be compatible with simulations,
so that for example it is possible to derive the overall correctness property
\[
  \kw{Clight}(\kw{a.c}) \oplus \kw{Clight}(\kw{a.c})
  \:\le\:
  \kw{Asm}(\kw{a.s}) \oplus \kw{Asm}(\kw{b.s})
\]
from the compiler correctness properties
associated with individual translation units.
CompCertO addresses this limitation
by introducing open semantics,
discussed in \S\ref{sec:bg:compcerto}.

\subsection{The CompCert Memory Model}

All languages in CompCert
share the same memory model \citep{compcertmm}.
The internal state that drives the transition steps
always contains a memory state $m \in \kw{mem}$
together with other language-specific state.

In essence,
a CompCert memory state
assigns to each possible memory address $(b, o) \in \kw{block} \times \mathbb{Z}$
a memory value $v \in \kw{memval}$
together with a permission level $p \in \kw{option}\,\kw{perm}$.
In addition,
a memory state contains a $\kw{nextblock}$ counter
which keeps track of the next block identifier to be allocated.
I will discuss these various components in more detail below.

\subsubsection{Memory Addresses}

The CompCert memory is divided in a number of \emph{blocks}.
As new blocks are allocated,
they are assigned a positive identifier $b \in \mathbb{N}^+$
in sequential order.
As mentioned above,
the $\kw{nextblock}$ counter within each memory state
keeps track of the smallest unallocated block identifier.
When a new block identifier is needed,
$\kw{nextblock}$ is incremented and its previous value
is used for the new block.
\[
  (b, o) \in \kw{ptr} = \kw{block} \times \mathbb{Z}
  \qquad
  (b, l, h) \in \kw{ptrrange} = \kw{block} \times \mathbb{Z} \times \mathbb{Z}
\]

Memory blocks represent independent address spaces.
Within each block,
a byte can be addressed using an offset $o \in \mathbb{Z}$.
When a new block is allocated,
a range of addresses $[\mathit{lo}, \mathit{hi})$ must be provided;
this range determines which addresses within the block are valid.
However,
rather than storing the range directly within the memory state,
the allocation operation uses it to assign initial permissions
for each address within the new block.

\subsubsection{Permissions}

Each memory address within a memory state
is assigned a permission level from the following hierarchy:
\[
  p \in \kw{option}\,\kw{perm} ::=
  \bot \mid
  \kw{nonempty} \mid
  \kw{readable} \mid
  \kw{writable} \mid
  \kw{freeable}
\]
The permissions are listed in increasing order of capability.
Each permission level encompasses
all operations allowed by lower levels,
so that for example the permission level $\kw{writable}$
represents the set of permissions
$\{ \kw{nonempty}, \kw{readable}, \kw{writable} \}$.

The $\kw{freeable}$ permission provides
exclusive access
and permits all operations
including load, store, pointer comparison, and deallocation.
The $\kw{writable}$
permission allows load, store, and pointer comparison operations
but prohibits freeing.
The $\kw{readable}$
permission restricts access to only load and
pointer comparison operations.
The $\kw{nonempty}$
indicates that the address is valid but permits only pointer comparisons.
Finally, the empty permission $\bot$
signifies addresses that are not yet allocated or have been previously freed, and no operations are permitted on such addresses.

When a block is first allocated,
addresses within the provided range
are assigned the permission level $\kw{freeable}$,
while all remaining addresses are assigned
empty permissions $\bot$.
Further memory operations may then decrease the permission level,
but can never increase it.
Memory operations which access a particular address
will first check that this address has sufficient permissions,
and fail if that is not the case.

Permissions play an important role
in the memory separation relation I will define in \S\ref{sec:ox:separation}.

\subsubsection{Memory Operations}

The memory operations include
$\kw{alloc}$ and $\kw{free}$
that allocate and deallocate memory blocks, respectively:
\begin{align*}
  \kw{alloc} & \::\: \kw{mem}
  \rightarrow \mathbb{Z}
  \rightarrow \mathbb{Z}
  \rightarrow \kw{mem} \times \kw{block}\\
  \kw{free} & \::\: \kw{mem}
  \rightarrow \kw{ptrrange}
  \rightarrow \kw{option}(\kw{mem})
\end{align*}
Each memory value $\kw{memval}$ represents the contents of exactly one byte of memory.
It may be stored as a concrete byte,
or may be identified as a particular one-byte fragment
within a larger, more abstract value
(for instance, the third byte of a given pointer).
Sequences of memory values can be loaded from or stored to memory via the following byte-level operations:
\begin{align*}
  \kw{loadbytes} & \::\: \kw{mem}
  \rightarrow \kw{ptr}
  \rightarrow \mathbb{Z}
  \rightarrow \kw{option}(\kw{list}(\kw{memval}))\\
  \kw{storebytes} & \::\: \kw{mem}
  \rightarrow \kw{ptr}
  \rightarrow \kw{list}(\kw{memval})
  \rightarrow \kw{option}(\kw{mem})\\
\end{align*}
For greater convenience,
the memory model also provides
higher-level operations
that work with typed values ($\kw{val}$)
such as integers, floats, and pointers,
eliminating the need to manipulate raw bytes directly:
\begin{align*}
  \kw{load} & \::\: \kw{mem}
  \rightarrow \kw{ptr}
  \rightarrow \kw{option}(\kw{val})\\
  \kw{store} & \::\: \kw{mem}
  \rightarrow \kw{ptr}
  \rightarrow \kw{val}
  \rightarrow \kw{option}(\kw{mem})
\end{align*}

The exact representation of memory values
is not essential to the work discussed in this section.
Therefore
I will not discuss the specifics further,
but refer the interested reader to \citet{compcertmmv2}
for more background on this topic.

\subsubsection{Memory Transformations}

The compilation passes of CompCert
often transform the structure of the memory state:
multiple blocks can merged into one;
new blocks may be introduced in the target memory
and blocks may be dropped from the source memory.
To express these transformations,
CompCert introduces \emph{memory extensions} and \emph{memory injections}
as possible relations between source- and target-level memory states.

% In CompCertO,
% these memory transformations are generalized and consolidated
% into a notion of \emph{CompCert Kripke Logical Relations} (CLKRs),
% which play an important role in defining simulation conventions.
% The underlying idea is that
% if two memory states are related by a CKLR,
% then memory operations which succeed at the source level
% should also succeed on at the target level,
% and their outcomes should in turn be related
% by the CKLR.

% Unfortunately,
% these memory transformations are difficult to use
% to express the relationships between
% different \emph{fragments} of a single memory state.
% The notion of \emph{separation relation} introduced below
% seeks to fill this gap.

\section{Certified abstraction layer}
\label{sec:bg:cal}

The Certified Abstraction Layer (CAL) framework
facilitates verification through
systematic decomposition into abstraction layers.
Originally developed with CompCertX semantics,
CAL has since been generalized to work with various semantic models.
Notable instantiations include
Interaction Specifications\citep{rbgs-cal},
free monad semantics\citep{thesis},
and object-based semantics with non-determinism\citep{popl22}.

At a conceptual level,
CAL comprises the following components:

\begin{itemize}
  \item
    \emph{Layer Interfaces $\mathcal{L}$}:
    A layer interface $L \in \mathcal{L}$
    formally specifies the behaviors
    of a collection of operations defined in a given signature.
    Each operation may interact with a designated abstract state.
    The abstraction of state allows for clear specification
    without revealing internal implementation details.

  \item
    \emph{Implementations $\mathcal{M}$}:
    Implementations $M \in \mathcal{M}$
    realize operations in an overlay layer interface
    by utilizing operations defined in
    an underlay interface.
    The implementations
    are associated with an operator
    $+ : \mathcal{M} \times \mathcal{M} \rightarrow \mathcal{M}$
    that groups two implementations together
    to form a larger implementation.

  \item
    \emph{Implementation Interpretation}:
    The operator $\llbracket\ \rrbracket : \mathcal{M} \times \mathcal{L} \rightarrow \mathcal{L}$
    interprets an implementation $M \in \mathcal{M}$
    given an underlay interface $L \in \mathcal{L}$.
    The underlay interface provides specifications
    for the uninterpreted calls in the implementation.
    The interpretation $\llbracket M \rrbracket L$
    is itself an interface
    that captures the behavior
    of operations in the implementation's signature.

  \item
    \emph{Refinement Ordering}:
    The partial order $\le \:\subseteq\: \mathcal{L} \times \mathcal{L}$
    between layer interfaces
    formally establishes the correctness criteria.
    Refinement ordering ensures that
    the behavior provided by one layer interface
    is fully captured by another,
    possibly at different abstraction levels.
    When states differ across abstraction layers,
    refinement incorporates explicit simulation relationships
    to manage state transformations coherently.

  \item
    \emph{Monotonicity of Interpretation}:
    An important property of the interpretation operator
    is that it is monotonic with respect to refinement ordering:
    \[
      \begin{prooftree}
        \hypo{L_1 \le_R L_2}
        \infer1{\llbracket M \rrbracket L_1 \le_R \llbracket M \rrbracket L_2}
      \end{prooftree}
    \]

\end{itemize}

These conceptual ingredients
enable CAL to form a generic and highly flexible framework.
The correctness of an individual
abstraction layer is formally expressed
by asserting that
an implementation $M$ correctly
realizes an overlay interface $L_2$
on top of an underlay interface $L_1$,
denoted as
\[
  L_1 \:\vdash_R\: M : L_2
  \quad
  :\Leftrightarrow
  \quad
  L_2 \le_R \llbracket M \rrbracket L_1
\]
Here $R$
is an abstract relation
that specifies how states are related
between the two interfaces.

The correctness of individual layers
can be composed to form the correctness
of the overall system
by applying the composition property:
\[
  \begin{prooftree}
    \hypo{L_1 \:\vdash_R\: M : L_2}
    \hypo{L_2 \:\vdash_S\: N : L_3}
    \infer2{L_1 \:\vdash_{R \cdot S}\: M \mathbin+ N : L_3}
  \end{prooftree}
\]

% The generic framework is formalized in Coq as described in
% Figure~\ref{fig:cal-module-type}.

% \begin{figure}
%   \centering
% \begin{minted}{coq}
% Module Type CAL.

%   Parameter Layer : Type -> Type -> Type.
%   Parameter Impl : Type -> Type -> Type.

%   Parameter impl_layer : forall {E F S}, Impl E F -> Layer E S -> Layer F S.
%   Parameter impl_compose : forall {E F G}, Impl E F -> Impl F G -> Impl E G.

%   Parameter ref : forall {E S1 S2} (R: S1 -> S2 -> Prop),
%       Layer E S1 -> Layer E S2 -> Prop.

%   Definition correct `(L1: Layer E1 S1) `(L2: Layer E2 S2) `(M: Impl E1 E2) :=
%       exists R, ref R L2 (impl_layer M L1).

% End CAL.
% \end{minted}
%   \caption{The CAL}
%   \label{fig:cal-module-type}
% \end{figure}

\subsection{CompCertX}

The CompCertX is a variant of CompCert
designed to support the CAL framework
that aims to allow CompCert programs
to play the role of layer implementations.

A layer interface $L$ is a tuple $\langle S, \sigma \rangle$,
where $S$ is the type of abstract states,
and $\sigma$ assigns each operation $f$ a primitive
$\sigma^f$:
\[
  \sigma^f \subseteq (\kw{val}^* \times \kw{mem} \times S)
  \times (\kw{val} \times \kw{mem} \times S)
\]
When $\sigma^f(\kw{args}, \kw{res}, \kw{state})$
holds,
it means that the primitive $\sigma^f$
takes the arguments $\kw{args}$,
the memory $\kw{mem}$,
and the state $\kw{state}$,
and returns the result $\kw{res}$
and the updated state $\kw{state'}$.

The ClightX language has identical syntax
as the Clight language.
The difference is that
the ClightX semantics
is parameterized over layer interfaces $\mathcal{L}$
that provides interpretations for
the calls to the primitives in the layer interface.
Given a layer interface $L := \langle S, \Sigma \rangle$,
The semantics of $\kw{ClightX}_L$
additionally takes the abstract state $S$,
and when it calls a primitive in $L$,
$\Sigma$ is used to interpret the call.
For the CAL interpretation,
the interpretation of an implementation $M$
with an underlay interface $L := \langle S, \Sigma \rangle$
is the layer interface defined as:
\[
  \llbracket M \rrbracket L \::=\: \langle S, \kw{ClightX}_L(M) \rangle
\]

Forward simulation serves
as the refinement order to establish correctness
of layer implementations.
Moreover,
the abstract states
are gradually realized in lower layers
as global variables stored in the memory state.
To support this,
the simulation relation is defined
by two components:
\[
  R^r \subseteq S_2 \times S_1 \qquad
  R^m \subseteq S_2 \times \kw{mem}
\]
The overlay abstract state $S_2$
preserves part of the underlay abstract states $S_1$
through the relation $R^r$,
while also introducing new abstract states
that are stored in the underlay as part of the memory state,
as described by $R^m$.
This design
allows the abstract state to be
progressively materialized across layers,
enabling a smooth transition
from high-level specifications
to low-level implementations.

I have neglected the details of
the compilation of ClightX programs,
and the composition of simulation relations
across layers.
Interested readers are referred to \citep{popl15}
for more details.

% \subsection{Free Monad and Monad Homomorphism}

% Another instance of the CAL is presented in \citep{thesis}
% by using free monads as the layer interface.

% This instance takes a more rigid type system
% where the layer interfaces
% are typed by an effect signature $E$,
% which is a set of primitive operations.
% Each element in the signature
% is a pair of primitive call $m$
% and its associcated return type $N$,
% written as $(m: N) \in E$.
% A layer interface is a tuple
% \[
%   L := \langle E, S, \sigma \rangle
% \]
% where $S$ is the type of abstract states,
% and $\sigma$ associates
% a fallible state monad to each operation $(m: N) \in E$
% as the interpretation of the primitive call.
% \[
%   \sigma^m : S \rightarrow (N \times S)_\bot \,.
% \]

% The layer implementation $M: E \rightarrow F$ associates a free monad
% over $E$ to each operation $(m: N) \in F$:
% \[
%   M^m : \mathcal{T}_E(N)
% \]

% The composition is defined in terms of interpreting the effects of
% free monad as described in Figure~\ref{fig:freer-monad}.

% Refinement

% The CAL instance using free monad is defined in Figure~\ref{fig:freer-monad}. It
% is worth mentioning that the conventional definition of free monad is not
% definable in Coq because of the strict positivity requirement\citep{one-monad}.
% The definition in our formalization is actually called the \textit{freer
% monad}\citep{freer-monad}.

% \begin{figure}
%   \centering
% \begin{minted}{coq}
%   Record esig := { op :> Type; ar : op -> Type; }.
%   Definition Spec (E: esig) (S: Type) := forall (n: E), S -> option (ar n * S).
%   Inductive Free {E: esig} {A: Type} :=
%     | Pure : A -> Free
%     | Bind (n: E) : (ar n -> Free) -> Free.
%   Definition Impl (E: esig) (F: esig) := forall (n: F), Free E (ar n).

%   Fixpoint exec' {E A S} (T: Free E A) (L: Spec E S) (s: S) : option (A * S) :=
%     match T with
%     | Pure a => Some (a, s)
%     | Bind e_op k => match L e_op s with
%                       | Some (a, s') => exec' (k a) L s'
%                       | None => None
%                       end
%     end.
%   Definition exec {E F S} (M: Impl E F) (L: Spec E S) : Spec F S :=
%     fun f_op s => exec' (M f_op) L s.

%   Fixpoint seq {E A B} (T1: Free E A) (T2: A -> Free E B) : Free E B :=
%     match T1 with
%     | Pure a => T2 a
%     | Bind e_op k => Bind e_op (fun a => seq (k a) T2)
%     end.

%   Fixpoint compose' {E F A} (T: Free F A) (M: Impl E F) : Free E A :=
%     match T with
%     | Pure a => Pure a
%     | Bind f_op k => seq (M f_op) (fun a => compose' (k a) M)
%     end.
%   Definition compose {E F G} (M1: Impl E F) (M2: Impl F G) : Impl E G :=
%     fun g_op => compose' (M2 g_op) M1.

%   Definition ref `(L1: Spec E S1) `(L2: Spec E S2) (R: S1 -> S2 -> Prop) :=
%     forall m n s1 s1' s2, L1 m s1 = Some (n, s1') -> R s1 s2 ->
%     exists s2', L2 m s2 = Some (n, s2') -> R s1' s2'.
% \end{minted}
%   \caption{The CAL instance using free monad}
%   \label{fig:freer-monad}
% \end{figure}

\section{CompCertO}
\label{sec:bg:compcerto}

Extending CompCert to support compositional verification
is challenging and has been an active research area.
CompCertO\citep{compcerto} addresses this challenge
through open semantics and simulation conventions,
which form the foundation for this thesis.

\subsection{Open Semantics}
\label{sec:bg:language-interfaces}

To model translation units and linking,
CompCertO must describe interactions across component boundaries—function calls and returns.
It uses \emph{language interfaces}
to type component boundaries.

\begin{definition}
  A \emph{language interface} $A = \langle A^\que, A^\ans \rangle$
  is a set of questions $A^\que$ and a set of answers $A^\ans$.
\end{definition}
% For example,
% the language interface for C is
% $\langle \kw{ident} \times \kw{val}^* \times \kw{mem}, \kw{val} \times \kw{mem} \rangle$.
% For example,
% assembly-level interactions are formulated in terms of
% low-level register state and code addresses.
I will write $\mathbf{0}$
for the empty language interface $\langle \varnothing, \varnothing \rangle$,
and $\mathbf{1}$
for the language interface
with only one possible question and answer,
$\langle \{*\}, \{*\} \rangle$.

Transition systems
are then assigned a type $A \twoheadrightarrow B$.
The following figure
shows the general shape of interaction of a component
with the environment.
The language interface $B$
is used between the component and its client.
The component
receives a question $q \in B^\que$
and is responsible for answering it
with an answer $r \in B^\ans$.
In the course of its execution,
it may perform outgoing calls
requiring answers from other components
using the language interface $A$.

\[
  \centering
  \begin{tikzpicture}[yscale=0.25,xscale=0.50]
    \draw (1,-1) rectangle (5,11) node[midway] {$L$};
    \scriptsize
    \draw[<-] (0,10) node[left] {$m_1 \in A^\que$} -- (1,10)
    node[above=1.5em,midway] (B) {\normalsize $A$};
    \draw[<-] (5,10) -- (6,10) node[right] {$q \in B^\que$}
    node[above=1.5em,midway] (A) {\normalsize $B$};
    \draw[->] (0,8) node[left] {$n_1 \in A^\ans$} -- (1,8) ;
    \node[right] at (-1,5.5) {$\:\vdots$};
    \draw[<-] (0,2) node[left] {$m_n \in A^\que$} -- (1,2);
    \draw[->] (6,0) node[right] {$r \in B^\ans$} -- (5,0);
    \draw[<-] (1,0) -- (0,0) node[left] {$n_n \in A^\ans$};
    \draw[<<-] (A) -- (B);
  \end{tikzpicture}
\]

\begin{definition}[Transition system]
  A CompCertO \emph{transition system} $L : A \twoheadrightarrow B$
  is a tuple
  \[
    L = \langle S, {\rightarrow}, I, X, Y, F \rangle \,.
  \]
  The relation ${\rightarrow} \subseteq S \times S$
  is a transition relation on the set of states $S$.
  The relation $I \subseteq B^\que \times S$
  assigns to each question of $B$
  a set of \emph{initial states}.
  The relation $F \subseteq S \times B^\ans$
  designates \emph{final states} which terminate the computation
  with a corresponding answer.
  External calls are specified
  by the $X \subseteq S \times A^\que$,
  which designates \emph{external states}
  together with a question of $A$,
  and $Y \subseteq S \times A^\ans \times S$,
  which selects a \emph{resumption state}
  based on the external call's answer.
\end{definition}

An execution can be understood as a sequence
that begins in an initial state,
evolves through internal transitions,
alternates with external calls and responses,
and eventually reaches a final state.
In particular, executions take the form
\[
  q \mathrel{I} s_0 \rightarrow^*
  s_1 \mathrel{X} m_1 \envstep
  n_1 \mathrel{Y^{s_1}} s_1' \rightarrow^*
  s_2 \mathrel{\cdots}
  s_n \mathrel{X} m_n \envstep
  n_n \mathrel{Y^{s_n}} s_n' \rightarrow^*
  s_f \mathrel{F} r
  \,.
\]
The component is activated by an incoming call,
described by a question $q \in B^\que$,
which is used to determine the transition system's initial state.
As it executes,
the transition system may perform outgoing calls,
asking questions
$m_1, \ldots, m_n \in A^\que$
and receiving corresponding answers
$n_1, \ldots, n_n \in A^\ans$.
Execution terminates with
the top-level answer $r \in B^\ans$.

The execution of a transition system
is represented as an interaction trace
\[
  t :=
  q \sysstep
  (m_1 \envstep n_1) \sysstep
  \cdots \sysstep
  (m_n \envstep n_n) \sysstep
  r
\]
when
the internal transition steps
are not important.
Here $\sysstep$ denotes internal execution
and $\envstep$ denotes environment-controlled steps.
I will write $L \vDash t$
to mean that the transition system $L$
admits the interaction trace $t$.

Under this definition,
the source and target semantics of CompCertO can be described as
\[
  \kw{Clight}(\kw{p.c}) : \mathcal{C}\at\kw{mem} \twoheadrightarrow \mathcal{C}\at\kw{mem}
  \qquad \text{and} \qquad
  \kw{Asm}(\kw{p.s}) : \mathcal{A}\at\kw{mem} \twoheadrightarrow \mathcal{A}\at\kw{mem} \,.
\]
The language interface
$\mathcal{C}\at\kw{mem} = \langle \mathcal{C}\at\kw{mem}^\que, \mathcal{C}\at\kw{mem}^\ans \rangle$
describes the kind of interactions used Example~\ref{ex:overview:clightsem}:
\begin{align*}
  \mathcal{C}\at\kw{mem}^\que & :=
  \{ f(\vec{v})@m \mid f \in \kw{ident}, \vec{v} \in \kw{val}^*, m \in \kw{mem} \}
  \\
  \mathcal{C}\at\kw{mem}^\ans & :=
  \{ v@m \mid v \in \kw{val}, m \in \kw{mem} \}
\end{align*}

\begin{example}[Clight semantics] \label{ex:overview:clightsem}
  Consider the translation unit $\kw{rb.c}$ shown in Fig.~\ref{fig:bq-code}.
  Its semantics is given by
  the transition system $\kw{Clight}(\kw{rb.c})$,
  which admits the following interaction trace:
  \[
    \kw{Clight}(\kw{rb.c}) \quad \vDash \quad
    \kw{inc1}()@[\kw{c_1} \mapsto 2]
    \: \rightarrowtail \:
    2@[\kw{c_1} \mapsto 3]
  \]
  Note that the memory is updated to store the new value of the counter $\kw{c1}$.
  By contrast, $\kw{bq.c}$
  does not directly modify the memory,
  but it makes outgoing calls which may have that effect:
  \[
    \kw{Clight}(\kw{bq.c}) \:\: \vDash \:\:
    \kw{deq}()@m
    \rightarrowtail
    \big( \kw{inc_1}()@m \leadsto i@m' \big)
    \rightarrowtail
    \big( \kw{get}(i)@m' \leadsto v@m'' \big)
    \rightarrowtail
    v@m''
    \,.
  \]
\end{example}

Note here I use round parentheses for
the \emph{open} transition system $\kw{Clight}(-)$
as opposed to the original closed semantics $\kw{Clight}[-]$.

\subsection{Simulations}

The types of
$\kw{Clight}(p) : \C \at \kw{mem} \twoheadrightarrow \C \at \kw{mem}$
and
$\kw{Asm}(p') : \A \at \kw{mem} \twoheadrightarrow \A \at \kw{mem}$
raise the question of the relationship
between source-level interactions
in $\C \at \kw{mem}$
and corresponding target-level interactions
in $\A \at \kw{mem}$.
Compositional compiler correctness only makes sense
with respect to a particular calling convention.
CompCertO makes this explicit:
simulations operate in the context of specified
\emph{simulation conventions},
which introduce a form of two-dimensional typing for simulations.

\begin{definition}[Simulation convention]
  \label{def:bg:sc}
  A \emph{simulation convention}\footnote{
    The original notation in \cite{compcerto} is $\mathbb{R} : A \Leftrightarrow B$.
  }
  between the language interfaces
  $A: \langle A^\que, A^\ans \rangle$
  and
  $B: \langle B^\que, B^\ans \rangle$
  is a tuple
  \[
    \mathbb{R} : A \twoheadleftrightarrow B
    := \langle W, \sysstep, \mathbb{R}^\que, \mathbb{R}^\ans \rangle
  \]
  where $W$ is a set of Kripke worlds
  together with an accessibility relation $\sysstep$
  that is reflexive and transitive.
  $\mathbb{R}^\que \subseteq \mathcal{R}_W (A^\que, B^\que)$
  and $\mathbb{R}^\ans \subseteq \mathcal{R}_W (A^\ans, B^\ans)$
  are relations indexed by the Kripke worlds.
  I will also use the following notation:
  \[
    w \Vdash q_a \mathbin{\mathbb{R}^\que} q_b
    \ :\Leftrightarrow\
    q_a \mathbin{\mathbb{R}_w^\que} q_b
    \quad \text{and} \quad
    w \Vdash r_a \mathbin{\mathbb{R}^\ans} r_b
    \ :\Leftrightarrow\
    r_a \mathbin{\mathbb{R}_w^\ans} r_b
  \]
\end{definition}

The Kripke worlds are used to enforce invariants
on the corresponding pairs of questions and answers.
This is made explicit by the following definition
of simulations.

\begin{definition}[Simulation]
  To establish a simulation
  of a transition system $L_1: A_1 \twoheadrightarrow B_1$
  by a transition system $L_2: A_2 \twoheadrightarrow B_2$,
  a simulation convention
  $\mathbb{R}_B : B_1 \twoheadleftrightarrow B_2$
  must be first specified
  for their incoming calls,
  and a simulation convention
  $\mathbb{R}_A : A_1 \twoheadleftrightarrow A_2$
  for their outgoing calls.
  A simulation between $L_1$ and $L_2$
  is witnessed by a simulation relation
  $R_{w \in W_B} \subseteq S_1 \times S_2$
  between the states of $L_1$ and $L_2$
  indexed by the Kripke worlds in $\mathbb{R}_B$
  that satisfies the simulation properties
  in Figure~\ref{fig:bg:simint}.

  The simulation is written as
  \[
    \phi : L_1 \le_{\mathbb{R}_A \twoheadrightarrow \mathbb{R}_B} L_2
    \qquad\qquad
    \begin{tikzcd}[row sep=2ex, column sep=2ex]
      A_1 \ar[rr, twoheadrightarrow, "L_1"]
      \ar[dd, leftrightarrow, "\mathbb{R}_A"'] &&
      B_1 \ar[dd, leftrightarrow, "\mathbb{R}_B"] \\
      & \phi & \\
      A_2 \ar[rr, twoheadrightarrow, "L_2"'] && B_2
    \end{tikzcd}
    \,.
  \]
\end{definition}

\begin{figure}
  \small
  \[
    \begin{array}{c@{\qquad}c@{\qquad}c}
      \begin{tikzcd}[row sep=3.5ex, column sep=3.5ex]
        q_1 \ar[dd, "w_B \Vdash \mathbb{R}_B^\que"', dash] \ar[rr, dash, "I_1"] &&
        s_1 \ar[dd, "w'_B \Vdash R", dash, dashed] \\
        && \\
        q_2 \ar[rr, "I_2"', dash, dashed] &&
        s_2
      \end{tikzcd}
      &
      \begin{tikzcd}[row sep=3.5ex, column sep=3.5ex]
        s_1 \ar[rr] \ar[dd, "w_B \Vdash R"', dash] &&
        \!\!{}_1 \:\, s_1' \ar[dd, "w_B' \Vdash R", dash, dashed] \\
        && \\
        s_2 \ar[rr, dashed] &&
        \!\!{}_2^* \:\, s_2'
      \end{tikzcd}
      %      \begin{tikzcd}[sep=large]
      %        s_1 \ar[r] \ar[d, "{(w_A, w_B) \Vdash R}"', dash] &
      %        s_1' \ar[d, "{(w_A,w_B) \Vdash R}", dash, dashed] \\
      %        s_2 \ar[r, dashed] &
      %        \!\!\!{}^* \: s_2'
      %      \end{tikzcd}
      &
      \begin{tikzcd}[row sep=3.5ex, column sep=3.5ex]
        s_1 \ar[rr, "F_1", dash] \ar[dd, "w_B \Vdash R"', dash] &&
        r_1 \ar[dd, "w_B' \Vdash \mathbb{R}_B^\ans", dash, dashed] \\
        && \\
        s_2 \ar[rr, "F_2"', dash, dashed] &&
        r_2
      \end{tikzcd}
      \vspace{1.2ex} \\
      \text{(a) Initial states} &
      \text{(b) Internal states} &
      \text{(c) Final states}
    \end{array}
  \]
  \[
    \begin{array}{c}
      \begin{tikzcd}[row sep=4.5ex, column sep=4.5ex]
        s_1 \ar[rr, "X_1", dash] \ar[dd, "w_B \Vdash R"', dash] &&
        m_1 \ar[rr, dotted, dash] \ar[dd, "w_A"', "{} \Vdash \mathbb{R}_A^\que", dash, dashed] &&
        n_1 \ar[rr, "Y_1^{s_1}", dash] \ar[dd, "w_A'"', "{} \Vdash \mathbb{R}_A^\ans", dash] &&
        s_1' \ar[dd, "w_B' \Vdash R", dash, dashed]
        \\
        &&
        \\
        s_2 \ar[rr, "X_2"', dash, dashed] &&
        m_2 \ar[rr, dotted, dash] &&
        n_2 \ar[rr, "Y_2^{s_2}"', dash, dashed] &&
        s_2'
      \end{tikzcd}
      \vspace{1ex} \\
      \text{(d) Outgoing calls}
    \end{array}
  \]
  \vspace{1.5ex}
  \[
    \begin{array}{c@{\qquad}l}
      \vspace{2.8ex}
      (a) &
      {
        \begin{aligned}
          \forall\; w_B, q_1, q_2, s_1 .\;
          & q_1 \xrightarrow{I_1} s_1
          \:\wedge\:
          w_B \Vdash q_1 \mathbin{\mathbb{R}_B^\que} q_2 \:\Rightarrow
          \\
          \exists\; w'_B, s_2.\;
          & q_2 \xrightarrow{I_2} s_2
          \:\wedge\:
          w_B \sysstep w'_B
          \:\wedge\:
          w'_B \Vdash s_1 \mathbin{R} s_2
        \end{aligned}
      }
      \\
      \vspace{2.8ex}
      (b) &
      {
        \begin{aligned}
          \forall\; w_B, s_1, s_2, s_1' .\;
          & s_1 \rightarrow_1 s_1'
          \:\wedge\:
          w_B \Vdash s_1 \mathbin{R} s_2 \:\Rightarrow \\
          \exists\; w'_B, s_2.\;
          & \bigl(s_2 \rightarrow^+_2 s_2'
            \:\vee\:
            s_2 \rightarrow^*_2 s_2'
            \:\wedge\:
            s_2' \le s_2
          \bigr)
          \:\wedge\:
          w'_B \Vdash s_1' \mathbin{R} s_2'
        \end{aligned}
      }
      \\
      \vspace{2.8ex}
      (c) &
      {
        \begin{aligned}
          \forall\; w_B, s_1, s_2, r_1 .\;
          & s_1 \xrightarrow{F_1} r_1
          \:\wedge\:
          w_B \Vdash s_1 \mathbin{R} s_2 \:\Rightarrow \\
          \exists\; w'_B, r_2.\;
          & r_1 \xrightarrow{F_2} r_2
          \:\wedge\:
          w_B \sysstep w'_B
          \:\wedge\:
          w'_B \Vdash r_1 \mathbin{\mathbb{R}_B^\ans} r_2
        \end{aligned}
      }
      \\
      (d) &
      {
        \begin{aligned}
          \forall\; w_B, s_1, s_2, m_1 .\;
          & s_1 \xrightarrow{X_1} m_1
          \:\wedge\:
          w_B \Vdash s_1 \mathbin{R} s_2
          \:\Rightarrow \\
          \exists\; w_A, m_2.\;
          & s_2 \xrightarrow{X_2} m_2
          \:\wedge\:
          w_A \Vdash m_1 \mathbin{\mathbb{R}_A^\que} m_2
          \:\wedge\:\\
          \forall\; w'_A&, n_1, n_2, s_1'.\;
          n_1 \xrightarrow{Y_1^{s_1}} s_1'
          \:\wedge\:
          w_A \sysstep w'_A
          \:\wedge\:
          w'_A \Vdash n_1 \mathbin{\mathbb{R}_A^\ans} n_2
          \:\Rightarrow \\
          & \exists\; w'_B, s'_2.\;
          n_2 \xrightarrow{Y_2^{s_2}} s_2'
          \:\wedge\:
          w_B \sysstep w'_B
          \:\wedge\:
          w'_B \Vdash s_1' \mathbin{R} s_2'
        \end{aligned}
      }
    \end{array}
  \]
  \caption{Simulation properties for internal steps (a,b,c)
  and outgoing calls (d).}
  \label{fig:bg:simint}
\end{figure}

The simulation properties together
guarantees that each interaction event
in the source transition system
is simulated by a corresponding interaction event
as prescribed by the simulation convention.
Specifically,
when the environment
invokes the transition systems in property (a),
it chooses world $w_B$
that relates the incoming questions.
Ultimately,
at the final state in property (c),
the answers must be related
at world $w_B'$
that is accessible from $w_B$.
Although intermediate worlds
are allowed for the internal steps,
the transitivity of $\sysstep$
guarantees that the final world $w_B'$
is accessible from the original world $w_B$.

Conversely,
when $L_1$ and $L_2$
perform outgoing calls,
the simulation must exhibit
a world $w_A$ where the questions are related.
It is then guaranteed by the environment
that the answers must be related
at world $w_A'$
that is accessible from $w_A$
as in property (c).
The alternating choice of worlds
is also reflected in the alternating usage
of the universal and existential quantifiers.

There is an identity simulation convention $\idsc_A : A \twoheadleftrightarrow A$
for every language interface $A$;
given $L_1, L_2 : A \twoheadrightarrow B$,
I will often write
a simulation of type $L_1 \le_{\idsc_A \twoheadrightarrow \idsc_B} L_2$
simply as $L_1 \le L_2$.
%
Compiler correctness
is expressed in terms of a convention
$\mathbb{C} : \C \at \kw{mem} \leftrightarrow \A \at \kw{mem}$
and can be stated as follows:
\[
  \kw{CompCert}(p) = p'
  \quad \Rightarrow \quad
  \phi^{cc} \::\:
  \kw{Clight}(p)
  \:\le_{\mathbb{C} \twoheadrightarrow \mathbb{C}}\:
  \kw{Asm}(p')
  \:.
\]

\begin{remark}
  The definition given here differs slightly
  from the original one in CompCertO.
  In the original setting,
  the Kripke worlds used to relate questions and answers
  are required to be identical,
  and world transitions are encoded implicitly
  within the relation $R^\que$ and $R^\ans$
  using modal Kripke relators.
  In contrast,
  I make these transitions explicit,
  since later chapters will generalize them further.
  For a more detailed discussion of
  how Kripke world transitions are handled in CompCertO, see \cite{compcerto21tr}.
\end{remark}

\subsection{Compositional Structure}
\label{sec:bg:compcert-compose}

Compared with the original CompCert,
CompCertO has a richer compositional structure.

\subsubsection{Vertical Composition}
% Figure~\ref{fig:compcerto}
% summarizes the compositional structure of the framework.

Just as in the original CompCert,
where simulations can be composed vertically
to combine the correctness of individual compilation phases
into an overall correctness theorem,
CompCertO supports the same principle.
To achieve this, simulation conventions must themselves compose.
Specifically, given simulation conventions
$\mathbb{R} : A \twoheadleftrightarrow B$ and
$\mathbb{R}' : B \twoheadleftrightarrow C$
they compose into
$\mathbb{R} \fatsemi \mathbb{R}' : A \twoheadleftrightarrow C$
\footnote{
  The original notation in \cite{compcerto} is $\mathbb{R} \cdot \mathbb{R}'$.
}
.

\begin{theorem}
  \label{thm:bg:sc}
  The composition $\fatsemi$ is associative
  and admits the identity morphism $\idsc_A : A \twoheadrightarrow A$.
  In other words,
  language interfaces together with simulation conventions
  form a category, denoted $\mathbf{SC}$.
\end{theorem}

This is used by the vertical composition principle \kw{sim}-$\vcomp$ for simulations,
which allows simulation squares to be composed vertically,
as illustrated by the following diagram:
\[
  \begin{prooftree}
    \hypo{\phi: L_1 \le_{\mathbb{R} \twoheadrightarrow \mathbb{S}} L_2}
    \hypo{\psi: L_2 \le_{\mathbb{R'} \twoheadrightarrow \mathbb{S'}} L_3}
    \infer2[\kw{sim}-$\fatsemi$]
    {\phi \fatsemi \psi: L_1 \le_{
        \mathbb{R} \fatsemi \mathbb{R'}
      \twoheadrightarrow \mathbb{S} \fatsemi \mathbb{S'}}
    L_3}
  \end{prooftree}
  \qquad
  \begin{array}{c}
    \begin{tikzcd}[row sep=1ex, column sep=2ex]
      A_1 \ar[rr, twoheadrightarrow, "L_1"]
      \ar[dd, leftrightarrow, "\mathbb{R}"]
      && B_1
      \ar[dd, leftrightarrow, "\mathbb{S}"]
      \\
      & \phi &
      \\
      A_2 \ar[rr, twoheadrightarrow, "L_2"]
      \ar[dd, leftrightarrow, "\mathbb{R'}"]
      && B_2
      \ar[dd, leftrightarrow, "\mathbb{S'}"]
      \\
      & \psi &
      \\
      A_3 \ar[rr, twoheadrightarrow, "L_3"] && B_3
    \end{tikzcd}
  \end{array}
\]

\subsubsection{Horizontal Composition}
\label{bg:compcerto:linking}
Moreover,
the \emph{semantic linking} operator $\oplus$
models the interaction between different program components.
\[
  \oplus_A : (A \twoheadrightarrow A) \times
  (A \twoheadrightarrow A) \rightarrow
  (A \twoheadrightarrow A)
\]

The transition system $L_1 \oplus L_2$
generally mirrors the execution of $L_1$ or $L_2$,
but when $L_1$ makes an external call
to a function provided by $L_2$ (and vice versa),
$L_1 \oplus L_2$ instantiates a new copy of $L_2$ to handle the call internally.
This copy executes until it reaches a final state,
at which point its outcome is used to resume
the suspended execution of $L_1$.
For example
$\kw{Clight}(\kw{rb.c}) \oplus \kw{Clight}(\kw{bq.c})$
admits the trace
\[
  \kw{deq}()@[\kw{c1} \mapsto 2, \kw{buf} \mapsto \{v_0, v_1, v_2, v_3\}]
  \quad\rightarrowtail\quad
  v_2@[\kw{c1} \mapsto 3, \kw{buf} \mapsto \{v_0, v_1, v_2, v_3\}]
  \,.
\]

Importantly,
the $\oplus$ operator has the following properties:
\[
  \begin{prooftree}
    \hypo{\phi : L_1 \le_{\mathbb{R} \twoheadrightarrow \mathbb{R}} L_2}
    \hypo{\psi : L'_1 \le_{\mathbb{R} \twoheadrightarrow \mathbb{R}} L'_2}
    \infer2[\kw{sim}-$\oplus$]
    {\phi \oplus \psi : L_1 \oplus L'_1 \le_{\mathbb{R} \twoheadrightarrow \mathbb{R}} L_2 \oplus L'_2}
  \end{prooftree}
  \quad
  \text{and}
  \quad
  \phi^{asm}: \kw{Asm}(p) \oplus \kw{Asm}(p') \le \kw{Asm}(p + p')
\]
where $p + p'$
represents the linked assembly program.
These properties
ensures the horizontal composition
of the source programs' behavior
is faithfully implemented
by the compiled and linked assembly program.
For example,
if $\kw{rb.c}$ and $\kw{bq.c}$
are compiled to $\kw{rb.s}$ and $\kw{bq.s}$,
respectively,
then the following simulation holds:
\[
  \bigl(\phi^{cc}(\kw{rb.c}) \oplus \phi^{cc}(\kw{bq.c})\bigr)
  \fatsemi \phi^{asm} \::\:
  \kw{Clight}(\kw{rb.c}) \oplus \kw{Clight}(\kw{bq.c})
  \le_{\mathbb{C} \twoheadrightarrow \mathbb{C}}
  \kw{Asm}(\kw{rb.s} + \kw{bq.s})
\]

Unfortunately,
simulations cannot generally be composed horizontally
using $\kw{sim}$-$\oplus$.
This limitation arises from semantic linking's symmetric nature,
which enables mutually recursive interaction between $L_1$ and $L_2$.
For $\oplus$ composition to be possible,
the transition systems must operate
over a single language interface,
and likewise simulations must operate
with respect to a single simulation convention
($\kw{sim}$-$\oplus$).

%To work around this restriction,
%CompCertO introduces a rich algebra of \emph{simulation convention refinements},
%which play the role of a second kind of two-dimensional object.
%These refinements can compose
%with simulations to modify their types,
%and are used to massage per-phase
%simulation proofs with varied conventions into
%an overall compiler correctness theorem
%which fits \kw{sim}-$\oplus$.

% \paragraph{Evaluation}

% Example~\ref{ex:compcerto}
% illustrates the flexibility of the CompCertO semantic model,
% but also some of its limitations.
% The language interface $\mathcal{C}_\kw{m}$
% forces the specifications
% to be formulated in terms of low-level memory states,
% and they remained tied to the particular concrete representation
% used by the code in Fig.~\ref{fig:code}.
% Moreover,
% the rigidity inherent $\oplus$ composition
% makes it difficult in general
% to handle situations which involve heterogeneous language interfaces.

\subsection{Simulation between C and Asm components}
\label{sec:bg:simulation-between-c}

The simulation convention $\mathbb{C}$
plays an important role
of connecting the C-level programs
with the assembly-level ones.
Consider a C-level program $\kw{a.c}$
that interacts with an assembly-level program $\kw{b.s}$.
To model this,
we introduce a C-level specification
$L_\kw{b}: \mathcal{C} \at \kw{mem} \twoheadrightarrow \mathcal{C} \at \kw{mem}$
which describes the behavior of $\kw{b.s}$
at the C-level.
The connection is established via the following simulation:
\[
  \phi_\kw{b} \::\: L_\kw{b} \le_{\mathbb{C} \twoheadrightarrow \mathbb{C}} \kw{Asm}(\kw{b.s})
\]
so that
if the combined system $\kw{Clight}(\kw{a.c}) \oplus L_\kw{b}$
satisfies the desired specification,
then correctness can be transferred to
the compiled and linked assembly program $\kw{Asm}(\kw{a.s} + \kw{b.s})$.

However,
the simulation convention $\mathbb{C}$ is
constructed compositionally,
using Kleene algebraic operators
(\cite[Section~5]{compcerto}).
In outline,
\[
  \mathbb{C} := \mathcal{R}^{*} \:\fatsemi \:\kw{wt} \:\fatsemi \:\kw{CA} \:\fatsemi \:\kw{vainj}
\]
Here,
\begin{itemize}
  \item
    $\mathcal{R} : \C \at \kw{mem} \twoheadleftrightarrow \C \at \kw{mem}$
    is a sum of refinement conventions
    $\mathcal{R} := \kw{injp} + \kw{inj} + \kw{ext} + \kw{vainj} + \kw{vaext}$
    is a sum of refinement conventions,
    where the sum operator allows the caller to choose,
    and the Kleene star enables repeated composition.
    Each of these conventions
    is an instance of
    CompCertO's Kripke logical relation(CKLR)
    which enjoys the following parametricity property.
    \begin{theorem}[Parametricity{\cite[Theorem~4.3]{compcerto}}]
      \label{thm:parametricity}

      For the languages
      $L \in \{\kw{Clight}, \kw{RTL}, \kw{Asm}\}$,
      \[
        \forall\ \mathbb{R} \in \kw{CKLR}.\ L(p) \le_{\mathbb{R} \rightarrow \mathbb{R}} L(p)
      \]
    \end{theorem}

  \item $\kw{wt} : \C \at \kw{mem} \twoheadrightarrow \C \at \kw{mem}$
    ensures the well-typedness of the arguments
    and the return values.
  \item $\kw{CA} : \C \at \kw{mem} \twoheadrightarrow \A \at \kw{mem}$
    formalizes the calling convention,
    mapping arguments and return values
    into assembly-level registers and stack.
  \item $\kw{vainj} : \A \at \kw{mem} \twoheadrightarrow \A \at \kw{mem}$
    injects memory states,
    ensuring compatibility between C and assembly-level views of memory.
\end{itemize}

The simulation $\phi_\kw{b}$
under $\mathbb{C}$
is quite complicated,
because of the intricate structure of the simulation convention.
However,
the proof can be significantly simplified
with the help of the parametricity theorem.

We first show
the goal can be achieved
with the following the proof obligations:
\begin{gather}
  L_\kw{b} \le_{\mathcal{R} \twoheadrightarrow \mathcal{R}} L_\kw{b}
  \label{eq:c-asm-1}\\
  L_\kw{b} \le_{\kw{wt} \twoheadrightarrow \kw{wt}} L_\kw{b}
  \label{eq:c-asm-2}\\
  L_\kw{b} \le_{\kw{injp} \cdot \kw{CA} \twoheadrightarrow \kw{inj} \cdot \kw{CA}} \kw{Asm(b.s)}
  \label{eq:c-asm-3}
\end{gather}
By applying {\cite[Theorem~5.6]{compcerto}}
to (\ref{eq:c-asm-1}), we get:
\begin{equation}
  \label{eq:c-asm-4}
  L_\kw{b} \le_{\mathcal{R}^* \twoheadrightarrow \mathcal{R}^*} L_\kw{b}
\end{equation}
At the same time,
an immediate instance of
the Theorem~\ref{thm:parametricity} is:
\begin{equation}
  \label{eq:c-asm-5}
  \kw{Asm}(\kw{b.s}) \le_{\kw{vainj} \twoheadrightarrow \kw{vainj}} \kw{Asm}(\kw{b.s})
\end{equation}
and by vertical composition
of (\ref{eq:c-asm-4}),
(\ref{eq:c-asm-2}), (\ref{eq:c-asm-3}),
and (\ref{eq:c-asm-5}), we get:
\begin{equation}
  L_\kw{b}
  \le_{\mathcal{R}^{*} \cdot \kw{wt} \cdot \kw{injp} \cdot \kw{CA} \cdot \kw{vainj}
    \twoheadrightarrow
  \mathcal{R}^{*} \cdot \kw{wt} \cdot \kw{inj} \cdot \kw{CA} \cdot \kw{vainj}} \kw{Asm(b.s)}
\end{equation}
By {\cite[Lemma~5.7, Lemma~5.8]{compcerto}},
we have the following refinement properties
between simulation conventions:
\[
  \kw{inj} \cdot \kw{CA}
  \sqsubseteq
  \kw{CA} \cdot\kw{inj}
  \qquad
  \kw{injp} \cdot \kw{wt}
  \sqsubseteq
  \kw{wt} \cdot\kw{injp}
\]
These commutations allow
the $\kw{injp}$ to be absorbed into $\mathcal{R}^*$,
and $\kw{inj}$ to be absorbed into $\kw{vainj}$,
thus completing the proof.

Although the proof obligations
(\ref{eq:c-asm-1})--(\ref{eq:c-asm-3})
remain to be manually verified,
they are quite straightforward.
Moreover,
later work \cite{compcerto-dr}
investigated the refinement conventions
in more details,
and managed to introduce
a direct simulation convention, $\kw{CA}_\kw{injp}$,
which is equivalent to $\mathbb{C}$
but easier to use.
This makes proving the simulations such as $\phi_\kw{b}$
considerably easier.
This approach further
paves the way toward
certified compilation of multi-threaded programs
\cite{compcertoc}.

\section{Game Semantics}

Game semantics\citep{cspgs}
models computation as an interactive process
between a component and its environment.
Unlike traditional operational semantics,
which focus on internal state transitions,
game semantics explicitly represents
interactive aspects of execution,
making it well-suited for compositional verification
of multiple interacting components.

Central to this approach is the notion of a \emph{game},
defined as a structured interaction
between two distinct players.
The proponent represents the component or system
under consideration,
while the opponent models the environment
or external context interacting with the proponent.
Interactions between these two players
occur as moves that can be categorized as
questions and answers.
A sequence of these moves constitutes a \emph{play},
capturing a particular execution scenario.
An example of a play have the following form:
\[
  \begin{tikzpicture}[baseline=(current bounding box.center)]
    \node (q) at (0,0) {$\kw{run}$};
    \node at (0.5,0) {$\cdot$};
    \node (m1) at (1.25,0) {$\underline{\kw{read}()}$};
    \node at (2,0) {$\cdot$};
    \node (n1) at (2.25,0) {$n$};
    \node at (2.5,0) {$\cdot$};
    \node (m2) at (3.75,0) {$\underline{\kw{write}(n + 1)}$};
    \node at (5,0) {$\cdot$};
    \node (n2) at (5.5,0) {$\kw{ok}$};
    \node at (6,0) {$\cdot$};
    \node (r) at (6.75,0) {$\underline{\kw{done}}$};

    \draw[{Latex}-, shorten <=2pt, shorten >=2pt] (q.north).. controls +(0.5cm, 0.7cm) and +(-0.5cm, 0.7cm) .. (r.north);
    \draw[{Latex}-, bend left=30, shorten <=2pt, shorten >=2pt] (m1.north) to (n1.north);
    \draw[{Latex}-, bend left=30, shorten <=2pt, shorten >=2pt] (m2.north) to (n2.north);
  \end{tikzpicture}
\]
The underlined moves are those of the system.
The arrows come from the answer moves
to their corresponding question moves.
This play is an example execution of a program
that reads an integer value from the environment,
increments it,
and writes the result back to the environment.

The permissible behaviors of a component
are characterized by a \emph{strategy},
which is a set of plays
that describes how the component
may react to the environment's moves.
Typically,
plays are ordered by the prefix relation.
An example strategy can be constructed
from the above play
by taking the prefix-closed set of plays:
\[
  \sigma_n \::=\: \big\downarrow (\kw{run} \cdot \underline{\kw{read}()} \cdot n \cdot \underline{\kw{write}(n + 1)} \cdot \kw{ok} \cdot \underline{\kw{done}})
\]
However,
such a strategy
only allows the environment to
respond with the answer $n$.
A specification of the program's behavior
is obtained by taking the union over all possible answers:
\[
  \sigma \::=\: \bigcup_{n \in \mathbb{N}} \sigma_n
\]

\subsection{Relation to CompCertO}

CompCertO's open semantics relates closely to game semantics\citep{compcerto21tr}.
A CompCertO execution trace
$q \sysstep (m_1 \envstep n_1) \sysstep (m_2 \envstep n_2) \sysstep r$
corresponds to the following play:
\[
  q
  \cdot
  \underline{m_1} \cdot n_1 \cdot \underline{m_2} \cdot n_2
  \cdot
  \underline{r}
\]
However,
it remains unclear
how the simulation conventions and simulations
can be formulated in terms of game semantics.
In this work,
I will develop a semantic model
based on strategies
and formally reveal the connection
between the two semantic models.

\section{Category Theory}

Category theory provides
a mathematical language for describing
composition and abstraction.
It captures common structural patterns
across different semantic models,
enabling abstract yet principled reasoning.

Category theory is particularly relevant to this thesis
because the three-dimensional algebraic structure
developed here—capturing horizontal, vertical, and spatial composition—aligns naturally with categorical concepts
such as morphisms, functors, and double categories.
By phrasing our framework in categorical terms,
we obtain both clarity of expression
and a set of well-studied algebraic laws
that guide the design of compositional reasoning principles.

\begin{table}
  \centering
  \begin{tabular}{llll}
    \toprule
    Category & Object & Morphism & See also\\
    \midrule
    $\mathbf{TS}$ & Language Interfaces & Transition Systems & Thm.~\ref{thm:ox:ts}\\
    $\mathbf{SC}$ & Language Interfaces & Simulation Conventions & Thm.~\ref{thm:bg:sc}\\
    $\mathbf{Lens}$ & Sets & Lens & Thm.~\ref{ox:def:lens-comp}\\
    $\mathbf{ETS}$ & Language Interfaces & Encapsulated Transition Systems & Thm.~\ref{ox:def:ets-comp}\\
    $\mathbf{SSC}$ & Language Interfaces & Stateful Simulation Conventions & Thm.~\ref{ox:def:ssc-comp}\\
    $\mathbf{Strat}$ & Effect Signatures & Strategies & Thm.~\ref{thm:strat:comp}\\
    \bottomrule
  \end{tabular}
  \caption{A list of categories introduced in this work}
  \label{bg:tab:cat}
\end{table}

The following section provides
a concise overview of several key concepts
that will be used throughout this work,
serving mainly as background for later developments.
For a more comprehensive introduction to category theory,
readers are encouraged to consult standard references
such as \cite{awodeyct}.

\subsection{Category and Double Category}

At its core,
a \emph{category} consists of:
\begin{itemize}
  \item Objects, which represents entities of interest
    such as sets, types, or language interfaces,
  \item Morphisms, which represent structure-preserving transformations
    between objects such as functions, or transition systems,
  \item A notion of composition of morphisms that is associative, and
  \item An identity morphism for each object that serves as a neutral element for composition.
\end{itemize}

This basic structure is enough to capture
many of the compositional operators we use
in verification frameworks.
For instance,
simulation conventions between language interfaces
form a category under composition.
A list of categories
that are introduced in this work
is given in \autoref{bg:tab:cat}.

A double category enriches this picture
by supporting
two different types of morphisms simultaneously
(horizontal and vertical),
together with squares that relate them.
This mirrors precisely the setting of our work:
horizontal morphisms represent program components,
vertical morphisms capture refinement relations,
and squares witness simulations between them.
The algebra of companions and conjoints
then expresses how horizontal and vertical dimensions interact
in a disciplined way.
This higher-dimensional perspective is critical
to making our compositional reasoning
both modular and systematic.

% double category $\mathbf{TSC}$ in \autoref{thm:ox:tsc}
% category $\mathbf{TS}$ in \autoref{thm:ox:ts}.

\subsection{Functor and Bifunctor}

A functor is a structure-preserving mapping
between categories:
it maps objects to objects and morphisms to morphisms
in a way that respects composition and identities.
Functors allow us to relate different semantic models systematically.
For example,
the embeddings I will present---%
such as the embedding of CompCertO's open transition systems
into the strategy model---%
are functorial in nature.
That is,
they translate not only the objects (language interfaces)
but also the morphisms (transition systems, simulation conventions)
with their relations (simulation squares)
in a way that preserves their algebraic structure.
This functorial perspective ensures that
results proved in one model can be
faithfully transported into another without loss of meaning.

A bifunctor generalizes the notion of a functor
to mappings that take two arguments,
such as tensor-like operators
that combine components simultaneously.
The spatial composition operator naturally
fits into this perspective.
More generally,
spatial composition combines semantic entities---%
transition systems, simulation conventions,
or simulation squares---%
with an auxiliary structure that evolves alongside them.
Crucially, this
auxiliary structure (such as external state)
can itself be organized into a category,
with states as objects
and their admissible transitions as morphisms.
When such a categorical view of the auxiliary structure
is paired with our semantic models,
the result is captured precisely by a bifunctor.
This framing makes explicit
how spatial composition uniformly lifts
both semantics and auxiliary structures
across the framework,
while preserving their compositional laws.


\documentclass[acmsmall,anonymous,review]{acmart}
\setcopyright{none}
%\usepackage{geometry}
%\usepackage[colorlinks]{hyperref}
\usepackage{ebproof}
%\usepackage{amsmath}
%\usepackage{amsthm}
%\usepackage{amssymb}
%\usepackage{libertine}
\usepackage{stmaryrd}
\usepackage{tikz}
\usepackage{tikz-cd}
\usetikzlibrary{calc}

%\newtheorem{definition}{Definition}
%\newtheorem{theorem}[definition]{Theorem}
\newtheorem{remark}{Remark}

\newcommand{\kw}[1]{\texttt{#1}}

\title{Mechanized Game Semantics for Compositional Verification}

\begin{abstract} %{{{
Recent work proposes to use game semantics for
refinement-based verification
of complex systems.
However,
mechanizing game semantics in a proof assistant presents many challenges.
%in part due to the complexity of most game models,
%and in part because the mathematics used in traditional descriptions of games
%are ill-suited to a type-theoretic formalization.
We show that the \emph{template games} of P.-A. Melli\`es,
and a recently proposed \emph{algebraic} formulation of game semantics
provide a viable approach.
We mechanize a simple game model in the Coq proof assistant,
enriched with a dually nondeterministic refinement ordering.
The result allows us to extend existing verification frameworks
in various directions.

First,
we use the model to provide a coarse-grained denotational semantics
of CompCert program components,
and use it to characterize and soundly implement
the traditional C control operators \texttt{setjmp} and \texttt{longjmp}.
We show that games makes it possible to
model aspects of the execution environments of programs
and demonstrate the verification of a system involving
two separate processes communicating through a pipeline.

Then,
we develop a theory of certified abstraction layers
within our new model.
By decoupling certified abstraction layers
from their low-level CompCert semantics,
and by imposing a stronger typing discipline
on layer interfaces and implementations,
we address long-standing limitations on their compositionality.
In particular,
we verify a page allocator
which can be used by an arbitrary number of
independent clients.
\end{abstract}
%}}}

\begin{document}

\maketitle

\section*{Plan} %{{{

The idea is to:
\begin{itemize}
  \item RBGS angle:
    establish its viability and gain experience
    with a specialized version
  \item CAL angle:
    wrap up CompCertOX with the right tools,
    explore extensions
\end{itemize}
If the scope ends up being too much,
there's several ways in which we could shrink it:
\begin{itemize}
  \item Drop one or both of the CompCert extensions
    (context switching, process environment)
  \item Have more basic goals for the CAL framework
\end{itemize}
But if everything goes well the timeline could be something like:
\begin{description}
  \item[By June 15 (Week 1)] Set down the basic framework
    \begin{itemize}
      \item Draft the key definitions and results we want in each section
      \item Pin down the basics of the game model + its interface
      \item Clean double category structure for CompCert LTS
    \end{itemize}
  \item[By June 22 (Week 2)] Bulk of the work
    \begin{itemize}
      \item Core maths written out in paper, Coq implementation well underway
      \item Good plan for the applications we want to demonstrate
      \item Paper framing settled, intro written
    \end{itemize}
  \item[By June 29 (Week 3)] Make it good
    \begin{itemize}
      \item Paper in submittable form, incl. related work
      \item Wrap up implementation as much as possible
    \end{itemize}
\end{description}
Generally speaking,
we could split up the work as follows:
\begin{itemize}
  \item J\'er\'emie would work on the RBGS and help more with writing
  \item Yu would focus on the CAL component and help more with
    Coq proofs.
\end{itemize}

%}}}

\section{Introduction}

Usual RBGS pitch.
Issues with certified abstraction layers:
depends on abstract state and simulation relations,
horizontal compositionality.

\section{Main Ideas} %{{{

\subsection{Game Semantics} %{{{

Basic ideas,

Game semantics view of effect signatures.

Strategies between effect signatures as 4-moves games.

%}}}

\subsection{Algebraic Effects} %{{{

%The new model we propose in \S\ref{sec:model}
%draws heavily from the work on algebraic effects.
Algebraic effects refine
the traditional monadic approach to computational effects.
In the traditional approach,
the available effects and their properties
are captured by a monad $\langle T, \eta, \mu \rangle$.
Computations with a result in the set $X$
are interpreted in $TX$.
The monad's unit $\eta_X : X \rightarrow TX$
embeds values as pure computations.
The multiplication $\mu_X : TTX \rightarrow X$
allows the computations $f : X \rightarrow TY$ and $g : Y \rightarrow TZ$
to be sequentially composed as
$\mu_Z \circ Tg \circ f : X \rightarrow TZ$.
The algebraic approach
restricts this framework to
monads generated by algebraic theories.
$TX$ consists of terms over the theory's signature
with variables in $X$,
considered up the theory's equations.

In a term representing a computation,
function symbols represent effects,
and their arguments represent the possible continuations
for the computation,
chosen or combined depending on the effect's outcome.
The computation proceeds inward,
with the environment resolving the choice of argument at each step.
The \emph{recursive} aspect of the term construction
allow computations to \emph{sequentially} trigger
an arbitrary number of effects
taken from the theory's signature.
Variables from the set $X$
denote termination with the corresponding result.
Equations characterize the behavior of effects
and allow us to reason about computations
using standard algebraic techniques.

Most effects used by programmers
are algebraic in the sense outlined above,
so the restriction to algebraic effects is not very onerous.
On the other hand,
algebraic effects have a key advantage:
unlike arbitrary monads,
algebraic theories are easy to compose.
For example,
the sum of two theories
incorporates the operations and equations from either one of them.
Their tensor product additionally
adds equations allowing arbitrary operations from one theory
to commute with arbitrary operations from the other.

Futhermore,
the algebraic approach allows the formulation of
generic effect \emph{handlers},
which generalize exception handlers
and transform computations by interpreting the effects of one theory
into another.
Writing $U^*$ for the monad associated with the theory $U$,
a handler defines a mapping of type $U^* X \rightarrow V^* Y$.
Because the term $u \in U^* X$
expresses all possible evolutions of the original computation,
handlers provide a very general setting
and in particular can express control operators.

%}}}

\subsection{Effect Signatures} %{{{

In this work,
we forego the use of algebraic theories
and instead restrict work with the simple notion of
effect signature defined below.

\begin{definition}{Effect signature}
An \emph{effect signature} is a set $E^\circ$ of operations
together with a family of sets $\big( E^\bullet_m \big)_{m \in E^\circ}$
specifying an \emph{arity} for each operation.
\end{definition}

[Comment on infinite arities]

Signatures offer less flexibility than full-blown theories,
but they are easier to manipulate.
They can be interpreted as simple games,
where operations correspond to one player's moves,
and arities correspond to the other's.
As we will see,
the restriction to signatures also makes possible
a finer-grained control of the shape of computations.

\paragraph{Term Constructions} %{{{

Given a signature $E$ and a set of variables $X$,
we can define the set of terms of depth one,
representing a computation which terminates after
triggering exactly one effect:
\[
  EX := \sum_{m \in E^\circ} \prod_{n \in E^\bullet_m} X
\]
For legibility,
we will write $\underline{m} \langle x_n \rangle_{n \in E^\bullet_m}$
for the term $\iota_m(n \mapsto x_n)$.

To represent computations with an arbitrary number of effects in $E$,
we can construct the free monad $E^*$,
which gives us the set of terms of arbitrary depth.
We can define it using the grammar:
\[
  t \in E^*X ::= \underline{m} ( t_1, \ldots, t_n ) \mid \underline{x}
  \qquad
  (m \in E^\circ, \: n \in E^\bullet_m, \: x \in X)
\]
Our key insight is that this construction and many others
can be internalized at the level of the signatures themselves:
given a signature $E$,
we can define a signature $\dagger E$
such that $\dagger E \, X = E^*X$.

This allows for a finer-grained control of the shape of computations,
by working at the level of the \emph{endofunctor} $E$,
and using the iteration construction $\dagger E$ only as needed.
Importantly,
this allows us to recover much compositional structure
that would normally be available only with the use of equational theories.
For example,
while the tensor product of algebraic theories
requires the addition of commutations equations
for $(U \otimes V)^*$ to be the right model,
a similar effect can be achieved with signatures alone
using a construction $\dagger E \otimes \dagger F$.

%}}}

\paragraph{Products of Signatures} %{{{

[Give a game interpretation for each one]

\[
  \begin{array}{l@{\:}l@{\qquad}l@{\:}l@{\qquad}l@{\:}l@{\qquad}l@{\:}l}
  (E_1 + E_2)^\circ &:=
    E_1^\circ + E_2^\circ &
  (E_1 + E_2)^\bullet_{\iota_i(m)} &:=
    E_{i,m}^\bullet &
  0^\circ &:= \varnothing
  \\
  (E_1 \times E_2)^\circ &:=
    E_1^\circ \times E_2^\circ &
  (E_1 \times E_2)^\bullet_{(m_1, m_2)} &:=
    E_{1,m_1}^\bullet + E_{2,m_2}^\bullet &
  1^\circ &:= \{*\} &
    1^\bullet_* &:= \varnothing
  \\
  (E_1 \otimes E_2)^\circ &:=
    E_1^\circ \times E_2^\circ &
  (E_1 \otimes E_2)^\bullet_{(m_1, m_2)} &:=
    E_{1,m_1}^\bullet \times E_{2,m_2}^\bullet &
  I^\circ &:= \{*\} &
    I^\bullet_* &:= \{*\}
  \end{array}
\]

%}}}

\paragraph{Signature Homomorphisms} %{{{

In our setting,
they will be \emph{signature homomorphisms},
which transform
terms of depth one over a signature $E$
into terms of depth one over a signature $F$.
This requires specifying
for each operation $m \in E$
an operation of $q \in F$,
and then for each argument position $j \in \mathsf{ar}(q)$
a corresponding position $i \in \mathsf{ar}(m)$.
\[
  E \rightarrow F \: := \:
  \prod_{m \in E} \sum_{q \in F} \,
    \big(\mathsf{ar}(q) \rightarrow \mathsf{ar}(m)\big)
  \: = \:
  \prod_{m \in E} \, F \big( \mathsf{ar}(m) \big)
\]
Note that signature homomorphisms
are in one-to-one correspondence with
natural transformations between the corresponding endofunctors:
\[
  E \rightarrow F \: = \:
  \forall X \cdot EX \rightarrow FX
\]
This corresponds to a uniform effect handler
transforming computations with an effect in $E$
into computations with an effect in $F$.

Note that a signature homomorphism of type
$I \rightarrow E$
corresponds to a choice of operation,
whereas a signature homomorphism of type
$E \rightarrow I$
corresponds to a choice of argument
for each possible operation.
In other words,
they encode costrategies and strategies
for the game associated with $E$.

%}}}

%}}}

\subsection{Object-based Semantics} %{{{

Object-based semantics are a precursor of game semantics.
The core idea is to characterize
the externally observable behavior of an object
without reference to its internal state.
Object-based semantics were originally formulated
in the coherence space model of linear logic;
like game semantics,
this gives them a flavor similar to trace models of process calculi,
but with strong typing and a strong polarization between
the system being modeled and its environment.

In this subsection,
we outline the high-level structures
involved in object-based semantics and
how they are realized in our own model.
Effect signatures take the place of coherence spaces,
and signature homomorphisms are used to define
the maps between them.

\paragraph{Linear Maps} %{{{

The starting point of Reddy's model
is the notion of linear map,
used to interpret linear logic in coherence spaces.
Roughly speaking,
in that context
linear maps are relations between coherence spaces which preserve
the respective role of the system and the environment.

There is one key difference between algebraic effects and
object-based semantics which we must account for.
Whereas algebraic effects model \emph{active} computations
which may trigger effects,
object-based and game semantics
model objects which must first be invoked
by an incoming query.
As a result,
the role of system and the environment
will be reversed between these two settings,
and we define our notion of linear map $E \multimap F$
as a signature homomorphism of type $F \rightarrow E$:
\[
  E \multimap F := \forall X \cdot FX \rightarrow EX
\]
A linear map of this type
describes a component which
\emph{uses} an interface $E$ to
\emph{provide} an interface $F$.

%}}}

\paragraph{Products} %{{{

Because of the opposite directions of
linear maps compared with signature homomorphisms,
the product and coproduct are switched.
To avoid confusion
and underscore their correspondence with
the similar constructions in Reddy's model,
we will use linear logic notations
when we use them in the context of object-based semantics:
\[
  E \oplus F := E \times F
  \qquad
  E \mathbin{\&} F := E + F
\]

%}}}

\paragraph{Iteration} %{{{

Objects are capable of handling
\emph{sequences} of successive invocations.
In fact,
a defining property of objects
is their ability to retain state from one invocation to the next.
Therefore,
in object-based semantics,
the behavior of objects can be characterized
as maps of type ${\dagger}E \multimap {\dagger}F$.

Nevertheless,
the ability to use $\dagger$ on demand
extends the expressivity of the model.
In particular,
properties of the free monad can be made explicit.
For example,
the behavior of \emph{stateless} objects
can be characterized
using maps of type ${\dagger} E \multimap F$.
The free monad's Kleisli extension
can then be used to recover a map of type
${\dagger} E \multimap {\dagger} F$.

%}}}

\paragraph{Horizontal Composition} %{{{

Although in the context of algebraic effects,
the usual formulation of the tensor product
requires the use of equational theories,
under an object semantics approach
it can be defined on mere signatures
in the way outlined above.
This allows us for example to compose two object semantics
$f_1 : {\dagger} E_1 \multimap {\dagger}F_1$ and
$f_2 : {\dagger}E_2 \multimap {\dagger}F_2$
to obtain:
\[
  f_1 \otimes f_2 : {\dagger}E_1 \otimes {\dagger}E_2
  \multimap {\dagger}F_1 \otimes {\dagger}F_2
\]
The result does not itself provide the interface of an object,
since its interface is not of the form $\dagger{-}$.
In particular,
the interface ${\dagger}F_1 \otimes {\dagger}F_2$
does not carry any information about
the interleaving between the invocations of $F_1$ and $F_2$,
underlining the composite nature of the resulting map,
where the state of the two objects are independent.

We can however handle queries into the interface
$\dagger(F_1 \mathbin{\&} F_2)$
by precomposing the result with the \emph{serialization} map:
\[
    \mathrm{ser}_{F_1,F_2} : F_1 \otimes F_2 \multimap {\dagger}(F_1 \mathbin{\&} F_2)
\]
On the left hand side,
\ldots

%}}}

\subsection{CompCertO} %{{{

[Brief explanation of the model and how it can embed.
Focus in particular on simulation conventions and explain
order enrichment in our model,
string diagrams]

Language interface $A = \langle A^\circ, A^\bullet \rangle$
can be embedded as the signature
\[
  \llbracket A \rrbracket := \{ m \mathbin: A^\bullet \mid m \in A^\circ \}
\]
Open transition system $L : A \twoheadrightarrow B$
can be embedded as
\[
  \llbracket L \rrbracket :
  \dagger \llbracket A \rrbracket \multimap \llbracket B \rrbracket
\]
This can be promoted to the regular map:
\[
  \llbracket L \rrbracket^* :
  \dagger \llbracket A \rrbracket \multimap \dagger \llbracket B \rrbracket
\]
Then composition embeds as expected.

Remains to see how simulation conventions can be accounted for.
Explain the difficulty
with mixing choices from the system and environment.

%}}}

\subsection{Certified Abstraction Layers} %{{{

Sum up the model per LICS'20 paper.

Layer interface $L$ for a signature $E$ can be embedded as:
\[
  \llbracket L \rrbracket : I \multimap \dagger E
\]
using a replay function of sorts and an initial state.
Layer implementation is $\llbracket M \rrbracket : \dagger E \multimap F$,
can be promoted to $\llbracket M \rrbracket^* : \dagger E \multimap \dagger F$.
Then for an appropriate notion of refinement $\sqsubseteq$ (see later section),
layer correctness can be formulated as:
\[
  \llbracket L' \rrbracket \sqsubseteq
  \llbracket M \rrbracket^* \circ \llbracket L \rrbracket
\]
Important note:
because we have hidden the state,
the simulation relation no longer appears as an explicit
parameter in correctness.

%}}}

%}}}

\section{Mechanized Game Semantics} %{{{

\subsection{Overview} %{{{

Features of the model:
\begin{itemize}
  \item Effect signatures as objects;
    add possible dagger but that could just be a flag
  \item Two-sided strategies as morphisms
  \item Dual nondeterminism
    (a deterministic subcategory can also be defined)
  \item Sum of signatures as tensor product
  \item Dagger comonad
\end{itemize}
Things we don't need to support for now:
\begin{itemize}
  \item Reentrancy
  \item More sophisticated games
\end{itemize}

%}}}

\subsection{Delineating the model} %{{{

The objects of the category must contain at least:
\[
  A ::= E \mid A \rhd B \mid {\dagger}A
\]
where $E$ is an effect signature.
There are two options:
\begin{enumerate}
  \item Maybe the simplest way would be to use the grammar above
    as the definition of objects.
  \item Effect signatures come with their own $\rhd$ and $\dagger$,
    but it's not clear it can be used as-is
\end{enumerate}
Once signatures have been defined,
it is useful to associate endofunctor to them
along the lines:
\begin{align*}
  [E] \:&=\: X \mapsto \sum_{m \in E} \prod_{n \in \mathrm{ar}(m)} X
  \\
  [A \rhd B] \:&=\: [A] \circ [B]
  \\
  [{\dagger}A] \:&=\: [A]^* \:=\: X \mapsto \mu Y \cdot [A] Y + X
\end{align*}
With this in mind we will expect the homomorphisms
to define natural transformations:
\[
  [A \multimap B] : 
  \forall X \cdot B X \rightarrow A X
\]
For the case $E \multimap F$ of signatures,
this is the same as specifying
a family $(e^m)_{m \in F}$ with $e^m \in E \big( \mathrm{ar}(m) \big)$,
in other word mapping each operation $m \in F$
to an operation $q \in E$ and
each outcome $r \in \mathrm{ar}(q)$
to an outcome $n \in \mathrm{ar}(m)$.

We could call those the \emph{linear} maps of our category,
with the expectation that a \emph{regular} map
$f^\dagger : {\dagger} A \multimap {\dagger} B$
would in addition be a \emph{monad homomorphism}
between the monads $[B]^*$ and $[A]^*$
and could always be defined in terms of
$f : {\dagger}A \multimap B$.

That's the basic story for simple algebraic games.
Then we can use my ACT 2021 techniques and
work over $\mathbf{DCPO}_{\bot!}$ or $\mathbf{CDLat}$
by using completions of term models,
and by tweaking the endofunctors
to give the result a game semantics flavor
rather than a coherence space flavor.

%}}}

\subsection{Double category} %{{{

To prepare the way for embedding simulation conventions,
we can define a double category based on the model.
The objects and horizontal morphisms
are the same as before.
The vertical morphisms of type $A \leftrightarrows B$
are \emph{adjunctions} $f^* \dashv f_*$
where $f^* : A \rightarrow B$ and $f_* : B \rightarrow A$
such that:
\[
  \mathrm{id}_A \le f_* \circ f^*
  \qquad \qquad
  f^* \circ f_* \le \mathrm{id}_B
\]
The 2-cell $x \le_{f \rightarrow g} y$
exists for
$x : A_1 \rightarrow B_1$,
$y : A_2 \rightarrow B_2$,
$f^* \dashv f_* : A_1 \rightarrow A_2$ and
$g^* \dashv g_* : B_1 \rightarrow B_2$
when one of the following equivalent conditions hold:
\begin{equation} \label{eqn:gc}
  x \le_{f \rightarrow g} y
  \quad :\Leftrightarrow \quad
  g^* \circ x \le y \circ f^*
  \quad \Leftrightarrow \quad
  x \circ f_* \le g_* \circ y
\end{equation}

This definitions makes sense in any order-enriched category,
and is a special case of a general construction on 2-categories
(see the nlab entries \emph{double category} and \emph{mate}).
Reading it in the context of our game model,
the lower adjoints $f^*$ and $g^*$
translate from their incoming low-level interactions
to their outgoing high-level interactions,
and the upper adjoints $f_*$ and $g_*$
translate the other way around.

Complete homomorphisms have both upper and lower adjoints,
In completely distributive lattices
they are themselves complete?
which makes it easy to define vertical morphisms
from relations (as spans) etc.

Could be depicted using elbow string diagrams.

%}}}

%}}}

\section{CompCert Programs}

In this section we would show how to embed CompCert programs into the model:
\begin{itemize}
  \item The types would be similar to the coherence space version
  \item For defining the embedding,
    high-level constructions instead of low-level traces
    could be more convenient but we can choose.
\end{itemize}
Things that would be new compared to what we've done so far:
\begin{itemize}
  \item With a clean double category for LTS,
    the simulation convention refinement
    $\mathbb{R} \sqsubseteq \mathbb{R}'$
    can be encoded as a simulation
    $\mathsf{id} \le_{\mathbb{R} \twoheadrightarrow \mathbb{R}'} \mathsf{id}$.
  \item We would use dual nondeterminism to embed simulation conventions.
\end{itemize}
Things we can forget about:
\begin{itemize}
  \item Don't distinguish crash vs. silent divergence
  \item Only layered/categorical structure,
    don't worry too much about mutual recursion
  \item This means at first we could just forget about component domains.
    Ultimately we might need to figure out some aspects
    to establish linking soundness,
    and definitely if we want to handle upcalls,
    but I think this is a good candidate for
    ``work around it for now''
\end{itemize}

\subsection{Open Transition Systems}

[Maybe it's appropriate to explain the definition of transition systems in section 2]

The labeled transition system in CompCertO uses a notion of domain to specify a
set of questions accepted by the system. The domain also determines whether a
question should be a cross component call or an external call when two
transition systems are composed. The composition is used to model linking
programs so the transition systems being composed use the same language
interface. As a result, it does not bring much trouble if the domain depends on
particular language interfaces. However, in a more general setting where
transition systems with different language interfaces interact with each other
we need to generalize the notion of domain to \emph{footprint}, which is
language independent and represents a set of identifiers owned by the transition
system.

Additionally, in categorical composition the transition systems form a layered
hierarchy where the function calls flow from the overlays to the underlays but
not the other way round. As a consequence, any attempts from the underlays to
call into overlays should be considered undefined behavior. Therefore, the
transition relation is parametrized by a footprint which is the identifiers
reserved for overlays.

\begin{definition}[Labeled Transition System]
Given an \emph{incoming} language interface $B$
and an \emph{outgoing} language interface $A$,
a \emph{labeled transition system for the game $A \twoheadrightarrow B$}
is a tuple $L = \langle S, \rightarrow, P, I, X, Y, F \rangle$.
The footprint $P$ is a set of reserved identifiers.
The relation
${\rightarrow^Q} \subseteq S \times \mathbb{E}^* \times S$ is
a \emph{transition relation} on the set of states $S$
parametrized by a footprint.
$I \subseteq D \times S$ 
assigns to each one a set of \emph{initial states}.
$F \subseteq S \times B^\bullet$
designates \emph{final states} together with corresponding answers.
External calls are specified by
$X \subseteq S \times A^\circ$,
which designates \emph{external states} together with
a question of $A$, and
$Y \subseteq S \times A^\bullet \times S$,
which is used to select a \emph{resumption state}
to follow an external state
based on the answer provided by the environment.
We write $L : A \twoheadrightarrow B$ when
$L$ is a labeled transition system for $A \twoheadrightarrow B$.
\end{definition}

Footprint itself is language independent but it can be associated with language
interfaces and serves as a predicate on questions and the convertion will be
implicit when the context is clear.

The identity transition system $\mathsf{id}$ simply passes through the calls
from the environment without doing anything. Two transition systems can be
composed as $L_1 \circ L_2$ where the outgoing calls from overlay system $L_2$
are propagated to the underlay $L_1$ as incoming calls.

\begin{definition}[Identity transition system] \label{def:lts-id}

The identity transition system $\mathsf{id}_A : A \twoheadrightarrow A$ can be
defined as:
\[
  \mathsf{id}_A :=
  \langle A^\circ + A^\bullet,\: \varnothing,\: \varnothing,\: I,\: X,\: Y,\: F \rangle
\]
where the components are
\[
  \begin{prooftree}
    \infer0[$\kw{i}^\circ$]{q \mathrel{I} \iota_1(q)}
  \end{prooftree}
  \qquad
  \begin{prooftree}
    \infer0[$\kw{x}^\circ$]{\iota_1(q) \mathrel{X} q}
  \end{prooftree}
  \qquad
  \begin{prooftree}
    \infer0[$\kw{x}^\bullet$]{r \mathrel{Y^{\iota_1(q)}} \iota_2(r)}
  \end{prooftree}
  \qquad
  \begin{prooftree}
    \infer0[$\kw{i}^\bullet$]{\iota_2(r) \mathrel{F} r}
  \end{prooftree}
\]
The identity transition system permits no internal steps. Questions from the
environment are immediately passed through as external calls($\kw{i}^\circ$,
$\kw{x}^\circ$) and the answer it receives is returned to the environment
without any operation ($\kw{i}^\bullet$, $\kw{x}^\bullet$).

\end{definition}

\begin{definition}[Composition of transition systems] \label{def:lts-comp}
Given two transition systems
$L_1 = \langle S_1, {\rightarrow_1}, P_1, I_1, X_1, Y_1, F_1 \rangle
: B \twoheadrightarrow C$ and
$L_2 = \langle S_2, {\rightarrow_2}, P_2, I_2, X_2, Y_2, F_2 \rangle
: A \twoheadrightarrow B$,
the composite transition system is defined as
\[
  L_1 \circ L_2 :=
  \langle S_1 + (S_2 \times S_1), {\rightarrow}, P_1 \cup P_2, I, X, Y, F \rangle
\]
with the following components.
\[
  \begin{prooftree}
    \hypo{q_C \mathrel{I_1} s_1}
    \infer1[$\kw{i}^\circ$]{q_C \mathrel{I} \iota_1(s_1)}
  \end{prooftree}
  \quad
  \begin{prooftree}
    \hypo{s_1 \mathrel{F_1} r_C}
    \infer1[$\kw{i}^\bullet$]{\iota_1(s_1) \mathrel{F} r_C}
  \end{prooftree}
  \quad
  \begin{prooftree}
    \hypo{s_2 \mathrel{X_2} q_A}
    \hypo{\forall i\,.\, q_A \notin P_i}
    \infer2[$\kw{x}^\circ$]{\iota_2(s_2, s_1) \mathrel{X} q_A}
  \end{prooftree}
  \quad
  \begin{prooftree}
    \hypo{r_A \mathrel{Y_2^{s_2}} s_2'}
    \infer1[$\kw{x}^\bullet$]{r_A \mathrel{Y^{\iota_2(s_2, s_1)}} \iota_2(s_2', s_1)}
  \end{prooftree}
  \quad
  \begin{prooftree}
    \hypo{s_1 \rightarrow_1^Q s_1'}
    \infer1[$\kw{r}_1$]{\iota_1(s_1) \rightarrow^Q \iota_1(s_1')}
  \end{prooftree}
\]
\[
  \begin{prooftree}
    \hypo{s_1 \mathrel{X_1} q_B}
    \hypo{q_B \mathrel{I_2} s_2}
    \hypo{q_B \notin P_1 \cup Q}
    \infer3[\kw{push}]{\iota_1(s_1) \rightarrow^Q \iota_2(s_2, s_1)}
  \end{prooftree}
  \quad
  \begin{prooftree}
    \hypo{s_2 \rightarrow_2^{P_1 \cup Q} s_2'}
    \infer1[$\kw{r}_2$]{\iota_2(s_2, s_1) \rightarrow^Q \iota_2(s_2', s_1)}
  \end{prooftree}
  \quad
  \begin{prooftree}
    \hypo{s_2 \mathrel{F_2} r_B}
    \hypo{r_B \mathrel{Y_1^{s_1}} s_1'}
    \infer2[\kw{pop}]{\iota_2(s_2, s_1) \rightarrow^Q \iota_1(s_1')}
  \end{prooftree}
\]
\end{definition}
The environment sends an incoming question to $L_1$ to initiate the execution of
the composite transition system($\kw{i}^\circ$), and $L_1$ proceeds according to
its own transition rules($\kw{r}_1$, $\kw{i}^\bullet$). Upon external calls, the
execution of $L_1$ is suspended and other transition system $L_2$ is activated,
where questions within the footprint of $L_1$ itself or the inherited footprint
are rejected($\kw{push}$). Then $L_2$ proceeds according to its transition rules
and gives the control back to $L_1$ when it has reached the final state
($\kw{r}_2$, $\kw{x}^\circ$, $\kw{x}^\bullet$, $\kw{pop}$). The footprint $P_1$
is passed to transition steps of $L_2$ as a restriction on the cross-component
call if $L_2$ itself is also a composite transition system.

We stick with CompCertO's forward simulation with simulation conventions as a
refinement order on transition systems. A simulation convention between language
interfaces $A$ and $B$ is a Kripke relation
$\mathbb{R} = \langle W, \mathbb{R}^\circ, \mathbb{R}^\bullet \rangle$ where $W$
is a set, $\mathbb{R}^\circ \in \mathcal{R}_W(A_1^\circ, A_2^\circ)$ and
$\mathbb{R}^\bullet \in \mathcal{R}_W(A_1^\bullet, A_2^\bullet)$, denoted by
$\mathbb{R} : A \Leftrightarrow B$. The simulation conventions ensure that
corresponding pairs of questions and answers are related consistently.  A
forward simulation on transition systems $L_1: A_1 \twoheadrightarrow B_1$ and
$L_2: A_2 \twoheadrightarrow B_2$ with simulation conventions
$\mathbb{R}_A : A_1 \Leftrightarrow A_2$ and
$\mathbb{R}_B : B_1 \Leftrightarrow B_2$ states that any transition in $L_1$ has
a corresponding transition sequence in $L_2$, denoted by
$L_1 \le_{\mathbb{R}_A \twoheadrightarrow \mathbb{R}_B} L_2$.

The transition system semantics model forms a double category where the objects
are language interfaces, the horizontal morphisms are transition systems, the
vertical morphisms are simulation conventions, and the 2-cells are
simulations. The 2-cells compose along the horizontal morphisms as well as the
vertical morphisms as depicted in the following properties and diagrams.
\[
  \begin{prooftree}
    \hypo{L_1 \le_{S \twoheadrightarrow T} L_1'}
    \hypo{L_2 \le_{R \twoheadrightarrow S} L_2'}
    \infer2{L_1 \circ L_2 \le_{R \twoheadrightarrow T} L_1' \circ L_2'}
  \end{prooftree}
  \qquad
  \begin{tikzcd}
    A \ar[r, ->>, "L_1"] \ar[d, <->, "R"] &
    B \ar[r, ->>, "L_2"] \ar[d, <->, "S"] &
    C \ar[d, <->, "T"] \\
    A' \ar[r, ->>, "L_1'"'] &
    B' \ar[r, ->>, "L_2'"'] &
    C'
  \end{tikzcd}
  \Rightarrow
  \begin{tikzcd}
    A \ar[rr, ->>, "L_1 \circ L_2"] \ar[d, <->, "R"] &&
    C \ar[d, <->, "T"] \\
    A' \ar[rr, ->>, "L_1' \circ L_2'"'] &&
    C'
  \end{tikzcd}
\]
\[
  \begin{prooftree}
    \hypo{L_1 \le_{R \twoheadrightarrow S} L_2}
    \hypo{L_2 \le_{T \twoheadrightarrow U} L_3}
    \infer2{L_1 \le_{R \cdot T \twoheadrightarrow S \cdot U} L_3}
  \end{prooftree}
  \qquad
  \begin{tikzcd}
    A_1 \ar[r, ->>, "L_1"] \ar[d, <->, "R"] & B_1 \ar[d, <->, "S"] \\
    A_2 \ar[r, ->>, "L_2"] \ar[d, <->, "T"] & B_2 \ar[d, <->, "U"] \\
    A_3 \ar[r, ->>, "L_3"] & B_3
  \end{tikzcd}
  \Rightarrow
  \begin{tikzcd}[row sep=large]
    A_1 \ar[r, ->>, "L_1"] \ar[dd, <->, "R \cdot T"] & B_1 \ar[dd, <->, "S \cdot U"] \\ \\
    A_3 \ar[r, ->>, "L_3"] & B_3
  \end{tikzcd}
\]

Finally, we prove that the categorical composition of assembly programs
approximates the horizontal composition from the original CompCertO
semantics.

\subsection{Denotational Semantics}

Explain the embedding, which goes:
\[
  L : A \twoheadrightarrow B
  \qquad \mapsto \qquad
  \llbracket L \rrbracket : {\dagger} A \multimap B
  \qquad \mapsto \qquad
  \llbracket L \rrbracket^\dagger : {\dagger} A \multimap {\dagger} B
\]
For simulation conventions, something like:
\[
  \mathbb{R} : A \Leftrightarrow B
  \qquad \mapsto \qquad
  \llbracket \mathbb{R} \rrbracket : A \leftrightarrows B
  \qquad \mapsto \qquad
  {\dagger}\llbracket \mathbb{R} \rrbracket :
    {\dagger}A \leftrightarrows {\dagger}B
\]
Open transition systems and forward simulations can be embedded into
object-based semantics as follows.

A language interface
$A = \langle A^\circ, A^\bullet \rangle$ can be read as an effect signature
which incorporates all the operations in the shape of questions in $A^\circ$ and
they have the same arity $A^\bullet$.
\[
  \llbracket A \rrbracket \mathrel{:=}
  \{ op \mathrel{:=} A^\circ, ar\ \_ \mathrel{:=} A^\bullet \}
\]

A transition system
$L: A \twoheadrightarrow B = \langle S, \rightarrow, P, I, X, Y, F \rangle$ can
then be interpreted as a linear map
$\dagger \llbracket A \rrbracket \multimap \llbracket B \rrbracket$, which is a
signature homomorphism of type
$ \llbracket B \rrbracket \rightarrow \dagger \llbracket A \rrbracket $
\[
  \llbracket L \rrbracket (q_B) \mathrel{:=} \bigsqcup_{q_BIs} run_L(s)
\]
\[
  run_L(s) \mathrel{:=}  \bigsqcup_{s\rightarrow^* s'\wedge s'Fr_B} \underline{r_B}
  \sqcup \bigsqcup_{s\rightarrow^* s' \wedge s'Xq_A} \underline{q_A}
  (r_A \mapsto \bigsqcup_{rY^{s'}s''} run_L(s''))
\]
The signature homomorphism takes an operation $q_B \in \llbracket B \rrbracket$
and activates the transition system with the operation by angelically choosing
an initial state. Starting from the initial state, the operation $run_L(s)$
builds a term in $E^*(ar(q_B))$ recursively by driving the transitions until a
final state is reached. The internal states are ignored as the game model
emphasizes on the externally observable behaviors. The angelic choices help
preserving all possible execution paths in the embedding.

A simulation convention $R : C \Leftrightarrow A$ between source program
language interface $C$ and target program language interface $A$ is encoded as
adjunctions $R^* \dashv R_* : C \leftrightarrows A$. Typically, the source
program is a high-level specification whereas the target program is specified
with implementation details.
\begin{align*}
  R^*(q_A) \mathrel{:=} \bigsqcup_{q_C R^\circ q_A} \underline{q_C}(r_C \mapsto \bigsqcap_{r_C R^\bullet r_A} r_A) \\
  R_*(q_C) \mathrel{:=} \bigsqcap_{q_C R^\circ q_A} \underline{q_A}(r_A \mapsto \bigsqcup_{r_C R^\bullet r_A} r_C)
\end{align*}
In the first case, the left adjoint $R^*: A \rightarrow C$ translates a
low-level call into a high-level call by making an angelic choice, which can be
viewed as a partial function abstracts away the implementation details from a
call in the target language. In case there is no corresponding representation in
the source language, the translation goes wrong. On the other hand, the target
program is allowed to choose the low-level representation it wants to resume
with.

In the second case, the right adjoint $R_*: C \rightarrow A$ works the other way
round by making a demonic choice on the operation, which results in the most
general representation of the specification in the target language. A target
program that refines such representation is a correct implementation of the
specification. On the other hand, the translation is free to choose however it
interprets the result in the form of the low-level interface which corresponds
to the angelic choice.

The forward simulations between transition systems can be embedded as
refinements in the game model, which categorically transport the 2-cells in the
category of language interfaces to the 2-cells in the enriched category of
effect signatures.

\subsection{Control Operators (optional, medium difficulty)}

Although we can't define things like setjmp/longjmp/context switching
directly in terms of the CompCertO semantics,
we can define them as operators on embedded games semantics
of CompCert programs,
which interpret calls to the relevant primitives
in the appropriate way.
In this part we could for example:
\begin{itemize}
  \item Give a specification of context switching in these terms
  \item Give a certified implementation at the assembly level
\end{itemize}
This would go a long way towards
a reformulation of (parts of) CCAL into RBGS.

\subsection{Processes and their Environment (optional, lesser difficulty)}

Define \emph{loaders} to ``close'' the C interface of a component.
This means we don't have to deal with memory states any more
and just get the observable behavior of a closed process.
That behavior could have a degrees of sophistication:
\begin{itemize}
  \item Just use CompCert events, but use game semantics to interpret them
  \item Still allow outgoing calls in the (semi-)closed semantics
  \item Have the loader include semantics for some standard library functions,
    and give a higher-level view of what they do
    using a new outgoing language interface.
\end{itemize}

If we do that,
we can model how closed CompCert processes interact
with the operating system,
and could give a small example like
specify and verify simple versions of
the \texttt{sort} and \texttt{uniq} commands,
then model the behavior of a
\texttt{sort|uniq}
pipeline.

Another reason to do this would be to close a gap in CompCertO,
where currently we do not prove that the initial invocation of main
(including the initial memory state)
is compatible with our calling convention.

\section{Certified Abstraction Layers}

\section{Related Work}

\section{Conclusion}

\appendix
\newpage

\noindent
For now,
the rest of this document contains my previous CompCertOX write-up.

\begin{figure}[h] % fig:ex {{{
  \begin{center}
    \begin{tikzcd}[row sep=1cm, column sep=1.5cm]
      1 \ar[d, double, dash] \ar[rrr, ->>, "L''"] &&&
      \mathcal{C}@K''
        \ar[d, <->, "S"]
        \ar[r, ->>, "\llbracket C \rrbracket@K''"] &
      \mathcal{C}@K''
        \ar[ddd, <->, "R \circ S"]
      \\
      1 \ar[d, double, dash] \ar[rr, ->>, "L'"] &&
      \mathcal{C}@K'
        \ar[d, <->, "R"]
        \ar[r, ->>, "\llbracket N \rrbracket@K'"] &
      \mathcal{C}@K'
        \ar[d, <->, "R"]
      \\
      1 \ar[d, double, dash] \ar[r, ->>, "L"] &
      \mathcal{C}@K
        \ar[d, double, dash] 
        \ar[r, ->>, "\llbracket M \rrbracket@K"] &
      \mathcal{C}@K
        \ar[r, ->>, "\llbracket N \rrbracket@K"] &
      \mathcal{C}@K
        \ar[d, double, dash]
      \\
      1 \ar[d, double, dash] \ar[r, ->>, "L"] &
      \mathcal{C}@K
        \ar[d, double, dash] 
        \ar[rr, ->>, "\llbracket M + N \rrbracket@K"] &&
      \mathcal{C}@K
        \ar[r, ->>, "\llbracket C \rrbracket@K"] &
      \mathcal{C}@K
        \ar[d, double, dash] 
      \\
      1 \ar[r, ->>, "L"] &
      \mathcal{C}@K
        \ar[rrr, ->>, "\llbracket C + M + N \rrbracket@K"] &&&
      \mathcal{C}@K
    \end{tikzcd}
  \end{center}
  \caption}}

\section{Overview} %{{{

Our usual
\href{https://certikos.github.io/rbgs-papers/thesis/thesis.pdf\#chapter.4}%
  {theory of abstraction layers}
can be formulated in CompCertO's double category of
language interfaces, simulation conventions and transition systems:
\begin{itemize}
  \item A layer interface $L$ with abstract states in $K$
    is represented as $L : 1 \twoheadrightarrow \mathcal{C}@K$
  \item Clight semantics define
    $\llbracket M \rrbracket : \mathcal{C} \twoheadrightarrow \mathcal{C}$
    and lift to
    $\llbracket M \rrbracket @ K :
     \mathcal{C}@K \twoheadrightarrow \mathcal{C}@K$
  \item Abstraction relations define simulation conventions
    $R : \mathcal{C}@K' \Leftrightarrow \mathcal{C}@K$
  \item Layer correctness
    $L \vdash_R M : L'$
    is encoded as
    $L' \le_{\mathsf{id} \twoheadrightarrow R}
     \llbracket M \rrbracket @K \circ L$
\end{itemize}
\autoref{fig:ex}
demonstrates how these ingredients may fit together
in a typical situation.

The following operators
are used to formulate vertical and horizontal composition
of abstraction layers.
In the homogenous case
$(A \twoheadrightarrow A) \times
 (A \twoheadrightarrow A) \rightarrow
 (A \twoheadrightarrow A)$,
they under-approximate $\oplus$
and can therefore be implemented by linking.
\begin{itemize}
  \item Categorical composition \hfill
    $\circ :
      (B \twoheadrightarrow C) \times
      (A \twoheadrightarrow B) \rightarrow
      (A \twoheadrightarrow C) \qquad$
  \item Flat composition \hfill
    $\uplus :
      (A \twoheadrightarrow B) \times
      (A \twoheadrightarrow B) \rightarrow
      (A \twoheadrightarrow B) \qquad$
\end{itemize}
To reconnect formally with Yu's work,
we can investigate the following further:
\begin{itemize}
  \item A CompCertO transition system $L : A \twoheadrightarrow B$
    can be embedded into $\dagger A \multimap B$.
  \item The cliques of $\dagger \mathcal{C}$
    are universal abstract states.
  \item We can map between effect signatures $E$
    and the language interfaces $\mathcal{C}, \mathcal{A}$.
\end{itemize}
We can then define
a principled embedding
into game semantics or coherence spaces.

\paragraph{Status}

This draft should mostly be accurate,
however we will have to work out some of the details and kinks.
Here is a list of issues:
\begin{itemize}
  \item Components which make outgoing calls
    to functions in their own domain
    cause problems in the relationship between
    $\circ$, $\uplus$ and $\oplus$.
  \item In fact the word ``domain'' is somewhat confusing,
    more precisely it's a set of symbols
    that are reserved by the component.
\end{itemize}

%A certified abstraction layer
%$L_2 \vdash_R M : L_1$
%establishes that:
%\[
%  L_1 \le_{\mathsf{id} \twoheadrightarrow R}
%  \llbracket M \rrbracket @K_2 \circ L_2
%\]
%Various properties of the operators involved
%ensure that
%for any context $C$,
%\[
%  \llbracket C \rrbracket@K_1 \circ L_1
%  \: \le_{\mathsf{id} \twoheadrightarrow R} \:
%  \llbracket C \rrbracket@K_2 \circ \llbracket M \rrbracket@K_2 \circ L_2
%  \: \le_{\mathsf{id} \twoheadrightarrow \mathsf{id}} \:
%  \llbracket C + M \rrbracket@K_2 \circ L_2
%  \,.
%\]
%In particular,
%this enables vertical composition:
%\[
%  L_1
%  \: \le_{\mathsf{id} \twoheadrightarrow R} \:
%  \llbracket M \rrbracket@K_2 \circ L_2
%  \: \le_{\mathsf{id} \twoheadrightarrow S} \:
%  \llbracket M + N \rrbracket@K_3 \circ L_3
%\]
%On the other hand,
%the flat composition operator:
%\[
%  L_1, L_2 : A \twoheadrightarrow B
%  \vdash
%  L_1 \uplus L_2 : A \twoheadrightarrow B
%\]
%allows us to carry out a similar

%}}}

\section{Categorical structure of CompCertO semantics} %{{{

The semantic model used in CompCertO
can be organized into a double category:
\begin{itemize}
  \item the objects are language interfaces;
  \item the horizontal morphisms are open transition systems;
  \item the vertical morphisms are simulation conventions;
  \item the 2-cells are simulations.
\end{itemize}
The CompCertO paper defines
the vertical composition of simulation conventions,
but focuses on a symmetric form of horizontal composition,
meant to model linking:
\[
  {\oplus} :
    (A \twoheadrightarrow A) \times
    (A \twoheadrightarrow A) \rightarrow
    (A \twoheadrightarrow A)
\]

In this section,
I complement the constructions on CompCertO's open semantics
to make their double category structure explicit.
In particular,
I introduce the simpler and more fundamental
horizontal composition operators:
\[
  \begin{array}{c@{\:}l}
  {\circ} &:
    (B \twoheadrightarrow C) \times
    (A \twoheadrightarrow B) \rightarrow
    (A \twoheadrightarrow C)
  \\
  {\uplus} &:
    (A \twoheadrightarrow B) \times
    (A \twoheadrightarrow B) \rightarrow
    (A \twoheadrightarrow B)
  \end{array}
\]
The linking operator $\oplus$
can then be recovered or characterized as the fixed point
\[
  L_1 \oplus L_2 :=
    \mu X \cdot (L_1 \uplus L_2) \circ X
  \,.
\]

\subsection{Horizontal category} %{{{

\subsubsection{Identity} %{{{

The identity transition system $\mathsf{id}_A : A \twoheadrightarrow A$
can be defined as:
\[
  \mathsf{id}_A :=
    \langle A^\circ + A^\bullet,\: \varnothing,\: \varnothing,\: I,\: X,\: Y,\: F \rangle
\]
The components are defined by the rules:
\[
  \begin{prooftree}
    \infer0{q \mathrel{I} \iota_1(q)}
  \end{prooftree}
  \qquad
  \begin{prooftree}
    \infer0{\iota_1(q) \mathrel{X} q}
  \end{prooftree}
  \qquad
  \begin{prooftree}
    \infer0{r \mathrel{Y^{\iota_1(q)}} \iota_2(r)}
  \end{prooftree}
  \qquad
  \begin{prooftree}
    \infer0{\iota_2(r) \mathrel{F} r}
  \end{prooftree}
\]

%}}}

\subsubsection{Composition} %{{{

Suppose we have the transition systems:
\begin{align*}
  L_1 &= \langle S_1, {\rightarrow_1}, D_1, I_1, X_1, Y_1, F_1 \rangle
    : B \twoheadrightarrow C
  \\
  L_2 &= \langle S_2, {\rightarrow_2}, D_2, I_2, X_2, Y_2, F_2 \rangle
    : A \twoheadrightarrow B
\end{align*}
The composite transition system is defined as
\[
  L_1 \circ L_2 :=
  \langle S, {\rightarrow}, D_1 \cup D_2, I, X, Y, F \rangle
\]
with the following components.
States are of the form:
\[
    S := S_1 + (S_2 \times S_1)
\]
Initially, the environment question activates $L_1$:
\[
  \begin{prooftree}
    \hypo{q_C \mathrel{I_1} s_1}
    \infer1{q_C \mathrel{I} \iota_1(s_1)}
  \end{prooftree}
  \qquad
  \begin{prooftree}
    \hypo{s_1 \rightarrow_1 s_1'}
    \infer1{\iota_1(s_1) \rightarrow \iota_1(s_1')}
  \end{prooftree}
  \qquad
  \begin{prooftree}
    \hypo{s_1 \mathrel{F_1} r_C}
    \infer1{\iota_1(s_1) \mathrel{F} r_C}
  \end{prooftree}
\]
When an external call is encountered,
the question is used to activate $L_2$:
\[
  \begin{prooftree}
    \hypo{s_1 \mathrel{X_1} q_B}
    \hypo{q_B \mathrel{I_2} s_2}
    \infer2{\iota_1(s_1) \rightarrow \iota_2(s_2, s_1)}
  \end{prooftree}
\]
Execution proceeds according to $L_2$,
\[
  \begin{prooftree}
    \hypo{s_2 \rightarrow_2 s_2'}
    \infer1{\iota_2(s_2, s_1) \rightarrow \iota_2(s_2', s_1)}
  \end{prooftree}
  \qquad
  \begin{prooftree}
    \hypo{s_2 \mathrel{X_2} q_A}
    \infer1{\iota_2(s_2, s_1) \mathrel{X} q_A}
  \end{prooftree}
  \qquad
  \begin{prooftree}
    \hypo{r_A \mathrel{Y_2^{s_2}} s_2'}
    \infer1{r_A \mathrel{Y^{\iota_2(s_2, s_1)}} \iota_2(s_2', s_1)}
  \end{prooftree}
\]
until a final state of $L_2$ is reached,
at which point $L_1$ is resumed:
\[
  \begin{prooftree}
    \hypo{s_2 \mathrel{F_2} r_B}
    \hypo{r_B \mathrel{Y_1^{s_1}} s_1'}
    \infer2{\iota_2(s_2, s_1) \rightarrow \iota_1(s_1')}
  \end{prooftree}
\]

%}}}

\subsubsection{Properties} %{{{

I will write:
\begin{itemize}
  \item $L_1 \le L_2$
    to mean $L_1 \le_{\mathsf{id} \twoheadrightarrow \mathsf{id}} L_2$,
  \item $L_1 \equiv L_2$
    to mean $L_1 \le L_2 \wedge L_2 \le L_1$.
\end{itemize}
The expected properties of the horizontal
identity and categorical composition
can then be formulated in the following way.

\begin{theorem}
For a transition system $L : A \twoheadrightarrow B$,
the following property holds:
\[
    L \circ \mathsf{id}_A \equiv \mathsf{id}_B \circ L \equiv L
    \,.
\]
Moreover, for the transition systems
\[
  \begin{tikzcd}
    A \ar[r, ->>, "L_3"] &
    B \ar[r, ->>, "L_2"] &
    C \ar[r, ->>, "L_1"] &
    D \,,
  \end{tikzcd}
\]
the following property holds:
\[
    (L_1 \circ L_2) \circ L_3 \equiv L_1 \circ (L_2 \circ L_3)
\]
\begin{proof}
It should be straightforward to verify
that the identity acts as a unit for composition.
Associativity can be verified using
the simulation relation:
\[
  \begin{array}{rr}
    \hline
    L_1 \circ (L_2 \circ L_3) & (L_1 \circ L_2) \circ L_3 \\
    \hline
    \iota_1(s_1) & \iota_1(\iota_1(s_1)) \\
    \iota_2(\iota_1(s_2), s_1) & \iota_1(\iota_2(s_2, s_1)) \\
    \iota_2(\iota_2(s_3, s_2), s_1) & \iota_2(s_3, \iota_2(s_2, s_1)) \\
    \hline
  \end{array}
\]
\end{proof}
\end{theorem}

%}}}

\begin{remark}[Domains and categorical composition]
  Unlike the horizontal composition operator $\oplus$
  and the flat composition operator $\uplus$ introduced in the next section,
  categorical composition does not make use of the component's domains
  to compute the behavior of the composite transition system.
  This means that in general,
  a component may exhibit meaningful behaviors on queries outside its domain.
  This is the case in particular for $\mathsf{id}_A$,
  where the domain is $\varnothing$ but
  every possible query is associated with a behavior.
  In turn it suggests a modification to CompCertO's
  language semantics which would make them act as ``passthough''
  on queries outside of a module's domain.

  I should also note that categorical composition as
  given in this section will require modifying
  the definition of CompCertO's transition systems slightly.
  As it stands,
  a component's domain $D$ is a set of queries of
  the incoming language interface.
  However,
  for the domain $D_1 \cup D_2$ used in $L_1 \circ L_2$
  to make sense when $B \neq C$,
  we will have to change it to a language-independent
  notion of domain, for example a set of identifiers.
  Language interfaces
  would then provide a way to recognize whether queries
  are part of a domain expressed in this general way.
\end{remark}

%}}}

\subsection{Simulations} %{{{

Horizontal composition is compatible with simulations in the following sense.
Given
$L_1 : B \twoheadrightarrow C$ and $L_2 : A \twoheadrightarrow B$,
which are simulated respectively by
$L_1' : B' \twoheadrightarrow C'$ and $L_2' : A' \twoheadrightarrow B'$,
the following property holds:
\[
  \begin{prooftree}
    \hypo{L_1 \le_{S \twoheadrightarrow T} L_1'}
    \hypo{L_2 \le_{R \twoheadrightarrow S} L_2'}
    \infer2{L_1 \circ L_2 \le_{R \twoheadrightarrow T} L_1' \circ L_2'}
  \end{prooftree}
\]
Diagrammatically,
this allows us to paste simulations squares horizontally:
\[
  \begin{tikzcd}
    A \ar[r, ->>, "L_1"] \ar[d, <->, "R"] &
    B \ar[r, ->>, "L_2"] \ar[d, <->, "S"] &
    C \ar[d, <->, "T"] \\
    A' \ar[r, ->>, "L_1'"'] &
    B' \ar[r, ->>, "L_2'"'] &
    C'
  \end{tikzcd}
  \qquad \Longrightarrow \qquad
  \begin{tikzcd}
    A \ar[rr, ->>, "L_1 \circ L_2"] \ar[d, <->, "R"] &&
    C \ar[d, <->, "T"] \\
    A' \ar[rr, ->>, "L_1' \circ L_2'"'] &&
    C'
  \end{tikzcd}
\]

Note that the vertical composition of simulation squares
corresponds to the usual composition of simulations already given
in the CompCertO paper:
\[
  \begin{prooftree}
    \hypo{L_1 \le_{R \twoheadrightarrow S} L_2}
    \hypo{L_2 \le_{T \twoheadrightarrow U} L_3}
    \infer2{L_1 \le_{R \cdot T \twoheadrightarrow S \cdot U} L_3}
  \end{prooftree}
  \hspace{3em}
  \begin{tikzcd}
    A_1 \ar[r, ->>, "L_1"] \ar[d, <->, "R"] & B_1 \ar[d, <->, "S"] \\
    A_2 \ar[r, ->>, "L_2"] \ar[d, <->, "T"] & B_2 \ar[d, <->, "U"] \\
    A_3 \ar[r, ->>, "L_3"] & B_3
  \end{tikzcd}
  \quad
  \begin{tikzcd}[row sep=large]
    A_1 \ar[r, ->>, "L_1"] \ar[dd, <->, "R \cdot T"] & B_1 \ar[dd, <->, "S \cdot U"] \\ \\
    A_3 \ar[r, ->>, "L_3"] & B_3
  \end{tikzcd}
\]

One last observation is that the identity transition system
allows us to formulate the refinement of simulation conventions itself
as a simulation square.
There is a nice symmetry
with the refinement of transition systems:
\[
  \begin{tikzcd}[sep=tiny]
    A \ar[dd, double, dash, "\mathsf{id}_A"'] \ar[rr, ->>, "L_1"] &&
    B \ar[dd, double, dash, "\mathsf{id}_B"] \\
    & L_1 \le L_2 & \\
    A \ar[rr, ->>, "L_2"'] &&
    B
  \end{tikzcd}
  \hspace{5em}
  \begin{tikzcd}[sep=tiny]
    A \ar[dd, <->, "S"'] \ar[rr, double, dash, "\mathsf{id}_A"] &&
    A \ar[dd, <->, "R"] \\
    & R \sqsupseteq S & \\
    B \ar[rr, double, dash, "\mathsf{id}_B"'] &&
    B
  \end{tikzcd}
\]

%}}}

\subsection{Flat composition} %{{{

The categorical composition of two transition systems
chains them together,
directing any outgoing calls of the first
to incoming calls of the second.
I now introduce another kind of composition
which lays them out side-by-side.

\begin{definition}
The \emph{flat composition} of the transition systems
\begin{align*}
  L_1 &= \langle S_1, {\rightarrow_1}, D_1, I_1, X_1, Y_1, F_1 \rangle
    : A \twoheadrightarrow B
  \\
  L_2 &= \langle S_2, {\rightarrow_2}, D_2, I_2, X_2, Y_2, F_2 \rangle
    : A \twoheadrightarrow B
\end{align*}
is the transition system
$L_1 \uplus L_2 : A \twoheadrightarrow B$
defined as:
\[
  L_1 \uplus L_2 :=
    \langle
      S_1 + S_2, \:
      {\rightarrow}, \:
      D_1 \cup D_2, \:
      I, \: X, \: Y, \: F
    \rangle
\]
The components are defined by the following rules,
where $i \in \{1, 2\}$:
\[
  \begin{prooftree}
    \hypo{q \mathrel{I_i} s}
    \hypo{q \in D_i}
    \infer2{q \mathrel{I} \iota_i(s)}
  \end{prooftree}
  \quad
  \begin{prooftree}
    \hypo{s \rightarrow_i s'}
    \infer1{\iota_i(s) \rightarrow \iota_i(s')}
  \end{prooftree}
  \quad
  \begin{prooftree}
    \hypo{s \mathrel{X_i} q}
    \infer1{\iota_i(s) \mathrel{X} q}
  \end{prooftree}
  \quad
  \begin{prooftree}
    \hypo{r \mathrel{Y_i^s} s'}
    \infer1{r \mathrel{Y^{\iota_i(s)}} \iota_i(s')}
  \end{prooftree}
  \quad
  \begin{prooftree}
    \hypo{s \mathrel{F_i} r}
    \infer1{\iota_i(s) \mathrel{F} r}
  \end{prooftree}
\]
\end{definition}

\noindent
I suspect the following properties hold when applicable:
\begin{itemize}
  \item $\mathsf{id} \uplus L \equiv L \uplus \mathsf{id} \equiv L$
  \item $(L_1 \uplus L_2) \circ L \equiv (L_1 \circ L) \uplus (L_2 \circ L)$
\end{itemize}

\begin{remark}
It may be necessary for $\uplus$
to act as \emph{passthrough}
for queries outside of its domain.
This would enable the correspondence with $\oplus$
described in the next section.

The main difficulty is that
this can only be done when $A = B$,
so we would want to achieve this effect indirectly.
One option would be to expect language semantics
to be \emph{passthrough} outside their domain
and to have a nondeterministic choice between the components
when we're outside the domain of both.
That is to say,
each component is \emph{inhibited by the other's domain}
instead of \emph{enabled by its own}.
In normal situations the behavior of both components
would be the same in the ``gap'' between the domains,
so it would only be nondeterminism on a formal level.
\end{remark}

%}}}

\subsection{Linking} %{{{

The linking operator $\oplus$
can be described as the limit:
\[
  L_1 \oplus L_2 :=
  \bigvee_{n \in \mathbb{N}}
    (L_1 \uplus L_2)^n \circ \bot_{D_1 \cup D_2}
\]
where:
\begin{itemize}
  \item
    $L^n$ is the $n$-fold composition $L \circ \cdots \circ L$;
  \item
    $D_1$ and $D_2$ are the respective domains of $L_1$ and $L_2$;
  \item
    $\bot_D$ is undefined on its domain $D$ and
    \emph{passthrough} outside of it.
\end{itemize}
Note in particular that $\mathsf{id} \equiv \bot_\varnothing$.

For certified abstraction layers,
our main interest is that $\oplus$ can be used
to established a connexion between
the various kinds of transition system compositions
and the semantics of the linked program.

\begin{theorem}
For two Clight programs $M$ and $N$,
the linked program $M + N$ is a correct implementation of
$\llbracket M \rrbracket \oplus \llbracket N \rrbracket$:
\[
    \llbracket M \rrbracket \oplus \llbracket N \rrbracket \le
    \llbracket M + N \rrbracket
\]
\begin{proof}
A linking theorem for the Asm language has already been proved.
The Clight proof should be similar.
\end{proof}
\end{theorem}

Then the key fact is that $\circ$ and $\uplus$
are both under-approximations of $\oplus$;
in other words, they are both implemented by linking:
\begin{align*}
  \llbracket M \rrbracket \circ \llbracket N \rrbracket \: &\le \:
    \llbracket M \rrbracket \oplus \llbracket N \rrbracket \: \le \:
    \llbracket M + N \rrbracket \\
  \llbracket M \rrbracket \uplus \llbracket N \rrbracket \: &\le \:
    \llbracket M \rrbracket \oplus \llbracket N \rrbracket \: \le \:
    \llbracket M + N \rrbracket
\end{align*}
The first property in particular
can be visualized as the simulation square:
\begin{equation}
  \begin{tikzcd}
    \mathcal{C} \ar[r, "\llbracket N \rrbracket"] \ar[d, double, dash] &
    \mathcal{C} \ar[r, "\llbracket M \rrbracket"] &
    \mathcal{C} \ar[d, double, dash] \\
    \mathcal{C} \ar[rr, "\llbracket M + N \rrbracket"'] & &
    \mathcal{C}
  \end{tikzcd}
  \label{eqn:linkingsquare}
\end{equation}

%}}}

\subsection{String diagrams} %{{{

Like 2-categories,
double categories admit a string diagram calculus where:
\begin{itemize}
  \item objects are represented by regions,
  \item horizontal morphisms are represented by vertical lines,
  \item vertical morphisms are represented by horizontal lines,
  \item 2-cells are represented by points.
\end{itemize}

The diagrams I have drawn so far
efficiently convey the type structure of the semantic framework;
they describe
the way language interfaces,
transition systems and simulation conventions
compose and interact.
But the \emph{simulation proofs} themselves
are literally found in the \emph{gaps} between
these entities.

String diagrams are useful because they turn this hierarchy on its head,
and offer a compelling visualization of the ways
complex simulations can be assembled from simpler ones.
A basic simulation square $f \le_{R \twoheadrightarrow S} g$ is drawn as:
\[
  \begin{tikzcd}[sep=small]
    A \ar[rr, ->>, "f"] \ar[dd, <->, "R"'] &&
    B \ar[dd, <->, "S"] \\
    & \phi & \\
    C \ar[rr, ->>, "g"'] &&
    D
  \end{tikzcd}
  \qquad \qquad
  \begin{tikzpicture}[baseline]
    \path[fill=blue!20] rectangle (-1,1);
    \path[fill=red!20] rectangle (1,1);
    \path[fill=green!20] rectangle (-1,-1);
    \path[fill=yellow!25] rectangle (1,-1);
    \draw (0,0) -- (0,+1) node[anchor=south] {$f$};
    \draw (0,0) -- (+1,0) node[anchor=west] {$S$};
    \draw (0,0) -- (0,-1) node[anchor=north] {$g$};
    \draw (0,0) -- (-1,0) node[anchor=east] {$R$};
    \node[draw,circle,fill=white] (sim) {$\phi$};
    \begin{scope}[inner sep=3mm]
      \node[below right] at (-1,1) {$A$};
      \node[below left] at (1,1) {$B$};
      \node[above right] at (-1,-1) {$C$};
      \node[above left] at (1,-1) {$D$};
    \end{scope}
  \end{tikzpicture}
\]

Much information can be elided from string diagrams.
Identity transition systems and simulation conventions
can be omitted completely.
Objects can be associated with a color,
eliminating redundant labeling.
For example,
here are depictions of
a simulation convention refinement property ($R \sqsupseteq S$),
a transition system refinement property ($L_1 \le L_2$),
and of the linking property (\ref{eqn:linkingsquare}):
\[
  \begin{tikzpicture}[baseline]
    \path[fill=magenta!20] (-1,1) rectangle (1,0);
    \path[fill=cyan!20] (-1,0) rectangle (1,-1);
    \draw (-1,0) node[left] {$S$}
      -- (0,0) node[draw,circle,fill=white,inner sep=0.5mm] {\tiny $\sqsubseteq$}
      -- (1,0) node[right,overlay] {$R$};
  \end{tikzpicture}
  \qquad \qquad
  \begin{tikzpicture}[baseline]
    \path[fill=magenta!20] (-1,1) rectangle (0,-1);
    \path[fill=cyan!20] (0,1) rectangle (1,-1);
    \draw (0,1) node[above] {$L_1$}
      -- (0,0) node[draw,circle,fill=white,inner sep=0.5mm] {\tiny $\le$}
      -- (0,-1) node[below] {$L_2$};
  \end{tikzpicture}
  \qquad \qquad
  \begin{tikzpicture}[baseline]
    \path[fill=blue!20] (-1,-1) rectangle (1,1);
    \begin{scope}
      \draw (-0.5, +1) node[above] {$\llbracket M \rrbracket$}
        .. controls +(-90:0.5) and +(180:0.2) .. (0,0);
      \draw (+0.5, +1) node[above] {$\llbracket N \rrbracket$}
        .. controls +(-90:0.5) and +(0:0.2) .. (0,0);
      \draw (0,0) -- (0,-1) node[below] {$\llbracket M + N \rrbracket$};
    \end{scope}
    \node[draw,circle,fill=white,inner sep=0.5mm] {\tiny $+$};
    \node[above right] at (-1,-1) {$\mathcal{C}$};
  \end{tikzpicture}
\]

\noindent
Skipping ahead,
here is a string diagram rendition of \autoref{fig:ex}.
\[
  \pgfdeclarelayer{nodes}
  \pgfsetlayers{main,nodes}
  \newcommand{\stens}{0.6}
  \begin{tikzpicture}[baseline,yscale=1.2,xscale=1.5]
    \footnotesize
    \tikzset{to path={
      .. controls ($(\tikztostart)!\stens!(\tikztostart -| \tikztotarget)$)
              and ($(\tikztotarget)!\stens!(\tikztostart -| \tikztotarget)$) ..
      (\tikztotarget) \tikztonodes}}

    % Boundary labels
    \begin{scope}
      \path (1,4) coordinate (L2) node[above] {$L''$};
      \path (2.5,4) coordinate (C) node[above] {$\llbracket C \rrbracket$};
      \path (3.5,3) coordinate (S) node[right] {$S$};
      \path (3.5,2) coordinate (R) node[right] {$R$};
      \path (0,0) coordinate (L) node[below] {$L$};
      \path (2,0) coordinate (T) node[below] {$\llbracket M+N+C \rrbracket$};
    \end{scope}

    % Simulation proofs
    \begin{pgfonlayer}{nodes}
      % Layer correctness
      \tikzset{every node/.style={rounded corners,draw,fill=white,inner sep=1mm}}
      \path (1,3) coordinate (LC2) node {$L' \vdash_S N : L''$};
      \path (0.5,2) coordinate (LC1) node {$L \vdash_R M : L'$};
      % Linking
      \tikzset{every node/.style={circle,draw,fill=white,inner sep=0.5mm}}
      \path (1.5,1.33) coordinate (LK1) node {$+$};
      \path (2,0.66) coordinate (LK2) node {$+$};
      % Parametricity
      \tikzset{every node/.style={circle,draw,fill=black,inner sep=0.5mm}}
      %\node (P2) at (4,2) {};
      %\node (P3) at (3,2) {};
    \end{pgfonlayer}

    % Regions
    \fill[color=magenta!20] (L2) to (LC2) -- (S) |- cycle;
    \fill[color=yellow!20] (LC2) to (LC1) -- (R) |- cycle;
    \fill[color=cyan!20] (LC1) to (L) -| (R) -- cycle;

    % Transition systems
    \begin{scope}[line width=1pt,inner sep=0.2mm]
      \draw (L2) to (LC2) to node[above left,pos=0.75] {$L'$} (LC1) to (L);
      \draw (T) to (LK2) to (LK1) to node[below left,pos=0.3] {$\llbracket M \rrbracket$} (LC1);
      \draw (LC2) to node[above right,pos=0.6] {$\llbracket N \rrbracket$} (LK1);
      \draw (LK2) to (C);
    \end{scope}

    % Simulation conventions
    \begin{scope}
      \draw (LC2) to (S);
      \draw (LC1) to (R);
    \end{scope}

    % Region labels
%    \begin{scope}[opacity=0.4]
%      \node[below right] at (-0.5,1) {$1$};
%      \node[above left] at (4,0) {$K$};
%      \node[left] at (4,3.5) {$K'$};
%      \node[below left] at (4,5) {$K''$};
%    \end{scope}
  \end{tikzpicture}
\]
Here the white region corresponds to
the empty language interface $1$,
whereas the colored regions correspond to
a version of the $\mathcal{C}$ language interface
extended to carry the various kinds of abstract states
used by the layer interfaces $L$, $L'$ and $L''$.
From the outer boundary of the diagram,
we can read the simulation property
\[
  \llbracket C \rrbracket@K'' \circ L''
  \le_{1 \twoheadrightarrow S \cdot R}
  \llbracket M + N + C \rrbracket@K \circ L
  \,.
\]
The diagram is a proof of this property,
constructed from the following components:
\begin{itemize}
  \item Layer correctness properties of the form
    \[
      L' \le_{\mathsf{id} \twoheadrightarrow R} \llbracket M \rrbracket@K \circ L
      \,,
    \]
    depicted as rectangular boxes.
  \item The linking property (\ref{eqn:linkingsquare}),
    lifted to operate with abstract state
    \[
      \llbracket M \rrbracket@K \circ \llbracket N \rrbracket@K
      \le_{\mathsf{id} \twoheadrightarrow \mathsf{id}}
      \llbracket M + N \rrbracket@K
      \,,
    \]
    depicted with a plus symbol.
  \item The compatibility of language semantics with abstraction relations
    \[
      \llbracket C \rrbracket@K'
      \le_{R \twoheadrightarrow R}
      \llbracket C \rrbracket@K
      \,,
    \]
    depicted as crossings between the horizontal lines ($R$)
    and vertical lines ($\llbracket C \rrbracket$).
\end{itemize}

%}}}

%}}}

\section{Certified abstraction layers} %{{{

I will use the notations and concepts outlined in
\href{https://certikos.github.io/rbgs-papers/thesis/thesis.pdf\#chapter.4}{Chapter 4}
of my thesis.
In particular,
\href{https://certikos.github.io/rbgs-papers/thesis/thesis.pdf#section.4.4}{\S 4.4}
reframes our CompCertX-based approach
into our more abstract formalism
and is a starting point for the following definitions.

\subsection{Abstract states} %{{{

In CompCertX,
the memory model is extended with an \emph{abstract state} component
which is used to specify the behavior of underlay primitives.

To extend CompCertO in a similar way,
given a set $K$ of abstract states,
we can introduce an operator to transform the language interface $A$
into the language interface $A@K$
where every question and answer
is annotated with an element of $K$.

\begin{definition}
For a language interface
$A = \langle A^\circ, A^\bullet \rangle$
we define the language interface:
\[
    A@K := \langle A^\circ \times K, \: A^\bullet \times K \rangle
    \,.
\]
\end{definition}

Then,
a transition system $L : A \twoheadrightarrow B$
can be lifted to $L@K : A@K \twoheadrightarrow B@K$,
which maintains an abstract state component
and threads it through the computation.

\begin{definition}
For a transition system
$L = \langle S, {\rightarrow}, D, I, X, Y, F \rangle$,
we define:
\[
    L@K \: := \:
      \langle S \times K, \: {\rightarrow} \times {=}_K, \:
            D \times K, \: I \times {=}_K, \: X \times {=}_K, \:
            Y_K, \: F \times {=}_K \rangle
\]
where the relation $Y_K$ is defined by the rule:
\[
  \begin{prooftree}
    \hypo{n \mathrel{Y^s} s'}
    \infer1{n@{k'} \mathrel{Y_K^{s@k}} s'@k'}
  \end{prooftree} 
\]
\end{definition}

Most of the relations involved in $L@K$
simply thread this component through unchanged (${-} \times {=}_K$).
At external calls,
we update the abstract state with
its value in the environment's answer.

A similar construction can be carried out for simulation conventions.

\begin{definition}
For a simulation convention $\mathbb{R} : A \leftrightarrow B$
with $\mathbb{R} = \langle W, \mathbb{R}^\circ, \mathbb{R}^\bullet \rangle$,
we define the simulation convention
$\mathbb{R}@K : A@K \leftrightarrow B@K$
in the following way:
\[
  \mathbb{R}@K \: := \:
    \langle
      W, \:
      \mathbb{R}^\circ \times {=}, \:
      \mathbb{R}^\bullet \times {=}
    \rangle
\]
\end{definition}

Together,
these definitions define a \emph{double endofunctor}
on the double category of
transition systems, simulation conventions and simulations.
The corresponding properties
are given as follows.

\begin{theorem}
  For the transition systems
  $L_1 : A \twoheadrightarrow B$ and
  $L_2 : B \twoheadrightarrow C$,
  we have:
  \[
    \mathsf{id}_A@K \equiv \mathsf{id}_{A@K}
    \qquad
    (g \circ f)@K \equiv g@K \circ f@K
  \]
  For the simulation conventions
  $R : A \leftrightarrow B$ and $S : B \leftrightarrow C$,
  we have:
  \[
    \epsilon_A@K \equiv \epsilon_{A@K}
    \qquad
    (R \cdot S)@K \equiv R@K \cdot S@K
  \]
  Finally,
  extending with abstract state preserves simulation squares:
  \[
    \begin{tikzcd}[row sep=large,column sep=large]
      A_1 \ar[r, ->>, "L_1"] \ar[d, <->, "R_A"'] &
      B_1 \ar[d, <->, "R_B"] \\
      A_2 \ar[r, ->>, "L_2"'] & B_2
    \end{tikzcd}
    \quad \Longrightarrow \quad
    \begin{tikzcd}[row sep=large, column sep=large]
      A_1@K \ar[r, ->>, "L_1@K"] \ar[d, <->, "R_A@K"'] &
      B_1@K \ar[d, <->, "R_B@K"] \\
      A_2@K \ar[r, ->>, "L_2@K"'] & B_2@K
    \end{tikzcd}
  \]
\end{theorem}

These properties essentially mean that
entire simulation diagrams
can be extended with abstract states at once.
For example,
if the following simulations hold:
\[
  \begin{tikzcd}[sep=large]
    A \ar[r, ->>, "f"] \ar[d, <->, "R"] &
    B \ar[r, ->>, "g"] &
    C \ar[r, ->>, "h"] \ar[d, <->, "S"] &
    D \ar[dd, <->, "T"]
    \\
    E \ar[rr, ->>, "x"] \ar[d, <->, "U"] &&
    F \ar[d, <->, "V"] &
    \\
    X \ar[rr, ->>, "\phi"] &&
    Y \ar[r, ->>, "\psi"] &
    Z
  \end{tikzcd}
\]
then we can conclude that the following simulations hold as well:
\[
  \begin{tikzcd}[row sep=large]
    A@K \ar[r, ->>, "f@K"] \ar[d, <->, "R@K"] &
    B@K \ar[r, ->>, "g@K"] &
    C@K \ar[r, ->>, "h@K"] \ar[d, <->, "S@K"] &
    D@K \ar[dd, <->, "T@K"]
    \\
    E@K \ar[rr, ->>, "x@K"] \ar[d, <->, "U@K"] &&
    F@K \ar[d, <->, "V@K"] &
    \\
    X@K \ar[rr, ->>, "\phi@K"] &&
    Y@K \ar[r, ->>, "\psi@K"] &
    Z@K
  \end{tikzcd}
\]
This will be especially useful for lifting
properties established in CompCertO
to the context of certified abstraction layers,
allowing for example versions of the
compiler correctness or linking properties
extended to include abstract state:
\[
  \begin{tikzcd}[row sep=large, column sep=huge]
    \mathcal{C}@K
      \ar[r, ->>, "\llbracket M \rrbracket@K"]
      \ar[d, <->, "\mathbb{C}@K"'] &
    \mathcal{C}@K
      \ar[d, <->, "\mathbb{C}@K"] \\
    \mathcal{A}@K
      \ar[r, ->>, "\llbracket C(M) \rrbracket@K"'] &
    \mathcal{A}@K
  \end{tikzcd}
  \qquad
  \begin{tikzcd}[sep=large]
    \mathcal{C}@K
      \ar[r, ->>, "\llbracket M \rrbracket@K"]
      \ar[d, double, dash] &
    \mathcal{C}@K
      \ar[r, ->>, "\llbracket N \rrbracket@K"] &
    \mathcal{C}@K
      \ar[d, double, dash] \\
    \mathcal{C}@K
      \ar[rr, ->>, "\llbracket M + N \rrbracket@K"'] &&
    \mathcal{C}@K
  \end{tikzcd}
\]


%}}}

\subsection{Layer interfaces} %{{{

Per the definitions in my thesis and our LICS'20 paper,
a layer interface can be described as a family of specifications:
\[
    \sigma^m : K \rightarrow \mathcal{P}^1(N \times K)
\]
where $(m \mathbin: N) \in E$ is an operation of the layer's signature.
In the case of the $\mathcal{C}$ language interface of CompCertO,
operations are of the form:
\[
    f(\vec{v}) : \mathsf{val}
    \qquad
    \text{where}
    \qquad
    f \in \mathsf{val}
    \qquad
    \vec{v} \in \mathsf{val}^*
\]

A layer interface specified in this style
can easily be represented as a CompCertO transition system
$\hat{\sigma} : 1 \twoheadrightarrow \mathcal{C}@K$,
defined as:
\[
  \hat{\sigma} := \langle
    \mathsf{val} \times \mathsf{mem} \times K,
    \varnothing,
    D,
    I,
    \varnothing,
    \varnothing,
    F
  \rangle
\]
A call into this transition system involves a single state.
At invocation,
we immediately query $\sigma$ to obtain the call's outcome
and save it in the transition system's state:
\[
  \begin{prooftree}
    \hypo{\sigma^{f(\vec{v})}(k) \ni (v', k')}
    \infer1{f(\vec{v})@m@k \mathrel{I} (v', m, k')}
  \end{prooftree}
\]
This single state admits no transition but is immediately final:
\[
  \begin{prooftree}
    \infer0{(v', m, k') \mathrel{F} v'@m@k'}
  \end{prooftree}
\]

The domain $D$ has to be specified in addition to $\sigma$,
which does not carry this information.
Alternatively,
we could use a more sophisticated embedding,
where $\sigma$ is defined in terms of a more abstract
effect signature $E$,
and where we specify a correspondance between
the operations of $E$ and $\mathcal{C}$ calls.

Using the definitions above,
the semantics of a Clight program $M$
on top of a layer interface
$L : 1 \twoheadrightarrow \mathcal{C}@K$
can be given as
\[
  \llbracket M \rrbracket @K \circ L \: : \:
  1 \twoheadrightarrow \mathcal{C}@K
  \,.
\]

%}}}

\subsection{Abstraction relations} %{{{

In CompCertX-based CertiKOS,
the abstraction relation between
an overlay with abstract states in $K_1$ and
an underlay with abstract states in $K_2$
is given as a pair of relations:
\[
  R^\mathsf{r} \subseteq K_1 \times K_2
  \qquad
  R^\mathsf{m} \subseteq K_1 \times \mathsf{mem}
\]
We also associate with each layer a set of global variables $G$
such that:
\[
  \begin{prooftree}
    \hypo{k_1 \mathrel{R^\mathsf{r}} m_2}
    \hypo{m_2 \cong_G m_2'}
    \infer2{k_1 \mathrel{R^\mathsf{r}} m_2'}
  \end{prooftree}
\]
where $\cong_G$ denotes the usual $\mathsf{Mem.unchanged\_on}$
relationship asserting that the two memories
associate the same contents to the global variables in $G$.

In CompCertO,
we can use these relations to define a
memory-extension-based simulation convention
$R : \mathcal{C}@K_1 \Leftrightarrow \mathcal{C}@K_2$
which captures CertiKOS-style abstraction between
overlay and underlay behaviors:
\begin{gather*}
 {\begin{prooftree}
    \hypo{k_1 \mathrel{R^\mathsf{r}} k_2}
    \hypo{k_1 \mathrel{R^\mathsf{m}} m_2}
    \hypo{m_1 \le_\mathsf{m} m_2}
    \hypo{m_1 \mathrel\text{no-perms-on} G}
    \hypo{\vec{v}_1 \le_\mathsf{v} \vec{v}_2}
    \infer5{f(\vec{v}_1)@m_1@k_1 \mathrel{R^\circ} f(\vec{v}_2)@m_2@k_2}
  \end{prooftree}}
\\[1em]
 {\begin{prooftree}
    \hypo{k_1 \mathrel{R^\mathsf{r}} k_2}
    \hypo{k_1 \mathrel{R^\mathsf{m}} m_2}
    \hypo{m_1 \le_\mathsf{m} m_2}
    \hypo{m_1 \mathrel\text{no-perms-on} G}
    \hypo{v'_1 \le_\mathsf{v} v'_2}
    \infer5{v'_1@m_1@k_1 \mathrel{R^\bullet} v'_2@m_2@k_2}
  \end{prooftree}}
\end{gather*}
Then we can
formulate the layer correctness property
$L \vdash_R M : L'$ as
\[
    L' \le_{\mathsf{id} \twoheadrightarrow R}
    \mathsf{Clight}(M)@K \circ L
    \,.
\]

%}}}

%}}}

\section{Coherence spaces} %{{{

%}}}

\section{Effect signatures} %{{{

%}}}

\end{document}


\documentclass[acmsmall,screen,review,anonymous,nonacm]{acmart}
\settopmatter{printfolios=true,printccs=false,printacmref=false}
\renewcommand\footnotetextcopyrightpermission[1]{}

% Packages {{{
\usepackage{booktabs}
\usepackage{bbm}
\usepackage{ebproof}
\usepackage{minted}
\usepackage{tikz-cd}
\usepackage{subcaption}
\usepackage{listings}
\usepackage{cmll}
\usepackage{stmaryrd}
\usetikzlibrary{patterns}
\usetikzlibrary{shapes}
\usetikzlibrary{decorations.pathmorphing}
\usetikzlibrary{3d}
\usetikzlibrary{calc}
%}}}

% Parameters {{{
\setcopyright{acmlicensed}
\copyrightyear{2023}
\acmYear{2023}
\acmDOI{XXXXXXX.XXXXXXX}
\acmConference[]{}{}{}
\acmPrice{15.00}
\acmISBN{978-1-4503-XXXX-X/18/06}

\bibliographystyle{ACM-Reference-Format}
\citestyle{acmauthoryear}

\hyphenation{Comp-Cert}
\hyphenation{Comp-CertX}
\hyphenation{Comp-CertO}
\hyphenation{Comp-CertM}
\hyphenation{Certi-KOS}

\ebproofset{
  right label template=\scriptsize\inserttext}

\lstset{
  language=C,
  basicstyle=\ttfamily\footnotesize,
  basewidth=0.5em,
  frame=single,
  numbers=left}

%}}}

% Macros {{{

% Notations {{{
\newcommand{\kw}[1]{\ensuremath{ \mathsf{#1} }}
\newcommand{\ifr}[1]{\mathrel{[{#1}]}}
\newcommand{\que}{\circ}
\newcommand{\ans}{\bullet}
\newcommand{\vref}{\le_\kw{v}}
\newcommand{\mext}{\le_\kw{m}}
\newcommand{\refby}{\preceq}
\newcommand{\scref}{\sqsupseteq}
\newcommand{\screfd}{\sqsubseteq}
\newcommand{\unitset}{\mathds{1}}
\renewcommand{\preceq}{\le}
\newcommand{\intl}[1]{#1^0}
%\newcommand{\caller}[1]{{\rtimes}#1}
%\newcommand{\callee}[1]{{\ltimes}#1}
\newcommand{\caller}[1]{\langle #1 ]}
\newcommand{\callee}[1]{[ #1 \rangle}
\newcommand{\lensarrow}{\leftrightarrows}
\newcommand{\lensle}{\equiv}
\newcommand{\idsc}{\mathbf{id}} % identity simulation convention
\newcommand{\jr}{\mathsf{Y}}
\newcommand{\vcomp}{\fatsemi}
\newcommand{\sepconj}{\oast}
%}}}

% Names of things {{{
\newcommand{\ClightP}{\ensuremath{ \mathsf{ClightP} }}
\newcommand{\Clight}{\ensuremath{ \mathsf{Clight} }}
%}}}

% Custom symbols {{{
\makeatletter
\providecommand*{\cupdot}{%
  \mathbin{%
    \mathpalette\@cupdot{}%
  }%
}
\newcommand*{\@cupdot}[2]{%
  \ooalign{%
    $\m@th#1\cup$\cr
    \hidewidth$\m@th#1\cdot$\hidewidth
  }%
}
\makeatother
%}}}

% String diagrams {{{
\colorlet{sdbg}{lightgray!50!white}
\colorlet{scsdbg}{lightgray!50!white}
\colorlet{tssdbg}{lightgray!50!white}
\colorlet{memsdbg}{ACMLightBlue!50!white}
\colorlet{mmemsdbg}{ACMBlue!50!white}
\colorlet{penvsdbg}{ACMGreen!50!white}

% String diagram picture
\tikzset{sdp/.style={
  x=4mm,
  y=3.5mm,
  z={(1.6mm,1.6mm)}
}}
% String diagram node
\tikzset{sdn/.style={
  draw,
  fill=white,
  shape=rectangle,
  rounded corners,
}}
% Nodes for terminator
\tikzset{bln/.style={
  sdn,
  shape=circle,
  inner sep=1pt,
}}
% Nodes for 3d transition systems
\tikzset{tst/.style={xslant=1,yscale=0.7}}
\tikzset{tsn/.style={sdn,tst}}
% Nodes for 3d simulation conventions
\tikzset{sct/.style={yslant=1,xscale=0.7,yscale=1.2}}
\tikzset{scn/.style={sdn,inner sep=2pt,sct}}
% Active region
\tikzset{act/.style={
  pattern=north west lines,
  opacity=0.33
}}

\newcommand{\companion}{
  node[sct] {\tikz\draw[-Stealth] (0,0);}
}
\newcommand{\conjoint}{
  node[sct,rotate=180] {\tikz\draw[-Stealth] (0,0);}
}

\newcommand{\flatcompanion}{
  node {\tikz\draw[-Stealth] (0,0);}
}
\newcommand{\flatconjoint}{
  node[rotate=180] {\tikz\draw[-Stealth] (0,0);}
}

% }}}

%}}}

\title{Unifying Compositional Verification and Certified Compilation
  with a Three-Dimensional Refinement Algebra}

% Authors {{{

\author{Yu Zhang}
\orcid{0000-0002-0778-3517}
\affiliation{
  \institution{Yale University}
  \city{New Haven}
  \state{CT}
  \country{USA}}
\email{yu.zhang.yz862@yale.edu}

\author{J\'er\'emie Koenig}
\orcid{0000-0002-3168-5925}
\affiliation{
  \institution{Yale University}
  \city{New Haven}
  \state{CT}
  \country{USA}}
\email{jeremie.koenig@yale.edu}

\author{Zhong Shao}
\orcid{0000-0001-8184-7649}
\affiliation{
  \institution{Yale University}
  \city{New Haven}
  \state{CT}
  \country{USA}}
\email{zhong.shao@yale.edu}

\author{Yuting Wang}
\orcid{0000-0003-3990-2418}
\affiliation{
  \institution{Shanghai Jiao Tong University}
  \city{Shanghai}
  \country{China}}
\email{yuting.wang@sjtu.edu.cn}

%}}}

\begin{document}
\newtheorem{remark}[theorem]{Remark}

\begin{abstract} %{{{
Formal verification is a gold standard
for building reliable computer systems.
\emph{Certified} systems in particular
come with a formal specification,
and a proof of correctness
which can easily be checked by a third party.

Unfortunately, verifying large-scale, heterogeneous systems
remains out of reach of current techniques.
Addressing this challenge
will require the use of compositional methods
capable of accommodating and interfacing
a range of program verification and certified compilation techniques.
%enabling the construction of certified systems
%from off-the-shelf certified components.
In principle,
compositional semantics
could play a role in enabling this;
in practice,
existing tools
tend to rely on
simple and specialized
operational models
which are difficult to interface with one another.

%This paper is concerned with bridging this gap.
We present a compositional semantics framework
%designed with complex system verification tasks in mind,
which can accommodate a broad range of verification techniques.
Its core is a three-dimensional algebra of refinement
which operates across program modules,
levels of abstraction, and
components of the system's state.
Our framework is mechanized in the Coq proof assistant
and we showcase its capabilities with multiple use cases.
\end{abstract}

%}}}

\maketitle

\section{Introduction} %{{{

% Preamble {{{

Programming language semantics
make formal verification possible
by providing a mathematical account of program execution.
In particular,
\emph{operational} semantics
are often used as a trusted ``ground truth''
of program behavior,
because they closely mirrors
the mechanical process of computation.

However,
reasoning about programs
directly in terms of their operational semantics
is often difficult because
traditional operational semantics act on a global state.
To reason about a given program,
we must examine for every possible program step
its effect on every component of the state.
Without additional structure,
this can rapidly become intractable.

Fortunately,
compositional proof techniques have been developed
for many kinds of program properties,
making it possible to break down proofs into
localized obligations.
For example,
techniques such as
program logics and logical relations
can be used to establish
correctness against Hoare-style specifications,
contextual equivalence,
and many other properties.
%
%Among these techniques,
%program logics have found
%broad adoption.
Modern program logics can deal with
complex memory layouts,
concurrency,
and sophisticated language features
while supporting
a high degree of automation.
This has allowed practitioners
to verify the correctness of 
increasingly complex algorithms and data structures.

%At the same time,
%traditional program logics
%also suffer limitations which
%restrict their applicability to
%some verification scenarios.
%other kinds of programs and
%other aspects of the software development process.

%}}}

\subsection{The program logic paradigm %{{{
  misses crucial aspects of software development}

%Since algorithms and data structures
%constitute the core of any software system,
%the ability to ensure their reliability
%in a broad variety of programming contexts is invaluable.
%At the same time,
Despite its success,
the paradigm embodied by
traditional program logics
does not account for all aspects of the software development process,
nor does it fully describe
the operation of a typical software artifact.
Concerns outside the scope of a typical program logic include the following:
\begin{itemize}
  \item To be executed,
    verified program components must first be \emph{compiled},
    and this process may compromise
    correctness results obtained at the source level.
  \item Operational semantics and program logics
    are typically designed for a single language, but
    many programs are built from components written
    in \emph{multiple languages}.
  \item Programs such as network servers and clients
    are algorithmically simple
    but conduct complex \emph{external interactions},
    which program logics rarely model or take into account.
\end{itemize}
The following example illustrates some of these limitations.

\begin{figure} % fig:readwritehello {{{
\centering
\begin{minipage}{.35\textwidth}
%\begin{lstlisting}[title={secret.c}]
%#include <unistd.h>
%char msg[] = "uryyb, jbeyq!\n";
%int main()
%{
%        write(1, msg, sizeof msg - 1);
%        return 0;
%}
%\end{lstlisting}
\begin{lstlisting}[title={secret.s}]
.globl main
main:   pushl $13
        pushl $msg
        call rot13
        pushl $1
        call write
        addl $12, %esp
        movl $0, %eax
        ret
.data
msg:    .string "hello, world!\n"
\end{lstlisting}
\vspace{1em}
\begin{lstlisting}[language=sh]
$ cc -o secret secret.s rot13.c
$ ./secret
uryyb, jbeyq!
$ cc -o decode decode.c rot13.c
$ ./secret | ./decode
hello, world!
\end{lstlisting}
\end{minipage}
\hspace{2.5em}
\begin{minipage}{.48\textwidth}
\begin{lstlisting}[title={rot13.c}]
void rot13(char *buf, int len)
{
  for (int i = 0; i < len; i++)
    if ('a' <= buf[i] && buf[i] <= 'z')
      buf[i] = (buf[i] - 'a' + 13) % 26 + 'a';
}
\end{lstlisting}
\begin{lstlisting}[title={decode.c}]
#include <unistd.h>
extern void rot13(char *, int);
int main()
{
        char buf[100];
        int n = read(0, buf, sizeof buf);
        rot13(buf, n);
        write(1, buf, n);
        return 0;
}
\end{lstlisting}
\end{minipage}
\caption{Two programs which use a common library
  are compiled and made to
  interact through a pipe.}
\label{fig:readwritehello}
\end{figure}
%}}}

\begin{example} \label{ex:readwritehello} %{{{
The code shown in \autoref{fig:readwritehello}
consists of two different programs
which are designed to work together and
use a common C library.
As illustrated in the usage scenario we have shown,
the 32-bit x86 assembly program \kw{secret.s} 
outputs a coded message
to be decyphered by \kw{decode.c}.
That is,
together the programs satisfy the following informal specification:
\begin{equation}
  \begin{minipage}{.9\textwidth}
  \it
  Suppose that,
  after compilation,
  \kw{secret.s} and \kw{decode.c}
  are each linked with \kw{rot13.c}.
  If the output of the first program
  is fed as input to the second,
  ``hello, world!'' will be displayed.
  \end{minipage}
  \label{eqn:hellospec}
\end{equation}

The programs are simple;
to verify that property (\ref{eqn:hellospec}) holds,
a reader with the right background
can mentally execute the code step by step
and convince themself that the programs will work as expected.
However, this task is complex in its own way
because it mobilises implicit knowledge and assumptions regarding
the C and x86 assembly languages,
the compiler's correctness with respect to the calling convention in use,
and some aspects of the Unix execution environment.
%
Likewise, any formal account of property (\ref{eqn:hellospec})
must involve these aspects of the problem as well,
encompassing all three of the challenges outlined at the beginning of this section.
%Property~(\ref{eqn:hellospec})
%also involves \emph{two different} programs.
To our knowledge,
there exists no program logic or verification framework
which can deal with this example.
\end{example}
%}}}

A fair amount of work has sought to address
the limitations outlined above.
For example,
the certified compiler CompCert \citep{compcert}
comes with a mechanized proof of correctness.
Better yet,
the Verified Software Toolchain (VST) \citep{vst}
provides a separation logic which interfaces with
the correctness proof of CompCert,
ensuring that properties obtained for C programs
can be formally transferred to the compiled assembly code.
In a further experiment,
a network server was verfied
by incorporating \emph{interaction trees} into VST
to model external interactions \cite{itrees}.
Operational semantics \cite{opsem-multi} and
program logics \cite{melocoton}
have also been developed for multi-language programs.

These efforts show that overcoming the limitations of
the program logic paradigm is possible,
but they constitute one-off adaptations to specific settings:
a particular specification logic,
set of interaction patterns,
language or combination thereof,
\emph{etc}.
To apply this methodology to Example~\ref{ex:readwritehello},
we would need to first develop a semantics and logic 
tailored to the situation at hand.
The result would be unlikely to apply directly
to another verification task.

%}}}

\begin{figure} % fig:paradigms {{{
  \small
  \[
    \begin{tikzcd}[column sep=-1em, row sep=small]
      &
      \begin{array}{c} \text{program} \\ \text{logic} \end{array}
      \ar[ddr, leftrightarrow] &
      \begin{array}{c} \text{logical} \\ \text{relation} \end{array}
      \ar[dd, leftrightarrow] &
      \begin{array}{c} \text{compositional} \\ \text{semantics} \end{array}
      \ar[ddl, leftrightarrow]
      \\
      {} \ar[rrrr, dotted, dash] &&&& {}
      \\
      & &
      \begin{array}{c} \text{operational} \\ \text{semantics} \end{array}
    \end{tikzcd}
    \qquad
    \begin{tikzcd}[column sep=-2.5em, row sep=tiny]
      &
      \begin{array}{c} \text{manual} \\ \text{proof} \end{array}
      \ar[ddr, leftrightarrow] &&
      \begin{array}{c}
        \text{compiler correctness} \\
        \text{and related results}
      \end{array}
      \ar[ddl, leftrightarrow]
      %\ar[ddd, leftrightarrow]
      \ar[ddr, leftrightarrow] &&
      \begin{array}{c} \text{program} \\ \text{logic} \end{array}
      \ar[ddl, leftrightarrow]
      \\
      {} \ar[rrrrrr, dotted, dash] &&&&&& {}
      \\
      &&
      \begin{array}{c}
        \text{compositional} \\
        \text{semantics}
      \end{array}
      \ar[dr, leftrightarrow, bend right]
      &&
      \hspace{-2em}
      \begin{array}{c}
        \text{compositional} \\
        \text{semantics}
      \end{array}
      \ar[dl, leftrightarrow, bend left]
      \\
      &&&
      \begin{array}{c}
        \text{environment} \\ \text{model}
      \end{array}
    \end{tikzcd}
  \]
  \caption{
    Approaches to program verification.
    The system being verified
    is modeled using the facilities shown below the line,
    and the techniques shown above
    are used to reason about its properties.
    Traditionally (left),
    the whole universe in which the computation occurs
    must be modeled in a monolithic and closed operational semantics.
    By using compositional semantics instead (right),
    both the model and reasoning techniques
    can be constructed out of reusable building blocks
    and adapted to various contexts and situations.
  }
  \label{fig:paradigms}
\end{figure}
%}}}

\subsection{Compositional semantics offer a more flexible approach} %{{{

The main difficulty with Example~\ref{ex:readwritehello}
is that formalizing property (\ref{eqn:hellospec})
in an existing framework
would require adjustments to the \emph{model}
within which we consider the behavior of
the programs \kw{secret.s} and \kw{decode.c}.
This is difficult to achieve
in frameworks based on traditional operational semantics
because they model a closed universe,
relying on \emph{proof techniques} for compositionality.

As illustrated in Fig.~\ref{fig:paradigms},
this state of affairs can be improved by
the use of compositional \emph{semantics}.
By their nature,
compositional semantics focus on
the behavior of open components and
their interactions with each other
and with the broader environment.
As a result,
they are more likely to be suitable
building blocks for
modeling complex, heterogeneous systems.

Recent work embracing this paradigm shows promising results.
For example,
whereas prior CompCert research
largely focuses on compositional proof techniques,
the work on CompCertO \cite{compcerto}
%,
%which identifies modeling heterogeneous systems
%as an important motivation,
shows that formulating the compiler's correctness result
directly in terms of a compositional semantics
is possible with a reasonable proof effort.
Likewise,
the DimSum framework \cite{dimsum}
successfully employs this approach to tackle
multi-language semantics and verification:
the framework can be used to stitch together
independent semantics for individual languages,
and to reason about refinement within and across
these languages.

%Historically,
%compositional semantics
%have been understood
%in relation to operational semantics
%as a way to provide
%a high-level, abstract representation
%for partial programs
%more amenable to reasoning.
%In that sense,
%compositional semantics fits into
%the traditional verification framework
%as a reasoning technique
%more than a model.
%In particular,
%traditional domain-theoretic semantics
%tend to provide a very high-level view
%of program behavior,
%abstracting away low-level details such as
%concrete state, execution steps,
%or even sometimes choices of evaluation strategies.
%However,
%the compositional semantics can be connected back to
%the more physical description offered by operational semantics
%through a \emph{full abstraction} theorem,
%which guarantees that
%program equivalence properties
%established using the compositional semantics
%reflect an operational notion of observational equivalence.
%Therefore in this context,
%full abstraction plays a role similar to
%that of program logic soundness.
%
%However,
%nothing in principle prevents us from using
%compositional semantics directly
%as a model for program execution.
%This is especially true
%in the context of approaches such as game semantics,
%which while compositional
%provide a very concrete and fairly low-level view
%of the computation process.
%Moreover,
%whereas operational semantics
%models a closed universe,
%by nature compositional semantics
%emphasize the open-ended nature of program components
%and the ways in which simple components interact
%to form a more complex system.
%This make it easier to model the ways in which
%even complete programs
%are embedded within a large execution environment.

At the same time,
%as we outline below,
compositional semantics remains
underdeveloped as a practical tool for verification,
and lacks a proper treatments of
many techniques which are routine in the context of
operational semantics and program logics.
%As we will see,
%many these techniques can fruitfully be adapted
%to the setting of compositional semantics,
%making it possible to tackle complex verification problems
%in a flexible manner.

%}}}

\subsection{Three dimensions of compositionality} %{{{

We will distinguish between several \emph{kinds} of compositionality which
semantic models, program logics, refinement frameworks
and other formal reasoning tools
can exhibit:
\begin{itemize}
\item \emph{Horizontal compositionality}
  refers to the ability to decompose behaviors and proofs
  along the structure of program.
  For example, denotational semantics are compositional in this sense.
  Likewise,
  the \emph{sequence} rule of Hoare logic
  is a horizontal composition principle.
\item \emph{Vertical compositionality} allows stepwise reasoning,
  as with transitive refinement and data abstraction mechanisms.
  For example,
  compiler correctness proofs make use of vertical compositionality
  when they combine correctness proofs for individual compilation phases.
\item \emph{Spatial compositionality} operates across the system state.
  This is the kind of compositionality enabled in separation logic
  by the separating conjunction $\ast$.
\end{itemize}
One barrier to the use of semantics
along the lines of Fig.~\ref{fig:paradigms}b
is that while
\emph{horizontally} compositional semantics
are a well-developed area of research,
there is comparatively less work investigating models which are
vertically and spatially compositional,
let alone the combination of all three.

%}}}

\subsection{Contributions} %{{{

This paper seeks to showcase and demonstrates the viability of
the paradigm illustrated in Fig.~\ref{fig:paradigms}b.

We present a generic semantic model---%
based on effect signatures and formulated in the style game semantics---%
which combines horizontal, vertical and spatial composition principles.
This model is flexible enough
to express the CompCertO semantics of C and assembly programs,
and to model the kind of process behavior and communication
required to handle Example~\ref{ex:readwritehello}.
%and to formalize property (\ref{eqn:hellospec}).

The multiple dimensions of compositionality
allow us to account for sophisticated reasoning techniques
such as data abstraction and memory separation,
and to capture---%
under a uniform notion of refinement---%
properties as varied as:
\begin{itemize}
  \item program correctness results;
  \item the semantics preservation theorem of CompCertO;
  \item the frame property of separation logic;
  \item representation independence for encapsulated state.
\end{itemize}
As we demonstrate with several examples,
these properties can then be combined
along the multiple dimensions
to construct sophisticated refinement proofs
of statements such as (\ref{eqn:hellospec}).

We present a high-level overview of our work in \S\ref{sec:overview}.
A technical description of our model is given in \S\ref{sec:model}.
We then showcase in \S\ref{sec:app} several uses of our model
beyond the aforementioned examples:
\begin{itemize}
  \item We explain in detail how
    CompCertO semantics and simulation proofs can be embedded,
    and model the \emph{loading} mechanism
    which turns an open program into a closed process;
  \item We give an account of \emph{secure compilation} in the new model and
    characterize a class of noninterference properties \emph{up to observation}
    which are already preserved by CompCert;
  \item We define an extension of CompCert's Clight language
    supporting encapsulated, module-local private variables
    and provide a correctness proof for the erasure of private annotations.
  \item We use our model to define a framework for
    \emph{certified abstraction layers} (CAL) \cite{popl15}.
    Unlike the original work on CAL,
    our layer framework does not modify the underlying compiler,
    and its meta-theory requires comparatively negligible effort.
\end{itemize}
We discuss related work in \S\ref{sec:rw}
and our conclusions in \S\ref{sec:conclusion}.
Our work has been mechanized in the Coq proof assistant
and we have submitted a preliminary artifact alongside this paper.

%}}}

%}}}

\section{Compositional Semantics for Verification} \label{sec:overview} %{{{

We will first describe the structure of our framework,
and illustrate it using a variety of examples.

\subsection{Overview} %{{{

Our framework consists of four kinds of objects,
each subject to some or all of four
different composition principles
(layered $\odot$,
 vertical $\fatsemi$,
 flat $\oplus$,
 spatial $\mathbin@$).
Before delving into any details,
we give a brief overview of how they fit together.

\paragraph{Semantic objects} %{{{

We introduce in \S\ref{sec:esig} the notion of \emph{effect signature} ($E, F\ldots$).
Signatures model interfaces between the components of a software system.
They serve
as horizontal endpoints for the \emph{strategies}
($L : E \rightarrow F$)
which model component behaviors, and
as vertical endpoints for the \emph{simulation conventions}
($\mathbf{R} : E_1 \leftrightarrow E_2$)
which model relationships between
views of the system at different levels of abstraction.
Finally,
\emph{refinements up to conventions}
($\phi : L_1 \le_{\mathbf{R} \rightarrow \mathbf{S}} L_2$)
involve the three kinds of objects above,
connecting them in the shape of a square (Fig.~\ref{fig:hvcomp}).

%}}}

\paragraph{Composition principles} %{{{

Refinement squares are the building block of compositional proofs
within our framework.
They can be assembled in the manner of puzzle pieces
alongside matching edges:
\begin{itemize}
\item Layered composition ($\odot$) acts horizontally.
  It connects strategies at a common endpoint (\emph{ie.}~effect signature)
  over which they are made to interact,
  and connects refinement squares alongside a common
  vertical edge (\emph{ie.}~simulation convention),
  which ensures that the refinement properties
  are based on compatible assumptions.
  %assumptions of one refinement proof
  %are established as guarantees by the other.
\item Vertical composition ($\fatsemi$)
  combines successive refinement steps,
  connecting simulation conventions alongside an intermediate signature,
  and refinement squares along a common strategy,
  which serves as an intermediate specification.
\end{itemize}
This basic framework is made more expressive
by the introduction of two additional forms of composition,
which coherently act on all objects from effect signatures to refinement squares:
\begin{itemize}
\item Flat composition ($\oplus$)
  serves as an alternative form of horizontal composition
  where components are laid out side by side
  instead of being made to interact.
\item Spatial composition ($\mathbin@$)
  equips our framework with a sophisticated infrastructure
  for compositional state.
\end{itemize}
We examine each of these constructions below.

%}}}

%}}}

\begin{figure} % fig:hvcomp {{{
  \[
  \begin{array}{c}
    \begin{tikzcd}[sep=0.5ex]
      E_1 \ar[dd, leftrightarrow, "\mathbf{R}"']
	  \ar[rr, "L_1"] &&
      F_1 \ar[dd, leftrightarrow, "\mathbf{S}"] \\
      & \phi & \\
      E_2 \ar[rr, "L_2"'] &&
      F_2
    \end{tikzcd}
  \end{array}
  \quad
  \begin{array}{c}
    \begin{prooftree}
      \hypo{L_1 : F \twoheadrightarrow G}
      \hypo{L_2 : E \twoheadrightarrow F}
      \infer2[\kw{ts}-$\odot$]{L_1 \odot L_2 : E \twoheadrightarrow G}
    \end{prooftree}
    \qquad
    \begin{prooftree}
      \hypo{\phi: L_1 \le_{\mathbf{S} \twoheadrightarrow \mathbf{T}} L_1'}
      \hypo{\psi: L_2 \le_{\mathbf{R} \twoheadrightarrow \mathbf{S}} L_2'}
      \infer2[\kw{sim}-$\odot$]{\phi \odot \psi :
	L_1 \odot L_2 \le_{\mathbf{R} \twoheadrightarrow \mathbf{T}} L_1' \odot L_2'}
    \end{prooftree}
    \\[1.5em]
    \begin{prooftree}
      \hypo{\mathbf{R} : E_1 \leftrightarrow E_2}
      \hypo{\mathbf{R}' : E_2 \leftrightarrow E_3}
      \infer2[\kw{sc}-$\vcomp$]{\mathbf{R} \vcomp \mathbf{R}' : E_1 \leftrightarrow E_3}
    \end{prooftree}
    \qquad
    \begin{prooftree}
      \hypo{\phi : L_1 \le_{\mathbf{R} \twoheadrightarrow \mathbf{S}} L_2}
      \hypo{\psi : L_2 \le_{\mathbf{R'} \twoheadrightarrow \mathbf{S'}} L_3}
      \infer2[\kw{sim}-$\vcomp$]{\phi \vcomp \psi : L_1 \le_{\mathbf{R} \vcomp \mathbf{R'} \twoheadrightarrow
	\mathbf{S} \vcomp \mathbf{S'}} L_3}
    \end{prooftree}
  \end{array}
\]
  \caption{Horizontal ($\odot$) and vertical ($\vcomp$)
    composition principles in our model.}
  \label{fig:hvcomp}
\end{figure}
%}}}

\subsection{Effect Signatures} \label{sec:esig} %{{{

Like interaction trees \cite{itrees},
our model uses \emph{effect signatures}
%\citep{some-algebraic-effects-ref}
to describe interfaces between the components of a system.
An effect signature simply enumerates
the external operations which
a component can invoke or implement,
and describes for each one
the set of possible outcomes.

\begin{definition} \label{def:esig}
An \emph{effect signature} is a set $E$ of \emph{questions}
together with an assignment $\kw{ar} : E \rightarrow \mathbf{Set}$
associating to each question $m \in E$
a set of \emph{answers} $n \in \kw{ar}(m)$.
We will often present them together as the set of bindings
$\{ (m \mathbin: N) \mid m \in E \wedge N = \kw{ar}(m_i) \}$.
%$\{ m \mathbin: \kw{ar}(m) \}_{m \in E}$.
\end{definition}

In the algebraic effects literature,
a question $(m \mathbin: N)$ in a signature $E$ is interpreted as
an algebraic operation of arity $N$, representing a possible effect.
The term $m(k_i)_{i \in N}$ is then a computation
which triggers the effect $m$ with possible continuations $k_i$.
Our terminology
%of \emph{questions} and \emph{answers}
follows more closely the game semantics tradition,
reflecting the fact that
we model only cross-component interactions.

\begin{example} \label{ex:fdsig} %{{{
Consider the execution environment
for the programs \kw{secret} and \kw{encode}
shown in \autoref{fig:readwritehello} and
described in Example~\ref{ex:readwritehello}.
Since our programs
do not use any command-line arguments or environment variables,
we can model their invocation with a single question:
\[
  \mathcal{P} \, := \,
    \{ \kw{run} : \mathbb{N} \}
  \,.
\]
The answer $x \in \mathbb{N}$ is the exit status of the process.
%which in our cases is expected to be zero.
Moreover, in the course of its execution
each process can invoke the \kw{read} and \kw{write} system calls.
We can describe this interface with the signature
\[
  \mathcal{S} \, := \, \{
    \kw{read}_i[n] \mathbin: \Sigma^* , \,
    \kw{write}_i[s] \mathbin: \mathbb{N} \, \mid \,
    i \in \mathbb{N}, \,
    n \in \mathbb{N}, \,
    s \in \Sigma^*
  \}
  \,,
\]
where $\Sigma := \{0,1\}^8$ is the alphabet of possible byte values.
In this formalism,
the program \kw{secret} will invoke
the operation
$\kw{write}_1[\texttt{"uryyb, jbeyq!\textbackslash{}n"}]
 \in \mathcal{S}$,
where $i := 1$ is the file descriptor associated with the standard output;
the outcome should be $14 \in \mathbb{N}$
if the operation succeeds.
\end{example}
%}}}

\begin{example}[CompCertO language interfaces] \label{ex:compcertosig} %{{{
The semantic model of CompCertO uses \emph{language interfaces}
of the form $A = \langle A^\circ, A^\bullet \rangle$
as the basis for component interactions.
These interfaces are similar to effect signatures,
but every question $q \in A^\circ$ uses the same set of answers $r \in A^\bullet$.
%In addition, CompCertO components are parameterized by
%a symbol table $\mathit{se} \in \kw{senv}$
%which defines a mapping between program identifiers
%and memory addresses.

For the C language,
questions are function calls of the form $f(\vec{v})@m$, where
$f$ identifies the function to be called,
$\vec{v} \in \kw{val}^*$ are the actual parameters, and
$m \in \kw{mem}$ is the memory state at the time of invocation;
answers take the form $v'@m'$ where
$v' \in \kw{val}$ is value returned by the function~$f$ and
$m' \in \kw{mem}$ is the new state of the memory.
This is captured by the effect signature
\[
  \mathcal{C} \mathbin@ \kw{mem} \:=\:
  \{ f(\vec{v})@m \mathbin: \kw{val} \times \kw{mem} \mid
     f \in \kw{val}, \:
     \vec{v} \in \kw{val}^*, \:
     m \in \kw{mem} \}
  \,.
\]
We will see that CompCertO language interfaces
can be systematically mapped to effect signatures,
and will elucidate the structure of
the spatial decomposition $\mathcal{C} \mathbin@ \kw{mem}$ below.
\end{example}
%}}}

%}}}

\begin{figure} % fig:overview:ts {{{
  \begin{subfigure}{0.35\textwidth}
    \centering
    \begin{tikzpicture}[yscale=0.15,xscale=0.30]
      \draw (1,-1) rectangle (5,11) node[midway] {$L$};
      \scriptsize
      \draw[->] (0,10) node[left] {$q \in F$} -- (1,10)
          node[above=1.5em,midway] (B) {\normalsize $F$};
        \draw[->] (5,10) -- (6,10) node[right] {$q_1 \in E$}
          node[above=1.5em,midway] (A) {\normalsize $E$};
        \draw[->] (6,8) node[right] {$r_1 \in \kw{ar}(q_1)$} -- (5,8) ;
        \node[right] at (6,5.5) {$\:\vdots$};
        \draw[->] (5,2) -- (6,2) node[right] {$q_n \in E$};
        \draw[->] (6,0) node[right] {$r_n \in \kw{ar}(q_n)$} -- (5,0);
      \draw[->] (1,0) -- (0,0) node[left] {$r \in \kw{ar}(q)$};
      \draw[->>] (A) -- (B);
    \end{tikzpicture}
    \subcaption{General shape}
    \label{fig:overview:ts:shape}
  \end{subfigure}
  \begin{subfigure}{0.30\textwidth}
    \centering
    \begin{tikzpicture}[yscale=0.15,xscale=0.30]
      \draw (1,-1) rectangle (4,11) node[midway] {$L_1$};
      \draw (5,-1) rectangle (8,11) node[midway] {$L_2$};
      %\draw (5,-1) rectangle (8,4) node[midway] {$L_2$};
      \draw[->] (0,10) -- (1,10) node[above=1em,midway] (C) {$G$};
        \draw[->] (4,10) -- (5,10) node[above=1em,midway] (B) {$F$};
          \draw[->] (8,10) -- (9,10) node[above=1em,midway] (A) {$E$};
          \draw[->] (9,9) -- (8,9);
          \draw[->] (8,8) -- (9,8);
          \draw[->] (9,7) -- (8,7);
        \draw[->] (5,7) -- (4,7);
        \draw[->] (4,3) -- (5,3);
          \draw[->] (8,3) -- (9,3);
          \draw[->] (9,2) -- (8,2);
          \draw[->] (8,1) -- (9,1);
          \draw[->] (9,0) -- (8,0);
        \draw[->] (5,0) -- (4,0);
      \draw[->] (1,0) -- (0,0);
      \draw[->>] (A) -- (B);
      \draw[->>] (B) -- (C);
    \end{tikzpicture}
    \subcaption{Composition}
    \label{fig:overview:ts:comp}
  \end{subfigure}
  \begin{subfigure}{0.25\textwidth}
    \centering
    \begin{tikzpicture}[yscale=0.15,xscale=0.30]
      \draw (5,-1) rectangle (8,4) node[midway] {$\kw{id}_E$};
      \draw[->] (4,3) node[left] {\scriptsize $q \in E$}
             -- (5,3) node[above=2em,midway] (A2) {$E$};
        \draw[->] (8,3) -- (9,3) node[above=2em,midway] (A1) {$E$}
                                 node[right] {\scriptsize $q$};
        \draw[->] (9,0) node[right] {\scriptsize $r \in \kw{ar}(q)$} -- (8,0);
      \draw[->] (5,0) -- (4,0) node[left] {\scriptsize $r$};
      \draw[->>] (A1) -- (A2);
    \end{tikzpicture}
    \vspace{1ex}
    \subcaption{Identity}
    \label{fig:overview:ts:id}
  \end{subfigure}
  \caption{Informal description of CompCertO's transition systems
    under \emph{layered} composition}  
  \label{fig:overview:ts}
\end{figure}
%}}}

\subsection{Strategies} \label{sec:strat} %{{{

We use effect signatures to assign coarse types
to program components and to their behaviors.
Specifically,
a~\emph{strategy} $L : E \rightarrow F$
models the behavior of a component which
uses operations of the signature $E$ to
implement the operations enumerated in $F$.

As depicted in Fig.~\ref{fig:overview:ts}a,
the environment can activate $L$ by asking a question $q \in F$,
which the component $L$ is expected to eventually answer
with a reply $r \in \kw{ar}(q)$.
In the process,
$L$ can perform an arbitrary number of queries $q_i \in E$,
which the environment answers with a response $r_i \in \kw{ar}(q_i)$.
The process can then begin anew with a question $q' \in F$,
and so on indefinitely.
We will write
\[
  L \:\vDash\: \big(q
    \rightarrowtail (q_1 \leadsto r_1)
    \rightarrowtail (q_2 \leadsto r_2)
    \rightarrowtail \cdots
    \rightarrowtail (q_n \leadsto r_n)
    \rightarrowtail r \big)
  \leadsto \big(q'
    \rightarrowtail \cdots
    \rightarrowtail r' \big)
  \leadsto \cdots
\]
to mean that $L$ accepts an execution trace of this kind.
Note that $\rightarrowtail$ denotes a part of the execution
where $L$ is in control,
whereas $\leadsto$ denotes a part of the execution
controlled by the environment.

\begin{example}[A specification for the program $\kw{decode}$] \label{ex:decodespec} %{{{
We use a component $\Sigma_\kw{decode} : \mathcal{S} \rightarrow \mathcal{P}$
to formulate a specification for the program $\kw{decode}$.
The component admits the execution trace
{\small\[
  \Sigma_\kw{decode} \vDash \kw{run}
    \rightarrowtail (\kw{read}_0[100] \leadsto \texttt{"uryyb, jbeyq!\textbackslash{}n"})
    \rightarrowtail (\kw{write}_1[\texttt{"hello, world!\textbackslash{}n"}] \leadsto 14)
    \rightarrowtail 0
  \,.
\]}
\end{example}
%}}}

\begin{example}[CompCertO semantics] \label{ex:compcertosem} %{{{
We explained in Example~\ref{ex:compcertosig} that
CompCertO language interfaces can be translated to effect signatures,
and showed the signature $\mathcal{C} \mathbin@ \kw{mem}$
used for C-level function calls and returns.
By the same token, CompCertO language semantics
can be translated to strategies as well.
For example,
the source language used by CompCertO
is a simplified version of~C called Clight,
and its semantics for a program $M$ can be used to define:
\[
  \kw{Clight}(M) : \mathcal{C} \mathbin@ \kw{mem}
    \twoheadrightarrow \mathcal{C} \mathbin@ \kw{mem}
  \,.
\]
The resulting transition systems will exhibit traces such as:
{\scriptsize
\begin{align*}
  \kw{Clight}(\kw{decode.c}) \:\vDash\:
  \kw{main}()@m &\rightarrowtail
  (\kw{read}(0, b, 100)@m[b \mapsto \textit{unspecified}] \leadsto
   14@m[b \mapsto \texttt{"uryyb, jbeyq!\textbackslash{}n"}]) \\& \rightarrowtail
  (\kw{rot13}(b, 14)@m[b \mapsto \texttt{"uryyb, jbeyq!\textbackslash{}n"}] \leadsto
   {*}@m[b \mapsto \texttt{"hello, world!\textbackslash{}n"}]) \\& \rightarrowtail
  (\kw{write}(1, b, 14)@m[b \mapsto \texttt{"hello, world!\textbackslash{}n"}] \leadsto
    14@m[b \mapsto \texttt{"hello, world!\textbackslash{}n"}]) \\& \rightarrowtail
  0@m[b \mapsto \textit{deallocated}]
\end{align*}
}%
This trace is more complicated than the one shown in Example~\ref{ex:decodespec};
among other things it involves low-level considerations regarding
the C memory model.
Nevertheless,
we will eventually use the CompCertO semantics of C and assembly
as a building block to model the scenario in Example~\ref{ex:readwritehello},
and connect them to the kind of high-level specifications we have seen so far.
\end{example}
%}}}

\begin{color}{gray}
\paragraph{Refinement Ordering} %{{{

In our model,
components can exhibit undefined behavior ($\bot$),
make use of angelic nondeterminism ($\vee$),
or perform an infinite number of queries in $E$.
To make this possible,
their behavior is represented coalgebraically as state transition systems,
and we define an associated notion of simulation.
Given the transition systems $L_1, L_2 : E \rightarrow F$,
a simulation is defined by a relation $\rho$ between their sets of states
which satisfies certain behavior preservation properties (Def.~\ref{def:sim}).
The existence of such a relation
proves that $L_2$ is \emph{more defined} than $L_1$,
and admits at least the same behaviors.
We will write
\[
  \rho : L_1 \le_{E \rightarrow F} L_2 \, , \qquad
  \text{or} \quad \rho : L_1 \le L_2 \, , \qquad
  \text{or just} \quad L_1 \le L_2
\]
when such a relation exists.
Since equality is a simulation relation
and simulation relations compose,
the relation $\le$ defined above is a preorder:
\[
  \epsilon_L : L \le L
  \qquad \qquad
  \begin{prooftree}
    \hypo{\rho : L_1 \le L_2}
    \hypo{\phi : L_2 \le L_3}
    \infer2{\rho \vcomp \phi : L_1 \le L_3}
  \end{prooftree}
\]
We take a ``proof irrelevant'' approach
and treat simulations of the same type as equal.
We will write $L_1 \equiv L_2$ when
both $L_1 \le L_2$ and $L_2 \le L_1$ hold
and consider transition systems \emph{up to} $\equiv$.

%}}}

\begin{example} \label{ex:decodesim} %{{{
We would like to show that the program $\kw{decode}$
satisfies the specification $\Sigma_\kw{decode}$
given in Example~\ref{ex:decodespec}.
To this end,
we will need to model the way in which
$\kw{decode.c}$ and $\kw{rot13.c}$ behave together
\emph{as a process}.
Assuming for now that such a model
$\llbracket \kw{decode} \rrbracket :
 \mathcal{S} \rightarrow \mathcal{P}$
exists,
our goal will be to establish a simulation
$
  \Sigma_\kw{decode}
  \le
  \llbracket \kw{decode} \rrbracket
$.
The model will involve CompCertO semantics
and will take into account the way the program is
compiled, linked and loaded.
%As a first step,
%we will show how to account for the source-level interaction
%between the components $\kw{decode.c}$ and $\kw{rot13.c}$.
\end{example}
%}}}
\end{color}

\paragraph{Layered Composition} %{{{

When a component $L_1 : F \rightarrow G$
uses an interface $F$ implemented by
another component $L_2 : E \rightarrow F$,
we can direct the questions asked by $L_1$ in $F$ to $L_2$
(Fig.~\ref{fig:overview:ts}b).
The result is the composite transition system
$L_1 \odot L_2 : E \rightarrow G$. %(Def.~X.Y).
This is the main form of horizontal composition which we will be using.

\begin{figure} % fig:code {{{
  \centering\footnotesize
  \begin{subfigure}{0.45\textwidth}
\begin{minted}{C}
static int c1, c2;
static V buf[N];

int inc1() { int i = c1++; c1 %= N; return i; }
int inc2() { int i = c2++; c2 %= N; return i; }
V get(int i) { return buf[i]; }
void set(int i, V val) { buf[i] = val; }
\end{minted}
  \subcaption{The translation unit $\kw{rb.c}$}
  \label{fig:rb}
  \end{subfigure}
  \hspace{4em}
  \begin{subfigure}{0.38\textwidth}
\begin{minted}{C}
extern int inc1(void);
extern int inc2(void);
extern V get(int i);
extern void set(int i, V val);

void enq(V val) { set(inc2(), val); }
V deq() { return get(inc1()); }
\end{minted}
  \subcaption{The translation unit $\kw{bq.c}$}
  \label{fig:bq}
  \end{subfigure}
  \caption{Running example, adapted from \citet{rbgs-cal}.
    The component $\kw{rb.c}$
    implements a ring buffer by encapsulating an array
    and two counters. It is used by the component
    $\kw{bq.c}$ to implement a
    bounded queue.}
  \label{fig:code}
%\caption{The state of a ring buffer,
%  made of two counters and a fixed-size array,
%  is encapsulated behind a simple interface.}
%\label{fig:rb}
%\caption{This component relies on the ring buffer primitives
%  provided in Fig.~\ref{fig:rb} to implement a bounded-size queue.}
%\label{fig:bq}
\end{figure}
%}}}

\begin{example}[Verifying a bounded queue] \label{ex:bq} %{{{
The code shown in Fig.~\ref{fig:code}
implements a bounded queue
with at most $N$ values of type $V$.
This is done in two steps.
The translation unit $\kw{rb.c}$
provides access to a ring buffer
in the form of an array as well as
two counters which wrap around
to stay in the interval $[0, N)$.
The translation unit $\kw{bq.c}$ then uses that interface
to implement the queue.
We can describe this situation using
high-level specifications with the following types:
\begin{align*}
  \Sigma_\kw{rb} &: \top \rightarrow E_\kw{rb}
  &
  E_\kw{rb} &:= \{
    \kw{inc}_1 \mathbin: \mathbb{N}, \,
    \kw{inc}_2 \mathbin: \mathbb{N}, \,
    \kw{get}[i] \mathbin: V, \,
    \kw{set}[i, v] \mathbin: \mathbbm{1} \, \mid \,
    i \in \mathbb{N}, \, v \in V
  \}
  \\
  \Sigma_\kw{bq} &: E_\kw{rb} \rightarrow E_\kw{bq}
  &
  E_\kw{bq} &:= \{
    \kw{enq}[v] \mathbin: \mathbbm{1}, \,
    \kw{deq} \mathbin: V \, \mid \,
    v \in V
  \}
\end{align*}
The specifications $\Sigma_\kw{rb}$ and $\Sigma_\kw{bq}$
will admit interaction traces such as:
{\small \begin{align*}
  \Sigma_\kw{rb} \vDash {} &
    (\kw{inc}_2 \rightarrowtail 0) \leadsto
    (\kw{set}[0, v] \rightarrowtail {*}) \leadsto &
  \Sigma_\kw{bq} \vDash {} &
    (\kw{enq}[v] \rightarrowtail
      (\kw{inc}_2 \leadsto 0) \rightarrowtail
      (\kw{set}[0, v] \leadsto {*}) \rightarrowtail
      {*}) \leadsto
  \\&
    (\kw{inc}_2 \rightarrowtail 1) \leadsto
    (\kw{set}[0, v'] \rightarrowtail {*}) \leadsto
  &&
    (\kw{enq}[v'] \rightarrowtail
      (\kw{inc}_2 \leadsto 1) \rightarrowtail
      (\kw{set}[1, v'] \leadsto {*}) \rightarrowtail
      {*}) \leadsto
  \\&
    (\kw{inc}_1 \rightarrowtail 0) \leadsto
    (\kw{get}[0] \rightarrowtail v)
  &&
    (\kw{deq} \rightarrowtail
      (\kw{inc}_1 \leadsto 0) \rightarrowtail
      (\kw{get}[0] \leadsto v) \rightarrowtail
      v)
\end{align*}}%
Layered composition allows us to compute their behavior
when we let them interact over $E_\kw{rb}$.
The resulting transition system
$\Sigma_\kw{bq} \odot \Sigma_\kw{rb} :
 \top \rightarrow E_\kw{bq}$
admits traces like the following one:
\[
  \Sigma_\kw{bq} \odot \Sigma_\kw{rb} \: \vDash \:
    (\kw{enq}[v] \rightarrowtail {*}) \leadsto
    (\kw{enq}[v'] \rightarrowtail {*}) \leadsto
    (\kw{deq} \rightarrowtail v)
  \,.
\]
\end{example}
%}}}

Layered composition is associative,
so that for $L_1 : G \rightarrow H$,
$L_2 : F \rightarrow G$ and 
$L_3 : E \rightarrow F$,
the following property holds:
\[
  (L_1 \odot L_2) \odot L_3 \: \equiv \:
  L_1 \odot (L_2 \odot L_3) \: : \: E \rightarrow H
\]
Moreover,
as shown in Fig.~\ref{fig:overview:ts}c,
for an effect signature $E$ we can define the identity transition system
$\kw{id}_E : E \rightarrow E$
which simply passes through every incoming query.
Identity transition systems satisfy
\[
  L \odot \kw{id}_E \: \equiv \:
  \kw{id}_F \odot L \: \equiv \:
  L \: : \: E \rightarrow F
  \,.
\]
Finally,
layered composition is compatible with simulations,
so that for exemple the simulations
\[
  \phi_\kw{bq} : \Sigma_\kw{bq} \le \llbracket \kw{bq.c} \rrbracket
  \quad \text{and} \quad
  \phi_\kw{rb} : \Sigma_\kw{rb} \le \llbracket \kw{rb.c} \rrbracket
\]
could be combined into the larger simulation
\[
  \phi_\kw{bq} \odot \phi_\kw{rb} \::\:
    \Sigma_\kw{bq} \odot \Sigma_\kw{rb} \:\le\:
    \llbracket \kw{bq.c} \rrbracket \odot
    \llbracket \kw{rb.c} \rrbracket
  \,.
\] 

%}}}

%}}}

\subsection{Data Abstraction and Vertical Composition} \label{sec:sconv} %{{{

% Motivation {{{

The functionality implemented in Example~\ref{ex:bq}
can be described
at different levels of abstraction.
The user may rely on a specification
$\Gamma_\kw{bq} : \top \rightarrow E_\kw{bq}$
modeling the queue as a sequence $\vec{q} \in V^*$.
However,
the refinement
$\Sigma_\kw{bq} \odot \Sigma_\kw{rb}$
will more likely use a state of the form
$(c_1, c_2, \vec{b}) \in \mathbb{N} \times \mathbb{N} \times V^N$,
more closely related to the
in-memory representation used by the actual code.

Data abstraction techniques
can be used to connect these two views.
For example, simulations enable a simple form of data abstraction.
In the case of
$\rho_\kw{bq} : \Gamma_\kw{bq} \le \Sigma_\kw{bq} \odot \Sigma_\kw{rb}$,
the relation $\rho_\kw{bq}$ will spell out the correspondence between
high- and low-level views of the system.
This simulation is possible because
the interface $E_\kw{bq}$ reveals no details about internal state
of either component.

The situation is more complicated
when a component's \emph{interactions} change
across levels of abstraction.
For example,
consider the code of $\kw{decode.c}$
in Fig.~\ref{fig:readwritehello}.
Seen as a \emph{process},
the program is invoked with the question
$\kw{run} \in \mathcal{P}$
and relies on system calls such as $\kw{read}_i[n] \in \mathcal{S}$.
However, at a lower level of abstraction,
the action $\kw{run}$
will take the form of a call to the $\kw{main}$ function
in the context of a carefully prepared initial memory state,
and likewise calls to $\kw{read}$ and $\kw{write}$
will take the form of C-level calls into the C standard library.

%}}}

\paragraph{Simulation Conventions}

To address this challenge,
we adapt to the setting of effect signatures
the notion of simulation convention used in CompCertO.
A simulation convention connects
the ways an interface is viewed at different levels of abstraction.
In CompCertO,
the compiler's correctness theorem involves
a simulation convention
$\mathbb{C} : \mathcal{C} \leftrightarrow \mathcal{A}$,
which is used to express the way in which
C-level function calls ($\mathcal{C}$) are encoded
as assembly-level interactions ($\mathcal{A}$).

We will build on this idea and
define a richer notion of simulation convention between effect signatures.
A simulation
$\phi : L_1 \le_{\mathbf{R} \rightarrow \mathbf{S}} L_2$
between %the transition systems
$L_1 : E_1 \twoheadrightarrow F_1$ and 
$L_2 : E_2 \twoheadrightarrow F_2$
is parameterized by two simulation conventions
$\mathbf{R} : E_1 \leftrightarrow F_1$ and
$\mathbf{S} : E_2 \leftrightarrow F_2$.
The simulation can then assume that incoming
source- and target-level questions in $F_1$ and $F_2$
will be related according to the convention $\mathbf{S}$,
and must prove that outgoing questions in $E_1$ and $E_1$
will be related according to $\mathbf{R}$.
Conversely, it can assume that
the environment's answers in $E$
will be related according to $\mathbf{R}$
and must prove that the components' answers in $F$
will be related according to $\mathbf{S}$.

The resulting compositional structure is shown in Fig.~\ref{fig:hvcomp}.
While program components compose \emph{horizontally} through linking ($\odot$),
simulation conventions compose \emph{vertically}
in the manner of simulation relations ($\vcomp$).
Simulations proofs are compatible with both $\odot$ and $\vcomp$
so that they constitute a two-dimensional notion of refinement.
We can also define the identity convention
$\idsc_E : E \leftrightarrow E$;
when $\mathbf{R}$ and $\mathbf{S}$ are both $\idsc$,
our sophisticated notion of simulation reduces to the usual one.

\begin{example}[Semantics preservation of CompCert]
\label{ex:bq-proof}
XXX: We can express the calling convention $\mathbb{C}$ in this form
and reuse the CompCertO correctness theorem.
\end{example}

\begin{example}[Building complex refinement proofs]
XXX: Explain how to establish a relationship between
high-level effect signatures like $E_\kw{bq}$, $E_\kw{rb}$
and the C language interfaces.
Show how to compose the refinements
$\Gamma_\kw{bq} \le \Sigma_\kw{bq} \odot \Sigma_\kw{rb}
\le \kw{Clight}(\kw{bq.c}) \odot \kw{Clight}(\kw{rb.c})
\le \kw{Asm}(\kw{bq.s} + \kw{rb.s})$.
\end{example}

%DimSum follows the approach used in the refinement calculus
%and uses dual nondeterminism to express similar
%abstraction relationship as a dynamic ``translation''
%between interacting components which use
%incompatible representations.
%Our own approach is closer to the one used in CompCertO
%but the notions of \emph{companion} and \emph{conjoint}
%introduced in \S{}X.Y %XXX
%suggest ways in which these two approaches could be unified.

%}}}

\begin{remark}[Morphisms in Context] %{{{
We will rely on the category theory convention
by which the same notation is used
for a functor's action on objects and morphisms.
When functors are combined and specialized,
objects and morphisms
may appear together in certain expressions.
For example,
applying the functor $U \times {-} + V : \mathbf{Set} \rightarrow \mathbf{Set}$
to a function $f : X \rightarrow Y$ yields
\[
  U \times f + V \: : \: U \times X + V \:\rightarrow\: U \times Y + V
  \qquad \text{(also known as }
  \kw{id}_U \times f + \kw{id}_V
  \text{)}
\]
Seeing $\kw{id}_A$ as the morphism part of the nullary functor $A$,
another interpretation is that
objects can simply denote their identity morphism.
%
In any case,
this idea generalizes to higher dimensions.
For example, given
$L_1 : A \twoheadrightarrow B$,
$L_2 : B \twoheadrightarrow C$,
$\mathbf{R} : A \leftrightarrow B$,
$\mathbf{S} : B \leftrightarrow C$ and
$\phi : L_1 \le_{\mathbf{R} \twoheadrightarrow B} B$,
we can write
\[
  L_2 \odot \phi \: : \:
  L_2 \odot L_1 \le_{\mathbf{R} \twoheadrightarrow C} L_2
  \qquad \text{and} \qquad
  \phi \vcomp \mathbf{S} \: : \:
  L_1 \le_{\mathbf{R} \vcomp \mathbf{S} \twoheadrightarrow \mathbf{S}} C
  \,.
\]
\end{remark}
%}}}

\subsection{Combining Effect Signatures} \label{sec:fcomp} %{{{

While
both the horizontal composition of transition systems
and the vertical composition of simulation conventions
can be extended to simulations,
neither applies to effect signatures,
which instead serve as 0-dimensional endpoints along which
these higher-level composition principles apply.
We now introduce
a composition operation $\oplus$ for effect signatures
which will apply to all higher-dimensional objects as well.

\begin{definition}[Sum of signatures]
A family $(E_i)_{i \in I}$ of effect signatures can be combined into
\[
  \bigoplus_{i \in I} E_i \, := \,
    \{ \iota_i(m) \mathbin: N \mid i \in I,\, (m \mathbin: N) \in E_i \}
  \,.
\]
We will focus on the binary case where $i \in \{1, 2\}$,
which we write $E_1 \oplus E_2$.
\end{definition}

The signature $E \oplus F$ contains the combined questions of $E$ and $F$.
Each question retains the same set of answers.
Many of the signatures we have seen can be decomposed using $\oplus$.

\begin{example}[Single-file interfaces]
We have seen that processes can be modeled as
transition systems of type $P : \mathcal{S} \rightarrow \mathcal{P}$,
where the signature $\mathcal{S}$ contains questions for
each file descriptor $i \in \mathbb{N}$.
We can express this as follows:
\[
  \mathcal{S} \: = \: \bigoplus_{i \in \mathbb{N}} \: \mathcal{F}
  \qquad \text{where} \qquad
  \mathcal{F} \: := \:
    \{ \kw{read}[n] : \Sigma^*, \kw{write}[s] : \mathbb{N} \mid n \in \mathbb{N}, s \in \Sigma^* \}
\]
In the following examples we focus on standard input and output,
and simplify $\mathcal{S} := \mathcal{F} \oplus \mathcal{F}$.
\end{example}

\begin{example}
Something else
\end{example}

\begin{figure} % fig:ecomp {{{
  \begin{gather*}
    \begin{prooftree}
      \hypo{L_1 : E_1 \twoheadrightarrow F_1}
      \hypo{L_2 : E_2 \twoheadrightarrow F_2}
      \infer2[\kw{ts}-$\oplus$]{
        L_1 \oplus L_2 : E_1 \oplus E_2 \twoheadrightarrow F_1 \oplus F_2}
    \end{prooftree}
    \hspace{8em}
    \begin{prooftree}
      \hypo{\mathbf{R} : E_1 \leftrightarrow F_1}
      \hypo{\mathbf{S} : E_2 \leftrightarrow F_2}
      \infer2[\kw{sc}-$\oplus$]{
        \mathbf{R} \oplus \mathbf{S} : E_1 \oplus E_2 \leftrightarrow F_1 \oplus F_2
      }
    \end{prooftree}
    \\[1em]
    \begin{array}{r@{}l}
      (L_1 \odot L_2) \oplus (L_1' \circ L_2') & {} \equiv
      (L_1 \oplus L_1') \odot (L_2 \oplus L_2') \\
      \kw{id}_E \oplus \kw{id}_F & {} \equiv \kw{id}_{E \oplus F}
    \end{array}
    \quad
    \begin{array}{r@{}l}
      (\mathbf{R}_1 \vcomp \mathbf{R}_2) \oplus (\mathbf{S}_1 \vcomp \mathbf{S}_2)
      & {} \equiv
      (\mathbf{R}_1 \oplus \mathbf{S}_1) \vcomp (\mathbf{R}_2 \oplus \mathbf{S}_2)
      \\
      \idsc_E \oplus \idsc_F & {} \equiv \idsc_{E \oplus F}
    \end{array}
    \\[1em]
    \begin{prooftree}
      \hypo{\phi: L_1 \le_{\mathbf{R}_1 \twoheadrightarrow \mathbf{S}_1} L_1'}
      \hypo{\psi: L_2 \le_{\mathbf{R}_2 \twoheadrightarrow \mathbf{S}_2} L_2'}
      \infer2[\kw{sim}-$\oplus$]{\phi \oplus \psi :
	L_1 \oplus L_2
        \le_{\mathbf{R}_1 \oplus \mathbf{R}_2 \twoheadrightarrow
             \mathbf{S}_1 \oplus \mathbf{S}_2}
	L_1' \oplus L_2'}
    \end{prooftree}
  \end{gather*}
  \caption{Signature composition ($\oplus$) for transition systems,
    simulation conventions and simulation proofs.}
  \label{fig:ecomp}
\end{figure}
%}}}

The compositional properties of $\oplus$
are summarized in Fig.~\ref{fig:ecomp} and discussed below.
The transition system
$L_1 \oplus L_2 : E_1 \oplus E_2 \rightarrow F_1 \oplus F_2$
is straightforward and lets $L_1$ and $L_2$ operate independently.
When a question $q \in F_1$ is asked
in the left-hand side component of $F_1 \oplus F_2$,
it is used to activate $L_1$ which executes
until the question is answered.
$L_2$ handles the questions of $F_2$ in a similar way.
Additional transition systems
can be defined in relation to $\oplus$, namely
\[
  \Delta_E : E \rightarrow E \oplus E \,,
  \qquad
  \gamma_{E,F} : E \oplus F \rightarrow F \oplus E \,,
  \qquad
  \pi_1^{E,F} : E \oplus F \rightarrow E \,,
  \qquad
  \pi_2^{E,F} : E \oplus F \rightarrow F \,.
\]
The transition system $\Delta_E$ passes along
questions received in two independent copies of $E$
but consolidates them into a single copy.
The projections $\pi_i^{E,F}$
can be used to ``forget'' the unused summand
These constructions are illustrated
in the following example.

\begin{example}[Composing processes] %{{{
We can define shell-like operators for composing processes.
Two processes $P, Q : \mathcal{S} \rightarrow \mathcal{P}$
can be combined into
$(P \mathbin\texttt{;} Q) : \mathcal{S} \rightarrow \mathcal{P}$.
To this end,
we define the scheduling component
$\kw{seq} : \mathcal{P} \oplus \mathcal{P} \rightarrow \mathcal{P}$
which invokes one process, then the other:
\[
  \kw{seq} \vDash
    \kw{run} \cdot (\kw{run}_1 \cdot n) \cdot (\kw{run}_2 \cdot m) \cdot m
\]
This component can be used to define:
\[
  P \mathbin\texttt{;} Q \: := \:
    \kw{seq} \odot (P \oplus Q)
             \odot (\mathcal{F} \oplus \gamma \oplus \mathcal{F})
             \odot (\Delta \oplus \Delta)
  \qquad
  \text{XXX add figure here}
\]
We could likewise model the shell operators $\texttt{\&\&}$ and $\texttt{||}$
by replacing $\kw{seq}$ with different scheduling policies.
In addition,
we can use a component
$\kw{fifo} : \top \rightarrow \mathcal{F}$
with behaviors such as:
{\small
\[
  \kw{fifo} \: \vDash \:
    (\kw{write}[\texttt{"hello, "}] \rightarrowtail 7) \leadsto
    (\kw{write}[\texttt{"world!\textbackslash{}n"}] \rightarrowtail 7) \leadsto
    (\kw{read}[100] \rightarrowtail \texttt{"hello, world!\textbackslash{}n"})
\]
}%
With $\kw{fifo}$ to model a buffer,
we can define:
\[
  P \mathbin\texttt{|} Q \: := \:
    \kw{seq} \odot (P \oplus Q)
         \odot (\mathcal{F} \oplus (\Delta \odot \kw{fifo}) \oplus \mathcal{F})
\]
This will eventually help us write (but XXX move later)
\[
  \mathsf{load}(\mathrm{secret.c})
  \mathbin{\texttt{|}}
  \mathsf{load}(\mathrm{rot13.c})
  \:\vDash\:
  \mathsf{run} \cdot
  \mathsf{write}_1[\text{``hello, world!''}] \cdot
  15 \cdot
  0
\]
\end{example}
%}}}

%The direct product of effect signatures
%gives our model's cartesian product:
%\[
%  \begin{prooftree}
%    \hypo{L_1 : A_1 \rightarrow B_1}
%    \hypo{L_2 : A_2 \rightarrow B_2}
%    \infer2{L_1 \with L_2 \,:\, A_1 \with A_2 \,\rightarrow\, B_1 \with B_2}
%  \end{prooftree}
%  \qquad
%  \gamma_{A,B} : A \with B \rightarrow B \with A
%  \qquad
%  \Delta_A : A \rightarrow A \with A
%  \qquad
%  {*}_A : A \rightarrow \top
%\]

The simulation convention $\mathbf{R} \oplus \mathbf{S}$
simply relates the questions and answers on each side
by using $\mathbf{R}$ and $\mathbf{S}$ respectively.
Constructing the composite simulation
is straightforward as well.

\begin{example}
XXX Something with $\mathbb{C}$ in the context of the encode/decode example?
\end{example}

%}}}

\subsection{Spatial Composition} %{{{

\begin{figure} % fig:xcomp {{{
  \begin{gather*}
    \begin{prooftree}
      \hypo{L : A \twoheadrightarrow B}
      \hypo{f : U \lensarrow V}
      \infer2[\kw{ts}-$\mathbin@$]{
        L \mathbin@ f : A \mathbin@ U \twoheadrightarrow B \mathbin@ V
      }
    \end{prooftree}
    \hspace{8em}
    \begin{prooftree}
      \hypo{\mathbf{R} : A \leftrightarrow B}
      \hypo{\mathbf{S} : U \leftrightarrow V}
      \infer2[\kw{sc}-$\mathbin@$]{
        \mathbf{R} \mathbin@ \mathbf{S} : A \mathbin@ U \leftrightarrow B \mathbin@ V
      }
    \end{prooftree}
    \\[1em]
    \begin{array}{r@{}l}
      (L_1 \odot L_2) \mathbin@ (f \circ g) & {} \equiv
      (L_1 \mathbin@ f) \odot (L_2 \mathbin@ g) \\
      \kw{id}_A \mathbin@ \kw{id}_U & {} \equiv \kw{id}_{A \mathbin@ U}
    \end{array}
    \quad
    \begin{array}{r@{}l}
      (\mathbf{R}_1 \vcomp \mathbf{R}_2) \mathbin@ (\mathbf{S}_1 \vcomp \mathbf{S}_2)
      & {} \equiv
      (\mathbf{R}_1 \mathbin@ \mathbf{S}_1) \vcomp (\mathbf{R}_2 \mathbin@ \mathbf{S}_2)
      \\
      \idsc_A \mathbin@ \idsc_U & {} \equiv \idsc_{A \mathbin@ U}
    \end{array}
    \\[1em]
    \begin{prooftree}
      \hypo{\phi: L \le_{\mathbf{R}_1 \twoheadrightarrow \mathbf{S}_1} L'}
      \hypo{\psi: f \le_{\mathbf{R}_2 \twoheadrightarrow \mathbf{S}_2} f'}
      \infer2[\kw{sim}-$\mathbin@$]{\phi \mathbin@ \psi :
	L \mathbin@ f
        \le_{\mathbf{R}_1 \mathbin@ \mathbf{R}_2 \twoheadrightarrow
             \mathbf{S}_1 \mathbin@ \mathbf{S}_2}
	L' \mathbin@ f'}
    \end{prooftree}
  \end{gather*}
  \caption{Spatial composition ($\mathbin@$) for transition systems,
    simulation conventions and simulation proofs.}
  \label{fig:xcomp}
\end{figure}
%}}}

The tensor product is another well-known operation on effect signatures,
which expects the client to
\emph{simultaneously} ask a question in each component:
\[
  \bigotimes_{i \in I} E_i \, := \,
    \textstyle
    \big\{ \langle m_i \rangle_{i\in I} : \prod_{i \in I} N_i \mathrel{\big|}
       \forall i \mathbin. (m_i \mathbin: N_i) \in E_i \big\}
\]
Unfortunately,
$\otimes$ does not generalize seamlessly to higher-dimensional components
in the same way $\oplus$ did.
Although the simulation convention $\mathbf{R} \otimes \mathbf{S}$
is straightforward to define,
it is not possible in general to define the transition system
$L_1 \otimes L_2 : E_1 \otimes E_2 \twoheadrightarrow F_1 \otimes F_2$
because there no reason to expect
the outgoing questions of $L_1$ and $L_2$
to synchronize in any way,
so that they could be combined
into questions of $E_1 \otimes E_2$.

Although a general form of $\otimes$
does not apply in our framework,
by restricting the right-hand side
to a form of \emph{passive} components
we obtain a form of \emph{spatial} composition
and a way to approach \emph{state} compositionally.
Specifically,
we start from the effect signature construction
\[
  E \mathbin@ U \::=\: E \otimes \{ u : U \mid u \in U \} \:=\:
    \{ m @ u : N \times U \,\mid\, (m \mathbin: N) \in E, \, u \in U \}
\]
It will play an important role
in our treatment of spatial composition and state encapsulation.

\begin{example}
In the CompCertO language interfaces,
%that in the language interfaces $\mathcal{C}_\kw{m}$ and $\mathcal{A}_\kw{m}$,
%used in CompCertO for the semantics of source and target programs,
every question and answer includes a global memory state $m \in \kw{mem}$.
%Following \citet{rbgs-cal},
%for a language interface $A$ and a set of global states $U$,
This allows us to use decompositions
such as $\mathcal{C} \mathbin@ \kw{mem}$ above, where
\[
  \mathcal{C} = \{ f(\vec{v}) \mathbin: \kw{val} \mid
      f \in \kw{ident}, \vec{v} \in \kw{val}* \}
\]
does not mention the memory state.
Compare with Example X.Y.
This will be useful later
to attach additional state or give specifications
which don't use the memory state at all.
\end{example}

\begin{example}[Abstract specifications] \label{ex:abspec} %{{{
The specification $\Gamma_\kw{bq}$ shown in Fig.~\ref{fig:spec}
gives an abstract description of the code in Fig.~\ref{fig:code}
by representing the queue state as a sequence $\vec{q}$.
Likewise $\Gamma_\kw{rb}$ uses the data $(b, c_1, c_2)$
to represent the contents of the buffer and the counter values.
Finally,
$\kw{bq.c}$ does not use any state of its own
and can be described by the simple 
specification
$
  \Sigma_\kw{bq} : \mathcal{C} \twoheadrightarrow \mathcal{C}
$.
%
As before,
we would hope to decompose a correctness proof
along the following lines:
\[
    \phi_1 : \Gamma_\kw{bq}
      \le_{\top \twoheadrightarrow ?}
      \Sigma_\kw{bq} \! \mathbin{\text{``}{\odot}\text{''}} \Gamma_\kw{rb}
    \qquad
    \phi_2 : \Sigma_\kw{bq}
      \le_{? \twoheadrightarrow ?}
      \kw{Clight}(\kw{bq.c})
    \qquad
    \phi_\kw{rb} : \Gamma_\kw{rb}
      \le_{\varnothing \twoheadrightarrow ?}
      \kw{Clight}(\kw{rb.c})
\]
However, the different types of states
prevent the components
from being composed directly.
%Below, we show how to extend transition systems
%to operate with an additional state component,
%so that we
%both side
%to pass along the other's state:
%\[
%  \small
%  \begin{tikzcd}[sep=8em]
%    \top \cong \top @ \kw{mem}
%    \ar[r, "L_\kw{bq}@ \kw{mem}", twoheadrightarrow] &
%    \mathcal{C} @ D_\kw{bq} @ \kw{mem} \cong
%    \mathcal{C} @ \kw{mem} @ D_\kw{bq}
%    \ar[r, "\kw{Clight}(M) @ D_\kw{bq}", twoheadrightarrow] &
%    \mathcal{C} @ \kw{mem} @ D_\kw{bq}
%    \,.
%  \end{tikzcd}
%\]
\end{example}
%}}}

To make the approach outlined above practical,
we must turn $@$ into a proper composition principle
and establish its action on
transition systems,
simulation conventions and
simulations.

\paragraph{Adjoining State} \label{sec:overview:slift} %{{{

We start by
outlining how the construction ${-} \mathbin@ U$
acts on transition systems
in the case of a fixed set $U$.
Namely,
given $L : A \twoheadrightarrow B$,
the transition system
$
  L \mathbin@ U : A \mathbin@ U \twoheadrightarrow B \mathbin@ U
$
transparently passes along
a state component of type $U$ as follows:
\begin{equation} \label{eqn:slift}
  \begin{prooftree}
  \hypo{
  L \:\vDash\: q \rightarrowtail
    (q_1 \leadsto r_1) \rightarrowtail
    \cdots \rightarrowtail
    (q_n \leadsto r_n) \rightarrowtail
    r
  }
  \infer1{
  L \mathbin@ U \:\vDash\: q@u_0 \rightarrowtail
    (q_1@u_0 \leadsto r_1@u_1) \rightarrowtail
    \cdots \rightarrowtail
    (q_n@u_{n-1} \leadsto r_n@u_n) \rightarrowtail
    r@u_n
  }
  \end{prooftree}
\end{equation}
Here, the value $u_0 \in U$
is initially received from the environment as part of the incoming question.
$L \mathbin@ U$ then mirrors the execution of $L$
but keeps track of this additional state component.
The state is attached to any outgoing question in $A$
and updated when the corresponding answer is received.
When $L$ terminates,
the final value of the state is returned with the answer in $B$.
%}}}

\begin{example} \label{ex:abspeclift} % How that helps, a bit. {{{
In Example~\ref{ex:abspec},
we were unable to state the relationship
between the code specification
$\Sigma_\kw{bq} : \mathcal{C} \twoheadrightarrow \mathcal{C}$
and the corresponding implementation
$\kw{Clight}(\kw{bq.c}) : \mathcal{C} \mathbin@ \kw{mem}
 \twoheadrightarrow \mathcal{C} \mathbin@ \kw{mem}$
due to their difference in type.
We can now formulate the requirement
$
  \phi_2 : \Sigma_\kw{bq} \mathbin@ \kw{mem} \le \kw{Clight}(\kw{bq.c})
$,
which expresses that $\kw{bq.c}$
makes the outgoing calls prescribed by $\Sigma_\kw{bq}$
but does not modify the global memory state.
The $@$ construction also allows us to interface $\Sigma_\kw{bq}$
with the abstract specification
$\Gamma_\kw{rb} : \top \twoheadrightarrow \mathcal{C} \mathbin@ D_\kw{rb}$
as
$
  (\Sigma_\kw{bq} \mathbin@ D_\kw{rb}) \odot \Gamma_\kw{rb} :
  \top \twoheadrightarrow \mathcal{C} \mathbin@ D_\kw{rb}
$,
where
$\Sigma_\kw{bq} \mathbin@ D_\kw{rb} :
 \mathcal{C} \mathbin@ D_\kw{rb}
 \twoheadrightarrow
 \mathcal{C} \mathbin@ D_\kw{rb}$
``passes through'' the abstract data component $D_\kw{rb}$
on which $\Gamma_\kw{rb}$ operates.
\end{example}
%}}}

\paragraph{Transforming State} %{{{

%Unfortunately,
%in our framework
%it is not possible in general
%to form the tensor product of transition systems.
%To see why, consider a hypothetical
%$
%  L_1 \otimes L_2 : A_1 \otimes A_2 \twoheadrightarrow B_1 \otimes B_2
%$.
%When a question is received in $B_1 \otimes B_2$,
%its $B_1$ and $B_2$ components can be used to activate
%the underlying transition systems $L_1$ and $L_2$.
%However, during their executions,
%$L_1$ and $L_2$ can ask
%arbitrary numbers of questions
%in $A_1$ and $A_2$.
%In general,
%there is no reason to expect that these questions will synchronize
%meaningfully to be combined
%into questions of $A_1 \otimes A_2$.

%As illustrated by
%Example~\ref{ex:abspeclift},
%extending components to ``pass through''
%additional state fields can be useful,
%but in some cases we need the ability
%to transform those fields as well.
%To this end,
It is possible to generalize the construction $L \mathbin@ U$
to incorporate a \emph{lens} $f : U \lensarrow V$
with a more sophisticated action on the state component
than a simple pass-through.
Lenses \cite{lenses} provides access to a field of type $U$ within $V$
through functions:
\[
  \begin{array}{c}
    \kw{get}_f : V \rightarrow U \\[1ex]
    \kw{set}_f : V \times U \rightarrow V
  \end{array}
  \quad
  \begin{array}{r@{\:}l}
    \kw{get}_f(\kw{set}_f(v, u)) &= u \\
    \kw{set}_f(v, \kw{get}_f(v)) &= v \\
    \kw{set}_f(\kw{set}_f(v, u_1), u_2) &= \kw{set}_f(v, u_2)
  \end{array}
  \qquad
  \begin{tikzpicture}[yscale=0.15,xscale=0.30,baseline=(V.base)]
    \draw (5,-1) rectangle (8,4) node[midway] {$f$};
    \draw[->] (4,3) node[left] (V) {$v \in V$} -- (5,3) node[above=1em,midway] (A2) {$V$};
      \draw[->] (8,3) -- (9,3) node[above=1em,midway] (A1) {$U$} node[right] {$\kw{get}_f(v)$};
      \draw[->] (9,0) node[right] {$u \in U$} -- (8,0);
    \draw[->] (5,0) -- (4,0) node[left] {$\kw{set}_f(v, u)$};
    \path (A1) -- node {$\rightleftarrows$} (A2);
  \end{tikzpicture}
\]
Operationally,
as illustrated above,
we think of a lens as a component
which behaves somewhat like
the identity transition system
(Fig.~\ref{fig:overview:ts}c).
When an incoming question $v \in V$ activates the components,
the view $\kw{get}_f(v) \in U$ is extracted and
forwarded as an outgoing question.
When this outgoing question is answered with an update $u \in U$,
the updated value $\kw{set}_f(v, u)$ is returned to the caller.

As with $L \mathbin@ U$,
in the transition system
$L \mathbin@ f : A \mathbin@ U \twoheadrightarrow B \mathbin@ V$,
every question and answer consists of a pair,
with one component from $A$ or $B$
and one component from the sets $U$ or $V$;
the first component is handled by $L$
while the second one is just carried along.
But now, when $L$ makes an outgoing call,
the second component
first passes through the lens $f$
to be projected into $U$:
\[
  \begin{tikzpicture}[yscale=0.2,xscale=0.5]
    \begin{scope}[gray]%[canvas is xz plane at y=0,gray]
      \draw (-3,-2) rectangle (-1,11) node[midway] {$L$};
      %\scriptsize
      \draw[->] (-4,10) -- (-3,10);
      \draw[->] (-1,10) -- (5,10);
      \draw[->] (5,8) -- (-1,8);
      \draw[->] (-1,1) -- (5,1);
      \draw[->] (5,-1) -- (-1,-1);
      \draw[->] (-3,-1) -- (-4,-1);
      \node at (-0.5,5) {$\vdots$};
    \end{scope}
    \begin{scope}[thick,yshift=-0.5cm] %[canvas is xz plane at y=0.3]
      \draw[fill=white] (2,7) rectangle (4,11) node[midway] {$f$};
      \draw[fill=white] (2,-1) rectangle (4,3) node[midway] {$f$};
      \draw[->] (-4,10) -- (2,10);
      \draw[->] (4,10) -- (5,10);
      \draw[->] (5,8) -- (4,8);
      \draw[->] (4,2) -- (5,2);
      \draw[->] (5,0) -- (4,0);
      \draw[->] (2,0) -- (-4,0);
      \draw[->] (2,8) -- (1.5,8) -- node[left] {$v_1$} (1.5,6) -- (2,6);
      \draw[->] (2,4) -- (1.5,4) -- node[left] {$v_{n-1}$} (1.5,2) -- (2,2);
      \node at (3,5.5) {$\vdots$};
    \end{scope}
    \begin{scope}[yshift=-0.25cm]%[canvas is xz plane at y=0.15]
      \path
        (-4,10) node[left] {$(\textcolor{gray}{q}, v_0)$}
        ( 5,10) node[right] {$(\textcolor{gray}{q_1}, u_0)$}
        ( 5, 8) node[right] {$(\textcolor{gray}{r_1}, u_1)$}
        ( 5, 1.5) node[right] {$(\textcolor{gray}{q_n}, u_{n-1})$}
        ( 5,-0.5) node[right] {$(\textcolor{gray}{r_n}, u_n)$}
        (-4,-0.5) node[left] {$(\textcolor{gray}{r}, v_n)$};
    \end{scope}
  \end{tikzpicture}
\]

In practice,
two kinds of lens turn out to be especially useful.
First,
every bijection is a lens,
and this can be used to define structural isomorphisms
such as $\gamma_{U,V} : U \times V \cong V \times U$.
Secondly, the trivial lens
$\langle V ] : \mathbbm{1} \leftrightarrows V$
where
$\kw{get}_{\langle V ]}(v) = *$ and
$\kw{set}_{\langle V ]}(v, *) = v$
can act as a ``terminator'',
which does not propagate any part of the state in $U$
but instead returns it unchanged to the caller.

%The remaining components of $\mathbin@$-composition
%are obtained in the following way.
The $\mathbin@$ construction can be further extended
to act on simulation conventions and simulations
to obtain the full compositional structure shown in Fig.~\ref{fig:xcomp}.
The composite simulation convention $\mathbf{R} \mathbin@ \mathbf{S}$
simply requires that the two fields within the questions and answers
of the composite language interfaces
be independently related by the corresponding simulation conventions.
Moreover, a relation $R \subseteq U \times V$
can be promoted to a simple simulation convention.
See \S\ref{sec:scomp} for details.

%}}}

\begin{example} \label{ex:bqcorrect} %{{{
Building on Example~\ref{ex:abspeclift},
consider the relationship between
the overall specification $
  \Gamma_\kw{bq} :
    \top \twoheadrightarrow \mathcal{C} \mathbin@ D_\kw{bq}
$
and its partial refinement
$
  (\Sigma_\kw{bq} \mathbin@ D_\kw{rb}) \odot \Gamma_\kw{rb} :
    \top \twoheadrightarrow \mathcal{C} \mathbin@ D_\kw{rb}
$.
To establish a simulation between them,
we use the abstraction relation
$R_\kw{bq} \subseteq D_\kw{bq} \times D_\kw{rb}$
shown at the bottom of Fig.~\ref{fig:spec}.
The refinement property can then be formulated as
$
  \phi_1 :
  \Gamma_\kw{bq}
  \:\le_{\top \twoheadrightarrow \mathcal{C} \mathbin@ R_\kw{bq}}\:
  (\Sigma_\kw{bq} \mathbin@ D_\kw{rb}) \odot
  \Gamma_\kw{rb}
$.
\end{example}
%}}}

%}}}

\begin{figure} % fig:spec {{{
  \small
%$\top = \langle \varnothing, \varnothing \rangle$
%$\vec{q} \in D_\kw{bq} := \kw{val}^*$,
\begin{align*}
\toprule
&
  \Gamma_\kw{bq} : \top \twoheadrightarrow \mathcal{C} \mathbin@ D_\kw{bq}
&  
  \Gamma_\kw{bq} &\vDash
      \kw{enq}(v) @ \vec{q}
      \:\rightarrowtail\:
      {*} @ \vec{q}v
&
  \hspace{-5em}
  \vec{q} \in D_\kw{bq} := \kw{val}^*, \:
  v \in \kw{val}
\\ &&
  \Gamma_\kw{bq} &\vDash
      \kw{deq}() @ v\vec{q}
      \:\rightarrowtail\:
      v @ \vec{q}
\\
\midrule
&
  \Sigma_\kw{bq} : \mathcal{C} \twoheadrightarrow \mathcal{C}
&
  \Sigma_\kw{bq} &\vDash
      \kw{enq}(v) \rightarrowtail
      (\kw{inc2}() \leadsto i) \rightarrowtail
      (\kw{set}(i, v) \leadsto *) \rightarrowtail
      *
&
  i \in \mathbb{N}, \:
  v \in \kw{val}
\\ &&
    \Sigma_\kw{bq} &\vDash
      \kw{deq}() \rightarrowtail
      (\kw{inc1}() \leadsto i) \rightarrowtail
      (\kw{get}(i) \leadsto v) \rightarrowtail
      v
\\ &
  \Gamma_\kw{rb} : \top \twoheadrightarrow \mathcal{C} \mathbin@ D_\kw{rb}
&
  \Gamma_\kw{rb} &\vDash
    \kw{inc1}()@(b, c_1, c_2) \rightarrowtail
    c_1@(b, c_1\!\!+\!\!1, c_2)
&
  \hspace{-8em}
  (b, c_1, c_2) \in D_\kw{rb} :=
    \kw{val}^N \times \mathbb{N} \times \mathbb{N},
\\ &&
  \Gamma_\kw{rb} &\vDash
    \kw{inc2}()@(b, c_1, c_2) \rightarrowtail
    c_2@(b, c_1, c_2\!\!+\!\!1)
&
  i \in \mathbb{N}, \:
  v \in \kw{val}
\\ &&
  \Gamma_\kw{rb} &\vDash
    \kw{set}(i, v)@(b, c_1, c_2) \rightarrowtail
    {*}@(b[i := v], c_1, c_2)
\\ &&
  \Gamma_\kw{rb} &\vDash
    \kw{get}(i)@(b, c_1, c_2) \rightarrowtail
    b_i@(b, c_1, c_2)
\\ \midrule &
  R_\kw{bq} \subseteq D_\kw{bq} \times D_\kw{rb}
&
        \vec{q} &\mathrel{R_\kw{bq}} (b, c_1, c_2) \: \Leftrightarrow \:
           (c_1 \le c_2 < N \wedge
            \vec{q} = b_{c_1} \cdots b_{c_2-1}) \vee {}
&
  (b, c_1, c_2) \in D_\kw{rb},
\\ &&
         & \hspace{16.3ex}
           (c_2 \le c_1 < N \wedge
            \vec{q} = b_{c_1} \cdots b_{N-1} b_0 \cdots b_{c_2 - 1})
&
  \vec{q} \in D_\kw{bq}
\\
\bottomrule
\end{align*}
\vspace{-2.5em}
  \caption{Abstract specifications for $\kw{bq.c}$ and $\kw{rb.c}$.
    The types are explained in \S\ref{sec:overview}.
    The overall specification $\Gamma_\kw{bq}$
    describes the queue operations in terms of
    a sequence values $\vec{q} \in D_\kw{bq} := \kw{val}^*$.
    Verification can be decomposed using the intermediate specifications
    $\Sigma_\kw{bq}$ and $\Gamma_\kw{rb}$ for
    $\kw{bq.c}$ and $\kw{rb.c}$.
    See Example~\ref{ex:abspec} for details.}
  \label{fig:spec}
\end{figure}
%}}}

%\begin{figure} % fig:overall {{{
%\[
%  \text{(a)}
%  \quad
%  \vcenter{\hbox{%
%  \begin{tikzpicture}[xscale=0.6,yscale=0.3]
%    \small
%
%    % Background
%    \fill[scsdbg] (-1,0) rectangle (4,9);
%    \fill[act] (0,9)
%      [rounded corners] -- (0,3)
%      [sharp corners] -- (1,2)
%      [rounded corners] -- (0,1)
%      [sharp corners] |- (-1,0) |- cycle;
%
%    % Strings
%    \draw (0,9) node[above] {$\mathcal{C}$}
%      [rounded corners] -- (0,3) -- (2,1)
%      [sharp corners] -- (2,0) node[below] {$\kw{mem}$};
%    \draw (1,7) node[bln] {} %node[above] {$\kw{mem}$}
%      -- (1,6) \flatcompanion
%      [rounded corners] -- (1,5) -- (2,4);
%    \draw (3,7) node[bln] {} node[above] {$m_0$}
%      -- (3,6) \flatconjoint
%      [rounded corners] -- (3,5)
%      [sharp corners] -- (2,4) node {$\bullet$}
%      [rounded corners] -- (2,3) -- (0,1)
%      [sharp corners] -- (0,0) node[below] {$\mathcal{A}$};
%
%    % Node
%    \node[sdn] at (1,2) {$\mathbb{C}$};
%
%  \end{tikzpicture}
%  }}
%  \qquad
%  \text{(b)}
%  \quad
%  \vcenter{\hbox{%
%  \begin{tikzpicture}[yscale=0.45,xscale=1.1]
%    \newcommand{\filltint}{30}
%    \small
%
%    \coordinate (b) at (0,2.7);
%
%    % Background areas
%    \fill[pattern=crosshatch,opacity=0.15]
%      (0,5) -| (3,1) -- (2.5,1)
%      [rounded corners] -- (2,2)
%      [sharp corners] -- (1,2)
%      [rounded corners] -- (1,3)
%      [sharp corners] -- (0,4) -- cycle;
%    \fill[ACMLightBlue!\filltint]
%      (-2,5) -| (0,4)
%      [rounded corners] -- (1,3)
%      [sharp corners] -- (1,2) -| cycle;
%    \fill[ACMBlue!\filltint] (-2,2)
%      [rounded corners] -- (2,2)
%      [sharp corners] -- (2.5,1)
%      [rounded corners] -- (2,0)
%      [sharp corners] -| cycle;
%    \fill[ACMRed!\filltint] (-2,0) |- (3,-3) -- (3,1) -- (2.5,1)
%      [rounded corners] -- (2,0)
%      [sharp corners] -- cycle;
%
%    \begin{scope}[opacity=0.5,outer sep=2pt]
%      \tiny
%      \node[above right] at (-2,4) {$\mathcal{C}$};
%      \node[below left] at (3,5) {$\top$};
%      \node at (0,1) {$\mathcal{C} \mathbin@ \kw{mem}$};
%      \node[above left] at (3,-3) {$\mathcal{A} \mathbin@ \kw{mem}$};
%    \end{scope}
%
%    % Strings
%    \begin{scope}
%      \small
%      \draw (0,5) node[above] {$\Gamma_\kw{bq}'$} -- (0,4)
%        [rounded corners] -- (-1,3) node[left] {$\Sigma_\kw{bq}$}
%        [rounded corners] -- (-1,-1) node[left,pos=0.5] {$\kw{bq.c}$}
%          node[left,pos=1] {$\kw{bq.s}$}
%        [sharp corners] -- (0,-2)
%          -- (0,-3) node[below] {$\kw{Asm}(\kw{bq.s+rb.s})$};
%      \draw (0,4)
%        [rounded corners] -- (1,3) node[right] {$\Gamma_\kw{rb}'$}
%        [rounded corners] -- (1,-1) node[right,pos=0.5] {$\kw{rb.c}$}
%          node[right,pos=1] {$\kw{rb.s}$}
%        [sharp corners] -- (0,-2);
%      \draw (-2,2) node[left] {$\mathcal{C} \mathbin@ \langle m_0 \rangle$}
%        [rounded corners] -- (2,2)
%        [sharp corners] -- (2.5,1) -- (3,1) node[right] {$\varnothing$};
%      \draw (2.5,1)
%        [rounded corners] -- (2,0)
%        [sharp corners] -- (-2,0) node[left] {$\mathbb{C}$};
%    \end{scope}
%
%    % Nodes
%    \begin{scope}[every node/.style={circle,draw,fill=white,inner sep=1pt}]
%      \node at (0,4) {$\phi_1'$};
%      \node at (-1,2) {$\phi_2'$};
%      \node at (+1,2) {$\phi_\kw{rb}'$};
%      \node at (-1,0) {$\pi_\kw{bq}$};
%      \node at (+1,0) {$\pi_\kw{rb}$};
%      \node[inner sep=2pt] at (0,-2) {$\ell$};
%      \node[inner sep=2pt] (z) at (2.5,1) {$z$};
%    \end{scope}
%  \end{tikzpicture}
%  }}
%\]
%  \caption{Simulation convention (a) and
%    overall proof of correctness (b) for our running example}
%  \label{fig:overall}
%\end{figure}
%%}}}

%}}}

\section{The Model} \label{sec:model} %{{{

Definition~\ref{def:esig} provides a formal description of effect signatures.
This section defines
the remaining objects and constructions involved in our model
and establishes their relevant properties.

In \S\ref{sec:model:strat} we examine strategies and their composition,
as in a traditional game semantics paper.
We also analyze our model under a more operational lens
and provide an algebraic characterization of the model.
This nontraditional view
serves as a base in \S\ref{sec:model:ref}
to develop the vertical part of the model
consisting of refinement conventions and refinement squares.

\subsection{Strategies} \label{sec:model:strat} %{{{

% Preamble {{{
Our model of component behaviors uses standard game semantics constructions.
Strategies are represented as sets of traces of the form
\[
  q \rightarrowtail
  (m_1 \leadsto n_1) \rightarrowtail
  (m_2 \leadsto n_2) \rightarrowtail
  \cdots \rightarrowtail
  (m_k \leadsto n_k) \rightarrowtail
  r
  \leadsto
  q' \rightarrowtail
  \cdots
\]
In the following
we will usually use the more compact notation
$q \underline{m}_1 n_1
   \underline{m}_2 n_2 \cdots
   \underline{m}_k n_k \underline{r}
 q' \cdots$,
where we have underlined the moves of the component.
%}}}

\begin{definition}[Strategy] %{{{
Consider two effect signatures $E, F$.
Then a \emph{play} for the game $E \rightarrow F$
is an element $s \in P_{E \rightarrow F}$
in the set generated by the recursive grammar:
\[
  s \in P_{E \rightarrow F} ::= q s_q \:;
  \quad
  s_q \in \bar{P}_{E \rightarrow F}^q ::=
    \underline{m} \mid
    \underline{m} n s_q \mid
    \underline{r} \mid
    \underline{r} s \,,
  \qquad
  (m \in E, q \in F, n \in \kw{ar}(m), r \in \kw{ar}(q))
\]
and ordered by the smallest relation $\sqsubseteq$ satisfying:
\[
  \begin{prooftree}
    \hypo{s_q \sqsubseteq_q s_q'}
    \infer1{q s_q \sqsubseteq q s_q'}
  \end{prooftree}
  \qquad
  \begin{prooftree}
    \infer0{\underline{m} \sqsubseteq_q \underline{m}n s_q}
  \end{prooftree}
  \qquad
  \begin{prooftree}
    \hypo{s_q \sqsubseteq_q s_q'}
    \infer1{\underline{m}n s_q \sqsubseteq_q \underline{m}n s_q'}
  \end{prooftree}
  \qquad
  \begin{prooftree}
    \infer0{\underline{r} \sqsubseteq_q \underline{r} s}
  \end{prooftree}
  \qquad
  \begin{prooftree}
    \hypo{s \sqsubseteq s'}
    \infer1{\underline{r} s \sqsubseteq_q \underline{r} s'}
  \end{prooftree}
\]
In other words, plays are even-length, non-empty prefixes
of sequences of the form
%\[
%  q_1 \underline{m}_{11} n_{11} \cdots \underline{m}_{1k_1} n_{1k_1} r_1
%  q_2 \underline{m}_{21} n_{21} \cdots \underline{m}_{2k_2} n_{2k_2} r_2 \cdots
%  \in
%  P_{E \rightarrow F}
%\]
given before,
ordered under the prefix relation $\sqsubseteq$.
A \emph{strategy} for the game $E \rightarrow F$
is then a set of plays
\[
  \sigma \subseteq P_{E \rightarrow F}
  \qquad \text{such that} \qquad
  \forall \, s_1 \, s_2 \cdot s_1 \sqsubseteq s_2 \wedge
    s_2 \in \sigma \Rightarrow s_1 \in \sigma
  \,.
\]
We will write $\sigma : E \rightarrow F$
and write the corresponding set of strategies as
$S_{E \rightarrow F}$.
%Consider an effect signature $E$ and a partially ordered set $X$.
%The \emph{coplays} for $E$ with a result in $X$
%are generated by the grammar
%\[
%  \qquad \qquad \qquad \qquad
%  s \in \bar{P}_E(X) ::=
%    \underline{x} \mid
%    \underline{m} \mid
%    \underline{m}ns
%    \,,
%  \qquad
%    \big( x \in X, \, m \in E, \, n \in \kw{ar}(m) \big)
%\]
%and ordered under the smallest relation $\sqsubseteq$ satisfying
%\[
%  x \le y \Rightarrow \underline{x} \sqsubseteq \underline{y}
%  \,,
%  \qquad
%  \underline{m} \sqsubseteq \underline{m}ns
%  \,,
%  \qquad
%  s \sqsubseteq t \Rightarrow \underline{m}ns \sqsubseteq \underline{m}nt
%  \,.
%\]
%Given the signatures $E$ and $F$,
%a \emph{play} for the game $E \rightarrow F$
%is an element $s \in P_{E \rightarrow F}$ of the poset
%\[
%  P_{E \rightarrow F} = \sum_{(q \mathbin: R) \in F} P_E(R) \,,
%\]
%where the play
%$s = \iota_q(s')$ is written as $qs'$,
%so that for example, complete plays ending with an answer $r \in R$ take the form
%%$s = \iota_q(\underline{m}_1 n_1 \cdots \underline{m}_k n_k \underline{r})$
%%is written
%$q \underline{m}_1 n_1 \cdots \underline{m}_k n_k \underline{r}$.
%Finally,
%a \emph{strategy} for $E \rightarrow F$
\end{definition}
%}}}

\begin{example} %{{{
In the previous section
we have informally describe the behavior of various components
by writing down traces generated by their executions.
It takes very little effort to turn such descriptions into formal strategies.
For example,
the behavior as a process of the program $\kw{secret}$
can be given as a strategy $\sigma : \mathcal{S} \rightarrow \mathcal{P}$
defined by:
\begin{align*}
  \sigma :=
  %\bigcup_{n \in \mathbb{N} } {\downarrow} (
   % &\kw{run} \cdot \underline{\kw{write}}_1[\texttt{"uryyb, jbeyq!"}]
   %          \cdot n \cdot \underline{*} \, ) \\ =
  \{
    &\kw{run} \cdot \underline{\kw{write}}_1[\texttt{"uryyb, jbeyq!"}] \, , \, \\
    &\kw{run} \cdot \underline{\kw{write}}_1[\texttt{"uryyb, jbeyq!"}]
             \cdot n \cdot \underline{*} \,,\, \\
    &\kw{run} \cdot \underline{\kw{write}}_1[\texttt{"uryyb, jbeyq!"}]
             \cdot n \cdot \underline{*} \cdot \kw{run} \cdot
                 \underline{\kw{write}}_1[\texttt{"uryyb, jbeyq!"}] \, , \,
    \ldots 
      \mid n \in \mathbb{N} \}
\end{align*}
Note that the process can be run again after it terminates,
although it will behave identically in any subsequent executions.
\end{example}
%}}}

\paragraph{Algebraic characterization} %{{{
While the model we have defined above
is straightforward and follows game semantics tradition,
it may not be completely clear at first
\emph{why} it is an appropriate choice.
In the remainder of this section
we will show the model is %in a precise sense
the most general possible which combines
\begin{itemize}
  \item unbounded angelic nondeterminism with
  \item execution shapes of the kind outlined in \S\ref{sec:strat}.
\end{itemize}
To this end,
we conduct an analysis of the model's (co)algebraic structure
along the lines of \citet{ags-act}.
The category $\mathbf{Sup}$ of
sup-lattices and sup-preserving functions
provides a model of nondeterminism;
the shape of the game $E \rightarrow F$
leads us to the particular sup-lattice
\begin{equation} \label{eqn:stratlat}
  S_{E \rightarrow F} \:\cong\:
  \mu Y \cdot
    \bigwith_{q \in F}
    \left(
    \mu Z \cdot
      \bigoplus_{m \in E}
      \left( \bigwith_{n \in \kw{ar}(n)} Z \right)_\bot
    \oplus
      \bigoplus_{r \in \kw{ar}(q)}
      Y_\bot
    \right)
  \:\in\:
  \mathbf{Sup}
  \,,
\end{equation}
where $\with$ and $\oplus$ respectively denote
the (coinciding) product and coproduct in $\mathbf{Sup}$, and where
$\mu X \cdot F X$ denotes
the (coinciding) initial algebra and terminal coalgebra
for an endofunctor
$F : \mathbf{Sup} \rightarrow \mathbf{Sup}$.

The remainder of this section
will elucidate the characterization (\ref{eqn:stratlat}) given above.
In the process, we build a toolbox of constructions and properties
connecting the
trace-based,
state-based and
algebraic views of
interactive computation.
This toolbox is
used as a foundation
in formulating the more sophisticated constructions
offered by our model.

%}}}

%}}}

\subsection{Games in the category of sup-lattices} %{{{

A sup-lattice is a partially ordered set with all joins.
We regard sup-lattices as spaces of computations
ordered under refinement,
where the joins $\bigvee_{i\in I} x_i$ model angelic nondeterminism.
The least element $\bot = \bigvee \varnothing$
captures undefined or undesirable behaviors,
including silent divergence.

We usually expect a function $f : S \rightarrow T$
between two sup-lattices $(S, {\le})$ and $(T, {\sqsubseteq})$
to preserve all joins so that
$f \bigl( \bigvee_{i \in I} \sigma_i \bigr) = \bigsqcup_{i \in I} f(\sigma_i)$.
In other words,
we expect operations on computations
to act independently on each angelic choice.
Since this property is satisfied by the identity function and
preserved by composition,
sup-lattices and sup-preserving functions
form a category $\mathbf{Sup}$.

\paragraph{Compositional structure} %{{{

As a model of linear logic,
$\mathbf{Sup}$ features a rich structure \cite{egg}
which can be interpreted in terms of interactive computations.
The initial object $\mathbf{0} := \{\bot\}$
only has the undefined behavior $\bot$;
the terminal object $\mathbf{1} := \{\bot \le \top\}$
also allows a successful but trivial computation.
More interestingly, two sup-lattices $S_1$ and $S_2$ can be composed:
\begin{itemize}
  \item $S_1 \with S_2$
    has the \emph{environment} choose $i \in \{1, 2\}$,
    then the computation proceeds as in $S_i$.
  \item $S_1 \oplus S_2$
    has the \emph{system} choose $i \in \{1, 2\}$,
    then the computation proceeds as in $S_i$.
  \item $S_1 \otimes S_2$ 
    lets a computation of $S_1$ and a computation of $S_2$
    proceed concurrently.
\end{itemize}
All three generalize to infinitary constructions.
Cartesian products ($\with$) and coproducts ($\oplus$)
come equipped with the expected
projections, injections and (co)tupling:
\begin{align*}
  \pi_j &: \textstyle \bigl(\bigwith_{i\in I} S_i\bigr) \rightarrow S_j
  &
  \forall i\in I \cdot f_i : S \rightarrow T_i &\:\vdash\:
  \textstyle \langle f_i \rangle_{i\in I} :
    S \rightarrow \bigl(\bigwith_{i\in I} T_i\bigr)
  \\
  \iota_j &: \textstyle S_j \rightarrow \bigl(\bigoplus_{i\in I} S_i\bigr)
  &
  \forall i\in I \cdot f_i : S_i \rightarrow T &\:\vdash\:
  \textstyle [ f_i ]_{i\in I} :
    \bigl(\bigoplus_{i\in I} S_i\bigr) \rightarrow T
\end{align*}
Referring to the interpretation above,
$\pi_j(\sigma) \in S_j$ is the behavior
taken by $\sigma \in \bigwith_i S_i$
when the environment chooses $i := j$, whereas
in $\iota_j(\sigma) \in \bigoplus_i S_i$,
the system chooses $i := j$
and the computation continues as $\sigma \in S_j$.
Finally,
$\sigma \in S$ and $\tau \in T$
can be combined into $\sigma \otimes \tau \in S \otimes T$;
their nondeterministic choices are independent,
so that $\otimes$ is strictly sup-preserving
in both arguments.

%}}}

\paragraph{Lifting} %{{{

One construction found in $\mathbf{Sup}$ which goes beyond
the standard multiplicative-additive linear logic connectives
is the \emph{lifting} operator
$X_\bot := \{\bot \le x_\bot \mid x \in X\}$.
By extending a sup-lattice $X$ with a new least element $\bot$,
lifting enables \emph{partial} observations of a computation,
corresponding to incomplete plays in the strategy model.
For example,
in $\bigoplus_i \bigoplus_j \mathbf{1}$
the choices of $i$ and $j$ are observed simultaneously,
but a computation in
$\bigoplus_i \bigl( \bigoplus_j \mathbf{1} \bigr)_\bot$
could go wrong after revealing the choice of $i$
but before choosing $j$.

%}}}

\paragraph{Fixed points} %{{{

Every endofunctor $F : \mathbf{Sup} \rightarrow \mathbf{Sup}$
has a unique fixed point $\mu F \in \mathbf{Sup}$,
with the defining property that
$\mu F \cong F (\mu F)$.
We will usually write out
$\mu (X \mapsto M)$ in full, as $\mu X \cdot M$,
instead of defining $F$ separately.
Essentially,
a computation in $\mu X \cdot M$
proceeds as in the expression $M$,
but any occurrence of the parameter $X$
is unfolded into another copy of $M$.
The sup-lattice $\mu F$ carries both
an initial algebra $c_F : F (\mu F) \rightarrow \mu F \in \mathbf{Sup}$ and
a terminal coalgebra $d_F : \mu F \rightarrow F (\mu F) \in \mathbf{Sup}$,
which are inverse to each other and
can be used to fold and unfold $\mu F$ as needed.
%
%Since $c_F$ is initial,
%any $F$-algebra (folding operator)
%$\phi : F S \rightarrow S \in \mathbf{Sup}$
%on a sup-lattice $S$ induces a unique mapping
%$[\phi] : \mu F \rightarrow S$
%which uses $\phi$ as an interpretation for $c_F$
%in the sense that $[\phi] \circ c_F = \phi \circ F [\phi]$.
%
This means that an $F$-algebra
$\alpha : F S \rightarrow S \in \mathbf{Sup}$
induces a unique
$[ \alpha ] : \mu F \rightarrow S \in \mathbf{Sup}$
and likewise an $F$-coalgebra $\delta : S \rightarrow F S$
induces a unique
$\langle \delta \rangle : S \rightarrow \mu F$
such that
\[
  \begin{tikzcd}
    F (\mu F) \ar[r, "c_F"] \ar[d, "{F [\alpha]}"'] &
    \mu F \ar[d, dashed, "{[\alpha]}"] \\
    F S \ar[r, "\alpha"] &
    S
  \end{tikzcd}
  \qquad
  \begin{tikzcd}
    S \ar[r, "\delta"] \ar[d, "\langle \delta \rangle"', dashed] &
    F S \ar[d, "F \delta"] \\
    \mu F \ar[r, "d_F"] &
    F (\mu F)
  \end{tikzcd}
\]

%}}}

\paragraph{Free sup-lattices} %{{{

The forgetful functor $|{-}| : \mathbf{Sup} \rightarrow \mathbf{Pos}$
has a left adjoint
$\mathcal{D} : \mathbf{Pos} \rightarrow \mathbf{Sup}$
which computes the sup-lattice $\mathcal{D}(U)$
generated by a partially ordered set $(U, {\le})$.
Concretely,
elements of this lattice can be described as
downward-closed subsets of $U$:
\[
  \mathcal{D}(U) :=
    \{ x \subseteq U \mid
      \forall u v \cdot u \le v \wedge v \in x \Rightarrow u \in x \}
\]
ordered under inclusion ($\subseteq$).
The adjunction property makes it easy to describe
sup-preserving functions out of $\mathcal{D}(U)$ as
simple monotonic functions on the generators in $U$:
\[
  f : \mathcal{D}(U) \rightarrow Y \in \mathbf{Sup}
  \qquad \Longleftrightarrow \qquad
  g : U \rightarrow |Y| \in \mathbf{Pos}
\]
%Indeed we can define $g(x) = \bigsqcup_{u \in x} f(u)$
%and conversely $f(u) = g(\{u\})$,
%where the singleton function
%$\{{-}\} : U \rightarrow |\mathcal{D}(U)|$
%is the adjunction's unit and
%$\bigcup : \mathcal{D}(|\mathcal{D}(U)|) \rightarrow \mathcal{D}(U)$
%is the counit.
These relationships are invaluable for mechanization
as they allow us to represent
sup-lattices and their homomorphisms
using much simpler sets of generators and monotonic functions.
\[
  \begin{array}{c}
    \mathbf{0} \cong \mathcal{D}(\varnothing) \\[1ex]
    \mathbf{1} \cong \mathcal{D}(\mathbbm{1})
  \end{array}
  \qquad
  \bigoplus_{i \in I} \mathcal{D}(P_i) \cong
  \bigwith_{i \in I} \mathcal{D}(P_i) \cong
  \mathcal{D}\left(\sum_{i \in I} P_i \right)
  \qquad
  \bigotimes_{i \in I} \mathcal{D}(P_i) \cong
  \mathcal{D}\left(\prod_{i \in I} P_i \right)
  \qquad
  \begin{prooftree}
    \hypo{F \mathcal{D} \cong \mathcal{D} G}
    \infer1{\mu F \cong \mathcal{D}(\mu G)}
  \end{prooftree}
\]

%}}}

%}}}

\subsection{Interactive computations} %{{{

To illustrate some of the constructions above,
consider an interactive computation
using external operations
in an effect signature $E$.





An \emph{interaction} $x \in \mathcal{I}_E(A)$ over $E$
with outcomes in $A$ is then simply a set of plays,
closed downward under the prefix relation $\sqsubseteq$:
\[
  \mathcal{I}_E(A) := \mathcal{D}(P_E(A), {\sqsubseteq})
\]
Then strategy $L : E \rightarrow F$ gives for each $(m : N) \in F$
an interaction
\[
  L(m) \in \mathcal{I}_E(N)
\]

\begin{definition}[Monad structure]
Plays and interactions are monads over $\mathbf{Ord}$.
\end{definition}

\begin{definition}[Layered composition]
substitution etc as in LICS
\end{definition}

\begin{definition}[Determinism]
define:
\[
  mns_1 \coh mns_2
\]
\end{definition}

%}}}

\subsection{}

\begin{definition}[Interactive transition system]
For an effect signature $E$ and a set $X$,
an \emph{action} over $E$ with an outcome in $X$ is
\[
  a \in \mathcal{A}_E^X(K) ::= \underline{x} \mid \underline{m} \mid \underline{m}n k
  \qquad
  (m \mathbin: N \in E, x \in X, k \in K)
\]
An \emph{interactive transition system} over $E$
with outcomes in $X$
is given by a set $S$ of states,
together with a transition relation:
\[
  \kw{next}_S : S \rightarrow \mathcal{D} \big( \mathcal{A}_E^X(S) \big)
    %\mathcal{D} \Big( X + \sum_{m \mathbin: N \in E} (S_\bot)^N \Big)
\]
The strategy associated with a state $\sigma \in S$ is then given by the set of traces:
\[
  [\sigma] := \{ \mathcal{A}(\sigma \mapsto [\sigma])(a) \mid a \in \kw{next}_S(\sigma) \}
\]
\end{definition}

\subsection{Layered Composition} \label{sec:base:ts} %{{{

\[
  \left( \bigoplus_{i \in I} X \right) \otimes
  \left( \bigwith_{i \in I} Y \right)
  \xrightarrow{[\iota_i \otimes \kw{id}]^{-1}}
  \bigoplus_{i \in I} \left( X \otimes
  \left( \bigwith_{i \in I} Y \right) \right)
  \xrightarrow{[\kw{id} \otimes \pi_i]}
  X \otimes Y
\]

Our model uses CompCertO's notion of transition system
as described in Definition~\ref{def:lts},
but we introduce the following notion of horizontal composition
(depicted in Fig.~\ref{fig:overview:ts}).

\begin{definition}[Transition system composition] \label{def:lcomp} %{{{
The transition system
$\kw{id}_A : A \twoheadrightarrow A$
is defined as
\[
  \kw{id}_A \::=\:
  \big\langle
    A^\que + A^\ans, \:
    \varnothing, \:
    \iota_1, \:
    \iota_1^{-1}, \:
    \iota_2, \:
    \iota_2^{-1}
  \big\rangle
  \,.
\]
The composition of
$
  L_1 = \langle S_1, {\rightarrow_1}, I_1, X_1, Y_1, T_1 \rangle
    : B \twoheadrightarrow C
$ and $
  L_2 = \langle S_2, {\rightarrow_2}, I_2, X_2, Y_2, T_2 \rangle
    : A \twoheadrightarrow B
$
is the transition system
$
  L_1 \odot L_2 :=
  \langle S, {\rightarrow}, I, X, Y, F \rangle
  : A \twoheadrightarrow C
$ defined as follows.
States are taken in the set
$
    S := S_1 + (S_2 \times S_1)
$.
When an call in $C$ activates $L_1$,
the left summand is used:
\[
  \begin{prooftree}
    \hypo{q_C \mathrel{I_1} s_1}
    \infer1{q_C \mathrel{I} \iota_1(s_1)}
  \end{prooftree}
  \qquad
  \begin{prooftree}
    \hypo{s_1 \rightarrow_1 s_1'}
    \infer1{\iota_1(s_1) \rightarrow \iota_1(s_1')}
  \end{prooftree}
  \qquad
  \begin{prooftree}
    \hypo{s_1 \mathrel{F_1} r_C}
    \infer1{\iota_1(s_1) \mathrel{F} r_C}
  \end{prooftree}
\]
When $L_1$ makes an outgoing call in $B$,
its current state is saved and
the question activates $L_2$.
The execution then
operates on the state of $L_2$
until a final state of $L_2$ is reached
and $L_1$ is resumed:
\[
  \small
  \begin{prooftree}
    \hypo{s_1 \mathrel{X_1} q_B}
    \hypo{q_B \mathrel{I_2} s_2}
    \infer2{\iota_1(s_1) \rightarrow \iota_2(s_2, s_1)}
  \end{prooftree}
  \quad
  \begin{prooftree}
    \hypo{s_2 \rightarrow_2 s_2'}
    \infer1{\iota_2(s_2, s_1) \rightarrow \iota_2(s_2', s_1)}
  \end{prooftree}
  \quad
  \begin{prooftree}
    \hypo{s_2 \mathrel{X_2} q_A}
    \infer1{\iota_2(s_2, s_1) \mathrel{X} q_A}
  \end{prooftree}
  \quad
  \begin{prooftree}
    \hypo{r_A \mathrel{Y_2^{s_2}} s_2'}
    \infer1{r_A \mathrel{Y^{\iota_2(s_2, s_1)}} \iota_2(s_2', s_1)}
  \end{prooftree}
  \quad
  \begin{prooftree}
    \hypo{s_2 \mathrel{F_2} r_B}
    \hypo{r_B \mathrel{Y_1^{s_1}} s_1'}
    \infer2{\iota_2(s_2, s_1) \rightarrow \iota_1(s_1')}
  \end{prooftree}
\]
\end{definition}
%}}}

%\begin{example} \label{ex:base:lcomp} %{{{
%Referring back to Example~\ref{ex:base:clightsem},
%consider the transition system:
%\[
%  \kw{Clight}(\kw{bq.c}) \odot \kw{Clight}(\kw{rb.c}) :
%  \mathcal{C}@\kw{mem} \twoheadrightarrow \mathcal{C}@\kw{mem}
%  \,.
%\]
%There,
%the calls of $\kw{bq.c}$ into $\kw{rb.c}$
%are turned into internal steps,
%triggering a switch between executions of the two components.
%For example,
%the call to $\kw{inc1}$ from $\kw{deq}$
%may proceed as follows:
%\begin{align*}
%  \kw{deq}()@m[\kw{c_1} \mapsto 5] \mathrel{I} \iota_1(u_0)
%  \rightarrow \cdots &\rightarrow \iota_1(u_1)
%  \rightarrow \iota_2(s_0, u_1)
%  \rightarrow \cdots \\ \cdots &\rightarrow \iota_2(s_k, u_1)
%  \rightarrow \iota_1(u_2) \rightarrow \cdots
%  \rightarrow \iota_1(u_5) \mathrel{F} v@m[\kw{c1} \mapsto 6]
%  \,.
%\end{align*}
%\end{example}
%%}}}

%}}}

\subsection{Kripke Relators} %{{{

We will rely on the Kripke relator framework
used in \citet{compcerto}.
Given two relations
$R \subseteq A \times B$ and
$S \subseteq U \times V$,
the relation
$(R \rightarrow S) \subseteq
 (A \rightarrow U) \times
 (B \rightarrow V)$
is defined in the usual way:
\[
  f \ifr{R \rightarrow S} g
  \quad:\Leftrightarrow\quad
  \forall (a, b) \in A \times B \mathbin.
    a \mathrel{R} b \Rightarrow
    f(a) \mathrel{S} g(b)
  \,.
\]
The more unusual powerset relator $\mathcal{P}^\le$
is used to express simulation diagrams.
Given $R \subseteq A \times B$,
the relation
$\mathcal{P}^\le(R) \subseteq \mathcal{P}(A) \times \mathcal{P}(B)$
is defined as:
\[
  x \ifr{\mathcal{P}^\le(R)} y
  \quad:\Leftrightarrow\quad
  \forall a \in x \mathbin.
  \exists b \in y \mathbin.
  a \mathrel{R} b
\]
For example,
suppose
$\alpha : A \rightarrow \mathcal{P}(A)$
and
$\beta : B \rightarrow \mathcal{P}(B)$
are transition relations.
The relation $R$ is a simulation relation between them
when the property
$
  \alpha \:\ifr{R \rightarrow \mathcal{P}^\le(R)}\: \beta
$
holds.

Components of complex data structures
must often be related in ways which depend
on the context in which they appear
(which may include a computation's history).
Relations can then be indexed over a set of \emph{worlds}
which capture the relevant contextual information.
A \emph{Kripke relation} over a set of worlds $W$,
written $R \in \mathcal{R}_W(A, B)$,
is a relation $R \subseteq W \times A \times B$.
We use the notation
$
  w \Vdash a \mathrel{R} b
$
to mean that $(w, a, b) \in R$,
and
$\Vdash a \mathrel{R} b$
to mean that $a$ and $b$ are related at all worlds.

It is often useful to let worlds evolve
by endowing $W$ with an \emph{accessibility} relation
${\leadsto} \subseteq W \times W$.
World transitions are then captured by modal relators,
which associates to a Kripke relation $R \in \mathcal{R}_W(A, B)$
the Kripke relations $\Diamond R$ and $\Box R$ of the same type, defined by:
\begin{align*}
  w \Vdash a \ifr{\Diamond R} b
  \:&:\Leftrightarrow\:
  \exists w' \mathbin. w \leadsto w' \wedge w' \Vdash a \mathrel{R} b
\\
  w \Vdash a \ifr{\Box R} b
  \:&:\Leftrightarrow\:
  \forall w' \mathbin. w \leadsto w' \Rightarrow w' \Vdash a \mathrel{R} b
\end{align*}
For example,
consider a Kripke frame $\langle W, {\leadsto} \rangle$ and
a simulation relation $R \in \mathcal{R}_W(A, B)$
between $\alpha : A \rightarrow \mathcal{P}(A)$
and $\beta : B \rightarrow \mathcal{P}(B)$,
The relators $\rightarrow$ and $\mathcal{P}^\le$
can be promoted to Kripke relators
by pointwise extension over the set of worlds.
The complex Kripke simulation property:
{\small
\[
  \forall w \in W \mathbin.
  \forall a \in A \mathbin.
  \forall b \in B \mathbin.
  w \Vdash a \mathrel{R} b \Rightarrow
  \forall a' \in \alpha(a) \mathbin.
  \exists b' \in \beta(b) \mathbin.
  \exists w' \in W \mathbin.
  w \leadsto w' \wedge w' \Vdash a' \mathrel{R} b'
\]
}
can then be stated simply as
$
  \Vdash \alpha \ifr{R \rightarrow \mathcal{P}^\le(\Diamond R)} \beta
$.

Finally,
the composite and cartesian product
of Kripke relations can be described as:
\[
  \begin{array}{c}
    %R \in \mathcal{R}_U(A, B) ,
    %S \in \mathcal{R}_V(B, C) \vdash
    %R \vcomp S \in \mathcal{R}_{U \times V}(A, C) &
    (u, v) \Vdash a \mathrel{[R \mathbin; S]} c \:\::\Leftrightarrow\:\:
      \exists b \mathbin.
        u \Vdash a \mathrel{R} b \:\wedge\: v \Vdash b \mathrel{S} c
  \\
    %R \in \mathcal{R}_U(A, B),
    %S \in \mathcal{R}_V(C, D) \vdash
    %{R \times S} \in \mathcal{R}_{U \times V}(A \times C, B \times D) &
    (u, v) \Vdash (a, c) \mathrel{[R \times S]} (b, d) \:\::\Leftrightarrow\:\:
      u \Vdash a \mathrel{R} b \:\wedge\: v \Vdash c \mathrel{S} d
  \end{array}
\]
%}}}

\subsection{Simulation Conventions} %{{{

Simulation conventions
characterize the relationship between
source- and target-level
questions and answers.
In CompCertO,
every pair of calls is related in isolation,
independently of any past or future calls.
Our notion of simulation convention
is more general,
and maintains state across calls.
%to operate on \emph{sequences} of calls.

\begin{remark}[Motivating stateful simulation conventions] \label{rem:base:ssc} %{{{
This will be useful in \S\ref{sec:encap}
when we introduce encapsulation.
Calls to a specification with encapsulated state
may produce traces like:
\begin{equation} \label{eqn:ssc:encap}
  \kw{inc}() \cdot 0 \cdot \kw{inc}() \cdot 1 \,\cdots {}
\end{equation}
However, a more concrete version (and eventually, the implementation)
may use explicit state:
\begin{equation} \label{eqn:ssc:explicit}
  \kw{inc}()@[\kw{c} \mapsto 0] \:\cdot\:
  0@[\kw{c} \mapsto 1] \:\cdot\:
  \kw{inc}()@[\kw{c} \mapsto 1] \:\cdot\:
  1@[\kw{c} \mapsto 2] \:\cdots\: {}
\end{equation}
The correspondence between
the questions and answers in (\ref{eqn:ssc:encap})
and those in (\ref{eqn:ssc:explicit})
cannot be formulated on a call-by-call basis
but must take into account the history of the computation.
\end{remark}
%}}}

State is maintained using Kripke worlds.
In CompCertO's version,
Kripke worlds are only used to ensure that
questions and answers for a given call
are related consistently.
We extend the definition to incorporate
caller and callee world transitions
as well as an initial world.

\begin{definition} \label{def:sconv} %{{{
A \emph{simulation convention}
$\mathbf{R} = \langle W, w_0, {\mapsto}, {\leadsto}, R^\que, R^\ans \rangle$
from $A$ to $B$ is specified by:
\begin{itemize}
  \item a set $W$ of worlds
    with an initial world $\intl{w} \in W$;
  \item a \emph{caller} accessibility relation ${\mapsto} \subseteq W \times W$;
  \item a \emph{callee} accessibility relation ${\leadsto} \subseteq W \times W$;
  \item a Kripke relation $R^\que \in \mathcal{R}_W\big(A^\que,\, B^\que\big)$
    between the language interfaces' questions, and
  \item a Kripke relation $R^\ans \in \mathcal{R}_W\big(A^\ans,\, B^\ans\big)$
    between their answers.
\end{itemize}
The accessibility relations are required to be reflexive and transitive.
We write $\mathbf{R} : A \leftrightarrow B$.
\end{definition}
%}}}

\begin{example} \label{ex:base:ssc} %{{{
Referring to Remark~\ref{rem:base:ssc},
we can formulate a simulation convention
where worlds capture the counter's value.
The callee may update it
but the caller must leave it unchanged.
Using the notation $\top = \mathbb{N} \times \mathbb{N}$
for the total relation,
we can define the simulation convention:
\[
  \mathbf{R} := \langle \mathbb{N}, 0, {=}, \top, R^\que, R^\ans \rangle
  \qquad
  \begin{prooftree}
    \infer0{n \Vdash \kw{inc}() \mathrel{R^\que} \kw{inc}()@[\kw{c} \mapsto n]}
  \end{prooftree}
  \qquad
  \begin{prooftree}
    \infer0{n + 1 \Vdash n \mathrel{R^\ans} n@[\kw{c} \mapsto n + 1]}
  \end{prooftree}
\]
The simulation convention $\mathbf{R}$ defined above
relates the sequences (\ref{eqn:ssc:encap}) and (\ref{eqn:ssc:explicit}).
\end{example}
%}}}

%The initial world $\intl{w}$ gives the simulation convention's initial state.
%While the environment is in control,
%the world may transition according to the relation $\mapsto$.
%When control is transferred to the system,
%the corresponding questions must be related by $w \Vdash R^\que$.
%World transitions may then occur according to $\leadsto$.
%Hence, when the system returns control to the environment,
%the corresponding answers
%will be related by $w' \Vdash R^\ans$,
%where $w'$ is a successor world of $w$ such that $w \leadsto w'$.
%The questions for any subsequent activation
%must in turn be related at a world $w''$ such that $w' \mapsto w''$,
%and so on and so forth indefinitely:
%\begin{align*}
%  \intl{w} \mapsto w_1 &\Vdash q^\sharp_1 \mathrel{R^\que} q^\flat_1 \\
%  w_1 \leadsto w_1' &\Vdash r^\sharp_1 \mathrel{R^\ans} r^\flat_1 \\
%  w_1' \mapsto w_2 &\Vdash q^\sharp_2 \mathrel{R^\que} q^\flat_2 \\
%  w_2 \leadsto w_2' &\Vdash r^\sharp_2 \mathrel{R^\ans} r^\flat_2 \\[-1ex]
%  &\:\:\vdots
%\end{align*}

\begin{definition}[Composition of Simulation Conventions] \label{def:sccomp}
The identity simulation convention
$\idsc_A : A \leftrightarrow A$
is given by
$\idsc_A := \langle
    \mathbbm{1}, *, {=}_\mathbbm{1}, {=}_\mathbbm{1}, {=}_{A^\que}, {=}_{A^\ans}
 \rangle$.
The simulation conventions
$\mathbf{R}_1 : A \leftrightarrow B$ and
$\mathbf{R}_2 : B \leftrightarrow C$,
compose into
$\mathbf{R}_1 \vcomp \mathbf{R}_2 : A \leftrightarrow C$,
which is defined with the following components:
\begin{align*}
  W &:= W_1 \times W_2 &
  R^\que &:= R_1^\que \mathbin; R_2^\que &
  (w_1, w_2) \mapsto (w_1', w_2') \: &:\Leftrightarrow \:
    w_1 \mapsto_1 w_1' \: \wedge \:
    w_2 \mapsto_2 w_2' \\
&&  R^\ans &:= R_1^\ans \mathbin; R_2^\ans &
  (w_1, w_2) \leadsto (w_1', w_2') \: &:\Leftrightarrow \:
    w_1 \leadsto_1 w_1' \: \wedge \:
    w_2 \leadsto_2 w_2'
  \,.
\end{align*}
\end{definition}

%}}}

\subsection{Simulations} \label{sec:base:sim} %{{{

We can now define our generalized notion of simulation.%
\footnote{
  We will again omit some details
  which our model retains,
  such as CompCert's approach to
  demonic nondeterminism,
  silent divergence,
  and termination preservation.
  They do not present any particular difficulty
  to extend to our model.
}
Consider a simulation
$
%\begin{tikzcd}
%  A_1 \ar[r, twoheadrightarrow, "L_1"] \ar[d, leftrightarrow, "\mathbf{R}_A"'] &
%  B_1 \ar[d, leftrightarrow, "\mathbf{R}_B"] \\
%  A_2 \ar[r, twoheadrightarrow, "L_2"] & B_2
%\end{tikzcd}
  \phi : L_1 \le_{\mathbf{R}_A \twoheadrightarrow \mathbf{R}_B} L_2
$.

The simulation simultaneously
plays the role of the caller ($\mapsto$) with respect to
the simulation convention $\mathbf{R}_A : A_1 \leftrightarrow A_2$ and
the role of the callee ($\leadsto$) with respect to $\mathbf{R}_B : B_1 \leftrightarrow B_2$.
Hence,
it will operate in the context of a Kripke frame
constructed from both $W_A$ and $W_B$.
The possible states of a simulation will be a subset
$W \subseteq W_A \times W_B$,
which must contain
the pair of initial worlds.
Between successive activations,
the environment may update the $W_B$ component.
Hence we require:
\[
  (w_A, w_B) \in W \:\wedge\:
  w_B \mapsto_B w_B' \:\Rightarrow\:
  (w_A, w_B') \in W
\]
When the components execute,
the worlds will evolve according to
the accessibility relation:
\[
  (w_A, w_B) \leadsto_{\bar{A}B} (w_A', w_B') \::\Leftrightarrow\:
  w_A \mapsto_A w_A' \:\wedge\: w_B \leadsto_B w_B'
\]
Reading the constituent transition relations
within $L_1, L_2$ as functions of type:
\begin{align*}
  I_1 &: B_1^\que \rightarrow \mathcal{P}(S_1) &
  {\rightarrow_1} &: S_1 \rightarrow \mathcal{P}(S_1) &
  F_1 &: S_1 \rightarrow \mathcal{P}(B_1^\ans)
  \\
  I_2 &: B_2^\que \rightarrow \mathcal{P}(S_2) &
  {\rightarrow_2} &: S_2 \rightarrow \mathcal{P}(S_2) &
  F_2 &: S_2 \rightarrow \mathcal{P}(B_2^\ans)
  \,,
\end{align*}
we can formulate the simulation properties for internal steps
as shown in \autoref{fig:simint}abc.

\begin{figure} % fig:simint {{{
  \small
  \[
    \begin{array}{c@{\qquad}c@{\qquad}c}
      \begin{tikzcd}[sep=tiny,column sep=0]
        q_1 \ar[dd, "w \Vdash \mathbf{R}_B^\que"', dash] \ar[rr, dash, "I_1"] &&
        s_1 \ar[dd, "w' \Vdash R", dash, dashed] \\
        & \leadsto_{\bar{A}B} & \\
        q_2 \ar[rr, "I_2"', dash, dashed] &&
        s_2
      \end{tikzcd}
      &
      \begin{tikzcd}[sep=tiny,column sep=0]
        s_1 \ar[rr, "t"] \ar[dd, "w \Vdash R"', dash] &&
        \!\!{}_1 \:\, s_1' \ar[dd, "w' \Vdash R", dash, dashed] \\
        & \leadsto_{\bar{A}B} & \\
        s_2 \ar[rr, "t", dashed] &&
        \!\!{}_2^* \:\, s_2'
      \end{tikzcd}
%      \begin{tikzcd}[sep=large]
%        s_1 \ar[r] \ar[d, "{(w_A, w_B) \Vdash R}"', dash] &
%        s_1' \ar[d, "{(w_A,w_B) \Vdash R}", dash, dashed] \\
%        s_2 \ar[r, dashed] &
%        \!\!\!{}^* \: s_2'
%      \end{tikzcd}
      &
      \begin{tikzcd}[sep=tiny, column sep=0]
        s_1 \ar[rr, "F_1", dash] \ar[dd, "w \Vdash R"', dash] &&
        r_1 \ar[dd, "w' \Vdash \mathbf{R}_B^\ans", dash, dashed] \\
        & \leadsto_{\bar{A}B} & \\
        s_2 \ar[rr, "F_2"', dash, dashed] &&
        r_2
      \end{tikzcd}
      \vspace{1ex} \\
      I_1 \ifr{\Vdash \mathbf{R}_B^\que \rightarrow
        \mathcal{P}^\le(\Diamond_{\bar{A}B} R)} I_2
      &
      {\rightarrow_1}
      \ifr{\Vdash R \rightarrow \mathcal{P}^\le(\Diamond_{\bar{A}B} R)}
      {\rightarrow_2^*}
      &
      F_1
      \ifr{\Vdash R \rightarrow \mathcal{P}^\le(\Diamond_{\bar{A}B} \mathbf{R}_B^\ans)}
      F_2
      \vspace{1.2ex} \\
      \text{(a) Initial states} &
      \text{(b) Internal states} &
      \text{(c) Final states}
    \end{array}
  \]
  \[
    \begin{array}{c}
      \begin{tikzcd}[sep=tiny, column sep=small]
        s_1 \ar[rr, "X_1", dash] \ar[dd, "w \Vdash R"', dash] &&
        m_1 \ar[rr, dotted, dash] \ar[dd, "w'"', "{} \Vdash \mathbf{R}_A^\que", dash, dashed] &&
        n_1 \ar[rr, "Y_1^{s_1}", dash] \ar[dd, "w''"', "{} \Vdash \mathbf{R}_A^\ans", dash] &&
        s_1' \ar[dd, "w''' \Vdash R", dash, dashed]
        \\
        & \leadsto_{\bar{A}B} && \leadsto_{AB} && \leadsto_{\bar{A}B} &
        \\
        s_2 \ar[rr, "X_2"', dash, dashed] &&
        m_2 \ar[rr, dotted, dash] &&
        n_2 \ar[rr, "Y_2^{s_2}"', dash, dashed] &&
        s_2'
      \end{tikzcd}
      \vspace{1ex} \\
      Z_1
      \mathrel{[
        \Vdash R \rightarrow \mathcal{P}^\le(
          \Diamond_{\bar{A}B} (
          \mathbf{R}_A^\que \times
            \Box_{AB} (
            \mathbf{R}_A^\ans \rightarrow
            \mathcal{P}^\le(\Diamond_{\bar{A}B} R))))
      ]}
      Z_2
      \vspace{1.3ex} \\
      \text{(d) Outgoing calls}
    \end{array}
  \]

  \caption{Stateful simulation properties for internal steps (a,b,c)
    and outgoing calls (d).}
  \label{fig:simint}
\end{figure}
%}}}

Conversely, for external calls,
the simulation plays the role of the environment.
We expect that:
\[
  (w_A, w_B) \in W \:\wedge\:
  w_A \leadsto_A w_A' \:\Rightarrow\:
  (w_A', w_B) \in W
\]
From the point of view of the simulation,
an external call makes a transition according to:
%the following accessibility relation:
%[NB we may want to restrict $\leadsto_B$ to $=$
%if this causes problems, but]
%Note that by allowing a transition $w_B \leadsto_B w_B'$,
%we are able to capture the effect that
%any reentrant call may have on the simulation state:
\[
  (w_A, w_B) \leadsto_{AB} (w_A', w_B') \::\Leftrightarrow\:
  w_A \leadsto_A w_A' \:\wedge\:
  w_B = w_B'
\]
By reading the action of transition systems at external calls
in terms of the functions:
\begin{align*}
  Z_1 &: S_1 \rightarrow
    \mathcal{P}(A_1^\que \times (A_1^\ans \rightarrow \mathcal{P}(S_1))) &
  Z_1(s_1) &:= \{ (q_1, Y_1^{s_1}) \mid s_1 \mathrel{X_1} q_1 \}
 \\
  Z_2 &: S_2 \rightarrow
    \mathcal{P}(A_2^\que \times (A_2^\ans \rightarrow \mathcal{P}(S_2))) &
  Z_2(s_2) &:= \{ (q_2, Y_2^{s_2}) \mid s_2 \mathrel{X_2} q_2 \}
  \,,
\end{align*}
we can then formulate the simulation condition for external calls
as presented in \autoref{fig:simint}d.

\begin{definition}[Open simulation] \label{def:sim}
There is a simulation
of $L_1 : A_1 \twoheadrightarrow B_1$
by $L_2 : A_2 \twoheadrightarrow B_2$
under the simulation conventions
$\mathbf{R}_A : A_1 \leftrightarrow A_2$ and
$\mathbf{R}_B : B_1 \leftrightarrow B_2$,
if there exists
\begin{itemize}
\item a set of worlds $W$,
closed under ${\leadsto_A} \times {\mapsto_B}$ and
such that
$(\intl{w}_A, \intl{w}_B) \in W \subseteq W_A \times W_B$;
and
\item
a Kripke relation $R \in \mathcal{R}_W(S_1, S_2)$
between the states of $L_1$ and $L_2$;
\end{itemize}
satisfying the properties given in
\autoref{fig:simint}.
We will write
$L_1 \preceq_{\mathbf{R}_A \twoheadrightarrow \mathbf{R}_B} L_2$.
In addition,
we will sometimes write
$L_1 \le L_2$ for $L_1 \le_{\idsc \twoheadrightarrow \idsc} L_2$ and
$L_1 \equiv L_2$ when both $L_1 \le L_2$ and $L_2 \le L_1$.
\end{definition}

%}}}

\subsection{Companions and Conjoints} \label{sec:overview:companion} %{{{

Transition systems and simulation conventions
are both one-dimensional objects
connecting language interfaces.
Although
they are fairly different in nature,
sometimes
a transition system and a simulation convention
represent in some sense equivalent transformations.

\begin{definition} %{{{
We say that a transition system $L : A \twoheadrightarrow B$ has:
\begin{itemize}
  \item a \emph{companion} $L^* : A \leftrightarrow B$ when
    $L^\triangle : A \le_{A \twoheadrightarrow L^*} L$
    and
    $L^\triangledown : L \le_{L^* \twoheadrightarrow B} B$;
  \item a \emph{conjoint} $L_* : B \leftrightarrow A$ when
    $L_\triangle : B \le_{L_* \twoheadrightarrow B} L$
    and
    $L_\triangledown : L \le_{A \twoheadrightarrow L_*} A$.
\end{itemize}
\end{definition}
%}}}

Concretely,
these properties mean that for certain simulation statements,
we can choose
whether a particular component
should appear as a transition system
or as a simulation convention.
This makes it possible to decompose proofs
along non-rectangular boundaries,
and generally affords us additional flexibility.
%
In practice,
companions and conjoints
can prove especially useful for small,
``administrative'' components
represented using lenses,
as enabled by the following property.

\begin{theorem}
Every lens $f : U \lensarrow V$
has a companion $f^* : U \leftrightarrow V$
and a conjoint $f_* : V \leftrightarrow U$.
\end{theorem}

\begin{example}
Suppose the transition systems
$L_1 : \top \twoheadrightarrow \mathcal{C} \mathbin@ U \mathbin@ V$ and
$L_2 : \top \twoheadrightarrow \mathcal{C} \mathbin@ V \mathbin@ U$
use state components listed in opposite orders.
We can use the lens
$\gamma : V \times U \lensarrow U \times V$
to reconcile their types.
To express that $L_2$ refines $L_1$ we can
use either one of the equivalent properties:
\[
  L_1
    \le
    (\mathcal{C} \mathbin@ \gamma) \odot L_2
  \quad \Leftrightarrow \quad
  L_1
    \le_{\top \twoheadrightarrow \mathcal{C} \mathbin@ \gamma_*}
    L_2
  \,.
\]
\end{example}

%\begin{definition}[Composite language interfaces] \label{def:litens} %{{{
%Given two language interfaces $A$ and $B$,
%the language interface $A \otimes B$ is defined as
%$
%  A \otimes B :=
%    \langle A^\que \times B^\que, \,
%            A^\ans \times B^\ans \rangle
%$.
%The language interface
%$\mathbf{I} = \langle \mathbbm{1}, \mathbbm{1} \rangle$
%is a unit for $\otimes$.
%In addition, for a set $U$
%we define the language interface
%$[U] := \langle U, U \rangle$.
%\end{definition}
%%}}}
%
%%Note that $\otimes$ is %(in essence)
%%associative and commutative,
%%and that:
%%\[
%%  [U \times V] = [U] \otimes [V]
%%  \,,
%%  \qquad
%%  [\mathbbm{1}] = \mathbf{I}
%%  \,,
%%  \qquad
%%  [\varnothing] = \top
%%  \,.
%%\]
%%
%%These operations on language interfaces
%%are mirrored at the level of simulation conventions.
%
%Note that
%we can recover $A \mathbin@ U := A \otimes [U]$
%as a special case.
%Moreover,
%the action of $\otimes$ on simulation conventions is straightforward:
%$\mathbf{R} : A_1 \leftrightarrow A_2$ and
%$\mathbf{S} : B_1 \leftrightarrow B_2$
%can be combined into
%$
%  \mathbf{R} \otimes \mathbf{S} :
%  A_1 \otimes B_1 \leftrightarrow
%  A_2 \otimes B_2
%$,
%%(Def.~\ref{def:sctens})
%which requires the $A$ and $B$ components
%of questions and answers
%to be related independently by $\mathbf{R}$ and $\mathbf{S}$.
%A relation $R \subseteq U_1 \times U_2$
%can also be promoted to a simulation convention
%$
%  [R] : [U_1] \leftrightarrow [U_2]
%$
%which uses $R$ as the underlying relation for both
%questions and answers;
%we will often use $[-]$ implicitly.
%The following example shows how this can be used.

\begin{example}[Refinement of abstract specifications] \label{ex:abspecref} %{{{

To interface client code with an abstract specification such as
$\Gamma_\kw{rb} : \top \twoheadrightarrow \mathcal{C} \mathbin@ D_\kw{rb}$
which does not affect the concrete memory state,
we can use the lens
$\langle \kw{mem} ] : \mathbbm{1} \leftrightarrows \kw{mem}$.
For $\kw{bq.c}$ we get
\[
    L :=
    (\kw{Clight}(\kw{bq.c}) \mathbin@ D_\kw{rb}) \odot
    (\mathcal{C} \mathbin@ \langle \kw{mem} ] \mathbin@ D_\kw{rb}) \odot
    \Gamma_\kw{rb} \,.
\]
As discussed in Example~\ref{ex:abspeclift},
$\kw{bq.c}$ itself does not modify the global memory either.
As a result we can state the correctness of $\kw{bq.c}$ as
$
  \phi_\kw{bq} :
  \Gamma_\kw{bq} \le_{\top \twoheadrightarrow \mathbf{R}_\kw{bq}} L
$,
where
$\mathbf{R}_\kw{bq} :=
 \mathcal{C} \mathbin@ \langle \kw{mem} ]^* \mathbin@ R_\kw{bq}$.
%\[
%  \begin{tikzpicture}[xscale=0.5,yscale=0.5,baseline=(R.base)]
%    \fill[scsdbg] (0,0) rectangle (6.4,2);
%    \fill[act] (0,0) rectangle (1,2);
%    \draw (1,2) node[above] {$\mathcal{C}$}
%      -- (1,0) node[below] {$\mathcal{C}$};
%    \draw (3,1) node[circle,draw,fill=white,inner sep=1pt] {}
%      -- (3,0) node[below,yshift=-0.3ex] {$\kw{mem}$};
%    \draw (5,2) node[above] {$D_\kw{bq}$}
%      -- (5,1) node[rounded corners,draw,fill=white] (R) {$R_\kw{bq}$}
%      -- (5,0) node[below] {$D_\kw{rb}$};
%  \end{tikzpicture}
%\]
%
Here %the simulation convention
$\langle \kw{mem} ]^* : \mathbbm{1} \leftrightarrow \kw{mem}$
is the companion of $\langle \kw{mem} ]$.
It accepts any $\kw{mem}$ field in the target question,
but requires it to be unchanged in the subsequent answer.
The associated property
$
  \langle \kw{mem} ]^\triangle :
    \mathbbm{1}
    \le_{\mathbbm{1} \lensarrow \langle \kw{mem} ]^*}
    \langle \kw{mem} ]
$
can be used to derive
$
  \phi_\kw{bq} :
  \Gamma_\kw{bq} \le_{\top \twoheadrightarrow \mathbf{R}_\kw{bq}} L
$
from the simulations
%given in Examples~\ref{ex:bqcorrect} and \ref{ex:abspeclift},
\[
%  \vcenter{\hbox{\begin{tikzpicture}[sdp]
%    % Left/bottom
%    \fill[tssdbg] (0,0,0) -- (0,3,0) -- (0,3,2)
%               -- (0,0,2) -- (2,0,2) -- (2,0,0) -- cycle;
%    \draw[thin,dotted] (0,0,0) -- (0,0,2);
%    \draw (0,2,1) node[scn,bln] {}
%      -- (0,1,1) \companion
%      -- (0,0,1) -- (1,0,1) node[tsn,bln] {};
%    % Top/right
%    \fill[tssdbg,opacity=0.6]
%      (0,3,0) -- (2,3,0) -- (2,0,0) -- (2,0,2) -- (2,3,2) -- (0,3,2) -- cycle;
%    \draw[thin,dotted] (2,3,0) -- (2,3,2);
%  \end{tikzpicture} }}
  \phi_1 :
    \Gamma_\kw{bq} \le_{\top \twoheadrightarrow \mathcal{C} \mathbin@ R_\kw{bq}}
      (\Sigma_\kw{bq} \mathbin@ D_\kw{rb}) \odot \Gamma_\kw{rb}
%  \vcenter{\hbox{\begin{tikzpicture}[sdp]
%    % Left/bottom background
%    \fill[tssdbg] (0,0,0) -- (0,4,0) -- (0,4,6)
%               -- (0,0,6) -- (6,0,6) -- (6,0,0) -- cycle;
%    \draw[thin,dotted] (0,0,0) -- (0,0,6);
%    \fill[act] (0,0,0) -- (0,4,0) -- (0,4,2) -- (0,0,2)
%      [rounded corners] -- (1,0,2)
%      [sharp corners] -- (2,0,3)
%      [rounded corners] -- (3,0,2)
%      [sharp corners] -- (6,0,2) -- (6,0,0) --cycle;
%
%    % Left/bottom strings
%    \draw (0,4,2) -- (0,0,2)
%      -- (2,0,2) node[tsn] {$\Sigma_\kw{bq}$}
%      [rounded corners] -- (3.5,0,2)
%      [sharp corners] -- (4.5,0,3);
%    \draw (0,4,4)
%      -- (0,2,4) node[scn] {$R_\kw{bq}$}
%      -- (0,0,4) node[above right,inner sep=1pt] {\tiny $D_\kw{rb}$}
%      [rounded corners] -- (3.5,0,4)
%      [sharp corners] -- (4.5,0,3)
%      node[tsn] {$\Gamma_\kw{rb}$}
%      -- (6,0,3);
%
%    % Top/right background
%    \fill[tssdbg,opacity=0.6]
%      (0,4,0) -- (6,4,0) -- (6,0,0) -- (6,0,6) -- (6,4,6) -- (0,4,6) -- cycle;
%    \fill[act] (0,4,0) -- (0,4,2)
%      [rounded corners] -- (2,4,2)
%      [sharp corners] -- (3,4,3)
%      -- (6,4,3) -- (6,0,3)
%      -- (6,0,0) -- (6,4,0) -- cycle;
%    \draw[thin,dotted] (6,4,0) -- (6,4,6);
%
%    % Top/right strings and nodes
%    \draw (0,4,2) node[left] {\footnotesize $\mathcal{C}$}
%      [rounded corners] -- (2,4,2)
%      [sharp corners] -- (3,4,3);
%    \draw (0,4,4) node[left] {\footnotesize $D_\kw{bq}$}
%      [rounded corners] -- (2,4,4)
%      [sharp corners] -- (3,4,3) node[tsn] {$\Gamma_\kw{bq}$}
%      -- (6,4,3) -- node[right] {\footnotesize $\top$} (6,0,3);
%
%  \end{tikzpicture} }}
  \quad \text{and} \quad
  \phi_2 :
    \Sigma_\kw{bq} \mathbin@ \kw{mem} \le \kw{Clight}(\kw{bq.c})
%  \vcenter{\hbox{\begin{tikzpicture}[sdp]
%    % Left/bottom background
%    \fill[tssdbg] (0,0,0) -- (0,3,0) -- (0,3,5)
%               -- (0,0,5) -- (4,0,5) -- (4,0,0) -- cycle;
%    \draw[thin,dotted] (0,0,0) -- (0,0,5);
%    \fill[act] (0,0,0) -- (0,3,0) -- (0,3,2) -- (0,0,2)
%      [rounded corners] -- (1,0,2)
%      [sharp corners] -- (2,0,3)
%      [rounded corners] -- (3,0,2)
%      [sharp corners] -- (4,0,2) -- (4,0,0) --cycle;
%
%    % Left/bottom strings
%    \draw (0,3,4) -- (0,0,4)
%      [rounded corners] -- (1,0,4)
%      [sharp corners] -- (2,0,3)
%      [rounded corners] -- (3,0,4)
%      [sharp corners] -- (4,0,4);
%    \draw (0,3,2) -- (0,0,2)
%      [rounded corners] -- (1,0,2)
%      [sharp corners] -- (2,0,3)
%      node[tsn] {$\kw{bq.c}$}
%      [rounded corners] -- (3,0,2)
%      [sharp corners] -- (4,0,2);
%
%    % Top/right background
%    \fill[tssdbg,opacity=0.6]
%      (0,3,0) -- (4,3,0) -- (4,0,0) -- (4,0,5) -- (4,3,5) -- (0,3,5) -- cycle;
%    \fill[act] (0,3,0) -- (0,3,2) -- (4,3,2)
%            -- (4,0,2) -- (4,0,0) -- (4,3,0) -- cycle;
%    \draw[thin,dotted] (4,3,0) -- (4,3,5);
%
%    % Top/right strings and nodes
%    \draw (0,3,2) node[left] {\footnotesize $\mathcal{C}$}
%      -- (2,3,2) node[tsn] {$\Sigma_\kw{bq}$} -- (4,3,2)
%      -- (4,0,2) node[right] {\footnotesize $\mathcal{C}$};
%    \draw (0,3,4) node[left] {\footnotesize $\kw{mem}$}
%      -- (4,3,4) -- (4,0,4) node[right] {\footnotesize $\kw{mem}$};
%
%  \end{tikzpicture} }}
  \: .
\]
To wit, the simulation property
\[
 \mathcal{C} \mathbin@ \langle \kw{mem} ]^\triangle \mathbin@ R_\kw{bq}
 \:\::\:\:
 \mathcal{C} \mathbin@ D_\kw{bq}
 \:\:\le_{\mathcal{C} \mathbin@ R_\kw{bq} \twoheadrightarrow
      \mathcal{C} \mathbin@ \langle \kw{mem} ]^*  \mathbin@ R_\kw{bq}}\:\:
 \mathcal{C} \mathbin@ \langle \kw{mem} ] \mathbin@ D_\kw{rb}
\]
can be horizontally composed with $\phi_1$ to obtain a simulation of type
\[
  %(\mathcal{C} \mathbin@ \langle \kw{mem} ]^\triangle \mathbin@ R_\kw{bq}) \odot \phi_1
  %\::\:
  \Gamma_\kw{bq} \: \le_{\top \twoheadrightarrow
    \mathcal{C} \mathbin@ \langle \kw{mem}]^* \mathbin@ R_\kw{bq}} \:
  (\mathcal{C} \mathbin@ \langle \kw{mem} ] \mathbin@ D_\kw{rb}) \odot
    (\Sigma_\kw{bq} \mathbin@ D_\kw{rb}) \odot \Gamma_\kw{rb}
  \,.
\]
The functoriality properties given in Fig.~\ref{fig:xcomp}
allow us to reveal the left-hand side of $\phi_2$ by rewriting
\[
 (\mathcal{C} \mathbin@ \langle \kw{mem} ] \mathbin@ D_\kw{rb}) \odot
   (\Sigma_\kw{bq} \mathbin@ D_\kw{rb})
 \: \equiv \:
 (\Sigma_\kw{bq} \mathbin@ \kw{mem} \mathbin@ D_\kw{rb}) \odot
   (\mathcal{C} \mathbin@ \langle \kw{mem} ] \mathbin@ D_\kw{rb})
\]
Vertical composition with $\phi_2$ does the rest,
as shown in the proof term
\[
  \phi_\kw{bq} \: :=
  \big( (\mathcal{C} \mathbin@
   \langle \kw{mem} ]^\triangle \mathbin@
   R_\kw{bq}) \odot \phi_1 \big) \vcomp
  \big( (\phi_2 \mathbin@ D_\kw{rb}) \odot
        (\mathcal{C} \mathbin@ \langle \kw{mem} ] \mathbin@ D_\kw{rb}) \odot
        \Gamma_\kw{rb} \big)
  \,.
\]
\end{example}
%}}}

%}}}

%\subsection{Compositional Structure} \label{sec:base:double} %{{{
%
%The composition of transitions systems and simulation conventions
%define the respective categories $\mathbf{TS}$ and $\mathbf{SC}$.
%In addition,
%simulations compose both horizontally and vertically,
%namely:
%\[
%  \begin{prooftree}
%    \hypo{
%      \phi_1 :
%      L_1
%      \preceq_{\mathbf{R}_B \twoheadrightarrow \mathbf{R}_C}
%      L_1'}
%    \hypo{
%      \phi_2 :
%      L_2
%      \preceq_{\mathbf{R}_A \twoheadrightarrow \mathbf{R}_B}
%      L_2'}
%    \infer2{
%      \phi_1 \odot \phi_2 \::\:
%      L_1 \odot L_2
%      \: \preceq_{\mathbf{R}_A \twoheadrightarrow \mathbf{R}_C} \:
%      L_1' \odot L_2'}
%  \end{prooftree}
%  \qquad
%  \begin{prooftree}
%    \hypo{\phi : L_1
%      \preceq_{\mathbf{R}_A \twoheadrightarrow \mathbf{R}_B}
%      L_2}
%    \hypo{\pi : L_2
%      \preceq_{\mathbf{S}_A \twoheadrightarrow \mathbf{S}_B}
%      L_3}
%    \infer2{
%      \phi \vcomp \pi \::\:
%      L_1 \:
%      \preceq_{\mathbf{R}_A \vcomp \mathbf{S}_A \twoheadrightarrow
%	   \mathbf{R}_B \vcomp \mathbf{S}_B}
%      \: L_3}
%  \end{prooftree}
%\]
%Overall,
%the compositional structure of our model
%can be summarized in the following way.
%
%\begin{theorem}
%Language interfaces,
%transition systems,
%simulation conventions and
%simulation properties
%form a thin double category $\mathbf{TSC}$.
%\end{theorem}
%
%%This characterization
%%gives a formal underpinning to the usual notions of
%%horizontal and vertical composition
%%found in existing work on compositional certified compilers.
%
%%\begin{lemma}[Horizontal composition of simulations]
%%\[
%%  \begin{prooftree}
%%    \hypo{
%%      L_1^\sharp
%%      \preceq_{\mathbf{R}_B \twoheadrightarrow \mathbf{R}_C}
%%      L_1^\flat}
%%    \hypo{
%%      L_2^\sharp
%%      \preceq_{\mathbf{R}_A \twoheadrightarrow \mathbf{R}_B}
%%      L_2^\flat}
%%    \infer2{
%%      L_1^\sharp \odot L_2^\sharp
%%      \preceq_{\mathbf{R}_A \twoheadrightarrow \mathbf{R}_C}
%%      L_1^\flat \odot L_2^\flat}
%%  \end{prooftree}
%%  \qquad \qquad
%%  \begin{tikzcd}
%%    A^\sharp \ar[r, twoheadrightarrow, "L_2^\sharp"]
%%	     \ar[d, leftrightarrow, "\mathbf{R}_A"] &
%%    B^\sharp \ar[r, twoheadrightarrow, "L_1^\sharp"]
%%	     \ar[d, leftrightarrow, "\mathbf{R}_B"] &
%%    C^\sharp \ar[d, leftrightarrow, "\mathbf{R}_C"]
%%    \\
%%    A^\flat \ar[r, twoheadrightarrow, "L_2^\flat"'] &
%%    B^\flat \ar[r, twoheadrightarrow, "L_1^\flat"'] &
%%    C^\flat
%%  \end{tikzcd}
%%\]
%%\end{lemma}
%%
%%\begin{theorem}[Vertical composition of simulations] \label{thm:svcomp}
%%\[
%%  \begin{prooftree}
%%    \hypo{L^\sharp
%%      \preceq_{\mathbf{R}_A \twoheadrightarrow \mathbf{R}_B}
%%      L^\natural}
%%    \hypo{L^\natural
%%      \preceq_{\mathbf{S}_A \twoheadrightarrow \mathbf{S}_B}
%%      L^\flat}
%%    \infer2{L^\sharp
%%      \preceq_{\mathbf{R}_A \vcomp \mathbf{S}_A \twoheadrightarrow
%%	   \mathbf{R}_B \vcomp \mathbf{S}_B}
%%      L^\flat}
%%  \end{prooftree}
%%\]
%%\end{theorem}
%%
%%\begin{theorem}[Layered composition of simulations] \label{thm:lcompsim}
%%Simulations compose as follows:
%%\[
%%  \begin{prooftree}
%%    \hypo{L_1^\sharp
%%          \le_{\mathbf{R}_B \twoheadrightarrow \mathbf{R}_C}
%%          L_1^\flat}
%%    \hypo{L_2^\sharp
%%          \le_{\mathbf{R}_A \twoheadrightarrow \mathbf{R}_B}
%%          L_2^\flat}
%%    \infer2{L_1^\sharp \odot L_2^\sharp
%%          \le_{\mathbf{R}_A \twoheadrightarrow \mathbf{R}_C}
%%          L_1^\flat \odot L_2^\flat}
%%  \end{prooftree}
%%  \qquad \qquad
%%  \begin{tikzcd}
%%    A^\sharp \ar[r, twoheadrightarrow, "L_2^\sharp"]
%%             \ar[d, Leftrightarrow, "\mathbf{R}_A"'] &
%%    B^\sharp \ar[r, twoheadrightarrow, "L_1^\sharp"]
%%             \ar[d, Leftrightarrow, "\mathbf{R}_B"] &
%%    C^\sharp \ar[d, Leftrightarrow, "\mathbf{R}_C"]
%%    \\
%%    A^\flat \ar[r, twoheadrightarrow, "L_2^\flat"'] &
%%    B^\flat \ar[r, twoheadrightarrow, "L_1^\flat"'] &
%%    C^\flat
%%  \end{tikzcd}
%%\]
%%\end{theorem}
%
%%}}}
%
%\subsection{Relationship with CompCertO} %{{{
%
%The simulation conventions used in CompCertO
%constitute a subset of our own.
%In CompCertO,
%the caller may specify an arbitrary world with each new call,
%and the callee must relate the answers at that same world.
%Hence,
%under our definition they
%take the form $\langle W, *, \top, {=}, R^\que, R^\ans \rangle$.
%When simulation conventions of this form are used,
%our definition of simulation coincides with that of CompCertO,
%so that in particular
%CompCertO's correctness theorem can be reused as-is.
%%and combined with any correctness result
%%obtained for Clight programs.
%%
%Moreover,
%CompCertO defines a notion of simulation convention \emph{refinement},
%whereby a simulation convention
%can replace another in all simulation statements.
%Having defined a proper categorical structure for $\mathbf{TS}$,
%in our setting
%it is possible to encode simulation convention refinement as:
%\[
%  \mathbf{R} \sqsubseteq \mathbf{S} : A \leftrightarrow B
%  \quad :\Leftrightarrow \quad
%    \kw{id}_A \le_{\mathbf{R} \twoheadrightarrow \mathbf{S}} \kw{id}_B
%\]
%The interaction of $\sqsubseteq$ with simulations properties
%is then just an instance of horizontal composition.
%
%Finally,
%our layered composition operator $\odot$
%\emph{under-approximates}
%CompCertO's semantic linking operator $\oplus$.
%Since CompCert's syntactic linking of assembly programs
%is known to implement $\oplus$,
%this shows that linking is also a correct implementation of
%the layered composition $\odot$.
%
%\begin{theorem}[Linking implements layered composition] \label{thm:linking}
%For two assembly programs $M_1, M_2$,
%%For $L_1, L_2 : A \twoheadrightarrow A$,
%\[
%  \kw{Asm}(M_1) \odot \kw{Asm}(M_2)
%  \:\le\:
%  \kw{Asm}(M_1) \oplus \kw{Asm}(M_2)
%  \:\le\:
%  \kw{Asm}(M_1 + M_2)
% \,.
%\]
%\end{theorem}
%
%This means that when a system is compositionally
%specified and verified at the Clight level,
%and an overall correctness property is derived
%in terms of $\odot$,
%we can combine it with the compiler's correctness theorem
%to obtain guarantees about the linked assembly program.
%
%%}}}

%}}}

\section{Refinement Conventions}



\section{Compositional State} \label{sec:scomp} %{{{

In the previous section
we introduced a generalization of CompCertO semantics,
building a uniform two-dimensional structure
involving horizontal ($\odot$) and vertical ($\vcomp$) composition.
We now introduce spatial composition ($\mathbin@$)
to obtain our full three-dimensional framework.

\subsection{Lenses} %{{{

One peculiarity of the spatial composition operator $\mathbin@$
is that its action on behaviors
accepts transition systems on the left-hand side, but
only accepts \emph{lenses} on the right-hand side.
To fit lenses into the compositional structure
we defined in \S\ref{sec:base},
we interpret lenses as transition systems.

\begin{definition}
For a lens $f : U \lensarrow V$,
the transition system $[f] : [U] \twoheadrightarrow [V]$
is defined as:
\[
  [f] := \big\langle
    V + V, \: \varnothing, \:
    \iota_1, \: X, \: Y, \: \iota_2^{-1}
  \big\rangle
  \qquad \text{with} \qquad
  \begin{prooftree}
    \hypo{\kw{get}_f(v) = u}
    \infer1{\iota_1(v) \mathrel{X} u}
  \end{prooftree}
  \qquad
  \begin{prooftree}
    \hypo{\kw{set}_f(v, u') = v'}
    \infer1{u' \mathrel{Y^{\iota_1(v)}} \iota_2(v')}
  \end{prooftree}
  \quad.
\]
A relation $R \subseteq U \times V$
can likewise be promoted to
$[R] := \langle \mathbbm{1}, *, \top, \top, R, R \rangle :
 [U] \leftrightarrow [V]$.
\end{definition}

We can also define a simplified notion of simulation
involving only lenses and relations.

\begin{definition}[Lens simulation]
The lens
$f_1 : U_1 \lensarrow V_1$
is simulated by
$f_2 : U_2 \lensarrow V_2$
with respect to
$R_U \subseteq U_1 \times U_2$ and 
$R_V \subseteq V_1 \times V_2$
when the following property holds: 
\[
  f \lensle_{R_U \lensarrow R_V} g
  \quad :\Leftrightarrow \quad
  \kw{get}_f \ifr{R_V \rightarrow R_U} \kw{get}_g
  \:\wedge\:
  \kw{set}_f \ifr{R_V \times R_U \rightarrow R_V} \kw{set}_g
\]
\end{definition}

Lenses, relations, and lens simulations
compose in the expected ways,
and they embed functorially into
the structure established in \S\ref{sec:base}.
This means that the following properties hold:
\[
\begin{array}{r@{\:}l@{\qquad}r@{\:}l}
  [\kw{id}_U : U \lensarrow U]
    &\:\equiv\: \kw{id}_{[U]} : [U] \twoheadrightarrow [U]
  &
  [f \circ g] &\equiv [f] \odot [g]
  \\
  {[{=}_U \subseteq U \times U]}
    &\:\equiv\: \idsc_{[U]} : [U] \leftrightarrow [U]
  &
  [R ; S] &\equiv [R] \vcomp [S]
\end{array}
\qquad
\begin{prooftree}
  \hypo{\sigma : f \equiv_{R_U \lensarrow R_V} g}
  \infer1{[\sigma] : [f] \le_{[R_U] \twoheadrightarrow [R_V]} [g]}
\end{prooftree}
\]

%}}}

\subsection{Spatial Composition} \label{sec:scomp:tr} \label{sec:scomp:sc} %{{{

We now proceed to define the $\mathbin@$ construction.
Recall that at the level of language interfaces and sets,
we have $A \mathbin@ U := \langle A^\que \times U, A^\ans \times V \rangle$.
The horizontal and vertical parts of $\mathbin@$ are as follows.

\begin{definition}[Spatial composition] \label{def:lift} \label{def:sctens} %{{{
For a transition system
$L = \langle S, {\rightarrow}, I, X, Y, F \rangle : A \twoheadrightarrow B$
and a lens $f : U \rightarrow V$,
we define
$L \mathbin@ f :=
 \langle S \times V, {\rightarrow_f}, I_f, X_f, Y_f, F_f \rangle
 : A \mathbin@ U \twoheadrightarrow B \mathbin@ V$
where:
\begin{gather*}
 {\begin{prooftree}
    \hypo{q \mathrel{I} s}
    \infer1{q@v \mathrel{I_f} s@ v}
  \end{prooftree}}
  \quad
 {\begin{prooftree}
    \hypo{s \rightarrow s'}
    \infer1{s@v \rightarrow_f s'@v}
  \end{prooftree}}
  \quad
 {\begin{prooftree}
    \hypo{s \mathrel{F} r}
    \infer1{s@v \mathrel{F_f} r@v}
  \end{prooftree}}
  \quad
 {\begin{prooftree}
    \hypo{s \mathrel{X} m}
    \hypo{\kw{get}_f(v) = u}
    \infer2{s@v \mathrel{X_f} m@u}
  \end{prooftree}}
  \quad
 {\begin{prooftree}
    \hypo{n \mathrel{Y}^s s'}
    \hypo{\kw{set}_f(v, u') = v'}
    \infer2{n@u' \mathrel{Y^{s@v}_f} s'@v'}
  \end{prooftree}}
\end{gather*}
In addition,
given $\mathbf{R}_1 : A \leftrightarrow B$
and $\mathbf{R}_2 : U \leftrightarrow V$,
we define
$\mathbf{R}_1 \mathbin@ \mathbf{S}_2 :
 A \mathbin@ U \leftrightarrow B \mathbin@ V$
as
\[
  \mathbf{R}_1 \mathbin@ \mathbf{R}_2 \: := \:
    \big\langle
      W_1 \times W_2, \:
      (\intl{w}_1, \intl{w}_2), \:
      {\mapsto_1} \times {\mapsto_2}, \:
      {\leadsto_1} \times {\leadsto_2}, \:
      R_1^\que \times R_2^\que, \:
      R_1^\ans \times R_2^\ans
    \big\rangle
  \,.
\]
\end{definition}
%}}}

\begin{theorem} %{{{
The constructions defined above
satisfy the properties listed in Fig.~\ref{fig:xcomp}.
\end{theorem}
%}}}

%}}}

\subsection{Memory Separation} \label{sec:overview:sepalg} %{{{

Spatial composition
allows us to separate
complex states into different fields;
we can then reason about components
independently of the fields which they do not access,
and use $\mathbin@$
to connect these components with the rest of the system.
However, eventually this abstract description
must be refined into a concrete program
acting on a global memory,
where all state has been consolidated.

To achieve this in a way which preserves compositionality,
we use a \emph{partial commutative monoid}
over the CompCert memory model.
This provides an operation $\bullet$
which can be used to decompose a memory state $m$ into
a number of \emph{shares}
$
  m_1 \bullet \cdots \bullet m_n
$.
This construction
is similar in spirit to the \emph{algebraic memory model}
of \citet{ccal};
its construction is explained in Appendix~A. %\ref{app:sep}.

The properties of $\bullet$
and its interaction with memory operations
ensure that CompCert semantics satisfy
a \emph{frame} property,
meaning that they are insensitive to
additional memory shares:
\begin{equation} \label{eqn:overview:sepalg:frame}
  \begin{prooftree}
  \hypo{
  L \:\vDash\: q@m_0 \rightarrowtail
    (q_1@m_1 \leadsto r_1@m_1') \rightarrowtail
    \cdots \rightarrowtail
    (q_n@m_n \leadsto r_n@m_n') \rightarrowtail
    r@m'}
  \infer1{
   {\begin{array}{r@{\:}l}
    L \:\vDash\: q@(m_0 \bullet w_0) \rightarrowtail
      \big( q_1@(m_1 \bullet w_0) &\leadsto r_1@(m_1' \bullet w_1) \big) \rightarrowtail
      \cdots \\ \cdots \rightarrowtail
      \big( q_n@(m_n \bullet w_{n-1}) &\leadsto r_n@(m_n' \bullet w_n) \big) \rightarrowtail
      r@(m' \bullet w_n)
   \end{array}} }
  \end{prooftree}
\end{equation}
The similarity of (\ref{eqn:overview:sepalg:frame})
with the behavior (\ref{eqn:slift})
of the transition system $L \mathbin@ U$ (\S\ref{sec:overview:slift})
is no coincidence.
Reading $\bullet$ as a \emph{join} relation
$\jr \subseteq (\kw{mem} \times \kw{mem}) \times \kw{mem}$,
we can state one in terms of the other.

\begin{theorem}[Frame property for Clight] \label{thm:clightframe}
The Clight semantics satisfies
\[
  \kw{FP}(M) :
  \kw{Clight}(M) \mathbin@ \kw{mem}
  \le_{A \mathbin@ \jr \twoheadrightarrow B \mathbin@ \jr}
  \kw{Clight}(M)
  \,,
  \: \text{where} \:
  (m_1, m_2) \mathrel{\jr} m  :\Leftrightarrow
  m_1 \bullet m_2 = m
  \,.
\]
\end{theorem}

It will often be the case that the join relation
is applied to the target of
simulation convention components
$\mathbf{R} : U \leftrightarrow \kw{mem}$ and
$\mathbf{S} : V \leftrightarrow \kw{mem}$.
In this case,
we will use the notation:
\[
  \mathbf{R} \sepconj \mathbf{S} : U \mathbin@ V \leftrightarrow \kw{mem}
  \qquad
  \mathbf{R} \sepconj \mathbf{S} :=
  (\mathbf{R} \mathbin@ \mathbf{S}) \vcomp \jr
  \,.
\]

%}}}

\begin{example} %{{{
To show that $\kw{rb.c}$
faithfully implements $\Gamma_\kw{rb}$,
we must explain how the abstract states of $D_\kw{rb}$
are realized in the concrete memory.
We use the relation
$R_\kw{rb} \subseteq D_\kw{rb} \times \kw{mem}$
defined by:
\[
  (b, c_1, c_2) \: \mathrel{R_\kw{rb}} \:
  [\kw{buf} \mapsto \{b_0, \ldots, b_{N-1}\}, \,
   \kw{c1} \mapsto c_1, \,
   \kw{c2} \mapsto c_2]
  \,.
\]
At the implementation level,
the memory state passed to $\kw{rb.c}$
will contain $\kw{buf}$, $\kw{c1}$ and $\kw{c2}$,
whose values must match the high-level abstract state
and will be updated according to the specification.
The remaining part of the memory should not be changed by $\kw{rb.c}$.
This can be expressed as
\begin{equation} \label{eqn:rbcorrect}
  \phi_\kw{rb} :
  \Gamma_\kw{rb}
  \le_{\varnothing \twoheadrightarrow
       \mathcal{C} \mathbin@
       \langle \kw{mem} ]^* \sepconj R_\kw{rb}}
  \kw{Clight}(\kw{rb.c})
\end{equation}

Conveniently,
to establish the property above,
it suffices to show
$\phi_\kw{rb}^\kw{min} :
  \Gamma_\kw{rb}
  \le_{\varnothing \twoheadrightarrow \mathcal{C} \mathbin@ R_\kw{rb}}
  \kw{Clight}(\kw{rb.c})
$.
In other words,
we can prove the correctness of $\kw{rb.c}$
in the context of a minimal memory share
which contains only the variables $\kw{buf}$, $\kw{c1}$ and $\kw{c2}$.
We can then use the Clight frame property for $\kw{rb.c}$
and the absorption property
$z : \varnothing \sqsubseteq \varnothing \sepconj \langle\kw{mem}]^*$
to derive
$
  \phi_\kw{rb} :=
  \big(
    \phi_\kw{rb}^\kw{min} \mathbin@ \langle mem ]^*
    \vcomp
    \kw{FP}(\kw{rb.c})
  \big) \odot z
$.
\end{example}
%}}}

\subsection{Encapsulated State} \label{sec:overview:encap} %{{{

%The constructions we have introduced so far
%make it possible to manage global state
%and control interference between components,
%but do not support true encapsulation.
Finally,
in \S\ref{sec:encap}
we show how our model can be extended
with persistent component-local state.
%
%inspired by \citet{feedback,caots}.
%Concretely,
%a component $\Sigma : A \rightarrow B$
%consists of a set $U$ of \emph{private states},
%an underlying transition system of type
%$L : A \twoheadrightarrow B \mathbin@ U$,
%and an initial state $u \in U$.
%When $\Sigma$ is activated for the first time,
%the initial state is adjoined to incoming question
%to activate $L$.
%When $L$ terminates,
%the updated state is saved
%to be used with the next activation.
%
%\paragraph{Encapsulation Primitive} %{{{

The compositional structure and constructions we have described
embed into the extended model.
%
In addition,
the model supports an \emph{encapsulation} primitive
$
  [ u \rangle : U \lensarrow \mathbbm{1}
$.
When this component is activated
by an incoming question $* \in \mathbbm{1}$,
the initial state $u \in U$ is used
as an outgoing question.
When an answer $u' \in U$ is received,
the component stores $u'$ as the next state
and returns control to the caller.
This allows $[ u \rangle$ to act
as a state encapsulation primitive.

%}}}

\paragraph{Representation Independence} %{{{

The hallmark of state encapsulation is
the idea that two component which exhibit
the same externally observable behavior
should be indistinguishable,
even if their internal details differ.
Within our framework,
this follows from the property:
\[
  \zeta : u \mathbin{R} v
  \quad\Longrightarrow\quad
  [\zeta\rangle : [u\rangle \le_{R \twoheadrightarrow \mathbbm{1}} [v\rangle
\]
[XXX: incorporate explanation from POPL author response]

%}}}

\paragraph{Implementing Encapsulated State} %{{{

Encapsulated state must eventually be realized as global state.
The conjoint simulation convention
$[u\rangle_* : \mathbbm{1} \leftrightarrow U$
can be used to express this.
Concretely, $[u\rangle_*$ requires the first target question
to carry the value $u$.
When the question is answered with a new state~$u'$,
this new state replaces $u$.
The next question is
expected to carry the value $u'$,
and so on.
In other words,
the simulation convention $[u\rangle_*$
requires the target system to be provided with a state component of type $U$,
maintained across successive activations and
initially set to the value $u$.

%}}}

%}}}

\begin{example} %{{{
The component
$\Gamma'_\kw{bq} :=
 (\mathcal{C} \mathbin@ [\epsilon\rangle) \odot \Gamma_\kw{bq} :
 \top \twoheadrightarrow \mathcal{C}$
describes the behavior of an initially empty bounded queue.
The set of abstract states $D_\kw{bq}$ is used to define it,
but is not exposed as part of its interface,
so that client code will only observe call traces
where state is implicit:
\[
  \Gamma'_\kw{bq} \: \vDash \:
    \kw{enq}(v_1) \cdot
    {*} \cdot
    \kw{enq}(v_2) \cdot
    {*} \cdot
    \kw{deq}() \cdot
    v_1 \cdot
    \kw{enq}(v_3) \cdot
    {*} \cdot
    \kw{deq}() \cdot
    v_2 \, \cdots
\]
Likewise,
we can use
$d_0 := (\{{*},{*},\ldots\}, 0, 0) \in D_\kw{rb}$
to define
$\Gamma'_\kw{rb} :=
 (\mathcal{C} \mathbin@ [d_0\rangle) \odot \Gamma_\kw{rb} :
 \top \twoheadrightarrow \mathcal{C}$
as an encapsulated specification for
the ring buffer data structure.
Representation independence, together with the fact
$\zeta_\kw{bq} : \epsilon \mathbin{R_\kw{bq}} d_0$
that the initial states are related,
means that we can prove:
\[
  \phi_1' \: := \:
  (\mathcal{C} \mathbin@ [\zeta_\kw{bq}\rangle) \odot \phi_1
  \: : \:
  \Gamma'_\kw{bq} \: \le \: \Sigma_\kw{bq} \odot \Gamma'_\kw{rb}
\]
That is,
state encapsulation not only makes it easier
to interface $\Sigma : \mathcal{C} \twoheadrightarrow \mathcal{C}$
with $\Gamma'_\kw{rb} : \top \twoheadrightarrow \mathcal{C}$,
but it also means the simulation
$\phi_1'$ can be stated in terms of the identity simulation convention.

Next, consider the implementation
of $\Gamma'_\kw{rb}$
by $\kw{rb.c}$
in terms of concrete memory.
The initial memory share $m_0 := \kw{init\_mem}(\kw{rb.c})$
associated with $\kw{rb.c}$ satisfies
$\zeta_\kw{rb} : d_0 \mathrel{R_\kw{rb}} m_0$.
This allows us to prove:
\[
  \phi_\kw{rb}' \: := \:
    \Big(
      \mathcal{C} \mathbin@
      \langle \kw{mem} ]^* \sepconj
      \big(
        [\zeta_\kw{rb} \rangle \vcomp [m_0\rangle_\triangledown
      \big)
    \Big) \odot \phi_\kw{rb}
  \: : \:
  \Gamma'_\kw{rb}
    \le_{\varnothing \twoheadrightarrow
      \mathcal{C} \mathbin@
        \langle m_0 \rangle}
    \kw{Clight}(\kw{rb.c})
  \,,
\]
where the simulation convention component
$\langle m_0 \rangle :=
 \langle \kw{mem} ]^* \sepconj [m_0\rangle_* :
 \mathbbm{1} \leftrightarrow \kw{mem}$
expresses the idea that
the memory state introduced at the target level is split into two halves.
One half will contain $\kw{buf}$, $\kw{c1}$ and $\kw{c2}$;
it must be initialized to $m_0$
and preserved by the environment from one call to the next.
The other half is unconstrained
but is guaranteed to be left unchanged by $\kw{rb.c}$.

Since the client component
$\Sigma_\kw{bq} \mathbin@ \kw{mem}$,
by construction,
does not affect the memory at all,
this incoming simulation convention
can easily be incorporated into the property:
\[
  \phi_2' \: := \:
    (\Sigma_\kw{bq} \mathbin@ \langle m_0 \rangle)
    \vcomp
    \phi_2
  \: : \:
  \Sigma_\kw{bq}
    \le_{\mathcal{C} \mathbin@ \langle m_0 \rangle
         \twoheadrightarrow
         \mathcal{C} \mathbin@ \langle m_0 \rangle}
    \kw{Clight}(\kw{bq.c})
\]
Revisiting the challenge articulated in Example~\ref{ex:abspec},
we can then give the following proof:
\[
  \phi_1'
  \:\vcomp\:
  (\phi_2' \vcomp \pi_\kw{bq}) \odot
  (\phi_\kw{rb}' \vcomp \pi_\kw{rb}) \odot z
  \:\vcomp\:
  \ell
  \quad : \quad
  \Gamma_\kw{bq}'
  \:
  \le_{\varnothing \twoheadrightarrow
       (\mathcal{C} \mathbin@ \langle m_0 \rangle) \vcomp \mathbb{C}}
  \:
  \kw{Asm}(\kw{bq.s} + \kw{rb.s})
\]
\end{example}
%}}}

\subsection*{Encapsulated State} \label{sec:encap} %{{{

When a transition system $L : A \twoheadrightarrow B$
performs an outgoing call in $A$,
the internal state $s$ is preserved
until an answer resumes the execution.
However,
no state is preserved between incoming calls in~$B$.
Each question $q \in B^\que$ initializes a fresh state $s$
such that $q \mathrel{I} s$
regardless of any previous calls. % made into $L$.

To allow a component to retain state across calls,
we could modify the definition of transition systems
to include a persistent state $K$ with an initial value $\intl{k}$.
The initial and final state predicates
\begin{equation} \label{eqn:psts}
  \intl{k} \in K
  \qquad
  I \subseteq K \times B^\que \times S
  \qquad
  F \subseteq S \times B^\ans \times K
\end{equation}
could then access and update this persistent state,
with the understanding that on the first activation,
the persistent state $\intl{k}$ would be used for initialization,
and that subsequently the updated value produced by $F$
in the context of one activation
would be used to initialize the next one.

\paragraph{Components} %{{{

Luckily,
the situation we have described above
can already be encoded in our model with minimal effort
by using the language interface $B \mathbin@ K$:
we define a component with persistent state as a tuple
$(\intl{k} \in K \mid L)$
where
$L : A \twoheadrightarrow B \mathbin@ K$.
A similar approach can be used for lenses.

\begin{definition} \label{def:slts} %{{{
A \emph{persistent transition system}
$(u \in U \mid L) : A \rightarrow B$
consists of:
\begin{itemize}
  \item a set of states $U$ with a distinguished initial state $u \in U$;
  \item a transition system $L : A \rightarrow B \mathbin@ U$.
\end{itemize}
The identity persistent transition system for $A$ is defined as
$\kw{id}_A := (* \in \mathbbm{1} \mid \kw{id}) : A \rightarrow A$
and the composition of
$(u \mid L_1) : B \rightarrow C$ and
$(v \mid L_2) : A \rightarrow B$ is defined as:
\[
  (u \in U \mid L_1) \,\odot\, (v \in V \mid L_2) \::=\:
  \big( (u, v) \in U \times V \mid
        (L_1 \mathbin@ V) \odot L_2 \big)
  \::\: A \rightarrow C
\]
A similar construction applies to lenses,
and the spatial composition of the persistent transition system
$(u \in U \mid L) : A \rightarrow B$
with the persistent lens
$(v \in V \mid f) : X \rightarrow Y$
can be given as
\[
  (u \in U \mid L) \mathbin@ (v \in V \mid f) :=
  \big( (u, v) \in U \times V \mid
    (B \mathbin@ \gamma_{U,X} \mathbin@ V) \odot
    (L \mathbin@ f) \big)
  \,.
\]
A functorial embedding
maps $L : A \twoheadrightarrow B$
to its persistent counterpart $(* \in \mathbbm{1} \mid L) : A \rightarrow B$.
In addition,
the new framework allows us to define the encapsulation primitive:
\[
  [ u \rangle \: := \:  (u \in U \mid \kw{id}_U)
  \: : \: U \rightarrow \mathbbm{1}
  \,.
\]
\end{definition}
%}}}

%}}}

%Once private state has been encapsulated,
%in principle it can only be observed by the environment
%through the way the transition system responds
%to successive queries.
%In particular,
%constructions on stateful components
%should preserve the following notion of equivalence.
%
%\begin{definition}[Simple Simulation] \label{def:ssim}
%We will say that $\Sigma_1 : A \rightarrow B$
%is refined by $\Sigma_2 : A \rightarrow B$
%and write $\Sigma_1 \preceq \Sigma_2$
%when there exists a relation $R \subseteq K_1 \times K_2$
%such that:
%\begin{itemize}
%  \item the initial states $\intl{k}_1, \intl{k}_2$ are related by $R$;
%  \item the transition systems satisfy
%    $L_1 \le_{\kw{id}_A \twoheadrightarrow B@R} L_2$.
%\end{itemize}
%We write $\Sigma_1 \equiv \Sigma_2$ when
%$\Sigma_1 \preceq \Sigma_2$ and
%$\Sigma_2 \preceq \Sigma_1$.
%\end{definition}

%}}}

%\subsection{Composition} %{{{
%
%The composite $\Sigma_1 \circ \Sigma_2$
%uses states of type $K_1 \times K_2$.
%Each side of the pair is updated
%when the corresponding component is active.
%Incoming questions in $C$ are routed to $L_1 : B \twoheadrightarrow C@K_1$,
%which we lift to pass through an additional state component of type $K_2$.
%Outgoing questions of $L_1$ in $B$ can then be routed to $L_2$,
%as depicted in the following diagram:
%\[
%  \begin{tikzpicture}[yscale=0.5]
%    \draw (0,2) node[left] {$K_2$} -- (2,2) .. controls +(0.5,0) and +(-0.5,0) .. (3,1);
%    \draw (0,1) node[left] {$K_1$} -- (1,1);
%    \draw (0,0) node[left] {$C$} -- (1,0);
%    \draw (2,0) -- node[below] {$B$} (3,0);
%    \draw (4,0) -- (5,0) node[right] {$A$};
%    \draw (1,1.5) rectangle node {$L_1$} (2,-0.5);
%    \draw (3,1.5) rectangle node {$L_2$} (4,-0.5);
%  \end{tikzpicture}
%\]
%Formally,
%composition is defined as follows.
%
%\begin{definition}[Composition] \label{def:slcomp}
%Given two stateful components
%$\Sigma_1 = (K_1 \mid L_1) : B \rightarrow C$ and
%$\Sigma_2 = (K_2 \mid L_2) : A \rightarrow B$,
%we define their composition
%$\Sigma_1 \circ \Sigma_2 : A \rightarrow C$
%in the following way:
%\[
%  \Sigma_1 \circ \Sigma_2 :=
%    ( K_1 \times K_2 \mid L_1@K_2 \odot L_2 )
%\]
%Note that this formulation
%implicitly uses the isomorphism
%\[
%  L_1@K_2 : B@K_2 \twoheadrightarrow (C@K_1)@K_2 \cong C@(K_1 \times K_2)
%  \,.
%\]
%\end{definition}
%
%
%
%
%\begin{lemma}
%  This is compatible with:
%  \begin{itemize}
%    \item simple simulations;
%    \item lifting hence associativity is preserved.
%  \end{itemize}
%
%\end{lemma}
%
%%}}}

%\subsection{Combining Hidden and Explicit State} %{{{
%
%It is possible for a stateful component
%to hide only part of the explicit state
%used by the underlying transition system.
%For example,
%in a transition system
%$L : A \rightarrow B@(K \times U) \cong B@U@K$,
%we may choose to hide only $K$,
%to obtain a stateful component
%$\Sigma = (K \mid L) : A \rightarrow B@U$
%which still exchanges explicit state of type $U$
%with the environment.
%This makes it possible to construct
%complex stateful components incrementally
%from simpler, stateless parts,
%by first promoting them using $\&$,
%then hiding state in a step-by-step fashion
%using $\kw{fbk}$.
%
%In addition,
%stateful components can be lifted
%to pass through explicit state.
%
%\begin{definition} \label{def:slift}
%The stateful component $\Sigma = (K \mid L) : A \rightarrow B$
%can be lifted to \[ \Sigma@U := (K \mid L@U) : A@U \rightarrow B@U \,. \]
%Note that this relies on the isomorphism
%\[
%  L@U : A@U \twoheadrightarrow (B@K)@U \cong (B@U)@K
%  \,.
%\]
%\end{definition}
%
%The following properties
%relate the promotion operator $\&$,
%the composition operators $\odot$ and $\circ$,
%and the hiding operator $\kw{fbk}$.
%
%\begin{lemma}
%  For $L_1 : B \twoheadrightarrow C@K_1$ and
%  $L_2 : A \twoheadrightarrow B@K_2$,
%  the following property holds:
%  \[
%    \&(L_1@K_2) \circ \&L_2 \equiv \&(L_1@K_2 \odot L_2)
%    \,.
%  \]
%  In addition, for $\Sigma_1 : B \rightarrow C@U_1$
%  and $\Sigma_2 : A \rightarrow B@U_2$,
%  the following property holds:
%  \[
%    \kw{fbk}_{U_1}(\Sigma_1) \circ \kw{fbk}_{U_2}(\Sigma_2) \equiv
%    \kw{fbk}_{U_1 \times U_2}(\Sigma_1@U_2 \circ \Sigma_2)
%    \,.
%  \]
%\end{lemma}
%
%%}}}

%\paragraph{Components} %{{{
%
%A component $L_1 : B \twoheadrightarrow C \in \mathbf{TS}$
%can interact with %a component
%$L_2 : A \twoheadrightarrow B \mathbin@ U$
%as the composite $(L_1 \mathbin@ U) \odot L_2 : A \twoheadrightarrow C \mathbin@ U$.
%This situation guarantees by construction that
%$L_1$ cannot interfere with the state component of type $U$
%used by $L_2$.
%Our approach to state encapsulation
%is to force the environment to
%interact with components in this manner.
%
%In the extended model,
%a component $L : A \rightarrow B \in \mathbf{TS}^\dagger$
%is defined using an underlying transition system
%of type $A \twoheadrightarrow B \mathbin@ U \in \mathbf{TS}$,
%where we have specified an
%additional state component $U \in \mathbf{Lens}$
%for the encapsulated states of $L$.
%
%}}}

\paragraph{Simulations} %{{{

Simulations of persistent transition systems
can be defined as simulations of the underlying components.
However,
to take into account the components' encapsulated states,
we must first incorporate those states into
the incoming simulation convention.

\begin{definition}
For $u \in U$,
the simulation conventions
$[ u \rangle^*$ and
$[ u \rangle_*$ are defined as:
\[
  [ u \rangle^* :=
    \big\langle U,\, u,\, {=},\, \top,\, \Gamma_U,\, \Gamma_U \big\rangle
    : U \leftrightarrow \mathbbm{1}
  \quad \text{and} \quad
  [ u \rangle_* :=
    \big\langle U,\, u,\, {=},\, \top,\, \Gamma_U^\kw{op},\, \Gamma_U^\kw{op} \big\rangle
    : \mathbbm{1} \leftrightarrow U
  \,,
\]
where $\Gamma_X \in \mathcal{R}_X(X, \mathbbm{1})$
is the smallest Kripke relation containing $x \Vdash x \mathrel{\Gamma_X} *$ for all $x \in X$.
\end{definition}

These elementary simulation conventions
internalize a state component of type $U$ in their Kripke worlds,
and express the guarantees provided by the environment
for encapsulated state:
on the first call, the source (resp. target) component
must be passed the initial state $u \in U$.
The component can modify the state
and perform a callee world transition
in $[ u \rangle^*$ (resp. $[ u \rangle_*$).
The caller cannot update the world between calls.
Once we can express these constraints,
defining simulations is easy.

\begin{definition} \label{def:ssim} %{{{
Simulations of persistent components are defined by
\[
  (u \in U \mid L_1)
  \le_{\mathbf{R} \rightarrow \mathbf{S}}
  (v \in V \mid L_2)
  \quad:\Leftrightarrow\quad
  L_1
  \le_{\mathbf{R} \twoheadrightarrow
       \mathbf{S} \mathbin@ ([ u \rangle^* \vcomp [ v \rangle_*)}
  L_2
  \,.
\]
\end{definition}
%}}}

Simulations defined as above compose horizontally and vertically,
and we recover for components with encapsulated state
the full structure of the original framework
outlined in Fig.~\ref{fig:xcomp}.

%}}}

%}}}

\section{Applications and Evaluation} \label{sec:app} %{{{

We now showcase further applications of our framework
and discuss its implementation.

\subsection{CompCertO semantics} %{{{

We first show how the CompCertO semantics
can be embedded into our framework,
thus allowing us to implement
the specifications with C programs.

The language interfaces for CompCertO
can be viewed as a coarse-grained form
of the effect signatures,
where the set of questions and answers
includes all the syntactically correct function calls
and return values.
Thus,
a language interface $A := \langle A^\que, A^\ans \rangle$
has its corresponding
effect signature $\llbracket A \rrbracket := \{ q : A^\ans \mid q \in A^\que \}$.

The semantics model present in this work
has no internal transitions,
thus we need to hide
the internal steps of CompCertO.
Given the CompCertO semantics
$L: A \twoheadrightarrow B = \langle S, \rightarrow, I, X, Y, F \rangle$,
we define the embedded semantics as
$\llbracket L \rrbracket : \llbracket A \rrbracket \rightarrow \llbracket B \rrbracket =
\langle S, I', X, Y', F \rangle$
where the internal steps transitions
are absorbed into
the initializations $I'$,
and the resumption $Y'$ as follows:
\[
  \begin{prooftree}
    \hypo{q_B \,I\, s}
    \hypo{s \rightarrow^* s'}
    \infer2{q_B \,I'\, s'}
  \end{prooftree}
  \qquad
  \begin{prooftree}
    \hypo{r_A \,Y^{s}\, s'}
    \hypo{s' \rightarrow^* s''}
    \infer2{r_A \,Y'^{s}\, s''}
  \end{prooftree}
\]

The embedding preserves the simulation relation.


Back to the bounded queue example in~\ref{ex:bq-proof},
the specification $\Gamma_\kw{bq}$ is implemented by
composition of two C components
$\llbracket \kw{Clight}(\kw{bq.c}) \rrbracket \odot
\llbracket \kw{Clight}(\kw{rb.c}) \rrbracket$.
Using the compiler correctness,
this is refined by
$\llbracket \kw{Asm}(\kw{bq.s}) \rrbracket \odot
\llbracket \kw{Asm}(\kw{rb.s}) \rrbracket$.
To further link the compiled programs,
we prove the embedding interacts with
CompCertO's horizontal composition
as follows:
\[
  \llbracket L_1 \rrbracket \odot \llbracket L_2 \rrbracket
    \sqsubseteq  \llbracket L_1 \oplus L_2 \rrbracket
  \qquad \text{for all} \; L_1, L_2 \in A \twoheadrightarrow A
\]
In particular, when combined with CompCertO's linking property
of assembly programs,
this means
\[
  \llbracket \kw{Asm}(\kw{bq.s}) \rrbracket
  \odot \llbracket \kw{Asm}(\kw{rb.s}) \rrbracket
  \sqsubseteq \llbracket \kw{Asm}(\kw{bq.s} + \kw{rb.s}) \rrbracket
\]
Thus, we have shown the
specification $\Gamma_\kw{bq}$
is implemented by the assembly programs
$\kw{bq.s}$ and $\kw{rb.s}$.


We omit the brackets for the open semantics
when the context is clear.


%}}}

\subsection{Modeling loading and the execution environments}

\begin{example} %{{{
The behavior of a CompCert program as a process
once it is \emph{loaded} and executed is given by:
\[
  \begin{prooftree}
    \hypo{L : \mathcal{C} \rightarrow \mathcal{C}}
    \infer1{\mathsf{load}_\mathcal{C}(L) : \mathcal{E} \rightarrow \mathcal{P}}
  \end{prooftree}
  \qquad
  \begin{prooftree}
    \hypo{L : \mathcal{C} \rightarrow \mathcal{C}}
    \infer1{\mathsf{load}_\mathcal{C}(L) : \mathcal{E} \rightarrow \mathcal{P}}
  \end{prooftree}
  \qquad
  \begin{prooftree}
    \hypo{L_1 \le_{\mathbb{C} \rightarrow \mathbb{C}} L_2}
    \infer1{\kw{load}_\mathcal{C}(L_1) \le \kw{load}_\mathcal{A}(L_2)}
  \end{prooftree}
\]
\end{example}

The CompCertO semantics explores
the intricate interaction among individual components,
which are eventually linked
through the linking property.
However, even after this linking,
the behavior of the resulting program remains
somewhat unclear.
It requires invocation
through a query
with a specific program state
and may entail interactions
with the environment.
To bring clarity to this process
and model the semantics of linked programs
akin to a process
running within an operating system,
we introduce the notion of a \textit{loader}.

On one end,
the loader uses
$\kw{entry}_\mathcal{A} : \mathcal{A} \rightarrow \mathcal{P}$
to invoke the main function.
\[
  \kw{entry}_\mathcal{A} \:\vDash\:
  \kw{run} \rightarrowtail
  (\vec{rs_0}[\kw{PC}\mapsto (b_\kw{main}, 0),
  \kw{RA} \mapsto \kw{null},
  \kw{RSP}\mapsto \kw{null}]@m_0 \leadsto
  \vec{rs}[\kw{RAX} \mapsto r]@m) \rightarrowtail r \\
\]
The registers $\vec{rs_0}$ and the memory $m_0$ are
initialized that
the program counter $\kw{PC}$ points to
the block $b_\kw{main}$
associated to the main function in the symbol table,
and the static variables
are properly initialized in the memory.
The return address $\kw{RA}$
and the stack pointer $\kw{RSP}$
are initialized to $\kw{null}$
according to the CompCertO's simulation convention.
At the end,
the value stored in the $\kw{RAX}$ is returned.

On the other end,
the component
$\kw{runtime} : \mathcal{E} \rightarrow \mathcal{C}$
acts as the conduit for runtime libraries
to interface the program with the operating system.
In our scenario,
the programs use $\kw{read}$ and $\kw{write}$
functions from $\kw{unistd.h}$
to perform I/O operations.
Thus, we implement the following minimalist runtime.

\begin{align*}
  \kw{runtime}_\mathcal{A} & \:\vDash\:
  \vec{rs}[\kw{PC} \mapsto (b_\kw{read}, 0),
    \kw{RDI} \mapsto 0,
    \kw{RSI} \mapsto (b, ofs),
    \kw{RDX} \mapsto n]@m \\
  & \rightarrowtail (\kw{read}_0[n] \leadsto data)
    \rightarrowtail \vec{rs'}[\kw{RAX} \mapsto \kw{len}(data)]@m[(b, ofs) \mapsto data] \\
  \kw{runtime}_\mathcal{A} & \:\vDash\:
    \vec{rs}[\kw{PC} \mapsto (b_\kw{write}, 0),
    \kw{RDI} \mapsto 1,
    \kw{RSI} \mapsto (b, ofs),
    \kw{RDX} \mapsto n]@m[(b, ofs) \mapsto data] \\
  & \rightarrowtail
  (\kw{write}_0[data[0..n]] \leadsto n')
  \rightarrowtail \vec{rs'}[\kw{RAX} \mapsto n']@m
\end{align*}

Following the x86 conventions,
arguments are passed through
the \kw{RDI}, \kw{RSI}, and \kw{RDX} registers.
The call to the \kw{read} function
stores the data read from the file descriptor 0
into the memory at the address $(b, ofs)$,
and returns the length of the data read.
The call to the \kw{write} function
writes the first $n$ bytes of data, $data[0..n]$,
from the memory at $(b, ofs)$
to the file descriptor 1.

Overall, the loader
for the CompCertO assembly semantics
$L_\mathcal{A}: \mathcal{A} \rightarrow \mathcal{A}$
is defined as
\[
  \kw{load}_\mathcal{A}(L_\mathcal{A}) : \mathcal{E} \rightarrow \mathcal{P}
  \::=\: \kw{entry}_\mathcal{A} \odot L_\mathcal{A}
  \odot \kw{runtime}_\mathcal{A}
\]

The loader for
the CompCertO's C semantics
$L_\mathcal{C}: \mathcal{C} \rightarrow \mathcal{C}$
is defined similarly
with the corresponding entry and runtime components.
\[
  \kw{entry}_\mathcal{C} \:\vDash\:
  \kw{run} \rightarrowtail (\kw{main}()@m_0 \leadsto r@m) \rightarrowtail r
\]
\begin{align*}
  \kw{runtime}_\mathcal{C} & \:\vDash\:
  \kw{read}(0, (b, ofs), n)@m \rightarrowtail
  (\kw{read}_0[n] \leadsto data)
  \rightarrowtail \kw{len}(data)@m[(b, ofs) \mapsto data] \\
  \kw{runtime}_\mathcal{C} & \:\vDash\:
  \kw{write}(1, b, n)@m[(b, ofs) \mapsto data] \rightarrowtail
  (\kw{write}_0[data[0..n]] \leadsto n')
  \rightarrowtail n'@m
\end{align*}
\[
  \kw{load}_\mathcal{C}(L_\mathcal{C}) : \mathcal{E} \rightarrow \mathcal{P}
  \::=\: \kw{entry}_\mathcal{C} \odot L_\mathcal{C}
  \odot \kw{runtime}_\mathcal{C}
\]

% TODO: more on why we need the C & Asm loaders
% TODO: similarity on the loader and embedding
% both go from C to Asm somehow


%}}}

\subsection{Certified Abstraction Layers} \label{sec:application:cal} %{{{

The bounded queue example in \S\ref{sec:overview} was ad-hoc
and relied on our framework as a versatile glue.
However, in many contexts additional structure is preferable.
The methodology of \citet{popl15}
divides the code of a large system into standardized
\emph{certified abstraction layers}.
The functionality exposed to client code at each layer
is specified in a \emph{layer interface}.
Within the terms of our formalism,
a layer interface is
a set $D$ of \emph{abstract states}
together with a specification
$\Gamma : \top \twoheadrightarrow \mathcal{C} \mathbin@ \kw{mem} \mathbin@ D$.
%and an initial state $d \in D$.
The semantics of client code then takes
this \emph{underlay} interface as a parameter:
\begin{equation} \label{eqn:layersem}
  %\forall D \in \mathbf{Set}
  %\: \mathbin. \:
  \Gamma :
    \top \twoheadrightarrow
    \mathcal{C} \mathbin@ \kw{mem} \mathbin@ D
  \quad \vdash \quad
  \kw{Clight}_{\Gamma}[M] :
    \top \twoheadrightarrow
    \mathcal{C} \mathbin@ \kw{mem} \mathbin@ D
  \,.
\end{equation}
A certified abstraction layer
involves an \emph{underlay} interface $\Gamma_1$,
an \emph{overlay} interface $\Gamma_2$,
a program module $M$
and an abstraction relation $R \subseteq D_2 \times (D_1 \times \kw{mem})$.
They must satisfy the property:
\[
  \Gamma_1 \vdash_R M : \Gamma_2
  \quad :\Leftrightarrow \quad
  \Gamma_2 \: \le_{\top \twoheadrightarrow \mathcal{C} \mathbin@ \hat{R}} \:
    \kw{Clight}_{\Gamma_1}[M]
  \,,
\]
where
$\hat{R} \subseteq (\kw{mem} \times D_2) \times (\kw{mem} \times D_1)$
extends $R$ to a relationship between
the entire states of the source and target programs.
The main challenge is then to prove the vertical composition property
\begin{equation} \label{eqn:calvcomp}
  \begin{prooftree}
    \hypo{\psi_{12} \: : \: L_1 \vdash_R M : L_2}
    \hypo{\psi_{23} \: : \: L_2 \vdash_S N : L_3}
    \infer2{\psi_{13} \: : \: L_1 \vdash_{R \cdot S} M + N : L_3}
  \end{prooftree}
\end{equation}

\paragraph{Implementing Layers}

This methodology is implemented in CompCertX,
a~modified version of CompCert where
every language semantics and correctness proof
has been updated
to take into account the abstract state and underlay interface.
A complex \emph{memory injection} is used in $\hat{R}$
to express the embedding of the source memory into the target,
alongside the concretized abstract state of the overlay.
Finally, the proof of vertical compositionality
is complex and largely monolithic,
involving aspects of
our frame property,
CompCertO's linking theorem, and more.
%it requires thousands of lines of Coq code.

By contrast,
the toolbox provided by our framework
makes it straightforward to formulate a comparable theory
of certified abstraction layers.
A layer-aware semantics can be defined as:
\[
  \kw{Clight}_\Gamma[M] := (\kw{Clight}(M) \mathbin@ D) \odot \Gamma
%    =
%  {\begin{tikzpicture}[xscale=0.25,yscale=0.12,baseline=(mem.base)]
%    % Background
%    \fill[tssdbg]
%      (1,-2) rectangle (14,7);
%    \fill[act]
%      (1,-2) rectangle (14,0);
%    % Wires
%    \draw (1,0) node[left] {\small$\mathcal{C}$}
%      -- (14,0) node[right] {\small$\top\,,$};
%    %\draw (0,2) node[left] {$\kw{mem}$}
%    %  -- (9,2) .. controls +(1,0) and +(-1,0) .. (11,5)
%    %  -- (13,5)
%    %    node[circle,inner sep=1pt,draw,fill=white] {}
%    %    node[right] {$\kw{mem}$};
%    \draw (1,2) node[left] (mem) {\small$\kw{mem}$}
%      -- (9,2) .. controls +(1,0) and +(-1,0) .. (10.5,1) -- (12,1);
%    \draw (1,4.5) node[left] {\small$D$}
%      -- (9,4.5) .. controls +(1,0) and +(-1,0) .. (10.5,2)
%      -- (11,2);
%    %\node[above,inner sep=1pt] at (10,0)
%    %  {\footnotesize $\mathcal{C}$};
%    % Boxes
%    \draw[fill=white,rounded corners]
%      ( 2,-1) rectangle ( 8,3) node[midway] {\small$\kw{Clight}(M)$}
%      (11,-1) rectangle (13,3) node[midway] {\small$\Gamma$};
%  \end{tikzpicture}}
\]
and does not require any compiler change.
Our memory join relation can be leveraged to define:
\[
  \hat{R} := 
    (\kw{mem} \mathbin@ R) \odot (\jr \mathbin@ D_1)
  \qquad
  R \cdot S :=
    S \vcomp (\kw{mem} \mathbin@ R) \vcomp (\jr \mathbin@ D_1)
\]
such that the composition property
$\alpha : (\hat{S} \vcomp \hat{R}) \cong \widehat{R \cdot S}$
holds by associativity of the join operation $\bullet$.
Finally,
the vertical composition property (\ref{eqn:calvcomp})
can be established with the single-line proof term:
\[
  \psi_{13} \: := \:
    \alpha \odot \big(
    \psi_{23} \vcomp
    \big( (\kw{Clight}(N) \mathbin@ R) \vcomp
          (\kw{FP}(N) \mathbin@ D_1) \big) \odot \psi_{12} \big)
  \,.
\]
We provide additional details in Appendix~B. %\ref{app:cal}.

%}}}

\subsection{Clight with Module-Local State} \label{sec:application:clightp} %{{{

Beyond verification-oriented applications,
incorporating state encapsulation into CompCert semantics
opens the door to new language features.
As an example,
we have defined a language called ClightP
which supports encapsulated module-local state
and can be soundly compiled to Clight.

\paragraph{Semantics}

In ClightP,
global variables can be declared \emph{private}.
Private variables cannot be accessed
from other translation units and are stored
in a separate
\emph{private environment} $p \in \kw{penv}$.
The semantics of a ClightP program $M$
are defined using an underlying transition system of type:
\[
  \kw{ClightP}(M) :
    \mathcal{C} \mathbin@ \kw{mem}
    \twoheadrightarrow
    \mathcal{C} \mathbin@ \kw{mem} \mathbin@ \kw{penv}
\]
We can then extract from the program $M$
the initial private environment $p_0 = \kw{init\_penv}(M)$
and obtain the encapsulated semantics
$\kw{ClightP}\langle M \rangle :
    \mathcal{C} \mathbin@ \kw{mem} \rightarrow
    \mathcal{C} \mathbin@ \kw{mem}$ as:
\[
  \kw{ClightP} \langle M \rangle :=
%\:
%  \begin{tikzpicture}[xscale=0.35,yscale=0.17,baseline=(m.base)]
%    % Background
%    \fill[tssdbg] (1,-2) rectangle (11,6.5);
%    \fill[act] (1,-2) rectangle (11,0);
%    % Wires
%    \draw (1,0) node[left] {$\mathcal{C}$}
%      -- (11,0) node[right] {$\mathcal{C}$};
%    \draw (1,2) node[left] (m) {$\kw{mem}$}
%      -- (11,2) node[right] {$\kw{mem}$};
%    \draw (3,4)
%        node[circle,inner sep=1pt,draw,fill=white] {}
%        node[left] {$p_0$}
%        -- (4,4);
%    % Boxes
%    \draw[fill=white,rounded corners]
%      ( 4,-1) rectangle (10,5) node[midway] (N) {$\kw{ClightP}(M)$};
%  \end{tikzpicture} \: =
  (\mathcal{C} \mathbin@ \kw{mem} \mathbin@ [ p_0 \rangle) \odot
    \kw{ClightP}(M)
  \,.
\]
Note that the resulting type
means that ClightP semantics in this form
can be composed directly.

\paragraph{Compiling to Clight}

We have defined a simple transformation $M' := \kw{ClightUnP}(M)$
which turns a ClightP program $M$ into a regular Clight program $M'$
by erasing the $\kw{private}$ annotations from all variables.
We can then show the associated correctness property:
\[
  \kw{ClightP}\langle M \rangle \le_{
    \mathcal{C} \mathbin@ \kw{mem} \sepconj \langle \kw{mem} ]^*
    \twoheadrightarrow
    \mathcal{C} \mathbin@ \kw{mem} \sepconj [ \kw{m_0} \rangle_*}
  \kw{Clight}(M')
  \,,
\]
where $m_0$ is a memory share computed from $M$
containing the initial values of its private variables.
The incoming simulation convention $\kw{mem} \sepconj [m_0\rangle_*$
requires $m_0$ to be added to the target global memory state.
The outgoing convention $\kw{mem} \sepconj \langle \kw{mem} ]^*$
allows the target program to include this additional memory region
into its outgoing calls,
with a guarantee that it will not be changed.

% The correctness property
% can be decomposed into two parts:
% \[
%   \mathcal{C} \mathbin@ \kw{mem} \mathbin@ [p_0 \rangle \le_{
%     \mathcal{C} \mathbin@ \kw{mem} \bullet R_p
%     \twoheadrightarrow
%     \mathcal{C} \mathbin@ \kw{mem} \bullet [ \kw{m_0} \rangle_* }
%   \mathcal{C} \mathbin@ \kw{mem}
% \]
% and
% \[
%   \kw{ClightP}(M) \le_{
%     \mathcal{C} \mathbin@ \kw{mem} \bullet \langle \kw{mem} ]^*
%     \twoheadrightarrow
%     \mathcal{C} \mathbin@ \kw{mem} \bullet R_p }
%   \kw{Clight}(M')
% \]
%where the relation $R_p \subseteq \kw{penv} \times \kw{mem}$
%connects the private environment and the memory,
%and becomes hidden in the correctness property
%as shown in Figure~\ref{fig:clightp}.

% The first part can be derived from
% \[
%   [p_0 \rangle \le_{R_p \twoheadrightarrow [ \kw{m_0} \rangle_* } \kw{mem}
% \]

\paragraph{Composition}

One challenge is that the correctness property depicted above
is not directly compositional,
because the incoming and outgoing simulation conventions are different.
Fortunately,
the frame property for Clight ensures that
the correctness properties for multiple ClightP translation units
can be combined in a meaningful way.
See Appendix~C for more details. %\ref{sec:appendix:clightp}

%\begin{example}[CAL]
%Certified abstraction layers can be interpreted
%using \kw{ClightP} as
%\[
%  L_1 \vdash M : L_2 :\Leftrightarrow
%  L_2 \le_{\varnothing \twoheadrightarrow \mathcal{C}\mathbin@ [\kw{mem}\rangle_* }
%  \kw{ClightP}\langle M \rangle \odot [\kw{mem}\rangle \odot L_1
%\]
%Then the composition becomes simpler with
%the auxiliary property:
%\[
%  \kw{ClightP}\langle M \rangle \odot [\kw{mem}\rangle \le_{ \mathcal{C} \mathbin@ \kw{mem} \twoheadrightarrow \mathcal{C}\mathbin@ [\kw{mem}\rangle_* }
%  \kw{ClightP}\langle M \rangle
%\]
%\end{example}

% }}}

\subsection{Secure Compilation}

\subsection{Implementation} %{{{

\begin{table}
  \caption {Significant lines of code for each part of the framework}
  \label{tbl:coqwc}
  \small
  \begin{tabular}{lrlr}
    \toprule
    Component & SLOC &
    Component & SLOC\\
    \midrule
    Layered Composition (\S\ref{sec:overview:lcomp}, \S\ref{sec:base}) & 1,029
    &
    Bounded Queue Example (\S\ref{sec:overview}) & 1,691
    \\
    Spatial Composition (\S\ref{sec:overview:scomp}, \S\ref{sec:scomp})& 440
    &
    Certified Abstraction Layers (\S\ref{sec:application:cal}) & 2,247
    \\
    Memory Separation (\S\ref{sec:overview:sepalg}) & 1,151
    &
    ClightP (\S\ref{sec:application:clightp}) & 4,031
    \\
    State Encapsulation (\S\ref{sec:overview:encap}, \S\ref{sec:encap}) & 2,321
    \\
    \midrule
    && Total & 12,910\\
    \bottomrule
  \end{tabular}
\end{table}

We have mechanized all structures, theorems and properties
presented in this paper using Coq.
Our development comprises 12,910 SLOC of Coq,
and the detailed breakdown is given in Table \ref{tbl:coqwc}.
%The artifact is built on top of CompCertO v0.1,
%and made modest changes to the original semantics.

% \subsection{Ad-hoc Verification} \label{sec:application:ad-hoc}

% We have already seen in \S\ref{sec:overview}
% that states have to be explicitly attached to transition systems
% in order to identify the interfaces across different components.
% It immediately becomes overwhelming,
% and great care have to be taken
% in order to put together all the pieces correctly.
% In contrast,
% once the states in the specifications
% are encapsulated by defining
% \[
%   \Gamma_\kw{bq}: \top \rightarrow \mathcal{C} := (\epsilon \in \kw{val}^* \mid L_\kw{bq}) \qquad
%   \Gamma_\kw{rb}: \top \rightarrow \mathcal{C} := ((\bot, 0, 0) \in \kw{val}^N \times \mathbb{N} \times \mathbb{N} \mid L_\kw{rb}),
% \]
% and the static variables in the $\kw{Clight}$ programs are
% converted to private ones
% \[
%   \kw{ClightP} \langle \kw{bq.cp} \rangle : \mathcal{C} @ \kw{mem} \rightarrow \mathcal{C} @ \kw{mem} \qquad
%   \kw{ClightP} \langle \kw{rb.cp} \rangle : \mathcal{C} @ \kw{mem} \rightarrow \mathcal{C} @ \kw{mem},
% \]
% the interfaces across different components are identified
% except for an innocent memory state to be passed through.
% With the individual correctness
% \[
%   \pi_\kw{bq}:\ \Gamma_\kw{bq} \le \Sigma_\kw{bq} \circ \Gamma_\kw{rb} \qquad
%   \pi_\kw{bq'}:\ \Sigma_\kw{bq} @ \kw{m} \le \kw{ClightP}\langle \kw{bq.cp} \rangle \qquad
%   \pi_\kw{rb}:\ \Gamma_\kw{rb} @ \kw{m} \le \kw{ClightP} \langle \kw{rb.cp} \rangle
% \]
% the complete proof can be presented
% as a string diagram in Figure~\ref{fig:application:example}.


% \subsection{Certified Abstraction Layers} \label{sec:application:cal}

% Certified abstraction layers (CAL) framework
% provides a theoretical foundation
% for large scale systems verification,
% by allowing
% the high-level abstract specification
% to be progressively refined
% through a series of abstraction layers.
% The correctness of individual components is denoted by
% \[
%   L^\sharp \vdash_R M : L^\flat,
% \]
% which says that
% the behavior of an overlay specification $L^\sharp: \top \twoheadrightarrow \mathcal{C} @ D^\sharp$
% is refined by an implementation $M: \mathcal{C} \twoheadrightarrow \mathcal{C}$
% executed on top of an underlay $L^\flat: \top \twoheadrightarrow \mathcal{C} @ D^\flat$
% witnessed by the abstraction relation $R \subseteq D^\sharp \times D^\flat$.
% Abstraction layers are composed according to the following rule
% \[
%   \begin{prooftree}
%     \hypo{L^\natural \vdash_R M : L^\sharp}
%     \hypo{L^\flat \vdash_S N : L^\natural}
%     \infer2{L^\flat \vdash_{R \circ S} M + N : L^\sharp}
%   \end{prooftree}
% \]

% However, there lacks a satisfying instance of the framework.
% The closest one uses CompCertX with primitives
% to implementation the semantics of the framework,
% where the layer correctness is interpreted as
% (roughly expressed using our notations)
% \[
%   L^\sharp \vdash_R M : L^\flat :\Leftrightarrow L^\sharp @ \kw{mem} \le_{\varnothing \twoheadrightarrow [R]} \kw{ClightX}_{L^\flat @ \kw{mem}} [M],
% \]
% where $\kw{ClightX_L[M]}$ interprets
% a module $M$ of Clight functions
% with external calls being interpreted
% according to the underlay interface $L$,
% and the simulation relation $[R] \subseteq (D^\sharp \times \kw{mem}) \times (D^\flat \times \kw{mem})$
% allows the abstract state to be realized
% by concrete memory state.
% Memory injection is used so that the abstraction layers
% are composible to the CompCertX compiler.

% Although it has been applied to verify CertiKOS,
% the above framework has two major limitations.
% On one hand,
% the $\kw{ClightX}$ semantics is not truly open
% in that the behaviors of external calls have to be provided by the underlay.
% As a consequence,
% the process of layer construction requires
% the well-formedness of the underlay.
% On the other,
% the composition on abstraction relations
% becomes much complicated
% when it incorporates the memory injection.

% By contrast, with CompCertO semantics and the state encapsulation,
% the abstraction relation can be encapsulated,
% which gives a simpler interpretation of the abstraction layers:
% \[
%   L^\flat \vdash M : L^\sharp :\Leftrightarrow L^\sharp @ \kw{mem} \le \kw{ClightP} \langle M \rangle \circ L^\flat @ \kw{mem}
% \]
% The truly openness of CompCertO semantics
% enables modules to be verified independent of underlay semantics.
% For example, the program $\kw{bq.cp}$ is verified
% with reference to its specification $\Sigma_\kw{bq}$
% without relying on any underlay specifications.
% Moreover,
% since the abstract relation has been encapsulated,
% the vertical composition of certified abstraction layers
% can be established as a corollary of
% the vertical composition of encapsulated semantics.
% Furthermore,
% with the ability of combining memory fragments,
% a form of horizontal composition of
% abstraction layers can be established
% in a straightforward way,
% which would be complicated due to
% the intricacy between the abstraction relation
% and the memory state.

%}}}

%}}}

\section{Related and Future Work} \label{sec:rw} %{{{

\begin{table} % tbl:features {{{
  \caption{Various verification methods and frameworks,
    and whether they support
    horizontal compositionality (H),
    refinement and vertical compositionality (V),
    spatial compositionality (S),
    correctness-preserving compilation (C),
    extensible interaction model (E),
    multiple languages (M),
    \ldots
    [Note that we don't claim to supersede the functionality of all of these
    but only examine a subset of features. Also limit the scope of wour comparison:
    things that are mechanized in Coq? That can be viewed as semantics framework?]
  }
  \small
  \begin{tabular}{l c@{\:\:}c@{\:\:}c c@{\:\:}c@{\:\:}c l}
    \toprule
    & H & V & S & C & E & M & Notes \\
    \midrule
    Hoare logic \cite{hoarelog} & $\checkmark$ \\
    Refinement calculus \cite{refcal} & $\checkmark$ & $\checkmark$ \\
    Separation logic \cite{seplog} & $\checkmark$ & & $\checkmark$ \\
    \quad VST \cite{vst} & $\checkmark$ & & $\checkmark$ & $\checkmark$ \\
    \quad CCR \cite{ccr} & $\checkmark$ & $\checkmark$ & $\checkmark$ \\
    \quad Iris (?) \\
    \midrule
    CompCert \cite{compcert} & & $\checkmark$ & & $\checkmark$ \\
    \quad Compositional CompCert \cite{compcompcert} & $\checkmark$ & $\checkmark$ & & $\checkmark$
      &&& Too complex \\
    \quad SepCompCert \cite{sepcompcert} & 
      $\sim$ & $\checkmark$ & & $\checkmark$ &&& \\
    \quad CompCertM \cite{compcertm} &
      $+$ & $\checkmark$ & & $\checkmark$ \\
    \quad CompCertO \cite{compcerto} &
      $\checkmark$ & $\checkmark$ & & $\checkmark$ \\
    \midrule
    CAL \cite{popl15} & $\checkmark$ & $\checkmark$ & & $\checkmark$ & & $\checkmark$ \\
    Interaction Trees \cite{itrees} & \\
    Melocoton \cite{melocoton} & $\checkmark$ & & $\checkmark$ & & & $\checkmark$ \\
    Dim-Sum \cite{dimsum} \\
    \midrule
    Present work & $\checkmark$ & $\checkmark$ & $\checkmark$ & $\checkmark$ &
      $\checkmark$ & $\checkmark$ \\
    \bottomrule
  \end{tabular}
  \label{tbl:features}
\end{table}
%}}}

Finally, we briefly discuss past and future research
relevant to the work and goals we have described.

\paragraph{CompCert-based Verification Frameworks} %{{{

We have already discussed CompCert and CompCertO
extensively in \S\ref{sec:compcert}.
CompCertM \cite{compcertm} is another project
which builds on CompCert
to provide a compositional verification framework.
Like CompCertO,
it introduces a better model of the interaction between
C and assembly programs
and more flexibility in simulation conventions.
However, while it permits some form of localized state,
CompCertM does not support
full-blown data abstraction and state encapsulation
of the kind we have presented.
See \citet{compcerto}
for a detailed comparison between Compositional CompCert,
CompCertM and CompCertO.

We have also touched on
certified abstraction layers and CompCertX in \S\ref{sec:application:cal}.
Subsequent work has extended CAL to support concurrency \cite{ccal}.
There are more recent treatments of CAL which,
like our work,
attempt to streamline the underlying theory
\citep{popl22,rbgs-cal},
%In particular a limited version of the construction ${-} \mathbin@ U$
%operating on a fixed set $U$ appears in \citet{rbgs-cal}.
but this work has not been mechanized
or interfaced with CompCert.

%Interaction trees \cite{itree,itrees} provide
%another framework for compositional semantics
%formalized in the Coq proof assistant
%which presents similarities with our own.
%though their interface with CompCert is also less comprehensive.

%}}}

\paragraph{Separation Logic} %{{{

For the most part,
the frameworks discussed above
do not provide program-level verification facilities,
but rather focus on a more coarse-grained, module-level ``glue''.
Likewise,
we have assumed that elementary module correctness properties
such as $\phi_1$, $\phi_2$ and $\phi_\kw{bq}^\kw{min}$
were provided by the user%
\footnote{Our example is simple enough that,
  in our implementation,
  manual simulation proofs were
  sufficient.}
and focused on the problem of
connecting such proofs.
Nevertheless,
program logics in general and separation logic in particular
are relevant to our work in the following ways.

First, it would be beneficial to incorporate
such program logics into our framework.
For example, \citet{popl15} provides
a rudimentary Clight program logic which
can be used to help prove abstraction layers correct. 
It may be useful to investigate whether
the Clight separation logic provided by
the Verified Software Toolchain \cite{vst}
could be interfaced with our model.

Secondly,
spatial composition is in fact
the defining feature of separation logic.
Our treatment of memory separation
draws extensively from
separation algebra \cite{sepalg},
an approach to building models of separation logic.
More recently,
Conditional Contextual Refinement (CCR) \cite{ccr}
combined (vertical) refinement and (spatial) separation logic into
a unified, mechanized framework.
CCR however does not support state encapsulation
or certified compilation.

%}}}

\paragraph{Multi-language Semantics} %{{{

One aspect of our framework which is not developed much
is the possibility of reasoning across languages.
In Compositional CompCert and CompCertM,
assembly programs are given C-level semantics,
making it possible to directly reason about composite programs
(but only for Asm code which behaves according the C calling convention).
CAL uses the opposite approach and can translate
C-level layer specification into assembly behaviors.
Recent work on the DimSum framework \cite{dimsum}
attempts to give a more general account of
multi-language semantics
by introducing wrappers to translate between
different languages.

These various approaches all attempt
to represent \emph{horizontally} what
the simulation conventions of CompCertO represent vertically.
In our framework,
the notions of companion and conjoint
could provide a natural way to formalize
approaches of this kind,
so that for example the CompCertO calling convention
$\mathbb{C} : \mathcal{C} \leftrightarrow \mathcal{A}$
would be in companion/conjoint relationships with
adapter components
$\mathcal{CA} : \mathcal{A} \twoheadrightarrow \mathcal{C}$
and
$\mathcal{AC} : \mathcal{C} \twoheadrightarrow \mathcal{A}$.
The complexity of CompCertO's convention as presently stated
makes this challenging,
but we do not believe it to be a fundamental issue.

%}}}

%\paragraph{Categorical Structures} %{{{
%
%To our knowledge,
%the work described in this paper
%constitutes the first explicit use of
%double categories in the context of compositional verification,
%although the approach was initially suggested in \citet{compcerto}.
%String diagrams for double categories
%were developed and shown to be sound in \citet{dcsd}.
%Monoidal categories and their string diagrams
%are more common---%
%\citet{rosetta} provides a good introduction.
%%a good introduction is provided by \citet{rosetta}.
%
%The construction $\mathbf{C}^\dagger$ presented in \S\ref{sec:encap}
%bears some resemblance to---and was inspired by---%
%the \emph{free category with feedback} construction \cite{feedback,caots}.
%Indeed,
%traced monoidal categories
%could provide a basis for encapsulation
%in a version of our framework supporting reentrancy and mutual recursion,
%as in interaction trees.
%The present version
%based on an elementary encapsulation primitive $[ u \rangle$
%is simpler but less powerful.
%
%%}}}

%}}}

\section{Conclusion} \label{sec:conclusion} %{{{

Combining compositional semantics,
abstraction,
encapsulation and certified compilation
is an important step towards
the construction of large-scale systems certified end-to-end.
Moreover,
we believe that
the underlying algebraic structures that we have uncovered
in this process
constitute an elegant conceptual framework
with applications beyond the present work,
and may become an important facet of
future certified systems engineering work.

%}}}

\bibliography{../references}

\section{Things to redistribute or drop} %{{{

\subsection{Overview and Terminology} \label{sec:terminology} %{{{

There are several well-known techniques which can be used
formally or informally
to help us understand and reason about
the code in Example~\ref{ex:bq}.
Our model seeks to give a semantics-level account
of the following features.

\paragraph{Horizontal Composition} %{{{

A proof technique or semantic model is usually deemed compositional
when the properties or behaviors it describes
can be broken down along the syntactic structure of the program.
For example,
a core rule of Hoare logic
allows us to verify the correctness of the program $C_1 \mathbin; C_2$
by first verifying the correctness properties $\{P\} C_1 \{Q\}$ and $\{Q\} C_2 \{R\}$
for the component programs $C_1$ and $C_2$,
then combining them to derive the property
$\{P\} C_1 \mathbin; C_2 \{R\}$
for the composite whole.
We refer to this kind of phenomenon as
\emph{horizontal} compositionality.
Note that for this to work,
the properties must agree on a common \emph{interface},
in this case a state assertion $Q$
which must hold after $C_1$ finished executing
but before $C_2$ starts.

%}}}

\paragraph{Memory Separation and Spatial Composition} %{{{

While data abstraction and vertical compositionality enable
the transformation of state representation
across levels of abstraction,
a complementary feature is the ability to divide
state into components
which different parts of the system
act on independently.
In Example~\ref{ex:bq},
the functions $\kw{inc1}$, $\kw{inc2}$ and $\kw{set}$
act respectively on the variables $\kw{c1}$, $\kw{c2}$ and $\kw{buf}$
but do not affect any other components of the state.
Therefore,
it should be possible to reason about each one
in terms of their associated state component only.
We will call this \emph{spatial} compositionality.

Spatial compositionality is
a central feature of separation logic \cite{seplog},
where the separating conjunction
can be used to combine specifications
operating on arbitrary but disjoint parts of the memory.
Our approach is more strongly typed,
in the sense that the different parts of the state
accessed by a component
are visible explicitly in its interface.
Each can be refined independently,
and components which act on different fields
can nevertheless be combined.

%}}}

%}}}

%\begin{example}
%Moreover,
%while C does not facilitate abstraction and encapsulation,
%they also play a role.
%Since the variables $\kw{c1}$, $\kw{c2}$ and $\kw{buf}$
%are declared $\kw{static}$ and
%their addresses are not leaked, % by the code in $\kw{rb.c}$,
%we can reason that under normal circumstances
%the environment will not access them directly.
%In fact, a user of the code in Fig.~\ref{fig:code}
%could ignore them and picture the queue
%as an abstract sequence of values.
%%$\kw{enq}$'ed values waiting to be $\kw{deq}$'ed.
%
%An effective verification framework
%must provide a way to make these intuitions formal.
%Figure~\ref{fig:spec} demonstrates one approach.
%The top-level specification $\Gamma_\kw{bq}$
%describes the overall behavior of $\kw{enq}$ and $\kw{deq}$
%using an abstract sequence $\vec{q}$ as the only state.
%The verification task can then be decomposed using
%the intermediate specifications for each file.
%The specification
%$\Gamma_\kw{rb}$
%describes the lower-level ring buffer data structure.
%The specification
%$\Sigma_\kw{bq}$ describes
%the sequence of calls into $\kw{rb.c}$
%which are necessary to realize the queue operations
%in terms of the ring buffer primitives,
%and does not carry any state at all.
%The relation $R$
%explains how queue states
%are realized in terms of the ring buffer.
%\end{example}

%}}}

\section*{Was: Background and Approach} %{{{

We start in \S\ref{sec:compcert} and \S\ref{sec:compcerto}
with a summary of the CompCert work we build on.
Our own framework is described
in \S\ref{sec:overview:lcomp} and \S\ref{sec:overview:scomp},
serving as a base for the techniques explained in the rest of the section.

\subsection{Whole-Program Semantics in CompCert} \label{sec:compcert} %{{{

As a \emph{certified} compiler,
CompCert comes with a specification
and a correctness proof,
mechanized in a proof assistant.
To state a specification for the compiler,
the mathematical development which accompanies
CompCert
includes a formalization of the source (Clight) and target (Asm) languages.

\paragraph{Transition Systems} %{{{

CompCert uses transition systems
to define language semantics.
For example,
the semantics $\kw{Clight}[p]$
of a source program $p$
are described by a set $S$ of states along with:%
\begin{itemize}
  \item a distinguished subset of \emph{initial} states $I \subseteq S$;
  \item a transition relation ${\rightarrow} \subseteq S \times S$;
  \item a relation $F \subseteq S \times \kw{int}$ which identifies
    \emph{final} states and the associated process exit codes.
\end{itemize}
We use infix notation for the relations $\rightarrow$ and $F$.
In addition, we will often write $x \mathrel{R} y \mathrel{S} z$
to mean $x \mathrel{R} y \mathrel\wedge y \mathrel{S} z$.
An execution of $p$ starts with an initial state $s_0 \in I$,
performs a number of transitions
$
  I \ni s_0 \rightarrow s_1 \rightarrow \cdots \rightarrow s_n \mathrel{F} x
$,
and terminates with status $x$ when the final state $s_n$ is reached.%
\footnote{%
  This glosses over important technical details:
  CompCert uses transition labels to model interaction with the operating system,
  permits some forms of demonic nondeterminism,
  and takes special care to handle infinite executions.
  However, the changes required in compositional extensions
  are largely orthogonal to these details,
  so we will not discuss them in depth.}
%States with no $\rightarrow$ or $F$ successors
%are said to \emph{go wrong} and denote undefined behaviors.

%In the Clight semantics,
%states contains the current control stack,
%environments storing values of temporary variables,
%as well as a global memory state.
%States used by the Asm semantics
%consist only of the registers and global memory.
%CompCert also formalizes %syntax and semantics for
%various intermediate languages, which are
%not part of the specification
%but are used in the construction of the proof.

%}}}

\paragraph{Simulations} %{{{

Once we have described
the behavior of source and target programs,
we must state their relationship.
The correctness theorem of CompCert is established as a \emph{simulation} ($\le$):
\begin{equation}
    \kw{CompCert}(p) = p'
    \quad\Longrightarrow\quad
    \kw{Clight}[p] \le \kw{Asm}[p']
    \,.
    \label{eqn:ccc-wp}
\end{equation}

Given the transition systems $L_1$ and $L_2$,
a simulation between them is a relation $\rho \subseteq S_1 \times S_2$
between the states of the source $L_1$
and the states of the target $L_2$.
This relation must satisfy several conditions
which ensure that
every execution of $L_1$ gives rise
to a corresponding execution of $L_2$:
\begin{itemize}
  \item there is for every source initial state $s_1 \in I_1$
    a related target initial state
    ($\exists s_2 \mathbin. s_1 \mathrel\rho s_2 \in I_2$);
  \item for related states $s_1 \mathrel\rho s_2$,
    a source transition $s_1 \rightarrow_1 s_1'$ must be matched by
    a target transition sequence $s_2 \rightarrow_2^* s_2'$ such that
    the resulting states $s_1' \mathrel\rho s_2'$ are again related;
  \item for related states $s_1 \mathrel\rho s_2$,
    if $s_1$ is final in $L_1$ with an outcome $x$,
    then $s_2 \mathrel{F_2} x$ as well.
\end{itemize}
We write $\rho : L_1 \le L_2$ when these conditions are satisfied,
or just $L_1 \le L_2$ when such a $\rho$ exists.

%}}}

\paragraph{Vertical Compositionality} %{{{

The key to proving (\ref{eqn:ccc-wp}) is the compositionality of simulations.
CompCert uses a dozen compilation phases,
which progressively transform %$p$ into $p'$:
$
  p = p_0 \longmapsto p_1 \longmapsto \cdots \longmapsto p_n = p'
$.
To derive the correctness theorem,
a simulation proof is established for each phase:
\begin{equation}
  \kw{Clight}[p] \:=\:
  \kw{Clight}[p_0] \:\le\: \kw{RTL}[p_1] \:\le\: \cdots \:\le\: \kw{Asm}[p_n]
  \:=\: \kw{Asm}[p']
  \label{eqn:corrsteps}
\end{equation}
When the target $L_2$ of a simulation $\pi : L_1 \le L_2$
is the source of a simulation $\rho : L_2 \le L_3$,
the two can be combined, and the composite
$\pi \vcomp \rho$
is in turn a simulation of type $L_1 \le L_3$.
This allows
the successive simulation proofs in (\ref{eqn:corrsteps})
to be combined into the correctness property (\ref{eqn:ccc-wp}).

%}}}

\paragraph{Horizontal Compositionality} %{{{

A serious limitation of CompCert
is that its semantics
only describe the behavior of complete programs.
For example, in Fig.~\ref{fig:code}
neither $\kw{rb.c}$ nor $\kw{bq.c}$ provide a $\kw{main}()$ function,
and as a result their Clight semantics are undefined.
CompCert can compile $\kw{rb.c}$ and $\kw{bq.c}$,
but in this situation
the correctness property (\ref{eqn:ccc-wp})
does not provide any guarantees.

To account for this situation
at the semantic level,
we need to assign a behavior $\kw{Clight}(\kw{rb.c})$
to individual translation units such as $\kw{rb.c}$.
We must then define an operator $\oplus$ to model the \emph{linking} process
which happens before $\kw{rb.c}$
is run as part of a larger program.
This operator should be compatible with simulations,
so that for example we may derive the overall correctness property
\[
  \kw{Clight}(\kw{bq.c}) \oplus \kw{Clight}(\kw{rb.c})
  \:\le\:
  \kw{Asm}(\kw{bq.s}) \oplus \kw{Asm}(\kw{rb.s})
\]
[XXX need to introduce $+$ which is used in $\odot$ vs $\oplus$ theorem later]
from the compiler correctness properties
associated with the individual translation units.

%}}}

%}}}

\subsection{Compositional Semantics in CompCertO} \label{sec:compcerto} %{{{

Achieving horizontal compositionality in CompCert
is a surprisingly complex task and
has been an active area of research for the past decade.
Below we explain the solution retained in CompCertO
\cite{compcerto},
which we use as a starting point.

\paragraph{Open Semantics} %{{{

To model translation units and linking,
we must describe interactions across component boundaries%
---namely, function calls and returns.
To this end,
CompCertO
uses a notion of \emph{open} transition system.
As an example,
consider the case of a C translation unit such as \kw{rb.c}:
\begin{itemize}
\item
To initialize a transition system ($I$),
we must give the name $f \in \kw{ident}$
of a function being invoked,
actual parameters $\vec{v} \in \kw{val}^*$,
and the current state $m \in \kw{mem}$ of the global memory.
\item
When the component terminates ($F$),
instead of a single integer outcome $x \in \kw{int}$ it must provide
a return value $v' \in \kw{val}$ and an updated memory state $m' \in \kw{mem}$.
\end{itemize}
%In other words,
%executions now take the form:
%\[
%  f(\vec{v})@m
%  \:\mathrel{I}\:
%  s_0 \:\rightarrow\: s_1 \:\rightarrow\: \cdots \:\rightarrow\: s_n
%  \:\mathrel{F}\:
%  v'@m'
%\]
This models function calls \emph{into} the component.
In addition,
the component itself may
perform outgoing calls
by associating to certain \emph{external} states ($X$)
a description of the call,
and the possible resumption states ($Y$)
which may result from the call's outcome.
Transition system now contain:
\begin{equation}
 \begin{array}{c}
  I \subseteq (\kw{ident} \times \kw{val}^* \times \kw{mem}) \times S
  \qquad
  {\rightarrow} \subseteq S \times S
  \qquad
  F \subseteq S \times (\kw{val} \times \kw{mem})
  \\
  X \subseteq S \times (\kw{ident} \times \kw{val}^* \times \kw{mem})
  \qquad
  Y \subseteq S \times (\kw{val} \times \kw{mem}) \times S
 \end{array}
 \label{eqn:compcomp-lts}
\end{equation}
%We will again use infix notation for $X$,
We use the notation $q = f(\vec{v})@m \in \kw{ident} \times \kw{val}^* \times \kw{mem}$
for function call specifications,
and write $r \mathrel{Y^s} s'$ when $(s, \, r, \, s') \in Y$.
This means that
after $s$ triggers an external call ($s \mathrel{X} q$)
which returns an answer $r = v'@m' \in \kw{val} \times \kw{mem}$,
the execution resumes with state~$s'$.
%In other words,
%executions will take the form
Executions take the form
\[
  q \mathrel{I} s_0 \rightarrow^*
  s_1 \mathrel{X} q_1 \leadsto
  r_1 \mathrel{Y^{s_1}} s_1' \rightarrow^*
  s_2 \mathrel{\cdots}
  s_n \mathrel{X} q_n \leadsto
  r_n \mathrel{Y^{s_n}} s_n' \rightarrow^*
  s_f \mathrel{F} r
  \,,
\]
corresponding to an interaction trace
$
  q \rightarrowtail
  (q_1 \leadsto r_1) \rightarrowtail
  \cdots \rightarrowtail
  (q_n \leadsto r_n) \rightarrowtail
  r
$.
Here we use $\rightarrowtail$ to denote internal execution
and $\leadsto$ to denote a step where the environment is in control.
We will write $L \vDash t$
to mean that the transition system $L$
admits the interaction trace $t$.

%The component is activated by an incoming call,
%described by a question $q \in B^\que$.
%%which is used to determine the transition system's initial state.
%As it executes,
%the transition system may perform outgoing calls,
%asking questions
%$q_1, \ldots, q_n \in A^\que$
%and receiving corresponding answers
%$r_1, \ldots, r_n \in A^\ans$.
%Execution terminates with
%the top-level answer $r \in B^\ans$.

\begin{example}[Clight semantics] \label{ex:overview:clightsem} %{{{
Consider the translation unit $\kw{rb.c}$ shown in Fig.~\ref{fig:code}.
Its semantics is given by 
the transition system $\kw{Clight}(\kw{rb.c})$,\!%
\footnote{%
  We use round parentheses for
  the \emph{open} transition system $\kw{Clight}(-)$
  as opposed to the original closed semantics $\kw{Clight}[-]$.
  }
which admits the following interaction trace:
\[
  \kw{Clight}(\kw{rb.c}) \quad \vDash \quad
  \kw{inc1}()@[\kw{c1} \mapsto 2]
  \: \rightarrowtail \:
  2@[\kw{c1} \mapsto 3]
\]
Note that the memory is updated to store the new value of the counter $\kw{c1}$.
By contrast, $\kw{bq.c}$
does not directly modify the memory,
but it makes outgoing calls which may have that effect:
\[
  \kw{Clight}(\kw{bq.c}) \:\: \vDash \:\:
  \kw{deq}()@m
  \rightarrowtail
  \big( \kw{inc1}()@m \leadsto i@m' \big)
  \rightarrowtail
  \big( \kw{get}(i)@m' \leadsto v@m'' \big)
  \rightarrowtail
  v@m''
  \,.
\]
\end{example}
%}}}

The model we have described so far
makes sense for C programs,
but cross-component interactions take different forms
in the case of other languages.
For example,
assembly-level interactions are formulated in terms of
low-level register state and code addresses.
To deal with this diversity,
CompCertO introduces a rudimentary form of typing for transition systems,
where the form taken by \emph{outgoing} and \emph{incoming} interactions
are specified by a notion of language interface.

\begin{definition} \label{def:li} \label{def:lts} %{{{
A \emph{language interface} $A = \langle A^\que, A^\ans \rangle$
is a set of questions $A^\que$ and a set of answers $A^\ans$.
Then a \emph{transition system} $L : A \twoheadrightarrow B$
is a tuple $L = \langle S, {\rightarrow}, I, X, Y, F \rangle$
consisting of:
\begin{itemize}
  \item a set $S$ of states and
    a transition relation ${\rightarrow} \subseteq S \times S$;
  \item a relation $I \subseteq B^\que \times S$
    which assigns possible \emph{initial states}
    to each question of $B$;
  \item a relation $F \subseteq S \times B^\ans$
    which specifies \emph{final states} together with
    corresponding answers in $B$;
  \item a relation $X \subseteq S \times A^\que$
    which identifies \emph{external states} and
    corresponding questions of $A$;
  \item a relation $Y \subseteq S \times A^\ans \times S$,
    which identifies \emph{resumption states}.
\end{itemize}
\end{definition}
%}}}

Under this definition,
the source and target semantics of CompCertO can be described as
\[
  \kw{Clight}(p) : \mathcal{C}_\kw{m} \twoheadrightarrow \mathcal{C}_\kw{m}
  \qquad \text{and} \qquad
  \kw{Asm}(p') : \mathcal{A}_\kw{m} \twoheadrightarrow \mathcal{A}_\kw{m} \,.
\]
We will focus on
$\mathcal{C}_\kw{m} = \langle \mathcal{C}_\kw{m}^\que, \mathcal{C}_\kw{m}^\ans \rangle$
describes the kind of interactions used Example~\ref{ex:overview:clightsem}:
\[
  \mathcal{C}_\kw{m}^\que :=
    \{ f(\vec{v})@m \mid f \in \kw{ident}, \vec{v} \in \kw{val}^*, m \in \kw{mem} \}
  \,,
  \quad
  \mathcal{C}_\kw{m}^\ans :=
    \{ v@m \mid v \in \kw{val}, m \in \kw{mem} \}
  \,.
\]

%}}}

\paragraph{Simulations} %{{{

The types of
$\kw{Clight}(p) : \mathcal{C}_\kw{m} \twoheadrightarrow \mathcal{C}_\kw{m}$ and
$\kw{Asm}(p') : \mathcal{A}_\kw{m} \twoheadrightarrow \mathcal{A}_\kw{m}$
raise the question of the relationship
between source-level interactions
in $\mathcal{C}_\kw{m}$
and corresponding target-level interactions
in $\mathcal{A}_\kw{m}$.
Compositional compiler correctness only makes sense
with respect to a particular calling convention.
%Rather than modeling the calling convention implicitly
%as part of the assembly semantics,
%
CompCertO makes this explicit:
simulations operate in the context of specified
\emph{simulation conventions},
%which express the relationships between
%the source and target components'
%interactions with the environment.
which introduce a form of two-dimensional typing for simulations.

To establish a simulation
of a transition system $L_1: A_1 \twoheadrightarrow B_1$
by a transition system $L_2: A_2 \twoheadrightarrow B_2$,
we must first specify a simulation convention
$\mathbf{R}_B : B_1 \leftrightarrow B_2$
for their incoming calls,
and a simulation convention
$\mathbf{R}_A : A_1 \leftrightarrow A_2$
for their outgoing calls.
The simulation can then be stated as
\[
  \phi : L_1 \le_{\mathbf{R}_A \twoheadrightarrow \mathbf{R}_B} L_2
  \,.
\]
There is an identity simulation convention $\idsc_A : A \leftrightarrow A$
for every language interface $A$;
given $L_1, L_2 : A \twoheadrightarrow B$,
we will often write
a simulation of type $L_1 \le_{\idsc_A \twoheadrightarrow \idsc_B} L_2$
simply as $L_1 \le L_2$.
%
Compiler correctness
is expressed in terms of a convention
$\mathbb{C} : \mathcal{C}_\kw{m} \leftrightarrow \mathcal{A}_\kw{m}$
and can be stated as follows:
\[
  \kw{CompCert}(p) = p'
  \quad \Rightarrow \quad
  \pi_p \::\:
  \kw{Clight}(p)
  \:\le_{\mathbb{C} \twoheadrightarrow \mathbb{C}}\:
  \kw{Asm}(p')
  \:.
\]
[XXX]
Finally, the case where $L_1$ and $L_2$ are both identities
is a \emph{simulation convention refinement}, written:
\[
  \mathbf{R} \sqsubseteq \mathbf{S} : A \leftrightarrow B
  \quad :\Leftrightarrow \quad
  \kw{id}_A \le_{\mathbf{R} \twoheadrightarrow \mathbf{S}} \kw{id}_B
\]

%We will write
%for a simulation of this kind,
%which can depicted in pasting and string diagrams as:
%\[
%  \begin{tikzcd}[sep=tiny]
%    A^\sharp
%      \ar[rr, twoheadrightarrow, "L^\sharp"]
%      \ar[dd, leftrightarrow, "\mathbf{R}_A"'] &&
%    B^\sharp
%      \ar[dd, leftrightarrow, "\mathbf{R}_B"] \\
%    & \pi & \\
%    A^\flat
%      \ar[rr, twoheadrightarrow, "L^\flat"'] &&
%    B^\flat
%  \end{tikzcd}
%  \qquad \qquad
%  \begin{tikzpicture}[xscale=0.4,yscale=0.35,baseline=(R)]
%    % Background
%    \begin{scope}
%      \fill[ACMYellow!50] (0,2) rectangle (2,4);
%      \fill[ACMRed!50] (0,0) rectangle (2,2);
%      \fill[ACMGreen!50] (2,2) rectangle (4,4);
%      \fill[ACMBlue!50] (2,0) rectangle (4,2);
%    \end{scope}
%    % Region labels
%    \begin{scope}[every node/.style={opacity=0.5}]
%      \scriptsize
%      \node[below right] at (0,4) {$B^\sharp$};
%      \node[above right] at (0,0) {$B^\flat$};
%      \node[below left] at (4,4) {$A^\sharp$};
%      \node[above left] at (4,0) {$A^\flat$};
%    \end{scope}
%    % Strings
%    \begin{scope}
%      \footnotesize
%      \draw (2,4) node[above] {$L^\sharp$}
%         -- (2,0) node[below] {$L^\flat$};
%      \draw (0,2) node[left] (R) {$\mathbf{R}_B$}
%         -- (4,2) node[right] {$\mathbf{R}_A$};
%    \end{scope}
%    % Node
%    \node[draw,fill=white,rounded corners] at (2,2) {$\pi$};
%  \end{tikzpicture}
%\]
%Simulation conventions are relational in nature.

%}}}

\paragraph{Compositional Structure} %{{{

Figure~\ref{fig:compcerto}
summarizes the compositional structure of the framework.
The simulation conventions $\mathbf{R} : A \leftrightarrow B$ and
$\mathbf{R}' : B \leftrightarrow C$
compose into
$\mathbf{R} \vcomp \mathbf{R}' : A \leftrightarrow C$.
This is used by the vertical composition principle \kw{sim}-$\vcomp$ for simulations,
which allows them to be pasted vertically.
%\[
%  \begin{prooftree}
%    \hypo{\mathbf{R} : A \leftrightarrow B}
%    \hypo{\mathbf{R}' : B \leftrightarrow C}
%    \infer2{(\mathbf{R} \vcomp \mathbf{R}') : A \leftrightarrow C}
%  \end{prooftree}
%  \qquad
%  \begin{prooftree}
%    \hypo{\phi : L_1 \le_{\mathbf{R} \twoheadrightarrow \mathbf{S}} L_2}
%    \hypo{\psi : L_2 \le_{\mathbf{R'} \twoheadrightarrow \mathbf{S'}} L_3}
%    \infer2{(\phi \vcomp \psi) : L_1 \le_{\mathbf{R} \vcomp \mathbf{R'} \twoheadrightarrow
%      \mathbf{S} \vcomp \mathbf{S'}} L_3}
%  \end{prooftree}
%\]
Moreover,
the \emph{semantic linking} operator $\oplus$
models the interaction between different program components,
%The transition system $L_1 \oplus L_2$
%generally mirrors the execution of $L_1$ or $L_2$,
%but when $L_1$ makes an external call
%to a function provided by $L_2$ (and vice versa),
%$L_1 \oplus L_2$ instantiates a new copy of $L_2$ to handle the call internally.
%This copy executes until it reaches a final state,
%at which point its outcome is used to resume
%the suspended execution of $L_1$.
so that for example
$\kw{Clight}(\kw{rb.c}) \oplus \kw{Clight}(\kw{bq.c})$
admits the trace
\[
  \kw{deq}()@[\kw{c1} \mapsto 2, \kw{buf} \mapsto \{v_0, v_1, v_2, v_3\}]
  \quad\rightarrowtail\quad
  v_2@[\kw{c1} \mapsto 3, \kw{buf} \mapsto \{v_0, v_1, v_2, v_3\}]
  \,.
\]

Unfortunately,
despite what the shape of the diagram in Fig.~\ref{fig:compcerto} may suggest,
simulations cannot in general be pasted horizontally
along boundaries of the kind $\mathbf{R} : A \leftrightarrow B$.
This is due to the symmetric nature of semantic linking,
which lets $L_1$ and $L_2$ interact in a mutually recursive way.
For $\oplus$ composition to be possible,
the transition systems must operate
over a single language interface
($\kw{ts}$-$\oplus$),
and likewise simulations must operate
with respect to a single simulation convention
($\kw{sim}$-$\oplus$).

%To work around this restriction,
%CompCertO introduces a rich algebra of \emph{simulation convention refinements},
%which play the role of a second kind of two-dimensional object.
%These refinements can compose
%with simulations to modify their types,
%and are used to massage per-phase
%simulation proofs with varied conventions into
%an overall compiler correctness theorem
%which fits \kw{sim}-$\oplus$.

%}}}

\begin{example}[Refinement-based verification with CompCertO] \label{ex:compcerto} %{{{
Following the blueprint in Fig.~\ref{fig:spec},
we can attempt to use the framework outlined above
for refinement-based verification.
This would involve defining the specifications
$L_\kw{bq}, M_\kw{bq}, L_\kw{rb} :
 \mathcal{C}_\kw{m} \twoheadrightarrow \mathcal{C}_\kw{m}$
and proving:
\[
  \rho_1 :
  L_\kw{bq} \le M_\kw{bq} \oplus L_\kw{rb}
  \,, \qquad
  \rho_2 :
  M_\kw{bq} \le \kw{Clight}(\kw{bq.c})
  \,, \qquad
  \rho_3 :
  L_\kw{rb} \le \kw{Clight}(\kw{rb.c})
  \,.
\]
These simulations could then be combined with
CompCertO's correctness theorem and the linking property
$
  \ell :
    \kw{Asm}(\kw{bq.s}) \oplus \kw{Asm}(\kw{rb.s})
    \le
    \kw{Asm}(\kw{bq.s} + \kw{rb.s})
$
to establish end-to-end correctness as:
\[
  \rho_1 \vcomp (\rho_2 \oplus \rho_3) \vcomp (\pi_\kw{bq} \oplus \pi_\kw{rb}) \vcomp \ell
  \::\:
  L_\kw{bq}
    \le_{\mathbb{C} \twoheadrightarrow \mathbb{C}}
    \kw{Asm}(\kw{bq.s} + \kw{rb.s})
  \,.
\]
\end{example}
%}}}

\paragraph{Evaluation} %{{{

Example~\ref{ex:compcerto}
illustrates the flexibility of the CompCertO semantic model,
but also some of its limitations.
The language interface $\mathcal{C}_\kw{m}$
forces the specifications
to be formulated in terms of low-level memory states,
and they remained tied to the particular concrete representation
used by the code in Fig.~\ref{fig:code}.
Moreover,
the rigidity inherent $\oplus$ composition
makes it difficult in general
to handle situations which involve heterogeneous language interfaces.

%}}}

%}}}

\subsection{Layered Composition} \label{sec:overview:lcomp} %{{{

The restriction of semantic linking
$
  {\oplus}_A : (A \twoheadrightarrow A) \times (A \twoheadrightarrow A)
  \rightarrow (A \twoheadrightarrow A)
$
to \emph{homogeneous} components
is at odds with
CompCertO's multiplicity of language interfaces.
In \S\ref{sec:base},
we describe a \emph{layered composition} operator $\odot$
which is more fundamental and more flexible:
\[
  {\odot}_{A,B,C} :
    (B \twoheadrightarrow C) \times
    (A \twoheadrightarrow B) \rightarrow
    (A \twoheadrightarrow C)
  \,.
\]
The transition system $L_1 \odot L_2$,
is depicted in \autoref{fig:overview:ts:comp}.
Incoming calls in $C$ activate $L_1$.
The outgoing calls of $L_1$ in $B$ are then handled by $L_2$, and
the outgoing calls of $L_2$ in $A$
are directed back to the environment.
The identity $\kw{id}_A : A \twoheadrightarrow A$
simply passes calls through.

%This mode of composition is hinted at in \citet{compcerto}.
%We provide a formal definition in \S\ref{sec:base:ts}
%and show that
%it defines a category $\mathbf{TS}$ of
%language interfaces and transition systems.
%In particular,
%the unit for $\odot$ is
%the transition system $\kw{id}_A : A \twoheadrightarrow A$
%depicted in \autoref{fig:overview:ts:id},
%which echoes the incoming question as an outgoing one
%and propagates the answer back to the caller.

Because $\odot$ connects transition systems
along one side only
(matching the outgoing calls of the first one with
the incoming calls of the second),
the resulting compositional structure
is more flexible and uniform
than the one induced by~$\oplus$;
the corresponding rules are shown
in Fig.~\ref{fig:xcomp}.
%At the same time,
%as an  \emph{under-approximation} of semantic linking,
%the behavior described by layered composition
%can still be implemented by linking assembly programs.
%
At the same time, layered composition
\emph{under-approximates} $\oplus$,
enabling the following property.
%Syntactic linking of assembly programs ($+$) is known to implement $\oplus$,
%and therefore layered composition as well.

\begin{theorem}[Linking implements layered composition] \label{thm:linking}
For two assembly programs $M_1, M_2$,
%For $L_1, L_2 : A \twoheadrightarrow A$,
\[
  \ell\::\:
  \kw{Asm}(M_1) \odot \kw{Asm}(M_2)
  \:\le\:
  \kw{Asm}(M_1) \oplus \kw{Asm}(M_2)
  \:\le\:
  \kw{Asm}(M_1 + M_2)
 \,.
\]
\end{theorem}

%This means that when a system is compositionally
%specified and verified at the Clight level,
%and an overall correctness property is derived
%in terms of $\odot$,
%we can combine it with the compiler's correctness theorem
%to obtain guarantees about the linked assembly program.


%}}}

\begin{example} \label{ex:compcerto-sd} % composing bq.c, rb.c {{{
Revisiting Example~\ref{ex:compcerto},
note that the transition systems $L_\kw{bq}$ and $L_\kw{rb}$
used as specifications
accept incoming calls but never perform external calls.
This can be reflected in their type as
$L_\kw{bq}, L_\kw{rb} : \top \twoheadrightarrow \mathcal{C}_\kw{m}$,
with $\top = \langle \varnothing, \varnothing \rangle$
as a trivial language interface.
The simulations become:
\[
  \rho_1 :
  L_\kw{bq} \le M_\kw{bq} \odot L_\kw{rb}
  \,, \qquad
  \rho_2 :
  M_\kw{bq} \le \kw{Clight}(\kw{bq.c})
  \,, \qquad
  \rho_3 :
  L_\kw{rb} \le_{\varnothing_{\top,\mathcal{C}_\kw{m}}
    \twoheadrightarrow \idsc_{\mathcal{C}_\kw{m}} } \kw{Clight}(\kw{rb.c})
  \,,
\]
where the simulation convention
$\varnothing_{A,B} : A \leftrightarrow B$
disallows any interaction.
We can again derive:
\[
  \rho_1 \vcomp (\rho_2 \oplus \rho_3) \vcomp (\pi_\kw{bq} \oplus \pi_\kw{rb}) \vcomp \ell
  \::\:
  L_\kw{bq}
    \le_{\mathbb{C} \twoheadrightarrow \varnothing \vcomp \mathbb{C}}
    \kw{Asm}(\kw{bq.s} + \kw{rb.s})
  \,.
\]
[XXX should be $\odot$, also position of $\varnothing$ is suspect]
\end{example}
%}}}

%\begin{table}[b] % tbl:compcerto {{{
%  \caption{Summary of the CompCertO model,
%    with notations and applicable composition principles.}
%  \label{tbl:compcerto}
%  \begin{tabular}{
%    llc
%    c@{\:\:\:\:}c@{\:\:\:\:}c@{}c
%  }
%    \toprule
%    Role & Components & Notation & \multicolumn{4}{c}{Compose} \\
%    & & && H & V
%    \\
%    \midrule
%      Interface
%        & Language interfaces & $A, B, C$ && &
%    \\
%      Behavior
%        & Transition systems & $L : A \twoheadrightarrow B \in \mathbf{TS}$ &&
%            $\oplus$ &
%    \\
%      Abstraction
%        & Simulation conventions & $\mathbf{R} : A \leftrightarrow B \in \mathbf{SC}$ &&
%            & $\vcomp\,$
%    \\
%      Refinement
%        & Simulations &
%          $\pi :
%           L_1 \le_{\mathbf{R} \twoheadrightarrow \mathbf{S}} L_2
%             \in \mathbf{TSC}$ &&
%          $\oplus$ & $\vcomp\,$
%    \\
%    \bottomrule
%  \end{tabular}
%\end{table}
%%}}}

\begin{table} % tbl:roadmap -- Roadmap and notations {{{
  \caption{
    Summary of
    the various kinds of objects in our framework,
    together with the corresponding notations
    and applicable horizontal (H), vertical (V) and spatial~(S)
    composition operations.
} \label{tbl:roadmap}
  \small
  \begin{tabular}{
    l lc c@{\,\:}c@{\:}c @{\quad\:} lc c@{\,\:}c@{\:}c
  }
    \toprule
    Role & Component & Notation & H & V & S
         & Component\hspace{-1em} & Notation & H & V & S
    \\
    \midrule
      Interface
        & Language interface & $A, B, C$
	& & & $\mathbin@$
        & Set & $U, V$
	& & & $\times$
    \\
      Behavior
        & Transition system & $L : A \twoheadrightarrow B$
	& $\odot$ & & $\mathbin@$
        & Lens & $f : U \lensarrow V$
	& $\circ$ & & $\times$
    \\
      Abstraction
        & Simulation convention \hspace{-1em} & $\mathbf{R} : A \leftrightarrow B$
	& & $\vcomp$ & $\mathbin@$
        & Relation & $R \subseteq U \times V$
	& & $\mathbin;$ & $\times$
    \\
      Refinement
        & Simulation &
          $\pi :
           L_1 \le_{\mathbf{R} \twoheadrightarrow \mathbf{S}} L_2$
        & $\odot$ & $\vcomp$ & $\mathbin@$
        & Simulation &
         $\sigma : f \equiv_{R \lensarrow S} g$
	& $\circ$ & $\mathbin;$ & $\times$
    \\
    \bottomrule
  \end{tabular}
\end{table}
%}}}

%}}}

\ifdefined\withappendix

\appendix

\newpage

\section{Memory Separation in CompCert} \label{app:sep} %{{{

% The constructions we have introduced so far
% make it possible to manage and encapsulate persistent state
% in the context of CompCertO,
% with certified abstraction layers
% as one key application.
% The global memory state used in the semantics of CompCert languages
% can then be regarded as one possible kind of state among others,
% and replaced in specifications by more abstract data representations.

% Unfortunately,
% because of the monolithic nature of CompCert's memory,
% abstracting only part of it is challenging
% and yields complex simulation conventions.
% In Example~\ref{ex:rbcorrect},
% the abstraction relation had to involve
% not only the whole target memory state,
% but also the residual source memory state
% not subject to abstraction,
% and used complex properties to express their relationships.
% In other words,
% instead of focusing on the particular fragment of the memory
% which we seek to abstract away,
% we are to forced to characterize the effect of partial abstraction
% on the context as well,
% even though the remaining areas of the memory
% are not meaningfully involved in the task at hand.

% In this section,
% we use techniques from separation logic
% to address this problem.
% We propose to equip the CompCert memory model
% with a structure akin to separation algebra \cite{something-for-sa}
% and incorporate the resulting construction
% within the framework of simulation conventions,
% CompCert Kripke logical relations,
% and state encapsulation.

\subsection{The CompCert Memory Model}

In essence,
a CompCert memory state
assign to each possible memory address $i \in \kw{block} \times \mathbb{Z}$:
\begin{itemize}
  \item a permission level $p \in \kw{option}\,\kw{perm}$;
  \item a memory value $v \in \kw{memval}$.
\end{itemize}
In addition,
a memory state contains a $\kw{nextblock}$ counter
which keeps track of the next block identifier to be allocated.
We discuss these various components in more detail below.

\subsubsection{Memory Addresses}

The CompCert memory is divided in a number of \emph{blocks}.
As new blocks are allocated,
they are assigned a positive identifier $b \in \mathbb{N}_*$
in sequential order.
As mentioned above,
the $\kw{nextblock}$ counter within each memory state
keeps track of the smallest unallocated block identifier.
When a new block identifier is needed,
$\kw{nextblock}$ is incremented and its previous value
is used for the new block.

Memory blocks represent independent address spaces.
Within each block,
a byte can be addressed using an offset $o \in \mathbb{Z}$.
When a new block is allocated,
a range of addresses $[\mathit{lo}, \mathit{hi})$ must be provided;
this range determines which addresses within the block are valid.
However,
rather than storing the range directly within the memory state,
the allocation operation uses it to assign initial permissions
for each address within the new block.

\subsubsection{Permissions}

Each memory address within a memory state
is assigned a permission level among the following:
\[
  p \in \kw{option}\,\kw{perm} ::=
    \bot \mid
    \kw{nonempty} \mid
    \kw{readable} \mid
    \kw{writable} \mid
    \kw{freeable}
\]
The permissions are listed above in increasing order,
so that for example the permission level $\kw{writable}$ 
represents the set of permissions
$\{ \kw{nonempty}, \kw{readable}, \kw{writable} \}$.
Permissions play an important role
in the memory separation relation we define.

When a block is first allocated,
addresses within the provided range
are assigned the permission level $\kw{freeable}$;
all remaining addresses are assigned
empty permissions $\bot$.
Further memory operations may then decrease the permission level,
but can never increase it.
Memory operations which access a particular address
will first check that this address has sufficient permissions,
and fail if that is not the case.

\subsubsection{Memory Values}

Each memory value represents the contents of exactly one byte of memory.
It may be stored as a concrete byte,
or may be identified as a particular one-byte fragment
within a larger, more abstract value
(for instance, the third byte of a given pointer).

The exact representation of memory values
is not essential to the work discussed in this section.
Therefore
we will not discuss the specifics further,
but refer the interested reader to \citet{compcertmmv2}
for more background on this topic.

\subsubsection{Memory Transformations}

The compilation passes of CompCert
often transform the structure of the memory state:
multiple blocks can merged into one;
new blocks may be introduced in the target memory
and blocks may be dropped from the source memory.
To express these transformations,
CompCert introduces \emph{memory extensions} and \emph{memory injections}
as possible relations between source- and target-level memory states.

In CompCertO,
these memory transformations are generalized and consolidated
into a notion of \emph{CompCert Kripke Logical Relations} (CLKRs),
which play an important role in defining simulation conventions.
The underlying idea is that
if two memory states are related by a CKLR,
then memory operations which succeed at the source level
should also succeed on at the target level,
and their outcomes should in turn be related
by the CKLR.

Unfortunately,
these memory transformations are difficult to use
to express the relationships between
different \emph{fragments} of a single memory state.
The notion of \emph{separation relation} introduced below
seeks to fill this gap.

\subsection{Separation Relations} %{{{

To express memory separation in CompCert,
and define a \emph{join} relation
$J \subseteq (\kw{mem} \times \kw{mem}) \times \kw{mem}$.
We will write $J(m_1, m_2, m)$ as:
\[
  m_1 \bullet m_2 \equiv m
  \,,
\]
understood to mean that
the memory states $m_1$ and $m_2$
can be merged into $m$.
This relation satisfies the properties listed in Fig.~\ref{fig:sepalg}
and defines a separation algebra in the sense of \citet{freshlook}.

\begin{figure}
  \begin{gather*}
    m_1 \bullet m_2 \equiv m \:\wedge\:
      m_1 \bullet m_2 \equiv m' \:\Rightarrow\:
      m = m'
      \\
    % m_1 \bullet m_2 \equiv m \:\wedge\:
    %   m_1' \bullet m_2 \equiv m \:\Rightarrow\:
    %   m_1 = m_1'
    %   \text{ XXX}
    %   \\
    m_1 \bullet m_2 \equiv m \:\Rightarrow\:
      m_2 \bullet m_1 \equiv m
      \\
    m_1 \bullet m_2 \equiv m_{12} \:\wedge\:
      m_{12} \bullet m_3 \equiv m \:\Rightarrow\:
      \exists m_{23} \mathrel.
      m_2 \bullet m_3 \equiv m_{23} \:\wedge\:
      m_1 \bullet m_{23} \equiv m
      \\
    m \bullet \kw{empty} \equiv m
  \end{gather*}
  \caption{Properties of separation algebras
    in relational form. See also \citet{freshlook}.}
  \label{fig:sepalg}
\end{figure}

In addition to these structural properties,
the join relation must be compatible
with CompCert's memory operations.
If an operation which reads from the memory succeeds on a fragment,
it should succeed with the same result on a larger memory state:
\[
  \begin{prooftree}
    \hypo{\kw{op}(m_1) = \kw{Some}\,v}
    \hypo{m_1 \bullet m_2 \equiv m}
    \infer2{\kw{op}(m) = \kw{Some}\,v}
  \end{prooftree}
\]
Likewise,
operations which updates the memory
should be insensitive to additional fragments:
\[
  \begin{prooftree}
    \hypo{\kw{op}(m_1) = \kw{Some}\,m_1'}
    \hypo{m_1 \bullet m_2 \equiv m}
    \infer2{\exists m' \mathrel.
      m_1' \bullet m_2 \equiv m' \wedge
      \kw{op}(m) = \kw{Some}\,m'}
  \end{prooftree}
\]

Together,
these properties allow us to derive
versions of the \emph{frame rule}
for CompCert languages:
if a program can successfully execute on $m_1$ alone
to yield a new memory fragment $m_1'$,
then executing it on a larger memory state
$m_1 \bullet m_2$ will succeed as well,
and yield a memory state $m_1' \bullet m_2$
where the irrelevant portion $m_2$
has not been modified.

Moreover,
executions which affect disjoint parts of the memory
can be considered independently.
Specifically, from the rules above
we can derive the property:
\[
  \begin{prooftree}
    \hypo{\kw{op}_1(m_1) = \kw{Some}\,m_1'}
    \hypo{\kw{op}_2(m_2) = \kw{Some}\,m_2'}
    \hypo{m_1 \bullet m_2 \equiv m}
    \infer3{\exists m' \mathrel.
      \kw{op}_1(\kw{op}_2(m)) =
	%\kw{op}_2(\kw{op}_1(m)) =
	m' \:\wedge\:
      m_1' \bullet m_2' \equiv m'}
  \end{prooftree}
\]
As in separation logic,
this facilitates reasoning
about program components
which affect the memory state in independent ways.

Below we explain how a separation relation can be defined
for the CompCert memory model.

\subsubsection{Memory Contents}

A CompCert memory state essentially defines a map of type
\[
  \kw{ptr} \rightarrow \kw{option}\,\kw{perm} \times \kw{memval} \,,
\]
which assigns to every possible address
a permission level and a memory value.
Figure~\ref{fig:sepdef}
shows the definition of a simple separation relation
for the contents of individual memory cells.
This relation can then be extended to the whole map
in the obvious way.

\begin{figure}
  \begin{subfigure}{0.45\textwidth}
    \centering
    \fbox{$J_\kw{contents}$}
    \begin{gather*}
      (p, v) \in \kw{option}\ \kw{perm} \times \kw{memval} \\[1ex]
      (\bot, \kw{undef}) \bullet (p, v) \equiv (p, v) \\
      (p, v) \bullet (\bot, \kw{undef}) \equiv (p, v)
    \end{gather*}
    \subcaption{Memory contents}
    \label{fig:sepdef:contents}
  \end{subfigure}
  \begin{subfigure}{0.45\textwidth}
    \centering
    \fbox{$J_\kw{nextblock}$}
    \begin{gather*}
      (\mathit{nb}, a) \in \kw{block} \times \kw{bool}
      \\[1ex]
     {\begin{prooftree}
	\hypo{\mathrm{max}(\mathit{nb}_1, \mathit{nb}_2) = \mathit{nb}}
	\hypo{\lnot (a_1 \wedge a_2)}
	\infer2{(\mathit{nb}_1, a_1) \bullet (\mathit{nb}_2, a_2) \equiv
	  (\mathit{nb},\, a_1 \mathbin\vee a_2)}
      \end{prooftree}}
    \end{gather*}
    \subcaption{Fresh blocks}
    \label{fig:sepdef:fresh}
  \end{subfigure}
  \caption{%
    Basic ingredients for separation algebras
    of the CompCert memory model.}
  \label{fig:sepdef}
\end{figure}

\subsubsection{Block Validity}

A more challenging issue is the treatment of $\kw{nextblock}$.
When a memory state $m$ is separated into $m_1 \bullet m_2 \equiv m$,
the fragments $m_1$ and $m_2$ will share a common view of the address space.
However,
they each carry their own copy of the $\kw{nextblock}$ counter.
As a result,
performing independent allocations in each fragment
will break the separation property,
because the new blocks will be assigned conflicting names.

As a starting point,
we solve this problem by
making sure that new blocks
can only be allocated in one of the fragments.
In addition to the $\kw{nextblock}$ counter,
memory states carry a boolean flag
indicating whether allocations are permitted.
When memory fragments are joined,
this flag can only be set in one of the fragments.
Figure~\ref{fig:sepdef:fresh}
shows the corresponding separation algebra
for the $\kw{nextblock}$ counter.

%}}}

% \subsection{Frame rule} %{{{

% The compatibility of memory operations with our separation algebras
% can be used to show that
% more complex ways to manipulate memory states
% enjoy similar properties.
% Ultimately this allows us to derive
% a kind of \emph{frame rule} for the Clight semantics.
% We can state this informally as follows:
% \[
%   \begin{prooftree}
%     \hypo{\Clight(p) : m_1 \leadsto m_2}
%     \infer1{\Clight(p) : m_1 \bullet m \leadsto m_2 \bullet m}
%   \end{prooftree}
% \]
% In other words,
% if the program $p$ safely acts on a memory state $m_1$
% to transform it into a memory state $m_2$,
% then we can frame a memory fragment $m$ onto $m_1$
% and expect the program to leave that fragment intact.
% Intuitively, this holds because
% if $p$ ever needed or affected any of the memory present
% in fragment $m$,
% it would have gone wrong on $m_1$ alone.

% To formalize this property in the context of CompCertO,
% we can promote the memory separation relation
% to a simulation convention:
% \[
%   \forall A \:.\quad
%   A@{\bullet} : A@(\kw{mem} \times \kw{mem}) \leftrightarrow A@\kw{mem}
% \]
% We will then compare the ``source''-level semantics
% \[
%   \Clight(p)@\kw{mem} :
%     \mathcal{C}@(\kw{mem} \times \kw{mem}) \twoheadrightarrow
%     \mathcal{C}@(\kw{mem} \times \kw{mem})
%   \,,
% \]
% which acts on one of the memory fragments
% but leaves the other one unchanged,
% to the concrete semantics of $p$ acting on the total memory state:
% \[
%   \Clight(p) : \mathcal{C}@\kw{mem} \leftrightarrow \mathcal{C}@\kw{mem}
%   \,.
% \]
% This yields the following property.

% \begin{lemma}[Frame rule for Clight]
% \[
%   \Clight(p)@\kw{mem}
%   \le_{\mathcal{C}@{\bullet} \twoheadrightarrow \mathcal{C}@{\bullet}}
%   \Clight(p)
% \]
% \end{lemma}

%}}}

%}}}

\section{Certified Abstraction Layers} \label{app:cal} %{{{

We present the proof for
layer composition
step by step in this section.

Given the individual layer correctness:
\[
  \psi_{12} \: : \: L_1 \vdash_R M : L_2
  \quad
  \psi_{23} \: : \: L_2 \vdash_S N : L_3
  \,,
\]
we can thread the abstraction relation $R$
through the program $N$
\begin{equation}
  \label{eq:1}
  \kw{Clight}(N)\mathbin@ R :
  \kw{Clight}(N) \mathbin@ D_2
  \le_{\mathcal{C}\mathbin@ \kw{mem} \mathbin@ R \twoheadrightarrow \mathcal{C}\mathbin@ \kw{mem} \mathbin@ R}
p  \kw{Clight}(N) \mathbin@ \kw{mem} \mathbin@ D_1
  \,,
\end{equation}
and use the frame rule to combine together the memory fragments
\begin{equation}
  \label{eq:2}
  \kw{FP}(N)\mathbin@ D_1 :
  \kw{Clight}(N) \mathbin@ \kw{mem} \mathbin@ D_1
  \le_{\mathcal{C}\mathbin@ \jr \mathbin@ D_1 \twoheadrightarrow \mathcal{C}\mathbin@ \jr \mathbin@ D_1}
  \kw{Clight}(N) \mathbin@ D_1
  \,.
\end{equation}
By vertically composing (\ref{eq:1}) and (\ref{eq:2}), we have
the following self-simulation property
\begin{equation}
  \label{eq:3}
  \psi :=
  \kw{Clight}(N)\mathbin@ R \vcomp \kw{FP}(N)\mathbin@ D_1 :
  \kw{Clight}(N) \mathbin@ D_2
  \le_{\mathcal{C}\mathbin@ \hat{R} \twoheadrightarrow \mathcal{C}\mathbin@ \hat{R}}
  \kw{Clight}(N) \mathbin@ D_1
  \,.
\end{equation}
The simulation (\ref{eq:3}) can then be horizontally composed
with the underlay correctness $\psi_{12}$
\begin{equation}
  \label{eq:4}
  \psi \odot \psi_{12} :
  \kw{Clight}(N) \mathbin@ D_2 \odot L_2
  \le_{\top \twoheadrightarrow \mathcal{C}\mathbin@ \hat{R}}
  \kw{Clight}(N) \mathbin@ D_1 \odot \kw{Clight}(M) \mathbin@ D_1 \odot L_1
  \,.
\end{equation}
Finally, we put the overlay correctness on top of (\ref{eq:4})
\begin{equation}
  \label{eq:5}
  \psi_{23}\vcomp (\psi \odot \psi_{12}) :
  L_3
  \le_{\top \twoheadrightarrow \mathcal{C}\mathbin@ (\hat{S} \vcomp \hat{R})}
  (\kw{Clight}(N) \mathbin \odot \kw{Clight}(M)) \mathbin@ D_1 \odot L_1
  \,,
\end{equation}
and
by applying
the structural isomorphism
$\alpha : (\hat{S} \vcomp \hat{R}) \cong \widehat{R \cdot S}$,
we obtain the conclusion in \S\ref{sec:application:cal}
\[
  \alpha \odot \big(
  \psi_{23} \vcomp (\psi \odot \psi_{12}) \big) :
  L_3
  \le_{\top \twoheadrightarrow \mathcal{C}\mathbin@ \widehat{R \cdot S}}
  (\kw{Clight}(N) \mathbin \odot \kw{Clight}(M)) \mathbin@ D_1 \odot L_1
  \,.
\]
%}}}

\section{Clight with module-local state} \label{sec:appendix:clightp} %{{{

\newcommand{\clightp}[1]{\kw{ClightP} \langle #1 \rangle}

We present the proof for
composing the correctness of ClightP compilation
in this section.

First of all,
the frame property extends to the $\kw{ClightP}$ semantics:
\[
  \kw{FP'}: \clightp{M} \le_{\mathcal{C} \mathbin@ \jr \twoheadrightarrow \mathcal{C} \mathbin@ \jr} \clightp{M}
\]
Then given the correctness for $M$ and $N$
\[
  \pi_M: \clightp{M} \le_{\mathbf{R} \twoheadrightarrow \mathbf{S}(m_0)} \kw{Clight}(M')
  \quad
  \pi_N: \clightp{N} \le_{\mathbf{R} \twoheadrightarrow \mathbf{S}(n_0)} \kw{Clight}(N')
  \,,
\]
where $\mathbf{R}$ and $\mathbf{S}(m_0)$ are shorthands for
$\mathcal{C} \mathbin@ \kw{mem} \bullet \callee{\kw{mem}}^*$
and
$\mathcal{C} \mathbin@ \kw{mem} \bullet \caller{m_0}_*$,
we utilize the following properties:
\[
  \begin{array}{c}
    \phi_M : \clightp{M} \le_{\mathbf{R} \twoheadrightarrow \mathbf{R}} \clightp{M}
    := (\clightp{M} \mathbin@ \callee{\kw{mem}}^*) \vcomp \kw{FP'}  \\[1ex]
    \phi_N : \kw{Clight}(N') \le_{\mathbf{S}(m_0) \twoheadrightarrow \mathbf{S}(m_0)} \kw{Clight}(N')
    := (\kw{Clight}(N') \mathbin@ \caller{m_0}_*) \vcomp \kw{FP'} \\[1ex]
    \alpha : \mathbf{R} \sqsubseteq \mathbf{R} \vcomp \mathbf{R}
    \qquad
    \beta : \mathbf{S}(n_0) \vcomp \mathbf{S}(m_0) \sqsubseteq \mathbf{S}(n_0 \bullet m_0)
  \end{array}
  \,.
\]
where the refinement between simulation conventions $\alpha$ and $\beta$ follows
the associativity of the $\bullet$ operator.
By composing together the properties,
we obtain the composite correctness of \kw{ClightP} compilation:
\[
  \beta \odot ((\pi_N \vcomp \phi_N) \odot (\phi_M \vcomp \pi_M) ) \odot \alpha :
  \clightp{N} \odot \clightp{M} \le_{\mathbf{R} \twoheadrightarrow \mathbf{S}(n_0 \bullet m_0)} \kw{Clight}(N') \odot \kw{Clight}(M')
  \,.
\]


%}}}

\fi

\end{document}
\endinput

\section{Cut String Diagrams Material} %{{{

\paragraph{A Geometric Analogy}

At this point we invite the reader
to consider the high-level
algebra of composition that
transition systems and simulations conform to.
In what is described above:
\begin{itemize}
  \item Transition systems are 0-dimensional objects,
    akin to the vertices of a graph or
    the points of a topological space.
    They do not compose
    but provide ``endpoints'' for simulations.
  \item Simulations are 1-dimensional objects,
    similar to edges or paths;
    they connect with each~other
    when the 0-dimensional target of one coincides with
    the 0-dimensional source of another.
    %at their 0-dimensional endpoints.
\end{itemize}
%In this analogy,
%proving CompCert correct
%boils down to constructing a simulation path
%from the source to the target semantics.
%This involves using intermediate programs and language semantics
%to identify ``waypoints'' and reduce the problem
%to proving more elementary simulations.
%
%Algebraically speaking,
%we have described a
%\emph{category} of transition systems and simulation relations.
%
We will see that
when we incorporate more composition principles into the framework,
the dimensionality of transition systems and simulation proofs will increase.
Ultimately,
simulations in our framework will be 3-dimensional objects;
the principle will remain, however,
that objects of dimension $n+1$ can be connected
alongside a common boundary of dimension $n$.

%Ultimately,
%horizontal and spatial composition
%will turn simulation proofs into 3-dimensional objects.
%An important starting point for our work
%will be to provide a rigorous account of the way
%these different composition principles interact,
%and to introduce tools such as string diagrams
%which can help leverage the physical intuitions
%outlined above to deal with the complexity
%of the objects we manipulate.
%\emph{Category theory} provides a systematic study
%of compositional structures of this kind
%\citep{rosetta},
%and our approach draws heavily from it.
%However,
%to the extent possible,
%in our exposition we have tried to avoid
%assuming familiarity with category theory
%on the reader's part.

  \text{(c)}
  \raisebox{-0.5\height}{
    \begin{tikzpicture}[xscale=0.22,yscale=0.4]
      \small
      \fill[color=ACMBlue!20] (-4,+2) rectangle (+4,-2);
      \begin{scope}[rounded corners]
        % Input wires
        \draw (-3,+2) node[above] {$L_1$} -- (-3,+1) -- (0,0);
        \draw (-1,+2) node[above] (L2) {$L_2$} -- (-1,+1) -- (0,0);
        \draw (+3,+2) node[above] (Ln) {$L_n$} -- (+3,+1) -- (0,0);
        \path (L2) -- node[yshift=-1pt] {$\cdots$} (Ln);
        % Output wires
        \draw (-3,-2) node[below] {$L_1'$} -- (-3,-1) -- (0,0);
        \draw (-1,-2) node[below] (M2) {$L_2'$} -- (-1,-1) -- (0,0);
        \draw (+3,-2) node[below] (Mm) {$L_m'$} -- (+3,-1) -- (0,0);
        \path (M2) -- node[yshift=-1pt] {$\cdots$} (Mm);
      \end{scope}
      \node[circle,draw,fill=white,inner sep=1pt] {$\phi$};
    \end{tikzpicture}
  }

  \text{(d)}
  \raisebox{-0.5\height}{
    \begin{tikzpicture}[xscale=0.7,yscale=0.35]
      \small
      \fill[color=ACMBlue!20] (-3,+2) rectangle (+2.2,-4);
      \begin{scope}[rounded corners]
        \draw (-2.5,+2) node[above] {$C$}
           -- (-2.5,-4) node[below] {$C$};
        \draw (0,2) node[above] {$L_\kw{bq}$} -- (0,0);
        \draw (0,0)
          -- (-1,-1) node[left,inner sep=1pt] {\footnotesize $\kw{bq.c}$}
          -- (-1,-4) node[below] {$\kw{Asm}(\kw{bq.s})\:$};
        \draw (0,0)
          -- (+1,-1) node[right,inner sep=1pt] {\footnotesize $\kw{rb.c}$}
          -- (+1,-4) node[below] {$\:\kw{Asm}(\kw{rb.s})$};
      \end{scope}
      \begin{scope}[every node/.style={circle,draw,fill=white,inner sep=1pt}]
        \node at (0,0) {$\phi$};
        \node at (-1,-2.5) {\scriptsize $\pi_\kw{bq}$};
        \node at (+1,-2.5) {\scriptsize $\pi_\kw{rb}$};
      \end{scope}
    \end{tikzpicture}
  }

    (c,d) We use string diagram representations for simulations;
    the abbreviations $\kw{bq.c}$ and $\kw{rb.c}$
    denote the semantics
    $\kw{Clight}(\kw{bq.c})$ and $\kw{Clight}(\kw{rb.c})$
    of the corresponding files.

\paragraph{String Diagrams} %{{{

As morphisms of a monoidal category,
simulations in Compositional CompCert
admit a \emph{string diagram} notation.
A simulation relation
$
  \phi : L_1 \oplus L_2 \oplus \cdots \oplus L_n \le
        L_1' \oplus L_2' \cdot \cdots \oplus L_m'
$
is depicted as shown in Fig.~\ref{fig:pasting}c.
There, we use a single node $\phi$
as the entire simulation proof,
but diagrams with more complex structures
are used to denote composite simulation relations:
\begin{itemize}
\item
$\phi_1 \vcomp \phi_2$
is depicted
by connecting the output wires of
the diagram $\phi_1$
to the input wires of $\phi_2$;
\item
we represent
$\phi_1 \oplus \phi_2$
by the horizontal juxtaposition of the two diagrams.
\end{itemize}
Using these conventions,
the simulation relation
$\kw{id}_C \oplus (\phi \vcomp (\pi_\kw{bq} \oplus \pi_\kw{rb}))$
%(\ref{eqn:ccex})
above can be depicted
as shown in Fig.~\ref{fig:pasting}d.
Note that the simulation $\kw{id}_C$
does not have to be represented as an explicit node.

String diagrams can be devised for a variety of 
two-dimensional structures;
we use many different kinds in our exposition below.
Their geometry
captures the compositional structure and properties
of the underlying objects,
more comprehensively %compactly and accurately
than the pasting diagram
we used in Fig.~\ref{fig:pasting}b.

%}}}

\paragraph{String Diagrams} %{{{

This equips the model
with the structure of a \emph{double category}.

Double categories admit a
string diagram notation \cite{dcsd}
which we use to represent simulation proofs.
\autoref{fig:compcerto}a shows the general form
of a diagram for the simulation
\[
  \phi \: : \:
  L_1 \odot L_2 \odot \cdots \odot L_n
  \: \le_{
    \mathbf{R}_1 \vcomp \mathbf{R}_2 \vcomp \cdots \mathbin \mathbf{R}_k
    \twoheadrightarrow
    \mathbf{S}_1 \vcomp \mathbf{S}_2 \vcomp \cdots \mathbin \mathbf{S}_l
  } \:
  L_1' \odot L_2' \odot \cdots \odot L_m'
  \,.
\]
%In diagrams of this kind,
Regions %of the plane
are labeled by language interfaces.
Horizontal morphisms (transition systems)
are represented by vertical lines,
with the composition
$L_1 \odot L_2 \odot \cdots \odot L_n$
running from left to right.
Vertical morphisms (simulation conventions)
are represented by horizontal lines,
with the composition
$\mathbf{R}_1 \vcomp \mathbf{R}_2 \vcomp \cdots \vcomp \mathbf{R}_k$
running from top to bottom.
Identity morphisms can be omitted,
and the simulations
$
  \kw{id}_L :
    L \le_{\kw{id} \twoheadrightarrow \kw{id}} L
$ and $
  \kw{id}_\mathbf{R} :
    \kw{id} \le_{\mathbf{R} \twoheadrightarrow \mathbf{R}} \kw{id}
$
can be represented by naked vertical and horizontal lines.
Diagrams with matching boundaries
can be connected horizontally or vertically,
per (\ref{eqn:hvcomp}).

%}}}

\begin{figure} % fig:compcerto {{{
  \text{(a)}
  \quad
  \raisebox{-0.5\height}{\begin{tikzpicture}[xscale=0.2,yscale=0.16]
    \footnotesize
    \newcommand{\filltint}{30}

    % Coordinates
    \path (0,0) coordinate (C)
      (-3,+5) coordinate (L1c) +(0,+4) coordinate (L1)
      (-1,+5) coordinate (L2c) +(0,+4) coordinate (L2)
      (+3,+5) coordinate (Lnc) +(0,+4) coordinate (Ln)
      (+5, 3) coordinate (S1c) +(+4,0) coordinate (S1)
      (+5, 1) coordinate (S2c) +(+4,0) coordinate (S2)
      (+5,-3) coordinate (Snc) +(+4,0) coordinate (Sn)
      (-3,-5) coordinate (M1c) +(0,-4) coordinate (M1)
      (-1,-5) coordinate (M2c) +(0,-4) coordinate (M2)
      (+3,-5) coordinate (Mnc) +(0,-4) coordinate (Mn)
      (-5, 3) coordinate (R1c) +(-4,0) coordinate (R1)
      (-5, 1) coordinate (R2c) +(-4,0) coordinate (R2)
      (-5,-3) coordinate (Rnc) +(-4,0) coordinate (Rn)
      ;

    % Background regions
    \fill[ACMBlue!\filltint] (C)
      [rounded corners] -- (L1c)
      [sharp corners] -- (L1) -- (L2)
      [rounded corners] -- (L2c)
      [sharp corners] -- cycle;
    \fill[ACMLightBlue!\filltint] (C)
      [rounded corners] -- (Lnc)
      [sharp corners] -- (Ln) -| (S1)
      [rounded corners] -- (S1c)
      [sharp corners] -- cycle;
    \fill[ACMGreen!\filltint] (C)
      [rounded corners] -- (S2c)
      [sharp corners] -- (S2) -- (S1)
      [rounded corners] -- (S1c)
      [sharp corners] -- cycle;
    \fill[ACMYellow!\filltint] (C)
      [rounded corners] -- (Mnc)
      [sharp corners] -- (Mn) -| (Sn)
      [rounded corners] -- (Snc)
      [sharp corners] -- cycle;
    \fill[ACMOrange!\filltint] (C)
      [rounded corners] -- (M1c)
      [sharp corners] -- (M1) -- (M2)
      [rounded corners] -- (M2c)
      [sharp corners] -- cycle;
    \fill[ACMRed!\filltint] (C)
      [rounded corners] -- (M1c)
      [sharp corners] -- (M1) -| (Rn)
      [rounded corners] -- (Rnc)
      [sharp corners] -- cycle;
    \fill[ACMPurple!\filltint] (C)
      [rounded corners] -- (R2c)
      [sharp corners] -- (R2) -- (R1)
      [rounded corners] -- (R1c)
      [sharp corners] -- cycle;
    \fill[ACMDarkBlue!\filltint] (C)
      [rounded corners] -- (L1c)
      [sharp corners] -- (L1) -| (R1)
      [rounded corners] -- (R1c)
      [sharp corners] -- cycle;

    % Region labels
    \begin{scope}[opacity=0.66,outer sep=1pt]
      \tiny

      % Language interfaces
      \path (R1) |- node[below right] {$Z_0$} (L1);
      \path (R1) -- node[right] {$Z_1$} (R2);
      \path (Rn) |- node[above right] {$Z_l$} (M1);
      \path (M1) -- node[above] {$Y'$} (M2);
      \path (Mn) -| node[above left] {$A_k$} (Sn);
      \path (L1) -- node[below] {$Y$} (L2);
      \path (Ln) -| node[below left] {$A_0$} (S1);
      \path (S1) -- node[left] {$A_1$} (S2);

      % Dot dot
      \path (L2) -- node[below,yshift=-2pt] {$\cdots$} (Ln);
      \path (S2) -- node[left,yshift=3pt,xshift=-2pt] {$\vdots$} (Sn);
      \path (M2) -- node[above] {$\cdots$} (Mn);
      \path (R2) -- node[right,yshift=3pt,xshift=2pt]  {$\vdots$} (Rn);
    \end{scope}

    % Strings
    \begin{scope}
      \draw (C)
        [rounded corners] -- (L1c)
        [sharp corners] -- (L1) node[above] {$L_1$};
      \draw (L2) node[above] {$L_2$}
        [rounded corners] -- (L2c)
        [sharp corners] -- (C);
      \draw (C)
        [rounded corners] -- (Lnc)
        [sharp corners] -- (Ln) node[above] {$L_n$};
      \draw (C)
        [rounded corners] -- (Mnc)
        [sharp corners] -- (Mn) node[below] {$L'_m$};
      \draw (M2) node[below] {$L'_2$}
        [rounded corners] -- (M2c)
        [sharp corners] -- (C);
      \draw (C)
        [rounded corners] -- (M1c)
        [sharp corners] -- (M1) node[below] {$L'_1$};
    \end{scope}
    \begin{scope}%[thick]
      \draw (S1) node[right] {$\mathbf{R}_1$}
        [rounded corners] -- (S1c)
        [sharp corners] -- (C);
      \draw (C)
        [rounded corners] -- (S2c)
        [sharp corners] -- (S2) node[right] {$\mathbf{R}_2$};
      \draw (Sn) node[right] {$\mathbf{R}_k$}
        [rounded corners] -- (Snc)
        [sharp corners] -- (C);
      \draw (R1) node[left] {$\mathbf{S}_1$}
        [rounded corners] -- (R1c)
        [sharp corners] -- (C);
      \draw (C)
        [rounded corners] -- (R2c)
        [sharp corners] -- (R2) node[left] {$\mathbf{S}_2$};
      \draw (Rn) node[left] {$\mathbf{S}_l$}
        [rounded corners] -- (Rnc)
        [sharp corners] -- (C);
    \end{scope}

    % Node
    \node[draw,fill=white,circle,inner sep=2pt] at (C) {$\phi$};

  \end{tikzpicture}}
  \qquad
  \text{(b)}
  \quad
  \raisebox{-0.5\height}{\begin{tikzpicture}[xscale=0.85,yscale=0.5]
    \footnotesize
    \newcommand{\filltint}{30}

    % Background regions
    \fill[ACMBlue!\filltint]
      (-3, 0)
      [rounded corners] -- (2,0)
      [sharp corners] -- (2.5,1)
      [rounded corners] -- (2,2)
      [sharp corners] -- (0,2) -- (0,3) -| cycle;
    \fill[ACMRed!\filltint]
      (-3, 0)
      [rounded corners] -- (2,0)
      [sharp corners] -- (2.5,1) -- (3.5,1) -- (3.5,-3) -| cycle;
    \fill[pattern=crosshatch,opacity=0.15]
      (3.5,1) -- (2.5,1)
      [rounded corners] -- (2,2)
      [sharp corners] -| (0,3) -| cycle;

    % Region labels
    \begin{scope}[opacity=0.66,outer sep=1pt]
      \tiny
      \node[below right] at (-3,3) {$\mathcal{C}_\kw{m}$};
      \node[below left] at (3.5,3) {$\top$};
      \node[above right] at (-3,-3) {$\mathcal{A}_\kw{m}$};
      \node[above left] at (3.5,-3) {$\mathcal{A}_\kw{m}$};
    \end{scope}

    % Strings
    \begin{scope}
      % Transition systems
      \draw (0,3) node[above] {$L_\kw{bq}$}
         -- (0,2)
         [rounded corners]
         -- (-1, 1) node[left,inner sep=1pt] {$\kw{bq.c}$}
         -- (-1,-1) node[left,inner sep=1pt] {$\kw{bq.s}$}
         [sharp corners]
         -- (0,-2)
         -- (0,-3) node[below] {$\kw{Asm}(\kw{bq.s} + \kw{rb.s})$};
      \draw (0,2)
         [rounded corners]
         -- (1, 1) node[right,inner sep=1pt] {$\kw{rb.c}$}
         -- (1,-1) node[right,inner sep=1pt] {$\kw{rb.s}$}
         [sharp corners]
         -- (0,-2);
      % Simulation conventions
      \draw (-3,0) node[left] {$\mathbb{C}$}
         [rounded corners] -- (2,0)
         [sharp corners] -- (2.5,1) -- (3.5,1) node[right] {$\varnothing$};
      \draw (0,2)
         [rounded corners] -- node[above] {\tiny $\varnothing$} (2,2)
         [sharp corners] -- (2.5,1);
    \end{scope}

    % Nodes
    \begin{scope}[every node/.style={draw,fill=white,circle,inner sep=2pt}]
       \node at (0,2) {$\phi$};
       \node[inner sep=1pt] at (-1,0) {$\pi_\kw{bq}$};
       \node[inner sep=1pt] at (+1,0) {$\pi_\kw{rb}$};
       \node at (0,-2) {$\ell$};
       \node at (2.5,1) {$z$};
    \end{scope}
  \end{tikzpicture}}
%  \quad
%  \text{(c)}
%    \small
%  \begin{tikzcd}[sep=1ex,row sep=0.5ex]
%    \top
%      \ar[rr, equal]
%      \ar[dddd, leftrightarrow, "\varnothing"']
%      &&
%    \top
%      \ar[dd, leftrightarrow, "\varnothing"']
%      \ar[rrrr, "L_\kw{bq}"]
%      &&&&
%    \mathcal{C}_\kw{m}
%      \ar[dd, equal]
%    %  \ar[rr, "C"] &&
%    %\mathcal{C}_\kw{m}
%    %  \ar[dddddd, leftrightarrow, "\mathbb{C}"]
%    \\
%    &&&& \phi
%    \\
%    & z &
%    \mathcal{C}_\kw{m}
%      \ar[dd, leftrightarrow, "\mathbb{C}"]
%      \ar[rr, "\kw{Clight}(\kw{rb.c})"] &&
%    \mathcal{C}_\kw{m}
%      \ar[dd, leftrightarrow, "\mathbb{C}"]
%      \ar[rr, "\kw{Clight}(\kw{bq.c})"] &&
%    \mathcal{C}_\kw{m}
%      \ar[dd, leftrightarrow, "\mathbb{C}"]
%    \\
%    &&
%    & \pi_1 & & \pi_2
%    %&& \pi_C \!\!
%    \\
%    \mathcal{A}_\kw{m} \ar[rr, equal] \ar[dd, equal] &&
%    \mathcal{A}_\kw{m} \ar[dd, equal] \ar[rr, "\kw{Asm}(\kw{rb.s})"'] &&
%    \mathcal{A}_\kw{m} \ar[rr, "\kw{Asm}(\kw{bq.s})"'] &&
%    \mathcal{A}_\kw{m} \ar[dd, equal]
%    \\
%    &  &&& \ell
%    \\
%    \mathcal{A}_\kw{m} \ar[rr, equal] &&
%    \mathcal{A}_\kw{m} \ar[rrrr, "\kw{Asm}(\kw{rb+bq.s})"'] && &&
%    \mathcal{A}_\kw{m} %\ar[rr, "C'"'] &&
%    %\mathcal{A}_\kw{m}
%  \end{tikzcd}
  \caption{
    Under layered composition,
    the CompCertO model is a \emph{double category}
    and admits a string diagram notation for its simulations.
    Shown here are (a) the general form
    (b) the simulation described in Example~\ref{ex:compcerto-sd}.}
  \label{fig:compcerto}
\end{figure}
%}}}

From Example~\ref{ex:abspec}:
\[
  \begin{tikzpicture}[yscale=0.44,xscale=1.1,baseline=(z.base)]
    \newcommand{\filltint}{30}
    \small

    \coordinate (b) at (0,2.7);

    % Background areas
    \fill[ACMPurple!\filltint] (-2,4) -| (0,5) -| cycle;
    \fill[ACMLightBlue!\filltint] (-2,4) -| (0,2) -| cycle;
    \fill[pattern=crosshatch,opacity=0.15]
      (0,5) -| (3,1) -- (2.5,1)
      [rounded corners] -- (2,2)
      [sharp corners] -- (1,2)
      [rounded corners] -- (1,3)
      [sharp corners] -- (0,4) -- cycle;
    \fill[ACMDarkBlue!\filltint]
      (0,4) |- (1,2) [rounded corners] -- (1,3) [sharp corners] -- cycle;
    \fill[ACMBlue!\filltint] (-2,2)
      [rounded corners] -- (2,2)
      [sharp corners] -- (2.5,1)
      [rounded corners] -- (2,0)
      [sharp corners] -| cycle;
    \fill[ACMRed!\filltint] (-2,0) |- (3,-3) -- (3,1) -- (2.5,1)
      [rounded corners] -- (2,0)
      [sharp corners] -- cycle;

    % Breaks
    \begin{scope}[
      every path/.style={
        draw=white,
        decorate,decoration={zigzag,aspect=0,amplitude=0.8pt},line width=2.5pt,
        opacity=1
      }]
      \draw (-2,4) node[left] {$?$} -- (0,4);
      \draw (0,4) -- (0,2);
      \draw (1,2) -- (-2,2) node[left] {$?$};
    \end{scope}

    \begin{scope}[opacity=0.5,outer sep=2pt]
      \tiny
      \node[above right] at (-2,4) {$\mathcal{C} \mathbin@ D_\kw{bq}$};
      \node[below left] at (3,5) {$\top$};
      \node at (0,1) {$\mathcal{C} \mathbin@ \kw{mem}$};
      \node[above left] at (3,-3) {$\mathcal{A} \mathbin@ \kw{mem}$};
      \node[above right] at (0,2.4) {$\mathcal{C} \mathbin@ D_\kw{rb}$};
      \node[above right] at (-2,2) {$\mathcal{C}$};
    \end{scope}

    % Strings
    \begin{scope}
      \small
      \draw (0,5) node[above] {$\Gamma_\kw{bq}$} -- (0,4)
        [rounded corners] -- (-1,3) node[left] {$\Sigma_\kw{bq}$}
        [rounded corners] -- (-1,-1) node[left,pos=0.5] {$\kw{bq.c}$}
          node[left,pos=1] {$\kw{bq.s}$}
        [sharp corners] -- (0,-2)
          -- (0,-3) node[below] {$\kw{Asm}(\kw{bq.s+rb.s})$};
      \draw (0,4)
        [rounded corners] -- (1,3) node[right] {$\Gamma_\kw{rb}$}
        [rounded corners] -- (1,-1) node[right,pos=0.5] {$\kw{rb.c}$}
          node[right,pos=1] {$\kw{rb.s}$}
        [sharp corners] -- (0,-2);
      \draw (1,2)
        [rounded corners] -- (2,2)
        [sharp corners] -- (2.5,1) -- (3,1) node[right] {$\varnothing$};
      \draw (2.5,1)
        [rounded corners] -- (2,0)
        [sharp corners] -- (-2,0) node[left] {$\mathbb{C}$};
    \end{scope}

    % Nodes
    \begin{scope}[every node/.style={circle,draw,fill=white,inner sep=1pt}]
      \node at (0,4) {$\phi_1$};
      \node at (-1,2) {$\phi_2$};
      \node at (+1,2) {$\phi_\kw{rb}$};
      \node at (-1,0) {$\pi_\kw{bq}$};
      \node at (+1,0) {$\pi_\kw{rb}$};
      \node[inner sep=2pt] at (0,-2) {$\ell$};
      \node[inner sep=2pt] (z) at (2.5,1) {$z$};
    \end{scope}
  \end{tikzpicture}
\]

The simulation properties can be depicted as bends: %{{{
\[
    L^\triangle :
    \begin{tikzpicture}[scale=0.4,baseline=0.25cm] %{{{
      % Background
      \begin{scope}
        \fill[ACMBlue!50] (0,2) -- (0,1)
          [rounded corners] -- (1,1)
          [sharp corners] -- (1,0) -- (2,0) |- cycle;
        \fill[ACMLightBlue!50] (0,1)
          [rounded corners] -- (1,1)
          [sharp corners] -- (1,0) -| cycle;
      \end{scope}
      % Region labels
      \begin{scope}[opacity=0.5]
        \tiny
        \node[above right] at (0,0) {$B$};
        \node[below left] at (2,2) {$A$};
      \end{scope}
      % Strings
      \begin{scope}
        \footnotesize
        \draw (0,1) node[left] {$L^*$}
          [rounded corners] -- (1,1)
          [sharp corners] -- (1,0)
          node[below] {$L$};
      \end{scope}
    \end{tikzpicture}
    %}}}
    \qquad
    L^\triangledown : \:
    \begin{tikzpicture}[scale=0.4,baseline=0.25cm] %{{{
      % Background
      \begin{scope}
        \fill[ACMBlue!50] (1,2)
          [rounded corners] -- (1,1)
          [sharp corners] -- (2,1) |- cycle;
        \fill[ACMLightBlue!50] (0,2) -- (1,2)
          [rounded corners] -- (1,1)
          [sharp corners] -- (2,1) -- (2,0) -| cycle;
      \end{scope}
      % Region labels
      \begin{scope}[opacity=0.5]
        \tiny
        \node[above right] at (0,0) {$B$};
        \node[below left] at (2,2) {$A$};
      \end{scope}
      % Strings
      \begin{scope}
        \footnotesize
        \draw (1,2) node[above] {$L$}
          [rounded corners] -- (1,1)
          [sharp corners] -- (2,1)
          node[right] {$L^*$};
      \end{scope}
    \end{tikzpicture}
    %}}}
    \qquad
    \qquad
    L_\triangle : \:
    \begin{tikzpicture}[scale=0.4,baseline=0.25cm] %{{{
      % Background
      \begin{scope}
        \fill[ACMBlue!50] (1,0)
          [rounded corners] -- (1,1)
          [sharp corners] -- (2,1) |- cycle;
        \fill[ACMLightBlue!50] (0,0) -- (1,0)
          [rounded corners] -- (1,1)
          [sharp corners] -- (2,1) -- (2,2) -| cycle;
      \end{scope}
      % Region labels
      \begin{scope}[opacity=0.5]
        \tiny
        \node[below right] at (0,2) {$B$};
        \node[above left] at (2,0) {$A$};
      \end{scope}
      % Strings
      \begin{scope}
        \footnotesize
        \draw (1,0) node[below] {$L$}
          [rounded corners] -- (1,1)
          [sharp corners] -- (2,1)
          node[right] {$L_*$};
      \end{scope}
    \end{tikzpicture}
    %}}}
    \:\quad
    L_\triangledown :
    \begin{tikzpicture}[scale=0.4,baseline=0.25cm] %{{{
      % Background
      \begin{scope}
        \fill[ACMBlue!50] (2,2) -- (1,2)
          [rounded corners] -- (1,1)
          [sharp corners] -- (0,1) -- (0,0) -| cycle;
        \fill[ACMLightBlue!50] (1,2)
          [rounded corners] -- (1,1)
          [sharp corners] -- (0,1) |- cycle;
      \end{scope}
      % Region labels
      \begin{scope}[opacity=0.5]
        \tiny
        \node[below right] at (0,2) {$B$};
        \node[above left] at (2,0) {$A$};
      \end{scope}
      % Strings
      \begin{scope}
        \footnotesize
        \draw (1,2) node[above] {$L$}
          [rounded corners] -- (1,1)
          [sharp corners] -- (0,1)
          node[left] {$L_*$};
      \end{scope}
    \end{tikzpicture}
    %}}}
\]
%}}}

\subsection{Higher-Dimensional Structure} \label{sec:overview:high-dimension} %{{{

[TODO: rewrite to introduce string diagrams for LTS and simulation conventions,
as well as 3D algebra, but with minimal reliance on categorical jargon or
the 2D string diagrams for simulation proofs that we no longer
introduce before]

\newcommand{\pasv}{%
  \begin{tikzpicture}[baseline=0.1ex,x=1.2ex,y=1.2ex]
    %\fill[sdbg] (0,0) rectangle (1,1);
    \draw (0,0) rectangle (1,1);
  \end{tikzpicture}}
\newcommand{\actv}{%
  \begin{tikzpicture}[baseline=0.1ex,x=1.2ex,y=1.2ex]
    %\fill[sdbg] (0,0) rectangle (1,1);
    \fill[pattern=north west lines] (0,0) rectangle (1,1);
    \draw (0,0) rectangle (1,1);
  \end{tikzpicture}}

Recall that
Compositional CompCert,
by introducing semantic linking $\oplus$,
turned transition systems
from 0- to 1-dimensional objects,
and simulations
from 1- to 2-dimensional.
Then CompCertO,
with language interfaces,
reintroduced non-trivial 0-dimensional objects
into the framework.

With spatial composition,
we have once again shifted
dimensionality by one:
\begin{itemize}
\item
interfaces are now 1-dimensional and compose spatially ($\mathbin@$),
\item
transition systems ($\mathbin@$, $\odot$) and
simulation conventions ($\mathbin@$, $\,\vcomp\,$) are now 2-dimensional,
and \item simulations are 3-dimensional ($\mathbin@$, $\odot$, $\,\vcomp\,$).
\end{itemize}
Moreover,
to make sure a composition
$A \mathbin@ U_1 \mathbin@ \cdots \mathbin@ U_n$
is well-formed and
only involves a language interface on the left,
we can use two 0-dimensional endpoints
$\actv$ and $\pasv$ as follows:
\[
  \begin{tikzcd}
    \actv \ar[r, bend left, "A"] &
    \pasv \ar[r, bend left, "U_1"] &
    \pasv \ar[r, bend left, "\cdots"] &
    \pasv \ar[r, bend left, "U_n"] &
    \pasv
  \end{tikzcd}
\]
That is, we describe
language interfaces as $A : \actv \curvearrowright \pasv$ and
sets as $U : \pasv \curvearrowright \pasv$.
Spatial composition in general can then be given as
$
  {\mathbin@}_{\alpha,\beta,\gamma} :
    (\alpha \curvearrowright \beta) \times
    (\beta \curvearrowright \gamma) \rightarrow
    (\alpha \curvearrowright \gamma)
$,
where
$\mathbin@_{\actv\,,\,\pasv\,,\,\pasv}$ is the operation we have described and
$\mathbin@_{\pasv\,,\,\pasv\,,\,\pasv}$ is the cartesian product of sets.

This structure, and
the ways in which $\mathbin@$ interacts with horizontal and vertical composition,
can be captured with string diagram notations for (now 2-dimensional)
transition systems and simulation conventions.
In these diagrams,
horizontal and vertical composition
retain their orientation
while spatial composition
runs in the complementary direction.

Finally,
spatial composition can also be incorporated
as \emph{depth}
into simulation diagrams,
which then become 3-dimensional.
For the sake of legibility,
we will usually omit the internal structure of such string diagrams,
and depict only their boundary,
ie.\@ the type of the corresponding simulations.

\begin{figure} % fig:3dsd {{{
\[
  \text{(a)} \quad
  \vcenter{\hbox{%
  \begin{tikzpicture}[sdp]

    %% Left and bottom faces

    % Background area
    \fill[tssdbg] (0,0,0) -- (0,6,0) -- (0,6,8)
               -- (0,0,8) -- (8,0,8) -- (8,0,0) -- cycle;
    \draw[thin,dotted] (0,0,0) -- (0,0,8);
    \fill[act] (0,0,0) -- (0,6,0)
      -- (0,6,2) -- (0,0,2)
      [rounded corners] -- (1,0,2)
      [sharp corners] -- (2,0,3)
      [rounded corners] -- (3,0,2) -- (5,0,2)
      [sharp corners] -- (6,0,4) -- (8,0,4)
      -- (8,0,0) -- cycle;

    % Strings
    \draw (0,6,2) -- (0,0,2)
      [rounded corners] -- (1,0,2)
      [sharp corners] -- (2.5,0,3)
      [rounded corners] -- (4,0,2) -- (5,0,2)
      [sharp corners] -- (6,0,4)
      -- (8,0,4) node[right] {\footnotesize $\top$};
    \draw (0,4,3.5)
      node[scn,bln] {}
      -- (0,3,3.5) \companion
      -- (0,0,3.5)
      node[above right,inner sep=1pt] {\tiny $\kw{mem}$}
      [rounded corners] -- (1,0,3.5)
      [sharp corners] -- (2.5,0,3)
      node[tsn] {$\kw{bq.c}$}
      [rounded corners] -- (4,0,4)
      [sharp corners] -- (4.5,0,4)
      node[tsn,bln] {};
    \draw (0,6,6)
      -- (0,2.7,6)
      node[scn] {$R_\kw{bq}$}
      -- (0,0,6)
      node[above right,inner sep=1pt] {\tiny $D_\kw{rb}$}
      [rounded corners] -- (5,0,6)
      [sharp corners] -- (6,0,4)
      node[tsn] (rb) {$\Gamma_\kw{rb}$}
      -- (8,0,4);

    %% Center label

    \node%[draw,circle,inner sep=1pt]
       at (4,3,4) {$\phi_\kw{bq}$};

    %% Top and right

    % Background
    \fill[tssdbg,opacity=0.6]
      (0,6,0) -- (8,6,0) -- (8,0,0) -- (8,0,8) -- (8,6,8) -- (0,6,8) -- cycle;
    \draw[thin,dotted] (8,6,0) -- (8,6,8);
    \fill[act]
      (0,6,0) -- (0,6,2)
      [rounded corners] -- (3,6,2)
      [sharp corners] -- (4,6,4)
      -- (8,6,4) -- (8,0,4) -- (8,0,0) -- (8,6,0) -- cycle;

    % Strings and nodes
    \draw (0,6,2) node[above left,inner sep=1pt] {\footnotesize $\mathcal{C}$}
      [rounded corners] -- (3,6,2)
      [sharp corners] -- (4,6,4)
      -- (8,6,4) -- (8,0,4);
    \draw (0,6,6) node[above left, inner sep=1pt] {\footnotesize $D_\kw{bq}$}
      [rounded corners] -- (3,6,6)
      [sharp corners] -- (4,6,4)
      node[tsn] {$\Gamma_\kw{bq}$};

  \end{tikzpicture}
  }}
  \qquad
  \text{(b)} \quad
  \vcenter{\hbox{%
  \begin{tikzpicture}[sdp]

    %% Left and bottom faces

    % Background area
    \fill[tssdbg] (0,0,0) -- (0,6,0) -- (0,6,8)
               -- (0,0,8) -- (8,0,8) -- (8,0,0) -- cycle;
    \draw[thin,dotted] (0,0,0) -- (0,0,8);
    \fill[act] (0,0,0) -- (0,6,0)
      -- (0,6,2) -- (0,0,2)
      [rounded corners] -- (2,0,2)
      [sharp corners] -- (4,0,3.5)
      [rounded corners] -- (6,0,2)
      [sharp corners] -- (8,0,2)
      -- (8,0,0) -- cycle;

    % Strings
    \draw (0,6,2) -- (0,0,2)
      [rounded corners] -- (2,0,2)
      [sharp corners] -- (4,0,3.5)
      [rounded corners] -- (6,0,2)
      [sharp corners] -- (8,0,2); % node[right] {$\top$};
    \draw (0,6,6)
      -- (0,3,6)
      node[scn] {$R_\kw{rb}$}
      [rounded corners] -- (0,2,6)
      [sharp corners] -- (0,1,5);
    \draw (0,4,3.5)
      node[scn,bln] {}
      -- (0,3,3.5) \companion
      [rounded corners] -- (0,2,3.5)
      [sharp corners] -- (0,1,5) node[sct] {$\bullet$}
      -- (0,0,5) node[above right,inner sep=1pt] {\tiny $\kw{mem}$}
      [rounded corners] -- (2,0,5)
      [sharp corners] -- (4,0,3.5)
      node[tsn] {$\kw{rb.c}$}
      [rounded corners] -- (6,0,5.5)
      [sharp corners] -- (8,0,5.5);

    %% Center label

    \node%[draw,circle,inner sep=1pt]
       at (4,3,4) {$\phi_\kw{rb}$};

    %% Top and right

    % Background
    \fill[tssdbg,opacity=0.6]
      (0,6,0) -- (8,6,0) -- (8,0,0) -- (8,0,8) -- (8,6,8) -- (0,6,8) -- cycle;
    \draw[thin,dotted] (8,6,0) -- (8,6,8);
    \fill[act]
      (0,6,0) -- (0,6,2)
      [rounded corners] -- (3,6,2)
      [sharp corners] -- (4,6,4)
      -- (8,6,4) -- (8,3,4)
      [rounded corners] -- (8,2,2)
      [sharp corners] -- (8,0,2) -- (8,0,0) -- (8,6,0) -- cycle;

    % Strings and nodes
    \draw (0,6,2) node[above left,inner sep=1pt] {\footnotesize $\mathcal{C}$}
      [rounded corners] -- (3,6,2)
      [sharp corners] -- (4,6,4)
      -- (8,6,4) node[above] {\footnotesize $\top$}
      -- (8,3,4)
      [rounded corners] -- (8,2,2)
      [sharp corners] -- (8,0,2) node[right] {\footnotesize $\mathcal{C}$};
    \draw (8,3,4) node[scn] {$\varnothing$}
      [rounded corners] -- (8,2,5.5)
      [sharp corners] -- (8,0,5.5) node[right] {\footnotesize $\kw{mem}$};
    \draw (0,6,6) node[above left, inner sep=1pt] {\footnotesize $D_\kw{rb}$}
      [rounded corners] -- (3,6,6)
      [sharp corners] -- (4,6,4)
      node[tsn] {$\Gamma_\kw{rb}$};

  \end{tikzpicture}
  }}
\]
  \caption{
    Three-dimensional string diagrams
    for the correctness properties of $\kw{bq.c}$ and $\kw{rb.c}$.
  }
  \label{fig:3dsd}
\end{figure}
%}}}

%\paragraph{Categorical Structure} %{{{
%
%These constructions satisfy many properties
%which are well-understood in the context of category theory.
%For example, the properties
%\[
%  \epsilon_A \otimes \epsilon_B = \epsilon_{A \otimes B}
%  \qquad \text{and}
%  \qquad
%  (\mathbf{R}_1 \otimes \mathbf{R}_2) \cdot
%  (\mathbf{S}_1 \otimes \mathbf{S}_2) =
%  (\mathbf{R}_1 \cdot \mathbf{S}_1) \otimes
%  (\mathbf{R}_2 \cdot \mathbf{S}_2)
%  \,,
%\]
%and various properties of the invertible simulation conventions:
%\[
%  \lambda_A : A \otimes \mathbf{I} \cong A \,,
%  \qquad
%  \alpha_{ABC} : (A \otimes B) \otimes C \cong A \otimes (B \otimes C) \,,
%  \qquad
%  \gamma_{AB} : A \otimes B \cong B \otimes A \,,
%\]
%equip %the category
%$\mathbf{SC}$
%%of language interfaces and simulation conventions
%with the structure of a \emph{symmetric monoidal category}.
%Likewise, the properties
%\[
%  [{=}_U] = \epsilon_{[U]} \,,
%  \qquad
%  [R \cdot S] = [R] \vcomp [S] \,,
%  \qquad
%  [R \times S] = [R] \otimes [S]
%\]
%can be captured by describing
%$[-] : \mathbf{Rel} \rightarrow \mathbf{SC}$
%as a \emph{monoidal functor}
%from the symmetric monoidal category $\mathbf{Rel}$
%of sets and relations
%to the symmetric monoidal category $\mathbf{SC}$.
%
%%This categorical description %of the compositional structure
%%of simulation conventions
%%brings with it useful tools.
%%In essence,
%Symmetric monoidal categories capture
%the algebra of systems or processes which
%compose both in series and parallel
%\cite{rosetta}.
%In the case of simulation conventions,
%the process is one of concretization
%from a high-level, abstract representation
%of component interactions
%to a more concrete and low-level one.
%Series composition ($\cdot$)
%allows us to carry out this process in a stepwise manner,
%while parallel composition ($\otimes$)
%allows us to operate independently on various components
%of questions and answers.
%This intuition is backed by the formal language of string diagrams.
%
%%}}}
%
%\paragraph{String Diagrams} %{{{
%
%As implied by the properties above,
%a composite morphism in a symmetric monoidal category
%can often be written in a variety of equivalent ways.
%String diagrams provide a more economical representation,
%where these equivalences are captured
%by simple geometric intuition.
%For example, consider the following situation:
%\[
%  \begin{prooftree}
%    \hypo{
%      \begin{array}{c}
%	w : A \leftrightarrow B \\
%	x : \mathbf{I} \leftrightarrow C \\
%	y : C \leftrightarrow D \\
%	z : B \otimes D \leftrightarrow E
%      \end{array}
%    }
%    \infer1{\mathbf{R} : A \leftrightarrow E}
%  \end{prooftree}
%  \quad
%  \begin{array}{r@{}l}
%    \mathbf{R} := {} &
%    \lambda_A^{-1} \cdot
%    (A \otimes x) \cdot
%    (w \otimes y) \cdot
%    z 
%    \\[0.5ex]
%    = {} &
%    \lambda_A^{-1} \cdot
%    (w \otimes (x \cdot y)) \cdot z
%    \\[0.5ex]
%    = {} &
%    w \cdot \lambda_B^{-1} \cdot
%    (B \otimes (x \cdot y)) \cdot z
%    \\
%    \vdots \:\, &
%  \end{array}
%  \qquad
%  \mathbf{R} :=
%  \vcenter{\hbox{%
%    \begin{tikzpicture}[scale=0.4,inner sep=2pt,baseline=(w.base)]
%      \fill[scsdbg] (-1, 0) rectangle (3, 6);
%      \draw[rounded corners]
%            (0, 6) node[above] {$A$}
%         -- (0, 3) node[draw,fill=white,circle] (w) {$w$}
%         -- (0, 2) node[below left,inner sep=1pt] {$B$} -- (1, 1)
%         -- (1, 0) node[below] {$E$};
%      \draw[rounded corners]
%            (2, 5) node[draw,fill=white,circle] {$x$}
%         -- node[midway,right] {$C$} (2, 3) node[draw,fill=white,circle] {$y$}
%         -- (2, 2) node[below right,inner sep=0pt] {$D$} -- (1, 1) node[draw,fill=white,circle] {$z$};
%    \end{tikzpicture}
%  }}
%\]
%Here,
%we define a simulation convention $\mathbf{R}$ from various components
%using categorical operations.
%On the left,
%we show the type of every variable,
%and give several equivalent definitions for $\mathbf{R}$.
%The string diagram on the right
%captures the same information.
%Note that
%string diagrams
%are \emph{formal} diagrams
%which denote a particular morphism
%with the same rigor
%as traditional notation.
%
%The string diagrams we use to represent simulation conventions
%can be read from top to bottom.
%Vertical lines denote language interfaces,
%and horizontal juxtaposition represent tensor products.
%Since it is the unit for $\otimes$,
%the language interface $\mathbf{I}$ is not explicitly represented.
%Nodes connect a group of lines above to a group of lines below
%and denote elementary simulation conventions,
%and are connected vertically to denote sequential composition.
%Like the language interface $\mathbf{I}$,
%the identity simulation convention $\epsilon$ is omitted,
%and may appear as a vertical line without an intervening node.
%Based on these conventions,
%the string diagram above can be read as:
%\begin{align*}
%  \mathbf{R}
%     = \begin{tikzpicture}[scale=0.35,inner sep=2pt,baseline=(x.base)]
%         \fill[scsdbg] (-1,4) rectangle (3,6);
%         \draw (0,6) node[above] {$A$} -- (0,4) node[below] {$A$};
%         \draw (2,5) node[draw,circle,fill=white] (x) {$x$}
%           -- (2,4) node[below] {$C$};
%       \end{tikzpicture}
%       \:\cdot\:
%       \begin{tikzpicture}[scale=0.35,inner sep=2pt,baseline=(w.base)]
%         \fill[scsdbg] (-1,2) rectangle (3,4);
%         \draw (0, 4) node[above] {$A$}
%         -- (0, 3) node[draw,fill=white,circle] (w) {$w$}
%         -- (0, 2) node[below] {$B$};
%         \draw (2, 4) node[above] {$C$}
%         -- (2, 3) node[draw,fill=white,circle] (y) {$y$}
%         -- (2, 2) node[below] {$D$};
%       \end{tikzpicture}
%       \:\cdot\:
%       \begin{tikzpicture}[scale=0.35,inner sep=2pt,baseline=(z.base)]
%         \fill[scsdbg] (-1,0) rectangle (3,2);
%         \draw (2, 2) node[above] {$D$}
%           .. controls +(0,-0.5) and +(0.5,0.5) .. (1,1);
%         \draw (0, 2) node[above] {$B$}
%           .. controls +(0,-0.5) and +(-0.5,0.5) .. (1,1) node[draw,circle,fill=white] (z) {$z$}
%           -- (1,0);
%       \end{tikzpicture}
%    &= \lambda_A^{-1} \cdot
%       \left(
%         \begin{tikzpicture}[scale=0.35,inner sep=2pt,baseline=(x.base)]
%           \node (x) at (0,5) {$x$};
%           \fill[scsdbg] (-1,4) rectangle (1,6);
%           \draw (0,6) node[above] {$A$} -- (0,4) node[below] {$A$};
%         \end{tikzpicture}
%         \otimes
%         \begin{tikzpicture}[scale=0.35,inner sep=2pt,baseline=(x.base)]
%           \fill[scsdbg] (1,4) rectangle (3,6);
%           \draw (2,5) node[draw,circle,fill=white] (x) {$x$}
%             -- (2,4) node[below] {$C$};
%         \end{tikzpicture}
%       \right) \:\cdot\:
%       \left(
%         \begin{tikzpicture}[scale=0.35,inner sep=2pt,baseline=(w.base)]
%           \fill[scsdbg] (-1,2) rectangle (1,4);
%           \draw (0, 4) node[above] {$A$}
%           -- (0, 3) node[draw,fill=white,circle] (w) {$w$}
%           -- (0, 2) node[below] {$B$};
%         \end{tikzpicture}
%         \otimes
%         \begin{tikzpicture}[scale=0.35,inner sep=2pt,baseline=(y.base)]
%           \fill[scsdbg] (1,2) rectangle (3,4);
%           \draw (2, 4) node[above] {$C$}
%           -- (2, 3) node[draw,fill=white,circle] (y) {$y$}
%           -- (2, 2) node[below] {$D$};
%         \end{tikzpicture}
%       \right) \:\cdot\:
%       \begin{tikzpicture}[scale=0.35,inner sep=2pt,baseline=(z.base)]
%         \fill[scsdbg] (-1,0) rectangle (3,2);
%         \draw (2, 2) node[above] {$D$}
%           .. controls +(0,-0.5) and +(0.5,0.5) .. (1,1);
%         \draw (0, 2) node[above] {$B$}
%           .. controls +(0,-0.5) and +(-0.5,0.5) .. (1,1) node[draw,circle,fill=white] (z) {$z$}
%           -- (1,0);
%       \end{tikzpicture} \\
%    &= \lambda_A^{-1} \cdot (\epsilon_A \otimes x) \cdot (w \otimes y) \cdot z
%    \,.
%\end{align*}
%%There are other ways to decompose the diagram,
%%which yield some of the alternate formulas for $\mathbf{R}$
%%shown above.
%%But conversely,
%%the string diagrams representations of these formulas
%%are all identical,
%%up to topological deformations
%%which correspond to the axioms of monoidal categories.
%
%%}}}

%}}}



%}}}

\newpage

\begin{figure}[h] %{{{
  \textbf{Notations}
  \\[1em]
  \begin{tabular}{llcllc}
    Basic component & Def.~\ref{def:lts} &
    $L : A \twoheadrightarrow B$ &
    Stateful component & Def.~\ref{def:slts} &
    $\Sigma : A \rightarrow B$
    \\
    Basic convention & Def.~\ref{def:simconv} &
    $\mathbb{R} : A^\sharp \Leftrightarrow A^\flat$ &
    Stateful convention & Def.~\ref{def:sconv} &
    $\mathbf{R} : A^\sharp \leftrightarrow A^\flat$
    \\
    Basic simulation & Def.~\ref{def:sim} &
    $L^\sharp \le_{\mathbb{R}_A \twoheadrightarrow \mathbb{R}_B} L^\flat$ &
    Stateful simulation & Def.~\ref{def:ssim} &
    $\Sigma^\sharp \preceq_{\mathbf{R}_A \rightarrow \mathbf{R}_B} \Sigma^\flat$
  \end{tabular}
  \\[1em]
  \textbf{Layered composition}
  \\[1em]
  \begin{tabular}{cc@{\qquad}cc}
    Def.~\ref{def:lcomp} &
  {$\begin{prooftree}
      \hypo{L_1 : B \twoheadrightarrow C}
      \hypo{L_2 : A \twoheadrightarrow B}
      \infer2{L_1 \odot L_2 : A \twoheadrightarrow C}
    \end{prooftree}$}
    &
    Def.~\ref{def:slcomp} &
  {$\begin{prooftree}
      \hypo{\Sigma_1 : B \rightarrow C}
      \hypo{\Sigma_2 : A \rightarrow B}
      \infer2{\Sigma_1 \circ \Sigma_2 : A \rightarrow C}
    \end{prooftree}$}
    \vspace{1em} \\
    Thm.~\ref{thm:lcompsim} &
  {$\begin{prooftree}
      \hypo{L_1^\sharp
            \le_{\mathbb{R}_B \twoheadrightarrow \mathbb{R}_C}
            L_1^\flat}
      \hypo{L_2^\sharp
            \le_{\mathbb{R}_A \twoheadrightarrow \mathbb{R}_B}
            L_2^\flat}
      \infer2{L_1^\sharp \odot L_2^\sharp
            \le_{\mathbb{R}_A \twoheadrightarrow \mathbb{R}_C}
            L_1^\flat \odot L_1^\flat}
    \end{prooftree}$} &
    Thm.~\ref{thm:slcompsim} &
  {$\begin{prooftree}
      \hypo{\Sigma_1^\sharp
            \preceq_{\mathbf{R}_B \rightarrow \mathbf{R}_C}
            \Sigma_1^\flat}
      \hypo{\Sigma_2^\sharp
            \preceq_{\mathbf{R}_A \rightarrow \mathbf{R}_B}
            \Sigma_2^\flat}
      \infer2{\Sigma_1^\sharp \circ \Sigma_2^\sharp
            \preceq_{\mathbf{R}_A \rightarrow \mathbf{R}_C}
            \Sigma_1^\flat \circ \Sigma_1^\flat}
    \end{prooftree}$}
  \end{tabular}
  \\[1em]
  \textbf{Vertical composition}
  \\[1em]
  \begin{tabular}{cc@{\qquad}cc}
    Def.~\ref{def:ccomp} & {$
    \begin{prooftree}
      \hypo{\mathbb{R} : A^\sharp \Leftrightarrow A^\natural}
      \hypo{\mathbf{S} : A^\natural \Leftrightarrow A^\flat}
      \infer2{\mathbb{R} \cdot \mathbf{S} : A^\sharp \Leftrightarrow A^\flat}
    \end{prooftree}
    $} &
    Def.~\ref{def:sccomp} & {$
    \begin{prooftree}
      \hypo{\mathbf{R} : A^\sharp \leftrightarrow A^\natural}
      \hypo{\mathbf{S} : A^\natural \leftrightarrow A^\flat}
      \infer2{\mathbf{R} \vcomp \mathbf{S} : A^\sharp \leftrightarrow A^\flat}
    \end{prooftree}
    $}
    \vspace{1em} \\
    Thm.~\ref{thm:vcomp} & {$
    \begin{prooftree}
      \hypo{L^\sharp
        \le_{\mathbb{R}_A \twoheadrightarrow \mathbb{R}_B}
        L^\natural}
      \hypo{L^\natural
        \le_{\mathbf{S}_A \twoheadrightarrow \mathbf{S}_B}
        L^\flat}
      \infer2{L^\sharp
        \le_{\mathbb{R}_A \cdot \mathbf{S}_A \twoheadrightarrow
             \mathbb{R}_B \cdot \mathbf{S}_B}
        L^\flat}
    \end{prooftree}
    $} &
    Thm.~\ref{thm:svcomp} & {$
    \begin{prooftree}
      \hypo{\Sigma^\sharp
        \preceq_{\mathbf{R}_A \twoheadrightarrow \mathbf{R}_B}
        \Sigma^\natural}
      \hypo{\Sigma^\natural
        \preceq_{\mathbf{S}_A \twoheadrightarrow \mathbf{S}_B}
        \Sigma^\flat}
      \infer2{\Sigma^\sharp
        \preceq_{\mathbf{R}_A \vcomp \mathbf{S}_A \twoheadrightarrow
             \mathbf{R}_B \vcomp \mathbf{S}_B}
        \Sigma^\flat}
    \end{prooftree}
    $}
  \end{tabular}
  \\[1em]
  \textbf{Adjoining explicit state}
  \\[1em]
  \begin{tabular}{cc@{\qquad}cc}
    Def.~\ref{def:lift} &
    {$
    \begin{prooftree}
      \hypo{L : A \twoheadrightarrow B}
      \infer1{L@K : A@K \twoheadrightarrow B@K}
    \end{prooftree}
    $} &
    Def.~\ref{def:slift} &
    {$
    \begin{prooftree}
      \hypo{\Sigma : A \rightarrow B}
      \infer1{\Sigma@K : A@K \rightarrow B@K}
    \end{prooftree}
    $}
    \vspace{1em} \\
    & &
    Def.~\ref{def:liftsconv} &
    {$
    \begin{prooftree}
      \hypo{\mathbf{R} : A^\sharp \leftrightarrow A^\flat}
      \infer1{\mathbf{R}@\langle K^\sharp, K^\flat \rangle :
        A^\sharp@K^\sharp \leftrightarrow A^\flat@K^\flat}
    \end{prooftree}
    $}
    \vspace{1em} \\
    & &
    Thm.~\ref{thm:liftssim} &
    {$
    \begin{prooftree}
      \hypo{\Sigma^\sharp
        \preceq_{\mathbf{R}_A \rightarrow \mathbf{R}_B}
        \Sigma^\flat}
      \infer1{\Sigma^\sharp@K^\sharp
        \preceq_{\mathbf{R}_A@\langle K^\sharp, K^\flat \rangle \rightarrow
                 \mathbf{R}_B@\langle K^\sharp, K^\flat \rangle}
        \Sigma^\flat@K^\flat}
    \end{prooftree}
    $}
  \end{tabular}
  \\[1em]
  \textbf{Embedding simple components}
  \[
    \begin{prooftree}
      \hypo{L : A \twoheadrightarrow B}
      \infer1{\&L : A \rightarrow B}
    \end{prooftree}
    \qquad
    \begin{prooftree}
      \hypo{\mathbb{R} : A^\sharp \Leftrightarrow A^\flat}
      \infer1{\&\mathbb{R} : A^\sharp \leftrightarrow A^\flat}
    \end{prooftree}
    \qquad
    \begin{array}{c}
      \&(L_1 \odot L_2) \equiv \&L_1 \circ \&L_2
      \\[1ex]
      \&(\mathbb{R} \cdot \mathbf{S}) \equiv
        \&\mathbb{R} \vcomp \&\mathbf{S}
    \end{array}
    \qquad
    \begin{prooftree}
      \hypo{L^\sharp
        \le_{\mathbb{R}_A \twoheadrightarrow \mathbb{R}_B}
        L^\flat}
      \infer1{\&L^\sharp
        \preceq_{\&\mathbb{R}_A \rightarrow \&\mathbb{R}_B}
        \&L^\flat}
    \end{prooftree}
  \]
  \\[1em]
  \textbf{Encapsulating state}
  \[
    \begin{prooftree}
      \hypo{\Sigma : A \rightarrow B@K}
      \infer1{\kw{fbk}_K(\Sigma) : A \rightarrow B}
    \end{prooftree}
    \qquad
    \begin{prooftree}
      \hypo{\Sigma^\sharp
        \preceq_{\mathbf{R}_A \rightarrow
                 \mathbf{R}_B@\langle K^\sharp,K^\flat \rangle}
        \Sigma^\flat}
      \infer1{\kw{fbk}_{K^\sharp}(\Sigma^\sharp)
        \preceq_{\mathbf{R}_A \rightarrow \mathbf{R}_B}
        \kw{fbk}_{K^\flat}(\Sigma^\flat)}
    \end{prooftree}
    \qquad
    \kw{fbk}_\mathbbm{1}(\Sigma) \equiv \Sigma
  \]
  \vspace{1ex}
  \[
    \kw{fbk}_{K_1}(\Sigma_1) \circ \kw{fbk}_{K_2}(\Sigma_2) \equiv
    \kw{fbk}_{K_1 \times K_2}(\Sigma_1@K_2 \circ \Sigma_2)
  \]
  \caption{Summary of key notations, definitions and properties}
    %Constructions on the left-hand side operate in terms of
    %the original semantic framework of CompCertO.
    %We extend that framework
    %to account for persistent encapsulated state,
    %shown on the right.
    %Construction which enable the manipulation of
    %encapsulated state are shown at the bottom.}
  \label{fig:overview}
\end{figure}
%}}}

\tableofcontents

\section*{New material} %{{{

\subsection*{Protected explicit state}

The Kripke relation
$\Lambda_U \in \mathcal{R}_V(\mathbbm{1}, U)$
is defined by the rule:
\[
  \begin{prooftree}
    \infer0{u \Vdash * \ifr{\Lambda_U} u}
  \end{prooftree}
\]

\begin{definition}
For a set $U$,
the simulation convention $\caller{U} : I \Leftrightarrow U$
is defined as:
\[
  \caller{U} := \big\langle U,
      \Lambda_U,
      \Lambda_U
    \big\rangle
\]
\end{definition}

\begin{definition}
For a pointed set $U$,
the stateful simulation convention
$\callee{U} : I \leftrightarrow U$
is defined as
\[
  \callee U \: := \: \big\langle
      U, \,
      \Lambda_U, \,
      \Lambda_U, \,
      {=}_U, \,
      \top_U
    \big\rangle
\]
\end{definition}

\[
  \begin{tikzcd}[sep=large]
    \mathcal{C}\otimes\kw{mem}
      \ar[r, "\mathsf{ClightP}(M)"]
      \ar[d, equals] &
    \mathcal{C}\otimes\kw{mem}
      \ar[r, equals]
      \ar[d, leftrightarrow, "\mathcal{C} \otimes \kw{mem} \otimes \callee{p_0}"] &
    \mathcal{C}\otimes\kw{mem}
      \ar[dd, leftrightarrow, "\mathcal{C} \otimes \kw{mem} \otimes \callee{m_0}"]
    \\
    \mathcal{C} \otimes \kw{mem}
      \ar[r, "\mathsf{ClightP} \langle M \rangle"]
      \ar[d, leftrightarrow, "\mathsf{C} \otimes \kw{mem} \otimes \caller{\kw{mem}}"'] &
    \mathcal{C} \otimes \kw{mem} \otimes \kw{penv}
      \ar[d, leftrightarrow, "\mathcal{C} \otimes \kw{mem} \otimes R"]
    \\
    \mathcal{C} \otimes \kw{mem} \otimes \kw{mem}
      \ar[d, leftrightarrow, "\mathcal{C} \otimes {\bullet}"'] &
    \mathcal{C} \otimes \kw{mem} \otimes \kw{mem}
      \ar[d, leftrightarrow, "\mathcal{C} \otimes {\bullet}"]
      \ar[r, equals] &
    \mathcal{C} \otimes \kw{mem} \otimes \kw{mem}
      \ar[d, leftrightarrow, "\mathcal{C} \otimes {\bullet}"]
    \\
    \mathcal{C} \otimes \kw{mem}
      \ar[r, "\mathsf{Clight}(M)"] &
    \mathcal{C} \otimes \kw{mem}
      \ar[r, equals] &
    \mathcal{C} \otimes \kw{mem}
  \end{tikzcd}
\]

%}}}

\section{Certified Abstraction Layers} \label{sec:cal} %{{{

This section will be dropped.

{
\color{gray}
A cleaner version of our OOPSLA story.
Here we must go from:
\begin{itemize}
  \item A fully abstract version where the layer interface
    has encapsulated abstract state,
    but does not change the memory at all
  \item A version where this is realized by an encapsulated
    memory component,
    which is added when the layer is invoked,
    and re-separated when it returns control to the client
    (refinement can act on that individual memory fragment).
  \item The concrete implementation version
    where the state is part of the global memory
    (refinement shown via
    simulation up to ${-} \bullet m \equiv {-}$).
\end{itemize}
}

We have shown in \ref{sec:base:abrel} that
abstraction relations are unwieldy,
especially when they are promoted to simulation conventions.

In general, the abstraction relations have the form
$R \subseteq K^\sharp \times (\kw{mem} \times K^\flat)$
so that the abstraction layers gradually refine
the concrete memory values and low-level abstract states
into high-level abstract states.
The abstraction relations are then promoted to simulation conventions
$\hat{R}: \mathcal{C}@(\kw{mem}\times K^\sharp)
\Leftrightarrow \mathcal{C}@(\kw{mem}\times K^\flat)$.
However, abstraction relations are not compatible with
vertical composition.
In other words, the following property does not hold
\[
   \hat{R \circ S} \sqsubseteq \hat{R}; \hat{S}
\]

The reason is that the abstraction relations
are playing two roles at the same time.
One is to refine the memory values to the abstract representations,
and the other is to embed the memory fragment
into the entire unified memory model.
Therefore, we seek to decouple the two tasks.
The $\ClightP$ language tackles the second task
and provides a more tractable $\kw{penv}$ interface
than the monolithic memory.
This leaves us the first task to solve.
With the help of state encapsulation,
the first task can be solved in a clean and elegant manner
as we will present.

\subsection{Layer Interfaces} %{{{

A layer interface with abstract states in $D$
can be defined using a transition system:
\[
  L : \mathbf{1} \twoheadrightarrow \mathcal{C}@D
\]
To interface with the client code,
we can hide the abstract state and lift the component to:
\[
  \Sigma := \kw{fbk}_D(\&L)@\kw{mem} : \mathbf{1} \rightarrow \mathcal{C}@\kw{mem}
\]
For example, we can hide the abstract state
from bounded queue and ring buffer interface in the example \ref{ex:rbspec}.
Note that the client may not modify their abstract states,
and may even not be aware of the existence of such states.
\[
  \Sigma_\kw{bq} := \kw{fbk}(\&L_\kw{bq}): \mathbf{1} \rightarrow \mathcal{C}@\kw{mem} \qquad
  \Sigma_\kw{rb} := \kw{fbk}(\&L_\kw{rb}): \mathbf{1} \rightarrow \mathcal{C}@\kw{mem}
\]

%}}}

\subsection{Layer Implementation}
\label{sec:cal:impl}

Given two transition systems manipulating states
at different abstraction levels
$L^\sharp: \mathbf{1} \twoheadrightarrow A@K^\sharp$
and
$L^\flat: \mathbf{1} \twoheadrightarrow A@K^\flat$,
the simulation between them is witnessed
by an abstraction relation $R \subseteq K^\sharp \times K^\flat$
such that
\[
  L^\sharp \le_{\kw{id} \twoheadrightarrow A@R} L^\flat
\]

Once the states are encapsulated,
the signatures of the two transition systems are identified.
As a consequence, the abstraction relation is concealed accordingly.
\[
  \kw{fbk}_{K^\sharp}(\& L^\sharp) \preceq \kw{fbk}_{K^\flat}(\& L^\flat)
\]
The secret is the simulation invariant.

The benefits of doing so:
\begin{itemize}
\item The self-simulation property for the client is no longer necessary.
  The client is ignorant of the representations.
  Decoupled the process of transforming the abstract state
  and assembling them into the memory.
  Again the secret is the simulation invariant.
\item The issues with composition of abstraction relations are solved
\end{itemize}

For the layer correctness,
we exploit the $\ClightP$ semantics as the implementation.
Then the correctness can be formulated as
\[
  \Sigma^\flat \vdash M : \Sigma^\sharp
  \Leftrightarrow
  \Sigma^\sharp \preceq \ClightP(M) \circ \Sigma^\flat
\]
The abstraction relation
$R \subseteq K^\sharp \times (\kw{penv} \times K^\flat)$
has once again been concealed.
Consequently, the vertical composition of abstraction layers
can be proved
by the monotonicity and associativity of layered composition
in a straightforward manner.
\[
  \begin{prooftree}
    \hypo{\Sigma^\flat \vdash M : \Sigma^\natural}
    \hypo{\Sigma^\natural \vdash N : \Sigma^\sharp}
    \infer2{\Sigma^\flat \vdash M, N : \Sigma^\sharp}
  \end{prooftree}
\]

Back to the bounded queue and ring buffer example,
we can prove the followings in the new framework
\[
  \Sigma_\kw{rb} \vdash M_\kw{bq} : \Sigma_\kw{bq}
  \qquad
  \varnothing \vdash M_\kw{rb} : \Sigma_\kw{rb}
\]
and then compose them together
\[
  \varnothing \vdash M_\kw{rb}, M_\kw{bq} : \Sigma_\kw{bq}
\]

{
\color{gray}
\subsection{Layer Implementation} %{{{

The correctness property $L^\flat \vdash M : L^\sharp$
must be established as a simulation of the form:
\[
  \kw{fbk}(\&L^\sharp)@\kw{mem}
  \le_\mathbb{R}
  \&\Clight(M) \circ \kw{fbk}(\&L^\flat)@\kw{mem}
  :
  \mathbf{1} \rightarrow \mathcal{C}@\kw{mem}
\]
Here the simulation convention $\mathbb{R}$
must exclude from the source memory
the region used in the target memory
to store the persistent state and stack frames used by $M$.
It must also ensure that
this region remains unchanged in the target memory
between successive activations of $M$.
However,
the exact representation used
to represent the hidden abstract state of $L^\sharp$
is itself hidden within the simulation.

\paragraph{Layer Correctness}

To prove a particular layer implementation correct,
we first focus on the way $M$ acts on its private fragment.
We give an abstraction relation
$R \subseteq D^\sharp \times (D^\flat \times \kw{mem})$
such that:
\begin{equation}
  L^\sharp
  \:\le_{\mathbf{1} \rightarrow \kw{id}@R}\:
  \Clight(M)@D^\flat \circ L^\flat@\kw{mem}
  \qquad \text{and} \qquad
  \intl{d}^\sharp \mathrel{R} \big( \intl{d}^{\,\flat}, \intl{m} \big)
  \,.
  \label{eqn:lc}
\end{equation}
Here $\intl{m}$ is the initial memory fragment for the module $M$,
derived from the definitions within $M$ itself.
Note that we can carry out this proof without regard for the context memory.
There are no particular conditions on $R$ other than
initial state being related.

\paragraph{Adding Context Memory}

By hiding internal state,
\autoref{eqn:lc} can be used to establish:
\begin{align*}
  \kw{fbk}_{D^\sharp}(\&L^\sharp) \le {} &
  \kw{fbk}_{D^\flat \times \kw{mem}} \big(
    \&(\Clight(M)@D^\flat \circ L^\flat@\kw{mem})
    \big) \\ \equiv {} &
  \kw{fbk}_\kw{mem}(\&\Clight(M)) \circ \kw{fbk}_{D^\flat}(\&L^\flat)
  : \mathbf{1} \rightarrow \mathcal{C}
  \,,
\end{align*}
however this does not take into account the context memory,
or the way in which the context memory and the memory used by $M$
are merged into the global memory
at the implementation level.
To achieve this we must use our memory separation primitive
and the frame rule for $\Clight$.
}

%Let me think about that but two things that come to mind:
%The first one is, for linking to work, you also need to do that for internal calls since the call from f to g will eventually become an internal call in [F + G] which will have to be matched with the cross-component interaction in [F] ⊕ [G].
%The second one is, think about the layer implementation case. We know that L : C@K ↠ C@K is refined by [[M]] : C@mem ↠ C@mem which operates in terms of a memory fragment that only contains the globals that implement abstract state K, and whatever stack blocks [[M]] allocates.
%Now the state for these transition systems is hidden so that we actually have a direct simulation between fbk(&L) : C → C and fbk(&L') : C → C. Both can then be lifted to fbk(&L)@mem, fbk(&L')@mem : C@mem → C@mem to be interfaced with context code. But note that in the execution of fbk(&L')@mem case there are now two different memory states involved: the context one which is left unchanged, and the 

%}}}

\subsection{Horizontal composition} %{{{

We first define the product.
\begin{definition}[Product] \label{def:prod}
  Given transition systems
\[
  L_1 = \langle S_1, {\rightarrow_1}, I_1, X_1, Y_1, T_1 \rangle
    : A \twoheadrightarrow B@K_1
  \quad \text{and} \quad
  L_2 = \langle S_2, {\rightarrow_2}, I_2, X_2, Y_2, T_2 \rangle
    : A \twoheadrightarrow B@K_2
\]
  we define
  $L_1 \cupdot L_2: A \twoheadrightarrow B@(K_1\times K_2)$
  as follows.
  \[
    S := (S_1 \times K_2) + (S_2 \times K_1)
  \]
  \[
    \begin{prooftree}
      \hypo{q@k_1 \mathrel{I_1} s_1}
      \infer1{q@(k_1, k_2) \mathrel{I} \iota_1(s_1@k_2)}
    \end{prooftree}
    \qquad
    \begin{prooftree}
      \hypo{s_1 \rightarrow_1 s'_1}
      \infer1{\iota_1(s_1@k_2) \rightarrow \iota_1(s'_1@k_2)}
    \end{prooftree}
    \qquad
    \begin{prooftree}
      \hypo{s_1 \mathrel{X_1} m}
      \infer1{\iota_1(s_1@k_2) \mathrel{X} m}
    \end{prooftree}
  \]
  \[
    \begin{prooftree}
      \hypo{n \mathrel{Y_1}^{s_1} s'_1}
      \infer1{n \mathrel{Y}^{s_1@k_2} \iota_1(s'_1@k_2)}
    \end{prooftree}
    \qquad
    \begin{prooftree}
      \hypo{s_1 \mathrel{F_1} r@k_1}
      \infer1{\iota_1(s_1@k_2) \mathrel{F} r@(k_1,k_2)}
    \end{prooftree}
  \]
  the symmetric cases are elided
\end{definition}

Then apply it to the components with encapsulated states.
\begin{definition}[Product] \label{def:sprod}
  Given two stateful components
$\Sigma_1 = (K_1 \mid L_1) : A \rightarrow B$ and
$\Sigma_2 = (K_2 \mid L_2) : A \rightarrow B$,
we define their composition
$\Sigma_1 \cup \Sigma_2 : A \rightarrow B$
in the following way:
\[
  \Sigma_1 \cup \Sigma_2 :=
    ( K_1 \times K_2 \mid L_1 \cupdot L_2 )
\]
\end{definition}

The horizontal composition can be proved
by the monotonicity and interchangeability.
\[
  \begin{prooftree}
    \hypo{\Sigma_1^\flat \vdash M : \Sigma_1^\sharp}
    \hypo{\Sigma_2^\flat \vdash N : \Sigma_2^\sharp}
    \infer2{\Sigma_1^\flat \cup \Sigma_2^\flat
      \vdash M, N : \Sigma_1^\sharp \cup \Sigma_2^\sharp}
  \end{prooftree}
\]

For example, this indicates that we can further
decompose the ring buffer layer into three layers,
and verify them independently.

%}}}

\subsection{Upcalls} %{{{

%}}}

%}}}

\section{Cut material to keep for now} %{{{

\begin{remark}[to incorporate or cut]
In particular,
in the absence of demonic nondeterminism,
CompCert's notion of \emph{forward simulation} is appropriate:
given two transition systems $L_1$ and $L_2$,
it suffices to exhibit a relation between their possible states
such that:
\begin{itemize}
  \item initial state of $L_1$ have related initial states in $L_2$;
  \item state transitions in $L_1$ have corresponding sequences of transitions
    from related states in $L_2$;
  \item related state which produce a final outcome in $L_1$
    have a corresponding final outcome in $L_2$.
\end{itemize}
To take into account event traces,
the simulation works under the assumption that
$L_1$ and $L_2$ are fed the same inputs by the environement,
and requires that they produce identical outputs.
The existence of a simulation relation satisfying these properties
shows that the behavior of $L_1$ is refined by that of $L_2$;
we say that $L_1$ is simulated by $L_2$ and write $L_1 \le L_2$.
\end{remark}

\begin{definition} [Simulation Convention Refinement] \label{def:scref}
  Given the stateful simulation conventions
  $\mathbf{R} : A^\sharp \leftrightarrow A^\flat$ and
  $\mathbf{S} : A^\sharp \leftrightarrow A^\flat$,
  the refinement between $\mathbf{R}$ and $\mathbf{S}$ is defined as:
  \[
    \mathbf{R} \sqsubseteq \mathbf{S} :\Leftrightarrow
    \mathbf{1} \preceq_{\mathbf{S} \twoheadrightarrow \mathbf{R}} \mathbf{1}
  \]
We write $\mathbf{R} \equiv \mathbf{S}$ when
$\mathbf{R} \sqsubseteq \mathbf{S}$ and
$\mathbf{S} \sqsubseteq \mathbf{R}$.
\end{definition}

The notion $\mathbf{1}$ represents the identify transition system.
Essentially, the refinement corresponds to the following indefinite condition:
\begin{align*}
  \forall w^R_1\ q^\sharp_1\ q^\flat_1.\ \intl{w^R} \mapsto w^R_1
  \Vdash q^\sharp_1 \mathrel{\mathbf{R}^\que} q^\flat_1 &\rightarrow
  \exists w^S_1.\ \intl{w^S} \mapsto w^S_1 
  \Vdash q^\sharp_1 \mathrel{\mathbf{S}^\que} q^\flat_1 \wedge\\
  \forall w^S_2\ r^\sharp_1\ r^\flat_1.\ w^S_1 \leadsto w^S_2
  \Vdash r^\sharp_1 \mathrel{\mathbf{S}^\ans} r^\flat_1 &\rightarrow
  \exists w^R_2.\ w^R_1 \leadsto w^R_2
  \Vdash r^\sharp_1 \mathrel{\mathbf{R}^\ans} r^\flat_1 \wedge\\
  \forall w^R_3\ q^\sharp_2\ q^\flat_2.\ w^R_2 \mapsto w^R_3
  \Vdash q^\sharp_2 \mathrel{\mathbf{R}^\que} q^\flat_2 &\rightarrow
  \exists w^S_3.\ w^S_2 \mapsto w^S_3
  \Vdash q^\sharp_3 \mathrel{\mathbf{S}^\que} q^\flat_3 \wedge\\
  \forall w^S_4\ r^\sharp_2\ r^\flat_2.\ w^S_3 \mapsto w^S_4
  \Vdash r^\sharp_2 \mathrel{\mathbf{S}^\ans} r^\flat_2 &\rightarrow
  \exists w^R_4.\ w^R_3 \mapsto w^R_4
  \Vdash r^\sharp_2 \mathrel{\mathbf{R}^\ans} r^\flat_2 \wedge\\[-1ex]
  &\:\:\vdots
\end{align*}

Similar to the refinement of the stateless simulation conventions,
a stateful simulation convention $\mathbf{S}$
is considered more general than $\mathbf{R}$
if the refinement $\mathbf{R} \sqsubseteq \mathbf{S}$ holds.
In particular, questions related by any worlds of $\mathbf{R}$
are also related under some worlds of $\mathbf{S}$;
when response is returned,
answers related at any successive worlds of $\mathbf{S}$
are also related under some successive worlds of $\mathbf{R}$.
However, because the questions and replies related
by a stateful simulation convention
are subject to the transition of its world,
the refinement unfolds indefinitely as the world evolves.

In general, in order to prove the stateful simulation between components
one has to design the simulation relation and invariant.
However, for proving simulation convention refinement,
there is not much to say about the states in the identity transition system.
So the simulation invariant is the key ingredient to prove such properties.


\begin{theorem}[Sequential rule of simulation convention] \label{thm:scseq}
  \[
    \begin{prooftree}
      \hypo{\mathbf{R'} \sqsubseteq \mathbf{R}}
      \hypo{L_1 \preceq_{\mathbf{R} \twoheadrightarrow \mathbf{S}} L_2}
      \hypo{\mathbf{S} \sqsubseteq \mathbf{S'}}
      \infer3{L_1 \preceq_{\mathbf{R'} \twoheadrightarrow \mathbf{S'}} L_2}
    \end{prooftree}
  \]
\end{theorem}

\section{ClightP}
\label{sec:clightp-1}


{
\color{gray}
[Note: this could be swapped with Section 3
because it showcases the use of persistent state,
but probably does not have much to gain
from the memory separation framework.]

Can we add a $\mathsf{private}$ keyword or storage class to Clight,
formulate a semantics and show a correctness proof
for erasure of the keyword?

Main challenge: how to define the semantics of the new keyword
in a way that's convenient.

We need a good name for this language.
For now I will use \ClightP{}.
}

Unlike the temporaries,
the module-local variables can also have type array or struct.
Therefore, we extend the type $\kw{val}$ with composite types to $\kw{cval}$
and define the $\kw{penv}$ as a map from identifiers to $\kw{cval}$.
Lifting the variables from memory to a separate environment
means that their address cannot be taken.
So we introduce accessors $\kw{lcval}$ to simulate left values,
which represent memory locations in $\Clight$,
so that the module-static variables can be evaluated to left values similarly
and updated accordingly.
Under this approach, struct assignment is not supported
because it is only possible to assign by value.
However, this enforces the program not to pass the address
of the private variables to other modules
so that undesired behaviors can be avoided.

\begin{gather*}
  cv \in \kw{cval} \mathrel{::=} \kw{Val}(v:\kw{val})
                     \mathrel{|} \kw{Arr}(sz:\mathbb{N},a:\mathbb{Z} \rightarrow \kw{cval})
  \\
  \kw{penv} \mathrel{::=} \kw{ident} \rightarrow \kw{cval}
  \\
  l \in \kw{lcval} \mathrel{::=} \kw{Lval}(i: \kw{ident})
                          \mathrel{|} \kw{Lloc}(l:\kw{lcval}, x:\mathbb{Z})
\end{gather*}

We reuse the $\Clight$ expressions,
and add the following expressions to access the private variables.
Similarly, we reuse the statements and small-step transitions
and add extra cases for updating the private states.
\begin{gather*}
  e \mathrel{::=} \cdots \mathrel{|} \kw{Epvar}(i:\kw{ident})
  \mathrel{|} \kw{Eaccess}(e:\kw{expr}, x: \kw{expr})
  \\[2ex]
  \kw{pread} \mathrel{:} \kw{penv} \rightarrow \kw{lcval} \rightarrow \kw{option}\ \kw{val}\\
  \kw{pwrite} \mathrel{:} \kw{penv} \rightarrow \kw{lcval} \rightarrow \kw{val} \rightarrow \kw{option}\ \kw{penv}
  \\[2ex]
  {\begin{prooftree}
    \hypo{\kw{pe}[i] = \lfloor \kw{Val}(v) \rfloor }
    \infer1{\kw{m},\kw{pe} \vdash \kw{Epvar}(i) \downarrow v}
  \end{prooftree}}
  \qquad
  {\begin{prooftree}
    \hypo{\kw{m}, \kw{pe} \vdash e \Downarrow loc}
    \hypo{\kw{m},\kw{pe} \vdash x \downarrow \kw{Vint}(i)}
    \hypo{\kw{pread}(\kw{pe}, \kw{Lloc}(loc, i)) = \lfloor \kw{Val}(v) \rfloor }
    \infer3{\kw{m},\kw{pe} \vdash \kw{Eaccess}(e, x) \downarrow v}
  \end{prooftree}}
  \\[2ex]
  {\begin{prooftree}
    \infer0{\kw{m},\kw{pe} \vdash \kw{Epvar}(i) \Downarrow \kw{Lval}(i)}
  \end{prooftree}}
  \qquad
  {\begin{prooftree}
    \hypo{\kw{m},\kw{pe} \vdash e \Downarrow loc}
    \hypo{\kw{m},\kw{pe} \vdash x \downarrow \kw{Vint}(i)}
    \infer2{\kw{m}, \kw{pe} \vdash \kw{Eaccess}(e, x) \Downarrow \kw{Lloc}(loc, i)}
  \end{prooftree}}
  \\[2ex]
  {\begin{prooftree}
    \hypo{\kw{m}, \kw{pe} \vdash a_1 \Downarrow loc}
    \hypo{\kw{m}, \kw{pe} \vdash a_2 \downarrow v}
    \hypo{\kw{pwrite}(pe, loc, v) = \lfloor pe' \rfloor }
    \infer3{(\kw{m}, \kw{pe}, \kw{Sassign}(a_1, a_2)) \rightarrow
      (\kw{m}, \kw{pe'}, \kw{Sskip})}
  \end{prooftree}}
\end{gather*}

A \ClightP{} program can be compiled to Clight
by erasing the \texttt{private} annotations
and turning privates variable into regular
global variables.
{
\color{gray}
Proving the correctness of this transformation
should not be too difficult.
We can just use a memory extension or injection.
The only new part is that we must express
the simulation convention for the underlying transition systems
in a way that relates the source private environment
to the target (public) memory state.
The twist here is that
the externally observable simulation convention
should just enforce the empty permissions in the source memory.
The relation between the private state and the target memory
should be existentially quantified.
But this means we need requirements on the initial target memory as well.
We will have to set up our extended notion of simulation
in a way that supports those things.
}

$\ClightP$ expressions are turned into $\Clight$ expressions
by replacing accesses to the private variables
with accesses to the corresponding memory locations.
\begin{gather*}
  \kw{transl\_expr}(\kw{Epvar}(i)) = \kw{Evar}(i)\\
  \kw{transl\_expr}(\kw{Eaccess}(e, \iota_2(i))) = \kw{Oadd}(\kw{transl\_expr(e)}, \kw{transl\_expr(i)})
\end{gather*}
The semantics of $\Clight$ is typed-directed
so the offset calculated by $\kw{Oadd}$
depends on the type of the array elements.
After the expressions are transated, the statements
are immediately valid $\Clight$ statements.

To establish the simulation between
the source $\ClightP$ program and the target $\Clight$ program,
we essentially transform the persistent environment into memory fragment,
and merge the fragment with the regular memory state.
We define a relation $\kw{pe} \rhd \kw{m}$
to denote that the the persistent environment $\kw{pe}$
can be concretized to the memory $\kw{m}$ under the global symbol table $\kw{se}$.
\[
  \begin{prooftree}
    \hypo{\forall i \mapsto cv \in \kw{pe}, \exists i \mapsto b \in se,
      (b,0) \leadsto_{\kw{m}} cv}
    \infer1{\kw{pe} \rhd \kw{m}}
  \end{prooftree}
\]
We define $(b, o) \leadsto_{\kw{m}} cv$ as follows:
\begin{gather*}
  {
  \begin{prooftree}
    \hypo{\kw{load}(\kw{m}, b, o) = \lfloor v \rfloor }
    \infer1{(b, o) \leadsto_{\kw{m}} \kw{Val}(v)}
  \end{prooftree}
  }
  \quad
  {
  \begin{prooftree}
    \hypo{\forall i, (b, o+\kw{offset}(a, i)) \leadsto_{\kw{m}} a[i]}
    \infer1{(b, o) \leadsto_{\kw{m}} \kw{Arr}(sz, a)}
  \end{prooftree}
  }
\end{gather*}
The auxiliary function $\kw{offset}$
calculates the offset based on the type information
which is elided for readability.

The other part of memory should remain the same.
We exploit the join operator defined in \ref{app:sep}.

%}}}

\section{Stashed examples}

\begin{example}[Layer specifications] \label{ex:rbspec} %{{{
We can formulate a specification for
the program component $\kw{rb.c}$ as follows.
The state of the ring buffer
is expressed as a tuple
$(f, c_1, c_2) \in S_\kw{rb} := \kw{val}^N \times \mathbb{N} \times \mathbb{N}$.
Operations do not otherwise access the memory,
so the specification will be of type
\[
  L_\kw{rb} : \mathbf{1} \twoheadrightarrow \mathcal{C}@S_\kw{rb}
  \,.
\]
To define it, we construct a simple transition system such that
all executions take the shape
\[
  q@(f, c_1, c_2) \:\mathrel{I}\: (v', f', c_1', c_2')
                  \:\mathrel{F}\: v'@(f', c_1', c_2')
  \,.
\]
The predicates $X$, $Y$ and $\rightarrow$ are empty.
As suggested above, $F$ is in essence the identity relation.
This leaves us to define $I$ which specifies the component's
actual behavior:
\[ \begin{array}{c@{\qquad}c}
 {\begin{prooftree}
    \hypo{i < N}
    \infer1{
      \kw{set}(i, v)@(f, c_1, c_2)
      \mathrel{I_\kw{rb}}
      (\kw{undef}, f[i := v], c_1, c_2)}
  \end{prooftree}}
  &
 {\begin{prooftree}
    \hypo{c_1' = (c_1 + 1) \mathbin{\mathrm{mod}} N}
    \infer1{
      \kw{inc1}@(f, c_1, c_2)
      \mathrel{I_\kw{rb}}
      (c_1, f, c_1', c_2)}
  \end{prooftree}}
  \vspace{1em}
  \\
 {\begin{prooftree}
    \hypo{i < N}
    \infer1{
      \kw{get}(i)@(f, c_1, c_2)
      \mathrel{I_\kw{rb}}
      (f_i, f, c_1, c_2)
    }
  \end{prooftree}}
  &
 {\begin{prooftree}
    \hypo{c_2' = (c_2 + 1) \mathbin{\mathrm{mod}} N}
    \infer1{
      \kw{inc1}@(f, c_1, c_2)
      \mathrel{I_\kw{rb}}
      (c_2, f, c_1, c_2')}
  \end{prooftree}}
\end{array} \]
We can then define
$L_\kw{rb} := \langle
  S_\kw{rb},\:
  \varnothing,\:
  I_\kw{rb},\:
  \varnothing,\:
  \varnothing,\:
  {=}
 \rangle$.

A similar approach can be use to define
$L_\kw{bq} : \mathbf{1} \twoheadrightarrow \mathcal{C}@S_\kw{bq}$,
where the states in $S_\kw{bq} := \kw{val}^*$
are simply lists enumerating the contents of the queue.
Here the operations will be specified as follows:
\[
  \begin{prooftree}
    \hypo{|\vec{q}| < N}
    \infer1{\kw{enq}(v)@\vec{q} \:\mathrel{I_\kw{bq}}\: (\kw{undef}, \vec{q}v)}
  \end{prooftree}
  \qquad\qquad
  \begin{prooftree}
    \hypo{\vec{q} = v\vec{p}}
    \infer1{\kw{deq}(\epsilon)@\vec{q} \:\mathrel{I_\kw{bq}}\: (v, \vec{p})}
  \end{prooftree}
\]
Again we can define $L_\kw{bq} := \langle
  S_\kw{bq},\:
  \varnothing,\:
  I_\kw{bq},\:
  \varnothing,\:
  \varnothing,\:
  {=}
\rangle$.
\end{example}
%}}}

\begin{example}[Interfacing $L_\kw{rb}$ with client code] \label{ex:context} %{{{
Building on Example~\ref{ex:rbspec},
consider the problem of interfacing
the client code in $\kw{bq.c}$ with the underlay interface $L_\kw{rb}$.
The types
\[
  L_\kw{rb} : \mathbf{1} \twoheadrightarrow \mathcal{C}@S_\kw{rb}
  \qquad
  \text{and}
  \qquad
  \Clight(\kw{bq.c}) : \mathcal{C}@\kw{mem} \twoheadrightarrow \mathcal{C}@\kw{mem}
\]
are not directly compatible,
given that $L_\kw{rb}$ manipulates a state of type $S_\kw{rb}$
and $\kw{rb.c}$ expects a memory state of type $\kw{mem}$.
The solution is to lift each one to ``pass through''
the state of the other:
\[
  \begin{tikzcd}[column sep=huge]
    \mathbf{1}@\kw{mem}
    \ar[r, "L_\kw{rb}@\kw{mem}"] &
    \mathcal{C}@S_\kw{rb}@\kw{mem} \cong
    \mathcal{C}@\kw{mem}@S_\kw{rb}
    \ar[r, "\Clight(\kw{bq.c})@S_\kw{rb}"] &
    \mathcal{C}@\kw{mem}@S_\kw{rb}
  \end{tikzcd}
\]
Implicitly taking into account the isomorphisms
\[
  \mathbf{1} \cong \mathbf{1}@\kw{mem}
  \qquad
  \text{and}
  \qquad
  \mathcal{C}@S_\kw{rb}@\kw{mem} \cong
  \mathcal{C}@\kw{mem}@S_\kw{rb} \cong
  \mathcal{C}@(S_\kw{rb} \times \kw{mem})
  \,,
\]
they can then be composed into
\begin{gather*}
  \Clight(\kw{bq.c})@S_\kw{rb} \odot
  L_\kw{rb}@\kw{mem} :
  \mathbf{1} \twoheadrightarrow \mathcal{C}@(S_\kw{rb} \times \kw{mem})
  \,.
\end{gather*}

To establish that this combination implements the overlay interface $L_\kw{bq}$,
we can lift the latter to:
\[
  L_\kw{bq}@\kw{mem} : \mathbf{1} \twoheadrightarrow
    \mathcal{C}@(S_\kw{bq} \times \kw{mem})
  \,.
\]
We will then need to define a simulation convention
explaining the relationship between
the states of type $S_\kw{bq}$ used by the specification and
the states of type $S_\kw{rb}$ used by the implementation.
\end{example}
%}}}

\section{Old intro}

\subsection{Verification Frameworks} \label{sec:intro:bigpict} %{{{

Building large-scale certified systems
requires the ability
to model and specify those systems compositionally,
so that verification can be carried out
on components of a manageable size.
In addition,
the verification of large heterogeneous systems---%
for example,
computer systems involving combinations of
hardware, software and network components---%
will require formal models versatile enough
to account for the large variety of
operational paradigms and interfaces involved.

Devising models that are up to the task is challenging,
but existing research has laid much of the necessary groundwork.
Denotational semantics and category theory
excel at describing and manipulating compositional structures.
They tend to focus on the externally observable behavior of components,
abstracting away any internal details which are irrelevant
to the ways in which components interact and combine.
In principle,
they could be used to achieve
large-scale compositionality for certified components.
However,
category theory and denotational semantics
have not seen widespread adoption
for certified software engineering.

By necessity,
many certified software projects use
specialized semantic models,
chosen first and foremost
to make verification tractable
in the context of a particular
programming language or verification target.
Any compositional structures they provide
are likewise fine-tuned to their particular setting.
In this context,
mandating the use of any one model
%in order to achieve interoperability between
%certified system components
is unrealistic.
Instead,
researchers should attempt to establish
a hierarchy of semantic models
with varying degrees of generality.
Simple models could be used
in specific contexts in order to facilitate verification.
At the same time,
the resulting specifications and proofs
could be embedded into more flexible models
where they could interface
with other components.

With that said,
the high level of abstraction and generality of
existing compositional semantics
is not the only thing
standing in the way of their use for verification.
As a general rule,
work of this kind has focused on characterizing exactly
the space of behaviors which can be defined in a particular language.
By contrast,
verification often operates in much more open-ended settings.
The focus is the relationship between specifications and implementations,
involving both abstraction and program refinement.
A better understanding of how these concepts fit into
the paradigm of compositional semantics
is therefore another important task
to make the construction of
large-scale, heterogeneous, certified systems
tractable.

%In this paradigm,
%suitable high-level models would need to account
%for specifications, refinement and abstraction,
%which have not been a traditional focus
%for denotational semantics
%but which are the bread and butter of
%many verification frameworks.
%Conversely,
%compositional structures
%used in low-level models to facilitate verification
%should ideally be designed in such a way that
%they can be preserved when embedded into richer settings.
%This would allow compositional reasoning
%to cut across components of different kinds,
%even when they were originally verified
%using different low-level frameworks.

%In what follows,
%we use this lens
%to examine recent work on
%the certified compiler CompCert.

%We present a formal account
%of both horizontal and vertical compositionality
%as well as the \emph{certified abstraction layer}
%techniques used to verify
%the operating system kernel CertiKOS.
%We identify \emph{double categories}
%as an account of structures found in CompCertO,
%an extension of CompCert
%which provides a compositional semantic preservation theorem.
%We outline a high-level account
%of this semantic model
%and show how these high-level structures
%can be used to facilitate
%an implementation of certified abstraction layers
%within the framework provided by CompCertO.

%}}}

% xx where do interaction trees fit in the picture

\subsection{CompCert} %{{{

Work on
the certified C compiler CompCert \cite{compcert}
illustrates many challenging aspects of compositional verification.
CompCert is a C compiler written in the Coq proof assistant
which comes with a formal, mechanized proof of correctness:
the semantics of the source and target languages
are described as labeled transition systems,
and a simulation proof
shows that the behavior of the compiled program
refines the behavior of the source program.

%The original correctness theorem of
Originally,
CompCert
only modeled the compilation of whole programs.
To overcome this limitation,
researchers first attempted to make
the transition system model used by CompCert
more compositional,
and to update the compiler's semantic preservation property
to operate at the level of individual translation units
\cite{compcompcert}.
Because
this came at a high cost in terms of proof effort,
subsequent work on verified separate compilation
turned instead to the development of compositional
\emph{proof techniques}
within the context of a closed, whole-program semantics
(\S\ref{sec:related:compcert}).
%Another successful approach
%was explored by the work
However,
the recent work on CompCertO \cite{compcerto}
revisits compositional semantic preservation,
addressing its challenges
by incorporating data refinement
as a first-class citizen.
The flexibility
gained by this approach
makes it possible %---as in CompCertM---%
to reuse much of CompCert's existing
correctness proofs,
and to address any difficulties in composition
using external reasoning.

The semantic model of CompCertO
remains fairly specialized:
its goal is to minimize any changes needed to CompCert,
to eliminate any unnecessary complexity,
and to enable compositionality
to the exact extent required
for compositional semantic preservation to work.
%In particular,
%the model does not account for
%encapsulated state;
%it describes the behavior of individual function calls,
%independently of any prior history,
%and expect any persistent state
%to be passed by the environment at entry
%alongside the names and parameters of the function to be invoked.
%
Yet
the directness of the approach
%in terms of the programme outlined in \S\ref{sec:intro:bigpict},
%CompCertO's approach also opens up the possibility
opens the door to
a compositional embedding
of CompCertO's semantics and proofs
into richer models.
In fact,
we will show that CompCertO's model
already exhibits a surprisingly rich compositional structure,
and that once this structure has been brought to light,
it can be extended to account for encapsulated state
using fairly general constructions.

%We will show that CompCertO's model
%can be equipped with the structure of
%a \emph{double category}.
%Based on this view
%of CompCertO's open semantics,
%we can further extend the framework
%to support state encapsulation.
%Moreover,
%once they are brought to light,
%we can give an account of
%these compositional structures
%in terms of simpler and more abstract models,
%such as Reddy's
%coherence space model of objects
%\cite{objsem}.
%
%The main difficulty encountered in this work
%is the difference in data representation
%in the semantics of source and target programs.
%In CompCert's closed semantics,
%these differences play no role in
%the externally observable behavior of programs.
%Consequently,
%simulation proofs can capture these differences
%in the simulation relations they use.
%Simulation relations are existentially quantified
%within proofs,
%and can remain hidden in correctness statements.
%By contrast,
%in the context of compositional semantics,
%cross-component interactions which occur
%within a linked program become observable,
%and these internal details can no longer be ignored.

%}}}

\subsection{Certified abstraction layers} %{{{

The divide between abstract semantic models
and concrete verification projects
also exists in the context of \emph{certified abstraction layers},
a technique which
allows a complex program to be verified in steps,
and which was used in the construction of
the certified operating system kernel CertiKOS
\cite{popl15,ccal}.

Under this approach,
the verification begins first with the lowest-level layer of code,
which other parts of the program rely on.
Once verified,
this code can be given a high-level specification,
which hides the implementation details
and makes it possible to reason about client code
in terms of an abstract view
of the lower layer's state.
This abstract state
can be accessed only by calling into
the layer's interface,
realizing a form of state encapsulation
and data abstraction.

To implement this methodology,
CertiKOS uses a modified version of CompCert
called CompCertX,
which parameterizes the compiler's semantics and correctness proof
with a layer interface.
This requirement [is a problem but now we have a general-purpose CompCertO].

%This approach was 
%
%There are limitations to the way this approach
%was implemented for the verification of CertiKOS.
%This work reused the CompCert semantics,
%by augmenting its memory model
%with a layer-dependent \emph{abstract state} component,
%making it possible to connect the code verification
%with the compiler's correctness proof.
%However,
%this means that the formulation of certified abstraction layers
%used in this context
%was intimately tied to CompCert-specific constructs.

In addition,
while this approach allows code
to be verified in a piecemeal manner,
and allows reasoning at an appropriate level of abstraction
for each layer,
the method is not fully compositional
in the sense that it relies on \emph{closed} semantics.
The behavior of a given abstraction layer
can only be characterized
once a specification for the lower layer it builds on has been given,
reducing the flexibility of the framework.
This also forces verification to proceed
in a linear way,
so that when two parts of the code
are independent,
one must nonetheless be verified
as a client to a layer which includes the other.

To address these limitations,
more abstract models have been proposed
for certified abstraction layers
\cite{rbgs-cal,popl22},
inspired by game semantics and
coherence space models of objects \cite{objsem}.
These models have not been used
in the context of practical verification tasks,
but shed light on the underlying structures
involved in this methodology.

Based on our framework,
we propose a formulation of certified abstraction layers
which incorporates the best of both worlds:
on one hand,
like the original formulation,
it is based on CompCertO semantics
and seamlessly integrates
with the compiler's correctness theorem;
on the other hand,
like more recent work,
the categorical structures underlying its construction
are made explicit,
facilitating a more compositional approach
to certified abstraction layers.
To illustrate these capabilities,
we demonstrate the use of this framework
by verifying a simple example
found in prior work.

%}}}

\begin{example}[Certified Abstraction Layers] \label{ex:overview:lint} %{{{

Software systems are often constructed in layers:
basic data structures and functionality
are implemented by low-level code.
We can then rely on this code
without concern its implementation details
or the data representation it uses.
Instead,
a programmer writing client code
will understand and reason about
this layer of code
in terms of a more abstract mental picture
of its operation.
For example,
when using the functions $\kw{enq}$ and $\kw{deq}$
shown in Fig.~\ref{fig:code},
we can think of the bounded queue they provide
as a simple sequence of elements,
and ignore the mechanics
of the ring buffer used to implement it.
\emph{Certified abstraction layers}
formalize this methodology within CompCert
and were used to verify the operating system kernel
CertiKOS \cite{popl15}.

Modeling layers required a modification of CompCert's semantics
to incorporate an \emph{underlay interface}
described using an \emph{abstract state}.
%to achieve a limited form of compositionality.
The closed semantics of CompCert can be described as
\[
  \chi : \top \twoheadrightarrow \mathcal{C} \mathbin@ \kw{mem}
  \: \vdash \:
  \kw{Clight}_\chi[M] : \top \twoheadrightarrow \mathbf{I}
  \,,
  \quad \text{ where }
  \top := \langle \varnothing, \varnothing \rangle
  \text{ and }
  \mathbf{I} := \langle \mathbbm1, \mathbbm1 \rangle
  \,.
\]
The transition system $\kw{Clight}_\chi[M]$
is invoked with a trivial question ${*} \in \mathbf{I}^\que$,
which initiates the execution of the $\kw{main}$ function of $M$.
When $M$ invokes an external function,
the behavior of that function is obtained from the parameter $\chi$.
In the CertiKOS proof,
abstraction layers are formalized by using
a variant CompCertX,
whose semantic model can be described as:
\[
  \forall D \in \mathbf{Set}
  \: \mathbin. \:
  L^\flat :
    \top \twoheadrightarrow
    \mathcal{C} \mathbin@ \kw{mem} \mathbin@ D
  \: \vdash \:
  \kw{Clight}_{L^\flat}[M] :
    \top \twoheadrightarrow
    \mathcal{C} \mathbin@ \kw{mem} \mathbin@ D
  \,.
\]
This allows the semantics of $M$ to be evaluated
in the context of the \emph{underlay} interface $L^\flat$,
whose primitives are described in terms of
an \emph{abstract state} memory component of type $D$.
A specification for the code in \autoref{fig:code}
may use an abstract state in $S_\kw{bq} := \kw{val}^*$.
The corresponding layer interface
$L_\kw{bq} : \top \twoheadrightarrow
 \mathcal{C} \mathbin@ \kw{mem} \mathbin@ S_\kw{bq}$
will then generate traces such as:
\[
  L_\kw{bq} \:\vDash\:
  \kw{enq}(v) \mathbin@ m \mathbin@ \vec{q}
  \:\rightarrowtail\:
  \kw{undef} \mathbin@ m \mathbin@ \vec{q}v
  \qquad%\qquad
  L_\kw{bq} \:\vDash\:
  \kw{deq}() \mathbin@ m \mathbin@ v\vec{q}
  \:\rightarrowtail\:
  v \mathbin@ m \mathbin@ \vec{q}
\]
We can then evaluate and reason about any client code
in terms of this abstract representation:
\[
  \kw{Clight}_{L_\kw{bq}} \big[\,
  \begin{minipage}{13em}
\begin{minted}{C}
void rot() { enq(deq()); }
\end{minted}
  \end{minipage} \,\big]
\:\vDash\:
%  \qquad
%  \Rightarrow
%  \qquad
  \kw{rot}() \mathbin@ m \mathbin@ v\vec{q}
  \:\rightarrowtail\:
  {*} \mathbin@ m \mathbin@ \vec{q}v
  \,.
\]

We will use certified abstraction layers
to illustrate the flexibility of CompCertO's approach
and as an example application for the techniques we introduce.
\end{example}
%}}}

\begin{example}[Layer Semantics] %{{{
As noted in Example~\ref{ex:overview:lint},
the CertiKOS verification effort
required the entire correctness proof of CompCert
to be modified to operate in terms of an underlay interface.
We can achieve a similar effect in CompCertO
without any modification to the compiler,
by defining
\[
  \kw{Clight}_{L^\flat}[M] :=
    (\kw{Clight}(M) \mathbin@ D^\flat) \odot L^\flat
  \,.
\]
We will see that CompCertO's open simulations
make it possible to formulate layer correctness
in a reasonably straightforward way as well.
\end{example}

%Using this construction,
%a C layer interface $L^\flat$ which uses abstract states in $D$
%can be modeled as a transition system of type
%$
%  L^\flat : \top \twoheadrightarrow \mathcal{C}_\kw{m}@D
%$,
%using questions and answers of the form:
%\begin{align*}
%  (\mathcal{C}_\kw{m}@D)^\que &:=
%    \{ f(\vec{v})@(m, d) \mid
%       f \in \kw{ident},
%       \vec{v} \in \kw{val}^*,
%       m \in \kw{mem},
%       d \in D \}
%  \\
%  (\mathcal{C}_\kw{m}@D)^\ans &:=
%    \{ v'@(m', d') \mid
%       v' \in \kw{val},
%       m' \in \kw{mem},
%       d' \in D \}
%\end{align*}
%This leaves the question of evaluating client code
%running on top of the underlay $L$.
%}}}

\begin{example}[Abstraction relations] \label{ex:overview:absrel} %{{{
In \S\ref{sec:overview:slift},
we noted that a layer specification
$L^\sharp :
 \top \twoheadrightarrow \mathcal{C} \mathbin@ \kw{mem} \mathbin@ D^\sharp$
and its implementation
$\kw{Clight}_{L^\flat}[M] :
 \top \twoheadrightarrow \mathcal{C}_\kw{m}@D^\flat$
%in terms of an underlay $L^\flat$
are not directly comparable, owing to their
use of different abstract states.
We now show how to construct,
using the techniques that we have introduced,
a simulation convention suitable for
stating the desired correctness property.

Note the decomposition
$\mathcal{C}_\kw{m}@D \cong \mathcal{C} \otimes [\kw{mem}] \otimes [D]$.
When a layer specification $L^\sharp$ is implemented,
part of its abstract state $D^\sharp$ is realized as concrete values
stored in the global memory,
and part of it reflects the abstract state of the underlay in $D^\flat$.
The details of this can be expressed using a relation
$R \subseteq (\kw{mem} \times D^\sharp) \times (\kw{mem} \times D^\flat)$,
allowing us to state the layer correctness property as
\[
  L^\flat \vdash_R M : L^\sharp \quad :\Leftrightarrow \quad
    L^\sharp \le_{\top \twoheadrightarrow \mathcal{C} \otimes [R]}
    \llbracket M \rrbracket_{L^\flat}
  \,.
\]
To ensure that this relation is compatible with client code,
we must require that
\[
  \forall C \mathbin.
    \kw{Clight}(C)@D^\sharp
    \le_{\mathcal{C} \otimes [R] \twoheadrightarrow \mathcal{C} \otimes [R]}
    \kw{Clight}(C)@D^\flat
  \,.
\]
%Now consider an assembly version
%$L^\sharp_\mathcal{A} :
% \top \twoheadrightarrow \mathcal{A} \otimes [\kw{mem}] \otimes [D^\sharp]$
%of the specification, such that
%\[
%  L^\sharp
%  \le_{\top \twoheadrightarrow \mathbb{C} \otimes D^\sharp}
%  L^\sharp_\mathcal{A}
%  \,.
%\]
\end{example}
%}}}

\end{document}
%%% Local Variables: 
%%% mode: LaTeX
%%% TeX-command-extra-options: "-shell-escape"
%%% End:


\chapter{The Strategy Model}
\label{ch:strat}

The preceding chapters
extended the CompCertO semantics
with stateful structures
to address data abstraction in verification tasks
through a modular approach.
The example of implementing a bounded queue with a ring buffer
is representative of
libraries that implement
specific data structures or algorithms.

In this chapter,
this chapter further improves the verification framework
by using traces of events
to model the behavior of components.
I will also investigate the approach to
verify actual executable programs
with this toolbox.

First,
notice that the specification language used in the previous chapters
is still the CompCertO transition system.
In formal verification,
generally speaking,
the specification is expected to be expressed
in a higher-level language that is more descriptive
and easier to understand.
To this end,
I introduce a strategy-based semantics model
that centers on the traces of events
instead of the internal transition steps
as the language of specifications.
Recall that
traces have already been used as a shorthand way
to describe the behavior of a component
throughout this dissertation,
such as:
\[
  L \:\vDash\: q \rightarrowtail
  (q_1 \leadsto r_1) \rightarrowtail
  \cdots \rightarrowtail
  (q_n \leadsto r_n) \rightarrowtail
  r
\]
In this chapter,
I present a formal semantics around the notion of traces,
and show that
it can integrate with the CompCertO semantics.

Second,
I demonstrate the power of this framework
through an example that,
while simple in terms of programming,
poses several nontrivial verification challenges.
First, the verification must account for
the process of loading the executable
as a running process,
which requires modeling the surrounding execution environment.
Moreover, programs are rarely standalone;
they are typically built from multiple components
whose individual behaviors may be simple
but whose interactions---especially
when treated as independent processes---can be subtle
and complex.
Finally, these components are not always written
in the same programming language.
Real systems often combine code written
in different languages,
and a convincing verification
must bridge these heterogeneous boundaries.

\begin{table}
  \centering
  \begin{tabular}{ll}
    \toprule
    Notation & Description\\
    \midrule
    E, F & Effect Signature\\
    $\sigma : E \rightarrow F$ & Strategy (Def.~\ref{def:strat})\\
    $\sigma \odot \tau$ & Layered Composition of Strategies (Def.~\ref{def:lcomp})\\
    $\sigma \oplus \tau$ & Flat Composition of Strategies (Def.~\ref{def:strat:flat-comp})\\
    $\sigma \at f$ & Spatial Composition of Strategies (Def.~\ref{def:strat:spatial-comp})\\
    $\sigma^*$ & Regular Closure of Strategies (Def.~\ref{sec:strat:regular-closure})\\
    $[f] : [U] \rightarrow [V]$ & Lens as Strategy (Def.~\ref{def:strat:stateful-lens})\\
    $\llbracket L \rrbracket : \llbracket A \rrbracket \rightarrow \llbracket B \rrbracket$ & Transition System as Strategy (Def.~\ref{ox:def:lts-lens})\\
    \bottomrule
  \end{tabular}
  \caption{Summary of notations}
  \label{tab:strat:notations}
\end{table}

In the remainder of this chapter,
I progressively develop the strategy model
and its supporting machinery.
Section \ref{sec:strat:overview} begins with an informal overview
that introduces the central ideas at an intuitive level.
Section \ref{sec:strat:model} then presents a precise formalization of the model.
Building on this foundation,
Sections \ref{sec:strat:composition} and \ref{sec:strat:regular-closure}
explore different ways of constructing strategies
and establish their key properties.
Section \ref{sec:strat:loaders} puts these ideas
into practice by implementing loaders for executable programs
and presenting a full formal proof of the running example.
Finally, Table \ref{tab:strat:notations} summarizes the notations
used throughout the chapter,
providing a concise reference for
the technical development that follows.

\section{Overview}
\label{sec:strat:overview}

In this section,
I first introduce
another verification task
as a running example
that is algorithmically simple
but involves complex interaction among components
and requires modeling
the complete cycle of compiling, linking, and executing
the programs.
I then demonstrate
how to approach the verification task
using the strategy model.

\subsection{Running Example}

In this chapter,
I use the code shown in \autoref{fig:readwritehello}
as a running example.
It consists of two different programs that
use a common C library and
are designed to work together.
As illustrated in the usage scenario shown earlier,
the 32-bit x86 assembly program \kw{secret.s}
outputs a coded message
to be deciphered by \kw{decode.c}.
In particular,
the programs together satisfy the following informal specification:
\begin{equation}
  \begin{minipage}{.9\textwidth}
    \it
    Suppose that,
    after compilation,
    \kw{secret.s} and \kw{decode.c}
    are each linked with \kw{rot13.c}.
    If the output of the first program
    is fed as input to the second,
    ``hello, world!'' will be displayed.
  \end{minipage}
  \tag{a}
  \label{eqn:hellospec}
\end{equation}

The programs are simple;
to verify that property (\ref{eqn:hellospec}) holds,
a reader with the right background
can mentally execute the code step by step
and convince themselves that the expected outcome will occur.
However, this task is complex in its own way
because it mobilizes implicit knowledge and assumptions regarding
the C and x86 assembly languages,
the compiler's correctness with respect to the calling convention in use,
and some aspects of the Unix execution environment.
%
Likewise, any formal account of property (\ref{eqn:hellospec})
must involve these aspects of the problem as well,
encompassing all three of the challenges outlined at the beginning of this section.
%Property~(\ref{eqn:hellospec})
%also involves \emph{two different} programs.
To my knowledge,
there exists no program logic or verification framework
that can deal with this example.

\subsection{Strategy Model, Informally}

To address these challenges and enable verification of such complex heterogeneous systems,
the strategy model I present in this chapter
is built around the notion of
traces
that are presented as a sequence of events.
The events are typed using
effect signatures ($E, F, \ldots$).
The signatures also describe
the interfaces of the software components.

\begin{definition}
  An \emph{effect signature} is a set $E$ of \emph{questions}
  together with an assignment $\kw{ar} : E \rightarrow \mathbf{Set}$
  associating to each question $m \in E$
  a set of \emph{answers} $n \in \kw{ar}(m)$.
  They are often presented together as the set of bindings
  $\{ m \mathbin: N \mid m \in E \wedge N = \kw{ar}(m) \}$.
\end{definition}

Unlike the conventional algebraic effect literature,
in the semantics model presented in this chapter,
triggering effects operates in a manner similar to
making a regular function call
instead of
abruptly terminating the current execution
with the continuation captured
in the delimited context.
The handler is simply the callee
and it must return an answer to the caller
who triggered the effect.

\begin{example}
  Consider the execution environment
  for the programs \kw{secret} and \kw{encode}
  shown in \autoref{fig:readwritehello}.
  Since the programs
  do not use any command-line arguments or environment variables,
  their invocation can be modeled with a single question:
  \[
    \mathcal{P} \, := \,
    \{ \kw{run} : \mathbb{N} \}
    \,.
  \]
  The answer $x \in \mathbb{N}$ is the exit status of the process.
  %which in our cases is expected to be zero.
  Moreover, in the course of its execution
  each process can invoke the \kw{read} and \kw{write} system calls.
  This interface is described with the signature
  \[
    \mathcal{S} \, := \, \{
      \kw{read}_i[n] \mathbin: \Sigma^* , \,
      \kw{write}_i[s] \mathbin: \mathbb{N} \, \mid \,
      i \in \mathbb{N}, \,
      n \in \mathbb{N}, \,
      s \in \Sigma^*
    \}
    \,,
  \]
  where $\Sigma := \{0,1\}^8$ is the alphabet of possible byte values.
  In this formalism,
  the program \kw{secret} will invoke
  the operation
  $\kw{write}_1[\texttt{"uryyb, jbeyq!"}]
  \in \mathcal{S}$,
  where $i := 1$ is the file descriptor associated with the standard output;
  the outcome should be $14 \in \mathbb{N}$
  if the operation succeeds.
\end{example}

Strategies $\sigma : E \rightarrow F$
model the behavior of components
by specifying the actions
taken by the component
in response to the possible actions of the environment,
and is represented as a set of traces.
In general,
the traces have the following shape:
\[
  \sigma \:\vDash\:
  \big(q_1 \sysstep
    (m_1 \envstep n_1) \sysstep
    \cdots \sysstep
    (m_k \envstep n_k) \sysstep
  r_1\big)
  \envstep
  \big(q_2 \sysstep
    \cdots \sysstep
  r_2\big)
  \envstep
  \cdots
\]
where $q_i \in F$ are the events triggered by the environment,
and the component triggers events $m_i \in E$.
The arrow $\sysstep$ represents a step in the execution of the component,
while the arrow $\envstep$ represents a step in the execution of the environment.

It is also worth mentioning that
the traces can include repeated invocations from the caller environment
as shown by the $q_2$ following the $r_1$ in the trace.
This means the traces
have persistent state intrinsically.

\begin{example}
  \label{ex:decodespec}
  The strategies
  $\Gamma_\kw{secret}, \Gamma_\kw{decode} : \mathcal{S} \rightarrow \mathcal{P}$
  formulate specifications for the commands $\kw{secret}$ and $\kw{decode}$.
  The processes admit the execution traces
  {\small
    \begin{align*}
      \Gamma_\kw{secret} &\vDash \kw{run}
      \sysstep (\kw{write}_1[\texttt{\textnormal{"uryyb, jbeyq!\textbackslash{}n"}}] \envstep 14)
      \sysstep 0
      \\
      \Gamma_\kw{decode} &\vDash \kw{run}
      \sysstep (\kw{read}_0[100] \envstep \texttt{"uryyb, jbeyq!"})
      \sysstep (\kw{write}_1[\texttt{"hello, world!"}] \envstep 14)
      \sysstep 0
      \,.
  \end{align*}}
  For the counter mentioned in \autoref{sec:oe:persistent-state},
  the following strategy can be used:
  \[
    \sigma_\kw{cnt} \vDash \big(\kw{inc}() \sysstep 0\big) \envstep
    \big(\kw{inc}() \sysstep 1\big) \envstep \cdots
  \]
\end{example}

The space of strategies
is equipped with a least element $\bot$
and a refinement relation $\le$.
The $\bot$ element represents the undefined behavior.
The refinement relation
\[
  \rho \::\: L_1 \le L_2
\]
holds when the target strategy $L_2$
exhibits at least the behavior
of the source strategy $L_1$.
This can be used for modeling
the correctness of a program
in terms of a specification.

To show that the program $\kw{decode}$
satisfies the specification $\Sigma_\kw{decode}$
given in \autoref{fig:readwritehello}
requires modeling the way in which
$\kw{decode.c}$ and $\kw{rot13.c}$ behave together
\emph{as a process}.
Assuming that
$\llbracket \kw{decode} \rrbracket :
\mathcal{S} \rightarrow \mathcal{P}$,
models the response of the combined program
to the trigger $\kw{run} \in \mathcal{P}$
in terms of system calls performed over the interface $\mathcal{S}$,
the goal will be to establish a refinement
$
\Sigma_\kw{decode}
\le
\llbracket \kw{decode} \rrbracket
$.
The model $\llbracket - \rrbracket$ will involve CompCertO semantics
and take into account the way the program is
compiled, linked and loaded.
%As a first step,
%we will show how to account for the source-level interaction
%between the components $\kw{decode.c}$ and $\kw{rot13.c}$.

The refinement relation between strategies
is a simplified notion
because it does not take into account
the data abstraction.
For that, a counterpart to the simulation conventions is needed
and it will be the main topic in the next chapter.

\subsection{Compare with CompCertO Semantics}

Before proceeding to the formal development,
it is instructive to compare the strategy model
with the CompCertO semantics to understand their relationship
and respective strengths. The strategy model is
similar to the CompCertO semantics.
I compare the two semantics models
in the following aspects.

\subsubsection{Compositional Semantics}

The open transition system of CompCertO
excels at verifying the correctness of compiler passes.
Its design focuses on faithfully capturing execution steps,
making it highly effective for
reasoning about compiler transformations.
However,
for broader software verification tasks,
this operational orientation becomes a limitation.
Real software systems are typically compositions of heterogeneous components,
each governed by its own semantics.
One approach would be to translate all components
into CompCertO's transition system semantics
so that all proofs can be carried out there.
While feasible, this is often cumbersome.

A more systematic solution is to adopt a compositional semantics
that serves as glue between components.
Compositional semantics emphasize observable events and structured interactions
rather than internal execution steps,
thereby enabling heterogeneous components to be reasoned about together.
In fact,
CompCertO's open transition system already
incorporates some aspects of compositional semantics,
but its operational bias reflects
its primary goal of compiler verification.

The strategy model,
by contrast,
shifts the balance decisively toward compositional reasoning.
As a hybrid between operational and denotational semantics,
strategies deemphasize low-level execution details
and instead highlight observable behaviors and interactions.
This balance makes the strategy model a sweet spot:
operational enough to remain concrete and mechanizable,
but denotational enough to support modularity, abstraction, and scalability.
As such, it offers broader applicability than CompCertO,
serving not only compiler correctness
but also the verification of heterogeneous, interacting systems.

\subsubsection{Language Interfaces and Effects Signatures}

The semantic model of CompCertO uses \emph{language interfaces}
of the form $A = \langle A^\circ, A^\bullet \rangle$
as the basis for component interactions.
These interfaces resemble effect signatures
in that they describe the possible questions and answers exchanged
between a component and its environment.
However,
CompCertO's interfaces are comparatively coarse,
every question $q \in A^\circ$ is associated with the same set of answers $r \in A^\bullet$.

Effect signatures,
by contrast,
are more typed and structured.
They not only specify, for each operation,
the precise type of its result,
but also rigorously constrain
the set of possible operations
in the first place.
In other words,
effect signatures capture both
what operations are admissible and
how their results must be shaped,
whereas CompCertO's language interfaces treat all operations uniformly.

For the C language,
for example,
questions are function calls of the form $f(\vec{v})@m$, where
$f$ identifies the function to be called,
$\vec{v} \in \kw{val}^*$ are the actual parameters, and
$m \in \kw{mem}$ is the memory state at the time of invocation;
answers take the form $v'@m'$ where
$v' \in \kw{val}$ is the value returned by the function~$f$ and
$m' \in \kw{mem}$ is the new state of the memory.
This is captured by the effect signature
\[
  \mathcal{C} \mathbin@ \kw{mem} \:=\:
  \{ f(\vec{v})@m \mathbin: \kw{val} \times \kw{mem} \mid
    f \in \kw{val}, \:
    \vec{v} \in \kw{val}^*, \:
  m \in \kw{mem} \}
  \,.
\]
I will show that CompCertO language interfaces
can be systematically mapped to effect signatures.

\subsubsection{Transition Systems and Traces}

When the internal state transitions
are ignored,
the execution of a transition system
can be represented as a single-use trace:
an execution begins with a question from environment
and terminates with an answer to that question.
For instance,
the program $\kw{decode.c}$ will exhibit traces such as:
{\scriptsize
  \begin{align*}
    \kw{Clight}(\kw{decode.c}) \:\vDash\:
    \kw{main}()@m &\rightarrowtail
    (\kw{read}(0, b, 100)@m[b \mapsto \textit{unspecified}] \leadsto
    14@m[b \mapsto \texttt{\textnormal{"uryyb, jbeyq!\textbackslash{}n"}}]) \\& \rightarrowtail
    (\kw{rot13}(b, 14)@m[b \mapsto \texttt{\textnormal{"uryyb, jbeyq!\textbackslash{}n"}}] \leadsto
    {*}@m[b \mapsto \texttt{\textnormal{"hello, world!\textbackslash{}n"}}]) \\& \rightarrowtail
    (\kw{write}(1, b, 14)@m[b \mapsto \texttt{\textnormal{"hello, world!\textbackslash{}n"}}] \leadsto
    14@m[b \mapsto \texttt{\textnormal{"hello, world!\textbackslash{}n"}}]) \\& \rightarrowtail
    0@m[b \mapsto \textit{deallocated}]
  \end{align*}
}%

In contrast,
the strategy model
generalizes this notion.
A trace in the strategy model
may involve repeated invocations from the caller environment,
and, crucially,
the behavior of later invocations
can depend on the execution of earlier ones.
This endows the strategy model
with persistent state as an intrinsic feature.
From this perspective,
a transition system can be seen as
a special case of the strategy model,
suitable when internal state transitions are irrelevant
or need not be tracked explicitly.

The trace above is notably more complicated than
the high-level one given in Example~\ref{ex:decodespec}.
This complexity stems from low-level details of
the C memory model,
which must be faithfully represented in the CompCertO semantics.
Nevertheless,
these semantics of C and assembly will eventually
serve as building blocks for modeling the scenario in Fig~\ref{fig:readwritehello},
where they will be connected to the kind of high-level specifications
that have been presented so far.

\section{Strategy Model}
\label{sec:strat:model}

In this section,
I give a formal characterization of the strategy model.

\subsection{Strategies}

Preceding sections have already informally described many strategies
using interaction traces of the form
\[
  q \rightarrowtail
  (m_1 \leadsto n_1) \rightarrowtail
  (m_2 \leadsto n_2) \rightarrowtail
  \cdots \rightarrowtail
  (m_k \leadsto n_k) \rightarrowtail
  r
  \leadsto
  q' \rightarrowtail
  \cdots
  \,.
\]
Such traces will also be represented more compactly as
$
q \underline{m_1} n_1
\underline{m_2} n_2 \cdots
\underline{m_k} n_k \underline{r}
q' \cdots
$,
where underlined actions
are those of the components,
alternating with the actions of the environment.
Beyond the requirement of alternation
between the component and the environment actions,
the traces of interest
must also be well-bracketed~\citep{gamesemantics}.
The corresponding constraints
are formalized using the notion of position,
as illustrated in the following diagram.
\[
  \begin{tikzcd}
    P_{E,F} \ar[r, bend left, "q \in F"] &
    P_{E,F}^q \ar[r, bend left, "m \in E"] \ar[l, bend left, "r \in \kw{ar}(q)"] &
    P_{E,F}^{qm} \ar[l, bend left, "n \in \kw{ar}(m)"]
  \end{tikzcd}
  % \qquad
  % \begin{array}{r@{}ll}
  %   s \in P_{E,F} &{} ::= \epsilon \mid q s^q &
  %   \bigl( q \in F \bigr)
  %   \\[1ex]
  %   s^q \in P_{E,F}^q &{} ::= \underline{m} s^{qm} \mid \underline{r} s &
  %   \bigl( m \in E, r \in \kw{ar}(q) \bigr)
  %   \\[1ex]
  %   s^{qm} \in P_{E,F}^{qm} &{} ::= \epsilon \mid n s^q &
  %   \bigl( n \in \kw{ar}(m) \bigr)
  % \end{array}
\]
Consider a component
that interacts with its handler via the effect signature $E$,
and with its client via the effect signature $F$.
The component progresses through a sequence of positions,
governed by the structure of the interaction:
\begin{itemize}
  \item It starts at the position $P_{E,F}$,
    where the only permissible event is
    an incoming question $q \in F$,
    resulting in a transition to the position $P_{E,F}^q$;
  \item At position $P_{E,F}^q$,
    the next move is either:
    \begin{itemize}
      \item an outgoing question $m \in E$,
        in which case the position transitions to $P_{E,F}^{qm}$;
      \item an outgoing answer $r \in \kw{ar}(q)$,
        in which case the position transitions to $P_{E,F}^q$.
    \end{itemize}
  \item At position $P_{E,F}^{qm}$,
    the only allowed move is
    an incoming answer $n \in \kw{ar}(m)$,
    after which the position returns to $P_{E,F}^q$.
\end{itemize}

\begin{definition}[Strategy] \label{def:strat}
  A \emph{play} in the game $E \rightarrow F$
  is an element $s \in P_{E,F}$
  in the set generated by the grammar:
  \[
    \begin{array}{cccc}
      \begin{prooftree}
        \hypo{q \in F}
        \hypo{s \in P_{E,F}^q}
        \infer2{qs \in P_{E,F}}
      \end{prooftree}
      &
      \begin{prooftree}
        \hypo{m \in E}
        \hypo{s \in P_{E,F}^{qm}}
        \infer2{\underline{m}s \in P_{E,F}^q}
      \end{prooftree}
      &
      \begin{prooftree}
        \hypo{n \in \kw{ar}(m)}
        \hypo{s \in P_{E,F}^q}
        \infer2{ns \in P_{E,F}^{qm}}
      \end{prooftree}
      \\[2em]
      \begin{prooftree}
        \infer0{\epsilon \in P_{E,F}}
      \end{prooftree}
      &
      \begin{prooftree}
        \hypo{r \in \kw{ar}(q)}
        \hypo{s \in P_{E,F}}
        \infer2{\underline{r}s \in P_{E,F}^q}
      \end{prooftree}
      &
      \begin{prooftree}
        \infer0{\epsilon \in P_{E,F}^{qm}}
      \end{prooftree}
    \end{array}
  \]
  The plays are ordered by the prefix relation $\sqsubseteq$
  following the rules:
  \[
    \begin{prooftree}
      \hypo{s_q \sqsubseteq_q s_q'}
      \infer1{q s_q \sqsubseteq q s_q'}
    \end{prooftree}
    \qquad
    \begin{prooftree}
      \infer0{\underline{m} \sqsubseteq_q \underline{m}n s_q}
    \end{prooftree}
    \qquad
    \begin{prooftree}
      \hypo{s_q \sqsubseteq_q s_q'}
      \infer1{\underline{m}n s_q \sqsubseteq_q \underline{m}n s_q'}
    \end{prooftree}
    \qquad
    \begin{prooftree}
      \infer0{\underline{r} \sqsubseteq_q \underline{r} s}
    \end{prooftree}
    \qquad
    \begin{prooftree}
      \hypo{s \sqsubseteq s'}
      \infer1{\underline{r} s \sqsubseteq_q \underline{r} s'}
    \end{prooftree}
  \]
  % We will write
  % \[
  %   \sigma \subseteq P_{E\twoheadrightarrow F}
  %   \qquad \text{such that} \qquad
  %   \forall s\ t.
  %   s \sqsubseteq t \wedge t \in \sigma \Rightarrow s \in \sigma
  %   \,.
  % \]
  Then a \emph{strategy} $\sigma : E \rightarrow F$
  is a prefix-closed subset of $P_{E,F}$.
  % where any two $s_1, s_2 \in \sigma$ satisfy $s_1 \coh s_2$.
\end{definition}

When using sets of traces as semantics,
prefix closure is essential
for establishing a meaningful comparison between strategies.
Consider two trace sets $\{ s, t \}$ and $\{ t \}$,
where $s \sqsubseteq t$.
These sets are conventionally identified
as equivalent because they convey the same information.
Without enforcing this identification through prefix closure,
the natural ordering on trace sets yields
only a preorder rather than a partial order,
since distinct sets can be mutually comparable.
The standard resolution is to take equivalence classes
under this preorder,
where the prefix-closed completion naturally emerges
as the canonical representative of each equivalence class,
thereby restoring the desired partial order structure.

\subsection{Example of Strategy}

In the previous section,
the behavior of various components
was described informally
through representative execution traces.
Now that we have formally introduced
the definition of strategies,
these informal descriptions can be transformed
into rigorous formal strategies
using the constructions presented above.

\subsubsection{Strategy with Persistent State}
\label{sec:strat:ex:persistent}

The behavior of a queue with infinite capacity
can be modeled as follows.
For a starting state $\vec{q} \in V^*$,
the strategy $\sigma_{\vec{q}} : \emptysig \rightarrow E_\kw{bq}$
is defined by the following rules:
\[
  \begin{prooftree}
    \infer0{\epsilon \in \sigma_{\vec{q}} }
  \end{prooftree}
  \qquad
  \begin{prooftree}
    \hypo{s \in \sigma_{\vec{q}} }
    \infer1{\kw{deq} \cdot \underline{v} \cdot s \, \in \, \sigma_{v \vec{q}} }
  \end{prooftree}
  \qquad
  \begin{prooftree}
    \hypo{s \in \sigma_{\vec{q} v}}
    \infer1{\kw{enq}[v] \cdot \underline{*} \cdot s \, \in \, \sigma_{\vec{q}} }
  \end{prooftree}
\]
Note that as expected, the strategy never performs any outgoing calls
but only interacts over $E_\kw{bq}$.
The behavior of a queue that is initially empty
is described by $\sigma_\epsilon : \emptysig \rightarrow E_\kw{bq}$,
where $\epsilon$ is the empty queue.
And the state is retained from one execution to the next.

\subsubsection{Strategy from Downset}

The behavior of the process $\kw{./secret}$
as a strategy $\Gamma\kw{secret} : \mathcal{S} \rightarrow \mathcal{P}$
can be defined as:
\[
  \Gamma_\kw{secret} :=
  \bigcup_{n \in \mathbb{N}}
  {\downarrow} \,
  \kw{run} \cdot \underline{\kw{write}}_1[\texttt{"uryyb, jbeyq!"}]
  \cdot n \cdot \underline{*}
  \,.
\]
The principal downset ${\downarrow}\, s$ for a play $s$
is the set $\{t \mid t \sqsubseteq s\}$ of all its prefixes.
Since it is downward-closed, it constitutes a valid strategy.
The strategy includes all possible return values
following the write action,
reflecting the non-deterministic environment behavior.

Unlike the previous example,
the strategy only contains plays of one execution.
Section \ref{sec:strat:regular-closure} will show
that the regular closure $(-)^*$
enables the process to restart after it terminates.

% \todo{add specification for decode and the overall strategy}

\subsubsection{Strategy from Stateful Lens}

The following example
derives a strategy from a stateful lens.

\begin{definition}[Strategy Interpretation of Stateful Lenses]
  \label{def:strat:stateful-lens}
  A stateful lens $f : U \leftrightarrows V$ with states in $P$
  can be given a strategy interpretation
  $[f] : [U] \rightarrow [V]$
  using simple effect signatures of the form
  $[U] := \{u : U \mid u \in U\}$.
  For a state $p \in P$,
  the strategy $[f]_p : [U] \rightarrow [V]$ is defined by
  \begin{center}
    \begin{prooftree}
      \infer0{\epsilon \in [f]_p}
    \end{prooftree}
    \qquad
    \begin{prooftree}
      \hypo{s \in [f]_p^v}
      \infer1{vs \in [f]_p}
    \end{prooftree}
    \qquad
    \begin{prooftree}
      \hypo{\kw{get}_f(v, p) = u}
      \hypo{s \in [f]_p^{vu}}
      \infer2{\underline{u}s \in [f]_p^v}
    \end{prooftree}
    \qquad
    \begin{prooftree}
      \hypo{v = v'}
      \hypo{s \in [f]_p}
      \infer2{\underline{v'}s \in [f]_p^v}
    \end{prooftree}
    \\
    \vspace{10pt}
    \begin{prooftree}
      \infer0{\epsilon \in [f]_p^{vu}}
    \end{prooftree}
    \qquad
    \begin{prooftree}
      \hypo{\kw{set}_f(v, p, u') = (v', p')}
      \hypo{s \in [f]_{p'}^{v'}}
      \infer2{u's \in [f]_p^{vu}}
    \end{prooftree}
  \end{center}
\end{definition}

Notice that the definition above
includes a non-deterministic choice from the system.
At the position $[f]^v_p$,
there are two possible next moves,
either returning $v$ unchanged
or passing on the $u$ component as an outgoing questions
and updating it based on the corresponding answer.
This underlines that the interpretation $[f]$
is not meant to be used as is
but is expected to be combined with a ``real'' strategy
from which it will inherit these scheduling choices.
In principle,
this phenomenon is similar to the situation with
transition systems as discussed in \S\ref{sec:ox:lens}.

% At the same time,
% the strategy interpretation
% is quite useful as an intermediate construction.
% In particular,
% it allows us to reuse
% refinement conventions of type
% $\mathbf{R} : [U] \leftrightarrow [V]$
% and refinement squares of type
% $[f] \le_{\mathbf{R} \twoheadrightarrow \mathbf{S}} [g]$
% in order to complement the horizontal structure
% defined for stateful lenses
% with the same vertical constructions we used for strategies.
% The following definition will be useful in this context.

\subsection{Behavioral Refinement and Determinism}

The space of strategies
is equipped with the least element $\varnothing$,
and the partial order $\subseteq$.
However,
this ordering is not immediately
suitable as a notion of behavioral refinement.
As seen in preceding examples,
both system and environment non-deterministic behaivors
can be encoded within the strategy.
Consequently,
a larger strategy
may permit
additional behaviors from the environment, the system, or both.

To address this issue,
I adopt set inclusion
as a working notion of behavioral refinement,
under the assumption that
all non-determinism arises from the environment.
A more comprehensive notion of refinement relation%
---capable of handling both sources of non-determinism---
will be introduced in \S\ref{ch:rc}.
The determinism of a strategy
is formalized via the coherence relation.
\begin{definition}[Coherence]
  The \emph{coherence} relation ${\coh} \subseteq P_{E,F} \times P_{E,F}$
  is defined by the following rules:
  \[
    \begin{array}{cccc}
      \begin{prooftree}
        \infer0{\epsilon \coh s}
      \end{prooftree}
      &
      \begin{prooftree}
        \infer0{s \coh \epsilon}
      \end{prooftree}
      &
      \begin{prooftree}
        \infer0{\epsilon \coh^{qm} s}
      \end{prooftree}
      &
      \begin{prooftree}
        \infer0{s \coh^{qm} \epsilon}
      \end{prooftree}
      \\[2ex]
      \begin{prooftree}
        \hypo{q_1 = q_2 \Rightarrow s_1 \coh^{q_1} s_2}
        \infer1{q_1 s_1 \coh q_2 s_2}
      \end{prooftree}
      &
      \begin{prooftree}
        \hypo{s_1 \coh s_2}
        \infer1{\underline{r} s_1 \coh^q \underline{r} s_2}
      \end{prooftree}
      &
      \begin{prooftree}
        \hypo{s_1 \coh^{qm} s_2}
        \infer1{\underline{m} s_1 \coh^q \underline{m} s_2}
      \end{prooftree}
      &
      \begin{prooftree}
        \hypo{n_1 = n_2 \Rightarrow s_1 \coh^q s_2}
        \infer1{n_1 s_1 \coh^{qm} n_2 s_2}
      \end{prooftree}
    \end{array}
  \]
\end{definition}
The empty play $\epsilon$ is coherent with all plays.
More generally,
two plays are coherent if,
whenever the environment actions are the same,
the corresponding system responses must be identical.
This effectively eliminates the non-deterministic choices from the system.

\begin{definition}[Deterministic Strategy]
  A strategy $\sigma : E \rightarrow F$ is \emph{deterministic}
  if:
  \[
    \forall s_1, s_2 \in \sigma, \quad s_1 \coh s_2
  \]
\end{definition}
Then for deterministic strategies $\sigma$ and $\tau$,
a simple refinement relation can be defined:
\[
  \sigma \le \tau \quad :\Leftrightarrow\quad \sigma \subseteq \tau
\]

\section{Composition}
\label{sec:strat:composition}

The previous section
illustrates how strategies are defined
as sets of plays.
While this provides a foundation for understanding strategy behavior,
real systems require building complex strategies from simpler components.
In this section,
I show that strategies can be defined
by composing smaller strategies,
enabling modular construction of system specifications.

\subsection{Layered Composition}

The layered
composition of
$\sigma : F \rightarrow G$ with
$\tau : E \rightarrow F$
allows the strategies to synchronize
over the signature $F$.
Their interaction over the intermediate signature is then hidden from
the composite strategy $\sigma \odot \tau : E \rightarrow G$.
Layered composition can be defined
at the level of individual plays.

\begin{definition}[Layered Composition of Strategies] \label{def:lcomp} %{{{
  The \emph{identity} strategy $\kw{id}_E : E \rightarrow E$
  is defined as:
  \[
    \kw{id}_E :=
    \bigl(
      \{\epsilon\} \cup
      \{ m \underline{m} \mid m \in E \} \cup
      \{ m \underline{m} n \underline{n} \mid m \in E, n \in \kw{ar}(m) \}
    \bigr)^*
    \,.
  \]
  In addition, two strategies
  $\sigma : F \rightarrow G$ and
  $\tau : E \rightarrow F$
  compose to yield a strategy
  $\sigma \odot \tau : E \rightarrow G$.
  Individual plays compose
  according to the relations:
  \begin{align*}
    {\odot} : P_{F,G} \times P_{E,F}
    &\rightarrow \mathcal{P}(P_{E,G}) &
    \epsilon \odot t &:= \{ \epsilon \} \\&&
    qs \odot t &:= \{ qw \mid w \in s \odot^q t \}
    \\
    {\odot}^q : P_{F,G}^q \times P_{E,F}
    &\rightarrow \mathcal{P}(P_{E,G}^q) &
    \underline{r} s \odot^q t &:=
    \{ \underline{r} w \mid w \in s \odot t \} \\&&
    \underline{m} s \odot^q t &:=
    \{ w \mid \exists t' \cdot t = mt' \wedge w \in s \odot^{qm} t' \}
    \\
    {\odot}^{qm} : P_{F,G}^{qm} \times P_{E,F}^m
    &\rightarrow \mathcal{P}(P_{E,G}^q) &
    s \odot^{qm} \underline{u} t &:=
    \{ \underline{u} w \mid w \in s \odot^{qmu} t \} \\&&
    s \odot^{qm} \underline{n} t &:=
    \{ w \mid \exists s' \cdot s = ns' \wedge w \in s' \odot^q t \}
    \\
    {\odot}^{qmu} : P_{F,G}^{qm} \times P_{E,F}^{mu}
    &\rightarrow \mathcal{P}(P_{E,G}^{qu}) &
    s \odot^{qmu} \epsilon &:= \{ \epsilon \} \\&&
    s \odot^{qmu} v t &:= \{ v w \mid w \in s \odot^{qm} t \}
  \end{align*}
  The \emph{layered composition} of $\sigma$ and $\tau$ can then be defined as
  $
  \sigma \odot \tau \: := \:
  \bigcup_{(s,t) \in \sigma \times \tau}
  s \odot t
  $.
\end{definition}

\begin{theorem}
  \label{thm:strat:comp}
  The composition $\odot$ is associative
  and admits the identity strategy $\kw{id}_E : E \rightarrow E$ as units.
  Thus,
  strategies form a category $\mathbf{Strat}$.
\end{theorem}

\subsection{Flat Composition}
In addition to layered composition,
strategies can also be combined side-by-side.
Specifically, two strategies
$\sigma_1 : E \rightarrow F_1$ and
$\sigma_2 : E \rightarrow F_2$
can be used to independently handle
the two components of an incoming effect signature $F_1 \oplus F_2$.
The strategy
$\sigma_1 \oplus \sigma_2 : E_1 \oplus E_2 \rightarrow F_1 \oplus F_2$
is straightforward and lets $\sigma_1$ and $\sigma_2$ operate independently.
When a question $q \in F_1$ is asked
in the left-hand side component of $F_1 \oplus F_2$,
it is used to activate $\sigma_1$ which executes
until the question is answered.
$\sigma_2$ handles the questions of $F_2$ in a similar way.

I now introduce
a composition operation $\oplus$ operating on the effect signatures themselves,
which will act on all higher-dimensional objects as well.

\begin{definition}[Sum of signatures]
  A family $(E_i)_{i \in I}$ of effect signatures can be combined into
  \[
    \bigoplus_{i \in I} E_i \, := \,
    \{ \iota_i(m) \mathbin: N \mid i \in I,\, (m \mathbin: N) \in E_i \}
    \,,
  \]
  which uses the set of operations $\iota_i(m) \in \sum_i E_i$
  and uses for each one the arity assigned to it
  in its signature of origin $E_i$.
  The binary case where $i \in \{1, 2\}$
  will be written as $E_1 \oplus E_2$.
\end{definition}

The signature $E \oplus F$ contains the combined questions of $E$ and $F$.
Each question retains the same set of answers.
Many of the signatures seen so far can be decomposed using $\oplus$.

\begin{example}[Per-file interfaces]
  Processes can be modeled as
  strategies of type $P : \mathcal{S} \rightarrow \mathcal{P}$,
  where the signature $\mathcal{S}$ contains questions for
  each file descriptor $i \in \mathbb{N}$.
  This signature can be decomposed as
  $
  \mathcal{S} \: = \: \bigoplus_{i \in \mathbb{N}} \: \mathcal{F}
  $, where $
  \mathcal{F} \: := \:
  \{ \kw{read}[n] : \Sigma^*, \kw{write}[s] : \mathbb{N} \mid n \in \mathbb{N}, s \in \Sigma^* \}
  $.
  Since the examples focus on standard input ($i=0$) and output ($i=1$),
  it can be simplified as
  \[\mathcal{S} := \mathcal{F} \oplus \mathcal{F}\,.\]
\end{example}

Additional strategies
can be defined in relation to $\oplus$, namely
\[
  \Delta_E : E \rightarrow E \oplus E \,,
  \qquad
  \gamma_{E,F} : E \oplus F \cong F \oplus E \,,
  \qquad
  \pi_1^{E,F} : E \oplus F \rightarrow E \,,
  \qquad
  \pi_2^{E,F} : E \oplus F \rightarrow F \,.
\]
The strategy $\Delta_E$ passes along
questions received in two independent copies of $E$
but consolidates them into a single copy.
The projections $\pi_i^{E,F}$
can be used to ``forget'' the unused summand
of the signature $E \oplus F$.
These constructions are useful
for composing processes in the next section.

\begin{definition}[Flat Composition]
  \label{def:strat:flat-comp}
  The strategy $\pi_i : E_1 \oplus E_2 \rightarrow E_i$
  can be defined as:
  \[
    \pi_i := \bigl(
      \{ \epsilon \} \cup
      \{ m \, \underline{\iota_i(m)} \mid m \in E_i \} \cup
      \{ m \, \underline{\iota_i(m)} \, n \, \underline{n} \mid
      m \in E_i, n \in \kw{ar}(m) \}
    \big)^*
    \,.
  \]
  Moreover,
  two strategies
  $\sigma_1 : E \rightarrow F_1$ and
  $\sigma_2 : E \rightarrow F_2$
  can be combined into
  $\langle \sigma_1, \sigma_2 \rangle : E \rightarrow F_1 \oplus F_2$.
  Individual plays combine as follows:
  \begin{align*}
    \langle qs_1 , s_2 \rangle &:=
    \{ \iota_1(q) \, w \mid w \in \langle s_1, s_2 \rangle^q_1 \} &
    \langle s_1 , qs_2 \rangle &:=
    \{ \iota_2(q) \, w \mid w \in \langle s_1, s_2 \rangle^q_2 \}
    \\
    \langle \underline{r} s_1, s_2 \rangle_1^q &:=
    \{ \underline{r} w \mid w \in \langle s_1, s_2 \rangle \} &
    \langle s_1, \underline{r} s_2 \rangle_2^q &:=
    \{ \underline{r} w \mid w \in \langle s_1, s_2 \rangle \}
    \\
    \langle \underline{m} s_1, s_2 \rangle_1^q &:=
    \{ \underline{m} w \mid w \in \langle s_1, s_2 \rangle^{qm}_1 \} &
    \langle s_1, \underline{m} s_2 \rangle_2^q &:=
    \{ \underline{m} w \mid w \in \langle s_1, s_2 \rangle^{qm}_2 \}
    \\
    \langle n s_1, s_2 \rangle^{qm}_1 &:=
    \{ n w \mid w \in \langle s_1, s_2 \rangle^q_1 \} &
    \langle s_1, n s_2 \rangle^{qm}_2 &:=
    \{ n w \mid w \in \langle s_1, s_2 \rangle^q_2 \}
  \end{align*}
  Then
  $\langle \sigma_1, \sigma_2 \rangle :=
  \bigcup_{(s_1, s_2) \in \sigma_1 \times \sigma_2}
  \langle s_1, s_2 \rangle$.
  In addition,
  for $\sigma_1 : E_1 \rightarrow F_1$ and
  $\sigma_2 : E_2 \rightarrow F_2$,
  the flat composition is defined as
  \[
    \sigma_1 \oplus \sigma_2 \: := \:
    \langle \sigma_1 \odot \pi_1, \, \sigma_2 \odot \pi_2 \rangle
    \: : \:
    E_1 \oplus E_2 \rightarrow F_1 \oplus F_2
    \,.
  \]
\end{definition}

\begin{theorem}[Properties of $\oplus$]
  The flat composition is compatible
  with the layered composition in the following sense:
  \[
    \begin{array}{r@{}l}
      (L_1 \odot L_2) \oplus (L_1' \odot L_2') & {} \equiv
      (L_1 \oplus L_1') \odot (L_2 \oplus L_2') \\
      \kw{id}_E \oplus \kw{id}_F & {} \equiv \kw{id}_{E \oplus F}
    \end{array}
  \]
  In other words, the $\oplus$ operator defines a bifunctor
  on the category of strategies:
  \[
    \oplus : \mathbf{Strat} \times \mathbf{Strat} \rightarrow \mathbf{Strat}
  \]
\end{theorem}

%Note that despite the notations we have used,
%flat composition does not constitute a cartesian product of strategies,
%because there are strategies of type
%$E \rightarrow F_1 \oplus F_2$
%which do not take the form $\langle \sigma_1, \sigma_2 \rangle$.
%For example,
%questions in $F_1$ may determine
%how later questions in $F_2$ are answered.

\subsection{Composing processes}
\label{sec:strat:compose-process}

Shell-like operators can be defined for composing processes.
Two processes $P, Q : \mathcal{S} \rightarrow \mathcal{P}$
can be combined into
$(P \mathbin\texttt{;} Q) : \mathcal{S} \rightarrow \mathcal{P}$.
To this end,
I define the scheduling component
$\kw{seq} : \mathcal{P} \oplus \mathcal{P} \rightarrow \mathcal{P}$
which invokes one process, then the other:
\[
  \kw{seq} \vDash
  \kw{run} \cdot (\underline{\kw{run}_1} \cdot n) \cdot (\underline{\kw{run}_2} \cdot m) \cdot m
\]
This component can be used to define:
\[
  P \mathbin\texttt{;} Q \: := \:
  \kw{seq} \odot (P \oplus Q)
  \odot (\mathcal{F} \oplus \gamma \oplus \mathcal{F})
  \odot (\Delta \oplus \Delta)
\]
The $P \oplus Q$ component
emits the system events with signature $(\mathcal{F} \oplus \mathcal{F}) \oplus (\mathcal{F} \oplus \mathcal{F})$.
The auxiliary component
$(\mathcal{F} \oplus \gamma \oplus \mathcal{F})$
and
$(\Delta \oplus \Delta)$
combine the events on standard input and output
into a unified effect signature
of the form $\mathcal{F} \oplus \mathcal{F}$,
which is used by a regular process.

The shell operators $\texttt{\&\&}$ and $\texttt{||}$ could likewise be modeled
by replacing $\kw{seq}$ with different scheduling policies.
Beyond sequencing,
\emph{data flow} can also be modeled
between processes.
To this end,
I introduce
$\kw{fifo} : \mathbf{0} \rightarrow \mathcal{F}$
representing a FIFO buffer.
Operationally,
$\kw{fifo}$ can accept writes
and serve reads
according to the following rules:
\[
  \kw{fifo} := \sigma_\epsilon
  \qquad
  \begin{prooftree}
    \hypo{w \in \sigma_{s_1 \cdot s_2}}
    \infer1{\kw{write}[s_2] \cdot \underline{\kw{len}(s_2)} \cdot w \in \sigma_{s_1}}
  \end{prooftree}
  \qquad
  \begin{prooftree}
    \hypo{s = s_1 \cdot s_2}
    \hypo{\kw{len}(s_1) = n}
    \hypo{w \in \sigma_s}
    \infer3{\kw{read}[n] \cdot \underline{s_1} \cdot w \in \sigma_s}
  \end{prooftree}
\]
This fifo buffer accepts write events
and accumulates the written data in order.
When a read event occurs with a specified amount,
it returns that number of characters
from the front of the buffer,
maintaining the first-in-first-out ordering
of the stored data.
For example, $\kw{fifo}$ has the following behavior:
{\small
  \[
    \kw{fifo} \: \vDash \:
    (\kw{write}[\texttt{"hello, "}] \rightarrowtail 7) \leadsto
    (\kw{write}[\texttt{"world!\textbackslash{}n"}] \rightarrowtail 7) \leadsto
    (\kw{read}[100] \rightarrowtail \texttt{"hello, world!\textbackslash{}n"})
  \]
}%
With $\kw{fifo}$ serving as an intermediate buffer,
the pipe operator can be defined:
\[
  P \mathbin\texttt{|} Q \: := \:
  \kw{seq} \odot (P \oplus Q)
  \odot (\mathcal{F} \oplus (\Delta \odot \kw{fifo}) \oplus \mathcal{F})
  \,.
\]
Intuitively, the pipe connects
the output of P to the input of Q via the buffer,
while the sequencing ensures that
the communication proceeds in an orderly fashion.
The structure of this composition is illustrated
in the following diagram.
\begin{center}
  \begin{tikzpicture}[
      box/.style = {draw, minimum width=10mm, minimum height=8mm, align=center},
      wire/.style = {thick},
      dot/.style  = {circle, fill, inner sep=1.3pt}
    ]

    % --- blocks
    \node[box]                                  (seq)  {seq};
    \node[box, right=10mm of seq, yshift= 8mm]  (P)    {P};
    \node[box, right=10mm of seq, yshift=-8mm]  (Q)    {Q};
    \node[box, right=35mm of seq]               (fifo) {fifo};
    \node[right=40mm of P, yshift= 2mm]         (Px)   {$\mathcal{F}$};
    \node[right=40mm of Q, yshift=-2mm]         (Qx)   {$\mathcal{F}$};
    \node[left=10mm of seq]                     (Sx)   {$\mathcal{P}$};
    \node[left=3mm of P]                        (Pl)   {$\mathcal{P}$};
    \node[left=3mm of Q]                        (Ql)   {$\mathcal{P}$};
    \node[right=5mm of P, yshift=-2mm]          (Pr)   {$\mathcal{F}$};
    \node[right=5mm of Q, yshift=+2mm]          (Qr)   {$\mathcal{F}$};

    % merge dot
    \node[dot, right=28mm of seq] (m) {};

    % convenient left/right x positions for long wires

    \coordinate (TopL) at ($(seq.north)+(0,8mm)$);
    \coordinate (TopR) at ($(fifo.north)+(0,8mm)$);
    \coordinate (BotL) at ($(seq.south)+(0,-8mm)$);
    \coordinate (BotR) at ($(fifo.south)+(0,-8mm)$);

    % --- internal wiring
    \draw[wire] (Sx) -- (seq.west);
    \draw[wire] (seq.east)+(0,2mm) to[out=0,in=180,looseness=1.2] (P.west);
    \draw[wire] (seq.east)+(0,-2mm) to[out=0,in=180,looseness=1.2] (Q.west);
    \draw[wire] (P.east)+(0,-1mm) to[out=0,in=120,looseness=1.2] (m);
    \draw[wire] (Q.east)+(0,+1mm) to[out=0,in=240,looseness=1.2] (m);
    \draw[wire] (m) -- (fifo.west);
    \draw[wire] (P.east)+(0,+2mm) -- (Px);
    \draw[wire] (Q.east)+(0,-2mm) -- (Qx);

    % --- dashed enclosure with label
    \node[
      fit=(seq)(P)(Q)(fifo)(TopL)(TopR)(BotL)(BotR),
      draw, dashed, rounded corners, inner sep=8mm,
      label={[anchor=north west]north west:{\small $P \mid Q$}}
    ] {};
  \end{tikzpicture}
\end{center}

Using this construction,
the relationship between the behaviors can be expressed
$\Gamma_\kw{secret}$ and $\Gamma_\kw{encode}$
to formulate a partial account of property (\ref{eqn:hellospec}).
% Specifically,
% we expect the behavior
% \[
%   \Gamma_{(\ref{eqn:hellospec})} \vDash \kw{run}
%   \rightarrowtail (\kw{write}_1[\texttt{"hello, world!\textbackslash{}n"}] \leadsto 14)
%   \rightarrowtail 0
% \]
% to admit the refinement square
% $
% \phi_{(\ref{eqn:hellospec})} :
% \Gamma_{(\ref{eqn:hellospec})} \le
% \Gamma_\kw{secret} \mathbin{\texttt{|}} \Gamma_\kw{decode}
% $.
\[
  \phi_{(\ref{eqn:hellospec})} :
  \Gamma_{(\ref{eqn:hellospec})} \subseteq
  \Gamma_\kw{secret} \mathbin{\texttt{|}} \Gamma_\kw{decode}
  \,.
\]

Note that since this model does not support concurrency,
the construction $P \mathbin{\texttt{|}} Q$ above
can only offer a sequential approximation of
the corresponding Unix shell operator.
The example shown in
\autoref{fig:readwritehello}
to illustrate the issues that come up
when the horizon of verification
is pushed beyond the boundary of a fixed language or model,
but providing a realistic account of Unix processes
remains beyond the scope of the present work.

\subsection{Spatial Composition}

The spatial composition equips strategies
with external states.
Inspired by the approach in \S\ref{sec:ox:lens},
the spatial composition is defined
as a special case of the tensor product.

Like the sum used by flat composition,
the tensor product is another well-known operation on effect signatures,
which expects the client to
\emph{simultaneously} ask a question in each component:
\[
  \bigotimes_{i \in I} E_i \, := \,
  \textstyle
  \big\{ \langle m_i \rangle_{i\in I} : \prod_{i \in I} N_i \mathrel{\big|}
  \forall i \mathbin. (m_i \mathbin: N_i) \in E_i \big\}
\]

\begin{definition}
  The tensor product
  $\sigma_1 \otimes \sigma_2 : E_1 \otimes E_2 \rightarrow F_1 \otimes F_2$
  of the strategies
  $\sigma_1 : E_1 \rightarrow F_1$ and
  $\sigma_2 : E_2 \rightarrow F_2$
  is defined by pairing up their plays pairwise
  when they exhibit a similar structure, on a move-by-move basis.
  The process can be described by the following rules:
  \begin{align*}
    \epsilon \otimes \epsilon & := \{ \epsilon \} \\
    q_1 s_1 \otimes q_2 s_2 & := \{ (q_1, q_2) s \mid s \in s_1 \otimes^{q_1, q_2} s_2 \}\\
    \underline{r}_1 s_1 \otimes^{q_1,q_2} \underline{r}_2 s_2 & := \{ \underline{(r_1, r_2)} s \mid s \in s_1 \otimes s_2 \} \\
    \epsilon \otimes^{q_1m_1,q_2m_2} \epsilon & := \{ \epsilon \} \\
    \underline{m}_1 s_1 \otimes^{q_1,q_2} \underline{m}_2 s_2 & := \{ \underline{(m_1, m_2)} s \mid s \in s_1 \otimes^{q_1m_1, q_2m_2} s_2 \} \\
    n_1s_1 \otimes^{q_1m_1,q_2m_2} n_2s_2 & := \{ (n_1, n_2) s \mid s \in s_1 \otimes^{q_1, q_2} s_2 \}
  \end{align*}
  The resulting strategy can be given as
  $\sigma_1 \otimes \sigma_2 :=
  \bigcup_{(s_1, s_2) \in \sigma_1 \times \sigma_2} s_1 \otimes s_2$.
\end{definition}

\begin{definition}
  \label{def:strat:spatial-comp}
  The spatial composition operator $\mathbin@$ is defined:
  \begin{itemize}
    \item for an effect signature $E$ and a set $U$,
      as $E \mathbin@ U := E \otimes [U]$;
    \item for a strategy $\sigma$ and a stateful lens $f$,
      as $\sigma \mathbin@ f := \sigma \otimes [f]$.
      % \item for the refinement conventions $\mathbf{R}$ and $\mathbf{S}$
      %   as $\mathbf{R} \mathbin@ \mathbf{S} := \mathbf{R} \otimes \mathbf{S}$.
  \end{itemize}
\end{definition}

\begin{theorem}[Properties of $\at$]
  The spatial composition is compatible
  with the layered composition in the following sense:
  \[
    \begin{array}{r@{}l}
      (L_1 \odot L_2) \mathbin@ (f \circ g) & {} \equiv
      (L_1 \mathbin@ f) \odot (L_2 \mathbin@ g) \\
      \kw{id}_A \mathbin@ \kw{id}_U & {} \equiv \kw{id}_{A \mathbin@ U}
    \end{array}
  \]
  In other words, the $\at$ operator defines a bifunctor
  on the category of strategies and stateful lenses:
  \[
    \at : \mathbf{Strat} \times \mathbf{Lens} \rightarrow \mathbf{Strat}
  \]
\end{theorem}

\section{Regular Closure}
\label{sec:strat:regular-closure}

The composition operations introduced in the previous section
provide powerful tools for building complex strategies.
However, unlike $\sigma_{\vec{q}}$ above,
many strategies of interest are stateless
in the sense that every incoming question
is handled in the same way
regardless of any previous history.
Following \citet{objsem},
such strategies are called \emph{regular}.

A common approach to
construct a regular strategy
is to define a \emph{single-use} strategy
that describes the behavior of one execution,
and then apply the regular closure operator
to repeat the execution
infinitely many times.
Plays in the single-use strategy are of the form
$ q \underline{m_1} n_1 \cdots \underline{m_k} n_k \underline{r} $
or prefixes thereof.

\begin{definition}[Single-use Strategy]
  A strategy $\sigma$ is \emph{single-use}
  when its plays are of the form
  $ q \underline{m_1} n_1 \cdots \underline{m_k} n_k \underline{r} $
  or prefixes thereof.
\end{definition}

Because processes do not retain state
from one execution to the next,
the strategies $\Gamma_\kw{secret}$ and $\Gamma_\kw{encode}$
are both single-use.
They can no longer respond to subsequent questions
after they have terminated.
The regular closure operator
extends the single-use strategy
by repeating itself

\begin{definition}[Regular Closure]
  Consider a strategy $\sigma : E \rightarrow F$.
  Given two plays $s, t \in P_{E,F}$
  the play $s \rhd t$ initially proceeds as $s$
  but goes on to proceed as $t$ if $s$ ends
  with $\epsilon \in P_{E,F}$
  when a question $q \in F$ is expected.
  Formally,
  I can define
  $\rhd^x : P_{E,F}^x \times P_{E,F} \rightarrow P_{E,F}^x$
  as follows:
  \begin{align*}
    qs \rhd t &:= q ( s \rhd^q t ) &
    \underline{m} s \rhd^q t &:= \underline{m} (s \rhd^{qm} t) &
    n s \rhd^{qm} t &:= n (s \rhd^q t) \\
    \epsilon \rhd t &:= t &
    \underline{r} s \rhd^q t &:= \underline{r} (s \rhd t) &
    \epsilon \rhd^{qm} t &:= \epsilon
  \end{align*}

  The \emph{regular closure} $\sigma^* : E \rightarrow F$
  allows the strategy $\sigma$
  to start over with each new incoming question:
  \[
    \epsilon \in \sigma^*
    \qquad \qquad
    s \in \sigma \wedge t \in \sigma^* \Rightarrow
    s \rhd t \in \sigma^*
  \]
\end{definition}

With the regular closure applied,
the process $\Gamma^*_\kw{secret}$ restarts
after each execution,
yielding an unbounded sequence of identical executions:
\begin{align*}
  \Gamma^*_\kw{secret}
  \:\vDash\:
  & \big(\kw{run}
    \sysstep (\kw{write}_1[\texttt{\textnormal{"uryyb, jbeyq!\textbackslash{}n"}}] \envstep 14)
  \sysstep 0\big) \\
  \:\envstep\:
  & \big(\kw{run}
    \sysstep (\kw{write}_1[\texttt{\textnormal{"uryyb, jbeyq!\textbackslash{}n"}}] \envstep 14)
  \sysstep 0\big) \\
  \envstep
  & \cdots
\end{align*}

Formally, I define a regular strategy
as the regular closure of a single-use strategy.
\begin{definition}[Regular Strategy]
  $\sigma$ is a \emph{regular strategy}
  when it is the regular closure $\sigma = \tau^*$
  of a single-use strategy $\tau$.
\end{definition}

\subsection{Properties of Regular Closure}

The regular closure enjoys several important properties.

\begin{theorem}
  \label{thm:regular-closure}
  The regular closure perserves determinism and refinement order:
  \[
    \begin{prooftree}
      \hypo{\sigma \ \kw{deterministic}}
      \hypo{\sigma \ \kw{single\text{-}use}}
      \infer2{\sigma^* \ \kw{deterministic}}
    \end{prooftree}
    \qquad
    \begin{prooftree}
      \hypo{\phi : \sigma \subseteq \tau}
      \infer1{\phi^* : \sigma^* \subseteq \tau^*}
    \end{prooftree}
    \,,
  \]
\end{theorem}

The determinism of $\sigma^*$
depends crucially on $\sigma$ being single-use.
If $\sigma$ were not single-use,
the repeated restarts could introduce non-deterministic behavior.
For instance,
consider the counter strategy $\sigma_\kw{inc}$
which is deterministic on its own,
but becomes non-deterministic
under the regular closure.
\[
  \sigma_\kw{inc} := \downarrow
  \kw{inc} \cdot \underline{0} \cdot \kw{inc} \cdot \underline{1}
\]
Particularly, the regular closure $\sigma_\kw{inc}^*$ is not deterministic
because it admits the following two plays
that violate the determinism:
\[
  \sigma_\kw{inc}^* \vDash \kw{inc} \cdot \underline{0} \cdot \kw{inc} \cdot \underline{1}
  \qquad
  \sigma_\kw{inc}^* \vDash \kw{inc} \cdot \underline{0} \cdot \kw{inc} \cdot \underline{0}
\]

Regular closure operator also interacts well
with other operators.
\begin{lemma}[Regular Closure Compatibility] \label{lem:regular-closure-comp}
  Let $\sigma$ be a single-use strategy,
  and $\tau$ a regular strategy.
  The regular closure operator
  is compatible with composition on the right by $\tau$,
  in the sense that
  \[
    \sigma^* \odot \tau = (\sigma \odot \tau)^*
    \,.
  \]
  Moreover,
  the regular closure operator
  commutes with state lifting:
  \[
    \sigma^* \at U = (\sigma \at U)^*
    \,.
  \]
  In other words,
  regular closure can be ``pushed through''
  both composition with a regular strategy
  and the addition of external state.
\end{lemma}
These commutativity properties are particularly useful
when CompCertO semantics are embedded.

\subsection{Embedding CompCertO semantics}

The language semantics and correctness properties
defined by the certified compiler CompCertO
can be used within this model.

CompCertO
uses a notion of \emph{open} transition system
to describe interactions across component boundaries.
These boundaries are specified using
language interfaces of the form
$A := \langle A^\que, A^\ans \rangle$,
which translate to effect signatures
$\llbracket A \rrbracket := \{ q : A^\ans \mid q \in A^\que \}$.

Recall that
a CompCertO \emph{transition system} $L : A \twoheadrightarrow B$
is a tuple $L = \langle S, {\rightarrow}, I, X, Y, F \rangle$
where executions take the form
\[
  q \mathrel{I} s_0 \rightarrow^*
  s_1 \mathrel{X} q_1 \leadsto
  r_1 \mathrel{Y^{s_1}} s_1' \rightarrow^*
  s_2 \mathrel{\cdots}
  s_n \mathrel{X} q_n \leadsto
  r_n \mathrel{Y^{s_n}} s_n' \rightarrow^*
  s_f \mathrel{F} r
  \,,
\]
corresponding to an interaction trace
$
q \rightarrowtail
(q_1 \leadsto r_1) \rightarrowtail
\cdots \rightarrowtail
(q_n \leadsto r_n) \rightarrowtail
r
$.

To describe the strategy associated with a CompCertO transition system,
I first formalize the set of plays generated by an internal state $s \in S$ as follows:
\[
  \begin{prooftree}
    \hypo{s \rightarrow^* s' \: X \: m \leadsto n \: Y^{s'} \: s''}
    \hypo{s'' \Vdash w}
    \infer2{s \Vdash \underline{m}nw}
  \end{prooftree}
  \qquad
  \begin{prooftree}
    \hypo{s \rightarrow^* s' \: F \: r}
    \infer1{s \Vdash \underline{r}}
  \end{prooftree}
\]
For an invocation on the transition system,
the play $qw$ will then result when $q \mathrel{I} s \vDash w$.
To handle subsequent invocations,
the process is iterated using the regular closure operator
defined in \S\ref{sec:strat:regular-closure}:
\[
  \llbracket L \rrbracket \: := \:
  \Big(\bigcup_{q \in B^\que} \{ qw \mid \exists s \cdot q \: I \: s \wedge s \Vdash w \} \Big)^*
  \,.
\]

Using \autoref{thm:regular-closure} and \autoref{lem:regular-closure-comp},
the following properties of the embedding hold:
\begin{theorem}[Embedding properties]
  \label{thm:strat:embedding}
  Given transition systems $L_1: B \twoheadrightarrow C$ and $L_2: A \twoheadrightarrow B$,
  \[
    \llbracket L_1 \rrbracket \odot \llbracket L_2 \rrbracket
    \subseteq \llbracket L_1 \odot L_2 \rrbracket
    \,.
  \]
  In other words,
  the embedding defines an oplax functor
  from the category of transition systems
  to the category of strategies:
  \[
    \llbracket - \rrbracket : \mathbf{LTS} \rightarrow \mathbf{Strat} \,.
  \]
\end{theorem}

\section{Loaders}
\label{sec:strat:loaders}

Having established the foundational theory of strategies and their composition,
I now turn to practical applications.
Verifying functionalities
of library code substantially benefits from
CompCertO's open semantics.
However,
the openness hinders reasoning
on the behavior of executables.
For the executables,
it is desirable to
model them
in terms of the \textit{process behavior};
the behaviors are self-contained,
and can be characterized
by the sequence of system calls
they perform.
To bridge the gap
between the open semantics of
the process behavior,
I introduce the notion of a \textit{loader}.

\subsection{Assembly Loader}

An assembly-level transition system
$L : \mathcal{A} \twoheadrightarrow \mathcal{A}$
is loaded via the loader:
\[
  \kw{load}_\mathcal{A}(L)
  : \mathcal{S} \rightarrow \mathcal{P}
  \::=\: \kw{entry}_\mathcal{A} \: \odot \: L
  \: \odot \: \kw{runtime}_\mathcal{A}
  \,.
\]

On one end,
the loader launches
the component as a process
by using the
$\kw{entry}_\mathcal{A} : \mathcal{A} \rightarrow \mathcal{P}$
to invoke its main function:
\[
  \kw{entry}_\mathcal{A} \:\vDash\:
  \kw{run} \rightarrowtail
  (\vec{rs_0}[\kw{PC}\mapsto \kw{main},
      \kw{RA} \mapsto \kw{null},
    \kw{RSP}\mapsto \kw{null}]@m_0 \leadsto
  \vec{rs}[\kw{RAX} \mapsto r]@m) \rightarrowtail r \,.
\]
The registers $\vec{rs_0}$ and the memory $m_0$ are
initialized that
the program counter $\kw{PC}$ holds a pointer value
that points to
the $\kw{main}$ function,
and the static variables
are properly initialized in the memory.
The return address $\kw{RA}$
and the stack pointer $\kw{RSP}$
are initialized to $\kw{null}$
according to CompCertO's simulation convention.
At the end,
the value stored in $\kw{RAX}$ is returned.

On the other end,
the $\kw{runtime} : \mathcal{S} \rightarrow \mathcal{A}$
acts as the conduit for runtime libraries
to interface the program with the operating system.
In this scenario,
the programs only use $\kw{read}$ and $\kw{write}$
functions from $\kw{unistd.h}$
to perform I/O operations.
Thus, I implement the minimalist runtime:

{\footnotesize
  \begin{align*}
    \kw{runtime}_\mathcal{A} & \:\vDash\:
    \vec{rs}[\kw{PC} \mapsto \kw{read},
      \kw{RDI} \mapsto 0,
      \kw{RSI} \mapsto b,
    \kw{RDX} \mapsto n]@m[b \mapsto unspecified] \\
    & \rightarrowtail (\kw{read}_0[n] \leadsto s)
    \rightarrowtail \vec{rs'}[\kw{RAX} \mapsto \kw{len}(s)]@m[b \mapsto s] \\
    \kw{runtime}_\mathcal{A} & \:\vDash\:
    \vec{rs}[\kw{PC} \mapsto \kw{write},
      \kw{RDI} \mapsto 1,
      \kw{RSI} \mapsto b,
    \kw{RDX} \mapsto n]@m[b \mapsto s] \\
    &  \rightarrowtail
    (\kw{write}_1[s[0:n]] \leadsto n')
    \rightarrowtail \vec{rs'}[\kw{RAX} \mapsto n']@m[b \mapsto s]
  \end{align*}
}

Following the x86 conventions,
arguments are passed via
the \kw{RDI}, \kw{RSI}, and \kw{RDX} registers.
The \kw{read} function
loads a sequence of bytes
from the standard input,
stores them into the memory
where the pointer value $b$ points to,
and returns the length of the byte sequence.
Conversely, the \kw{write} function
writes the first $n$ bytes
of the byte sequence $s$
to the standard output,
and the return value $n'$ indicates
the number of bytes that are successfully written.

With the assembly loader,
I then formally formulate the property (\ref{eqn:hellospec})
as follows:
\[
  \Gamma_\kw{(\ref{eqn:hellospec})} \subseteq
  \kw{load}_\mathcal{A}(\llbracket \kw{Asm}(\kw{secret.s} + \kw{rot13.s}) \rrbracket)
  \mid
  \kw{load}_\mathcal{A}(\llbracket \kw{Asm}(\kw{decode.s} + \kw{rot13.s}) \rrbracket)
  \,.
\]

\subsection{Clight Loader}

Reasoning about the process behavior
directly at the level of assembly programs
is intricate because of
the large abstraction gap between
the strategy-level specifications
and the assembly semantics.
Therefore,
I also introduce a $\kw{Clight}$ loader
to divide the proof
into manageable pieces.

\kw{Clight} programs
are not directly loaded as processes.
That said,
loaded \kw{Clight} programs
serves as an intermediate
verification step
to simplify the proof.
Similar to the assembly loader,
the \kw{Clight} loader is defined
using the auxiliary strategies
$\kw{entry}_\mathcal{C}$ and
$\kw{runtime}_\mathcal{C}$:
{\small
  \begin{align*}
    & \kw{entry}_\mathcal{C} \:\vDash\:
    \kw{run} \rightarrowtail
    (\kw{main}()@m_0 \leadsto r@m) \rightarrowtail r \\
    & \kw{runtime}_\mathcal{C} \:\vDash\:
    \kw{read}(0, b, n)@m[b \mapsto unspecified]
    \rightarrowtail (\kw{read}_0[n] \leadsto s)
    \rightarrowtail \kw{len}(s)@m[b \mapsto s] \\
    & \kw{runtime}_\mathcal{C} \:\vDash\:
    \kw{write}(1, b, n)@m[b \mapsto s]
    \rightarrowtail
    (\kw{write}_1[s[0:n]] \leadsto n')
    \rightarrowtail n'@m[b \mapsto s]\\
    & \kw{load}_\mathcal{C}(L)
    \::=\: \kw{entry}_\mathcal{C} \: \odot \: L
    \: \odot \: \kw{runtime}_\mathcal{C}
\end{align*}}%

\begin{theorem}[Loader simulation]
  \label{thm:strat:loader-simulation}
  The events in
  $\kw{entry}_\mathcal{C}$ and $\kw{runtime}_\mathcal{C}$
  have one-to-one correspondence
  with the events in the assembly loader
  modulo
  the simulation convention $\mathbb{C}$.
  Thus, the loaders
  transport
  the $\mathbb{C}$-related CompCertO simulations
  into
  refinements between process behaviors
  in the following sense:
  \[
    \begin{prooftree}
      \hypo{\phi: L_1 \le_{\mathbb{C} \twoheadrightarrow \mathbb{C}} L_2}
      \infer1{\phi^\ell: \kw{load}_\mathcal{C}(\llbracket L_1 \rrbracket) \subseteq
      \kw{load}_\mathcal{A}(\llbracket L_2 \rrbracket)}
    \end{prooftree}
  \]
\end{theorem}

\subsection{Complete Proof}
\label{sec:strat:complete-proof}
Revisiting
the task formalized as property (\ref{eqn:hellospec}),
I have shown
the high level specification
$\Gamma_{(\ref{eqn:hellospec})}$
is implemented by
piping the strategies
$\Gamma_\kw{secret}$ and $\Gamma_\kw{decode}$
in \S\ref{sec:strat:compose-process}.
I now proceed to
prove the strategies
are respectively implemented
by the actual programs
by proving
the correctness of the \kw{Clight} programs,
and then utilizing the CompCertO's compiler correctness
to transport the refinements
into the assembly level.

To utilize the power of
transitive reasoning,
I first define the
program-level specifications
$\Sigma_\kw{secret}$ and $\Sigma_\kw{decode}$
that are closer
to the \kw{Clight} programs behavior-wise.
{\small
  \begin{align*}
    \Sigma_\kw{secret} & \:\vDash\: &&
    \kw{main()}@m[b \mapsto \texttt{"hello, world!\textbackslash{}n"}] \\
    & \rightarrowtail &&
    (\kw{write}(1, b, 14)@m [b \mapsto \texttt{"uryyb, jbeyq!\textbackslash{}n"}]
    \leadsto 14@m[b \mapsto \texttt{"uryyb, jbeyq!\textbackslash{}n"}]) \\
    & \rightarrowtail && 0@m[b \mapsto deallocated] \\
    \Sigma_\kw{decode} & \:\vDash\ &&
    \kw{main()}@m \\
    & \rightarrowtail && ( \kw{read}(0, b, 100)@m[b \mapsto unspecified]
    \leadsto \kw{len}(s)@m[b \mapsto s]) \\
    & \rightarrowtail && (\kw{write}(1, b, \kw{len}(s))@m[b\mapsto \kw{rot13}(s)]
    \leadsto n@m[b\mapsto \kw{rot13}(s)]) \\
    & \rightarrowtail && 0@m[b \mapsto deallocated]
    \,,
\end{align*}}
The overall proof is structured as follows:
\begin{align}
  \Gamma_\kw{secret}
  & \subseteq \kw{load}_\C(\Sigma_\kw{secret}) \nonumber\\
  & \subseteq \kw{load}_\C(\llbracket L_\kw{secret} \rrbracket
  \odot \llbracket \kw{Clight}(\kw{rot13.c}) \rrbracket) \nonumber\\
  & \subseteq \kw{load}_\C(\llbracket L_\kw{secret}
  \odot \kw{Clight}(\kw{rot13.c}) \rrbracket) \tag{by \autoref{thm:strat:embedding}}\\
  & \subseteq \kw{load}_\mathcal{A}(\llbracket \kw{Asm}(\kw{secret.s})
  \odot \kw{Asm}(\kw{rot13.s}) \rrbracket)
  \tag{by \autoref{thm:strat:loader-simulation}}
  \\
  & \subseteq \kw{load}_\mathcal{A}(\llbracket \kw{Asm}(\kw{secret.s} + \kw{rot13.s}) \rrbracket)
  \tag{by CompCertO's linking property}
  \\
  \Gamma_\kw{decode}
  & \subseteq \kw{load}_\C(\Sigma_\kw{decode}) \nonumber\\
  & \subseteq \kw{load}_\C(\llbracket \kw{Clight}(\kw{decode.c}) \rrbracket
  \odot \llbracket \kw{Clight}(\kw{rot13.c}) \rrbracket) \nonumber\\
  & \subseteq \kw{load}_\C(\llbracket \kw{Clight}(\kw{decode.c})
  \odot \kw{Clight}(\kw{rot13.c}) \rrbracket) \tag{by \autoref{thm:strat:embedding}}\\
  & \subseteq \kw{load}_\mathcal{A}(\llbracket \kw{Asm}(\kw{decode.s})
  \odot \kw{Asm}(\kw{rot13.s}) \rrbracket) \tag{by \autoref{thm:strat:loader-simulation}}\\
  & \subseteq \kw{load}_\mathcal{A}(\llbracket \kw{Asm}(\kw{decode.s} + \kw{rot13.s}) \rrbracket)
  \tag{by CompCertO's linking property}\\
  \Gamma_{(\ref{eqn:hellospec})}
  & \subseteq \Gamma_\kw{secret} \mid \Gamma_\kw{decode} \nonumber\\
  & \subseteq \kw{load}_\mathcal{A}(\kw{Asm}(\kw{secret.s} + \kw{rot13.s}))
  \mid \kw{load}_\mathcal{A}(\kw{Asm}(\kw{decode.s} + \kw{rot13.s}))\nonumber
\end{align}
Following the compositional approach,
the remaining proof obligations reduce to
verifying the correctness of individual components:
\begin{gather*}
  \Gamma_\kw{decode} \subseteq \kw{load}_\mathcal{C}(\Sigma_\kw{decode})
  \qquad
  \Sigma_\kw{decode} \subseteq
  \llbracket \kw{Clight}(\kw{decode.c}) \rrbracket \odot \llbracket \kw{Clight}(\kw{rot13.c}) \rrbracket
  \\
  \Gamma_\kw{secret} \subseteq \kw{load}_\mathcal{C}(\Sigma_\kw{secret})
  \qquad
  \Sigma_\kw{secret} \subseteq
  \llbracket L_\kw{secret} \rrbracket \odot \llbracket \kw{Clight}(\kw{rot13.c}) \rrbracket
  \\
  L_\kw{secret} \le_{\mathbb{C} \twoheadrightarrow \mathbb{C}} \kw{Asm}(\kw{secret.s})
\end{gather*}
Most of these obligations are straightforward to establish,
relying on previously defined semantics and component interfaces.
Among these,
the last one is a CompCertO simulation
directly proved between
a C-level transition system
and its corresponding assembly-level transition system,
and such proofs are discussed in \autoref{sec:bg:simulation-between-c}.

% and prove
% they meet
% the strategy-level specifications
% via the loader:
% \[
%   \phi_\kw{decode}: \Gamma_\kw{decode} \subseteq \kw{load}_\mathcal{C}(\Sigma_\kw{decode})
%   \qquad
%   \phi_\kw{secret}: \Gamma_\kw{secret} \subseteq \kw{load}_\mathcal{C}(\Sigma_\kw{secret})
%   \,.
% \]%
% Then, the rest of the proof
% only involves the CompCertO semantics.
% In particular,
% the following properties
% state that
% the programs correctly implement
% their corresponding specifications:
% \begin{gather*}
%   \pi_\kw{secret}: \Sigma_\kw{secret} \le L_\kw{secret} \oplus \kw{Clight}(\kw{rot13.c})
%   \qquad
%   \pi'_\kw{secret}: L_\kw{secret} \le_{\mathbb{C} \rightarrow \mathbb{C}} \kw{Asm}(\kw{secret.s})
%   \\
%   \pi_\kw{decode}: \Sigma_\kw{decode} \le \kw{Clight}(\kw{decode.c}) \oplus \kw{Clight}(\kw{rot13.c})
%   \,.
% \end{gather*}
% where $L_\kw{secret}$
% is a transition system
% defined in terms of the $\mathcal{C}$ language interface
% that captures the behavior of
% the assembly program $\kw{secret.s}$.
% Combining the above simulations
% with CompCertO's compiler correctness,
% we obtain:

% \todo{here it should be simulation instead of refinement. maybe we can use prooftree here}
% \begin{align*}
%   \psi_\kw{secret} := \pi_\kw{secret} \fatsemi
%   (\pi'_\kw{secret} \oplus \phi^\kw{cc}_\kw{rot13})
%   \fatsemi \ell
%   & \quad :\quad \Sigma_\kw{secret} \le_{\mathbb{C} \rightarrow \mathbb{C}}
%   \kw{Asm}(\kw{secret.s} + \kw{rot13.s}) \\
%   \psi_\kw{decode} := \pi_\kw{decode} \fatsemi
%   (\phi^\kw{cc}_\kw{decode} \oplus \phi^\kw{cc}_\kw{rot13})
%   \fatsemi \ell
%   & \quad :\quad \Sigma_\kw{decode} \le_{\mathbb{C} \rightarrow \mathbb{C}}
%   \kw{Asm}(\kw{decode.s} + \kw{rot13.s})
%   \,.
% \end{align*}
% Note the $\oplus$ operator here
% is CompCertO's linking operator,
% which should not be confused
% with the flat composition
% on strategies.
% Eventually,
% the property (\ref{eqn:hellospec})
% is witnessed by the following proof:
% \[
%   \phi_\kw{(\ref{eqn:hellospec})} \fatsemi
%   (\phi_\kw{secret} \fatsemi \psi_\kw{secret}^\ell \mid
%   \phi_\kw{decode} \fatsemi \psi_\kw{decode}^\ell) \: : \:
%   \Gamma_\kw{(\ref{eqn:hellospec})} \le \kw{load}_\mathcal{A}(\kw{secret.s} + \kw{rot13.s})
%   \mid \kw{load}_\mathcal{A}(\kw{decode.s} + \kw{rot13.s})
% \]

% The definitions of
% the $\kw{Clight}$ loader and
% the $\kw{Clight}$ level specifications,
% and the detailed proof
% of directly proving simulation
% between $L_\kw{secret}$ and $\kw{Asm}(\kw{secret.s})$
% can be found in later sections.

% The program-level specification
% implements the strategy
% in the following sense:
% {\small
%   \[
%     \begin{prooftree}
%       \hypo{\phi_\kw{decode}: \Gamma_\kw{decode} \le \kw{load}_\mathcal{C}(\Sigma_\kw{decode})}
%       \infer[no rule]1
%       {\phi_\kw{secret}: \Gamma_\kw{secret} \le \kw{load}_\mathcal{C}(\Sigma_\kw{secret})}
%     \end{prooftree}
%     \quad
%     \begin{array}{c}
%       \begin{tikzcd}[row sep=2.5ex, column sep=4ex]
%         \mathcal{S} \ar[dd, equal]
%         \ar[rrr, "\Gamma_\kw{secret}"]
%         &&&
%         \mathcal{P} \ar[dd, equal] \\
%         & &|[xshift=-4ex, overlay]| \phi_\kw{secret} & \\
%         \mathcal{S} \ar[r, "\kw{entry}"'] & \mathcal{C} \ar[r, "\Sigma_\kw{secret}"'] & \mathcal{C}\ar[r, "\kw{runtime}"']  & \mathcal{P}
%       \end{tikzcd}
%     \end{array}
%     \quad
%     \begin{array}{c}
%       \begin{tikzcd}[row sep=0.5ex, column sep=1ex]
%         \mathcal{S} \ar[dd, leftrightarrow, "\kw{runtime}_*"']
%         \ar[rr, "\Gamma_\kw{secret}"] &&
%         \mathcal{P} \ar[dd, leftrightarrow, "\kw{entry}^*"] \\
%         & \phi_\kw{secret} & \\
%         \mathcal{C} \ar[rr, "\Sigma_\kw{secret}"'] &&
%         \mathcal{C}
%       \end{tikzcd}
%     \end{array}
% \]}%
% It is worth mentioning that
% the components $\kw{entry}_\mathcal{C}$ and $\kw{runtime}_\mathcal{C}$
% can also be viewed as simulation conventions
% that translates signatures $\mathcal{S}$ and $\mathcal{P}$
% into the $\mathcal{C}$ language interface,
% as illustrated in the diagrams above.

\section{Summary}

This chapter has introduced the strategy model,
a semantic framework grounded in game semantics.
The model occupies a ``sweet spot''
between traditional denotational semantics
and operational semantics:
it retains the compositional clarity
of the former
while accommodating the step-by-step reasoning
of the latter.
In this sense,
the strategy model can be seen
as a hybrid approach
that combines the strengths of both traditions.

Within this model,
I developed a rich collection of compositional structures
that support the modular development and reasoning of systems.
To demonstrate the expressiveness
of the approach,
I applied these constructions
to a nontrivial verification task
involving complex external interactions,
ultimately delivering an end-to-end verification result%
---from the high-level specification
all the way down to the executed assembly programs.

At the same time,
this chapter highlighted a limitation of the current approach:
the notion of refinement developed here
remains intentionally simple.
It ensures basic correctness properties
but does not yet incorporate mechanisms for abstraction.
This motivates the developments in the next chapter,
where I will introduce
a more sophisticated treatment of refinement
based on refinement conventions.
These conventions address abstraction explicitly
and allow the strategy model
to handle both system nondeterminism and environmental nondeterminism
within a unified notion of play.


\chapter{Strategy Refinement}
\label{ch:rc}

The notion of simple refinement cannot handle
components at different levels of abstraction.
To address this challenge,
the notion of simulation convention used in CompCertO
is adapted to the setting of effect signatures and game semantics.
A simulation convention connects
the ways an interface is viewed at different levels of abstraction.
In CompCertO,
the compiler's correctness theorem involves
a simulation convention
$\mathbb{C} : \mathcal{C} \twoheadleftrightarrow \mathcal{A}$,
which is used to express the way in which
C-level function calls ($\mathcal{C}$) are encoded
as assembly-level interactions ($\mathcal{A}$).

Building on this idea,
a richer notion of \emph{refinement convention} between effect signatures is defined.
Then a refinement property
$\phi : L_1 \le_{\mathbf{R} \rightarrow \mathbf{S}} L_2$
between
$L_1 : E_1 \rightarrow F_1$ and
$L_2 : E_2 \rightarrow F_2$
is parameterized by two simulation conventions
$\mathbf{R} : E_1 \leftrightarrow E_2$ and
$\mathbf{S} : F_1 \leftrightarrow F_2$.
The corresponding refinement property assumes that incoming
source- and target-level questions in $F_1$ and $F_2$
will be related according to the convention $\mathbf{S}$,
and guarantees that outgoing questions in $E_1$ and $E_2$
will be related according to $\mathbf{R}$.
Conversely, it assumes that
the environment's answers in $E$
will be related according to $\mathbf{R}$
and guarantees that the components' answers in $F$
will be related according to $\mathbf{S}$.

%The resulting compositional structure is shown in Fig.~\ref{fig:hvcomp}.
%While program components compose \emph{horizontally} through linking ($\odot$),
%refinement conventions compose \emph{vertically}
%in the manner of simulation relations ($\vcomp$).
%Refinement proofs are compatible with both $\odot$ and $\vcomp$
%so that they constitute a two-dimensional notion.
%We can also define the identity convention
%$\idsc_E : E \leftrightarrow E$;
%when $\mathbf{R}$ and $\mathbf{S}$ are both $\idsc$,
%refinement squares reduce to simple refinements.

% \begin{example}[Semantics preservation of CompCert]
%   \label{ex:bq-proof}
%   We will see in \S\ref{sec:app}
%   that the correctness proof of CompCertO
%   can be put in the form of a refinement square:
%   \[
%     \kw{CompCert}(p) = p'
%     \quad \Longrightarrow \quad
%     \phi^\kw{cc}_p :
%     \kw{Clight}(p) \le_{\mathbb{C} \twoheadrightarrow \mathbb{C}} \kw{Asm}(p')
%   \]
%   where %the refinement convention
%   $\mathbb{C} : \mathcal{C} \leftrightarrow \mathcal{A}$
%   captures the calling convention used %by the compiler
%   to represent C calls at the level of assembly.
% \end{example}

%DimSum follows the approach used in the refinement calculus
%and uses dual nondeterminism to express similar
%abstraction relationship as a dynamic ``translation''
%between interacting components which use
%incompatible representations.
%Our own approach is closer to the one used in CompCertO
%but the notions of \emph{companion} and \emph{conjoint}
%introduced in \S{}X.Y %XXX
%suggest ways in which these two approaches could be unified.

%}}}

\begin{table}
  \centering
  \begin{tabular}{ll}
    \toprule
    Notation & Description\\
    \midrule
    $\mathbf{R} : E \leftrightarrow F$ & Refinement Convention\\
    $(m_1, m_2)(n_1, n_2)\backslash \mathbf{R}$ & Residual Refinement Convention after $(m_1, m_2)(n_1, n_2)$\\
    $\sigma \le_{\mathbf{R} \rightarrow \mathbf{S}} \tau$ & Refinement Square\\
    $\mathbf{R} \fatsemi \mathbf{S}$ & Vertical Composition of Refinement Conventions\\
    $\mathbf{R} \oplus \mathbf{S}$ & Flat Composition of Refinement Conventions\\
    $\mathbf{R} \at \mathbf{S}$ & Spatial Composition of Refinement Conventions\\
    \bottomrule
  \end{tabular}
  \caption{Summary of notations}
  \label{tab:rc:notations}
\end{table}

Section~\ref{sec:rc:overview} provides
a high-level overview of refinement squares,
introducing their central ideas in an intuitive manner.
Section~\ref{sec:rc:refconv}
then gives the formal definition of refinement conventions,
illustrates them with examples,
and establishes their key properties.
Section~\ref{sec:rc:refsq} defines refinement squares,
demonstrates how they capture the refinement property of strategies,
and explains how they integrate with the simulation framework of CompCertO.
Finally, Section~\ref{sec:rc:cal}
presents an instance of CAL formulated
within the strategy model.

For reference, Table~\ref{tab:rc:notations} summarizes the notations for refinement conventions and refinement squares used throughout this chapter.

\section{Overview}
\label{sec:rc:overview}
The inclusion order induces a simple notion of strategy refinement.
For example,
consider the strategies
$\sigma \subseteq \tau : E \rightarrow \{ {*} : \varnothing \}$,
where they only exhibit meaningful moves
to interact with the handlers
via the signature $E$.
Ignoring the initial move $*$,
plays of $\sigma$ and $\tau$ take the form
$
\underline{m_1} n_1 \underline{m_2} n_2 \cdots \underline{m_k}
$.
Operationally,
inclusion induces the following coinductive simulation property,
where $\underline{m}n \backslash \sigma$ is written
for the residual strategy $\{ s \mid \underline{m}ns \in \sigma \}$:
\begin{equation} \label{eqn:sim}
  \begin{array}{r@{\:}l}
    \sigma \le \tau :\Leftrightarrow {} &
    \forall m \cdot
    \underline{m} \in \sigma \Rightarrow
    \underline{m} \in \tau \wedge {} \\ &
    \forall n \cdot
    (\underline{m} n \backslash \sigma) \le
    (\underline{m} n \backslash \tau)
    \,.
  \end{array}
  \qquad
  \begin{tikzcd}
    \sigma \ar[r,dash] \ar[d, dash, "\le"'] &
    m \ar[r,dotted,dash] \ar[d,equal] &
    n \ar[r] \ar[d,equal] &
    (\underline{m} n \backslash \sigma) \ar[d, dash, dashed, "\le"]
    \\
    \tau \ar[r,dash, dashed] &
    m \ar[r,dotted,dash] &
    n \ar[r,dashed] &
    (\underline{m} n \backslash \tau)
  \end{tikzcd}
\end{equation}
In other words,
any behavior prescribed by the specification $\sigma$
must be mirrored by the refinement~$\tau$.

Refinement conventions and refinement squares
generalize this notion of refinement
to cover situations where the source $\sigma$ and target $\tau$
differ in their interactions with the environment.

Building on the example above, suppose
$\sigma : E_1 \rightarrow \{* : \varnothing\}$ and
$\tau : E_2 \rightarrow \{* : \varnothing\}$
now differ in the type of their outgoing interactions.
To relate them,
a notion of \emph{refinement convention}
$\mathbf{R} : E_1 \leftrightarrow E_2$
will be defined, establishing a correspondence between
the questions and answers of $E_1$ and $E_2$.
A~refinement \emph{up to} $\mathbf{R}$,
written in this case
$\sigma \le_{\mathbf{R} \rightarrow \{*:\varnothing\}} \tau$,
will correspond to the property
\begin{equation} \label{eqn:simupto}
  \begin{array}{l}
    \forall m_1 \cdot \underline{m_1} \in \sigma \Rightarrow {}
    %\\[0.66ex]
    \exists m_2 \cdot \underline{m_2} \in \tau \:\wedge\:
    m_1 \mathrel{\mathbf{R}^\circ} m_2 \:\wedge\: {}
    \\[0.66ex]
    \forall \, n_1 \, n_2 \cdot \:
    n_1 \mathrel{\mathbf{R}^\bullet_{m_1,m_2}} n_2 \Rightarrow
    \\[0.66ex] \quad
    \bigl(\underline{m_1} n_1 \backslash \sigma \bigr)
    %\le_{(m_1,m_2)(n_1,n_2) \backslash R \twoheadrightarrow \{*:\varnothing\}}
    \le_{\mathbf{R}_{m_1,m_2}^{n_1,n_2} \rightarrow \{*:\varnothing\}}
    \bigl(\underline{m_2} n_2 \backslash \tau \bigr)
  \end{array}
  \hspace{-1ex} %\quad
  \begin{tikzcd}
    \sigma \ar[r,dash] \ar[d, dash, "\le_\mathbf{R}"'] &
    m_1 \ar[r,dotted,dash] \ar[d,dash,dashed,"\mathbf{R}^\circ"'] &
    n_1 \ar[r] \ar[d,dash, "\mathbf{R}^\bullet_{m_1m_2}"'] &
    \bigl(\underline{m_1} n_1 \backslash \sigma \bigr)
    \ar[d, dash, dashed, "\le_{\mathbf{R}_{m_1m_2}^{n_1n_2}}"']
    \\
    \tau \ar[r,dashed,dash] &
    m_2 \ar[r,dotted,dash] &
    n_2 \ar[r,dashed] &
    \bigl(\underline{m_2} n_2 \backslash \tau \bigr)
  \end{tikzcd}
\end{equation}
Here,
the refinement convention provides a relation
$\mathbf{R}^\circ \subseteq E_1 \times E_2$
between the questions of $E_1$ and the questions of $E_2$;
furthermore, for related questions $m_1 \mathrel{\mathbf{R}^\circ} m_2$
the refinement convention provides a relation on answers
$\mathbf{R}^\bullet_{m_1,m_2} \subseteq \kw{ar}(m_1) \times \kw{ar}(m_2)$
and an updated refinement convention
$\mathbf{R}_{m_1m_2}^{n_1n_2}$ to be used for the next question
whenever the answers $n_1 \mathrel{\mathbf{R}^\bullet_{m_1m_2}} n_2$
are received.

As this example illustrates,
one source of complexity is
the \emph{alternating} character of (\ref{eqn:simupto}).
The strategies $\sigma$ and $\tau$
primarily exhibit moves to their handlers
as they make questions and receive answers from the environment.
While the \emph{client} is free to choose
matching questions $m_1$ and $m_2$,
it must be ready to accept for every answer $n_1$
any related $n_2$ which the handler could return.
In other words,
the kind of data abstraction realized by refinement conventions
inherently involves nondeterministic choices
from both the component and its environment.
In the literature on dual nondeterminism,
these are referred to as \emph{demonic} and \emph{angelic} choices~\citep{dndf}.
The refinement relation
$\le_\mathbf{R}$
exhibits opposite variance with respect to its two components:
it becomes larger
when $\mathbf{R}^\circ$ relates more questions,
but smaller when $\mathbf{R}^\bullet$
introduces additional constraints on answers.

This variance pattern is further complicated by the fact that
strategies simultaneously play
dual roles as both client and handler
on their outgoing and incoming interfaces respectively.
Consequently,
general refinement squares
must accommodate two different refinement conventions
with opposite variances.
From the handler's perspective,
this duality manifests as a complementary requirement:
when interacting with clients,
the handler must be robust enough
to handle all possible choices of matching questions
while retaining the flexibility to choose any valid answer.

\section{Refinement Conventions}
\label{sec:rc:refconv}

The construction of refinement conventions
follows the same general spirit as that of strategies.
However,
to address the challenges outlined above,
an essential technical innovation is required.
In particular,
to accommodate
the alternation between client and handler choices
inherent in refinement conventions,
the standard prefix ordering on plays must be extended.

\begin{definition} \label{def:refconv}
  \emph{Refinement conventions} of type $\mathbf{R} : E \leftrightarrow F$
  are constructed using plays of the form
  \[
    s \in P_{E \leftrightarrow F} \: ::= \:
    (m_1, m_2) \bot \: \mid \:
    (m_1, m_2) (n_1, n_2) \, s \: \mid \:
    (m_1, m_2) (n_1, n_2) \top
    \quad
    \left(
      \begin{array}{c@{\:}c}
        m_1 \in E \,, & n_1 \in \kw{ar}(m_1) \\
        m_2 \in F \,, & n_2 \in \kw{ar}(m_2)
      \end{array}
    \right)
    \,.
  \]
  As suggested by the notation,
  the plays are ordered by the smallest relation $\preceq$
  such that
  \[
    s_1 \preceq s_2 \:\Longrightarrow\:
    (m_1, m_2) \bot \:\preceq\:
    (m_1, m_2) (n_1, n_2) s_1 \:\preceq\:
    (m_1, m_2) (n_1, n_2) s_2 \:\preceq\:
    (m_1, m_2) (n_1, n_2) \top
    \,.
  \]
  Then refinement conventions are elements of
  \[
    S_{E \leftrightarrow F} \: := \:
    \mathcal{D} \big( P_{E \leftrightarrow F}, {\preceq} \big) \: = \:
    \{ \mathbf{R} \subseteq P_{E \leftrightarrow F} \mid
      \forall s \, t \cdot s \preceq t \wedge t \in \mathbf{R} \Rightarrow
    s \in \mathbf{R} \}
    \,.
  \]
\end{definition}

The plays of $P_{E \leftrightarrow F}$, interpreted as follows,
allow more simulations to succeed as
more and larger plays are added to the convention:
\begin{itemize}
  \item The play $(m_1, m_2)\bot$
    allows the questions $m_1$ and $m_2$
    to be related by $\mathbf{R}^\circ$.
    By default,
    \emph{all possible pairs of answers} $(n_1, n_2)$
    are permitted by $\mathbf{R}^\bullet_{m_1m_2}$.
    However, no questions are allowed beyond that point
    until plays of the following kind are added to the refinement convention.
  \item The play $(m_1, m_2)(n_1,n_2) s$
    extends the ``next'' convention $\mathbf{R}^{n_1,n_2}_{m_1,m_2}$
    with the play $s$.
    Importantly, it does \emph{not} modify
    the ``answers'' relation $\mathbf{R}^\circ$.
    As explained above, the pair $(n_1, n_2) \in \mathbf{R}^\circ_{m_1m_2}$
    was already---and remains---permitted.
    However,
  \item the play $(m_1,m_2)(n_1,n_2) \top$ \emph{disallows} the pair
    $(n_1, n_2) \notin \mathbf{R}^\circ_{m_1m_2}$ as related answers.
    Since this restricts the \emph{handler},
    simulations between client computations become easier to prove,
    which is why plays of the form $(m_1,m_2)(n_1,n_2)\top$ are the ``largest''.
\end{itemize}
Based on this interpretation,
the components $R^\circ$, $R^\bullet$
and $R_{m_1m_2}^{n_1n_2}$ could be defined as follows:
\begin{gather*}
  m_1 \mathrel{\mathbf{R}^\circ} m_2 \::\Leftrightarrow\:
  (m_1,m_2)\bot \in \mathbf{R}
  \,, \qquad
  n_1 \mathrel{\mathbf{R}^\bullet_{m_1m_2}} n_2 \::\Leftrightarrow\:
  (m_1,m_2)(n_1,n_2)\top \notin \mathbf{R}
  \,, \\[0.5ex]
  \mathbf{R}_{m_1m_2}^{n_1n_2} \: := \:
  (m_1,m_2)(n_1,n_2) \backslash \mathbf{R} \: = \:
  \{ s \mid (m_1,m_2)(n_1,n_2)s \in \mathbf{R} \}
  \,.
\end{gather*}
Note the negative involvement of $(m_1,m_2)(n_1,n_2)\top$
in the definition of $\mathbf{R}^\bullet$.
When this play appears in $\mathbf{R}$,
then by construction $\mathbf{R}$ must contain
all plays of the form $(m_1,m_2)(n_1,n_2)s$ as well.
However, they become meaningless
as a simulation can never proceed
in a way that they could influence.

\subsection{Ordering Refinement Conventions}

The root source of complexity
in the ordering of plays within the refinement convention lies in
the alternation of choices
between the client and the handler
as described in \autoref{sec:rc:overview}.
To illustrate this phenomenon more concretely,
consider the following refinement conventions,
presented in increasing order:
\begin{align*}
  \mathbf{R}_1 & : E \leftrightarrow E' := \varnothing \\
  \mathbf{R}_2 & := \mathbf{R}_1\ \cup\ \{(m_1, m'_1)\bot, (m_2, m'_2)\bot\} \\
  \mathbf{R}_3 & := \mathbf{R}_2\ \cup\ \{(m_1, m'_1)(n_1, n'_1)s_1, (m_1, m'_1)(n_2, n'_2)s_2\}\quad \text{where $s_1$ and $s_2$ are non-empty}  \\
  \mathbf{R}_4 & := \mathbf{R}_3\ \cup\ \downarrow\{(m_1, m'_1)(n_1, n'_1)\top\}
\end{align*}
where the signatures $E$ and $E'$ are defined as follows:
\[
  E  := \{ m_1 : \{n_1, n_2\},\ m_2 : \{ n_1, n_2 \} \} \quad
  E' := \{ m'_1 : \{n'_1, n'_2\},\ m'_2 : \{ n'_1, n'_2 \} \}
\]
From the client's perspective,
the relation $\mathbf{R}^\circ$ determines
which pairs of questions
it is permitted to initiate,
while $\mathbf{R}^\bullet$ determines
which pairs of answers it must be able to respond to.
As the refinement convention enlarges,
the client gains more flexibility in its choice of questions
and faces fewer obligations with respect to answers.
Consequently, larger conventions are more permissive for the client,
making the refinement proof easier to establish.

\begin{itemize}
  \item The proof fails with $\mathbf{R}_1$
    immediately,
    as the empty set leaves no valid choice of questions.
  \item With $\mathbf{R}_2$,
    the client may choose $(m_1, m'_1)$, or $(m_2, m'_2)$
    as questions.
    However, the handler is unconstrained
    in its choice of answers,
    and the client cannot further proceed
    upon receiving the answer.
  \item With $\mathbf{R}_3$,
    if the client selects $(m_1, m'_1)$ as questions,
    the handler's choice is restricted to $(n_1, n'_1)$ or $(n_2, n'_2)$.
    This restriction simplifies the client's proof obligation,
    since fewer cases need to be considered.
    The refinement then proceeds along $s_1$ or $s_2$,
    depending on the handler's choice.
  \item Once
    $(m_1, m'_1)(n_1, n'_1) \top$ is added,
    the handler is barred from choosing $(n_1, n'_1)$.
    The client's proof obligation is further simplified,
    as one potential branch is eliminated.
\end{itemize}

Dually, from the handler's perspective,
$\mathbf{R}^\circ$ specifies the pairs of questions
for which it must be ready to provide answers,
and $\mathbf{R}^\bullet$ specifies the answers
it is allowed to return.
As the refinement convention expands,
the handler faces stricter obligations,
since it must cover more possible questions
while being constrained to fewer admissible answers.
Thus, larger conventions are more restrictive
for the handler,
making the refinement proof harder to establish.

\begin{itemize}
  \item The proof trivially succeeds with $\mathbf{R}_1$,
    since no obligations exist.
  \item With $\mathbf{R}_2$,
    the handler must provide answers for
    $(m_1, m'_1)$ and $(m_2, m'_2)$,
    but fortunately,
    the client's freedom of choice ensures
    that the proof can still complete easily.
  \item The task becomes harder with $\mathbf{R}_3$:
    if the client chooses $(m_1, m'_1)$ as questions,
    the handler is restrained to answering either $(n_1, n'_1)$ or $(n_2, n'_2)$,
    after which,
    the proof continues along $s_1$ or $s_2$.
  \item $\mathbf{R}_4$ makes the situation more challenging,
    since the handler is no longer allowed
    to choose $(n_1, n'_1)$
    as a response to $(m_1, m'_1)$.
    This further restricts its strategy
    and increases the difficulty of the proof.
\end{itemize}

This ordering of refinement conventions
parallels the refinement order of simulation conventions
in CompCertO.
Formally,
the refinement relation $\mathbb{R} \sqsubseteq \mathbb{S}$ is defined as follows:
\begin{align*}
  \forall w\ m_1\ m_2.\ w \Vdash m_1 \mathbin{S^\que} m_2
  \rightarrow \exists v.\ (v \Vdash m_1 \mathbin{R^\que} m_2 \wedge
    \forall n_1\ n_2.\ v \Vdash n_1 \mathbin{R^\bullet} n_2 \rightarrow
  w \Vdash n_1 \mathbin{S^\bullet} n_2)
\end{align*}
Intuitively,
a more refined simulation convention $\mathbb{R}$
admits a broader range of questions
while restricting the admissible answers.
This asymmetry makes the simulation proof easier
for the client
(since it has more choices and fewer obligations)
but harder for the handler
(since it must cover more cases under tighter constraints).
As a consequence, CompCertO establishes the following property:
\begin{equation}\label{eq:convention-ordering}
  \begin{prooftree}
    \hypo{\mathbb{R} \sqsubseteq \mathbb{R}'}
    \hypo{L_1 \le_\kw{\mathbb{R} \twoheadrightarrow \mathbb{S}} L_2}
    \hypo{\mathbb{S} \sqsubseteq \mathbb{S}'}
    \infer3{L_1 \le_\kw{\mathbb{R'} \twoheadrightarrow \mathbb{S'}} L_2}
  \end{prooftree}
\end{equation}
In this rule, the handler's simulation convention $\mathbb{S}$
may be replaced by a less refined one $\mathbb{S}'$,
and the client's refinement convention $\mathbb{R}$
may be replaced by a more refined one $\mathbb{R}'$.
Both substitutions reduce the proof burden,
thereby ensuring that the simulation property is preserved.
In the next section,
an analogous property will be established
for strategy refinement under refinement conventions,
extending this intuition beyond simulation conventions.

\subsection{Example of Refinement Conventions}

Four examples of refinement conventions are given.

\subsubsection{Deriving a Refinement Convention from a Relation}

A set $U$ defines a signature $[U] := \{ u : U \mid u \in U \}$,
where both questions and answers are elements of $U$.
A relation $R \subseteq U \times V$
defines a refinement convention between $[U]$ and $[V]$,
where
the questions and answers
are permitted
as long as they are related by $R$.
\begin{definition}[Refinement convention interpretation of relations]
  A relation $R \subseteq U \times V$ defines
  a refinement convention $[R] : [U] \leftrightarrow [V]$ given by:
  \begin{align*}
    (u, v)\bot \in [R] &\::\Leftrightarrow\:
    u \mathrel{R} v \\
    (u,v)(u',v')\top \in [R] &\::\Leftrightarrow\:
    u \mathrel{R} v \wedge \lnot u' \mathrel{R} v' \\
    (u,v)(u',v')\,s \in [R] &\::\Leftrightarrow\:
    u \mathrel{R} v \wedge (u' \mathrel{R} v' \Rightarrow s \in [R]) \,.
  \end{align*}
\end{definition}
It is straightforward to show that
$[R]$ is downward closed
and prove that when when questions or answers
are admitted by the refinement convention,
they must be related by the underlying relation $R$:
\[
  u \mathbin{[R]^\que} v \:\Leftrightarrow\: u \mathbin{R} v\qquad
  u' \mathbin{[R]^\ans_{uv}} v' \:\Leftrightarrow\: u' \mathbin{R} v'
\]

\subsubsection{Stateful Refinement Conventions}

Like strategies
where the history of the events
can determine the next move,
refinement conventions
can also be history sensitive.
This capability becomes essential
when relating implementations
that maintain state differently,
as the appropriate refinement relationship
may depend on the current system state.

Consider again the counter example (Example~\ref{ex:decodespec}),
where the specification
encapsulates the state internally
while the implementation uses external state.
\begin{align*}
  \sigma_\kw{cnt} : \mathbf{0} \rightarrow E_\kw{cnt}
  & \vDash \big(\kw{inc}() \sysstep 0\big) \envstep
  \big(\kw{inc}() \sysstep 1\big) \envstep \cdots\\
  \sigma'_\kw{cnt} : \mathbf{0} \rightarrow E_\kw{cnt} \at \mathbb{N}
  & \vDash \big(\kw{inc}() \at 0 \sysstep 0 \at 1\big) \envstep
  \big(\kw{inc}() \at 1 \sysstep 1 \at 2\big) \envstep \cdots
\end{align*}
To relate these two implementations,
we need a refinement convention
that tracks the counter's current value,
since the validity of moves depends on
whether the external state matches the expected internal state.
The following history-sensitive refinement convention accomplishes this:
\begin{align*}
  (\kw{inc}, \kw{inc}\at i) \bot \in \mathbf{R}_k
  & \::\Leftrightarrow\: k = i\\
  (\kw{inc}, \kw{inc}\at i)(j, j'\at i') \top \in \mathbf{R}_k
  & \::\Leftrightarrow\: k = i \wedge (k \neq j \vee j \neq j' \vee i' \neq i + 1 )\\
  (\kw{inc}, \kw{inc}\at i)(j, j'\at i') s \in \mathbf{R}_k
  & \::\Leftrightarrow\: k = i \wedge
  (k = j \wedge j = j' \wedge i' = i + 1 \rightarrow s \in \mathbf{R}_{k+1})
\end{align*}
Here $\mathbf{R}_k$
represents the refinement convention when the counter state is $k$,
and the convention evolves as operations modify the counter state.
Particularly, question events are only matched
when the explicit state in the implementation matches
the hidden state,
and the returned answer must be the same
with the next state properly updated according to the answer.
The overall refinement convention is initialized at the initial state:
\[
  \mathbf{R}_\kw{cnt} \::= \: \mathbf{R}_0
\]

\subsubsection{Refinement Conventions for a Lens}
\label{sec:rc:refconv-lens}

Consider a stateful lens
$f: U \rightleftarrows V$
with states in $P$.
The refinement conventions
$f^* : [U] \leftrightarrow [V]$
and
$f_* : [V] \leftrightarrow [U]$
can be derived from the lens,
and are dual to each other.
The questions $u$ and $v$
are related if $u$ can be obtained from $v$
via the lens;
the answers $u'$ and $v'$ are related
if $v'$ is the result of updating $v$ with $u'$,
and the state is updated accordingly.
\begin{align*}
  (u, v)\bot \in f^*_p
  & \::\Leftrightarrow\:
  \kw{get}_{|f|} (v, p) = u \\
  (u, v)(u', v')\top \in f^*_p
  & \::\Leftrightarrow\:
  \kw{get}_{|f|} (v, p) = u \:\wedge\:
  \forall p'.\ \kw{set}_{|f|} ((v, p), u') \neq (v', p')\\
  (u, v)(u', v') s \in f^*_p
  & \::\Leftrightarrow\:
  \kw{get}_{|f|} (v, p) = u \:\wedge\:
  \forall p'.\ \kw{set}_{|f|} ((v, p), u') = (v', p') \rightarrow
  s \in f^*_{p'}
\end{align*}
The refinement convention $f^*$
is given by $f^*_{p_0}$,
and its dual $f_*$ is defined
by flipping the questions and answers.

\subsubsection{Refinement Convention from Simulation Convention}
\label{sec:rc:refconv-simconv}

The simulation conventions used in CompCertO
can likewise be translated to the richer notion of
refinement convention.
Recall that
a simulation convention
$\mathbb{R} : A \twoheadleftrightarrow B := \langle W, \sysstep, R^\que, R^\ans \rangle$
between the CompCertO language interfaces $A$ and $B$ is specified by
a set $W$ of worlds,
a Kripke relation $R^\que \subseteq \mathcal{R}_W (A^\que, B^\que)$
between questions
and
a Kripke relation $R^\ans \subseteq \mathcal{R}_W (A^\ans, B^\ans)$
between answers.

Kripke worlds are used to ensure that
questions and answers for a given call
are related consistently.
However, every pair of calls
is related in isolation,
independently of any past or future calls.
Thus, the following
refinement convention
embeds the simulation convention $\mathbb{R} : A \Leftrightarrow B$:
\begin{align*}
  (m_1, m_2) \bot \in \llbracket \mathbb{R} \rrbracket \: :\Leftrightarrow \: {}
  &\exists w \cdot m_1 \mathbb{R}^\que_w m_2
  \\
  (m_1, m_2) (n_1, n_2) \top \in \llbracket \mathbb{R} \rrbracket \::\Leftrightarrow\: {}
  &\exists w \cdot m_1 \mathbb{R}^\que_w m_2 \:\wedge\:
  \neg n_1 \mathbb{R}^\ans_w n_2
  \\
  (m_1, m_2) (n_1, n_2) s \in \llbracket \mathbb{R} \rrbracket \::\Leftrightarrow\: {}
  &\exists w \cdot m_1 \mathbb{R}^\que_w m_2 \:\wedge\:
  (n_1 \mathbb{R}^\ans_w n_2 \Rightarrow s \in \llbracket \mathbb{R} \rrbracket)
  \,.
\end{align*}

% \begin{prooftree}
%   \hypo{\mathbf{R} \sqsubseteq \mathbf{S} \Leftrightarrow
%     \llbracket \mathbf{R} \rrbracket
%   \le \llbracket \mathbf{S} \rrbracket}
%   \infer[no rule]1{\llbracket \mathbf{id} \rrbracket = \mathbf{id}
%     \qquad
%     \mathbf{R} \fatsemi \mathbf{S} =
%     \llbracket \mathbf{R} \rrbracket
%   \fatsemi \llbracket \mathbf{S} \rrbracket}
% \end{prooftree}
% \qquad \qquad
% \begin{prooftree}
%   \hypo{\phi : L_1 \le_{\mathbf{R}_A \twoheadrightarrow \mathbf{R}_B} L_2}
%   \infer1{\llbracket \phi \rrbracket :
%     \llbracket L_1 \rrbracket
%     \le_{\llbracket \mathbf{R}_A \rrbracket
%     \twoheadrightarrow \llbracket\mathbf{R}_B\rrbracket}
%   \llbracket L_2 \rrbracket}
% \end{prooftree}

The simulation convention used here is CompCertO's original one,
rather than the stateful simulation convention (Def.~\ref{def:oe:ssc}).
The reason is that the stateful variant
was specifically designed to
accommodate the ad-hoc encapsulation
employed in CompCertO's semantics.
By contrast,
in the strategy model,
state encapsulation is built in as a primitive feature.
Thus, for transition systems with encapsulated state in OpenTE,
the underlying transition system is simply embedded
into a strategy and
composed with the strategy model's native encapsulation primitive.

\subsection{Regular Refinement Conventions}

Similar to the notion of regular strategies,
a refinement convention is said to be regular
if the relation between the questions and between answers
is independent of the preceding questions and answers.

\begin{definition}[Regular refinement conventions]
  A refinement convention $\mathbf{R} : E \leftrightarrow F$
  is regular
  if the residual refinement convention,
  after any sequence of questions $(m_1, m_2)$
  and answers $(n_1, n_2)$,
  coincides with the original one.
  Formally,
  \[
    \mathbf{R}\ \kw{regular} \::\Leftrightarrow\:
    \forall m_1 m_2 n_1 n_2.\
    m_1 \mathbf{R}^\que m_2 \rightarrow
    n_1 \mathbf{R}^\ans_{m_1m_2} n_2 \rightarrow
    \mathbf{R} = \mathbf{R}^{n_1n_2}_{m_1m_2}
  \]
\end{definition}
Among the examples of the refinement conventions
given in the last section,
the ones derived from relations and simulation conventions
are regular.

\subsection{Composition of Refinement Conventions}

Having established the basic structure of refinement conventions,
this section turns to their compositional properties.
Just as strategies compose in multiple dimensions,
refinement conventions must also support various forms of composition
to enable modular reasoning about complex systems.

\subsubsection{Vertical Composition}

Vertical composition allows refinement conventions
to be chained
in a transitive manner,
much like composing ordinary relations.
However, the alternating nature of client and handler interactions
requires careful treatment of the mixed variance patterns
that I discussed earlier.

Before defining vertical composition,
I introduce the notion of an identity refinement convention,
which serves as the neutral element for composition.
The identity convention relates only identical questions and answers,
essentially expressing that no translation is needed.

\begin{definition}
  The \emph{identity refinement convention} $\idsc_E$
  associated with a signature $E$ is defined by:
  \begin{align*}
    (m_1, m_2) \bot \in \idsc_E &\::\Leftrightarrow\:
    m_1 = m_2 \\
    (m_1, m_2) (n_1, n_2) \top \in \idsc_E &\::\Leftrightarrow\:
    m_1 = m_2 \wedge n_1 \neq n_2 \\
    (m_1, m_2) (n_1, n_2) s \in \idsc_E &\::\Leftrightarrow\:
    m_1 = m_2 \wedge (n_1 = n_2 \Rightarrow s \in \idsc_E)
  \end{align*}
\end{definition}
Intuitively, the identity convention permits questions
and answers to be related
only when they are identical,
and similarly for answers.
The recursive structure maintains this property throughout the interaction.

Now I define vertical composition.
Conceptually, vertical composition works like relational composition:
$\mathbf{R} \fatsemi \mathbf{S}$ relates
questions $m_1$ and $m_3$
when there exists an intermediate question $m_2$ such that
$m_1$ is related to $m_2$ by $\mathbf{R}$
and $m_2$ is related to $m_3$ by $\mathbf{S}$.
However, the presence of answers and the need to track interaction history
significantly complicate the definition.

\begin{definition}[Vertical composition of refinement conventions] \label{def:vcomp}
  For the refinement conventions
  $\mathbf{R} : E_1 \leftrightarrow E_2$ and
  $\mathbf{S} : E_2 \leftrightarrow E_3$,
  the refinement convention
  $\mathbf{R \fatsemi S} : E_1 \leftrightarrow E_3$
  is defined as follows:
  \begin{align*}
    (m_1, m_3) \bot \in \mathbf{R} \fatsemi \mathbf{S} \: :\Leftrightarrow \: {}
    &\exists m_2 \cdot
    (m_1, m_2)\bot \in \mathbf{R} \:\wedge\:
    (m_2, m_3)\bot \in \mathbf{S}
    \\%[1.2ex]
    (m_1, m_3) (n_1, n_3) \top \in \mathbf{R} \fatsemi \mathbf{S} \::\Leftrightarrow\: {}
    &\exists m_2 \cdot
    (m_1, m_2)\bot \in \mathbf{R} \:\wedge\:
    (m_2, m_3)\bot \in \mathbf{S} \:\wedge\: {} \\
    &\,\forall n_2 \cdot
    (m_1, m_2)(n_1, n_2)\top \in \mathbf{R} \:\vee\:
    (m_2, m_3)(n_2, n_3)\top \in \mathbf{S}
    \\%[1.2ex]
    (m_1, m_3) (n_1, n_3) \, s \in \mathbf{R} \fatsemi \mathbf{S} \::\Leftrightarrow\: {}
    &\exists m_2 \cdot
    (m_1, m_2)\bot \in \mathbf{R} \:\wedge\:
    (m_2, m_3)\bot \in \mathbf{S} \:\wedge\: {} \\
    &\,\forall n_2 \cdot
    (m_1, m_2)(n_1, n_2)\top \in \mathbf{R} \:\vee\:
    (m_2, m_3)(n_2, n_3)\top \in \mathbf{S} \:\vee\: {} \\
    &\qquad \:\:\: s \,\in\,
    \bigl( (m_1, m_2) (n_1, n_2) \backslash \mathbf{R} \bigr) \fatsemi
    \bigl( (m_2, m_3) (n_2, n_3) \backslash \mathbf{S} \bigr)
    \,.
  \end{align*}
\end{definition}

The definition captures the essential idea that
answers $(n_1, n_3)$ are forbidden in the composition
if either the first step ($\mathbf{R}$) or the second step ($\mathbf{S}$) would forbid them.
This reflects the contravariant nature of answer relations:
restrictions from either component must be respected in the composition.
The third clause handles the recursive case,
where the interaction continues according to
the composition of the residual conventions.

\begin{remark}[Associativity of vertical composition] \label{rem:vcomp-assoc}
  It should be noted that the vertical composition
  of refinement conventions
  is not associative in general
  (although associativity holds
  in most practical cases that have been encountered).
  This limitation arises from the complex interplay
  between the variance patterns of questions and answers.
  A more detailed discussion on
  this phenomenon and a counter-example can be found
  in Appendix~A of \citep{compcertoe-tr}.
\end{remark}

\subsubsection{Spatial Composition}

In addition to vertical composition,
spatial composition is used to handle
refinement conventions that operate on
different parts of a compound interface.
Spatial composition allows us to combine
refinement conventions in parallel,
much like the tensor product of strategies.

For two refinement conventions
that operate on disjoint parts of an interface,
their spatial composition should combine them
in a way that preserves the structure of both.
The tensor product construction achieves this
by requiring that both component conventions
agree on their respective parts of any interaction.

\begin{definition}
  The tensor product
  $\mathbf{R} \otimes \mathbf{S} :
  E_1 \otimes F_1 \leftrightarrow E_2 \otimes F_2$
  of the refinement conventions
  $\mathbf{R} : E_1 \leftrightarrow E_2$ and
  $\mathbf{S} : F_1 \leftrightarrow F_2$
  is defined by the rules:
  \begin{align*}
    \bigl( (m_1,q_1), (m_2,q_2) \bigr)\bot \in \mathbf{R} \otimes \mathbf{S}
    \::\Leftrightarrow\: &
    (m_1, m_2)\bot \in \mathbf{R}
    \:\wedge\:
    (q_1, q_2)\bot \in \mathbf{S}
    \\
    \bigl( (m_1,q_1), (m_2,q_2) \bigr)
    \bigl( (n_1,r_1), (n_2,r_2) \bigr)\top
    \in \mathbf{R} \otimes \mathbf{S}
    \::\Leftrightarrow\: &
    (m_1, m_2)\bot \in \mathbf{R}
    \:\wedge\:
    (q_1, q_2)\bot \in \mathbf{S}
    \:\wedge\: \\
    & \bigl( (m_1,m_2)(n_1,n_2)\top \in \mathbf{R} \:\vee\:
    (q_1,q_2)(r_1,r_2)\top \in \mathbf{S} \bigr)
    \\
    \bigl( (m_1,q_1), (m_2,q_2) \bigr)
    \bigl( (n_1,r_1), (n_2,r_2) \bigr) \,s
    \in \mathbf{R} \otimes \mathbf{S}
    \::\Leftrightarrow\: &
    (m_1, m_2)\bot \in \mathbf{R}
    \:\wedge\:
    (q_1, q_2)\bot \in \mathbf{S} \: \wedge\: \\
    & \large\bigl( (m_1,m_2)(n_1,n_2)\top \in \mathbf{R} \: \vee \:
      (q_1,q_2)(r_1,r_2)\top \in \mathbf{S} \: \vee \:\\
      & s \in \bigl( (m_1,m_2)(n_1,n_2) \backslash \mathbf{R} \bigr) \otimes
    \bigl( (q_1,q_2)(r_1,r_2) \backslash \mathbf{S} \bigr)\large\bigr)
  \end{align*}
\end{definition}

The tensor product follows a straightforward pattern:
questions are related in the product if they are related
in both component conventions,
while answers are forbidden if they are forbidden
in either component convention.
This preserves the contravariant nature of answer relations
while maintaining the covariant nature of question relations.

The tensor product of strategies
is not well-behaved in general,
because layered composition may affect how plays
synchronize under the tensor product.
However, for refinement conventions,
the tensor product behaves more predictably
because it operates at the level of individual question-answer pairs
rather than full interaction sequences.

For notational convenience,
I will also use $\at$
to denote the spatial composition of refinement conventions,
as $\mathbf{R} \at \mathbf{S} := \mathbf{R} \otimes \mathbf{S}$.
This notation aligns with the spatial composition of strategies
and emphasizes the parallel nature of the operation.

\subsubsection{Flat Composition}

The third mode of composition is flat composition,
which extends the flat composition operation $\oplus$
from effect signatures and strategies to refinement conventions.
Flat composition is particularly useful for
modular development of refinement conventions,
allowing independent conventions to be combined
when they operate on disjoint parts of a signature.

Unlike spatial composition,
which works with tensor products,
flat composition operates at the level of signature components.
When two refinement conventions operate on separate components
of a signature union,
their flat composition handles each component independently.

\begin{definition}[Flat composition of refinement conventions]
  The conventions
  $\mathbf{R}_1 : E_1 \leftrightarrow F_1$ and
  $\mathbf{R}_2 : E_2 \leftrightarrow F_2$
  compose into
  $\mathbf{R}_1 \oplus \mathbf{R}_2 : E_1 \oplus E_2 \leftrightarrow F_1 \oplus F_2$,
  defined by:
  \begin{align*}
    \bigl(\iota_i(m_1), \iota_i(m_2)\bigr)\bot \in \mathbf{R}_1 \oplus \mathbf{R}_2
    \:&:\Leftrightarrow\:
    (m_1,m_2)\bot \in \mathbf{R}_i
    \\
    \bigl(\iota_i(m_1), \iota_i(m_2)\bigr)(n_1,n_2)\top \in \mathbf{R}_1 \oplus \mathbf{R}_2
    \:&:\Leftrightarrow\:
    (m_1,m_2)(n_1,n_2)\top \in \mathbf{R}_i
    \\
    \bigl(\iota_1(m_1), \iota_1(m_2) \bigr)(n_1,n_2) \, s \in \mathbf{R}_1 \oplus \mathbf{R}_2
    \:&:\Leftrightarrow\:
    s \in \bigl( (m_1,m_2)(n_1,n_2) \backslash \mathbf{R}_1 \bigr) \oplus \mathbf{R}_2
    \\
    \bigl(\iota_2(m_1), \iota_2(m_2) \bigr)(n_1,n_2) \, s \in \mathbf{R}_1 \oplus \mathbf{R}_2
    \:&:\Leftrightarrow\:
    s \in \mathbf{R}_1 \oplus \bigl( (m_1,m_2)(n_1,n_2) \backslash \mathbf{R}_2 \bigr)
  \end{align*}
\end{definition}

The flat composition definition reflects the componentwise nature of the operation.
Questions and answers are handled independently for each component,
with the injection function $\iota_i$ indicating which component
is being addressed.
The recursive structure ensures that
interactions within one component remain isolated
from interactions in the other component,
preserving the modular structure of the composition.

\subsubsection{Properties}

Having defined the three modes of composition,
their key algebraic properties can now be established.
These properties ensure that refinement conventions
form a well-structured algebraic system
that supports compositional reasoning.

\begin{theorem}[Properties of Refinement Convention Composition]
  \begin{gather*}
    \begin{prooftree}
      \hypo{\mathbf{R} : E_1 \leftrightarrow F_1}
      \hypo{\mathbf{S} : E_2 \leftrightarrow F_2}
      \infer2[\kw{sc}-$\oplus$]{
        \mathbf{R} \oplus \mathbf{S} : E_1 \oplus E_2 \leftrightarrow F_1 \oplus F_2
      }
    \end{prooftree}
    \hspace{8em}
    \begin{prooftree}
      \hypo{\mathbf{R} : A \leftrightarrow B}
      \hypo{\mathbf{S} : U \leftrightarrow V}
      \infer2[\kw{sc}-$\mathbin@$]{
        \mathbf{R} \mathbin@ \mathbf{S} : A \mathbin@ U \leftrightarrow B \mathbin@ V
      }
    \end{prooftree}
    \\
    \begin{array}{r@{}l}
      (\mathbf{R}_1 \vcomp \mathbf{R}_2) \mathbin@ (\mathbf{S}_1 \vcomp \mathbf{S}_2)
      & {} \equiv
      (\mathbf{R}_1 \mathbin@ \mathbf{S}_1) \vcomp (\mathbf{R}_2 \mathbin@ \mathbf{S}_2)
      \\
      \idsc_A \mathbin@ \idsc_U & {} \equiv \idsc_{A \mathbin@ U}
    \end{array}
    \quad
    \begin{array}{r@{}l}
      (\mathbf{R}_1 \vcomp \mathbf{R}_2) \oplus (\mathbf{S}_1 \vcomp \mathbf{S}_2)
      & {} \equiv
      (\mathbf{R}_1 \oplus \mathbf{S}_1) \vcomp (\mathbf{R}_2 \oplus \mathbf{S}_2)
      \\
      \idsc_E \oplus \idsc_F & {} \equiv \idsc_{E \oplus F}
    \end{array}
  \end{gather*}
\end{theorem}

The properties show the compatibility
of the flat composition ($\oplus$) and the spatial composition ($\mathbin@$)
with respect to the vertical composition ($\fatsemi$).
Should the vertical composition be associative,
the flat composition and the spatial composition
can be viewed as bifunctors.

\section{Refinement Squares}
\label{sec:rc:refsq}

With refinement conventions defined,
the formal notion of refinement squares
can now be established.

Refinement conventions are used
to express the general notion of a refinement square
$
\sigma \le_{\mathbf{R} \rightarrow \mathbf{S}} \tau
$.
Specifically,
$\mathbf{R}$ and $\mathbf{S}$
will be used to translate each play of $\sigma$ into
a \emph{challenge} for the strategy $\tau$.
Because of the alternating nature of refinement,
this challenge will involve nested $\forall$ and $\exists$ quantifiers
over the possible choices of questions and answers
offered by the refinement conventions.

\begin{definition}[Refinement Square]
  Consider two strategies
  $\sigma : E_1 \rightarrow F_1$ and
  $\tau : E_2 \rightarrow F_2$
  as well as two refinement conventions
  $\mathbf{R} : E_1 \leftrightarrow E_2$ and
  $\mathbf{S} : F_1 \leftrightarrow F_2$.
  It is said that there is a refinement square
  when the proposition
  $\sigma \le_{R \rightarrow S} \tau$
  defined below holds.
  %
  To this end,
  a family of relations
  $\unlhd^x_{\mathbf{R} \twoheadrightarrow \mathbf{S}}$
  is recursively defined between the possible plays of $\sigma$
  and the possible residuals of $\tau$.
  Using the short-hands
  $\mathbf{R}' := (m_1,m_2)(n_1,n_2) \backslash \mathbf{R}$ and
  $\mathbf{S}' := (q_1,q_2)(r_1,r_2) \backslash \mathbf{S}$,
  we can write:
  %\[
  %  \begin{array}{l@{\:}lr@{\:}l@{\:}r@{\:}l}
  %    {\le_{R \twoheadrightarrow S}} &\subseteq
  %    P_{E_1 \twoheadrightarrow F_1} \times S_{E_1 \twoheadrightarrow F_2} &
  %    \Bigl( \mathbf{R} &\in S_{E_1 \leftrightarrow E_2}, &
  %           \mathbf{S} &\in S_{F_1 \leftrightarrow F_2} \Bigr)
  %  \\[1.5ex]
  %    {\le_{R \twoheadrightarrow S}^{q_1,q_2}} &\subseteq
  %    P_{E_1 \twoheadrightarrow F_1}^{q_1} \times S_{E_1 \twoheadrightarrow F_2}^{q_2} &
  %    \Bigl( \mathbf{R} &\in S_{E_1 \leftrightarrow E_2}, &
  %           \mathbf{S} &\in S_{F_1 \leftrightarrow F_2}^{(q_1,q_2)} \Bigr)
  %  \\[1.5ex]
  %    {\le_{R \twoheadrightarrow S}^{q_1m_1, q_2m_2}} &\subseteq
  %    P_{E_1 \twoheadrightarrow F_1}^{q_1m_1} \times S_{E_1 \twoheadrightarrow F_2}^{q_2m_2} &
  %    \Bigl( \mathbf{R} &\in S_{E_1 \leftrightarrow E_2}^{(m_1,m_2)}, &
  %           \mathbf{S} &\in S_{F_1 \leftrightarrow F_2}^{(q_1,q_2)} \Bigr)
  %  \end{array}
  %\]
  %corresponding to the different stages of the execution,
  %as follows:
  \[
    \def\arraystretch{1.2}
    \begin{array}{r@{\:\:\:}l@{\:}l@{\quad:\Leftrightarrow\quad}l}
      \epsilon &
      \unlhd_{\mathbf{R} \rightarrow \mathbf{S}} &
      \tau &
      \epsilon \in \tau
      \\
      q_1 s &
      \unlhd_{\mathbf{R} \rightarrow \mathbf{S}} &
      \tau &
      \forall q_2 \cdot
      (q_1, q_2)\bot \in \mathbf{S} \Rightarrow
      s \unlhd_{\mathbf{R} \rightarrow \mathbf{S}}^{q_1,q_2}
      (q_2 \backslash \tau)
      \\[1ex]
      \underline{r}_1 s &
      \unlhd_{\mathbf{R} \rightarrow \mathbf{S}}^{q_1,q_2} &
      \tau &
      \exists r_2 \cdot
      (q_1, q_2)(r_1, r_2)\top \notin \mathbf{S} \:\wedge\:
      s \unlhd_{\mathbf{R} \rightarrow \mathbf{S}'}
      (\underline{r}_2 \backslash \tau )
      \\
      \underline{m_1} s &
      \unlhd_{\mathbf{R} \rightarrow \mathbf{S}}^{q_1,q_2} &
      \tau &
      \exists m_2 \cdot
      (m_1, m_2) \bot \in \mathbf{R} \:\wedge\:
      s \unlhd_{\mathbf{R} \rightarrow \mathbf{S}}^{q_1m_1,q_2m_2}
      (\underline{m_2} \backslash \tau )
      \\[1ex]
      \epsilon &
      \unlhd_{\mathbf{R} \rightarrow \mathbf{S}}^{q_1m_1,q_2m_2} &
      \tau &
      \epsilon \in \tau
      \\
      n_1 s &
      \unlhd_{\mathbf{R} \rightarrow \mathbf{S}}^{q_1m_1,q_2m_2} &
      \tau &
      \forall n_2 \cdot
      (m_1,m_2)(n_1, n_2)\top \notin \mathbf{R} \Rightarrow
      s \unlhd_{\mathbf{R}' \rightarrow \mathbf{S}}^{q_1,q_2}
      (n_2 \backslash \tau)
    \end{array}
  \]
  Then the existence of a refinement square can be formulated as:
  \[
    \sigma \le_{\mathbf{R} \rightarrow \mathbf{S}} \tau
    \::\Leftrightarrow\:
    \forall s \in \sigma \cdot
    s \unlhd_{\mathbf{R} \rightarrow \mathbf{S}} \tau
    \,.
  \]
  Similar to a simulation in CompCertO,
  a refinement square can be visualized as follows:
  \[
    \begin{tikzcd}[sep=small]
      E_1 \ar[rr, "\sigma"] \ar[dd, leftrightarrow, "\mathbf{R}"'] && F_1
      \ar[dd, leftrightarrow, "\mathbf{S}"]
      \\
      & \quad & \\
      E_2 \ar[rr, "\tau"] && F_2
    \end{tikzcd}
  \]
\end{definition}

\subsection{Composition of Refinement Squares}

Refinement squares are compatible with
the three fundamental forms of strategy composition:
horizontal, vertical, and spatial.
These operations ensure that refinement squares
fit seamlessly into the three-dimensional algebraic structure
that underpins the strategy model.
Flat composition adds further modularity.

\begin{theorem}[Composition of Refinement Squares]
  The refinement squares compose
  along the following three dimensions:
  \begin{itemize}
    \item Refinement squares compose horizontally
      as described by the rule $\kw{sim}$-$\odot$:
      \[
        \begin{prooftree}
          \hypo{\phi: \sigma_1 \le_{\mathbf{S} \rightarrow \mathbf{T}} \sigma_1'}
          \hypo{\psi: \sigma_2 \le_{\mathbf{R} \rightarrow \mathbf{S}} \sigma_2'}
          \infer2[\kw{sim}-$\odot$]{\phi \odot \psi :
          \sigma_1 \odot \sigma_2 \le_{\mathbf{R} \rightarrow \mathbf{T}} \sigma_1' \odot \sigma_2'}
        \end{prooftree}
        \qquad
        \begin{tikzcd}[sep=small]
          E_1 \ar[rr, "\sigma_2"] \ar[dd, leftrightarrow, "\mathbf{R}"]
          && E_2 \ar[rr, "\sigma_1"] \ar[dd, leftrightarrow, "\mathbf{S}"]
          && E_3 \ar[dd, leftrightarrow, "\mathbf{T}"]\\
          & \psi && \phi & \\
          F_1 \ar[rr, "\tau_2"]
          && F_2 \ar[rr, "\tau_1"]
          && F_3
        \end{tikzcd}
      \]
    \item Refinement squares compose vertically
      as described by the rule $\kw{sim}$-$\vcomp$:
      \[
        \begin{prooftree}
          \hypo{\phi : \sigma_1 \le_{\mathbf{R} \rightarrow \mathbf{S}} \sigma_2}
          \hypo{\psi : \sigma_2 \le_{\mathbf{R'} \rightarrow \mathbf{S'}} \sigma_3}
          \infer2[\kw{sim}-$\vcomp$]{\phi \vcomp \psi : \sigma_1 \le_{\mathbf{R} \vcomp \mathbf{R'} \rightarrow
          \mathbf{S} \vcomp \mathbf{S'}} \sigma_3}
        \end{prooftree}
        \qquad
        \begin{tikzcd}[sep=small]
          E_1 \ar[rr, "\sigma_1"] \ar[dd, leftrightarrow, "\mathbf{R}"] &&
          F_1 \ar[dd, leftrightarrow, "\mathbf{S}"] \\
          & \phi &\\
          E_2 \ar[rr, "\sigma_2"] \ar[dd, leftrightarrow, "\mathbf{R'}"] &&
          F_2 \ar[dd, leftrightarrow, "\mathbf{S'}"] \\
          & \psi & \\
          E_3 \ar[rr, "\sigma_3"] && F_3
        \end{tikzcd}
      \]
    \item Spatial composition of refinement squares
      \[
        \begin{prooftree}
          \hypo{\phi: L \le_{\mathbf{R}_1 \rightarrow \mathbf{S}_1} L'}
          \hypo{\psi: f \le_{\mathbf{R}_2 \rightarrow \mathbf{S}_2} f'}
          \infer2[\kw{sim}-$\mathbin@$]{\phi \mathbin@ \psi :
            L \mathbin@ f
            \le_{\mathbf{R}_1 \mathbin@ \mathbf{R}_2 \rightarrow
            \mathbf{S}_1 \mathbin@ \mathbf{S}_2}
          L' \mathbin@ f'}
        \end{prooftree}
      \]
      Additionally,
      refinement squares are compatible with flat composition.
      \[
        \begin{prooftree}
          \hypo{\phi: \sigma_1 \le_{\mathbf{R}_1 \rightarrow \mathbf{S}_1} \sigma_1'}
          \hypo{\psi: \sigma_2 \le_{\mathbf{R}_2 \rightarrow \mathbf{S}_2} \sigma_2'}
          \infer2[\kw{sim}-$\oplus$]{\phi \oplus \psi :
            \sigma_1 \oplus \sigma_2
            \le_{\mathbf{R}_1 \oplus \mathbf{R}_2 \rightarrow
            \mathbf{S}_1 \oplus \mathbf{S}_2}
          \sigma_1' \oplus \sigma_2'}
        \end{prooftree}
      \]
  \end{itemize}
\end{theorem}

\subsection{Properties of Refinement Squares}

Refinement squares
allow several key properties
from CompCertO
to be reformulated in a uniform and structured way.

Refinement squares
subsume the usual inclusion orderings
on the strategies and refinement conventions.
\begin{theorem}
  For all strategies $\sigma, \tau : E \rightarrow F$ and
  for all refinement conventions $\mathbf{R}, \mathbf{S} : E \leftrightarrow F$,
  the following hold:
  \[
    \begin{array}{c@{\qquad\qquad}c}
      \sigma \subseteq \tau \Rightarrow
      \sigma \le_{\kw{id}_E \rightarrow \kw{id}_F} \tau
      &
      \begin{tikzcd}[sep=small]
        E \ar[rr, "\sigma"] \ar[dd, leftrightarrow, "\kw{id}"] && F \ar[dd, leftrightarrow, "\kw{id}"] \\
        & \sigma \subseteq \tau & \\
        E \ar[rr, "\tau"] && F
      \end{tikzcd}
      \\
      \mathbf{R} \supseteq \mathbf{S} \Rightarrow
      \kw{id}_E \le_{\mathbf{R} \rightarrow \mathbf{S}} \kw{id}_F
      &
      \begin{tikzcd}[sep=small]
        E \ar[rr, "\kw{id}"] \ar[dd, leftrightarrow, "\mathbf{R}"] && E \ar[dd, leftrightarrow, "\mathbf{S}"] \\
        & \mathbf{R} \supseteq \mathbf{S} & \\
        F \ar[rr, "\kw{id}"] && F
      \end{tikzcd}
    \end{array}
  \]
\end{theorem}

Use the refinement square representing the ordering on refinement conventions,
the counterpart of \eqref{eq:convention-ordering}
using refinement squares
is formulated as follows:
\[
  \begin{tikzcd}[sep=small]
    E_1
    \ar[rr, rightarrow, "\kw{id}"]
    \ar[dd, leftrightarrow, "\mathbf{R'}"']
    && E_1
    \ar[rr, "\sigma"] \ar[dd, leftrightarrow, "\mathbf{R}"]
    && F_1
    \ar[rr, rightarrow, "\kw{id}"]
    \ar[dd, leftrightarrow, "\mathbf{S}"']
    && F_1
    \ar[dd, leftrightarrow, "\mathbf{S'}"]
    \\
    & \mathbf{R} \subseteq \mathbf{R'} && \phi && \mathbf{S'} \subseteq \mathbf{S}
    \\
    E_2
    \ar[rr, rightarrow, "\kw{id}"]
    && E_2 \ar[rr, "\tau"] &&
    F_2
    \ar[rr, rightarrow, "\kw{id}"]
    && F_2
  \end{tikzcd}
\]

A further development is
the notion of companion and conjoint of a strategy.
\begin{definition}[Companion and Conjoint]
  We say that a strategy $\sigma : E \rightarrow F$ has
  \begin{itemize}
    \item a \emph{companion} $\sigma^* : E \leftrightarrow F$
      when the following refinement squares hold:
      \[
        \begin{tikzcd}[sep=small]
          E \ar[rr, "\kw{id}"] \ar[dd, leftrightarrow, "\kw{id}"] && E \ar[dd, leftrightarrow, "\sigma^*"] \\
          & \kw{id}_E \le_{\kw{id}_E \rightarrow \sigma^*} \sigma &
          \\
          E \ar[rr, "\sigma"] && F
        \end{tikzcd}
        \qquad\qquad
        \begin{tikzcd}[sep=small]
          E \ar[rr, "\kw{id}"] \ar[dd, leftrightarrow, "\sigma^*"] && E \ar[dd, leftrightarrow, "\kw{id}"] \\
          & \sigma \le_{\sigma^* \rightarrow \kw{id}_F} \kw{id}_F &
          \\
          F \ar[rr, "\sigma"] && F
        \end{tikzcd}
      \]
    \item a \emph{conjoint} $\sigma_* : F \leftrightarrow E$
      when the following refinement squares hold:
      \[
        \begin{tikzcd}[sep=small]
          F \ar[rr, "\kw{id}"] \ar[dd, leftrightarrow, "\sigma_*"] && F \ar[dd, leftrightarrow, "\kw{id}"] \\
          & \kw{id}_F \le_{\sigma_* \rightarrow \kw{id}_F} \sigma &
          \\
          E \ar[rr, "\sigma"] && F
        \end{tikzcd}
        \qquad\qquad
        \begin{tikzcd}[sep=small]
          E \ar[rr, "\sigma"] \ar[dd, leftrightarrow, "\kw{id}"] && F \ar[dd, leftrightarrow, "\sigma_*"] \\
          & \sigma \le_{\kw{id}_E \rightarrow \sigma_*} \kw{id}_E &
          \\
          E \ar[rr, "\kw{id}"] && E
        \end{tikzcd}
      \]
  \end{itemize}
\end{definition}
In particular,
every stateful lens possesses both a companion and a conjoint
(cf. \autoref{sec:rc:refconv-lens}).
The encapsulation primitive $\encap{u}$
comes equipped with the
companion $\encap{u}^* : \mathbf{1} \leftrightarrow U$
and conjoint $\encap{u}_* : U \leftrightarrow \mathbf{1}$,
which serve to ``deencapsulate'' the state.

Finally,
refinement squares yield a straightforward formulation of
representation independence at the strategy level.
Unlike the CompCertO formulation,
no additional assumptions are needed to
enforce that the environment cannot mutate private state%
---this guarantee is intrinsic
to the strategy and refinement square structure.

\begin{theorem}[Representation Independence for Strategies]
  Given a relation $R \subseteq U \times V$,
  \[
    \zeta : u \mathbin{R} v \quad\Longrightarrow\quad
    \encap{\zeta} : \encap{u} \le_{R \rightarrow \kw{id}_\mathbf{1}} \encap{v}
  \]
\end{theorem}

\subsection{Embedding CompCertO Simulations}

Using the embedding,
simulations between CompCertO transition systems and simulation conventions
induce refinement squares between
the corresponding strategies and refinement conventions
within the strategy model.
\[
  \phi: L_1 \le_{\mathbb{R} \twoheadrightarrow \mathbb{S}} L_2
  \quad\Longrightarrow\quad
  \llbracket \phi \rrbracket :
  \llbracket L_1 \rrbracket \le_{\llbracket \mathbb{R} \rrbracket
  \rightarrow \llbracket \mathbb{S} \rrbracket}
  \llbracket L_2 \rrbracket
\]
In particular,
CompCertO's compiler correctness
corresponds to the following refinement square:
\[
  \llbracket \phi_p^\kw{cc} \rrbracket \: : \:
  \llbracket \kw{Clight}(\kw{p.c}) \rrbracket
  \le_{\llbracket \mathbb{C} \rrbracket
  \rightarrow \llbracket\mathbb{C}\rrbracket}
  \llbracket \kw{Asm}(\kw{p.s}) \rrbracket
  \,.
\]

The simulation considered here
is that of the original CompCertO,
rather than the stateful simulation,
for the same reason as discussed in \autoref{sec:rc:refconv-simconv}.

\subsubsection{The Bounded Queue Example}
\label{sec:rc:bq}

For the bounded queue example,
the following strategy-level specification is first defined.
\[
  \tau_\kw{bq} : \mathbf{0} \rightarrow E_\kw{bq} \at D_\kw{bq}
\]
This specification explicitly passes the abstract state $D_\kw{bq}$
where $D_\kw{bq} := V^*$ represents a sequence of elements in $V$.

The strategy-level specification $\tau_\kw{bq}$
is implemented by the CompCertO specification $L_\kw{bq}$ (Section~\ref{sec:ox:application})
by the following refinement square:
\[
  \psi_1 : \tau_\kw{bq} \le_{\mathbf{0} \rightarrow \mathbf{R}_\kw{bq} \at D_\kw{bq}}
  \llbracket L_\kw{bq} \rrbracket
  \qquad\qquad
  \begin{tikzcd}[sep=small]
    \mathbf{0}
    \ar[rr, "\tau_\kw{bq}"]
    \ar[dd, leftrightarrow, "\mathbf{0}"']
    && E_\kw{bq} \at D_\kw{bq}
    \ar[dd, leftrightarrow, "\mathbf{R_\kw{bq}} \at D_\kw{bq}"]
    \\
    &\psi_1&\\
    \mathbf{0}
    \ar[rr, "\llbracket L_\kw{bq} \rrbracket"']
    && \C \at D_\kw{bq}
  \end{tikzcd}
\]
where the refinement convention $\mathbf{R}_\kw{bq} : E_\kw{bq} \leftrightarrow \C$
translates the events in the effect signature $E_\kw{bq}$
into the function calls and return values conforming to the C-level interface $\C$.
At the same time,
the strategy-level specification $\tau_\kw{bq}$
refines the more abstract strategy
$\sigma_\epsilon$ from \autoref{sec:strat:ex:persistent}
with intrinsic encapsulation on the state
by the following refinement square:
\[
  \psi_2 : \sigma_\epsilon \le_{\mathbf{0} \rightarrow \encap{\epsilon}_*} \tau_\kw{bq}
  \qquad\qquad
  \begin{tikzcd}[sep=small]
    \mathbf{0}
    \ar[rr, "\sigma_\epsilon"]
    \ar[dd, leftrightarrow, "\mathbf{0}"']
    && E_\kw{bq}
    \ar[dd, leftrightarrow, "\encap{\epsilon}_*"]
    \\
    &\psi_2&
    \\
    \mathbf{0}
    \ar[rr, "\tau_\kw{bq}"']
    && E_\kw{bq} \at D_\kw{bq}
  \end{tikzcd}
\]

Combining the reasoning here
with the correctness property of $L_\kw{bq}$
established in Section~\ref{sec:ox:application}
and CompCertO's correctness and linking theorem,
a complete proof is obtained from the strategy $\sigma_\epsilon$
to the compiled assembly programs $\kw{bq.s} + \kw{rb.s}$.
\[
  \begin{tikzcd}[row sep=small, column sep=large]
    \mathbf{0}
    \ar[rr, "\sigma_\epsilon"]
    \ar[dd, leftrightarrow, "\mathbf{0}"']
    && E_\kw{bq}
    \ar[dd, leftrightarrow, "\encap{\epsilon}_*"]
    \\
    & \psi_2 &
    \\
    \mathbf{0}
    \ar[rr, "\tau_\kw{bq}"]
    \ar[dd, leftrightarrow, "\mathbf{0}"']
    && E_\kw{bq} \at D_\kw{bq}
    \ar[dd, leftrightarrow, "\mathbf{R}_\kw{bq} \at D_\kw{bq}"]
    \\
    & \psi_1 &
    \\
    \mathbf{0}
    \ar[rr, "\llbracket L_\kw{bq} \rrbracket"]
    \ar[dd, leftrightarrow, "\mathbf{0}"]
    && \C \at D_\kw{bq}
    \ar[dd, leftrightarrow, "(\kw{id} \at R_\kw{bq}) \fatsemi \mathbb{R}_\kw{rb}"]
    \\
    & \phi_\kw{\ref{sec:ox:application}} &
    \\
    \C \at \kw{mem}
    \ar[rr, "\llbracket \kw{Clight}(\kw{bq.c}) \odot \kw{Clight}(\kw{rb.c}) \rrbracket"]
    \ar[dd, "\mathbb{C}"]
    &&
    \C \at \kw{mem}
    \ar[dd, "\mathbb{C}"]
    \\
    &\kw{CompCertO}&
    \\
    \C \at \kw{mem}
    \ar[rr, "\llbracket \kw{Asm}(\kw{bq.s} + \kw{rb.c}) \rrbracket"]
    &&
    \C \at \kw{mem}
  \end{tikzcd}
\]

\subsection{Synchronization of Events}

Refinement conventions enforce a 1-to-1 mapping between
the moves of the source- and target-level strategies,
and require that their plays have similar structures.
However, in some cases
the relationship between events in the high-level view of the system
and their realization in low-level terms is more complex;
for example,
the high-level view of a TCP/IP connection as a stream of bytes
could model the transmission of a block of data as a single event,
whereas its realization in terms of low-level packets
may involve a complex interaction.

While the strict mapping enforced by refinement conventions
is a limitation,
situations like the one described above
can still be modeled within the formalism.
Suppose $\sigma : \mathcal{B} \rightarrow E$
uses the ``byte stream'' interface $\mathcal{B}$
while its refinement
$\tau : \mathcal{K} \rightarrow E$
is implemented in terms of
a network packet interface $\mathcal{K}$.
It remains possible to express their relationship
as a refinement square
$
\sigma \odot x \, \le_{\mathbf{R} \rightarrow E} \, \tau
$
with the help of auxiliary constructions
$x : \mathcal{X} \rightarrow \mathcal{B}$ and
$\mathbf{R} : \mathcal{X} \leftrightarrow \mathcal{K}$,
proceeding in two steps:
\begin{itemize}
  \item the effect signature $\mathcal{X}$ can provide
    a high-level, abstract representation of the packet interaction,
    and the strategy $x : \mathcal{X} \rightarrow \mathcal{B}$
    explains how byte stream operations are expanded into
    abstract packet interactions with more complex shapes;
  \item the refinement convention
    $\mathbf{R} : \mathcal{X} \leftrightarrow \mathcal{K}$
    can then be used to express the data abstraction component of
    the relationship, refining high-level abstract packets into
    their low-level actual representations,
    and encapsulating details such as TCP sequence numbers.
\end{itemize}

\section{Implementing CAL}
\label{sec:rc:cal}

Another instance of the CAL
is presented within the strategy model.
This construction closely mirrors
the OpenTE instance described in \autoref{sec:oe:cal},
but here the specification is
expressed as strategies
and simulations are replaced by refinement squares.

A layer interface consists of a strategy
together with its abstract state:
\[
  L \::=\: \langle S, \sigma : \mathbf{0} \rightarrow \C \at\kw{mem} \rangle
\]
If the strategy $\sigma$ is defined directly,
without first introducing an abstract state $S$
and then encapsulating it,
the following
refinement square is typically appealed to in order to make the hidden state explicit:
\[
  \sigma \le_{\mathbf{0} \rightarrow \kw{id}_{\C\at\kw{mem}} \at \encap{s_0}_*} \sigma_s
\]
where $s_0$ represents
the initial state of $S$,
and $\sigma_s: \mathbf{0} \rightarrow \C \at\kw{mem}\at S$
is an equivalent version of $\sigma$
with the state passed explicitly.
This refinement square effectively ``concretizes''
the abstract state,
ensuring that reasoning about $\sigma$
can be carried out in terms of a concrete state-passing form.

Given a layer interface $\langle S, \sigma \rangle$,
the interpretation of an implementation $M$
is defined as:
\[
  \llbracket M \rrbracket L \::=\:
  \langle S, \encap{p_0} \odot \llbracket \kw{ClightP}(M) \rrbracket \odot \sigma \rangle
\]
Note here the encapsulation primitive
is used within the strategy model
to encapsulate the persistent environment
of the ClightP program.
Layer correctness is established by proving
the following refinement square:
\[
  L_1 \vdash M : L_2 \::\Leftrightarrow\:
  \sigma_2 \le \encap{p_0} \odot \llbracket \kw{ClightP}(M) \rrbracket \odot \sigma_1
\]
To discharge such obligations,
an abstraction relation $R \subseteq S_2 \times (S_1 \times \kw{penv})$
is typically introduced to connect the state of two layers.
As in the OpenTE setting,
vertical composition of layers follows
directly from the monotonicity of the $\odot$ operator,
ensuring the compositionality of correctness proofs.

\section{Summary}

This section addressed
one of the central challenges
in developing the strategy model:
how to properly formulate the notion of refinement.
Unlike simpler settings,
refinement here must account for
nondeterminism arising from both the system under consideration
and its environment.
In the framework, the key innovation
is the introduction of refinement conventions.
By embedding the nondeterministic choices
directly into the notion of a play and its ordering,
refinement conventions allow the refinement relation
to faithfully represent
both system-level and environment-level behavior.
This provides a unified account of
how strategies can be compared while preserving semantic rigor.

Building on this foundation,
a variety of composition properties
for refinement were established.
These results ensure
that the strategy model
respects the three-dimensional algebraic structure.
Such compositionality is crucial,
as it guarantees that local refinement proofs
can be assembled into larger correctness arguments
without sacrificing coherence.

Finally, it was demonstrated that
the strategy model is expressive enough to embed CompCertO
and capture its compiler correctness results.
This embedding highlights
both the flexibility and the applicability of the framework.
On one hand, it shows that established verification frameworks
can be recovered as special cases;
on the other, it reveals how the strategy model
extends beyond existing approaches
by offering finer control
over state encapsulation and nondeterminism.
In this way,
this chapter completes the development of refinement in the strategy setting,
positioning it as a robust semantic foundation
for reasoning about realistic software systems.


\chapter{Application and Evaluation}
\label{ch:application}

\section{Applications}

The applications of the OpenTE framework
and the strategy framework
have been discussed in detail
within their respective chapters.
Here, I will briefly summarize these applications.

The OpenTE framework extends the CompCertO
with data abstraction and state encapsulation capabilities.
This extension enables three key applications.
First, it provides a modular proof of the bounded queue example
presented in \S\ref{sec:ox:application},
demonstrating how data abstraction can simplify verification of realistic data structures.
Second, it supports the development of the ClightP language
in \S\ref{sec:oe:clightp},
which leverages state encapsulation
to further streamline the verification process
by providing cleaner interfaces between specification and implementation.
Third, it enables an instantiation of the Certified Abstract Language (CAL)
framework developed in \S\ref{sec:oe:cal},
showing how abstract layers
can be systematically constructed and composed.

The strategy framework provides a more flexible and general verification foundation.
Its most significant application is the seamless integration with the CompCertO framework,
which enables end-to-end verification
from high-level strategic specifications to low-level compiled assembly programs.
This capability has been demonstrated through two substantial examples:
the rot13 system verification shown in \S\ref{sec:strat:complete-proof}
and the bounded queue verification in \S\ref{sec:rc:bq}.
Additionally, the strategy framework
has been used to provide a more general instantiation of the Certified Abstract Language (CAL)
as shown in \S\ref{sec:rc:cal},
illustrating its potential
as a unifying foundation for diverse verification approaches.

\section{Evaluation}

\begin{table}
  \caption{Components of the Rocq artifact,
  with lines of code counted by $\kw{coqwc}$.}
  \label{tbl:artifact}
  \begin{tabular}{lrrlrr}
    \toprule
    Component & \hspace{-5em} Definitions & Proofs &
    Application & \hspace{-5em} Definitions & Proofs \\
    \midrule
    %
    % Includes: coqrel/*.v
    $\kw{coqrel}$ library & 2,382 & 959 &
    %
    % Includes: examples/compcerto/*.v
    CompCertO embedding
    \hspace{-1em} & 1,000 & 1,743 \\
    %
    % Includes: anything under compcerto/
    CompCertO & 124,217 & 95,187 &
    %
    % Includes: examples/memsep/*.v
    Bounded queue example & 1,572 & 2,606 \\
    %
    % Includes: structures/*.v lattices/*.v
    Other support code & 271 & 491 &
    %
    % Includes: examples/process/*.v
    Rot13 example & 1,414 & 2,621 \\
    %
    % Includes: models/IntStrat.v
    Strategy model
    \hspace{-1em} & 2,198 & 3,252 &
    %
    % Includes: examples/cal/*.v
    CAL & 262 & 667 \\

    % Encapsulation.v Lifting.v TensorComp.v
    OpenTE framework & 1,136 & 2,297 &
    %
    % Includes: examples/clightp/*.v
    ClightP & 1,656 & 2,126 \\

    \bottomrule
  \end{tabular}
\end{table}

The lines of code for the Rocq proof development
accompanying this thesis
are summarized in \autoref{tbl:artifact}.
Each component serves a specific role
in the overall framework.

The $\kw{coqrel}$ library
provides a systematic approach
for working with logical relations,
particularly offering robust support for handling Kripke relations
that are essential
for developing the stateful simulation convention (Sec.~\ref{sec:oe:sim}).

The CompCertO component contains
the entire certified compiler infrastructure.

Supporting libraries include
utilities for downsets and abstract domains
used in developing the strategy model (Sec.~\ref{sec:strat:model}).

The strategy model part implements the core semantic framework,
including strategy semantics,
refinement conventions,
refinement squares,
and all associated compositional properties.

The OpenTE framework component implements spatial composition,
state encapsulation,
and stateful simulation built on top of
the CompCertO semantics.

The CompCertO embedding provides
the interface that integrates CompCertO semantics
into the strategy model,
enabling end-to-end verification.

Application-specific components
include the bounded queue verification,
which also incorporates support for memory separation (Sec.~\ref{sec:ox:separation}),
and the rot13 verification,
which includes the development of loaders (Sec.~\ref{sec:strat:loaders}).

The CAL component implements
the Certified Abstract Language framework
using the strategy model,
while the ClightP component encompasses
the development of ClightP semantics,
its compilation to Clight,
and the composition of correctness results.

The $\kw{coqrel}$ library was developed
entirely by Jérémie Koenig.
The CompCertO semantics
and the majority of the supporting code
were also developed primarily by him (though CompCertO builds upon the existing CompCert compiler),
with some proof contributions from myself.
The development of the strategy model
represents a joint effort between Jérémie and myself,
while the remaining components were written primarily by myself.


\chapter{Related Work}
\label{ch:related}

Finally, I will briefly discuss existing research work
relevant to the framework and goals described in this work.
I have already covered some of the foundational literature
in Section~\ref{sec:intro:litreview}.

\paragraph{Interaction Trees}

As a ``semantics toolbox'' of sorts,
interaction trees share some goals and techniques
with the model presented in this work.
In fact, an interaction tree $t : \kw{ITree}_E(X)$
can be interpreted into the framework as a strategy
$\langle t \rangle : E \rightarrow \{ {*} : X \}$.
However, strategies generalize ITrees in several ways:
\begin{itemize}
  \item Strategies are two-sided and encode incoming as well as outgoing interactions,
    forming the basis for layered composition.
  \item By design, ITrees must be executable programs,
    whereas strategies can be described logically using arbitrary Coq specifications.
  \item Strategies that exhibit the same external behavior are formally equal.
    By contrast, ITrees are compared using bisimulation equivalences.
    Equational reasoning requires Coq's setoid support,
    which can be slower and more fragile than rewriting with $\kw{eq}$.
  \item The strategies come with built-in notions of partial definition,
    refinement and data abstraction,
    whereas similar notions for ITrees
    have to be defined and tailored to a particular application.
\end{itemize}

\paragraph{Game Semantics}

The horizontal fragment of the framework is a particularly simple form of game semantics.
The framework's novelty resides in the vertical and spatial fragments,
for which, to my knowledge, there exists no precedent in the game semantics literature.
In particular, refinement conventions involve alternations of angelic and demonic choices;
it is surprising to find they can be modeled using a fairly standard approach,
although a rather unconventional ordering of plays must be used.
An interesting question for further research would be to investigate how far this can be pushed
and whether games more complex than effect signatures could admit their own forms of refinement conventions.

\paragraph{Refinement Calculus}

The refinement calculus \citep{refcal} was a source of inspiration for our framework.
One defining feature of the refinement calculus is \emph{dual nondeterminism},
which provides very powerful abstraction mechanisms.
At the same time, models like predicate transformers
do not deal with external interactions or state encapsulation.

\paragraph{CompCertO}

The semantic model of CompCertO~\citep{compcerto,compcerto-dr} introduced \emph{simulation conventions}
and the associated idea of a full-blown, two-dimensional refinement framework,
so it is worth pointing out the ways in which our framework generalizes the CompCertO model,
especially when it comes to refinement conventions:
\begin{itemize}
  \item CompCertO transition systems and simulation conventions use
    explicit states and Kripe worlds in their definitions,
    whereas strategies and refinement conventions provide canonical representations
    for the components' observable behaviors.
  \item
    Effect signatures are more general than the language interfaces used in CompCertO,
    which force all questions to use the same set of answers.
  \item
    CompCertO transition systems do not retain any history between successive incoming questions;
    as such, they cannot support the kind of state encapsulation which our framework enables.
    Likewise, simulation conventions only specify 4-way relationships
    between isolated pairs of questions and answers,
    but unlike refinement conventions they cannot be sensitive to the history of the computation.
\end{itemize}

\paragraph{Other CompCert-based Verification Frameworks}

CompCertM \citep{compcertm} is another project
which builds on CompCert
to provide a compositional verification framework.
Like CompCertO,
it introduces a better model of the interaction between
C and assembly programs
and more flexibility in simulation conventions.
However, while it permits some form of localized state,
CompCertM still does not support
full-blown data abstraction and state encapsulation
of the kind we have presented.
See \citep{compcerto,compcerto-dr}
for a detailed comparison between Compositional CompCert,
CompCertM and CompCertO.

We have also touched on
certified abstraction layers and CompCertX in \S\ref{sec:bg:cal}.
Subsequent work has extended CAL to support concurrency \cite{ccal}.
There are more recent treatments of CAL which,
like our work,
attempt to streamline the underlying theory
\citep{popl22,rbgs-cal},
%In particular a limited version of the construction ${-} \mathbin@ U$
%operating on a fixed set $U$ appears in \citet{rbgs-cal}.
but this work has not been mechanized
or interfaced with CompCert.

%Interaction trees \cite{itree,itrees} provide
%another framework for compositional semantics
%formalized in the Coq proof assistant
%which presents similarities with our own.
%though their interface with CompCert is also less comprehensive.

\paragraph{Separation Logic}

For the most part,
the frameworks discussed above
do not provide program-level verification facilities,
but rather focus on a more coarse-grained, module-level ``glue''.
Likewise,
we have assumed that elementary module correctness properties
such as $\phi_1$, $\phi_2$ and $\phi_\kw{bq}^\kw{min}$
were provided by the user%
\footnote{Our example is simple enough that,
  in our implementation,
  manual simulation proofs were
sufficient.}
and focused on the problem of
connecting such proofs.
Nevertheless,
program logics in general and separation logic in particular
are relevant to our work in the following ways.

First, it would be beneficial to incorporate
such program logics into our framework.
For example, \citet{popl15} provides
a rudimentary Clight program logic which
can be used to help prove abstraction layers correct.
It may be useful to investigate whether
the Clight separation logic provided by
the Verified Software Toolchain \citep{vst}
could be interfaced with our model.

Secondly,
spatial composition is in fact
the defining feature of separation logic.
Our treatment of memory separation
draws extensively from
separation algebra \citep{sepalg},
an approach to building models of separation logic.
More recently,
Conditional Contextual Refinement (CCR) \citep{ccr}
combined (vertical) refinement and (spatial) separation logic into
a unified, mechanized framework.
CCR however does not support state encapsulation
or certified compilation.

\paragraph{Multi-language Semantics}

We have demonstrated that our framework is able to reason across languages through non-trivial examples such as the one in Fig~\ref{fig:readwritehello}.
In Compositional CompCert and CompCertM,
assembly programs are given C-level semantics,
making it possible to directly reason about composite programs
(but only for Asm code, which behaves according to the C calling convention).
CAL uses the opposite approach and can translate
C-level layer specification into assembly behaviors.
Recent work on the DimSum framework \citep{dimsum}
attempts to give a more general account of
multi-language semantics
by introducing wrappers to translate between
different languages.

These various approaches all attempt
to represent \emph{horizontally} what
the simulation conventions of CompCertO represent vertically.
In our framework,
the notions of companion and conjoint
could provide a natural way to formalize
approaches of this kind,
so that, for example, the CompCertO calling convention
$\mathbb{C} : \mathcal{C} \leftrightarrow \mathcal{A}$
would be in companion/conjoint relationships with
adapter components
$\mathbb{C}_* : \mathcal{A} \rightarrow \mathcal{C}$
and
$\mathbb{C}^* : \mathcal{C} \rightarrow \mathcal{A}$.
The complexity of CompCertO's convention as presently stated
makes this challenging,
but we do not believe it to be a fundamental issue.

\paragraph{Event-based Semantics}
The DimSum framework~\citep{dimsum}
employs a language-agnostic, event-based semantics
as a generic framework
for multi-language semantics.
Both the DimSum framework
and our strategy model
feature rich compositional structures,
and support private states
across function invocations.
However, there are several key differences
set DimSum apart
from our approach.
First,
DimSum introduces
explicit angelic and demonic nondeterminism
alongside events.
These nondeterministic structures
facilitate the transformation and ordering
of event sequences
at different abstraction levels.
However,
this also adds complexity
due to the intricate
commuting properties between events
and nondeterministic choices.
In contrast,
our strategy model
adheres to a transitional approach
where plays solely consist of events.
Here, dual nondeterminism
is concealed within the construction of
refinement conventions and simulations,
activating only when necessary.
Second,
events in DimSum are not
well-bracketed,
allowing for modeling complex interactions
such as coroutines,
which are challenging to implement
within our current strategy model.
Generalization to
asynchronous games semantics
would be required to accommodate such behaviors.
Third,
the DimSum framework
does not support spacial composition.
Instead,
data abstraction must go through
the semantics wrapper,
which is a rather heavy mechanism.
Lastly,
the DimSum framework
features a four-pass compiler
that translates idealized source-
and target-level programs.
By contrast,
our strategy model
integrates a realistic optimizing compiler
that compiles C program into assembly.


\chapter{Conclusion and Future Work}
\label{ch:conclusion}

\section{Conclusion}

This thesis has advanced
the development of end-to-end certified systems
by combining several key ingredients:
compositional semantics, abstraction, state encapsulation, and certified compilation.
Taken together,
these elements provide a scalable framework
for modular reasoning about complex software systems,
enabling correctness guarantees
to be carried from high-level specifications
down to executable binaries.

Beyond the specific technical contributions,
this work also reveals an underlying algebraic structure
that unifies different semantic models
under a common three-dimensional compositional framework.
This structure not only clarifies
the design principles behind our constructions
but also provides a conceptual foundation
that could guide the development of future certified frameworks.
I believe this algebraic perspective
may grow into a central tool
in certified systems engineering,
supporting a broader range of applications
than those considered here.

\section{Future Work}

While this work establishes a solid foundation,
several promising directions remain open
for further exploration.

\subsection{Integration with Automated Reasoning}

Although our framework enables modular reasoning,
the verification of individual components
still relies heavily on manual effort.
Future work should focus on integrating automated reasoning tools
into the framework,
thereby reducing the proof burden
and improving scalability.
A natural candidate is
the DeepSEA language \citep{deepsea},
which synthesizes both implementations and correctness proofs
directly from high-level specifications.
Embedding such tools within our compositional framework
could make certified development significantly
more practical for large-scale systems.

\subsection{Embedding Concurrent Semantics Models}

Modern systems are inherently concurrent,
featuring interleaved executions and
shared-state interactions.
Extending the current framework to capture concurrency is
thus a critical next step.
Existing models such as Concurrent CAL (CCAL) \citep{ccal}
and multi-threaded variants of CompCertO \citep{compcertoc}
offer starting points.
Embedding these techniques
into our strategy model would significantly
enhance its expressiveness,
allowing the framework
to verify concurrent programs
with the same modularity and scalability demonstrated
in the sequential setting.

\subsection{Connections to Higher Category Theory}

The compositional structure of our framework suggests
deeper connections to higher category theory
that warrant formal investigation.
The three-dimensional nature of our composition operations---with strategies
forming 1-morphisms, refinement conventions
forming 2-morphisms,
and their vertical composition---naturally suggests
a bicategorical or even tricategorical structure.
Making these connections explicit
could provide more powerful compositional principles
and reveal additional algebraic properties
that enhance the framework's expressiveness.
Furthermore, higher categorical perspectives
might offer insights into generalizing our approach
to handle more complex compositional patterns,
such as those arising
in concurrent or probabilistic systems,
while maintaining the essential modularity properties
that make verification tractable.

By pursuing these directions,
I aim to push the boundary of certified verification
from modular, sequential components
toward realistic, concurrent, and automated settings.
Together, they represent the next stage
in building a comprehensive framework
for certified systems that are both scalable in practice
and sound in theory.


% Any chapters such as End Notes go after this.
\backmatter

% \bibliography{references}
% for your own sake, use a bibtex file, so all of the numbering of references will be done
% automatically.

\cleardoublepage
\phantomsection
\addcontentsline{toc}{chapter}{\bibname}
\bibliographystyle{ACM-Reference-Format}
\bibliography{references}

\end{document}
