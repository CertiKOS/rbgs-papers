\chapter{Conclusion and Future Work}
\label{ch:conclusion}

\section{Conclusion}

This thesis has advanced
the development of end-to-end certified systems
by combining several key ingredients:
compositional semantics, abstraction, state encapsulation, and certified compilation.
Taken together,
these elements provide a scalable framework
for modular reasoning about complex software systems,
enabling correctness guarantees
to be carried from high-level specifications
down to executable binaries.

Beyond the specific technical contributions,
this work also reveals an underlying algebraic structure
that unifies different semantic models
under a common three-dimensional compositional framework.
This structure not only clarifies
the design principles behind our constructions
but also provides a conceptual foundation
that could guide the development of future certified frameworks.
I believe this algebraic perspective
may grow into a central tool
in certified systems engineering,
supporting a broader range of applications
than those considered here.

\section{Future Work}

While this work establishes a solid foundation,
several promising directions remain open
for further exploration.

\subsection{Integration with Automated Reasoning}

Although our framework enables modular reasoning,
the verification of individual components
still relies heavily on manual effort.
Future work should focus on integrating automated reasoning tools
into the framework,
thereby reducing the proof burden
and improving scalability.
A natural candidate is
the DeepSEA language \citep{deepsea},
which synthesizes both implementations and correctness proofs
directly from high-level specifications.
Embedding such tools within our compositional framework
could make certified development significantly
more practical for large-scale systems.

\subsection{Embedding Concurrent Semantics Models}

Modern systems are inherently concurrent,
featuring interleaved executions and
shared-state interactions.
Extending the current framework to capture concurrency is
thus a critical next step.
Existing models such as Concurrent CAL (CCAL) \citep{ccal}
and multi-threaded variants of CompCertO \citep{compcertoc}
offer starting points.
Embedding these techniques
into our strategy model would significantly
enhance its expressiveness,
allowing the framework
to verify concurrent programs
with the same modularity and scalability demonstrated
in the sequential setting.

\subsection{Connections to Higher Category Theory}

The compositional structure of our framework suggests
deeper connections to higher category theory
that warrant formal investigation.
The three-dimensional nature of our composition operations---with strategies
forming 1-morphisms, refinement conventions
forming 2-morphisms,
and their vertical composition---naturally suggests
a bicategorical or even tricategorical structure.
Making these connections explicit
could provide more powerful compositional principles
and reveal additional algebraic properties
that enhance the framework's expressiveness.
Furthermore, higher categorical perspectives
might offer insights into generalizing our approach
to handle more complex compositional patterns,
such as those arising
in concurrent or probabilistic systems,
while maintaining the essential modularity properties
that make verification tractable.

By pursuing these directions,
I aim to push the boundary of certified verification
from modular, sequential components
toward realistic, concurrent, and automated settings.
Together, they represent the next stage
in building a comprehensive framework
for certified systems that are both scalable in practice
and sound in theory.
