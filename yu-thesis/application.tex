\chapter{Application and Evaluation}
\label{ch:application}

\section{Applications}

The applications of the OpenTE framework
and the strategy framework
have been discussed in detail
within their respective chapters.
Here, I will briefly summarize these applications.

The OpenTE framework extends the CompCertO
with data abstraction and state encapsulation capabilities.
This extension enables three key applications.
First, it provides a modular proof of the bounded queue example
presented in \S\ref{sec:ox:application},
demonstrating how data abstraction can simplify verification of realistic data structures.
Second, it supports the development of the ClightP language
in \S\ref{sec:oe:clightp},
which leverages state encapsulation
to further streamline the verification process
by providing cleaner interfaces between specification and implementation.
Third, it enables an instantiation of the Certified Abstract Language (CAL)
framework developed in \S\ref{sec:oe:cal},
showing how abstract layers
can be systematically constructed and composed.

The strategy framework provides a more flexible and general verification foundation.
Its most significant application is the seamless integration with the CompCertO framework,
which enables end-to-end verification
from high-level strategic specifications to low-level compiled assembly programs.
This capability has been demonstrated through two substantial examples:
the rot13 system verification shown in \S\ref{sec:strat:complete-proof}
and the bounded queue verification in \S\ref{sec:rc:bq}.
Additionally, the strategy framework
has been used to provide a more general instantiation of the Certified Abstract Language (CAL)
as shown in \S\ref{sec:rc:cal},
illustrating its potential
as a unifying foundation for diverse verification approaches.

\section{Evaluation}

\begin{table}
  \caption{Components of the Rocq artifact,
  with lines of code counted by $\kw{coqwc}$.}
  \label{tbl:artifact}
  \begin{tabular}{lrrlrr}
    \toprule
    Component & \hspace{-5em} Definitions & Proofs &
    Application & \hspace{-5em} Definitions & Proofs \\
    \midrule
    %
    % Includes: coqrel/*.v
    $\kw{coqrel}$ library & 2,382 & 959 &
    %
    % Includes: examples/compcerto/*.v
    CompCertO embedding
    \hspace{-1em} & 1,000 & 1,743 \\
    %
    % Includes: anything under compcerto/
    CompCertO & 124,217 & 95,187 &
    %
    % Includes: examples/memsep/*.v
    Bounded queue example & 1,572 & 2,606 \\
    %
    % Includes: structures/*.v lattices/*.v
    Other support code & 271 & 491 &
    %
    % Includes: examples/process/*.v
    Rot13 example & 1,414 & 2,621 \\
    %
    % Includes: models/IntStrat.v
    Strategy model
    \hspace{-1em} & 2,198 & 3,252 &
    %
    % Includes: examples/cal/*.v
    CAL & 262 & 667 \\

    % Encapsulation.v Lifting.v TensorComp.v
    OpenTE framework & 1,136 & 2,297 &
    %
    % Includes: examples/clightp/*.v
    ClightP & 1,656 & 2,126 \\

    \bottomrule
  \end{tabular}
\end{table}

The lines of code for the Rocq proof development
accompanying this thesis
are summarized in \autoref{tbl:artifact}.
Each component serves a specific role
in the overall framework.

The $\kw{coqrel}$ library
provides a systematic approach
for working with logical relations,
particularly offering robust support for handling Kripke relations
that are essential
for developing the stateful simulation convention (Sec.~\ref{sec:oe:sim}).

The CompCertO component contains
the entire certified compiler infrastructure.

Supporting libraries include
utilities for downsets and abstract domains
used in developing the strategy model (Sec.~\ref{sec:strat:model}).

The strategy model part implements the core semantic framework,
including strategy semantics,
refinement conventions,
refinement squares,
and all associated compositional properties.

The OpenTE framework component implements spatial composition,
state encapsulation,
and stateful simulation built on top of
the CompCertO semantics.

The CompCertO embedding provides
the interface that integrates CompCertO semantics
into the strategy model,
enabling end-to-end verification.

Application-specific components
include the bounded queue verification,
which also incorporates support for memory separation (Sec.~\ref{sec:ox:separation}),
and the rot13 verification,
which includes the development of loaders (Sec.~\ref{sec:strat:loaders}).

The CAL component implements
the Certified Abstract Language framework
using the strategy model,
while the ClightP component encompasses
the development of ClightP semantics,
its compilation to Clight,
and the composition of correctness results.

The $\kw{coqrel}$ library was developed
entirely by Jérémie Koenig.
The CompCertO semantics
and the majority of the supporting code
were also developed primarily by him (though CompCertO builds upon the existing CompCert compiler),
with some proof contributions from myself.
The development of the strategy model
represents a joint effort between Jérémie and myself,
while the remaining components were written primarily by myself.
