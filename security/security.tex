\documentclass[11pt]{article}
\usepackage{geometry}
\usepackage{amsmath}
\usepackage{amssymb}
\usepackage{hyperref}
\usepackage{libertine}
\usepackage[libertine]{newtxmath}
\usepackage{stmaryrd}
\usepackage{cmll}
\usepackage{bbm}

\title{Security properties in CompCertO/RBGS}
\author{J\'er\'emie Koenig}

\begin{document}

\maketitle

\begin{abstract} %{{{
In this document I summarize how I understand
secure compilation
in the context of CompCertO and refinement-based game semantics.
In a model with dual nondeterminism,
properties and hyperproperties
can be formulated as specifications.
Both are preserved by simple refinement as a result,
but the ways in which specific (hyper)properties interact with
a compiler's calling convention $\mathbb{C}$
must be investigated on a case-by-base basis.

For example, \emph{nonintereference} is
a special case of hyperproperty.
But since assembly-level contexts can potentially observe details
which are not visible at the C level,
noninterefence is not automatically preserved by CompCertO.
However,
source-level noninterference induces a target-level property
of noninterference \emph{up to $\mathbb{C}$}
which can be characterized precisely.

Based on this characterization,
I show how to equip the target program
with call wrappers which eliminate
the potential information leaks introduced by compilation
and turn noninterference up to $\mathbb{C}$
back into the full-blown version of target-level noninterference.
\end{abstract}
%}}}

\section{Specification model} %{{{

What follows is a tentative formalization of the model $\mathsf{NTree}$.
This model is similar in spirit to the one used in our LICS '20 paper
and it defines some kind of free monad in $\mathbf{CDLat}$,
but unlike in the LICS model,
non-deterministic choices do not commute with external operations.

\subsection{Interaction specifications} %{{{

A computation with external operations in an effect signature~$E$
can be specified with an \emph{interaction specification} of the form:
\begin{equation} \label{eqn:construct}
  x \in \mathcal{I}_E ::=
    \bigvee_{i \in I} \bigwedge_{j \in J}
      m_{ij} \big( x_{ijk} \big)_{k \in N_{ij}}
  \qquad \text{where} \quad
  (m_{ij} \mathbin: N_{ij}) \in E
\end{equation}
The notations $\bigvee, \bigwedge$ are respectively meant to denote
angelic and demonic choice
constituting the joins and meet of a refinement lattice.
As such it will be natural to omit unary choices from notations,
and likewise denote nullary choices with the corresponding
upper and lower bounds:
\begin{align*}
  m(x_k)_k &:= \bigvee_{i \in \mathbbm{1}} \bigwedge_{j \in \mathbbm{1}} m(x_k)_k
  &
  \bigvee_{i \in I} m_i(x_k)_k &:=
    \bigvee_{i \in I} \bigwedge_{j \in \mathbbm{1}} m_i(x_{ik})_k
    & \bot &:= \bigvee_{i \in \varnothing} ()
  \\ &&
  \bigwedge_{j \in J} m_j(x_k)_k &:=
    \bigvee_{i \in \mathbbm{1}} \bigwedge_{j \in J} m_j(x_{jk})_k
    & \top &:= \bigwedge_{i \in \varnothing} ()
\end{align*}

Note that we define interaction specifications inductively,
so on any ``branch'' within a term
the constructor (\ref{eqn:construct})
can only be applied a finite number of times.
However,
because the index sets $I$ and $J$ can be infinite,
we are able to model infinite computations
by ranging over finite approximations.

%We could also use the less ``algebraic'' and more ``computational'' notation
%$
%  \bigvee_{i \in I} \bigwedge_{j \in J} k \leftarrow m_{ij} \mathbin; x_k
%$
%and corresponding abbreviations.

%}}}

\subsection{Refinement lattice} %{{{

Interaction specifications are equipped with a form of refinement
such that
\[
  \bigvee_{i \in I} \bigwedge_{j \in J_i}
  m_{ij}\big(x_{ijk}\big)_{k \in N_{ij}}
  \quad\le\quad
  \bigvee_{u \in U} \bigwedge_{v \in V_u}
  q_{uv}\big(y_{uvk}\big)_{k \in R_{uv}}
\]
holds when
\[
  \forall i \, \exists u \,
  \forall v \, \exists j \mathbin.
  (m_{ij} = q_{uv}) \wedge
  \forall k \mathbin. (x_{ijk} \le y_{uvk})
  \,.
\]
We will consider the interaction specifications $x$ and $y$
to be equal when both $x \le y$ and $y \le x$.

This ordering induces a completely distributive refinement lattice,
where joins and meets can be computed as expected.
Consider a family of interaction specifications $(x_i)_{i \in I}$
which can each be written as
$x_i = \bigvee_{j \in J_i} \bigwedge_{k \in K_{ij}} m_{ijk}(y_{ijkl})_l$.
We can give the supremum and infimum as:
\[
  \bigvee_{i \in I} x_i \:=\:
    \bigvee_{(i, j) \in \sum_i J_i} \,
    \bigwedge_{k \in K_{ij}} \,
    m_{ijk}(y_{ijkl})_l
  \qquad \qquad
  \bigwedge_{i \in I} x_i \:=\:
    \bigvee_{u \in \prod_i J_i} \,
    \bigwedge_{(i, k) \in \sum_i K_{i u_i}}
    m_{i u_i k}(y_{i u_i kl})_l
\]

%}}}

\subsection{Sequential composition} %{{{

Interaction specifications can used to build an interaction specification monad.
For an effect signature $E$ and a set $V$ we define
\[
  \mathcal{I}_E(V) \: := \: \mathcal{I}_{E \oplus \{\eta v \mid v \in V \}}
  \,,
\]
which is the set of interaction specifications
over a version of the signature $E$ which has been extended
with a nullary operation $\eta v : \varnothing$
for each of the variables $v \in V$.
The action of $\mathcal{I}_E(-)$ on functions
transforms those terminal operations:
\begin{gather*}
  \mathcal{I}_E(f)\Big( \bigvee_i \bigwedge_j m_{ij}(x_{ijk})_k \Big) :=
    \bigvee_i \bigwedge_j \bar{f}\big(m_{ij}(x_{ijk})_k \big)
  \\
  \text{where}
  \quad
  \bar{f}\big(m(x_k)_k\big) =
  \begin{cases}
    \eta[f(v)] () & \text{if } m = \eta v () \\ 
    m(\mathcal{I}_E(f)(x_k))_k & \text{otherwise}
  \end{cases}
\end{gather*}
Then the monad's unit
$\eta : V \rightarrow \mathcal{I}_E(V)$ and multiplication
$\mu : \mathcal{I}_E(\mathcal{I}_E(V)) \rightarrow \mathcal{I}_E(V)$
are defined as
\begin{align*}
  & \eta(v) := \eta v () & \text{and} \quad &
  \mu \Big( \bigvee_i \bigwedge_j m_{ij}(x_{ijk})_k \Big) :=
    \bigvee_i \bigwedge_j \bar{\mu} \big( m_{ij}(x_{ijk})_k \big) 
 \\
  && \text{where} \quad & \bar{\mu}\big(m(x_k)_k\big) :=
  \begin{cases}
    x & \text{if } m = \eta x () \\
    m\big(\mu(x_k)\big)_k & \text{otherwise.}
  \end{cases}
\end{align*}

%}}}

\subsection{Strategy specifications} %{{{

We can give an interpretation of a signature $F$
by giving for each $(m \mathbin: N) \in F$
a~specification $\sigma_m \in \mathcal{I}_E(N)$
in another signature $E$.
We will say that
\[
  \sigma \in \prod_{(m \mathbin: N) \in F} \mathcal{I}_{E}(N)
\]
implements the signature $F$ in terms of the signature $E$,
and write $\sigma : E \rightarrow F$.

Combining $\sigma : E \rightarrow F$ with
a client specification $x \in \mathcal{I}_F$
yields a result $x\sigma \in \mathcal{I}_E$,
where every external operation occurring in $x$
is replaced with its implementation in $\sigma$.
Formally,
\[
  \Big( \bigvee_i \bigwedge_j m_{ij}\big(x_{ijk}\big)_k \Big) \sigma
  \::=\:
  \bigvee_i \bigwedge_j
    \big(k \mathbin\leftarrow \sigma_{m_{ij}} \mathbin; x_{ijk} \sigma \big)
  \,,
\]
where $v \mathbin\leftarrow x \mathbin; y$
denotes the Kleisli extension
$\mu \circ \mathcal{I}_E(v \mapsto y)$
being applied to $x$,
in this case instructing to continue the computation with $x_{ijk} \sigma$
if $\sigma_{m_j}$ yields the value $k$.

The operation described above serves as the basis for strategy composition.
Given the strategy specifications
$\sigma : E \rightarrow F$ and $\tau : F \rightarrow G$,
the composition $\tau \circ \sigma : E \rightarrow G$ is defined by
\[
  (\tau \circ \sigma)_m \: := \: \tau_m \sigma \: \in \: \mathcal{I}_E(N)
  \,,
  \qquad \text{where }
  (m \mathbin: N) \in G
  \,.
\]

%}}}

\subsection{Deterministic computations}

%}}}

\section{Security properties as specifications} %{{{

The specification model outlined in the previous section
departs from usual trace semantics approaches,
resembling free monad constructions like interaction trees more closely.
Nevertheless,
I will show below that
traditional trace properties and hyperproperties
can be reformulated as specifications within that model.

\subsection{Traces and properties} %{{{

As a system executes, we can record in a \emph{trace}
the observable interactions between the system and its environment
(and sometimes record the evolution of the system's internal state).
Trace semantics model a system's behavior as
the set of traces which may be observed as the system executes.

Under this traditional setup,
desirable aspects of system behavior
can be described as \emph{trace properties}.
A trace property is simply a predicate which identifies
acceptable traces.
A system satisfies a trace property when
every one of its traces is acceptable.


%}}}

\subsection{Safety properties} %{{{

\emph{safety} property:
\[
  \bigvee_{q_1} \bigwedge_{r_1 \mid q_1 r_1 \in P}
  \bigvee_{q_2} \bigwedge_{r_2 \mid q_1 r_1 q_2 r_2 \in P}
  \cdots
\]

\[
  \bigwedge_{i \in I} m_i \Big(
  \bigwedge_{j \in J_{iu}} m_{iuj} \Big(
  \bigwedge_{k \in K_{iujv}} m_{iujvk} \Big(
  \cdots
  \Big)_w
  \Big)_v
  \Big)_u
\]

\begin{align*}
  \phi P &:= \{ q \mid qrs \in P \}  \quad \text{(first action)} \\
  qr \backslash P &:= \{ s \mid qrs \in P \} \quad \text{(residual)}
\end{align*}

\[
  \Sigma : \mathbf{Prop} \rightarrow \mathbf{Spec}
  \qquad \qquad
  \Sigma := \mu F \mathbin. \bigvee_{q \in \phi[P]}
    \underline{q} \big( F(\rho_{qr}[P]) \big)_r
\]

%}}}

\subsection{Hyperproperties}

%}}}

\section{Data abstraction and noninterference}



\section{Secure compilation}

We have seen that properties and hyperproperties
are special cases of RBGS specifications.
As such they are preserved by simple refinement.
However,
the data abstraction component of a compiler's correctness property





\section*{Old model}

The specification model has:
\begin{itemize}
  \item finite traces of the form $q_1 r_1 q_2 r_2 \cdots q_n r_n$;
  \item a refinement ordering $\le$, with
  \item unbounded angelic choices $\bigvee_{i \in I} x_i$ and
  \item unbounded demonic choices $\bigwedge_{i \in I} x_i$.
\end{itemize}
The \emph{finite trace} $q_1 r_1 q_2 r_2 \cdots q_n r_n$
can be interpreted as an elementary specification requiring that:
\begin{quote}
  If the system is invoked with a question $q_1$,
  its response will be $r_1$;
  if it is then invoked with a new question $q_2$,
  its response will be $r_2$,
  and so on.
\end{quote}
The \emph{refinement} $\sigma \le \tau$
means that the specification $\tau$ is more strict than $\sigma$.
In other words $\tau$ imposes at least the same requirements as $\sigma$,
and any system which satisfies $\tau$ will satisfy $\sigma$ as well:
\begin{align*}
  \sigma \le \tau
  \:\:\Longleftrightarrow\:\:
  \mathrm{Requirements}(\sigma) &\subseteq
  \mathrm{Requirements}(\tau)
\end{align*}
From the definitions above we can see
the prefix ordering of traces is a special case of refinement:
\[
  q_1 r_1 \:\le\: q_1 r_1 q_2 r_2 \:\le\: \cdots \:\le\: q_1 r_1 q_2 r_2 \cdots q_n r_n
  \,,
\]
since longer traces impose additional requirements
compared with their prefixes.


\paragraph{Refinement lattice}

Specifications constitute a complete lattice under the refinement ordering.
\emph{Angelic} choices let the \emph{user} of the system
pick which specification they wish to rely on,
and \emph{demonic} choices let the \emph{implementer} pick.
They are the join and meet of the refinement lattice:
\[
  \bigvee_{i \in I} \sigma_i \:\le\: \tau
  \:\:\Leftrightarrow\:\:
  \forall i \in I \mathbin. \sigma_i \le \tau
  \qquad \qquad
  \sigma \:\le\: \bigwedge_{i \in I} \tau_i
  \:\:\Leftrightarrow\:\:
  \forall i \in I \mathbin. \sigma \le \tau_i
\]
Note that
an infinite trace $t$ can be represented as
a limit $\bigvee \mathsf{FP}(t)$ of its finite prefixes
$s \in \mathsf{FP}(t)$.

\paragraph{Implementations}

Usually,
the point of a specification $\sigma$ is to
eventually construct a system $x$ which satisfies it.
We use the same kind of mathematical objects
to represent specifications and implementations,
and will state the correctness of $x$ with respect to $\sigma$
as the refinement $\sigma \le x$.
However,
not every specification represents a possible implementation.

Compared with general specifications,
implementations exhibit a well-defined behavior
which can only be influenced by observable inputs.
In other words,
implementations take the form
$x = \bigvee S$
where $S$ is a set of pairwise coherent traces.
Traces are coherent when
the same inputs lead to the same outputs:
\[
  s \coh \epsilon \qquad\qquad
  \epsilon \coh t \qquad\qquad
  q_1 r_1 s \coh q_1' r_1' t
  \: :\Leftrightarrow \:
  (q_1 = q_1' \Rightarrow r_1 = r_1' \wedge s \coh t)
\]
I will write
\[
  \mathbf{Impl} := \Big\{ \bigvee S \mathrel\Big|
    \forall s, t \in S \mathbin. s \coh t  \, \Big\}
  \qquad \text{and} \qquad
  \mathrm{Impl}(\sigma) :=
    \{ x \in \mathbf{Impl} \mid \sigma \le x \}
  \,.
\]
We can observe that:
{\small \[
  \sigma \le \tau
  \:\:\Longrightarrow\:\:
  \mathrm{Impl}(\sigma) \supseteq
  \mathrm{Impl}(\tau)
\qquad
  \mathrm{Impl} \Big( \bigvee_{i \in I} \sigma_i \Big) \: = \:
  \bigcap_{i \in I} \, \mathrm{Impl}(\sigma_i)
\qquad
  \mathrm{Impl} \Big( \bigwedge_{i \in I} \sigma_i \Big) \: = \:
  \bigcup_{i \in I} \, \mathrm{Impl}(\sigma_i)
\]}%
However,
specifications are not completely characterized
by their set of implementations,
and even nonimplementable specifications
may be non-trivial.
For example, the specification
\[
  \sigma = q r \vee q' r_1 \vee q' r_2
  \qquad \text{where} \qquad
  r_1 \neq r_2
\]
is not implementable,
but it could appear as a component in a larger specification
$\tau \circ \sigma$ which is.
For example $\tau$ may not invoke the overconstrained $q'$ at all,
or could react in the same way to the answers $r_1$ and $r_2$.



\end{document}
