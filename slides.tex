\documentclass[handout]{beamer}
\usepackage{bussproofs}
\usepackage{stmaryrd}

\title{Lightweight Compositional Compcert}
\subtitle{a POPL paper pitch}
\author{J\'er\'emie Koenig}

\newcommand{\EC}{\text{EC}}
\newcommand{\nEC}{\overline{\EC}}
\newcommand{\ECEC}{\EC \Rightarrow \EC}
%\newcommand{\ECEC}{\bar{*} \vee *}
\newcommand{\kw}[1]{\texttt{#1}}
\newcommand{\word}[1]{\mathbf{#1}}
\newcommand{\sys}{\textbf{sys}}
\newcommand{\env}{\textbf{env}}
\newcommand{\ident}{\kw{ident}}
\newcommand{\val}{\kw{val}}
\newcommand{\mem}{\kw{mem}}
\newcommand{\Y}{\mathbf{Y}}
\newcommand{\E}{\mathcal{E}}

\setlength{\parskip}{1ex}

\AtBeginSection[]
{
   \begin{frame}
        \tableofcontents[currentsection]
   \end{frame}
}

\begin{document}

\begin{frame}
\titlepage
\end{frame}

\begin{frame}{Bottom line} %{{{
Deploy the same ideas we use for thread linking
in the context of separate compilation:
\begin{itemize}
\item Define a game semantics for Clight and Asm
\item Game semantics + extcall oracle = original semantics
\item CompcertX just turned into a compositional complier
\end{itemize}
\end{frame}
%}}}

\section{Game semantics for Compcert}

\begin{frame}{Games} %{{{
We build up from the HO arena and game $\EC$ defined by:
\begin{gather*}
M_\EC = \{ \kw{call}[f, \vec{v}, m] \} \cup
        \{ \kw{ret}[f, \vec{v}, m, v, m'] \} \\
\lambda_\EC(\kw{call}[\ldots]) = \kw{OQ} \hspace{3em}
\lambda_\EC(\kw{ret}[\ldots]) = \kw{PA} \\
* \vdash_\EC \kw{call}[\ldots] \hspace{2em}
\kw{call}[f, \vec{v}, m] \vdash_\EC \kw{ret}[f, \vec{v}, m, v, m']
\end{gather*}
Schematically:
\begin{center}
[figure]
\end{center}
Example play:
\begin{center}
[figure]
\end{center}
\end{frame}
%}}}

\begin{frame}{Type of modules} %{{{
We will interpret the semantics of modules in the game $\ECEC$:
\begin{center}
[figure]
\end{center}
Example play:
\begin{center}
[figure]
\end{center}
\end{frame}
%}}}

\begin{frame}{Linking operator} %{{{
For $\sigma_1, \sigma_2 : \ECEC$,
we want to define:
\[ \sigma_1 \bowtie \sigma_2 \]
It should be a straighforward combination
of $\circ$, $\cup$, and \kw{lfp},
though stronger typing may be required
in order to hide the implemented calls.

Linking theorem:
\[
        \AxiomC{$\sigma_1 \le_R \sigma_1'$}
        \AxiomC{$\sigma_2 \le_R \sigma_2'$}
        \BinaryInfC{$\sigma_1 \bowtie \sigma_2 \le_R \sigma_1' \bowtie \sigma_2'$}
        \DisplayProof
\]
($\le_R$ to be defined later)
\end{frame}
%}}}

\begin{frame}{Semantics} %{{{
For a Clight module $M$ with operational semantics $\rightarrow$,
the strategy $\llbracket M \rrbracket : \ECEC$ is defined by:
\[
	\AxiomC{$\kw{replay}(\kw{view}(\word{u}))
		\rightarrow^?
		\kw{Callstate}(f, \vec{v}, k, m)$}
	\UnaryInfC{$\llbracket M \rrbracket \ni
		\word{u} \cdot \kw{call}(f, \vec{v}, m)$}
	\DisplayProof
\]
\[
	\AxiomC{$\kw{replay}(\kw{view}(\word{u}))
		\rightarrow^?
		\kw{Returnstate}(v, \kw{K0}, m)$}
	\UnaryInfC{$\llbracket M \rrbracket \ni
		\word{u} \cdot \kw{ret}(v, m)$}
	\DisplayProof
\]

$\kw{replay}(\kw{view}(\word{u}))$ is the state obtained
by replaying the position $\word{u}$. \\
$\rightarrow^{?}$ is $\rightarrow$ iterated
until the first $\kw{Callstate}$ to an external function
or $\kw{Returnstate}$ to the null continuation.

The resulting strategies are innocent and well-bracketed.
\end{frame}
%}}}

%\begin{frame}{Fixpoint} %{{{
%Note that $\llbracket M \rrbracket$
%does not even service its own internal calls,
%instead relying on the environment to do that.
%
%We can introduce a fixpoint operator $\Y$:
%\[
%	\AxiomC{$\sigma : \ECEC$}
%	\UnaryInfC{$\Y \sigma : \EC$}
%	\DisplayProof
%\]
%The strategy $\Y \sigma$ is the composition
%of an infinite sequence of copies of $\sigma$,
%each using the next one to handle its function calls:
%\[ \text{[figure]} \]
%This way we can link strategies with $\cup$
%and don't need to worry about the domain of each module:
%the behavior whole program semantics is $\Y ( \llbracket M_1 \rrbracket \cup \llbracket M_2 \rrbracket \cup \cdots )$.
%\end{frame}
%%}}}

\section{Compcert KLRs}

\begin{frame}{Kripke logical relations} %{{{
\emph{Logical relations} are structure-preserving relations
that can be combined and built up along the type structure,
so that relations at all types can be obtained
from relations at base types.

\emph{Kripke} logical relations
are indexed over a set of ``possible world'',
so that a KLR is defined by:
\begin{itemize}
\item a set $W$ of \emph{worlds};
\item an \emph{accessibility relation} $(\leadsto) : \mathcal{R}(W, W)$ (often a preorder);
\item for each type $A$ and each world $w$
  a logical relation $R \llbracket A \rrbracket_w : \mathcal{R}(A_1, A_2)$ between the interpretations
  $A_1$, $A_2$ of $A$.
\end{itemize}

$w \leadsto w'$ can be read as: \emph{$w$ might evolve into $w'$}.
\end{frame} %}}}

\begin{frame}{Compcert KLRs} %{{{
My $\kw{simrel}$'s are Kripke logical relations for Compcert. \\
Roughly speaking, each one defines:
\begin{itemize}
\item a set of worlds (say, $\kw{meminj}$),
\item an accessibility relation (say, $\kw{inject\_incr}$),
\item world-indexed relations on values (say, $\kw{Val.inject}$)
  and memory states (say, $\kw{Mem.inject}$),
\end{itemize}
so that we can build up the corresponding world-indexed relation
for pretty much anything in Compcert
(in particular, states).
\end{frame}
%}}}

\begin{frame}{Compcert parametricity} %{{{
Expressed in this framework,
the monotonicity of Compcert semantics
is a \emph{parametricity theorem},
which gives us many interesting properties of the semantics
``for free''
(by plugging in an appropriately designed $\kw{simrel}$):
\begin{itemize}
\item stability under injection (\kw{inj});
\item preservation of invariants (\kw{inv});
\item \kw{Mem.nextblock} is increasing (``strong \kw{id}'');
\item \ldots
\end{itemize}

In fact,
the required properties of external calls
can nearly all be put in this form,
and can therefore be shown to be preserved by the semantics.
\end{frame}
%}}}

\begin{frame}{KLRs for strategies} %{{{
Let $(W, \leadsto, R)$ be a Compcert KLR.
Extending $R_w$ to the moves of $\EC$,
then of $\ECEC$ is straightforward.
From there we can define the simulation diagrams
for strategies.

Two \emph{justified sequences} are related if:
\begin{itemize}
\item They have the same overall structure;
\item For each corresponding pair of moves,
  there is a world $w$ such that they are related by $R_w$;
\item If a pair of moves with world $w$ justifies
  a pair of moves with world $w'$,
  then $w \leadsto w'$.
\end{itemize}

Two strategies are related if they take justified sequences
to justified sequences when played.

Linking theorem: preserved by
$\cup$, $\circ$, $\kw{lfp}$, therefore $\bowtie$.
\end{frame}
%}}}

\section{Separate compilation}

\begin{frame}{A problem?} %{{{
Compositional CompCert Princeton paper:
\begin{quote}
A deficiency of CompCert’s simulation proofs and of LSRs
was that they assumed conditions (1) and (2) at external calls,
but did not prove that these properties were preserved by compilation.
Directly imposing constraints (1) and (2)
onto the simulation clauses for internal steps does not work, however.
A compiled function should be allowed to
write to its own spill locations ---
just not to those of its caller.
\end{quote}
The go on to define their complicated \emph{structured injections}.
\end{frame}
%}}}

\begin{frame}{A solution!} %{{{
But because of the way we lay out our KLR accessibility relation
onto the justification structure of HO games,
our simulations already allow a function
to write to its own spill locations!

[Explain]

Not magical because we still have to prove the thing in question
(but we have Yuting and Pierre for that).

But at least our simulations express exactly the right constraints
in a very natural way, which is encouraging.
\end{frame}
%}}}

\begin{frame}{Connecting with the CompcertX theorem} %{{{
CompcertX's theorem should be expressible as:
\[
  \forall \E : \EC,
    \text{nonreentrant}(\E) \wedge
    \text{determined}(\E) \wedge
    \E \le_R \E \Rightarrow
    \llbracket M \rrbracket^\kw{Clight} \circ \E \le_R
    \llbracket C(M) \rrbracket^\kw{Asm} \circ \E
\]
where $R$ is a specific relation
(which includes the external call requirements).

We should be able to generalize it to:
\[
    \llbracket M \rrbracket^\kw{Clight} \le_R
    \llbracket C(M) \rrbracket^\kw{Asm}
\]
using the properties of our strategies
(in particular innocence).
\end{frame}
%}}}

\begin{frame}{Linking with fancy contexts} %{{{
Note that the generalized theorem allows us to link with
all kinds of contexts:
the only criterion is stability under $\le_R$.
This allows us to call things like \kw{longjmp} and \kw{cswitch},
implemented in assembly,
directly from C code.

A way to express the semantics of those is already
an improvement on the current Compcert,
and we could demonstrate by verifying a small program
which uses them.
\end{frame}
%}}}

\end{document}

