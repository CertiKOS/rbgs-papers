\section{Multi-module programs and horizontal composition}

The semantics introduced in the previous sections
allows us to characterize the behavior of an isolated
CompCert module,
however we have not yet given an account of
how modules interact with one another.
In this section,
we give a compositional semantics
for systems consisting of several CompCert modules.

\subsection{Assumptions}

We take as given a module syntax and semantics
such as the one defined in \S\ref{sec:modsem}:
\begin{itemize}
\item An elementary game $A$ specifies the form of the interactions
  between a module and its environment;
\item A set of programs $P$ defines the syntax of individual modules;
\item A semantic operator
  $\llbracket - \rrbracket :
   P \rightarrow \mathcal{P}(M_A^\kw{Q}) \times \mathcal{G}(A, A)$
  gives the interpretation of each module $M \in P$
  in terms of the set of questions it refuses
  and a strategy specifying its interactive behavior.
\end{itemize}
Furthermore,
we require the semantic objects to be well-formed
so that it satisfies the following properties:
\begin{itemize}
\item A module may only ask questions which it itself refuses;
  in other words it may not invoke itself.
\item A module may not specify any behavior for the questions
  it refuses.
\end{itemize}

To formalize these conditions,
we introduce the following construction.

\begin{definition}[Restriction]
For an elementary game $A$
and set of questions $X \subseteq M_A^\kw{Q}$,
the strategy $[X] : A \rightarrow A$
is defined as $[X] := \langle \{*\}, \delta_{[X]}, * \rangle$,
with $\delta_{[X]}$ defined by the rule:
\[
    m \in X \cup M_A^\kw{A} \vdash \delta_{[X]}(*, m) \ni (m, *)
\]
\end{definition}

That is,
for questions $m \in X$,
the strategy $[X]$ acts as the identity,
but questions $m \notin X$ are simply ignored.
Note that $[-]$ enjoys the following properties:
\begin{itemize}
\item $[\varnothing] \equiv \bot$, $[M_A^\kw{Q}] \equiv \kw{id}$;
\item $X \subseteq Y \Rightarrow [X] \sqsubseteq [Y]$;
\item $[X \cap Y] \equiv [X] \cdot [Y] \equiv [Y] \cdot [X]$;
\item $[X \cup Y] \equiv [X] \oplus [Y]$.
\end{itemize}

\begin{definition}
The semantic object
$(X, \sigma) \in \mathcal{P}(M_A^\kw{Q}) \times \mathcal{G}(A, A)$
is \emph{valid}
if the following properties hold:
\[
  \sigma \cdot [X] \sqsubseteq \bot \qquad
  [X] \cdot \sigma \sqsubseteq \sigma
\]
XXX: this does not prevent $\sigma$
from reacting to a question in $X$
once a number of questions in $\bar{X}$ have been asked.
Maybe the property is not needed though.

The set of valid semantic objects is defined as:
\[
    \mathbb{D}_A := \{
      (X, \sigma) \in \mathcal{P}(M_A^\kw{Q}) \times \mathcal{G}(A, A) \mid
      \sigma \cdot [X] \subseteq \bot \wedge
      [X] \cdot \sigma \subseteq \sigma
    \}
\]
\end{definition}

\subsection{Horizontal composition}

To model cross-module interactions,
we define the following operation on
the elements of $\mathbb{D}_A$.

\begin{definition}
The \emph{horizontal composition} of
$(X, f) \in \mathbb{D}$ and $(Y, g) \in \mathbb{D}$
is defined as:
\[
  (X, f) \bullet (Y, g) :=
    (X \cap Y, [X \cap Y] \cdot (f \oplus g)^\oplus)
\]
\end{definition}

The commutativity of $\bullet$ follows
from that of $\cap$ and $\oplus$.

Associativity:
By KA except it doesn't work when there's state.

