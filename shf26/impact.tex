\section{Broader Impacts}
\label{sec:impact}

\parhead{Technology Transfer}
The proposed project aims at developing a compositional verification
toolchain for C and related languages and creating a {\em
  certified} application binary interfaces (ABIs) for future
trustworthy heterogeneous systems. Such certified toolchain and ABIs
will greatly facilitate the verification of large-scale system
software, which in turn have a profound impact on the software
industry and the society in general.
Artifacts resulting from the project will be
made open-source to ensure rapid dissemination of ideas.

We plan to aggressively pursue technology transfer through several
different channels. First, PI Shao is the co-founder of CertiK, a
startup focusing on formal verification and auditing of smart
contracts and blockchain ecosystems.  CertiK has retargeted DeepSEA to
build verified smart contracts and developed a new verified compiler
backend for generating EVM code. We believe that our new compositional
verified compilation technology can be readily applied to greatly
enhance the power of this version of DeepSEA as well.
%%%
Second, together with his colleagues, under the DARPA V-SPELLS
program, PI Shao has been working on a Partitioned Trusted Execution
Environment firmware-level OS kernel (named PARTEE) for cyber-physical
systems with large legacy code base.  PARTEE will be deployed on the
self-driving car and UAV platforms to build ARM-TrustZone-based secure
enclaves. The compositional verification and compilation technologies
developed in this proposal will be tested and transferred into
realistic case studies under the PARTEE roject.

\parhead{Curriculum Development}
We plan to develop innovative certified system software and formal
verification curriculum at both the undergraduate and graduate
level. The two PIs plan to develop a new course on Formal Semantics
(CPSC 430) at Yale based on the Nominal CompCertO developed in this
project. Existing courses and textbooks on formal semantics focus on
somewhat outdated material and do not cover exciting new technologies
such as game semantics, nominal techniques, and verified
compilation. By using our verified compiler toolchain and common
composition patterns in CertiKOS as the basis, we plan to
create a set of programming labs (including specifying, programming,
compiling, verifying, and linking certified abstraction layers) where
students can learn modern compositional techniques through a more
hands-on approach.

PI Shao also plans to revamp his course on Language-Based Security
(CPSC 428) based on the verified OS kernel (CertiKOS) and hypervisor
(SeKVM) and compiler toolchain (Nominal CompCertO) and certified
programming tools (DeepSEA and VST) where students can gain hands-on
experience on both system design and verification.  Language-Based
Security aims to completely eliminate specific classes of
vulnerabilities introduced by the semantics of OS
kernels, virtual machines, and programming languages. Language-based
technologies, including program verification, formal methods, static
analyses, interpreters, and compilers are the key
components of next generation security systems. The newly revamped
course will survey the most promising compositional
techniques and show how to apply them to build
dependable system software.  We will
make the courses accessible to undergraduate students so they
can participate in the related
research. We will make our course materials freely available and
encourage our colleagues to use them at other universities.

\parhead{Community Building and Outreach.}  We plan to build a robust
community of researchers and practitioners through which the research
results and prototypes of this project can be timely disseminated and
leveraged by both academia and industry.  In Year 1, we will organize
a virtual monthly forum on the broad topic of compositional
formal verification for heterogeneous systems. 
Each forum will last an hour, offering one informal colloquium where a
speaker can present their past or ongoing research, share future
visions, and initiate discussion and interactions.  Once the monthly
forum grows, in Years 2-3, we will progress the forum into a yearly
in-person one-day workshop (colocated with a major programming
languages conference).  The workshop will include short presentation
sessions of visions and breakthrough results, tutorial sessions, a
panel, and a poster session.  Still, the presentations and panel will
be offered virtually at the same time to allow monthly forum members
to present and interact. Finally, we will develop a central website
portal to manage and advertise these activities. The portal will
contain training materials and text and video tutorials on how to use
LiDO*-related technologies, as well as examples in real-world
settings. It will also host the monthly forum and workshops, and will
serve as an avenue for advertising upcoming events and speakers and
store the past presentations.




