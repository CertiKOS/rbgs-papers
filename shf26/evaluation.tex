\section{Evaluation and Integration}

To evaluate and demonstrate the power and real-world applicability of
our new verification and verified compilation framework,
we will apply it to enhance existing and
build more advanced certified systems and programming tools.
In particular,
we will use the extended capabilities of our
certified toolchain to expand the scope and capabilities
of the correctness proof for the CertiKOS operating system kernel,
and demonstrate applications to the construction of
end-to-end certified heterogeneous systems.

\subsection{Concurrent Certified Abstraction Layers}

CertiKOS~\cite{certikos-osdi16} is a certified concurrent OS kernel
that supports fine-grained locking and can boot Linux in its guest VM
running on stock multicore x86 hardware.  CertiKOS is built from a
collection of certified concurrent abstraction layers
(CCAL)~\cite{ccal18} written in C or assembly.  The thread-safe
variant of CompCertX~\cite{dscal15} used by CCAL, however, suffers
from many limitations: it is not end-to-end and does not general
machine code; it does not support concrete machine-level finite
stacks; and it does not support certified linking (and loading) of
user- and kernel-level components. We plan to show that our new
verified compositional compiler can indeed address all these problems
and allow us to build much more powerful OS kernels.

Another target of application is SeKVM~\cite{sekvm21a,sekvm21b,tao21},
a new secure and formally verified Linux KVM Hypervisor developed at
Columbia University. SeKVM uses the same CCAL-based layered technology 
but only does the verification at the C source code level. We plan to
apply our new compositional verified compiler to enhance SeKVM and support
more extensible hypervisors with smaller TCBs and richer features.

Both CertiKOS and SeKVM verify the information-flow security of their
entire systems by first verifying a generalized noninterference
property (with declassification)~\cite{costanzo16} over the abstract
(deep) specification of their system-call layers, and then relying on
security-preserving contextual refinement to propagate the security
guarantees to cover the entire C and assembly implementation. The
initial single-core version of CertiKOS~\cite{costanzo16}, in
particular, used CompCertX~\cite{dscal15} to construct an end-to-end
security proof all the way to the assembly level.  We plan to explore
a similar approach in Normal CompCertO so it can also build end-to-end
security proofs for CertiKOS and SeKVM.

\subsection{More Accurate Process Model for CertiKOS} \label{sec:userspace}

In the CompCert assembly semantics which
the correctness proof of CertiKOS relies on,
the program is fixed and kept separate from the memory state.
As a result,
the correctness of CertiKOS is proved with respect to
an arbitrary but fixed, pre-loaded user-space program.

By contrast,
CompCertELF models CPU instruction decoding
and assigns semantics to concrete, binary machine code.
This means
we will be able to offer a much more complete and accurate model
of process execution in CertiKOS,
including system calls such as \texttt{exec}
which load binary process images from the file system.
In this context,
the CertiKOS specification could be given as a strategy
$
  \kw{CertiKOS} : \mathcal{E} \twoheadrightarrow \top \mathbin@ \mathcal{F}
$,
where the signature
$\top := \{ * : \varnothing \}$ contains a single operation
use to initiate the system's execution,
the set $\mathcal{F}$ describes the possible file system states,
and $\mathcal{E}$ describes the way the system interacts
with its environment (such as messages printed on the console).
A user of the specification could then
evaluate the behavior of a CertiKOS machine
started with certain commands stored as ELF binaries
on its file system,
formalizes as:
\[
  (\top \mathbin@ [ \mathtt{/sbin/init} \mapsto \cdots,
                   \mathtt{/bin/sh} \mapsto \cdots \rangle)
  \odot \kw{CertiKOS}
  : \mathcal{E} \twoheadrightarrow \top
  \,.
\]
and the correctness proof would allow this specification
to be replaced by the kernel's implementation.

\subsection{More Accurate Hardware Model for CertiKOS} \label{sec:hwm}

In the same vein,
the facilities offered by CompCertELF would allow us to formulate
a more accurate hardware model against which CertiKOS would be proven correct.
The low-level machine semantics in CompCertELF could be used to define
a component
$
  \kw{CPU} : \kw{IO} \twoheadrightarrow \top \mathbin@ \mathcal{M}
$.
This component would model the machine in terms of
an initial physical memory state in $\mathcal{M}$,
containing for example the kernel image initialized by a bootloader.
A UART model of the kind outlined in Example~\ref{ex:nicdriver}
could be used to complement this hardware model,
and a situation where CertiKOS runs on this hardware could be described as:
\[
  (\top \mathbin@ [\kw{certikos.elf}\rangle) \odot 
  \kw{CPU} \odot \kw{UART} : \kw{Ser} \twoheadrightarrow \top
  \,.
\]
The resulting strategy would describe communication with the machine
over its serial line
when it is started with the CertiKOS binary as its kernel image.
In the process,
the techniques described in Example~\ref{ex:nicdriver}
could be used to verify the UART driver in CertiKOS and
remove it from the trusted computing base.

\subsection{Proposed Work}

\paragraph*{Task 4a: Port the CertiKOS proof to this new setting}

We will be to adapt our CertiKOS work
to the setting of our new framework and compilation toolchain,
starting with the sequential version of the correctness proof.
Because our new framework can be used express
the certified abstraction layer constructions found in \cite{dscal15},
this task should be relatively straightforward.
In a second phase,
we will formalize the CCAL proof techniques \cite{ccal18}
in the context of our framework,
using the extensions provided by Nominal CompCertO.

\vspace*{-2ex}
\paragraph*{Task 4b: Build a more realistic machine model to verify CertiKOS against}

We will formalize the hardware model discussed above in \S\ref{sec:hwm}
and use it to extend the correctness proof of CertiKOS
all the way to its binary ELF image.
%Compared with Task~4b,
%this task should be relatively easier,
%as it will likely consist only of \emph{adding} more steps
%to the proof,
%as enabled by CompCertELF,
%but should not require modifying the existing base proof extensively.

\vspace*{-2ex}
\paragraph*{Task 4c: Build a more realistic user process model for CertiKOS}

We have outlined above the ways in which
the facilities provided by CompCertELF and the work described in
\S\ref{sec:compcertelf}
would bring new possibilities for the way in which
the CertiKOS proof handles user processes.
We will investigate
enhancing the CertiKOS specification and correctness proof
along the lines described in \S\ref{sec:userspace}.

%This task would be moderately difficult,
%as it would require modeling some kind of file system
%and incorporating the certified ELF loader developed in Task~3c
%into the kernel itself.
%However,
%a simple in-memory read-only file system
%incorporated directly into the \texttt{certikos.elf} image
%would be sufficient for this purpose.
%An interesting aspect of this task would also be
%to investigate what abstract, high-level view of the file system
%and its binary commands would be most convenient
%when reasoning about shell scripts for example,
%and how our refinement convention infrastructure and compilation toolchain
%could participate in connecting it
%to the actual low-level, binary version being run.

\vspace*{-2ex}
\paragraph*{Task 4d: Verify the UART console driver}

To complete the task of providing a more realistic hardware model,
we will also formally describe the operation of a standard UART
as suggested in Example~\ref{ex:nicdriver},
and verify the CertiKOS UART console driver against it.

\vspace*{-2ex}
\paragraph*{Task 4e: Model a situation involving multiple machines}

As a final challenge,
we will attempt to model and reason about
a situation where multiple machines running CertiKOS
and specific user space program
would communicate and interact with each other,
for example over their serial consoles.
This grand challenge will exercise every part of our work,
and would set the stage for future research
where the operation of entire networks 
could be modeled and proven correct.

