\documentclass{beamer}
\usepackage{stmaryrd}
\usepackage{galois}
\usepackage{bussproofs}
\newcommand{\bind}{\gg\!\!=}

% Macros {{{
\newcommand{\kw}[1]{\ensuremath{ \mathsf{#1} }}
\newcommand{\ifr}[1]{\ [{#1}]\ }
\newcommand{\ifrw}[2]{\ [{#1}]_{#2}\ }

\newcommand{\EC}{\kw{C}}
\newcommand{\simrel}{\kw{simrel}}
%}}}

\AtBeginSection[]
{
   \begin{frame}
        \tableofcontents[currentsection]
   \end{frame}
}

\begin{document}

\section{Introduction} %{{{

\begin{frame}{Refinement-based game semantics} %{{{
[A prospective ``glue'' for end-to-end verification]
\end{frame}
%}}}

\begin{frame}{Challenges with CompCert} %{{{
[Why CompCert is key component,
what it takes to
make it a ((certified component) compiler) not a
(certified (component compiler))]
\end{frame}
%}}}

\begin{frame}{Our approach} %{{{
Compared to Compositional CompCert,
our version uses:
\begin{itemize}
\item \emph{games} to specify
  the form of the interaction
  for each language;
\item \emph{simulation conventions} to
  establish a correpondance between the
  interactions of the source and target languages.
\end{itemize}

\vspace{1ex}
We use a \emph{logical relations} approach
to construct the conventions:
\begin{itemize}
\item \emph{CompCert KLRs} generalize memory injections and extensions;
\item \emph{parametricity theorems}
  establish properties of languages;
\item A simulation convention algebra
  can be used to reconcile the simulation conventions for
  incoming and external calls.
\end{itemize}
\end{frame}
%}}}

%}}}

\section{Transition systems} %{{{

\begin{frame}{Games} %{{{
A game $A = \langle M_A^Q, M_A^A \rangle$ specifies:
\begin{itemize}
\item a set $M_A^Q$ of \emph{questions} (function invocations), and
\item a set $M_A^A$ of \emph{answers} (corresponding returns).
\end{itemize}

\vfill
For example, the game $\mathcal{C}$ has:
\begin{itemize}
\item questions of the form $m \in M_\mathcal{C}^Q ::= f/\sigma(\vec{v})@m$;
\item answers of the form $n \in M_\mathcal{C}^A ::= v'@m'$.
\end{itemize}

\vfill
A component of type $A \Rightarrow B$
will accept incoming calls according to the game $B$ and
may perform external calls according to the game $A$.
For instance,
Clight components have type
$\mathcal{C} \Rightarrow \mathcal{C}$.
\end{frame}
%}}}

\begin{frame}{Game examples} %{{{
The games used in our development are:
\begin{center}
  \footnotesize
  \begin{tabular}{cllp{.48\textwidth}}
    \hline
    Name & Questions & Answers & Description \\
    \hline
    $\mathcal{C}$ & $(\kw{id}, \kw{sg}, \vec{v}, m)$ & $(v', m')$ &
      C-style function calls (Clight--RTL) \\
    $\mathcal{L}$ & $(\kw{id}, \kw{sg}, \kw{ls}, m)$ & $(\kw{ls}', m')$ &
      Arguments passed in abstract locations (LTL, Linear) \\
    $\mathcal{M}$ & $(\kw{id}, \kw{sp},\kw{ra},\kw{rs}, m)$ & $(\kw{rs}', m')$ &
      Arguments passed through in-memory stack (Mach) \\
    $\mathcal{A}$ & $(\kw{rs}, m)$ & $(\kw{rs}', m')$ &
      Assembly-style control transfers (Asm) \\
    \hline
  \end{tabular}
\end{center}
\end{frame}
%}}}

\begin{frame}{Transition systems} %{{{
A small-step transition system of type $A \Rightarrow B$
is given by a tuple
$L := \langle S, \rightarrow, I, X, Y, F \rangle$:
\begin{itemize}
\item $S$ is a set of states
\item ${\rightarrow} \subseteq S \times S$ is a transition relation;
\item $I \subseteq M_B^Q \times S$ specifies initial states;
\item $X \subseteq S \times M_A^Q$ identifies external calls;
\item $Y \subseteq S \times M_A^A \times S$ resumes the execution;
\item $F \subseteq S \times M_B^A$ specifies final states.
\end{itemize}

\vspace{1em}
This is similar to Compositional CompCert;
transition systems specify the behavior
of a single activation.
\end{frame}
%}}}

%}}}

\section{Simulations} %{{{

\begin{frame}{Simulation conventions} %{{{
To relate interactions which happen in terms of different games,
we introduce \emph{simulation conventions}.

\vspace{1em}
For the source and target games $A_1$ and $A_2$,
a simulation convention
$\mathbb{R} = \langle W, R^Q, R^A \rangle$
specifies:
\begin{itemize}
\item a set of \emph{worlds} $W$;
\item a $W$-indexed relation
$R^Q \subseteq W \times M_{A_1}^Q \times M_{A_2}^Q$;
\item a $W$-indexed relation
$R^A \subseteq W \times M_{A_1}^A \times M_{A_2}^A$.
\end{itemize}

\vspace{1em}
We write $\mathbb{R} : A_1 \Leftrightarrow A_2$.
\end{frame}
%}}}

\begin{frame}{Simulations} %{{{
CompCert's forward and backward simulations are modified to
take these changes into account.

\vspace{1em}
Simulations are parametrized by simulation conventions
for incoming and external calls.
For the transition systems $L_1 : A_1 \Rightarrow B_1$ and
$L_2 : A_2 \Rightarrow B_2$,
we will use the conventions $\mathbb{R}_A : A_1 \Leftrightarrow A_2$
and $\mathbb{R}_B : B_1 \Leftrightarrow B_2$ and write:
\[
    L_1 \le_{\mathbb{R}_A \Rightarrow \mathbb{R}_B} L_2
\]

The simulations add the following conditions:
\begin{itemize}
\item If the initial questions were related at a world $w \in W_B$,
  then the final answers must be related at that same world.
\item For each external call,
  there exists a world $v \in W_A$ that relates
  the source and target questions.
  The environment's answers must be realted at that world
  for the simulation to resume.
\end{itemize}

\end{frame}

\begin{frame}{Constructions}
[Composition and Kleene star for simulation conventions.]
\end{frame}

%}}}

\section{CompCert KLRs}

\begin{frame}{Logical relations and CompCert}
CompCert's memory model
is the algebraic structure at the core of
its language semantics.

\vspace{1em}
CompCert's memory extensions and injection
can be understood as logical relations:
elementary operations are compatible with them,
and we can systematically build up more complex relations
and prove larger relational properties,
up to and including simulation diagrams.

\vspace{1em}
Injections additionally use a set of worlds (\texttt{meminj}),
and an accessibility relation (\texttt{inject\_incr})
to make sure pointers are related consistently throughout
the memory.
We generalize from there
to \emph{CompCert KLRs}.
\end{frame}

\begin{frame}{Example: Memory model operations}
  \begin{align*}
      \kw{Mem.alloc} :
        &\Vdash R^\kw{mem} \rightarrow {=} \rightarrow {=} \rightarrow
        \Diamond (R^\kw{mem} \times R^\kw{block})
      \\
      \kw{Mem.free} :
        &\Vdash R^\kw{mem} \rightarrow R^\kw{ptrrange} \rightarrow
        \kw{option}^\le(\Diamond R^\kw{mem})
      \\
      \kw{Mem.load} :
        &\Vdash R^\kw{mem} \rightarrow R^\kw{ptr} \rightarrow
        \kw{option}^\le(R^\kw{val})
      \\
      \kw{Mem.store} :
        &\Vdash R^\kw{mem} \rightarrow R^\kw{ptr} \rightarrow R^\kw{val} \rightarrow
        \kw{option}^\le(\Diamond R^\kw{mem})
      \\
      \kw{Mem.perm} :
        &\Vdash R^\kw{mem} \rightarrow R^\kw{ptr} \rightarrow {\subseteq}
  \end{align*}
\end{frame}

\begin{frame}{Relational parametricity}
From CKLRs, we can derive simulation conventions
for games such as $\mathcal{C}$.

\vspace{1em}
For example
we define $\mathcal{C}[R] : \mathcal{C} \Leftrightarrow \mathcal{C}$
as follows:
\[
    \mathcal{C}[R] := \langle
      W, \:
      ({=} \times {=} \times {R^\kw{val}}^* \times R^\kw{mem}), \:
      \Diamond (R^\kw{val} \times R^\kw{mem})
    \rangle \,.
\]
\end{frame}

\section{Compiler Correctness}

\begin{frame}{Notable CKLRs}
The CKLRs used in our development are:
\begin{itemize}
  \item \textbf{ext}: memory extensions (uses a trivial set of worlds)
  \item \textbf{inj}: memory injections (\texttt{meminj},
    \texttt{Mem.inject}, etc.)
  \item \textbf{injp}: injections + CompCert's external call requirements
\end{itemize}

\vspace{1ex}
They satisfy the following properties:
\[
  \mathcal{C}[\kw{inj}] \cdot \mathcal{C}[\kw{inj}] \equiv
  \mathcal{C}[\kw{ext}] \cdot \mathcal{C}[\kw{inj}] \equiv
  \mathcal{C}[\kw{inj}] \cdot \mathcal{C}[\kw{ext}] \equiv
  \mathcal{C}[\kw{inj}] \,.
\]

Using this infrastructure,
we do not need extensive changes to the existing CompCert proofs:
although individual passes are not symmetric in terms of their
requirements vs. guarantees,
we will be able to exploit languages properties
to reconcile them at the scale of the whole compiler.
\end{frame}

\begin{frame}{Passes}
\begin{center}
  \tiny
  \begin{tabular}{lllp{.55\textwidth}}
    \hline
    Language/Pass & Outgoing & Incoming & Description \\
    \hline
    \textbf{Clight} & $\mathcal{C}$ & $\mathcal{C}$ &
      A simpler version of CompCert C;
      expressions have no side-effects. \\
    \emph{Eqn.}~(\ref{eqn:clight}) & $\mathcal{C}[\kw{injp}]^*$ & $\mathcal{C}[\kw{injp}]^*$ &
      \emph{Clight properties} \\
    \kw{SimplLocals} & $\mathcal{C}[\kw{injp}]$ & $\mathcal{C}[\kw{inj}]$ &
      Pulling non-adressable scalar local variables out of memory. \\
    \kw{Cshmgen} & \kw{id} & \kw{id} &
      Simplification of control structures;
      explication of type-dependent computations. \\
    \hline
    \textbf{Csharpminor} & $\mathcal{C}$ & $\mathcal{C}$ &
      Low-level structured language. \\
    \kw{Cminorgen} & $\mathcal{C}[\kw{injp}]$ & $\mathcal{C}[\kw{inj}]$ &
      Stack allocation of local variables whose address is taken. \\
    \hline
    \textbf{Cminor} & $\mathcal{C}$ & $\mathcal{C}$ &
      Low-level structured language,
      with explicit stack allocation of certain local variables. \\
    \kw{Selection} & $\mathcal{C}[\kw{ext}]$ & $\mathcal{C}[\kw{ext}]$ &
      Recognition of operators and addressing modes. \\
    \hline
    \textbf{Cminorsel} & $\mathcal{C}$ & $\mathcal{C}$ &
      Like Cminor, with machine-specific operators and addressing modes. \\
    \kw{RTLgen} & $\mathcal{C}[\kw{ext}]$ & $\mathcal{C}[\kw{ext}]$ &
      Construction of the CFG, 3-address code generation. \\
    \hline
    \textbf{RTL} & $\mathcal{C}$ & $\mathcal{C}$ &
      Register transfer language. \\
    \kw{Tailcall} & $\mathcal{C}[\kw{ext}]$ & $\mathcal{C}[\kw{ext}]$ &
      Recognition of tail calls. \\
    \kw{Renumber} & $\kw{id}$ & $\kw{id}$ &
      Postorder renumbering of the CFG. \\
    \emph{Eqn.}~(\ref{eqn:rtl}) & $\mathcal{C}[\kw{inj}]$ & $\mathcal{C}[\kw{inj}]$ &
      \emph{RTL properties} \\
    \kw{Allocation} & \kw{alloc} & \kw{alloc} &
      Register allocation \\
    \hline
    \textbf{LTL} & $\mathcal{L}$ & $\mathcal{L}$ &
      Location transfer language. \\
    \kw{Linearize} & \kw{id} & \kw{id} &
      Linearization of the CFG \\
    \hline
    \textbf{Linear} & $\mathcal{L}$ & $\mathcal{L}$ &
      Like LTL, but the CFG is replaced by
      a linear list of instructions \\
    \kw{CleanupLabels} & \kw{id} & \kw{id} &
      Removal of unreferenced labels. \\
    \kw{Debugvar} & \kw{id} & \kw{id} &
      Synthesis of debugging information. \\
    \kw{Stacking} & \kw{stacking} & \kw{stacking} &
      Laying out the activation records \\
    \hline
    \textbf{Mach} & $\mathcal{M}$ & $\mathcal{M}$ &
      Like Linear, with a more concrete view of the activation record \\
    \kw{Asmgen} & \kw{asmgen} & \kw{asmgen} &
      Emission of assembly code \\
    \hline
    \textbf{Asm} & $\mathcal{A}$ & $\mathcal{A}$ &
      Assembly language for x86 machines \\
    \hline
  \end{tabular}
\end{center}
\end{frame}

\begin{frame}{Correctness theorem}
the overall simulation convention for the whole compiler
can be given as:
\[
  \mathbb{R}_\kw{CompCert} : \mathcal{C} \Leftrightarrow \mathcal{A} :=
    \mathcal{C}[\kw{injp}]^* \cdot
    \mathcal{C}[\kw{inj}] \cdot
    \kw{alloc} \cdot
    \kw{stacking} \cdot
    \kw{asmgen} \,.
\]

From the relational parametricity of Clight and RTL
and the properties outlined above we can show
that the composition of passes is compatible with
$\mathbb{R}_\kw{CompCert}$ on both sides and
conclude that if
$\kw{CompCert}(p_s) = p_t$
we have:
\[
      \kw{Clight} \llbracket p_s \rrbracket
      \le_{\mathbb{R}_\kw{CompCert} \Rightarrow \mathbb{R}_\kw{CompCert}}
      \kw{Asm} \llbracket p_t \rrbracket \,.
\]
\end{frame}

\section{Conclusion}

\begin{frame}{Conclusion}
\end{frame}

\end{document}
