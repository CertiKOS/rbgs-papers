\documentclass[aspectratio=1610]{beamer}
\usepackage{libertine}
\usepackage{cmll}
\usepackage{bbm}

\begin{document}

\begin{frame}{Game interpretation of $\mathbf{Sup}$}
If we look at sup-lattices as \emph{spaces of strategies}, \\
we can interpret lattice constructions as game operators:
\begin{itemize}
  \item $S_1 \oplus S_2$ choice of the system
  \item $S_1 \with S_2$ choice of the environment
  \item $S_1 \otimes S_2$ parallel composition
  \item $S_\bot$ checkpoint
\end{itemize}

\vfill
Likewise a functor $F : \mathbf{Sup} \rightarrow \mathbf{Sup}$
denotes a \emph{partial} game which ends at $X$,
for example
\[
  \hat{E} X := \bigoplus_{q \in E}
    \Big( \bigwith_{r \in \mathsf{ar}(q)} X \Big)_\bot
\]
Functor composition $FG$ play the role of sequential composition of games, \\
with the free monad construction $F^*$ as Kleene iteration.

\vfill
For example $\hat{E}^* X$ is an interaction tree.
\end{frame}

\begin{frame}{Algebras an coalgebras}
The free monad can be folded and unfolded as expected ($d = c^{-1}$):
\[
  c : F( F^* X ) \oplus X \rightarrow F^*X
  \qquad \qquad
  d : F^*X \rightarrow F(F^*X) \oplus X
\]
One advantage of $\mathbf{Sup}$ is nice (co-)induction:
\begin{itemize}
  \item An algebra $\alpha : F S \oplus X \rightarrow S$
    can flatten $[\alpha] : F^* X \rightarrow S$;
  \item A coalgebra $\delta : S \rightarrow F S \oplus X$
    can generate $\langle \delta \rangle : S \rightarrow F^*X$.
\end{itemize}

\vfill
Definition for $\sigma \odot \tau$
uses coalgebra of the form
$\delta : S \rightarrow F S \oplus X$
with $S := S_{B \rightarrow C} \otimes S_{A \rightarrow B}$, \\
then compute $\langle \delta \rangle(\sigma \otimes \tau) \in S_{A \rightarrow C}$.

\vfill
In the process we can ``pull out'' things from one side, for ex:
\begin{itemize}
  \item Distributive law $\lambda : \Big(\bigoplus_i S_i \Big) \otimes S' \rightarrow
    \bigoplus_i (S_i \otimes S')$
  \item Monad strength $s : (F^* A) \otimes B \rightarrow F^* (A \otimes B)$
\end{itemize}
\end{frame}

\begin{frame}{Lenses}
A passive component $f : U \leftrightarrows V$
must now incorporate state on its own:
\begin{align*}
  \mathsf{get} &: V \times U \rightarrow U \\
  \mathsf{set} &: U \times V \rightarrow V
\end{align*}
Note that ``get'' remembers the last $U$ that we set. \\
We must also specify a starting state $u \in U$.

\vfill
This reveals the symmetry between the terminal lens
$\langle U ] : \mathbbm{1} \leftrightarrows U$ \\
and the encapsulation primitive
$[ u \rangle : U \leftrightarrows \mathbbm{1}$.

\vfill
Once we have such a definition,
we can define $\sigma \mathbin@ f$ \\
by using $f$ to build a coalgebra over $S_{E \rightarrow F} \otimes \mathcal{P}(U)$.

\end{frame}

\begin{frame}{Timeline}
\begin{itemize}
  \item Now: Sec 1--2 $\checkmark$
  \item Fri 21: Sec 3 (JK) strategies
  \item Mon 24: Sec 4 (JK) refinement
  \item Wed 26: Sec 5 (JK) spatial comp
  \item Fri 28: Sec 6 (YZ) applications
  \item Mon 01: Sec 7 related work
  \item 
  \item Thu 11: POPL deadline
\end{itemize}
\end{frame}

\end{document}
