\documentclass[handout]{beamer}
\usepackage{bussproofs}
\usepackage{stmaryrd}

\title{Lightweight Compositional Compcert}
\subtitle{a POPL paper pitch}
\author{J\'er\'emie Koenig}

\newcommand{\EC}{\text{EC}}
\newcommand{\nEC}{\overline{\EC}}
\newcommand{\ECEC}{\EC \Rightarrow \EC}
%\newcommand{\ECEC}{\bar{*} \vee *}
\newcommand{\kw}[1]{\texttt{#1}}
\newcommand{\word}[1]{\mathbf{#1}}
\newcommand{\sys}{\textbf{sys}}
\newcommand{\env}{\textbf{env}}
\newcommand{\ident}{\kw{ident}}
\newcommand{\val}{\kw{val}}
\newcommand{\mem}{\kw{mem}}
\newcommand{\Y}{\mathbf{Y}}
\newcommand{\E}{\mathcal{E}}

\setlength{\parskip}{1ex}

\begin{document}

\begin{frame}
\titlepage
\end{frame}

\begin{frame}{Bottom line} %{{{
Deploy the same ideas we use for thread linking
in the context of separate compilation:
\begin{itemize}
\item Define a game semantics for Clight and Asm
\item Game semantics + extcall oracle = original semantics
\item CompcertX just turned into a compositional complier
\end{itemize}
\end{frame}
%}}}

\begin{frame}{Games} %{{{
For this application, only 2 games: $\EC$ and $\ECEC$,
with moves
$e ::= \kw{call}(f, \vec{v}, m) \ \vert\ \kw{ret}(v, m')$.

In $\EC$, the players $\sys$ and $\env$ alternate:
\begin{itemize}
\item $\env$ plays $\kw{call}(f, \vec{v}, m)$;
\item $\sys$ plays $\kw{ret}(v, m)$.
\end{itemize}

In $\ECEC$ aka. $\nEC \vee \EC$,
$\env$ starts with $\kw{call}$ but
$\sys$ can call back in a well-bracketed way.
\end{frame}
%}}}

\begin{frame}{Semantics} %{{{
For a Clight module $M$ with operational semantics $\rightarrow$,
the strategy $\llbracket M \rrbracket : \ECEC$ is defined by:
\[
	\AxiomC{$\kw{replay}(\word{u})
		\rightarrow^?
		\kw{Callstate}(f, \vec{v}, k, m)$}
	\UnaryInfC{$\llbracket M \rrbracket \ni
		\word{u} \cdot \kw{call}(f, \vec{v}, m)$}
	\DisplayProof
\]
\[
	\AxiomC{$\kw{replay}(\word{u})
		\rightarrow^?
		\kw{Returnstate}(v, k, m)$}
	\UnaryInfC{$\llbracket M \rrbracket \ni
		\word{u} \cdot \kw{ret}(v, m)$}
	\DisplayProof
\]

$\kw{replay}(\word{u})$ is the state obtained
by replaying the position $\word{u}$. \\
$\rightarrow^{?}$ is $\rightarrow$ iterated
up to the first $\kw{Callstate}$ or a $\kw{Returnstate}$.
\end{frame}
%}}}

\begin{frame}{Fixpoint} %{{{
Note that $\llbracket M \rrbracket$
does not even service its own internal calls,
instead relying on the environment to do that.

We can introduce a fixpoint operator $\Y$:
\[
	\AxiomC{$\sigma : \ECEC$}
	\UnaryInfC{$\Y \sigma : \EC$}
	\DisplayProof
\]
The strategy $\Y \sigma$ is the composition
of an infinite sequence of copies of $\sigma$,
each using the next one to handle its function calls:
\[ \text{[figure]} \]
This way we can link strategies with $\cup$
and don't need to worry about the domain of each module:
the behavior whole program semantics is $\Y ( \llbracket M_1 \rrbracket \cup \llbracket M_2 \rrbracket \cup \cdots )$.
\end{frame}
%}}}

\begin{frame}{Game simulations} %{{{
[My generalization of Compcert simulations to games]
\end{frame}
%}}}

\begin{frame}{Connecting with the CompcertX theorem} %{{{
Using $\Y/\cup$ to define recursion and composition in one swoop would be nice,%
\footnote{Note the connexion with $\oplus$ as both horizontal and vertical composition}
but does it work for our purposes?

We should be able ``translate'' CompcertX's theorem to:
\[
	\forall \E : \EC,
		\Y (\llbracket M \rrbracket^\kw{Clight} \cup i_2(\E)) \le_R
		\Y (\llbracket C(M) \rrbracket^\kw{Asm} \cup i_2(\E))
\]
To transform it a into a per-module theorem,
we would need a simulation reflection theorem of the form:
\[
	\AxiomC{$\forall \E : \EC, \Y (\sigma_1 \cup i_2(\E)) \le_R \Y (\sigma_2 \cup i_2(\E))$}
	\UnaryInfC{$\sigma_1 \le_{R \Rightarrow R} \sigma_2$}
	\DisplayProof
\]
where $\sigma_1, \sigma_2 : \ECEC$.
\end{frame}
%}}}

\begin{frame}{Can it be done?} %{{{
It is non-trivial but I believe it can be worked out.
It may be necessary to introduce:
\begin{itemize}
\item the domain of modules in the types,
	so that we can sort out the moves of $M$ vs. $\E$, and/or
\item explicit composition,
	so that we can get $\Y$ into $\llbracket - \rrbracket$ and
	out of the simulation reflection theorem.
\end{itemize}
\end{frame}
%}}}

\end{document}

