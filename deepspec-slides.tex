\documentclass{beamer}
\usepackage{stmaryrd}
\usepackage{galois}

\AtBeginSection[]
{
   \begin{frame}
        \tableofcontents[currentsection]
   \end{frame}
}

\begin{document}

\section{Background}

\begin{frame}{DeepSpec glue} %{{{
\end{frame}
%}}}

\begin{frame} %{Building Certified Systems} %{{{
\begin{center}
  \fbox{%
    system description $\quad p \in P
    \quad \xrightarrow{\llbracket - \rrbracket} \quad
    \sigma \in \mathbb{D} \quad$ mathematical theory}
\end{center}
\begin{itemize}
\item \textbf{Refinement/Nondeterminism}
  \[ (\mathbb{D}, {\sqsubseteq}, {\sqcup}, {\sqcap})
     \mbox{ is a complete lattice} \]
\item \textbf{Compositionality}
  \begin{gather*}
    \llbracket p_1 + p_2 \rrbracket =
    \llbracket p_1 \rrbracket \oplus \llbracket p_2 \rrbracket
    \\
    \sigma_1 \sqsubseteq \sigma_1' \:\wedge\:
    \sigma_2 \sqsubseteq \sigma_2' \:\Rightarrow\:
    \sigma_1 \oplus \sigma_2 \sqsubseteq \sigma_1' \oplus \sigma_2'
  \end{gather*}
\item \textbf{Abstraction/Heterogeneity}
  \begin{gather*}
    (\mathbb{D}^\natural, {\sqsubseteq})
    \galois{\alpha}{\gamma}
    (\mathbb{D}^\sharp, {\sqsubseteq})
    \\
    \alpha(\sigma^\natural) \sqsubseteq \sigma^\sharp
    \: \Leftrightarrow \:
    \sigma^\natural \sqsubseteq_\mathbb{C} \sigma^\sharp
    \: \Leftrightarrow \:
    \sigma^\natural \sqsubseteq \gamma(\sigma^\sharp)
  \end{gather*}
\end{itemize}
\end{frame}
%}}}

\begin{frame}{Candidates} %{{{
\begin{center}
  \begin{tabular}{lcccc}
                 & V & H & Abstraction & \ldots \\
    CompCert LTS & $\bullet$ &   & &  \\
    CAL & $\bullet$ & & $\bullet$ & \\
    Interaction Semantics & $\bullet$ & $\bullet$ & & \\
    I-Trees & $\bullet$ & $\bullet$ & ? & \\
    R-B Game Semantics & $\bullet$ & $\bullet$ & $\bullet$ & $\bullet$ \\
  \end{tabular}
\end{center}
\end{frame}
%}}}

\begin{frame}{Game Semantics} %{{{
\begin{center}
  \begin{tabular}{lcc}
  \hline
   & Types & Terms \\
  \hline
  Denotational semantics & Objects & Morphisms \\
  Traditional & Domains & Continuous functions \\
  Game semantics & Games & Strategies \\
  \hline
  \end{tabular}
\end{center}

Game semantics are usually formulated as trace semantics,
but with a strong polarization between actions
of the system and actions of the environment.

The operational aspects
mean that game semantics can characterize
the interactive behavior of program components
more precisely than
usual domain-theoretic approaches.
\end{frame}
%}}}

\section{Refinement-Based Game Semantics}

\begin{frame}{Sequential Compositionality} %{{{
\end{frame}
%}}}

\begin{frame}{Nondeterminism} %{{{
\end{frame}
%}}}

\begin{frame}{Silent Divergence} %{{{
\end{frame}
%}}}

\begin{frame}{Interaction} %{{{
\end{frame}
%}}}

\begin{frame}{Abstraction} %{{{
\end{frame}
%}}}

\section{CompCert}

\begin{frame}{Semantics of Compilation Units} %{{{
\end{frame}
%}}}

\begin{frame}{Challenges for Compositionality} %{{{
\end{frame}
%}}}

\begin{frame}{Kripke Logical Relations} %{{{
\end{frame}
%}}}

\begin{frame}{Refinement Convention Algebra} %{{{
\end{frame}
%}}}

\begin{frame}{Compiler Correctness} %{{{
\end{frame}
%}}}

\section{Conclusion}

\begin{frame}{Role within DeepSpec} %{{{
Embed everything into (some version of) RBGS for interoperability.

Monad interface can probably be defined on interaction trees.
\end{frame}
%}}}

\begin{frame}{Future Work} %{{{
More complex games, concurrency, state.
\end{frame}
%}}}

\end{document}

