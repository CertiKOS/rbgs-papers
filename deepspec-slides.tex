\documentclass{beamer}
\usepackage{stmaryrd}
\usepackage{galois}
\usepackage{bussproofs}
\newcommand{\bind}{\gg\!\!=}

\AtBeginSection[]
{
   \begin{frame}
        \tableofcontents[currentsection]
   \end{frame}
}

\begin{document}

\section{Introduction}

\begin{frame}{Goal} %{{{

Tension:
\begin{itemize}
\item
  To make verification tractable,
  we need semantic models that are
  specialized to the task at hand.
\item 
  To link verified artefacts together and
  build large-scale certified systems,
  we need compatible semantic models.
\end{itemize}
\vfill

Our hope:
\begin{itemize}
\item
  To build a general-purpose, compositional
  semantic model supporting large-scale reasoning
  about heterogenous systems.
\item
  To enable linking disparate verification projects
  by making sure more specialized semantics
  can be embedded into ours.
\end{itemize}
\vfill

\end{frame}
%}}}

\begin{frame}{Building Certified Systems} %{{{
\small
\begin{center}
  \fbox{%
    system description $\quad p \in P
    \quad \xrightarrow{\llbracket - \rrbracket} \quad
    \sigma \in \mathbb{D} \quad$ mathematical theory}
\end{center}
\begin{itemize}
\item \textbf{Refinement/Nondeterminism}
  \[ (\mathbb{D}, {\sqsubseteq}, {\sqcup}, {\sqcap})
     \mbox{ is a complete lattice} \]
\item \textbf{Compositionality}
  \begin{gather*}
    \llbracket p_1 + p_2 \rrbracket =
    \llbracket p_1 \rrbracket \oplus \llbracket p_2 \rrbracket
    \\
    \sigma_1 \sqsubseteq \sigma_1' \:\wedge\:
    \sigma_2 \sqsubseteq \sigma_2' \:\Rightarrow\:
    \sigma_1 \oplus \sigma_2 \sqsubseteq \sigma_1' \oplus \sigma_2'
  \end{gather*}
\item \textbf{Abstraction/Heterogeneity}
  \begin{gather*}
    (\mathbb{D}^\natural, {\sqsubseteq})
    \galois{\alpha}{\gamma}
    (\mathbb{D}^\sharp, {\sqsubseteq})
    \\
    \alpha(\sigma^\natural) \sqsubseteq \sigma^\sharp
    \: \Leftrightarrow \:
    \sigma^\natural \sqsubseteq_\mathbb{C} \sigma^\sharp
    \: \Leftrightarrow \:
    \sigma^\natural \sqsubseteq \gamma(\sigma^\sharp)
  \end{gather*}
\end{itemize}
\end{frame}
%}}}

%\begin{frame}{Candidates} %{{{
%\begin{center}
%  \begin{tabular}{lcccc}
%                 & V & H & Abstraction & \ldots \\
%    CompCert LTS & $\bullet$ &   & &  \\
%    CAL & $\bullet$ & & $\bullet$ & \\
%    Interaction Semantics & $\bullet$ & $\bullet$ & & \\
%    I-Trees & $\bullet$ & $\bullet$ & ? & \\
%    R-B Game Semantics & $\bullet$ & $\bullet$ & $\bullet$ & $\bullet$ \\
%  \end{tabular}
%\end{center}
%\end{frame}
%%}}}

\begin{frame}{Game Semantics} %{{{
Basic idea:
\begin{center}
  \begin{tabular}{lcc}
  \hline
   & Types & Terms \\
  \hline
  Denotational semantics & Objects & Morphisms \\
  \hspace{1em} Traditional & Domains & Continuous functions \\
  \hspace{1em} Game semantics & Games & Strategies \\
  \hline
  \end{tabular}
\end{center}

\vfill
Game semantics:
\begin{itemize}
\item
  are usually formulated as trace semantics,
  but with a strong polarization between actions
  of the system and actions of the environment;
\item
  can characterize
  the interactive behavior of program components
  more precisely than
  usual domain-theoretic approaches,
  thanks to a strong operational flavor.
\end{itemize}
\vfill
\end{frame}
%}}}

\section{Refinement-Based Game Semantics}

\begin{frame}{Games} %{{{

We use games derived from very simple
\emph{elementary games},
which consist of
one \emph{environment question} followed by
one \emph{system answer}.
For C-style CompCert open modules,
questions of the form $f(\vec{v})@m$, and
answers consist of a return value and final memory state $v'@m'$.
\vfill

We will focus in particular on games of the form $!A \multimap B$,
where $A = (M, N)$ and $B = (Q, R)$ are elementary games,
so that the interaction will take the form:
\[
    q \, \mathbf{m} \, n \, \mathbf{m} \, n \cdots \mathbf{m} \, n \, \mathbf{r}
\]
In the CompCert context,
the game $B$ specifies the form of incoming calls and corresponding returns,
and the game $A$ specifies the form of external calls and returns.
\vfill

\end{frame}
%}}}

\begin{frame}{Strategies} %{{{

Following the usual approach,
we encode strategies as prefix-closed sets of traces.
More specifically,
for now our strategies are:
\begin{itemize}
\item \emph{Alternating:}
  system and environment play every other move;
\item \emph{Non-deterministic:}
  a strategy may allow multiple system behaviors,
  this induces a refinement lattice;
\item \emph{Receptive:}
  stragies do not restrict the environment
  but only specify the behavior of the system.
\end{itemize}
Because of the simplicity of the games we consider,
the traditional \emph{innocence} and \emph{well-bracketing} conditions
will also be satisfied.
\vfill

Concretely, we encode strategies as Kleisli morphisms
$Q \rightarrow \mathcal{I}_{M,N}(R)$ for the
\emph{interaction monad} described thereafter,
which allows a computation to perform output-input pairs.
\vfill

\end{frame}
%}}}

\begin{frame}{Traces} %{{{

For system moves $m \in M$ and environment moves $n \in N$
we will consider traces of the form:
\[
    s, t \in
    \mathcal{T}_{M,N}(A) ::=
    m \mid mnt \mid v \mid \Delta \mid \lightning \,,
\]
ordered by reflexivity plus the rules:
\[
  t \sqsubseteq \lightning , \quad
  m \sqsubseteq mnt ,
  \quad
  \AxiomC{$s \sqsubseteq t$}
  \UnaryInfC{$mns \sqsubseteq mnt$}
  \DisplayProof
\]

The special actions are:
\begin{itemize}
\item Traces may terminate with a return value $v \in A$;
\item Silent divergence is represented by $\Delta$;
\item Undefined behaviors are represented by $\lightning$.
\end{itemize}

\end{frame}
%}}}

\begin{frame}{Sequential Compositionality} %{{{

The terminating outcomes $v$ enable sequential compositionality:
we define the \emph{interaction monad},
whose elements:
\[ x \in \mathcal{I}_{M,N}(A) \]
are prefix-closed sets of traces, so that:
\[ x \subseteq \mathcal{T}_{M,N}(A) \,, \quad
   t \in x \wedge s \sqsubseteq t \Rightarrow s \in x \,. \]
The monad's unit is $\eta(v) = \{ v \}$ and
the binding operation is defined trace-wise:
$(x \bind f) = \bigcup_{t \in x} (t \bind f)$ with
\begin{gather*}
  \Delta \in (\Delta \bind f)
  \qquad
  t \in (\lightning \bind f)
  \qquad
  m \in (m \bind f)
  \\[1ex]
  \AxiomC{$t \in f(v)$}
  \UnaryInfC{$t \in (v \bind f)$}
  \DisplayProof
  \quad
  \AxiomC{$s \in (t \bind f)$}
  \UnaryInfC{$mns \in (mnt \bind f)$}
  \DisplayProof
\end{gather*}

\end{frame}
%}}}

\begin{frame}{Non-deterministic iteration} %{{{

Since unions and intersections preserve prefix closure,
they induce a complete lattice on $\mathcal{I}_{M,N}(A)$
with respect to trace containement.

This allows us to define a notion of Kleene star
on Kleisli morphisms of the form $f : A \rightarrow \mathcal{I}_{M,N}(A)$:
\[
  \begin{array}{r@{\:}l}
    f^0(a) &:= \eta(a) \\
    f^{n+1}(a) &:= f^n(a) \bind f
  \end{array}
  \qquad
  f^*(a) := \bigcup_n f^n(a) \,.
\]

We say that $f$ diverges silently at $a$
if there is an infinite sequence $(a_i)_{i \in \mathbb{N}}$
such $f(a_i) \supseteq \eta(a_{i+1})$ for all $i \in \mathbb{N}$,
and we define:
\begin{align*}
f_\Delta(a) &:= \begin{cases}
  \{\Delta\} & \mbox{if } f \mbox{ diverges at } a \\
  \varnothing & \mbox{otherwise}
\end{cases} \\
f^\omega(a) &:= f^*(a) \bind f_\Delta \\
f^\infty(a) &:= f^*(a) \cup f^\omega(a)
\end{align*}

\end{frame}
%}}}

\begin{frame}{Interaction} %{{{
For interaction,
we introduce the primitive $\mathbf{I} : M \rightarrow \mathcal{I}_{M,N}(N)$,
defined by $\mathbf{I}(m) := \{ m, mnn \}$.
Interaction introduces a new form of composition:
for $f : M \rightarrow \mathcal{I}_{P,Q}(N)$ and $x : \mathcal{I}_{M,N}(A)$,
we define
$x[f] : \mathcal{I}_{P,Q}(A)$,
which substitutes $f$ for the output-input pairs of $x$.
\vfill

This can be extended to the interactive composition
$g \odot f : A \rightarrow \mathcal{P,Q}(B)$ of the Kleisli morphisms
$f : M \rightarrow \mathcal{I}_{P,Q}(N)$ and
$g : A \rightarrow \mathcal{I}_{M,N}(B)$,
defined as:
\[ (g \odot f)(a) = g(a)[f] \]

Interactive substitution and composition satisfy the properties:
\begin{align*}
  x[\mathbf{I}] &= x &
  \mathbf{I}(m)[f] &= f(m) &
  x[g \odot f] &= x[g][f] \\
  g \odot \mathbf{I} &= g &
  \mathbf{I} \odot f &= f &
  h \odot (g \odot f) &= (h \odot g) \odot f
\end{align*}

\end{frame}
%}}}

\begin{frame}{Horizontal composition} %{{{
$\alpha \odot \beta$ corresponds to vertical composition in CertiKOS:
external calls of $\alpha$ are handled by the underlay $\beta$.
To define horizontal composition we need to iterate $\odot$.
\vfill

In the same way the sequential composition $\cdot$ (derived from $\bind$)
is iterated as $-^\infty$,
we define $f^\circledcirc$ as the iteration of $\cdot$,
so that morally:
\[
    f^\circledcirc = f \odot f \odot \cdots
\]
Then the horizontal composition of $\alpha$ and $\beta$
can be defined as:
\[ \alpha \bullet \beta := (\alpha \cup \beta)^\circledcirc \]
\vfill

\end{frame}
%}}}

\begin{frame}{Summary} %{{{

  \begin{center}
    \begin{tabular}{lccccccc}
      \hline
      Category & Appl. & Comp. & Unit & Join & Zero & Iteration \\
      \hline
      Functions & $f(x)$ & $f \circ g$ & $\mathsf{id}$ & & \\
      Monad &
        $x \bind f$ & $g \cdot f$ & $\eta$ &
        $\cup$ & $\varnothing$ & ${*}, {\infty}$ \\
      Interaction &
        $x[f]$ & $g \odot f$ & $\mathbf{I}$ &
        $\cup$ & $\varnothing$ & $\circledcirc$ \\
      \hline
    \end{tabular}
  \end{center}

\end{frame}
%}}}

\section{Challenges}

\end{document}

